\chapter{Project Functions}
\label{vl:tc-project}

As defined in the \dword{dune} Management Plan (DMP), the \dword{dune}
\dword{tc} generates and recommends technical decisions to the
\dword{dune} collaboration \dword{exb}, which
comprises all consortia scientific and technical leads. The board meets
regularly (approximately monthly) to review and resolve
technical issues with detector construction. It reports
through the \dword{exb} to collaboration management. The \dword{dune} \dword{tb}
is chaired by the \dword{dune} \dword{tcoord}. The
\dword{tc} \fixme{I agree with Tim about the abbreviations TC and TCn. They are too easily confused, and the second is inconsistent with other abbreviations, which are all capitals.} engineering team meets regularly (approximately weekly)
to discuss more detailed technical issues. The \Dword{tc} is not responsible for financial issues, which are referred to
the \dword{exb} and Resource Coordinator (RC) \fixme{Should this be a dword abbreviation? It is used later in this section and should probably be included in the glossary.} .

The \Dword{tc} has several major project support tasks:
\begin{itemize}
\item Ensure that each consortium has a well defined and complete
  scope, that interactions between the consortia are sufficiently
  well defined, and that any remaining scope is covered by \dword{tc}
  through \dword{comfund} or flagged as missing scope to the EB and RC. In
  other words, ensure that the full detector scope is
  identified. Monitor interactions and consortia progress in
  delivering what their scope requires.
\item Develop an overall project \dlong{ims}
  that includes reasonable production schedules, testing plans, and a
  well developed installation schedule from each consortium. Monitor
  the \dword{ims} as well as individual consortium schedules.
\item Ensure that appropriate engineering and safety standards are
  developed, understood, and agreed to by all key stakeholders and that these
  standards are conveyed to and understood by each
  consortium. Monitor the design and engineering work.
\item Ensure that all \dword{dune} requirements on \dword{lbnf} for
  conventional facilities, cryostat, and cryogenics are clearly
  defined and understood by each consortium. Negotiate scope
  boundaries with \dword{lbnf}. Monitor \dword{lbnf} progress on
  final conventional facility design, cryostat design, and cryogenics
  design.
\item Ensure that all technical issues associated with scaling from
  \dword{protodune} have sufficient resources to enable the detector to be fully integrated and
  installed, converging on
  decisions made by the consortia.
\item Ensure that each consortium has fully developed and reviewed the integration and \dword{qc} processes and that the requirements for the \dword{itf} are well defined.
\end{itemize}

\Dword{tc} is responsible for technical quality and the schedule but not
for consortia funding or budgets.  \Dword{tc} will try to resolve any issues it can but will likely push all
financial issues to the \dword{tb}, \dword{exb}, and RC for resolution.

\Dword{tc} maintains a web
page\footnote{\url{https://web.fnal.gov/collaboration/DUNE/DUNE\%20Project/\_layouts/15/start.aspx\#/}.}
with links to project documents. \Dword{tc} also maintains repositories of
project documents and drawings. These include the \dword{wbs},
schedule, risk register, requirements, milestones, strategy, detector
models, and drawings that define the \dword{dune} detector.

\section{\dword{dune} Project Description}

The following sections describe the systems that track the progress
of the \dword{dune} project as well as the ways in which this large, distributed organization is managed.

The \dword{dune} project builds on significant development in
previous large \dword{lartpc} detectors, ICARUS and MicroBooNE, and on substantial development from LBNE and LBNO. One of the most
important elements of the project development is successfully
constructing and operating \dword{protodune} detectors. These detectors
use full-size \dword{dune} components and processes. The construction of
\dword{protodune} has established teams, production lines, QA and QC
processes, installation, operation, and performance of the final
\dword{dune} detectors. Based on the success of \dword{protodune}, we have
reached advanced technical maturity, approaching
(80\%). The designs are significantly advanced
from \dword{protodune} to \dword{dune}. Most subsystems completed \dword{pdr}
or 60\% reviews on design modifications beyond \dword{protodune} in
advance of the \dword{tdr}. The overall level of design maturity is now $\sim$90\%. The breakdown of design maturity level for \dword{dsp} by
subsystem is provided in Table~\ref{tab:designmaturity}. The table
shows the final \dword{dune} design maturity at the time of \dword{protodune} and now
at the time of the \dword{tdr}, along with the estimated design effort or
weight of each subsystem. The APA conceptual design was developed in
2010 and prototyped at 40\% scale in the 35-ton detector. The
version deployed in \dword{pdsp} is close to that for \dword{dsp}
(85\%). The cold electronics low noise system design, including
feedthroughs, cables, and grounding, was successfully prototyped at large scale in MicroBooNE and \dword{pdsp} and is 90\% mature. The front end chip has gone through eight iterations and was successfully demonstrated in MicroBooNE and \dword{pdsp}
(90\%). The frontend motherboard has gone through a similar number of
iterations and was successfully demonstrated in \dword{pdsp}
(80\%). The ADC chip has evolved from a previous version used in
cryogenic conditions for XXX \fixme{I assume the XXX will be replaced.} (50\%). Key elements of the COLDATA chip
have been prototyped (25\%). The HV design has evolved from ICARUS, MicroBooNE, and 35-ton detector.  It has been prototyped in subsequent runs of the
35-ton detector and demonstrated in \dword{pdsp} (80\%). The photon
system ARAPUCA design has been prototyped at small scale and in
\dword{pdsp} (20\%). The mechanical design has been extensively
developed using the 35-ton detector and \dword{pdsp} (85\%). The DAQ artdaq
backend has been developed in several experiments, including the 35-ton detector
and \dword{pdsp}. The DAQ FELIX frontend has been developed by ATLAS
and prototyped in \dword{pdsp}.
\begin{dunetable}
  [\dword{dsp} design maturity]
  {|p{0.1\linewidth}|rp{0.1\linewidth}|rp{0.25\linewidth}|rp{0.2\linewidth}|}
  {tab:designmaturity}
  {\dword{dsp} design maturity}
  System & Weight & \dword{protodune} & \dword{dune}   \\ \toprowrule
  DSS & 10\% & 75\% &  85\% \\ \colhline
  APA & 30\% & 85\% &  95\% \\ \colhline
  CE  & 20\% & 80\% &  90\% \\ \colhline
  PDS & 10\% & 50\% &  65\% \\ \colhline
  HVS & 15\% & 80\% &  95\% \\ \colhline
  DAQ & 10\% & 60\% &  80\% \\ \colhline
  CISC & 5\% & 80\% &  90\% \\ \colhline \colhline
  Total& 100\% & 76\% & 90\% \\ \colhline
\end{dunetable}

The design maturity of the \dword{ddp} detector is also quite
advanced. It builds on working noble liquid \dwords{tpc} for dark
matter and neutrinoless double beta decay experiments. A significant
benchmark is the operation of the $3\times1\times1$ demonstrator at
CERN. A critical test will be the operation of \dword{pddp} at CERN. A
similar table of design maturity for \dword{ddp} will be included
here.

%%%%%%%%%%%%%%%%%%%%%%%%%%%%%%%%
\section{WBS}
\label{sec:fdsp-coord-wbs}

The \dword{dune} WBS \fixme{What does WBS stand for? I don't think it has yet been defined.} has six categories at level 1 as shown in
Figure~\ref{fig:WBS_level2}.
\fixme{Figure 3.1 looks like a table to me. Tables have columns and rows. This has columns and rows. Ergo, it's a table.}
\begin{dunefigure}[\dword{dune} WBS at level 2]{fig:WBS_level2}
  {High level \dword{dune} WBS to level 2.}
  \includegraphics[width=0.75\textwidth]{WBS_level2}
\end{dunefigure}
At level 2, WBS breaks down into seven items, both
for single-phase and dual-phase far detectors.

Each subsystem WBS is owned by the associated consortium. The WBS follow a common format with the first level (level 3) broken down in rough time sequence into the following categories:
\begin{enumerate}
  \item Management,
  \item Physics and Simulation,
  \item Design, Engineering, and R\&D,
  \item Production Setup,
  \item Production,
  \item Integration (ITF),
  \item Installation (\surf).
\end{enumerate}
Subsequent levels in the WBS generally follow consortia subsystem structure.
This WBS has been used as the framework to capture all of the costs
for the overall \dword{dune} project cost estimate.

%%%%%%%%%%%%%%%%%%%%%%%%%%%%%%%%
\section{Cost}
\label{sec:fdsp-coord-cost}

A preliminary total \dword{dune} project cost estimate scaled from
\dword{protodune} costs was provided to the Neutrino Cost Group in August
2018. Based on feedback from that review and updated cost estimates
based on \dword{dune} designs, an updated cost estimate has been
developed. A corresponding project schedule has been developed and
discussed in Section~\ref{sec:fdsp-coord-controls}. Cost estimates are
provided for both single-phase (\dword{dsp}) and dual-phase
(\dword{ddp}) detectors. The sequencing plan is described in
Volume~\volnumberexec:~\voltitleexec. A cost summary is
provided in Table~\ref{tab:cost_L1}.
\begin{dunetable}
  [\dword{dune} Project Cost at WBS level 1]
  {|p{0.07\linewidth}|p{0.15\linewidth}|p{0.2\linewidth}|p{0.2\linewidth}|p{0.2\linewidth}|}
  {tab:cost_L1}
  {\dword{dune} Project Cost for each detector}.
  WBS & Detector type & Detector \#1 & Detector \#2 & Detector \#3   \\ \toprowrule
  2.0 & \dword{dsp} & &  & \\ \colhline
  3.0 & \dword{ddp} & & & \\ \colhline
\end{dunetable}

The cost estimate includes the materials and services (M\&S) cost in
local currencies in 2018, converted to 2018 US\$. The cost estimate
includes the number of labor hours broken down into eight categories
(engineer, designer, technician, student, postdoc, graduate student,
scientist, faculty). There is no escalation included in the cost estimate.

The cost estimate is for one single-phase and one dual-phase far
detector, where the sequencing assumes that the single-phase detector
comes first. This implies that some common costs are captured in the
single-phase cost estimate. The cost estimate at WBS level two is
shown in Table~\ref{tab:cost_L2}.
\begin{dunetable}
  [\dword{dune} Project Cost at WBS level 2]
  {|p{0.05\linewidth}|p{0.3\linewidth}|p{0.25\linewidth}|p{0.35\linewidth}|}
  {tab:cost_L2}
  {\dword{dune} Project Cost at WBS level 2}
  WBS & Item & M\&S & Labor hours   \\ \toprowrule
  {\bf 2.0} & {\bf \dword{dsp}} & &             \\ \colhline
  2.1 & Technical Coordination & &  \\ \colhline
  2.2 & APA & &  \\ \colhline
  2.3 & CE & &  \\ \colhline
  2.4 & PDS & &  \\ \colhline
  2.8 & HV & &  \\ \colhline
  2.9 & DAQ & &  \\ \colhline
  2.10 & CISC & &  \\ \colhline
  {\bf 3.0} & {\bf \dword{ddp}} & &             \\ \colhline
  3.1 & Technical Coordination & &  \\ \colhline
  3.5 & CRP & &  \\ \colhline
  3.6 & CE & &  \\ \colhline
  3.7 & PDS & &  \\ \colhline
  3.8 & HV & &  \\ \colhline
  3.9 & DAQ & &  \\ \colhline
  3.10 & CISC & &  \\ \colhline
\end{dunetable}


No operations costs are included. While some parts of the detector are
accessible and equipment can be maintained, other parts are not. For
those systems with accessible equipment some costs are included for
replacement over the life of the detector.

Detailed cost estimates for each subsystem are available in Volume~\volnumbersp\ of the
\dword{dune} \dword{tdr} \dword{dsp}  and Volume~\volnumberdp\
\dword{ddp} for each consortium. A
companion cost book for \dword{dune} contains further details.
\fixme{What is the following list? It looks like a reminder that not all information has yet been included here.}

\begin{itemize}
 \item cost model, how many SP/DP detectors
 \item spares, labor categories, ...
 \item summary of consortia costs
 \item TC cost details?
\end{itemize}

%%%%%%%%%%%%%%%%%%%%%%%%%%%%%%%%
\section{MOU}
\label{sec:fdsp-coord-mou}

A \dword{mou} will be executed in which all deliverables will be
documented. The \dword{mou} will include the responsibilities of each
collaborating institution and their funding agency during the
construction phase of the experiment.

%%%%%%%%%%%%%%%%%%%%%%%%%%%%%%%%
\section{Budget}
\label{sec:fdsp-coord-budget}

The \dword{dune} \dword{tc} will be supported using the \dword{comfund} and by the host
country. The RC will oversee the \dword{comfund}. \fixme{Resource coordinator was previously abbreviated, and again the abbreviation needs to go in the glossary and use LATEX coding.}
A budget plan that outlines why the \dword{tc} is necessary will be reviewed and
approved by the \dword{rrb} annually. \fixme{This was how I read this sentence. However, it could instead be "The \dword{rrb} will review and approve the needs of the \dword{tc} annually in a budget plan."}

%%%%%%%%%%%%%%%%%%%%%%%%%%%%%%%%
\section{Schedule}
\label{sec:fdsp-coord-controls}

A series of tiered milestones have been developed for the \dword{dune}
project. The spokespersons and host
laboratory director are responsible for the tier 0 milestones. Three tier 0 milestones have been defined and the dates set:
\begin{enumerate}
\item Start main cavern excavation \hspace{2.58in} 2019
\item Start \dword{detmodule}~1 installation \hspace{2.1in} 2022
\item Start operations of \dword{detmodule}~ \fixme{The word module should be plural here.} 1 and 2 with beam \hspace{1in} 2026
\end{enumerate}
These dates will be revisited when the \dword{tdr} is reviewed. The \dword{tc} and \dword{lbnf} project
manager hold the Tier 1
milestones; these milestones will be defined in advance of the \dword{tdr} review. The consortia themselves hold the Tier 2
milestones.

Table~\ref{tab:DUNE_schedule} provides a high level version of the \dword{dune} milestones from the \dword{ims}.
\begin{dunetable}
  [Overall \dword{dune} Project Tier-1 milestones.]
  {p{0.84\linewidth}p{0.14\linewidth}}
  {tab:DUNE_schedule}
  {Overall \dword{dune} Project Tier-1 milestones.}
  Milestone & Date   \\ \toprowrule
  RRB Approval of Technical Design Review                       & 09/02/2019 \\ \colhline
  Beneficial Occupancy of Integration Test Facility             & 09/01/2021 \\ \colhline
  Construction of steel frame for Cryostat \#1 complete         & 12/17/2021 \\ \colhline
%  Construction of Mezzanine for Cryostat \#1 complete           & 01/17/2022 \\ \colhline
%  Begin integration/testing of Detector \#1 components at ITF   & 02/01/2022 \\ \colhline
  Beneficial Occupancy of Central Utility Cavern Counting room  & 04/16/2022 \\ \colhline
  Construction of steel frame for Cryostat \#2 complete         & 07/01/2022 \\ \colhline
%  Construction of Mezzanine for Cryostat \#2 complete           & 08/01/2022 \\ \colhline
  \textbf{Beneficial occupancy of Cryostat \#1}                 & \textbf{12/23/2022} \\ \colhline
  Cryostat \#1 ready for TPC installation                       & 05/01/2023 \\ \colhline
%  Begin integration/testing of Detector \#2 components at ITF   & 11/01/2023 \\ \colhline
  \textbf{Beneficial occupancy of Cryostat \#2}                 & \textbf{03/01/2024} \\ \colhline
%  Begin closing Temporary Construction Opening for Cryostat \#1 & 05/01/2024 \\ \colhline
  Cryostat \#2 ready for TPC installation                       & 08/01/2024 \\ \colhline
  Cryostat \#1 ready for filling                                & 10/01/2024 \\ \colhline
%  Begin closing Temporary Construction Opening for Cryostat \#2 & 07/18/2025 \\ \colhline
  \textbf{Detector \#1 ready for operations}                    & \textbf{10/01/2025} \\ \colhline
  Cryostat \#2 ready for filling                                & 12/05/2025 \\ \colhline
  \textbf{Detector \#2 ready for operations}                    & \textbf{12/18/2026} \\
\end{dunetable}

To monitor progress, the \Dword{tc} will maintain the \dword{ims} that links all consortium schedules
and contains the appropriate milestones.
%It is currently envisioned as three levels of control and notification milestones in addition to the detailed consortium schedules. The highest level contains external milestones, with the second level containing the key milestones for \dword{tc} to monitor deliverables and installation progress, and the third level containing the inter-consortium links.
The schedules will go under change control after each
consortium agrees to the notification milestone dates and the \dword{tdr} is
approved.
In addition to the overall \dword{ims} for construction and
installation, a schedule of key consortia activities has been developed during
2018 and 2019.

To ensure that the \dword{dune} detector remains on schedule, the
\dword{tc} will monitor schedule status from each consortium and organize
reviews of schedules and risks as appropriate.  As schedule problems
arise, the \dword{tc} will work with the affected consortium to resolve the
problems. If problems cannot be solved, the \dword{tc} will take the issue to the
\dword{tb} and \dword{exb}.

A monthly report with input from all consortia will be published by the
\dword{tc}. This will include updates on consortium technical progress and
updates from the \dword{tc}.

%%%%%%%%%%%%%%%%%%%%%%%%%%%%%%%%
\section{Risks}
\label{sec:fdsp-coord-risks}

\dword{dune} has implemented a risk registry in
DocDB-6443. This document includes tabs for consortia risks
and \dword{tc} risks. It includes a summary tab for the most significant overall
\dword{dune} risks. Tables~\ref{fig:dune_risks1} and~\ref{fig:dune_risks2} show the overall \dword{dune} risks. This registry
is updated approximately every year.
\fixme{Again, this has rows and columns, so it is a table, not a figure. It should be relabeled as such. Moreover, the text refers to it as a table, but the graphic itself is labeled a figure.}
\begin{dunefigure}[\dword{dune} overall risk register]{fig:dune_risks1}
  {First page of \dword{dune} overall risk register.}
  \includegraphics[width=0.99\textwidth]{DUNE_Risks_v4b1}
\end{dunefigure}
\begin{dunefigure}[\dword{dune} overall risk register]{fig:dune_risks2}
  {Second page of \dword{dune} overall risk register.}
  \includegraphics[width=\textwidth]{DUNE_Risks_v4b2}
\end{dunefigure}
\dword{lbnf} and \dword{dune}-US would like \dword{dune} to update and
expand this risk register to allow a Monte Carlo analysis of cost and
schedule risks to the US \fixme{I think the other chapters have used coding for this abbreviation to keep it consistent throughout the document.} project resulting from international
\dword{dune} risks. This request is under consideration.
Table~\ref{fig:tc_risks} summarizes the \dword{tc} risks.
\begin{dunefigure}[\dword{tc} risks]{fig:tc_risks}
  {\dword{tc} risk register.}
  \includegraphics[width=\textwidth]{TC_Risks_v4b}
\end{dunefigure}

\fixme{This table is referred to as a table in the text, but the graphic is labeled as a figure. It is a table and should be labeled as a table.}


Successfully operating \dword{protodune} retires many
potential risks in \dword{dune} itself. This includes most risks associated with the
technical design, production processes, \dword{qa}, integration,
and installation. Residual risks remain relating to design and
production modifications associated with scaling to \dword{dune}, mitigations
to known installation and performance issues in \dword{protodune}, underground
installation at \surf. and organizational growth.

The highest technical risks include development of a system to
deliver \SI{600}{kV} to the \dual cathode; general delivery of the
required \dword{hv}; cathode and \dword{fc} discharge to the cryostat
membrane; noise levels, particularly for the \dword{ce}; %cold TPC electronics,
number of dead channels; lifetime of components surpassing \dunelifetime{}; %20 years,
\dword{qc} of all components; verification of improved \dword{lem}
performance; verification of new cold  \dword{adc} and  \dword{coldata} performance;
argon purity; electron drift lifetime; \phel light yield;
incomplete calibration plan; and incomplete connection of design to
physics. Other major risks include insufficient funding, optimistic
production schedules, incomplete integration, and plans for testing and installation.

Key risks for \dword{tc} to manage include the following:
\begin{enumerate}
\item The consortia left too much scope  unaccounted for,so much falls
  to the \dword{tc} and \dword{comfund}.
\item Insufficient organizational systems in place to
  ensure that this complex international mega-science project,
  including \dword{tc}, \fnal as host laboratory, \surf, DOE, and all international
  partners continue to work together successfully to ensure
  appropriate rules and services are provided for the success of
  the project.
\item Inability of \dword{tc} to obtain sufficient personnel resources to
  ensure that \dword{tc} can oversee and coordinate all project tasks.  While the USA \fixme{This is why a code is needed for this abbreviation. Previously, US was used as the abbreviation for United States, but here USA is used. In addition, some authors may be adding periods to the abbreviation.} has a special responsibility to the
  \dword{tc} as host country, personnel resources should
  be directed to \dword{tc} from each collaborating country. Moreover, consortium deliverables are not 
  stand-alone subsystems, which makes them part of the risks for the project; they are all parts of a single \dword{detmodule}. This
  elevates the requirements for coordination among consortia.
\end{enumerate}

The consortia have provided preliminary versions of risk analyses that
have been collected on the \dword{tc} webpage. These are being developed into
an overall risk register that will be monitored and maintained by \dword{tc}
in coordination with the consortia.

%%%%%%%%%%%%%%%%%%%%%%%%%%%%%%%%
\section{Requirements}
\label{sec:fdsp-coord-requirements}

The scientific goals of \dword{dune} as described in the \dword{dune}
\dword{tdr} Volume~\volnumberexec:~\voltitleexec include
\begin{itemize}
\item a comprehensive program of neutrino oscillation measurements
  including searching for CP violations;
\item measurement of $\nu_{e}$ flux from a core-collapse supernova within our
  galaxy should one occur during \dword{dune} operations;
\item searching for baryon number violations.
\end{itemize}
These goals are behind a number of key detector requirements: drift
field, electron lifetime, system noise, photon detector light yield,
and event time resolution. The \dword{exb} has approved a list of high
level detector specifications, including those listed above. These are
maintained in edms-xxxx, and the high level requirements with
significant impact on physics are highlighted in
Table~\ref{tab:dunephysicsreqs}.
\begin{dunetable}
  [\dword{dune} physics-related specifications owned by \dword{exb}]
  {p{0.025\textwidth}p{0.06\textwidth}p{0.2\textwidth}p{0.35\textwidth}p{0.15\textwidth}p{0.1\textwidth}}
  {tab:dunephysicsreqs}
  {\dword{dune} high level system specifications owned by \dword{exb}}
  ID & System & Parameter & Physics Requirement Driver & Requirement & Goal \\ \toprowrule
  1   & HVS    & Minimum drift field &  Limit recombination, diffusion and space charge impacts on particle ID. Establish adequate \dword{s/n} for tracking. & >\SI{250}{V/cm} & \spmaxfield \\ \colhline
  2   & CE     & System noise & The noise specification is driven by pattern recognition and two-track separation.  & <\SI{1000}{enc} & ALARA \\ \colhline
  3   & PDS    & Light yield  & The light yield shall be sufficient to measure time of events with visible energy above 200 MeV.  Goal is 10\% energy measurement for visible energy of 10 MeV.  & >\SI{0.5}{pe/MeV} & >\SI{5}{pe/MeV}  \\ \colhline
  4   & PDS    & Time resolution  & The time resolution of the photon detection system shall be sufficient to assign a unique event time.  & $<\,\SI{1}{\micro\second}$ & $<\,\SI{100}{\nano\second}$  \\ \colhline
  5   & all    & liquid argon purity & The LAr purity shall be sufficient to enable drift e- lifetime of 3 (10)ms & $<$\,\SI{100}{ppt} & $<$\,\SI{30}{ppt} \\ \colhline
\end{dunetable}
Eleven other high level specifications that affect the
physics are owned by the \dword{exb} (listed in Table~\ref{tab:dunephysicsspecs}), and another twelve high
level engineering specifications are owned by the \dword{exb}.
\fixme{Should the abbreviations in the table be coded?}
\begin{dunetable}
  [\dword{dune} physics-related specifications owned by \dword{exb}]
  {p{0.025\textwidth}p{0.06\textwidth}p{0.2\textwidth}p{0.35\textwidth}p{0.15\textwidth}p{0.1\textwidth}}
  {tab:dunephysicsspecs}
  {\dword{dune} high level system specifications owned by \dword{exb}}
  ID & System & Parameter & Physics Requirement Driver & Requirement & Goal \\ \toprowrule
  6   & APA & Gaps between APAs  & minimize events lost due to vertex in gaps between APAs (15mm on same support beam, 30mm on adjacent beams) & <\SI{30}{mm} & <\SI{15}{mm} \\ \colhline
  7   & DSS & Drift field uniformity & tolerance on drift field due to component location & $<\,\SI{1}{\%}$  &   \\ \colhline
  8   & APA & wire angles  & 0$^\circ$ collection, $\pm$35.7$^\circ$ induction &  &  \\ \colhline
  9   & APA & wire spacing  & \SI{4.669}{mm} for U,V; \SI{4.790}{mm} for X,G &  &  \\ \colhline
  10  & APA & wire position tolerance  & & $\pm\,\SI{0.5}{mm}$  &  \\ \colhline
  11  & HVS & Drift field uniformity & tolerance on drift field due to HVS system & $<\,\SI{1}{\%}$  &  \\ \colhline
  12  & HVS & Cathode power supply ripple & very small compared to intrinsic electronics noise & $<\,\SI{100}{enc}$ &   \\ \colhline
  13  & CE & Frontend peaking time  & optimize vertex resolution & \SI{1}{\micro\second} &  \\ \colhline
  14  & CE & Signal saturation  & largest signals occur with multiple protons in the primary vertex & 500k $e^-$ &  \\ \colhline
  15  & cryo & LAr N$_2$ contamination  & optical attenuation length in liquid argon with 50~ppm of N$_2$ contamination is roughly 3~m & $<\,\SI{25}{ppm}$ &  \\ \colhline
  16  & all & Detector dead time  & risk of missing a supernova burst if all operating cryostats are offline & $<\,\SI{0.5}{\%}$ &  \\ \colhline
\end{dunetable}

Lower level detector specifications are held by the consortia and described in the \dword{dune} \dword{tdr} \dword{dsp}
Volume~\volnumbersp\ and \dword{ddp} Volume~\volnumberdp\ chapters for
each consortium.

The high level \dword{dune} requirements that drive the \dword{lbnf} design are
maintained in DocDB-112 and under change control. These are owned by
the \dword{dune} \dword{tc} and the \dword{lbnf} project manager.

\begin{itemize}
 \item Summary of consortia requirements?
 \item TC requirements (cleanliness, APA spacing, ...)?
\end{itemize}


%%%%%%%%%%%%%%%%%%%%%%%%%%%%%%%%
\section{Value Engineering}
\label{sec:fdsp-coord-ve}

Value engineering is the ongoing process of arriving at cost effective
solutions to the technical challenges of building the \dword{dune}
detector. \dword{dune} value engineering builds on significant
developments in \dword{lar} detectors going back to the 1970s, especially the large \dwords{lartpc} ICARUS and
\dword{microboone}. Prototyping by both LBNE and LBNO has significantly
advanced the value engineering process, leading to constructing
the \dword{protodune} detectors. These detectors validate the
\dword{dune} designs, confirming that necessary performance is
met. Any significant departure from current designs must account for
the success of \dword{protodune} and may require testing in a second
run of \dword{protodune}. The value engineering process is executed
at both the consortia and \dword{tc} level.

For example, the APA \fixme{Should this abbreviation be coded?} size has been optimized over the last 10 years,
using input from dimensions of the Ross shaft, shipping container size,
and long stainless steel tubes available from reliable vendors.
At this point, any significant change to the APA design would likely
lead to new and significant engineering costs.

Do we want to describe highlights of all of the VE that has been done
over the last 10 years? Do we want to discuss possible scope options/reductions?

%%%%%%%%%%%%%%%%%%%%%%%%%%%%%%%%
\section{Lessons Learned}
\label{sec:fdsp-coord-lessons}

A detailed list of lessons learned from constructing and operating
\dword{pdsp} is in DocDB-8255. These lessons have
driven the planning for \dword{dune} and have led to design changes in
\dword{dune}. Lessons learned will continue to be updated throughout the final
design stage and production. The methodologies are described in
Section~\ref{sec:lessons_learned}.

%%%%%%%%%%%%%%%%%%%%%%%%%%%%%%%%
\section{Interface to National Projects}
\label{sec:fdsp-coord-national}

\dword{dune} \dword{tc} works with each consortium leadership team. The
consortia leadership teams comprise the consortia leaders and 
consortia technical leaders as well as others appointed by the consortia
leaders. The consortia leadership teams usually have representatives
from the various national projects that provide the funding to build the consortia
deliverables. If the consortia leadership teams are not directly
embedded in the various national projects building their subsystem,
then the consortia leadership teams represent their national projects
to \dword{tc}. The consortia leadership teams are responsible for
providing their deliverables on time according to the
\dword{ims}. They are responsible for identifying any inconsistencies
between the \dword{ims} and their national project schedules and
bringing these issues to \dword{tc}. The consortia leadership teams
are responsible for reporting progress against the \dword{ims} to
\dword{tc}.

%%%%%%%%%%%%%%%%%%%%%%%%%%%%%%%%
\section{Reporting}
\label{sec:fdsp-coord-reporting}

The \dword{dune} project has provided regular monthly reports
since the final design and construction of \dword{protodune} began in
earnest in summer 2016. The project will continue these reports. Reporting will expand to include monthly reports
against the \dword{ims}. The \dword{dune} project provides regular
reports to the LBNC at reviews several times a year. The \dword{dune} project
produces reports from design, production, and operations reviews.

%%%%%%%%%%%%%%%%%%%%%%%%%%%%%%%%
%\section{Management of Schedule and Risks}
%\label{sec:fdsp-coord-mgmt}


%%%%%%%%%%%%%%%%%%%%%%%%%%%%%%%%
%\section{Design Process}
%\label{sec:fdsp-coord-designprocess}

%The design process follows the engineering safety process and review
%processes described in Chapter~\ref{vl:tc-review}. The current status
%of defining international code equivalences is described in
%Section~\ref{sec:esh_codes}.

%%%%%%%%%%%%%%%%%%%%%%%%%%%%%%%%
%\section{Integration Facility}
%\label{sec:fdsp-coord-itf}
%
%Do we explain the concept of ITF here in general terms? May be better
%in Chapter~\ref{vl:tc-facility}?
