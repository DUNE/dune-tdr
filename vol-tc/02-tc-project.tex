\chapter{Project Functions}
\label{vl:tc-project}

[Steve]

As defined in the \dword{dune} Management Plan (DMP), the \dword{dune}
Technical Board (TB) generates and recommends technical decisions to the
collaboration executive board (EB).
It consists of all consortia scientific and technical leads. It meets
on a regular basis (approximately monthly) to review and resolve any
technical issues associated with the detector construction. It reports
through the EB to collaboration management. The \dword{dune} TB
is chaired by the technical coordinator. The
\dword{tc} engineering team also meets on a regular basis (approximately monthly)
to discuss more detailed technical issues. \Dword{tc} does not have
responsibility for financial issues; that will instead be referred to
the EB and Resource Coordinator (RC).

\Dword{tc} has several major project support tasks that need to be accomplished:
\begin{itemize}
\item Assure that each consortium has a well defined and complete
  scope, that the interfaces between the consortia are sufficiently
  well defined and that any remaining scope can be covered by \dword{tc}
  through \dword{comfund} or flagged as missing scope to the EB and RC. In
  other words, assure that the full detector scope is
  identified. Monitor the interfaces and consortia progress in
  delivering their scope.
\item Develop an overall project \dlong{ims}
  that includes reasonable production schedules, testing plans and a
  well developed installation schedule from each consortium. Monitor
  the \dword{ims} as well as the individual consortium schedules.
\item Ensure that appropriate engineering and safety standards are
  developed and agreed to by all key stakeholders and that these
  standards are conveyed to and understood by each
  consortium. Monitor the design and engineering work.
\item Ensure that all \dword{dune} requirements on \dword{lbnf} for
  conventional facilities, cryostat and cryogenics have been clearly
  defined and understood by each consortium. Negotiate scope
  boundaries with \dword{lbnf}. Monitor \dword{lbnf} progress on
  final conventional facility design, cryostat design and cryogenics
  design.
\item Ensure that all technical issues associated with scaling from
  \dword{protodune} have sufficient resources to converge on
  decisions that enable the detector to be fully integrated and
  installed.
\item Ensure that the integration and \dword{qc} processes for each
  consortium are fully developed and reviewed and that the
  requirements on an \dword{itf} are well defined.
\end{itemize}

\Dword{tc} is responsible for technical quality and schedule and is not
responsible for consortia funding or budgets.  \Dword{tc} will try to help
resolve any issue that it can, but will likely have to push all
financial issues to the TB, EB and RC for resolution.

\Dword{tc} maintains a web
page\footnote{\url{https://web.fnal.gov/collaboration/DUNE/DUNE\%20Project/\_layouts/15/start.aspx\#/}.}
with links to project documents. \Dword{tc} maintains repositories of
project documents and drawings. These include the \dword{wbs},
schedule, risk register, requirements, milestones, strategy, detector
models and drawings that define the \dword{dune} detector.

%%%%%%%%%%%%%%%%%%%%%%%%%%%%%%%%
\section{WBS}
\label{sec:fdsp-coord-wbs}

%%%%%%%%%%%%%%%%%%%%%%%%%%%%%%%%
\section{Cost}
\label{sec:fdsp-coord-cost}

\begin{itemize}
 \item cost model, how many SP/DP detectors
 \item spares, labor categories, ...
 \item summary of consortia costs
 \item TC cost details?
\end{itemize}

%%%%%%%%%%%%%%%%%%%%%%%%%%%%%%%%
\section{MOU}
\label{sec:fdsp-coord-mou}

%%%%%%%%%%%%%%%%%%%%%%%%%%%%%%%%
\section{Budget}
\label{sec:fdsp-coord-budget}

%%%%%%%%%%%%%%%%%%%%%%%%%%%%%%%%
\section{Schedule}
\label{sec:fdsp-coord-controls}

A series of tiered milestones are being developed for the \dword{dune}
project. The Tier-0 milestones are held by the spokespersons and host
laboratory director. Three have been defined and the current milestones and
target dates are:
\begin{enumerate}
\item Start main cavern excavation \hspace{2.6in} 2019
\item Start \dword{detmodule}~1 installation \hspace{2.1in} 2022
\item Start operations of \dword{detmodule}~1--2 with beam \hspace{1in} 2026
\end{enumerate}
These dates will be revisited at the time of the \dword{tdr} review.  Tier-1
milestones will be held by the technical coordinator and \dword{lbnf} Project
Manager and will be defined in advance of the \dword{tdr} review. Tier-2
milestones will be held by the consortia.

A high level version of the \dword{dune} milestones from the \dword{ims}
can be seen in Table~\ref{tab:DUNE_schedule}.
\begin{dunetable}
  [Overall \dword{dune} Project Tier-1 milestones.]
  {p{0.84\linewidth}p{0.14\linewidth}}
  {tab:DUNE_schedule}
  {Overall \dword{dune} Project Tier-1 milestones.}
  Milestone & Date   \\ \toprowrule
  RRB Approval of Technical Design Review                       & 09/02/2019 \\ \colhline
  Beneficial Occupancy of Integration Test Facility             & 09/01/2021 \\ \colhline
  Construction of steel frame for Cryostat \#1 complete         & 12/17/2021 \\ \colhline
  Construction of Mezzanine for Cryostat \#1 complete           & 01/17/2022 \\ \colhline
  Begin integration/testing of Detector \#1 components at ITF   & 02/01/2022 \\ \colhline
  Beneficial Occupancy of Central Utility Cavern Counting room  & 04/16/2022 \\ \colhline
  Construction of steel frame for Cryostat \#2 complete         & 07/01/2022 \\ \colhline
  Construction of Mezzanine for Cryostat \#2 complete           & 08/01/2022 \\ \colhline
  \textbf{Beneficial occupancy of Cryostat \#1}                 & \textbf{12/23/2022} \\ \colhline
  Cryostat \#1 ready for TPC installation                       & 05/01/2023 \\ \colhline
  Begin integration/testing of Detector \#2 components at ITF   & 11/01/2023 \\ \colhline
  \textbf{Beneficial occupancy of Cryostat \#2}                 & \textbf{03/01/2024} \\ \colhline
  Begin closing Temporary Construction Opening for Cryostat \#1 & 05/01/2024 \\ \colhline
  Cryostat \#2 ready for TPC installation                       & 08/01/2024 \\ \colhline
  Cryostat \#1 ready for filling                                & 10/01/2024 \\ \colhline
  Begin closing Temporary Construction Opening for Cryostat \#2 & 07/18/2025 \\ \colhline
  \textbf{Detector \#1 ready for operations}                    & \textbf{10/01/2025} \\ \colhline
  Cryostat \#2 ready for filling                                & 12/05/2025 \\ \colhline
  \textbf{Detector \#2 ready for operations}                    & \textbf{12/18/2026} \\
\end{dunetable}

\Dword{tc} will maintain the \dword{ims} that links all consortium schedules
and contains appropriate milestones to monitor progress. The \dword{ims}
is envisioned to be maintained in MS-Project~\footnote{MicroSoft\texttrademark{} Project.} as it is expected
that many consortia will use this tool. It is currently envisioned as
three levels of control and notification milestones in addition to the
detailed consortium schedules. The highest level contains external
milestones, with the second level containing the key milestones for \dword{tc}
to monitor deliverables and installation progress, and the third level
containing the inter-consortium links. The schedules will go
under change control after agreement with each consortium on the
notification milestone dates and the \dword{tdr} is approved.

In addition to the overall \dword{ims} for construction and
installation, a schedule of key consortia activity in the period
2018--19 leading up to the \dword{tdr} has been developed.

To ensure that the \dword{dune} detector remains on schedule,
\dword{tc} will monitor schedule statusing from each consortium, will organize
reviews of schedules and risks as appropriate.  As schedule problems
arise \dword{tc} will work with the affected consortium to resolve the
problems. If problems cannot be solved, \dword{tc} will take the issue to the
TB and EB.

A monthly report with input from all consortia will be published by
\dword{tc}. This will include updates on consortium technical progress and
updates from \dword{tc} itself.

%%%%%%%%%%%%%%%%%%%%%%%%%%%%%%%%
\section{Risks}
\label{sec:fdsp-coord-risks}

\begin{itemize}
 \item Summary of high level risks
 \item Details of TC risks?
\end{itemize}


From IDR:

The successful operation of \dword{protodune} will retire a great many
potential risks to \dword{dune}. This includes most risks associated with the
technical design, production processes, \dword{qa}, integration
and installation. Residual risks remain relating to design and
production modifications associated with scaling to \dword{dune}, mitigations
to known installation and performance issues in \dword{protodune}, underground
installation at \surf and organizational growth.

The highest technical risks include: development of a system to
deliver \SI{600}{kV} to the \dual cathode; general delivery of the
required \dword{hv}; cathode and \dword{fc} discharge to the cryostat
membrane; noise levels, particularly for the \dword{ce}; %cold TPC electronics,
number of dead channels; lifetime of components surpassing \dunelifetime{}; %20 years,
\dword{qc} of all components; verification of improved \dword{lem}
performance; verification of new cold  \dword{adc} and  \dword{coldata} performance;
argon purity; electron drift lifetime; \phel light yield;
incomplete calibration plan; and incomplete connection of design to
physics. Other major risks include insufficient funding, optimistic
production schedules, incomplete integration, testing and installation
plans.

Key risks for \dword{tc} to manage include the following:
\begin{enumerate}
\item Too much scope is unaccounted for by the consortia and falls
  to \dword{tc} and \dword{comfund}.
\item Insufficient organizational systems are put into place to
  ensure that this complex international mega-science project,
  including \dword{tc}, \fnal as host laboratory, \surf, DOE and all international
  partners continue to successfully work together to ensure
  appropriate rules and services are provided to enable success of
  the project.
\item Inability of \dword{tc} to obtain sufficient personnel resources so as to
  ensure that \dword{tc} can oversee and coordinate all of its
  project tasks.  While the USA has a special responsibility towards
  \dword{tc} as host country, it is expected that personnel resources will
  be directed to \dword{tc} from each collaborating country. Related to this
  risk is the fact that consortium deliverables are not really
  stand-alone subsystems; they are all parts of a single \dword{detmodule}. This
  elevates the requirements on coordination between consortia.
\end{enumerate}

The consortia have provided preliminary versions of risk analyses that
have been collected on the \dword{tc} webpage. These are being developed into
an overall risk register that will be monitored and maintained by \dword{tc}
in coordination with the consortia.

%%%%%%%%%%%%%%%%%%%%%%%%%%%%%%%%
\section{Requirements}
\label{sec:fdsp-coord-requirements}

\begin{itemize}
 \item Summary of consortia requirements?
 \item TC requirements (cleanliness, APA spacing, ...)?
\end{itemize}

%%%%%%%%%%%%%%%%%%%%%%%%%%%%%%%%
\section{Value Engineering}
\label{sec:fdsp-coord-ve}

Principles and expectations for value engineering by consortia

%%%%%%%%%%%%%%%%%%%%%%%%%%%%%%%%
\section{Lessons Learned}
\label{sec:fdsp-coord-lessons}


%%%%%%%%%%%%%%%%%%%%%%%%%%%%%%%%
\section{Interface to National Projects}
\label{sec:fdsp-coord-national}


%%%%%%%%%%%%%%%%%%%%%%%%%%%%%%%%
\section{Reporting}
\label{sec:fdsp-coord-reporting}


%%%%%%%%%%%%%%%%%%%%%%%%%%%%%%%%
\section{Management of Schedule and Risks}
\label{sec:fdsp-coord-mgmt}


%%%%%%%%%%%%%%%%%%%%%%%%%%%%%%%%
\section{Design Process}
\label{sec:fdsp-coord-designprocess}

FNAL stds and international stds

%%%%%%%%%%%%%%%%%%%%%%%%%%%%%%%%
\section{Integration Facility}
\label{sec:fdsp-coord-itf}

Do we explain the concept of ITF here in general terms?

