\chapter{Environment, Safety, and Health}
\label{vl:tc-ESH}

\begin{comment}
A strong \dword{esh} program is essential to successfully complete
\dword{lbnf} and \dword{dune} at \dword{surf}. % and hosted by \dword{fnal}.
\dword{lbnf-dune} is internationally designed, coordinated and funded
through collaborating laboratories and universities.  \dword{lbnf}
comprises the world's highest intensity neutrino beam at \fnal and the
infrastructure necessary to support the large cryogenic experimental
detectors at \dword{surf}.
\end{comment}

\dword{lbnf-dune} is committed to protecting the health and safety of
staff, the community, and the environment, as stated in the
\dword{lbnf-dune} integrated \dword{esh} plan, as well as to ensuring a
safe work environment for \dword{dune} workers at all institutions and
protecting the public from hazards associated with constructing and
operating \dword{dune}.  Accidents and injuries are preventable, and
the \dword{esh} team will work with the global \dword{lbnf-dune}
project and collaboration to establish an injury-free workplace.
All work will be performed so as to preserve  the quality of the environment and
prevent property damage.

The \dword{lbnf-dune} \dword{esh} program complies with applicable
standards and local, state, federal, and international legal
requirements through the \dword{fnal} Work Smart set of standards and the
contract between \dword{fra} and the \dword{doe}
Office of Science (FRA-DOE). \fnal, as the host laboratory,
established the \dword{sdsd} to provide facility support.
\dword{sdsd} is responsible for support of \dword{lbnf-dune}
operations at \dword{surf}.

The \dword{lbnf-dune} \dword{esh} program strives to prevent
injuries or illness and seeks to continually improve safety and health
management.  To the maximum practical extent, all hazards must be
eliminated or minimized through substitution, or engineering or
administrative controls.  Where engineering or administrative controls
are not feasible, workers will use \dword{ppe}.

The \dword{lbnf-dune} \dword{esh} management system is
designed to work hand-in-hand with the \dword{surf} emergency
management systems to protect the public, workers, and the environment;
ensure compliance with the FRA-DOE contract and \fnal Work Smart
standards; and improve the \dword{dune} ability to meet or
exceed stakeholder expectations and execute the
scientific mission.  \dword{dune} uses a set of criteria to plan, direct,
control, coordinate, assure, and improve how \dword{esh} policies,
objectives, processes, and procedures are established, implemented,
monitored and achieved.

The \dword{lbnf} facilities at \dword{surf} are subject to
the requirements of the \dword{doe} Worker Safety and Health Program,
Title 10, Code Federal Regulations, Part 851 (10 CFR 851). These
requirements are promulgated through the \fnal Director Policy
Manual\footnote{\fnal Director's Policy Manual is:
  http://www.fnal.gov/directorate/Policy\_Manual.html}, and the \fnal
\dword{esh} Manual\footnote{\fnal \dword{esh} Manual is:
  http://esh.fnal.gov/xms/ESHQ-Manuals/feshm} (\dword{feshm}), which
align with the \dword{surf} \dword{esh} manual.


\section{\dshort{lbnf-dune} \dshort{esh} Management and Oversight}

The \dword{tcoord} and \dword{ipd} have responsibility for
implementation of the \dword{dune} \dword{esh} program for the construction and installation activities, respectively.  The
\dword{lbnf-dune} \dword{esh} manager reports to the
\dword{tcoord} and \dword{ipd} and is responsible for providing
\dword{esh} support and oversight for development and implementation of the 
\dword{lbnf-dune} \dword{esh} program. Figure~\ref{fig:dune_esh} shows
the \dword{lbnf-dune} \dword{esh} organization.


The \dword{dune} \dword{esh} coordinator reports to the
\dword{lbnf-dune} \dword{esh} manager and has primary responsibility
for \dword{esh} support and oversight of the \dword{dune} \dword{esh}
program for activities at collaborating insitutions as shown in
Figure~\ref{fig:dune_esh_construction}.  The far and near site
\dword{esh} coordinators are responsible for providing daily field support and
oversight for all installation activities at the \dword{surf}
and \dword{fnal} sites, as shown in Figure~\ref{fig:dune_esh_installation}.

Additional \dword{esh} subject matter experts (SMEs) are available to
provide supplemental support to the project through the \fnal
\dword{esh} Section. The \fnal \dword{esh} Section, FRA-DOE, CERN and
\dword{surf} will provide supplemental \dword{esh} oversight to
validate implementation of the \dword{lbnf-dune} \dword{esh}
program. FRA-DOE maintains a daily oversight presence at the far and
near sites.

The \dword{lbnf-dune} \dword{esh} plan defines the \dword{esh}
requirements applicable to installation activities at the \dword{surf}
site. Regular \dword{esh} walkthroughs will be conducted by
\dword{lbnf-dune} \dword{esh} management personnel. All
findings will be documented in the \fnal Predictive Solutions
database system.

\section{National Environmental Protection Act Compliance}

In compliance with the National Environmental Protection Act (NEPA) and in
accordance with \dword{doe} Policy 451.1, the
\dword{lbnf-dune} project performed an assessment  of %evaluation of %potential
environmental impacts that are possible during the construction and operation of the
project.  %An environmental assessment  was prepared %on 
This assessment~\cite{bib:docdb122}   identifies the
potential environmental impacts and the safety and health hazards
%identified 
that could occur or be present during the design, construction, and operating phases of
\dword{lbnf-dune}.  The environmental assessment presented an
analysis of the potential environmental consequences of the facility
and compared them to the consequences of a No Action Alternative. %The assessment 
It also included a detailed analysis of all potential environmental,
safety, and health hazards associated with construction and operation
of the facility.  The environmental assessment has been completed and
a finding of no significant impact (FONSI)~\cite{bib:docdb122} issued in September 2015.


\section{Codes/Standards Equivalencies}
\label{sec:esh_codes}

\dword{dune} will rely on significant contributions from international
partners. In many cases, an international partner will contribute
equipment for installation at \dword{fnal} or \dword{surf}, built
following international standards. \dword{fnal} has established a
process under the international agreement with \dword{cern}, detailed
in \dword{feshm}~\cite{FNAL:FESHM2000} Chapter 2110, to establish code equivalency between
USA and international engineering design codes and standards. This
process allows the laboratory to accept in-kind contributions from
international partners or purchase equipment designed using
international standards while ensuring an equivalent level
of safety.

At the time of this writing, \dword{fnal} has completed the following code
equivalencies:
\begin{itemize}
 \item pressure vessels designed using EN13445;
 \item structures designed using EN 1990, EN 1991, EN1993, EN 1999 (a
   subset of the Eurocodes), and EN 14620;
 \item CE-marked pressure piping systems designed using PED 97/23 EN 13480;
 \item CE-marked relief valves designed using PED 2014/68/EU EN ISO 4126;
 \item CE-marked electrical equipment for measurement and control; and
   laboratory use designed using IEC 61010-1 and IEC 61010-2-030.
\end{itemize}

As necessary, the laboratory code equivalency process will be followed
to establish equivalency to other international codes and
standards. The current list of completed code equivalencies can be
found in~\cite{bib:eshdocdb3303}. % the \dword{esh}Q Section Document Database DocDB-3303\footnote{\url{https://esh-docdbcert.fnal.gov/cgi-bin/cert/ShowDocument?docid=3303}}. 


\section{\dshort{esh} Requirements at Collaborating Laboratories and Institutions}

All work performed at collaborating institutions will be completed
following that institution's \dword{esh} policies and
programs. Equipment and operating procedures provided by the
collaborating institution will conform to the \dword{dune} project
\dword{esh} and integrated safety management policies and
procedures. The \dword{esh} organization at collaborating institutions
provides \dword{esh} oversight for work activities carried
out at their facilities.
%\dword{lbnf-dune} personnel will also follow the \dword{esh} manual and procedures at the collaborating institutions.

\section{\dshort{lbnf-dune} \dshort{esh} Program at \dshort{surf}}

\subsection{Site and Facility Access}

All \dword{dune} workers requiring access to the \dword{surf} site
register through the \fnal Global Services Office to receive the
necessary user training and a \fnal identification number that can be
used to apply for a \dword{surf} identification badge, through the
\dword{surf} Administrative Services Office. This is coordinated by
\dword{sdsd}. The \dword{surf} identification badge allows access to
the \dword{surf} site as part of the \dword{surf} Site Access Control
Program. The \fnal Global Services Office has extensive experience
with international collaborators.


\dword{surf} underground access will require that working groups
obtain a \dword{tap} for each daily access to the
underground areas.  All personnel within each working group must be
individually listed on the \dword{tap}, per the \dword{surf} Site
Access Control Program. All personnel are required to ``brass in and
out'' via the brass board located at the entrance to the Ross Cage prior to accessing the underground facilities.

\subsection{\dshort{esh} Training}


All personal performing work onsite at \dword{surf} are required to
attend \dword{surf} \dword{esh} Site Orientation prior to their work
at the site.  This includes \dword{surf} Surface and Underground
training classes, as well as associated Cultural Heritage
training. Arrangements will be made for all workers to complete this
training. In addition, unescorted-access training will be provided to
personnel for each underground working level (4850L and 4910L) at
which they will perform work.  The \dword{lbnf}/\dword{dune}
\dword{esh} management team will present a project-specific
introductory \dword{esh} presentation.

\subsection{Personnel Protective Equipment}

Personal protective equipment (PPE) is not a substitute for
engineering and administrative controls. These controls will be
implemented, to the extent feasible, to mitigate the hazard so that
the need for \dword{ppe} is reduced or eliminated.

Personnel must wear the following \dword{ppe} when on site the \dword{surf} site.
\begin{itemize}
\item At a minimum, all workers personnel shall wear
  steel-toed boots, long pants and shirts with 4~inch sleeves when
  performing non-office work at \dword{surf}. 
  \item All personnel entering the work site at the Ross (or Yates) Dry
    shall wear hard hats (brim facing forward), gloves,
    safety glasses with rigid side-shields and a reflective high
    visibility (e.g., orange) shirt, coat, or vest (minimum ANSI Class 2).
    Exceptions to these minimum requirements will be approved by the
    \dword{lbnf-dune} \dword{esh} manager and noted in the
    activity-specific \dword{ha} maintained by the \dword{dune} \dword{esh} coordinator.
  \item When working underground, all personnel will carry an Ocenco M20.2 self rescuer
    and a hard hat cap lamp.  Ocenco 7.5 emergency breathing apparatus devices
    will be stored underground for additional emergency support.
  \item Hard hats must meet the ANSI Z89.1 standard as defined by 29
    CFR 1926.100 and bear the Z89-.1 designation. 
   \item Eye protection must meet the requirement of 29 CFR
      1926.102. Safety glasses must be ANSI approved and be marked
      with the ANSI marking Z87.1 designation.
    \item Hearing protection must be appropriate to the work environment, as
      defined in the activity-based \dword{ha}.
    \item Workers will don any specialized \dword{ppe} required for specific work tasks as
      defined in the activity-based \dword{ha}. 
\end{itemize}

\subsection{Work Planning and Controls}

\begin{comment} Anne reworking
The goal of the work planning and \dword{ha} process is to
initiate careful thought about the hazards associated with work and
how the work can be performed safely. Careful planning of a job assures that
it is performed efficiently and safely. Work planning ensures the
scope of the job is understood, appropriate materials are available,
all hazards have been identified, mitigation efforts have been established, and
all affected employees understand what is expected of them. \dword{ha}
 is a critical part of work planning. All work activities
are subject to work planning and \dword{ha}. Depending on
the complexity of the task and the hazards involved, the \dword{ha} process
may be a mental exercise and verbal discussion, or it may be more
formal with a written hazard analysis and pre-job briefing. The Work
Planning and \dword{ha} program is documented in Chapters 2060 in
the \dword{feshm}. All work planning documentation is reviewed and
approved by \dword{dune} \dword{esh} coordinator and the \dword{dune}
\dword{irr} or \dword{orr} committees prior to the start of work activities.
\end{comment}

The goal of the work planning and \dword{ha} process is to determine how to do
the work safely, correctly, and efficiently. All work activities
are subject to work planning and \dword{ha}. 
The first steps are to 
initiate careful thought about the work, determine the scope of the job, identify the potential hazards associated with it, and determine mitigation strategies.
Depending on the complexity of the job and the hazards involved, the \dword{ha} process
may be limited to a mental exercise and verbal discussion, or it may be more
formal with a written \dword{ha} and pre-job briefing.
We ensure that all affected employees understand the full work plan and what is expected of them, and that they have all the appropriate materials to do the job properly.
The Work Planning and \dword{ha} program is documented in Chapters 2060 in the \dword{feshm} manual~\cite{feshm}. All work planning documentation is reviewed and
approved by the \dword{dune} \dword{esh} coordinator and the \dword{dune}
\dword{irr} or \dword{orr} committees prior to the start of work activities.

A work planning meeting will be held %each day/shift 
before each shift. %, prior to thestart of the shift's daily work activities.  
The meeting %should be 
is led by the shift supervisor, supported by the \dword{dune} \dword{esh} coordinator, and
attended by all personnel working on site during that shift. The meeting is intended to inform the workers of potential safety hazards and
hazard mitigations relating to the various work activities, ensure
that employees have the necessary \dword{esh} training and \dword{ppe}, answer any
questions relating to the work activities, and authorize the work
activities for that shift. The meeting is expected to last approximately 15 minutes, but 
may vary depending on the activities planned for the shift. 

A Safety Data Sheet (SDS) will be available for all chemicals and
hazardous materials that are used on site. All chemicals and hazardous
materials brought to the \dword{surf} site must be reviewed and approved by the
\dword{dune} \dword{esh} coordinator and the \dword{surf} \dword{esh}
Department before arriving at site.  SDS documentation will be
submitted to the \dword{dune} \dword{esh} coordinator prior to the
material arriving on site.

\subsection{Emergency Management}

Any injuries, accidents, or spills are to be reported immediately to the
\dword{lbnf-dune} \dword{esh} manager, from either the \dword{surf} Emergency
Response Coordinator or the installation manager, through the \dword{dune} far site
\dword{esh} coordinators and the \dword{dune} \dword{esh} coordinator. %All 
Any personnel that
experience any injury will be sent (or transported, if needed) to the Black Hills
Medical Clinic or Regional Health Lead-Deadwood hospital. %, as appropriate.  
The supervisor  completes an initial incident
investigation report and submits the report to the \dword{lbnf-dune}
\dword{esh} manager within 24 hours.


The \dword{surf} \dword{esh}
Manual\footnote{\url{http://sanfordlab.org/esh}} maintains the Emergency
Management and Emergency Response Plan 
(ERP)\footnote{\url{http://sanfordlab.org/esh/manual/32-emergency-response-plan-erp-policy}}
for the site. All personnel will receive ERP training and the ERP
flowchart for emergency notification process is posted at all
telephones. For all emergencies at the \dword{surf} site, personnel
contact Emergency Response personnel by using any building
phone, dialing the hoist operator, or by calling 911 from any outside
line (e.g., cell phone).

Emergency Response Groups that are not part of \dword{surf} but are recognized
outside resources are the Lawrence County Emergency Manager, the Lead and
Deadwood Fire Departments, Lawrence County Search and Rescue, Black
Hills Life Flight, and the Rapid City Fire Department HAZMAT team. The
\dword{surf} ERP is distributed to these resources to facilitate outside
emergency response.

\dword{sdsta} will maintain an Emergency Response incident command
system and an \dword{ert} on all shifts that can access the
underground sites with normal surface fire department response
times. This team provides multiple response capabilities for both
surface and underground emergencies but specializes in underground
rescue through MSHA Metal/Non-Metal Mine Rescue training. %The
%\dword{ert} has a defined training schedule, conduct regular
%walkthroughs in areas of response and emergency drills are conducted
%on both the surface and underground sites, for which all personnel on
%site are required to participate. \dword{ert} personnel includes a
%minimum of one emergency medical technician (EMT) and/or Paramedic per
%shift.
The \dword{ert} has a defined training schedule and conducts regular
walkthroughs in areas of response. The team conducts  emergency drills 
on both the surface and underground sites, in which all personnel on
site are required to participate. \dword{ert} personnel includes a
minimum of one emergency medical technician (EMT) and/or paramedic per
shift. 

\dword{surf} implements a guide program for both the surface and
underground areas. The guide program has an established training
program.  Visitors and other untrained personnel must be escorted by a trained
guide when on site at \dword{surf}. %The underground guide program requires %at a
In addition, a minimum of one guide is stationed on each %working 
level underground at all times where work is occurring. This guide provides
supplemental emergency support to unescorted, access-trained
personnel. Guides are trained as first responders to help in a medical
emergency until the ERT arrives.

In the case of an underground emergency such as fire or \dword{odh},
evacuation to the surface takes place through the Ross or Yates Shafts.
If full evacuation is not  possible, the refuge chamber on the 4850L
 can shelter up to 144 persons for 98 hours.

\subsection{Fire Protection, \dshort{odh} and Life Safety}

The \dword{dune} installation team is required to police their work areas
frequently and maintain good housekeeping. Teams generating common garbage and other
waste must disposed of it at frequent, regular intervals. % by the teams generating the waste. 
Containers will be provided for the
collection and separation of waste, trash, oily or used rags, and other
refuse.  Containers used for garbage and other oily, flammable, or
hazardous wastes, (e.g., caustics, acids, or harmful dusts) will be equipped with covers.  Chemical agents or
substances that might react to create a hazardous condition, must
be stored and disposed of separately, as assessed by the
\dword{lbnf-dune} \dword{esh} coordinator.

The \dword{lbnf-dune} \dword{esh} coordinator collects SDS documentation for chemicals and hazardous materials and determines proper storage cabinets for them.  The documentation is made readily available with the materials in the cabinets for the workers. 

%with SDS documentation (collected by the \dword{lbnf-dune} \dword{esh} coordinator) readily available. The\dword{lbnf-dune} \dword{esh} coordinator will evaluateappropriate storage cabinets for the \dword{ipd}. 


All open flame, welding, cutting, or grinding work activities require completion and approval of a \dword{fnal} ``hot'' work permit.  The \dword{dune}
\dword{esh} coordinator coordinates the issuance of the permit.
The team completing the work will be responsible for
providing all the required materials, personnel, and \dword{ppe} %protective equipment 
to conduct the hot work. All hot work permits must be
provided to the \dword{surf} \dword{esh} Department.

Cables installed for \dword{dune} are chosen to be
consistent with current \fnal standards for cable insulation and must
comply with recognized standards concerning cable fire resistance. 
This reduces the probability of a fire starting and of adverse health effects due to
combustion products of cable insulation materials.

Fire and life safety requirements for \dword{lbnf-dune} areas were
analyzed in the \dword{lbnf-dune} Far Site Fire and Life Safety
  Assessment~\citedocdb{14245}. ARUP provided code analysis, fire
modeling, egress calculations, and the design of the fire protection
features for \dword{lbnf} \dword{fscf}.  Additionally, the ARUP
consultant, SRK Consulting, modeled additional fire scenarios and the potential
spread of toxic fumes and heat in the drifts used by
\dword{lbnf-dune} for evacuation, verifying that the system design and evacuation 
    processes will be safe.   All caverns will be equipped with
fire detection and suppression systems, with both visual and audible
notification.  All fire alarms and system supervisory signals will be
monitored in the \dword{surf} Incident Command Center.  The
\dword{surf} \dword{ert} will respond with additional support from the
Lead and Deadwood Fire Departments and the county’s emergency management
department.

\dword{odh} requirements were assessed through the \dword{lbnf-dune}
\dword{odh} analysis. The caverns have been classified as \dword{odh} 1 and
the drifts are classified as  \dword{odh} 0. The caverns will be equipped with
an \dword{odh} monitoring and alarm system, with independent visual and
audible notification systems.  All \dword{odh} alarms and system
supervisory signals will be monitored in the \dword{surf} Incident
Command Center.  Each occupant entering an  \dword{odh} area will receive  \dword{odh}
training and carry Ocenco M20.3 escape packs.

Emergency conditions from smoke or \dword{odh} incidents underground are
primarily mitigated by the large ventillation rate in the \dword{surf}
underground area.

The facility emergency management plan will be reviewed and updated as
necessary during construction, installation, and operation activities
based on the egress strategy defined in the ARUP Fire and Safety
Report and the  \dword{surf} Emergency Management
  Plan\footnote{\url{http://sanfordlab.org/ehs/manual/31-emergency-management-policy}}.

Radon levels are presently monitored in the occupied underground
facilities. This monitoring program will extend to the \dword{lbnf}
underground areas in coordination with \dword{fnal} and \dword{sdsta}
\dword{esh} personnel.

\subsection{Earthquake Design Standards}

For surface and underground structures, the design standard is the
latest edition of the International Building Code (IBC) (2018), wherein
Chapter 16  is Structural Design and Section 1613 is Earthquake
Loads. The Lead seismic region, according to the American Society of
Civil Engineers (ASCE 7), is between the lowest risk (the same as that 
for \fnal) and the next level up. Both are minimal risks.

\subsection{Material handling and Equipment Operation}

All overhead cranes, gantry cranes, fork lifts, and motorized equipment,
e.g., trains and carts, will be operated only by trained
operators. Other equipment, e.g., scissor lifts, pallet jacks, hand
tools, and shop equipment, will be operated only by personnel trained
for the particular piece of equipment. 

Hoisting and rigging operations will be evaluated and planned.  A
member of the trained rigging team must identify the hazards and
determine the controls necessary to maintain an acceptable level of
risk.  A Hoisting and Rigging Lift Plan is required for complex and
critical lifts. This plan must be documented using the \fnal Hoisting
and Rigging Lift Plan or similar plan accepted by \fnal. The \dword{ha}
 documentation will include the development of critical lift
plans for specific phases of installation. % activities.

All equipment operating in the underground facility will be diesel or
electric powered. Diesel is allowed due to the large
ventillation rate in the underground area.  There will be no gasoline or propane powered
equipment in the underground facility.

\subsection{Stop Work Authority}

If a worker identifies any unanticipated or unsafe conditions %are identified, 
or non-compliant
practices occurring in their work activity, % are observed during construction activities, 
the trained worker is empowered and expected to stop the activity
 and notify their supervisor and \dword{dune} \dword{esh} coordinator of
this action. All workers on the \dword{dune} project have the
authority to stop work in any situation that presents an imminent
threat to safety, health, or the environment. Work may not resume
until the circumstances are investigated and the deficiencies corrected,
including the concurrence of the \dword{dune} \dword{ipd}
and \dword{lbnf-dune} \dword{esh} manager.

%%% Anne stopped here 11/6

\subsection{Operational Readiness}

The \dword{dune} review process consists of design, production,
installation, and operation reviews as described in
Chapter~\ref{vl:tc-review}. These reviews include lifting fixture
load testing and work planning and controls documentation. The
\dword{jpo} \dword{ro}  is involved at all stages of the review
process. All major stakeholders (including \dword{fnal}, \dword{surf} and
\dword{dune} collaborating institutions) will be involved as
appropriate. The \dword{ro}  will complete both system and process
readiness reviews to authorize installation activities at
\dword{surf}.  Operational readiness reviews will be completed prior
 to the operation of detector components.

\subsection{Lessons Learned}

The \dword{lbnf} project is currently working with \dword{sdsta} and the 
engineering consultant ARUP to implement \dword{esh} procedures and
protocols for training, emergency management, fire
protection, and life safety. The \fnal \dword{esh} Section, \dword{doe}, and
\dword{lbnf} \dword{esh} have completed a series of assessments of
critical \dword{sdsta} \dword{esh} programs including underground access,
emergency management, electrical safety, rigging, and fire
protection. The findings and lessons learned identified in these
\dword{esh} program assessments are tracked within the \fnal issues management
database, iTrack.

\dword{fnal} completed a review to
identify critical lessons learned from the previous underground
neutrino project NuMI/MINOS  in May 2009. The findings from this
exercise were documented in a report entitled Executive Summary of
Major NuMI Lessons Learned.  We are using the \dword{dune} lessons learned from
\dword{protodune}% are being utilized 
to further develop and enhance
the \dword{dune} engineering review and work planning and controls
processes.

Lessons learned are disseminated in areas of applicability and
flowed-down for appropriate implementation. Any action items
associated with lessons learned are tracked in iTrack. Lessons learned
are reviewed and evaluated by both \dword{fnal} and \dword{lbnf-dune} management.

\begin{comment}
The \dword{lbnf} project is presently implementing \dword{esh}
programs required for site access, training, work planning, and
emergency management for construction activities on the \dword{surf}
site. During the implementation of these programs, lessons learned will
be identified and addressed to improve the implementation of the \dword{lbnf-dune}
\dword{esh} programs.  The \dword{esh} programs will be fully
established and implemented when \dword{dune} activities start at
\dword{surf}.
\end{comment}