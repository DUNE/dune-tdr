\chapter{Integration Engineering}
\label{sec:fdsp-coord-integ-sysengr}
%\chapter{Integration}
%\label{vl:tc-integration}

The focus of the integration engineering is the mechanical and
electrical configuration and interfaces between each of the detector
systems. This includes verification that subassemblies and their
interfaces are correct (e.g., \dword{apa} and \single \dword{pds}). A
second major area is in the support of the \dwords{detmodule} and
their interfaces to the cryostat and cryogenics. A third major area is
assuring that the \dwords{detmodule} can be installed --- that the
integrated components can be moved into their final configuration. A
fourth major area is in the integration of the \dwords{detmodule} with
the necessary services provided by the conventional facilities.


The \dword{tc} engineering team maintains configuration data in
the appropriate format for the management of the detector
configuration. The consortia provide engineering data for their
subsystems to \dword{tc}. The \dword{tc} engineering team will work
with the \dword{lbnf} project team to integrate the full detector data
into the global \dword{lbnf} configuration files.

These tasks are carried out utilizing tools and practices
described in the rest of this chapter.

\section{Integration Models}
\label{sec:fdsp-coord-integ-models}
The single phase and dual phase detectors are large and made from many
intricate components. Fortunately, and for the most part, the
components are repetitive and are not overly complex geometrically. As
such, 3D mechanical modelling techniques are well suited for
representation of the detectors and management of its configuration.


At the same time, 3D modelling techniques are varied in the way items
are represented and in the way the techniques are practiced. This
requires that a set of 2D integration drawings are generated which are
clear and unambiguous across the collaboration. Such 2D drawings form
the basis of the 3D models and the basis of engineering design of all
components.

\subsection{Static Models}
\label{sec:fdsp-coord-integ-static}
Technical coordination is responsible for generating and maintaining
3D detector integration models and generating and maintaining
2D integration drawings. These models are static as they represent all
components at their design dimension. They do not represent effects of
gravity, tolerances, cold temperature and installation and assembly
clearances.

The 3D models are assembled by combining component models from various
consortia. The 3D model files are shared with the consortia in a
read-only fashion. The 2D integration drawings are then generated and
disseminated which show the interfaces to the level of detail that is
necessary to ensure proper fit and function. Issues that arise are
communicated with the consortia and resolution method is
determined. It is important to note that \dword{tc} engineering team
will not make any changes to any of the consortia component models.

It is the responsibility of the consortia to resolve any issues per
agreed method and provide updated models for reintegration. As such
models are kept in synchronization with only one-way integration and
without modification by anyone other than the consortia.

The level of detail in a model is managed actively as inordinate
levels of detail lead to large file sizes when models are combined and
then incorporated into global models of facilities. It is the
responsibility of the \dword{tc} engineering team to ensure the
appropriate level of detail is incorporated at each stage of model
integration.

Figure~\ref{fig:dune-sp_overall} shows the overall model of the Single
Phase Detector. One wall of the cryostat is removed for the interior
components to be visible. The detector is made from 25 rows.  Which
are the same in construction. At the end of row one and row 25, end
walls are installed which close the detection volume.
\begin{dunefigure}[Overall model of the Single-Phase Detector]{fig:dune-sp_overall}
  {Overall model of the Single-Phase Detector.}
  \includegraphics[width=0.85\textwidth]{Overall_model_SP.pdf}
\end{dunefigure}


As mentioned earlier, this model does not include all the details of
the detector components. The components are simplified to keep the
overall model complexity to a manageable level.

Figure~\ref{fig:dune-sp_row} shows the model of one row of the
detector module. Each row is made from six Anode Plane Assemblies,
four Cathode Plane Assemblies and eight Filed Cage plus Ground Plane
Assemblies. A total of 25 rows comprise one \dword{dsp} far detector.
\begin{dunefigure}[Overall model of one row of the Single-Phase Detector]{fig:dune-sp_row}
  {Model of one row of the Single-Phase Detector.}
  \includegraphics[width=0.85\textwidth]{Model_one_row_SP.pdf}
\end{dunefigure}


Figure~\ref{fig:dune-sp_transverse} and~\ref{fig:dune-sp_long} show
the cross sections of the detector in the transverse and longitudinal
directions respectively. Overall dimensions are also shown.
\begin{dunefigure}[Section view of the Single Phase Detector in the transverse direction.]{fig:dune-sp_transverse}
  {Section view of the Single Phase Detector in the transverse direction.}
  \includegraphics[width=0.85\textwidth]{SP_transverse_section.pdf}
\end{dunefigure}
\begin{dunefigure}[Overall model of the Single Phase Detector in the longitudinal direction]{fig:dune-sp_long}
  {Overall model of the Single Phase Detector in the longitudinal direction.}
  \includegraphics[width=0.85\textwidth]{SP_longitudinal_section.pdf}
\end{dunefigure}

(Note: Need analogous figures and explanations for Dual Phase)

\subsection{Envelope and Assembly Models}
\label{sec:fdsp-coord-integ-envelope}
Static models represent the detector and its components at their
design dimension exactly. Such exact dimensions are needed for the
detail component drawings and model to be completely compatible at all
times.

For installation and operation, other envelope models are needed which
will be more approximate in nature. Envelope models will be developed
to account for issues that impact installation and operation:
\begin{enumerate}
 \item Effects on the detector caused by distortion of the cryostat
   and detector support structure due to gravity,
 \item Effects on the detector caused by thermal contraction during
   detector filling and operation,
 \item Overall envelope models due to effects of component and
   assembly tolerances,
 \item Clearances needed during installation and envelopes needed for
   access and tooling,
 \item Reference models and drawings needed for installation stages
   and for control of assembly,
 \item Reference models and drawings needed for alignment and survey.
\end{enumerate}


The models and drawings described above will be generated from static
models. Combination models of above will also be generated to
represent combined effects. In all case, as with static modes, 2D
drawings will be created which form the basis of the installation
drawings.

Generation of envelope models and drawings are the responsibility of
\dword{tc} engineering team in coordination with consortia.


\section{Plan for Model and Drawing Storage and Dissemination}
\label{sec:fdsp-coord-integ-modelplan}
Drawings, models, schematics, production data and all other
engineering documents are created and shared by consortia and by
\dword{tc} engineering team. In addition, all interface drawings and
documentation are generated by \dword{tc} engineering team and shared
in a similar fashion.

Folders have been set up that allow for uploading and sharing
documents with appropriate protection. The structure of the folders
has been setup in a way that will suit each consortium. Not all
consortia will have similar folder structures or files and will adapt
the structure to their needs.

The folders and files reside on the \dword{edms}. This system and
similar structures have been used for \dword{protodune} and are being
used by \dword{lbnf}.

The following shows a high-level outline of the file structure. The
first section is for technical coordination files. The following
section is generic and is for a consortium. There will be one such
folder for each consortium.
\begin{enumerate}
 \item Technical Coordination
 \begin{enumerate}
  \item Mechanical drawings and files (Control by DUNE Lead Mechanical Engineer)
  \begin{enumerate}
    \item Far Detector general drawings for illustration purposes (control by TC)
    \item 3-D Model files of internal detector, for periodic upload to global model
    \item 2-D Interface drawing files    
    \item Alignment and survey files
    \item Integration Test facility
    \item Ash River installation test Technical Coordination
    \item QA/QC files
    \item Safety analysis and documentation
    \item Design reviews
  \end{enumerate}
  \item Electrical and electronics (Control by DUNE Lead Electrical Engineer)
  \begin{enumerate}
    \item Infrastructure requirements for grounding
    \item Consortia interface drawings
    \item Detector electronics grounding guidelines
    \item Detector Safety System
    \item QA/QC files
    \item Safety analysis and documentation
  \end{enumerate}
 \end{enumerate}
 \item Consortia files (One per consortium, control by consortia Technical Leads)
 \begin{enumerate}
   \item 3-D Model files
   \item 2-D Part drawing files
   \item Production files
   \item General grounding diagram
   \item System level block diagrams
   \item System level wiring diagrams
   \item Software and firmware plans
   \item Custom components, such as ASICs (one folder per component)
   \item PCB component (one folder per component)
   \item Cable component (one folder per component)
   \item Power supply component (one folder per component)
 \end{enumerate}
\end{enumerate}
(Note: add figure once the \dword{edms} structure is set up)



\section{Integration Drawings}
\label{sec:fdsp-coord-integ-drawings}
Within each detector, components from various consortia are assembled
and installed. The interfaces among components are developed and
managed through models and drawings as described earlier.

There are many such interfaces that are controlled. The following
section shows some of the interfaces, controlling dimensions and
configuration.

Figure~\ref{fig:dune-apa_interfaces_top} shows the interfaces for the
top APAs in the upper corner of the cryostat. It also shows position
of the cable penetration for the APA. Interfaces with cryostat
corrugations and LAr fill lines are also shown. The reference plane,
which is defined to be the plane of APA yokes is explained in the
alignment section.
\begin{dunefigure}[Interface between upper Anode Plane Assembly,
    Field Cage, cable trays and Detector Support System]{fig:dune-apa_interfaces_top}
  {Interface between upper Anode Plane Assembly, Field Cage, cable
    trays and Detector Support System.}
  \includegraphics[width=0.8\textwidth]{Interface_upper_apa.pdf}
\end{dunefigure}

Figure~\ref{fig:dune-apa_interfaces_bottom} shows the interfaces for
the bottom APAs in the lower corner of the cryostat. In both figures
the latch connection latch between the filed cages and APAs is also
shown.
\begin{dunefigure}[Interface between lower Anode Plane Assembly,
    Field Cage and service floor]{fig:dune-apa_interfaces_bottom}
  {Interface between lower Anode Plane Assembly, Field Cage and
    service floor.}
  \includegraphics[width=0.8\textwidth]{Interface_lower_apa.pdf}
\end{dunefigure}

Figure~\ref{fig:dune-apa_interfaces_mid} shows the interfaces for the
top of the central row of APAs with other components. In this case a
double latch connection is used. Similarly,
Fig~\ref{fig:dune-cpa_interfaces} shows the interfaces for top of the
CPAs with other components
\begin{dunefigure}[Interface between Anode Plane Assembly, upper Field
    Cages and Ground Plane]{fig:dune-apa_interfaces_mid}
  {Interface between Cathode Plane Assembly, upper Field Cages and Ground Plane.}
  \includegraphics[width=0.8\textwidth]{Interface_upper_mid_apa.pdf}
\end{dunefigure}
\begin{dunefigure}[Interface between Cathode Plane Assembly, upper Field
    Cages and Ground Plane]{fig:dune-cpa_interfaces}
  {Interface between Cathode Plane Assembly, upper Field Cages and Ground Plane.}
  \includegraphics[width=0.8\textwidth]{Interface_upper_cpa.pdf}
\end{dunefigure}

It should be noted that above integration drawing are derived directly
from the overall integration model. The overall integration model is
assembled from component models that are developed by
consortia. Interfaces are controlled by technical coordination and
consortia are responsible for maintaining their model files to be
compatible with the interfaces. During the design phase, models are
assembled and checked continuously. At the time of final design, all
interfaces will be fixed.

Component tolerances and installation clearances are managed through
additional models as described earlier.
Fig~\ref{fig:dune-apa_envelope} is a graphical representation
of Anode Plane and Cathode Plane Assemblies. It also shows their
relieve positions which defines how they are constrained in the
detector.
\begin{dunefigure}[Graphical representation of envelope dimensions and
    installation clearances for Anode Plane and Cathode Plane
    Assemblies in the warm state]{fig:dune-apa_envelope} {Graphical
    representation of envelope dimensions and installation clearances
    for Anode Plane and Cathode Plane Assemblies in the warm state.}
  \includegraphics[width=0.95\textwidth]{Warm_envelope_dimensions.pdf}
\end{dunefigure}

In this figure, design dimensions are shown. Component tolerances and
assembly tolerances for the upper and lowers APAs and CPA stacks have
been analyzed and are represented as envelope dimensions. An envelope
gap has been defined in order to account for tolerances in the support
system position and component tolerances. Taking all into account
pitch distances for APAs and CPAs have been defined in the warm state.

Figure~\ref{fig:dune-apa_envelope} also shows the design drift
distance in the warm state. The drift distance is defined as the
perpendicular distance between the surface of CPAs and the collection
wire plane of APAs.


Anode Plane and Cathode Plane Assemblies are supported in groups of
two or three on the detector support beams.  In the cold state, the
relative position of groups of Anode Plane and Cathode Plane
Assemblies change due to thermal contraction, while relative positions
within each group are relatively
constant. Figure~\ref{fig:dune-apa_envelope_cold} is a graphical
representation in the cold state.
\begin{dunefigure}[Graphical representation of envelope dimensions and
    installation clearances for Anode Plane and Cathode Plane
    Assemblies in the cold state]{fig:dune-apa_envelope_cold} {Graphical
    representation of envelope dimensions and installation clearances
    for Anode Plane and Cathode Plane Assemblies in the cold state.}
\end{dunefigure}

It should be noted that the design of the detector support system and
the cryostat have a direct impact on the cold state. In addition,
effects of gravity and buoyancy are also not represented. Such effects
are under study and will be represented in the models as design
progress.

Prior to installation of the detector, a set of cryogenic distribution
pipes are installed on the floor of the cryostat. In addition, the
membrane of the cryostat will have corrugations that impede
movement. Therefore, a temporary service floor will be installed. The
floor will be installed, and later removed, in
sections. Figure~\ref{fig:dune-floorpipes} shows this interface.

\begin{dunefigure}[Interface of cryogenic distribution pipes, service floor and detector.]{fig:dune-floorpipes} 
{Interface of cryogenic distribution pipes, service floor and detector.}
  \includegraphics[width=0.8\textwidth]{Interface_lower_mid_apa.pdf}
\end{dunefigure}


(Note: Need analogous figures and explanation for Dual Phase)


\section{Electrical System Block Drawings, Schematics, Layouts and Wiring Diagrams}
\label{sec:fdsp-coord-electrical}

The electrical design aspects of each consortia can be described by a
set of documents which in all cases includes a system level block
diagram and one-line power/grounding documentation.  Furthermore,
depending on what is being described, a complete set of schematics,
board production files and wiring diagrams will be required.  All
designs are subject to an electrical safety review prior to
installation and the safety review process should be begun though
technical coordination early in the design process.

Consortia are responsible for producing system level block
diagrams. Technical Coordination is responsible for ensuring that the
block diagrams are produced and are reviewed.  In some cases, multiple
system level block diagrams may be required.  An example of this is
for the Slow Controls Consortia, where multiple types of systems (RTD
readouts, Purity Monitors, Cameras, Pressure sensors, etc.) will each
require a separate system level block diagram. A system level block
diagram should show the conceptual blocks required to be implemented
in the design along with connections to other conceptual block
elements both within and outside of the consortia.  An example of a
system level block diagram can be found in
Figure~\ref{fig:electrical_blockdiagram_example}.
\begin{dunefigure}[Example system level block diagram]{fig:electrical_blockdiagram_example}
  {Electrical system level block diagram example SP cold electronics.}
 \includegraphics[width=0.85\textwidth]{Example_System_Level_Block-Diagram-SP_Cold_Electronics.pdf}
\end{dunefigure}

An electrical one-line drawing should represent the power and ground
distribution within the system being described.  The path of power and
ground distribution wiring between circuit elements should be
specifically noted along with wire types and sizes.  Power elements,
such as power supplies, fuses (or other protective circuit elements),
power connectors and pin ampacity should all be documented.

Electrical schematics show very specific details of how individual
components are connected.  Usually a schematic will represent a
printed circuit board design.  Schematics should call out specific
parts which are used in the design and include all interconnections.
In the case of a printed circuit board, layout files, manufacturing
specifications and bills of materials are also required to document
the design and allow for a safety review of any custom boards or
modules.

Wiring diagrams should include all wire and cable connections which
run between printed circuit boards or electronics modules.  Wires and
cables should be described within the contents of the diagram and
include identification of wire AWG, wire color, cable specification
and cable connectors and pinouts.


\section{Electrical Integration Documentation}
\label{sec:fdsp-coord-integ-electrical}
The previous section listed the documentation required to describe the
construction of a subsystem corresponding to the design
responsibilities within a consortium.  Interfaces which occur between
consortia subsystems must be documented and formally agreed upon
between the technical leads of the interfacing consortia and must be
verified by the technical coordination team.  Much of the type of
documentation required for describing a subsystem also applies to
documentation required for integration.

Documentation required for the interface between two electrical
subsystems includes a block diagram which identifies all connections
between the subsystems.  This block diagram must exist in the formal
interface document between consortia.  For each connection, additional
documentation must be provided which fully describes the interface
details. This additional detailed information can exist outside of the
primary interface document between consortia, but it must be pointed
to within that document.

One example of an integration interface is a signal cable which runs
between printed circuit boards belonging to different consortia.  For
a signal cable, interface documentation should include the connector
specifications, pinouts at each end of the cable and the pinout of the
board connectors.  Documentation should also include a description of
relevant electrical signal characteristics which may include signal
levels, function, protocol, bandwidth and timing information.  If
signals are referred to with different signal names between
subsystems, documentation of signal name cross reference must be
provided.

Consortia technical leads and the technical coordination team sign off
and approve the detailed documentation information on integration
interfaces which is not included in the primary consortia to consortia
interface document.


\section{Detector Survey and Alignment}
\label{sec:fdsp-coord-integ-survey}
The detector placement within the cryostat and thereby within the
cavern does not have a physics requirement. The requirement is driven
by the need for overall mechanical assembly in order that interfacing
parts are assembled and function properly.

In this section, reference frames for the detector are defined in order
that overall survey and alignment can be carried out with respect to
the cavern reference frame. (Note: we have not yet defined this)

For the single-phase detector, a reference flat and level plane is
defined which is coplanar with the upper Anode Plane Assembly yoke
planes. As such 75 yoke planes will define this reference flat and
level plane. This reference plane is set at exactly 781 mm below the
theoretical plane of the cryostat top membrane.

The detector reference plane is chosen to be coplanar with the upper
Anode Plane Assembly yoke planes because all the features including
the active area, are referenced to this plane.  Once this reference
plane is defined and established through survey, all vertical
distances within the detector are referenced and established to this
plane.  Figure~\ref{fig:dune-apa_interfaces_mid} shows the reference
plane in relation to the cryostat top membrane and Detector Support
System

During installation, the height of the Detector Support System beams
is set in accordance to this relationship. Adjustments are made in the
Detector Support System to establish all the beams to be in the
correct plane. The combined effects of gravity prior to fill and
buoyancy after the fill are calculated and further adjustment are made
to compensate. This will ensure that the detector is as close as
possible to design value after fill.

The transverse position of the detector is constrained to be in the
center of the cryostat. As such the mid-plane of the middle row Anode
Plane Assemblies is coplanar with the vertical mid-plane of the
cryostat. This relationship is verified during installation of the
Detector Support System beams and central row of Anode Plane
Assemblies. The outer rows are similarly aligned and surveyed with the
offset as shown in Figure~\ref{fig:dune-sp_row}.
\begin{dunefigure}[Position of feedthrough determining the longitudinal
    reference point of the detector]{fig:dss_feedthru}
  {Position of feedthrough determining the longitudinal reference point of the detector.}
  \includegraphics[width=0.9\textwidth]{SP_longitudinal_reference.pdf}
\end{dunefigure}


The longitudinal reference point of the detector within the cryostat
is defined by the position of single feedthrough of the central row
that is farthest from the cryostat opening. This feedthrough position
is shown in Figure~\ref{fig:dss_feedthru}.


(Note: The longitudinal position is not fixed at this time. Edit
section as needed once fixed)

(Note: Dual phase should be similar
transversely and longitudinally, vertically, it needs to be
determined)


\section{Interface Documents}
\label{sec:fdsp-coord-integ-interface}

Interface documents are developed and maintained in order to manage
the interfaces among consortia and between consortia and technical
coordination. There is a single document between two systems that have
an interface. Therefore, any system may have several interface
documents. In cases where no interface exists, an interface document
is not needed. The interfaces documents are managed by relevant
consortia technical leads and by \dword{tc} project engineers.

The content of the interface documents varies depending on the nature
of the interface. However, the structure of the documents are intended
to be common and define the interfaces in following sections:
\begin{enumerate}
 \item Definition: This section defines the interfacing systems
 \item Hardware: In this section, the interfacing hardware components,
   electrical and mechanical, are defined in general terms. As an
   example, the Anode Plane Assembly frame needs to support the Photon
   Detectors mounting brackets.
 \item Design: In this section, the dependencies in design
   methodology, sequence and standards are described. As with the
   previous example, the design of the Photon Detector mounting
   brackets depends on side tubes chosen for the Anode Plane Assembly.
 \item Production: Component production and an overall assembly is
   shared among interfacing system. This section details the
   responsibility.
 \item Testing: Similar to production, testing is a shared
   responsibility. In this section responsibilities for testing and
   the required equipment are defined.
 \item Integration: The integration of systems into installable units
   before insertion into the cryostat is defined in this section. The
   location, methodology, tooling and environment for integration are
   defined.
 \item Installation: Once assembled into an installable units,
   installation tasks and responsibilities are defined in this
   section. Any special transportation or installation tool or fixture
   is also defined.
 \item Commissioning: In this section, overall responsibilities for
   commissioning tasks as well as setting of parameters are defined.
 \item Appendices: It is intended that technical figures and
   interfaces are included in as much detail as necessary. Block
   diagrams which show interconnects and detailed documentation of
   each connections are needed. In this section such data are
   included.
\end{enumerate}

The interface documents will be developed and modified during the
technical design period. At the time of this writing, not all
documents have been fully developed. Once the technical design is
finished, the interface documents will be placed under revision
control.
