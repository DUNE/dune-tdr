\chapter{Detector Design and Construction Organization}
\label{vl:tc-overview}

\Dword{tc} is led by the \dword{tcoord}, who runs the \dword{tb} that
reviews \dword{dune} technical options. The \dword{tc} organizational
chart is shown in Fig.~\ref{fig:TC_org_chart}.
\begin{dunefigure}[\dword{tc} organizational chart]{fig:TC_org_chart}
  {\dword{dune} \dword{tc} organizational chart.}
  \includegraphics[width=0.99\textwidth]{TC_org_chart}
\end{dunefigure}


\section{DUNE Work Flow}
\label{sec:workflow}

Figure~\ref{fig:DUNE_workflow} shows the \dword{dune} work flow diagram.
\begin{dunefigure}[\dword{dune} work flow]{fig:DUNE_workflow}
  {\dword{dune} work flow.}
  \includegraphics[width=0.99\textwidth]{DUNE_workflow}
\end{dunefigure}

\dword{protodune2}??

\section{DUNE Far Detector Consortia}
\label{sec:consortia}

Construction of the \dword{dune} far \dwords{detmodule} is carried out
by ``consortia of collaboration institutions'' who assume
responsibility for detector subsystems.  Each consortium plans and
executes the construction, installation and commissioning of its
subsystem.

The \dword{dune} collaboration is pursuing far detector designs based
on both single and dual-phase readout technologies currently under
development for liquid Argon Time Projection Chambers (LArTPCs).  A
total of eleven Far Detector consortia have been formed to cover the
sub-systems required for the two detector types currently under
consideration.  In particular, three consortia (SP-APA, SP-TPC
Electronics and SP-Photon Detection) pursue sub-systems specific to
the single-phase design and another three consortia (DP-CRP, DP-TPC
Electronics and DP-Photon Detection) pursue designs for dual-phase
specific sub-systems.  An additional five consortia (HV System, DAQ,
Cryogenic Instrumentation/Slow Controls, Calibration, and Computing)
have responsibility for sub-systems common to both detector
technologies. The consortia are shown in Figure~\ref{fig:DUNE_consortia}.
\begin{dunefigure}[\dword{dune} consortia]{fig:DUNE_consortia}
  {\dword{dune} consortia.}
  \includegraphics[width=0.99\textwidth]{DUNE_consortia}
\end{dunefigure}



Management of the consortia is through an overall Consortium Leader
and a Technical Lead.  The Consortium Leader chairs an institutional
board composed of one representative from each of the collaboration
institutes contributing to the activities of the consortium.  Major
consortia decisions such as technology selections and assignment of
responsibilities within the institutions are expected to be passed
through its Institutional Board.  These decisions are then passed as
recommendation to the \dword{dune} \dword{exb}, described in more detail
below, for formal collaboration approval.

Figure~\ref{fig:DUNE_consortia_org} shows the \dword{dune} internal
consortia structure
\begin{dunefigure}[\dword{dune} Internal Consortia Structure]{fig:DUNE_consortia_org}
  {\dword{dune} Internal Consortia Structure}
  \includegraphics[width=0.99\textwidth]{DUNE_consortia_org}
\end{dunefigure}

Consortium responsibilities for sub-system deliverables are in many
cases shared by institutions supported by more than one funding
agency.  Each participating funding agency is expected to manage its
own internal project with responsibility for its assigned
deliverables.  Coordination of the separate internal projects
contributing to each consortium is the responsibility of the
consortium Technical Lead.  The Technical lead is responsible for
chairing a consortium Project Management Board incorporating the
separate managers from each of the internal projects that oversees the
inter-connections between the different efforts.

Something about consortia sub-structure …

\section{DUNE Collaboration Management}
\label{sec:dune_mgmt}

DUNE Management Team.

Figure~\ref{fig:DUNE_org} shows the \dword{dune} organizational chart.
\begin{dunefigure}[\dword{dune} org chart]{fig:DUNE_org}
  {\dword{dune} Organizational Chart}
  \includegraphics[width=0.99\textwidth]{DUNE_COllab_Mgmt}
\end{dunefigure}

Each of the consortia is represented on the \dword{dune} \dword{exb}
by its consortium leader.  The \dword{dune} \dword{exb} is the primary
collaboration decision-making body and as such includes
representatives from all major areas of activity within the
collaboration.  All collaboration decisions, especially those with
potential impacts on the \dword{dune} scientific program or connected
with the assignment of institutional responsibilities, pass through
the \dword{exb}.  \dword{exb} decisions are expected to be achieved
through consensus.  In cases where consensus cannot be obtained,
decision-making responsibility passes to the co-spokespersons.


\section{\dword{tc}}
\label{sec:tc}

Because the consortia operate as self-managed entities, a strong
\dword{tc} organization is required to ensure overall integration of
the detector elements and successful execution of the detector
construction project.  \dword{tc} areas of responsibility include project
oversight, systems engineering, quality assurance and safety.
\dword{tc} provides support to the \dword{jpo} (see
Section~\ref{sec:pm}) for planning and executing the required detector
integration and installation activities in the nearby surface
facilities and underground detector caverns at \surf.

\dword{tc} is headed by the \dword{tcoord}, who is an \dword{fnal}
employee and is appointed jointly by the \dword{fnal} director and the
\dword{dune} co-spokespersons.  A deputy technical coordinator is
selected from within the collaboration to assist the \dword{tc} in
carrying out their responsibilities.

The \dword{tc} organization supports the work of the consortia and
takes responsibility for integration of the detector
subsystems.  The organization includes teams focusing on project
coordination, detector integration and installation support.  The
project coordination team is led by a lead project controls
specialist, a \dword{qa} manager and an \dword{esh} manager.  The
detector integration team is directed by a lead mechanical and lead
electrical engineer, and inorporates an online computing coordinator.
The installation support team is headed by the coordinators for
activities associated with the integration of detector components on
the surface and installation of components in the underground areas.
Each of the three teams incorporates additional personnel that support
these individuals in carrying out their areas of responsibility.
Figure~\ref{fig:DUNE_tc} shows the \dword{dune} \dword{tc} organizational chart.
\begin{dunefigure}[\dword{dune} \dword{tc} org chart]{fig:DUNE_tc}
  {\dword{dune} \dword{tc} Organizational Chart}
  \includegraphics[width=0.99\textwidth]{DUNE_tc}
\end{dunefigure}



\section{DUNE Project Functions}
\label{sec:pm_functions}


\subsection{Safety}
\subsection{Engineering Integration (CAD Models, Drawings, Interfaces, Mechanical \& Electrical)}
\subsection{Change Control and Document Management (Requirements, Quality Assurance Database)}
\subsection{Schedule and Milestones}
\subsection{Partner Agreements and Financial Reporting}

\section{DUNE Project Management}
\label{sec:pm}


\fixme{I'm not clear on the distinction between board meetings:
  consortium board meetings with all consortium leads? Then tech board
  meetings are at a higher level? Who is involved? Then PMBs with
  funding agencies (since they are responsible for managing their
  projects?}

The \dword{tcoord} manages the overall detector construction project
through regular Technical Board meetings with the consortia leadership
teams and members of the \dword{tc} organization.  These board
meetings provide the primary forums for required interactions between
the consortia leadership teams.

Technical Board meetings are used to evaluate consortia design
decisions with potential impacts on overall detector performance,
ensure that interfaces between the different subsystems are well
understood and documented, and monitor the overall construction
project to identify and address both technical and interface issues as
they arise.

Project board meetings are used to ensure that the scopes of each
consortium are fully documented with assigned institutional
responsibilities, develop and manage risks held within a global
project registry, review and manage project change requests, and
monitor the status of the overall detector construction schedule.

Any decisions generated through these board meetings are passed to the
\dword{dune} \dword{exb} as recommendations for formal approval.
Depending on the specific agenda items for a meeting, the
\dword{tcoord} will invite additional members of the collaboration
with specific knowledge or particular expertise to participate.  In
addition, for major decisions, the \dword{tcoord} will officially
appoint three internal collaboration referees with no direct conflicts
of interest to engage in the process.
