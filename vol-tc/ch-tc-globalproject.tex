\chapter{Global Project Organization}
\label{vl:tc-global}

\section{Global Project Partners}
\label{sec:partners}

The \dword{lbnf} project is responsible for providing both the 
conventional facilities and supporting infrastructure (cryostats 
and cryogenic systems) that house the \dword{dune} \dword{fd} 
modules. \dword{lbnf} is a U.S. \dword{doe} project incorporating 
contributions from international partners and is headed by the 
\dword{lbnf} Project Director who also serves as the \dword{fnal} 
Deputy Director for \dword{lbnf}.  
The international \dword{dune} 
collaboration under the direction of its management team is 
responsible for the detector components.  
The 
\dword{dune} \dword{fd} construction project encompasses all 
activities required for designing and fabricating the 
detector elements and incorporates contributions from a number 
of international partners.  The 
organization of the global \dword{lbnf-dune}, which encompasses 
both project elements, is shown in Figure~\ref{fig:DUNE_global}.
\begin{dunefigure}[Global project organization]{fig:DUNE_global}
  {\dword{lbnf-dune} organization.}
  \includegraphics[width=0.8\textwidth]{FS_Integration_OrgChart_notitle}
\end{dunefigure}

In addition to the \dword{lbnf} and \dword{dune} pieces, the overall
coordination of installation activities in the underground caverns is
managed as a separate element of \dword{lbnf-dune} under the
responsibility of the \dword{ipd}, who is appointed by and reports
directly to the \dword{fnal} director.  To ensure coordination across
all elements of \dword{lbnf-dune}, the \dword{ipd} connects to both
the facilities and detector construction projects through ex-officio
positions on the \dword{lbnf} Project Management Board and
\dword{dune} \dword{exb}, respectively.  In carrying out these
responsibilities, the \dword{ipd} receives support from the
\dword{fnal} \dword{sdsd}.  The head of the \dword{sdsd} works closely
with \dword{surf} to ensure the appropriate infrastructure is in place
for safe operations.


The \dword{ipd} works closely with the \dword{lbnf} and \dword{dune}
teams in advance of these activities to coordinate planning
and assure that detector elements are properly integrated within the 
supporting infrastructure.  Although the \dword{ipd} is responsible 
for overall coordination of  installation activities
at \dword{surf}, the \dword{dune} consortia maintain responsibility 
for the installation and commissioning of their detector subsystems
and support these activities by providing dedicated personnel and
equipment resources.  Likewise, \dword{lbnf} retains responsibility
for the installation and commissioning of supporting infrastructure
items and provides dedicated resources to support these activities.          

\section{Experimental Facilities Interface Group}
\label{sec:efig}

The \dword{efig} is the body responsible for the required high-level
coordination between the \dword{lbnf} and \dword{dune} construction 
projects.  The \dword{lbnf} Project Director and \dword{ipd} 
co-chair the \dword{efig}.  \dword{efig} leadership also incorporates 
the four members of the \dword{dune} collaboration management 
team (co-spokespersons, \dword{tcoord}, and \dword{rcoord}).  
The \dword{efig} is responsible for steering the  
installation of the \dword{lbnf-dune} pieces and operates via the 
consensus of its leadership team.  If issues arise for which consensus 
cannot be achieved, decision-making responsibility is passed to the 
\dword{fnal} Director.

\section{Joint Project Office}
\label{sec:jpo}

The \dword{efig} is augmented by a \dword{jpo} that supports the
\dword{lbnf} and \dword{dune} projects as well as the integration
effort across \dword{lbnf} and \dword{dune}. The \dword{jpo} 
combines project activities and functions that exist within the 
\dword{lbnf} and \dword{dune} projects to ensure that they are 
properly coordinated across the global \dword{lbnf-dune}.  Members of 
the \dword{jpo} organization are drawn directly from the project 
teams supporting \dword{lbnf} and \dword{dune} as well as the 
team supporting the overall  installation effort.  
Figure~\ref{fig:DUNE_jpo} shows the current organizational 
structure of the \dword{jpo}, indicating the members being 
pulled in from the \dword{lbnf} (LBNF), \dword{dune} (\dword{tc}), and 
\dword{lbnf}/\dword{dune} Integration Office (IO) project teams.

\begin{dunefigure}[JPO organization chart]{fig:DUNE_jpo}
  {\dword{jpo} organization chart}
  \includegraphics[width=0.85\textwidth]{JPO_OrgChart2}
\end{dunefigure}
The team members focusing on specific project activities and 
functions within the \dword{jpo} are typically those carrying 
equivalent responsibilities within their home organization.  For 
example, the \dword{jpo} team responsible for building the fully 
integrated \threed CAD model of the detector within its supporting 
infrastructure and the surrounding facility includes the members 
of the \dword{lbnf} and \dword{dune} project teams responsible 
for integrating the individual elements.

\section{Coordinated Global Project Functions}
\label{sec:global_project}

\dword{jpo} functions include \dword{lbnf-dune} configuration and 
integration, installation planning and coordination, scheduling, 
safety assurance, technical review planning and oversight, development of partner agreements, and financial reporting.  At 
the time when \dword{lbnf} \dword{cf} delivers
\dword{aup} of the underground detector caverns at 
\dword{surf}, the \dword{jpo} will evolve to incorporate the 
stand-alone organization that will coordinate the 
onsite activities there required to install the detectors and 
supporting infrastructure under the direction of the \dword{ipd}.  
Some members of the \dword{lbnf} and \dword{dune} project 
teams are expected to become fully embedded within the 
\dword{jpo} at that point in time.  The expanded \dword{jpo} 
organization required to coordinate post-excavation
installation activities at \dword{surf} is described in 
Chapter~\ref{ch:tc-jpo}.  \dword{jpo} functions associated with 
its \dword{lbnf-dune} coordination role are described in the 
following sections.

\subsection{Safety}
\label{sec:dune_safety}

To ensure a consistent approach to safety across \dword{lbnf-dune},
there is a single \dword{lbnf-dune} \dword{esh} manager who reports directly 
to the \dword{lbnf} project director, \dword{ipd}, and \dword{dune}
management (via the \dword{dune} \dword{tcoord}).  This individual
directs separate safety teams responsible for implementing the 
\dword{lbnf-dune} \dword{esh} program within both the \dword{lbnf} 
and \dword{dune} projects as well as the \dword{lbnf}/\dword{dune}
installation activities at \dword{surf}. The safety organization is shown in figure~\ref{fig:dune_esh} and is described further in Chapter~\ref{vl:tc-ESH}.
\begin{dunefigure}[\dword{lbnf-dune} \dword{esh}]{fig:dune_esh}
  {High level \dword{lbnf-dune} \dword{esh} organization.}
  \includegraphics[width=0.85\textwidth]{DUNE_Safety_Org_Chart}
\end{dunefigure}
The \dword{lbnf-dune} \dword{esh} manager works directly with the \dword{fnal} 
and \dword{surf} safety teams to ensure that all project-related 
activities comply with the rules and regulations of the host 
organizations.  For example, the \dword{lbnf-dune} \dword{esh} manager 
works directly with the host safety organizations to develop the 
rules and regulations governing work in the underground areas 
at \dword{surf}, which are then consistently applied across all 
underground project activities.

The \dword{jpo} includes the engineering safety assurance team that
defines a common set of design and construction rules (mechanical and
electrical) to ensure consistent application of engineering standards
and engineering documentation requirements across \dword{lbnf-dune}.
This team works with the \dword{lbnf-dune} \dword{esh} manager to develop
equivalencies in codes and standards across the international project
as needed.  Following on lessons-learned from the processes employed
for the \dword{protodune} detectors, an important mandate of the
engineering safety assurance team is to ensure that safety issues
related to component handling and installation are incorporated within
the earliest stages of the design review process.  The \dword{jpo}
team will incorporate engineering resources to perform independent
validation of required mechanical analyses that ensure the structural
integrity of detector components through all stages of construction,
installation, and operation.

\subsection{Engineering Integration}
\label{sec:dune_engineering}

A central \dword{jpo} engineering team is responsible for building an
integrated model of the detectors within their supporting
infrastructure and the \dword{cf} that house them, as is described
further in Chapter~\ref{sec:fdsp-coord-integ-sysengr}.  The team
builds and maintains a full \threed CAD model of everything in the
underground detector caverns from the models of the individual
components provided by the \dword{lbnf} and \dword{dune} design teams.
Starting from the latest, approved version of the full CAD model, the
\dword{jpo} team incorporates approved changes as they are received
checks to ensure that no errors or spatial
conflicts are introduced into the model.  As part of this process a
series of \twod control drawings are produced from the \threed CAD
model to validate adherence with a selection of critical
component-to-component clearances within the integrated
model.  The updated working model is passed back to the individual
\dword{lbnf} and \dword{dune} design teams to validate that their
design modifications have been properly incorporated within the global
model.  After receiving the appropriate sign-offs from all parties,
the \dword{jpo} team tags a new frozen release of the model and makes
it available to the design teams as the current release against which
the next set of design changes will be generated.

Electrical engineers are incorporated within the central
\dword{jpo} team to ensure proper integration
of the detector electrical components.  This team is responsible 
for ensuring that detector grounding and shielding requirements, 
the maintenance of which are critical for detector performance, 
are strictly adhered to.  The team oversees the layout of 
electronics racks and cable trays both on the top of the cryostats 
and within the \dword{cuc} counting room that hosts the \dword{daq} 
electronics.  It also oversees the design of the power and cooling 
distribution systems that are required to support the electronics 
infrastructure.

The \dword{jpo} engineering team is responsible for documenting and
controlling the interfaces between the \dword{lbnf} and \dword{dune} 
projects as well as the interfaces between these projects and the 
\dword{lbnf}/\dword{dune}  installation activities 
at \dword{surf}.  To define these interfaces, the \dword{jpo} team 
develops formal documents, which subsequent to the approval of the 
relevant managers are placed under signature and versioning control.  
These documents are monitored regularly to ensure that no missing 
scope or technical incompatibilities are introduced at the boundaries 
between the projects.

\subsection{Change Control and Document Management}
\label{sec:dune_changecontrol}

The \dword{lbnf-dune} project partners have agreed to adopt 
the formal change control process developed previously for the 
\dword{lbnf} project.  The change control process applies to 
proposed modifications of requirements, technical designs, 
schedule, overall project scope and assigned responsibilities 
for individual scope items.  The formal \dword{lbnf-dune} 
change control process is described in~\citedocdb{82}.  The 
process includes separate decision paths for items affecting 
only \dword{dune} or \dword{lbnf} and incorporates an additional 
pathway for items affecting both projects.  A hierarchy of 
decision-making layers is built into each pathway based on 
pre-determined thresholds related to the extent of the proposed 
change.  The lowest-level change control body for modifications 
affecting both \dword{lbnf} and \dword{dune} is the \dword{efig}.  
The \dword{jpo} team includes a configuration manager who is 
responsible for formally implementing changes that are approved 
through this process.  After technical changes are incorporated 
into the global \threed CAD models, the lead \dword{lbnf}/\dword{dune} 
systems engineer is responsible for checking production drawings 
and verifying that no potential space conflicts have been 
introduced.  Under the direction of the configuration manager, 
all project changes are documented in detail and approved by 
the appropriate project partners using the \dword{lbnf} change 
control tool.

The configuration manager is responsible for administrating
and managing the \dword{lbnf-dune} document management system, 
which is hosted in the \dword{edms}.  All technical documents 
and drawings will be stored in the \dword{edms} system under 
formal signature and versioning control.  A product breakdown 
structure (PBS) database will be maintained to track the history 
of each detector component (and supporting infrastructure item) 
through construction, assembly, testing, transport and installation.  
The \dword{lbnf-dune} \dword{qa} manager who sits within the 
\dword{jpo} has responsibility for ensuring that all necessary 
documents and testing results used to validate component quality 
are stored within the PBS database.

\subsection{Schedule}
\label{sec:dune_schedule}

The \dword{jpo} team is responsible for creating a single project
schedule for \dword{lbnf-dune} incorporating all \dword{lbnf} and
\dword{dune} activities together with the installation activities at
\dword{surf} incorporating all interdependencies. A brief discussion
is provided in Section~\ref{sec:fdsp-coord-controls}. This schedule
will be used to track the status of the global enterprise.  The
project partners have agreed that the \dword{lbnf-dune} schedule will
be managed within the same Primavera \dword{p6} framework used to plan
and status the resource-loaded schedule of activities required for
U.S. \dword{doe} contributions to \dword{lbnf} and \dword{dune}.
Activities falling under the responsibility of other international
partners are included and directly linked within the \dword{p6}
schedule, but do not incorporate associated resource information
required for U.S. \dword{doe} activities.  The non-\dword{doe}
activities will not be tracked using the formal \dword{evms}
procedures required for the \dword{doe} project activities but rather
through regular assessments of progress towards completion by the
management teams responsible for those activities.  A substantial
number of milestones will be embedded within the schedule at an
appropriate level of granularity to allow for high-level tracking of
the project progress towards its completion.

\subsection{Review Process}
\label{sec:dune_review}

To achieve coherency in the review process across the
\dword{lbnf-dune} enterprise, reviews will be coordinated through the
\dword{jpo} as described in Chapter~\ref{vl:tc-review}.  \dword{dune}
collaboration management via the \dword{tcoord} has direct
responsibility for design and production reviews focusing on the
different detector elements.  Similarly, \dword{lbnf} project
management has responsibility for design and production reviews
covering the supporting infrastructure pieces within its scope.
Installation and \dwords{orr}, on the other hand, are the
responsibility of the \dword{ipd}.  Central coordination of the review
process through the \dword{jpo} ensures that issues related to
installation and operation are incorporated within all stages of the
review process.  Safety issues related to the handling and
installation of components are addressed starting from the earliest
design reviews with the development of detailed engineering notes
containing the required structural analysis through installation and
operations reviews with detailed hazard analyses.  The \dword{jpo}
team also takes responsibility for tracking review recommendations and
closing them as appropriate based on resulting actions.

\subsection{Partner Agreements and Financial Reporting}
\label{sec:dune_agreements}

Partner contributions to all project elements will be detailed 
in a series of written agreements.  In the case of \dword{lbnf}, 
these contributions will be spelled out in bilateral agreements 
between \dword{doe} and each of the contributing partners.  In 
the case of \dword{dune}, there will be an \dword{mou} 
detailing the contributions of all participating partners.  The 
\dword{mou} will detail the deliverables being provided by each 
partner and summarize required contributions to common items, 
for which the collaboration assumes shared responsibility.  
A series of more technical agreements describing the exact 
boundaries between partner contributions and the terms and 
conditions under which they will be delivered will lie just 
beneath the primary agreements.  The \dword{jpo} will coordinate 
production of these agreements and work to obtain the appropriate 
partner approvals on each.  

\section{\dword{protodune} Experience}
\label{sec:dune_protodune}

The global structure of \dword{lbnf-dune} is based heavily on 
the organization that successfully executed the construction,
installation, commissioning and operation of the \dword{protodune}
detectors at the \dword{cern}.  The onsite team at \dword{cern} 
responsible for the overall installation of 
detector and infrastructure components within the test beam 
facility played a critical role in the successful execution of the 
\dword{protodune} program.  The separate projects responsible for 
the construction of the detector and infrastructure components 
interacted effectively with the central, onsite team to minimize 
the issues encountered during the installation and commissioning 
process.  In cases where issues did arise, construction project 
team members interacted effectively with their counterparts on 
the onsite team to reach quick resolutions.

Some lessons learned from the \dword{protodune} experience have 
been applied in creating the \dword{lbnf-dune} organization for 
the \dword{dune} \dword{fd}.  The integration of installation 
safety issues into the early stages of the design review process 
is one such example.  Delays were encountered in getting approvals 
for the installation of some \dword{pdsp} components stemming from 
the absence of a coordinated approach in the review process for 
these items.  The creation of the \dword{jpo} team charged with 
organizing a coherent review process across \dword{lbnf-dune} is 
meant to address this issue directly.  In general, the successful 
implementation of the \dword{protodune} detectors demonstrates 
the capacity of the organizational structures to safely execute 
the project and meet performance requirements.

The team that led the installation of \dword{protodune} at
\dword{cern} also led the installation of \dword{minos} in the Soudan
mine in Minnesota and that experience is extraploated to the
installation at \dword{surf}. \dword{fnal} has established the
\dword{sdsd} to work with \dword{surf} to provide the supporting
infrastructure expected at a Laboratory for installation,
commissioning, and operation of the \dword{dune} detectors.
