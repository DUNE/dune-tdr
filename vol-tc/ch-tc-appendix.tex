%%%%%%%%%%%%%%%%%%%%%%%%%%%%%%%%
\section{Interface Documents}
\label{sec:fdsp-coord-interface}

A set of interface documents defines the scope of each subsystem and
with progressively more detail defines the detailed interfaces between
subsystems. There are three sets of interface documents. One set of
documents includes all of the consortia-to-consortia interfaces. A second
set includes the interfaces between the consortia and the facilities
(provided either by \dword{tc}, \dword{lbnf} or the \dword{integoff}). The
third set is between the consortia and the installation team. All
three sets are managed by \dword{jpo} engineering team.

The \dword{dune} interface documents are actively maintained in the
\dword{edms}, but a copy has been archived in DocDB that captures the
status of these documents at the time of this \dword{tdr}.  A matrix
with links to the interface documents in DocDB between consortia for
the \dword{spmod} are shown in
Table~\ref{tab:interface_sp_consortia}. 
The interface documents for the \dword{dpmod} are in preparation as of this writing. % sentence added by AH 12/2
\begin{dunetable}
  [\dshort{spmod} inter-consortium interface document matrix]
  {|p{0.08\linewidth}|rp{0.08\linewidth}||rp{0.08\linewidth}||rp{0.08\linewidth}||rp{0.08\linewidth}|rp{0.08\linewidth}||rp{0.08\linewidth}|rp{0.08\linewidth}|||}
  {tab:interface_sp_consortia}
  {\dword{spmod} consortium-to-consortium interface document matrix. All entries point to documents in the DUNE DocDB.}
       & PDS  & TPC Elec   & HV   & DAQ  & CISC & CAL  & COMP \\ \toprowrule
  APA  & \cite{bib:docdb6667} &  \cite{bib:docdb6670} & \cite{bib:docdb6673} & \cite{bib:docdb6676} &  \cite{bib:docdb6679} &  \cite{bib:docdb7048} &  \cite{bib:docdb7102} \\ \colhline
  PDS  &      &  \cite{bib:docdb6718} &  \cite{bib:docdb6721} & \cite{bib:docdb6727} &  \cite{bib:docdb6730} &  \cite{bib:docdb7051} & \cite{bib:docdb7105} \\ \colhline
  TPC Elec   &      &      & \cite{bib:docdb6739} & \cite{bib:docdb6742} & \cite{bib:docdb6745} & \cite{bib:docdb7054} & \cite{bib:docdb7108} \\ \colhline
  HV   &      &      &      & \cite{bib:docdb6736} & \cite{bib:docdb6787} & \cite{bib:docdb7066} & \cite{bib:docdb7120} \\ \colhline
  DAQ  &      &      &      &      & \cite{bib:docdb6790} & \cite{bib:docdb7069} & \cite{bib:docdb7123} \\ \colhline
  CISC &      &      &      &      &      & \cite{bib:docdb7072} & \cite{bib:docdb7126} \\ \colhline
  CAL  &      &      &      &      &      &      & \cite{bib:docdb6868} \\ 
\end{dunetable}

A matrix with links to the interface documents in DocDB between each consortium and the facility, installation, and DUNE physics for the \dword{spmod} are
shown in Table~\ref{tab:interface_sp_tc}.
\begin{dunetable}
  [\dshort{spmod} consortium-TC interface document matrix]
  {|p{0.15\linewidth}||rp{0.08\linewidth}||rp{0.08\linewidth}||rp{0.08\linewidth}||rp{0.08\linewidth}||rp{0.08\linewidth}||rp{0.08\linewidth}||rp{0.08\linewidth}||rp{0.08\linewidth}|}
  {tab:interface_sp_tc}
 {\dword{spmod} consortium-to-TC interface document matrix. All entries point to documents in the DUNE DocDB.}
                &  APA & PDS  & TPC Elec   & HV   & DAQ  & CISC & CAL  & COMP \\ \toprowrule
  Facility      & \cite{bib:docdb6967} & \cite{bib:docdb6970} & \cite{bib:docdb6973} & \cite{bib:docdb6985} & \cite{bib:docdb6988} & \cite{bib:docdb6991} & \cite{bib:docdb6829} & \cite{bib:docdb6841} \\ \colhline
  Installation  & \cite{bib:docdb6994} & \cite{bib:docdb6997} & \cite{bib:docdb7000} & \cite{bib:docdb7012} & \cite{bib:docdb7015} & \cite{bib:docdb7018} & \cite{bib:docdb6847} & \cite{bib:docdb6853} \\ \colhline
  Physics       & \cite{bib:docdb7075} & \cite{bib:docdb7078} & \cite{bib:docdb7081} & \cite{bib:docdb7093} & \cite{bib:docdb7096} & \cite{bib:docdb7099} & \cite{bib:docdb6865} &   \cite{bib:docdb6871}   \\ 
\end{dunetable}


%%%%%%%%%%%%%%%%%%%%%%%%%%%%%%%%
%\section{Cost}
%\label{sec:fdsp-coord-cost}
%
%\forlbnc{This section will discuss the cost estimate.}

%%%%%%%%%%%%%%%%%%%%%%%%%%%%%%%%
\section{Schedule Milestones}
\label{sec:fdsp-coord-controls}

A series of tiered milestones have been developed for the \dword{dune}
project. The spokespersons and host laboratory director are
responsible for the tier 0 milestones. Three tier 0 milestones have
been defined and the dates set:
\begin{enumerate}
\item Start main cavern excavation \hspace{2.58in} 2020
\item Start \dword{detmodule}~1 installation \hspace{2.1in} 2024
\item Start operations of \dwords{detmodule} \#1--2 with beam \hspace{0.8in} 2028
\end{enumerate}
These dates will be revisited after the U.S. \dword{lbnf} project is
reviewed. The \dword{tcoord}, \dword{ipd} and \dword{lbnf} project
manager hold the tier 1 milestones.  The consortia hold tier 2
milestones. Table~\ref{tab:DUNE_schedule} provides a high level version of the
\dword{dune} milestones from the \dword{lbnf-dune} schedule.


\begin{dunetable}
[\dshort{dune} schedule milestones]
{p{0.75\textwidth}p{0.18\textwidth}}
{tab:DUNE_schedule}
{\dword{dune} schedule milestones for first two far detector modules. Key DUNE dates and milestones, defined for planning purposes in this TDR, are shown in orange.  Dates will be finalized following establishment of the international project baseline.}
Milestone & Date   \\ \toprowrule
Final design reviews  & 2020 \\ \colhline
Start of APA production & August 2020 \\ \colhline
Start photosensor procurement & July 2021 \\ \colhline
Start TPC electronics procurement  & December 2021 \\ \colhline
%Start APA production at PSL/Chicago & Jan. / May 2021    \\ \colhline % Dec'19 per LBNC
Production readiness reviews  &  2022    \\ \colhline
\rowcolor{dunepeach} South Dakota Logistics Warehouse available& \sdlwavailable      \\ \colhline
Start of ASIC/FEMB production   & May 2022   \\ \colhline
%CPA factory 1 production begins & May 2022    \\ \colhline % Dec'19 per LBNC
%Begin \dshort{pd} module assembly for \dshort{detmodule} \#1 & September 2022     \\ \colhline % Dec'19 per LBNC
Start of DAQ server procurement &September 2022    \\ \colhline
\rowcolor{dunepeach} Beneficial occupancy of cavern 1 and \dshort{cuc}& \cucbenocc      \\ \colhline
Finish assembly of initial PD modules (80)      &March 2023    \\ \colhline
\rowcolor{dunepeach} \dshort{cuc} \dshort{daq} room available& \accesscuccountrm      \\ \colhline
Start of DAQ installation&      May 2023   \\ \colhline
Start of FC production for \dshort{detmodule} \#1       &September 2023   \\ \colhline
Start of CPA production for \dshort{detmodule} \#1&     December 2023   \\ \colhline
\rowcolor{dunepeach} Top of \dshort{detmodule} \#1 cryostat accessible& \accesstopfirstcryo      \\ \colhline
%First \dshort{pd} modules delivered to \dshort{sdwf} &   March 2024  \\ \colhline % Dec'19 per LBNC
Start TPC electronics installation on top of \dshort{detmodule} \#1     & April 2024   \\ \colhline
Start FEMB installation on APAs for \dshort{detmodule} \#1 &    August 2024    \\ \colhline
\rowcolor{dunepeach}Start of \dshort{detmodule} \#1 \dshort{tpc} installation& \startfirsttpcinstall      \\ \colhline
\rowcolor{dunepeach} Top of \dshort{detmodule} \#2 cryostat accessible& \accesstopsecondcryo      \\ \colhline %1/25
Complete FEMB installation on APAs for \dshort{detmodule} \#1   &March 2025    \\ \colhline
End DAQ installation    &May 2025    \\ \colhline
\rowcolor{dunepeach} End of \dshort{detmodule} \#1 \dshort{tpc} installation& \firsttpcinstallend      \\ \colhline %8/24
%DAQ operational for \dshort{detmodule} \#2 /  \#1 & May / Dec. 2025  \\ \colhline % Dec'19 per LBNC
\rowcolor{dunepeach}Start of \dshort{detmodule} \#2 \dshort{tpc} installation& \startsecondtpcinstall      \\ \colhline
End of FC production for \dshort{detmodule} \#1 &January 2026     \\ \colhline
End of APA production for \dshort{detmodule} \#1        &April 2026    \\ \colhline
%All TPC electronics components for module \#1 received in cavern & February 2026   \\ \colhline % Dec'19 per LBNC
%Begin installation of TPC electronics on top of cryostat \#1 & March 2026    \\ \colhline % Dec'19 per LBNC
%Start/end APA installation in \dshort{detmodule} \#1 & Apr. 2026 / Feb. 2027   \\ \colhline % Dec'19 per LBNC
\rowcolor{dunepeach} End \dshort{detmodule} \#2 \dshort{tpc} installation& \secondtpcinstallend      \\  \colhline
\rowcolor{dunepeach}Start detector module \#1 operations & July 2026 \\
\end{dunetable}




To monitor progress, \dword{jpo} scheduling team will maintain the
\dword{lbnf-dune} schedule that links all consortium schedules and
contains milestones for each consortia.  The schedules will go under
change control after each consortium agrees to the milestone dates,
the \dword{tdr} is approved, and the \dword{lbnf} project is baselined.

To ensure that the \dword{dune} detector remains on schedule,
\dword{tc} will monitor schedule status from each
consortium and organize reviews of schedules and risks as appropriate.
As schedule problems arise, \dword{tc} will work with the affected
consortium to resolve the problems. If problems cannot be solved, the
\dword{tcoord} will take the issue to the \dword{tb} and \dword{exb}.

A monthly report with input from all the consortia will be published by
\dword{tc} and provided to the \dword{lbnc}. This will
include updates on consortium and \dword{tc} technical progress
against the schedule.


%%%%%%%%%%%%%%%%%%%%%%%%%%%%%%%%
\section{Requirements}
\label{sec:fdsp-coord-requirements}

The scientific goals of \dword{dune} as described in %\dword{dune}
Volume~\volnumberexec:~\voltitleexec of this \dword{tdr}  include
\begin{itemize}
\item a comprehensive program of neutrino oscillation measurements
  including the search for \dword{cpv};
\item measurement of $\nu_{e}$ flux from a core-collapse supernova within our
  galaxy, should one occur during \dword{dune} operations; and 
\item a search for baryon number violation.
\end{itemize}
These goals motivate a number of key detector requirements: drift
field, electron lifetime, system noise, photon detector light yield
and time resolution. The \dword{exb} has approved a list of high-level
detector specifications, including those listed above. These will be
maintained in \dword{edms}, and the high-level requirements with
significant impact on physics (applying to both \dword{sp} and
\dword{dp} \dwords{detmodule} are highlighted in Table~\ref{tab:dunephysicsreqs}.
\begin{dunetable}
  [DUNE physics-related specifications owned by EB]
  {p{0.025\textwidth}p{0.06\textwidth}p{0.2\textwidth}p{0.35\textwidth}p{0.15\textwidth}p{0.1\textwidth}}
  {tab:dunephysicsreqs}
  {\dword{dune} physics-related specifications owned by \dword{exb}}
  ID & System & Parameter & Physics Requirement Driver & Requirement & Goal \\ \toprowrule
  1   & HVS    & Minimum drift field &  Limit recombination, diffusion and space charge impacts on particle ID. Establish adequate \dword{s/n} for tracking. & >\SI{250}{V/cm} & \spmaxfield \\ \colhline
  2   & TPC Elec     & System noise & The noise specification is driven by pattern recognition and two-track separation.  & <\SI{1000}{enc} & ALARA \\ \colhline
  3   & PDS    & Light yield  & The light yield shall be sufficient to measure time of events with visible energy above 200 MeV.  Goal is 10\% energy measurement for visible energy of 10 MeV.  & >\SI{0.5}{pe/MeV} & >\SI{5}{pe/MeV}  \\ \colhline
  4   & PDS    & Time resolution  & The time resolution of the photon detection system shall be sufficient to assign a unique event time.  & $<\,\SI{1}{\micro\second}$ & $<\,\SI{100}{\nano\second}$  \\ \colhline
  5   & all    & liquid argon purity & The LAr purity shall be sufficient to enable drift e- lifetime of 3 (10)ms & $<$\,\SI{100}{ppt} & $<$\,\SI{30}{ppt} \\ 
\end{dunetable}
Eleven other significant specifications for the \dword{spmod}
owned by the \dword{exb} are listed in Table~\ref{tab:dunephysicsspecs}
along with another twelve high-level engineering specifications.
\begin{dunetable}
  [DUNE high-level system specifications owned by the EB]
  {p{0.025\textwidth}p{0.06\textwidth}p{0.2\textwidth}p{0.35\textwidth}p{0.15\textwidth}p{0.1\textwidth}}
  {tab:dunephysicsspecs}
  {\dword{dune} high-level system specifications owned by \dword{exb}}
  ID & System & Parameter & Physics Requirement Driver & Requirement & Goal \\ \toprowrule
  6   & APA & Gaps between APAs  & minimize events lost due to vertex in gaps between APAs (15mm on same support beam, 30mm on adjacent beams) & <\SI{30}{mm} & <\SI{15}{mm} \\ \colhline
  7   & DSS & Drift field uniformity & tolerance on drift field due to component location & $<\,\SI{1}{\%}$  &   \\ \colhline
  8   & APA & wire angles  & 0$^\circ$ collection, $\pm$35.7$^\circ$ induction &  &  \\ \colhline
  9   & APA & wire spacing  & \SI{4.669}{mm} for U,V; \SI{4.790}{mm} for X,G &  &  \\ \colhline
  10  & APA & wire position tolerance  & & $\pm\,\SI{0.5}{mm}$  &  \\ \colhline
  11  & HVS & Drift field uniformity & tolerance on drift field due to HVS system & $<\,\SI{1}{\%}$  &  \\ \colhline
  12  & HVS & Cathode power supply ripple & very small compared to intrinsic electronics noise & $<\,\SI{100}{enc}$ &   \\ \colhline
  13  & TPC Elec & Front end peaking time  & optimize vertex resolution & \SI{1}{\micro\second} &  \\ \colhline
  14  & TPC Elec & Signal saturation  & largest signals occur with multiple protons in the primary vertex & 500k $e^-$ &  \\ \colhline
  15  & cryo & LAr N$_2$ contamination  & optical attenuation length in liquid argon with 50~ppm of N$_2$ contamination is roughly 3~m & $<\,\SI{25}{ppm}$ &  \\ \colhline
  16  & all & Detector dead time  & risk of missing a supernova burst if all operating cryostats are offline & $<\,\SI{0.5}{\%}$ &  \\ 
\end{dunetable}
The high level \dword{dune} requirements that drive the \dword{lbnf} design are
maintained in~\citedocdb{112} and under change control. These are owned by
the \dword{dune} \dword{tcoord} and \dword{lbnf} project manager.

Lower level detector specifications are held by the consortia and
described in the \dword{dune} \dword{tdr} 
Volumes~\volnumbersp{}, \voltitlesp{}, and~\volnumberdp{}, \voltitledp{}. A complete list of detector specifications is
provided in Section~\ref{sec:fdsp-app-requirements}.

%%%%%%%%%%%%%%%%%%%%%%%%%%% Anne moved up to keep reqs with reqs
\section{Full DUNE Requirements}
\label{sec:fdsp-app-requirements}

\fixme{The first tables in each section, the top-level requirements for SP and DP, respectively, repeat the first five specs. I've asked Brett to look at the code. Anne}

%%%%%%%%%%%%%%%%%%%%%%%%%%%%%%%
\subsection{Single-phase}
\label{sec:tc-req-sp}

% This file is generated, any edits may be lost.

% It defines macros which expand to corresponding
% specification values for subsystem SP-FD



\begin{longtable}{p{0.13\textwidth}p{0.23\textwidth}p{0.13\textwidth}p{0.27\textwidth}p{0.13\textwidth}}   
\caption{Specification for SP-FD \fixmehl{ref \texttt{tab:specs:SP-FD}}} \\

\rowcolor{dunesky}
  Label & Name  & Specification \newline (Goal) & Rationale & Validation \\  \colhline

\newtag{SP-FD-1}{ spec:min-drift-field }  & Minimum drift field  &  $>$\,\SI{250}{ V/cm} \newline ( $>\,\SI{500}{ V/cm}$ ) &  Lessens impacts of e-Ar recombination, e-lifetime, e-diffusion and space charge. &  ProtoDUNE \\ \colhline
    
    

  \newtag{SP-FD-2}{ spec:system-noise }  & System noise  &  $<\,\SI{1000}{enc}$ &  Provides $>$5:1 S/N on induction planes for  pattern recognition and two-track separation. &  ProtoDUNE and simulation \\ \colhline
    
    
\newtag{SP-FD-3}{ spec:light-yield }  & Light yield  &  $>\,\SI{0.5}{pe/MeV}$ \newline ( $>\,\SI{5}{pe/MeV}$ ) &  Rejects nucleon decay backgrounds from cosmogenic events near cathode. &   \\ \colhline
    
    
\newtag{SP-FD-4}{ spec:time-resolution-pds }  & Time resolution  &  $<\,\SI{1}{\micro\second}$ \newline ( $<\,\SI{100}{\nano\second}$ ) &  Enables \SI{1}{mm} position resolution for \SI{10}{MeV} SNB candidate events for instantaneous rate $<\,\SI{1}{m^{-3}ms^{-1}}$. &   \\ \colhline
    
    
\newtag{SP-FD-5}{ spec:lar-purity }  & Liquid argon purity  &  $<$\,\SI{100}{ppt} \newline ( $<\,\SI{30}{ppt}$ ) &  Provides $>$5:1 S/N on induction planes for  pattern recognition and two-track separation. &  Purity monitors and cosmic ray tracks \\ \colhline
    
    

  \newtag{SP-FD-6}{ spec:apa-gaps }  & Gaps between APAs   &  $<\,\SI{15}{mm}$ between APAs on same support beam; $<\,\SI{30}{mm}$ between APAs on different support beams &  Maintains fiducial volume.  Simplified contruction. &  ProtoDUNE \\ \colhline
    
    

  \newtag{SP-FD-7}{ spec:misalignment-field-uniformity }  & Drift field uniformity due to component alignment  &  $<\,1\,$\% throughout volume &  Maintains APA, CPA,  FC orientation and shape. &  ProtoDUNE \\ \colhline
    
    

  \newtag{SP-FD-8}{ spec:apa-wire-angles }  & APA wire angles  &  \SI{0}{\degree} for collection wires, \SI{35.7}{\degree} for induction wires &  Minimize inter-APA dead space. &  Engineering calculation \\ \colhline
    
    

  \newtag{SP-FD-9}{ spec:apa-wire-spacing }  & APA wire spacing  &  \SI{4.669}{mm} for U,V; \SI{4.790}{mm} for X,G &  Enables 100\% efficient MIP detection, \SI{1.5}{cm} $yz$ vertex resolution. &  Simulation \\ \colhline
    
    

  \newtag{SP-FD-10}{ spec:apa-wire-pos-tolerance }  & APA wire position tolerance  &  $\pm\,\SI{0.5}{mm}$ &  Interplane electron transparency; $dE/dx$, range, and MCS calibration. &  ProtoDUNE and simulation \\ \colhline
    
    

  \newtag{SP-FD-11}{ spec:hvs-field-uniformity }  & Drift field uniformity due to HVS  &  $<\,\SI{1}{\%}$ throughout volume &  High reconstruction efficiency. &  ProtoDUNE and simulation \\ \colhline
    
    

  \newtag{SP-FD-12}{ spec:hv-ps-ripple }  & Cathode HV power supply ripple contribution to system noise  &  $<\,\SI{100}{enc}$ &  Maximize live time; maintain high S/N. &  Engineering calculation, in situ measurement,   ProtoDUNE \\ \colhline
    
    
\newtag{SP-FD-13}{ spec:fe-peak-time }  & Front-end peaking time  &  \SI{1}{\micro\second} \newline ( Adjustable so as to see saturation in less than \SI{10}{\%} of beam-produced events ) &  Vertex resolution; optimized for \SI{5}{mm} wire spacing. &  ProtoDUNE and simulation \\ \colhline
    
    

  \newtag{SP-FD-14}{ spec:sp-signal-saturation }  & Signal saturation level  &  \num{500000} electrons &  Maintain calorimetric performance for multi-proton final state. &  Simulation \\ \colhline
    
    

  \newtag{SP-FD-15}{ spec:lar-n-contamination }  & LAr nitrogen contamination  &  $<\,\SI{25}{ppm}$ &  Maintain \SI{0.5}{PE/MeV} PDS sensitivity required for triggering proton decay near cathode. &  In situ measurment \\ \colhline
    
    

  \newtag{SP-FD-16}{ spec:det-dead-time }  & Detector dead time  &  $<\,\SI{0.5}{\%}$ &  Meet physics goals in timely fashion. &  ProtoDUNE \\ \colhline
    
    
\newtag{SP-FD-17}{ spec:cathode-resistivity }  & Cathode resistivity  &  $>\,\SI{1}{\mega\ohm/square}$ \newline ( $>\,\SI{1}{\giga\ohm/square}$ ) &  Detector damage prevention. &  ProtoDUNE \\ \colhline
    
    

  \newtag{SP-FD-18}{ spec:cryo-monitor-devices }  & Cryogenic monitoring devices  &   &  Constrain uncertainties on detection efficiency, fiducial volume. &  ProtoDUNE \\ \colhline
    
    

  \newtag{SP-FD-19}{ spec:adc-sampling-freq }  & ADC sampling frequency  &  $\sim\,\SI{2}{\mega\hertz}$ &  Match \SI{1}{\micro\second} shaping time. &  Nyquist requirement and design choice \\ \colhline
    
    
\newtag{SP-FD-20}{ spec:adc-number-of-bits }  & Number of ADC bits  &  \num{12} bits \newline ( \num{13} bits ) &  ADC noise contribution negligible (low end); match signal saturation specification (high end). &  Engineering calculation and design choice \\ \colhline
    
    

  \newtag{SP-FD-21}{ spec:ce-power-consumption }  & Cold electronics power consumption   &  $<\,\SI{50}{ mW/channel} $ &  No bubbles in LAr to redice HV discharge risk. &  ProtoDUNE \\ \colhline
    
    

  \newtag{SP-FD-22}{ spec:data-rate-to-tape }  & Data rate to tape  &  $<\,\SI{30}{PB/year}$ &  Cost.  Bandwidth. &  ProtoDUNE \\ \colhline
    
    

  \newtag{SP-FD-23}{ spec:sn-trigger }  & Supernova trigger  &  $>\,\SI{90}{\%}$ efficiency for SNB within \SI{100}{kpc} &  $>\,$90\% efficiency for SNB within 100 kpc &  Simulation and bench tests \\ \colhline
    
    

  \newtag{SP-FD-24}{ spec:local-e-fields }  & Local electric fields  &  $<\,\SI{30}{kV/cm}$ &  Maximize live time; maintain high S/N. &  ProtoDUNE \\ \colhline
    
    

  \newtag{SP-FD-25}{ spec:non-fe-noise }  & Non-FE noise contributions  &  $<<\,\SI{1000}{enc} $ &  High S/N for high reconstruction efficiency. &  Engineering calculation and ProtoDUNE \\ \colhline
    
    

  \newtag{SP-FD-26}{ spec:lar-impurity-contrib }  & LAr impurity contributions from components  &  $<<\,\SI{30}{ppt} $ &  Maintain HV operating range for high live time fraction. &  ProtoDUNE \\ \colhline
    
    

  \newtag{SP-FD-27}{ spec:radiopurity }  & Introduced radioactivity  &  less than that from $^{39}$Ar &  Maintain low radiological backgrounds for SNB searches. &  ProtoDUNE and assays during construction \\ \colhline
    
    

  \newtag{SP-FD-28}{ spec:dead-channels }  & Dead channels  &  $<\,\SI{1}{\%}$ &  Contingency for possible efficiency loss for $>\,$20 year operation.  &  ProtoDUNE \\ \colhline
    
    


\end{longtable} 

% This file is generated, any edits may be lost.

% It defines macros which expand to corresponding
% specification values for subsystem SP-APA



\begin{longtable}{p{0.13\textwidth}p{0.33\textwidth}p{0.13\textwidth}p{0.17\textwidth}p{0.13\textwidth}}   
\caption{Specification for SP-APA \fixmehl{ref \texttt{tab:specs:SP-APA}}} \\

\rowcolor{dunesky}
  Label & Description  & Specification \newline (Goal) & Rationale & Validation \\  \colhline


  \newtag{SP-APA-1}{ spec:apa-unit-size }  & Overall dimensions of a single anode plane assembly  &  \SI{6.0}{m} tall $\times$ \SI{2.3}{m} wide &  Maximum size allowed for fabrication, transportation, and installation.  &  ProtoDUNE-SP  \\ \colhline
    
    

  \newtag{SP-APA-2}{ spec:apa-active-area }  & APAs should be sensitive over most of the full area of an APA frame, limiting dead regions in the detector volume.  &  Maximize total active area. &  Maximize area for data collection  &  ProtoDUNE-SP  \\ \colhline
    
    

  \newtag{SP-APA-3}{ spec:apa-wire-tension }  & APA wires shall not touch during operation and break risk must be kept to a minimum.   &  \SI{6}{N} $\pm$ \SI{1}{N} &  Prevent contact beween wires and minimize  break risk &  ProtoDUNE-SP \\ \colhline
    
    

  \newtag{SP-APA-4}{ spec:apa-bias-voltage }  & APAs should produce optimal and uniform induction and collection signal shapes.  &  The setup, including boards, must hold 150\% of max operating voltage. &  Headroom in case adjustments needed &  E-field simulation sets wire bias voltages. ProtoDUNE-SP confirms performance. \\ \colhline
    
    

  \newtag{SP-APA-5}{ spec:apa-frame-planarity }  & Overall twist of the APA frame.  &  $<$\SI{5}{mm} &  APA transparency.  Ensures wire plane spacing change of $<$0.5 mm.  &  ProtoDUNE-SP \\ \colhline
    
    

  \newtag{SP-APA-6}{ spec:apa-bad-channels }  & Number of channels incapable of recording signals.  &  $<$1\%, with a goal of $<$0.5\% &  Reconstruction efficiency &  ProtoDUNE-SP \\ \colhline
    
    


\end{longtable} 
% This file is generated, any edits may be lost.

% It defines macros which expand to corresponding
% specification values for subsystem SP-HV



\begin{longtable}{p{0.25\textwidth}p{0.7\textwidth}}   
\caption{Specification for SP-HV \fixmehl{ref \texttt{tab:specs:SP-HV}}} \\

\rowcolor{dunesky}
\newtag{SP-HV-1}{ spec:power-supply-stability } & Name: Maximize power supply stability \\ 
    Description & The external voltage source should provide the required voltages stably and reproducibly.   \\  \colhline
    
    Specification &   \\   \colhline
    
    Rationale &     \\ \colhline
    Validation &   \\
   \colhline
\rowcolor{dunesky}
\newtag{SP-HV-2}{ spec:hv-connection-redundancy } & Name: Provide redundancy in all \dword{hv} connections. \\ 
    Description & Redundant connections allow for operations if there is a single-point failure.   \\  \colhline
    Specification (Goal) &  Two-fold  ( Four-fold ) \\   \colhline
    
    Rationale &     \\ \colhline
    Validation &   \\
   \colhline


\end{longtable} 
% This file is generated, any edits may be lost.
\begin{footnotesize}
%\begin{longtable}{p{0.14\textwidth}p{0.13\textwidth}p{0.18\textwidth}p{0.22\textwidth}p{0.20\textwidth}}
\begin{longtable}{p{0.12\textwidth}p{0.18\textwidth}p{0.17\textwidth}p{0.25\textwidth}p{0.16\textwidth}}
\caption{TPC electronics specifications \fixmehl{ref \texttt{tab:spec:SP-ELEC}}} \\
  \rowcolor{dunesky}
       Label & Description  & Specification \newline (Goal) & Rationale & Validation \\  \colhline

    \\ \rowcolor{dunesky} \newtag{SP-FD-2}{ spec:system-noise } & Name: System noise \\
    Description & The total system noise seen by each wire should be no more than 1000 enc of noise. It is expected that random noise on the FE amplifier will be the dominant contribution to the total system noise.   \\  \colhline
    Specification &  $<\,\SI{1000}{enc}$ \\   \colhline
    Rationale &   Provides $>$5:1 S/N on induction planes for  pattern recognition and two-track separation.  \\ \colhline
    Validation & ProtoDUNE and simulation  \\
   \colhline

    \\ \rowcolor{dunesky} \newtag{SP-FD-13}{ spec:fe-peak-time } & Name: Front-end peaking time \\
    Description & The FE peaking time shall be set so as to optimize vertex resolution.    \\  \colhline
    Specification (Goal) &  \SI{1}{\micro\second}  ( Adjustable so as to see saturation in less than \SI{10}{\%} of beam-produced events ) \\   \colhline
    Rationale &   Vertex resolution; optimized for \SI{5}{mm} wire spacing.  \\ \colhline
    Validation & ProtoDUNE and simulation  \\
   \colhline

   \newtag{SP-FD-14}{ spec:sp-signal-saturation }  & Signal saturation level  &  \num{500000} $e^-$ \newline (Adjustable so as to see saturation in less than \SI{10}{\%} of beam-produced events) &  Maintain calorimetric performance for multi-proton final state. &  Simulation \\ \colhline
    
   
  \newtag{SP-FD-19}{ spec:adc-sampling-freq }  & ADC sampling frequency  &  $\sim\,\SI{2}{\mega\hertz}$ &  Match \SI{1}{\micro\second} shaping time. &  Nyquist requirement and design choice \\ \colhline
    
    \\ \rowcolor{dunesky} \newtag{SP-FD-20}{ spec:adc-number-of-bits } & Name: Number of ADC bits \\
    Description & The ADC shall digitize the charge deposited on the wires with12 bits precision.   \\  \colhline
    Specification (Goal) &  \num{12} bits  ( \num{13} bits ) \\   \colhline
    Rationale &   ADC noise contribution negligible (low end); match signal saturation specification (high end).  \\ \colhline
    Validation & Engineering calculation and design choice  \\
   \colhline

    \\ \rowcolor{dunesky} \newtag{SP-FD-21}{ spec:ce-power-consumption } & Name: Cold electronics power consumption  \\
    Description & The CE power consumption shall remain below 50 mW/channel.    \\  \colhline
    Specification &  $<\,\SI{50}{ mW/channel} $ \\   \colhline
    Rationale &   No bubbles in LAr to redice HV discharge risk.  \\ \colhline
    Validation & ProtoDUNE  \\
   \colhline

   
  \newtag{SP-FD-25}{ spec:non-fe-noise }  & Non-FE noise contributions  &  $<<\,\SI{1000}\,e^- $ &  High S/N for high reconstruction efficiency. &  Engineering calculation and ProtoDUNE \\ \colhline
    
   
  \newtag{SP-FD-28}{ spec:dead-channels }  & Dead channels  &  $<\,\SI{1}{\%}$ &  Minimize the degradation in physics performance over the $>\,20$-year detector operation. &  ProtoDUNE and bench tests \\ \colhline
    

   \newtag{SP-ELEC-1}{ spec:num-FE-baselines }  & Number of baselines in the front-end amplifier  &  2.0 \newline ( 2.0 ) &  Use a single type of amplifier for both induction and collection wires &  ProtoDUNE \\ \colhline
    
   \newtag{SP-ELEC-2}{ spec:gain-FE-amplifier }  & Gain of the front-end amplifier  &  $\sim\SI{20}{mV/fC}$ \newline (Adjustable in the range \SIrange{5}{25}{mV/fC}) &  The gain of the FE amplifier is obtained from the maximum charge to be observed without saturation and from the operating voltage of the amplifier, that depends on the technology choice. &   \\ \colhline
    
   \newtag{SP-ELEC-3}{ spec:FEMB-data-link }  & Data transmission speed between the FEMB and the WIB  &  4.0 \newline ( 2.0 ) &  Balance between reducing the number of links and reliability and power issues when increasing the data transmission speed. &  ProtoDUNE, Laboratory measurements on bit error rates \\ \colhline
    
   
  \newtag{SP-ELEC-4}{ spec:num-channels-FEMB }  & Number of channels per front-end motherboard  &  \num{128} &  The total number of wires on one side of an APA, 1,280, must be an integer multiple of the number of channels on the FEMBs. &  Design \\ \colhline
    
   \newtag{SP-ELEC-5}{ spec:FEMB-data-link }  & Number of links between the FEMB and the WIB  &  \num{4} at \SI{1.28}{Gbps} \newline (\num{2} at \SI{2.56}{Gbps}) &  Balance between reducing the number of links and reliability and power issues when increasing the data transmission speed. &  ProtoDUNE, Laboratory measurements on bit error rates \\ \colhline
    
   
  \newtag{SP-ELEC-6}{ spec:number-FEMB-per-WIB }  & Number of FEMBs per WIB  &  \num{4} &  The total number of FEMB per WIB is a balance between the complexity of the boards, the mechanics inside the WIEC, and the required processing power of the FPGA on the WIB.  &  ProtoDUNE, Design \\ \colhline
    
   
  \newtag{SP-ELEC-7}{ spec:WIB-data-link }  & Data transmission speed between the WIB and the DAQ backend  &  \SI{10}{Gbps} &  Balance between cost and reduction of the number of optical fiber links for each WIB. &  ProtoDUNE, Laboratory measurements on bit error rates \\ \colhline
    
   
  \newtag{SP-ELEC-8}{ spec:cold-cables-xsec }  & Cross section of cold cables  &  \SI{2.5}{in} &  Avoid the need for further changes to the APA frame and for routing the cables along the cryostat walls &  Tests on APA frame prototypes \\ \colhline
    


\label{tab:specs:SP-ELEC}
\end{longtable}
\end{footnotesize} 
% This file is generated, any edits may be lost.

% It defines macros which expand to corresponding
% specification values for subsystem SP-PDS



\begin{longtable}{p{0.25\textwidth}p{0.7\textwidth}}   
\caption{Specification for SP-PDS \fixmehl{ref \texttt{tab:specs:SP-PDS}}} \\

\rowcolor{dunesky}
\newtag{SP-PDS-1}{ spec:ly-uniformity } & Name: Light yield uniformity \\ 
    Description & The light yield uniformity shall remain less than 50\%.    \\  \colhline
    Specification (Goal) &  < \num{50}\%  ( < \num{20}\% ) \\   \colhline
    
    Rationale &   Rejection of low-energy background. Improved energy resolution through light-based calorimetric measurements.  \\ \colhline
    Validation &   \\
   \colhline
\rowcolor{dunesky}
\newtag{SP-PDS-2}{ spec:spatial-localization } & Name: Spatial localization \\ 
    Description & Events inside the active volume shall be localized in 3D  to within < \SI{100}{\cm} using light signals.   \\  \colhline
    Specification (Goal) &  < \SI{100}{\cm}  ( < \SI{50}{\cm} ) \\   \colhline
    
    Rationale &     \\ \colhline
    Validation &   \\
   \colhline
\rowcolor{dunesky}
\newtag{SP-PDS-3}{ spec:env-light-exposure } & Name: Environmental light exposure \\ 
    Description & Blue/UV Light exposure to the PD modules should be minimized.  No exposure to sunlight at any time.  UV-Filtered light (>\SI{400}{nm}) during all exposure.   \\  \colhline
    Specification (Goal) &  \num{0} sunlight; ALARA other sources  ( ALARA ) \\   \colhline
    
    Rationale &     \\ \colhline
    Validation &   \\
   \colhline
\rowcolor{dunesky}
\newtag{SP-PDS-4}{ spec:env-humidity-limit } & Name: Environmental humidity limit \\ 
    Description & All working environments with exposed TPB coatings must maintain <\SI{50}{\%} Relative Humidity (RH) at  \SI{70}{\degree F}.   \\  \colhline
    Specification (Goal) &  < \SI{50}{\%} RH at \SI{70}{\degree F}  ( ALARA ) \\   \colhline
    
    Rationale &     \\ \colhline
    Validation &   \\
   \colhline
\rowcolor{dunesky}
\newtag{SP-PDS-5}{ spec:light-tightness } & Name: Light-tight cryostat \\ 
    Description & Noise rate in photon detectors due to external sources must be less that <\SI{10}{\%} of that induced by radiological background.   \\  \colhline
    Specification (Goal) &  <\SI{10}{\%}  ( ALARA ) \\   \colhline
    
    Rationale &     \\ \colhline
    Validation &   \\
   \colhline
\rowcolor{dunesky}
\newtag{SP-PDS-6}{ spec:ed-light } & Name: Light from electrical discharge \\ 
    Description & Induced PD single-PE event rate due to flashing from HV electrical discharging or corona effect shall be less than <\SI{10}{\%} of that induced by radiological background.   \\  \colhline
    Specification (Goal) &  <\SI{10}{\%}  ( ALARA ) \\   \colhline
    
    Rationale &     \\ \colhline
    Validation &   \\
   \colhline
\rowcolor{dunesky}
\newtag{SP-PDS-7}{ spec:mech-deflection } & Name: Mechanical deflection (static) \\ 
    Description & The PDS shall move no more than \SI{5}{\mm} relative to  the  horizontal and vertical orientation of APA (or move in any direction at all?)   \\  \colhline
    Specification (Goal) &  $<$\SI{5}{\milli\meter}  ( ALARA ) \\   \colhline
    
    Rationale &     \\ \colhline
    Validation &   \\
   \colhline
\rowcolor{dunesky}
\newtag{SP-PDS-8}{ spec:apa-install } & Name: Clearance for installation through APA side tubes \\ 
    Description & PD modules must fit and be secured to the APA through slots in one side of the APA, as designed in concert with APA group.   \\  \colhline
    
    Specification &  $>$\SI{1}{\milli\meter} \\   \colhline
    
    Rationale &     \\ \colhline
    Validation &   \\
   \colhline
\rowcolor{dunesky}
\newtag{SP-PDS-9}{ spec:pds-compatible } & Name: No mechanical interference with APA, SP-CE and SP-HV detector elements (clearance) \\ 
    Description & SP-PD system, including cables and mechanical supports, must fit within APA and not interfere with SP-CE.   \\  \colhline
    
    Specification &  $>$\SI{1}{\milli\meter} \\   \colhline
    
    Rationale &     \\ \colhline
    Validation &   \\
   \colhline
\rowcolor{dunesky}
\newtag{SP-PDS-10}{ spec:pds-cable } & Name: PD cable routing APA intrusion \\ 
    Description & The SP-PD cable system must be installed prior to APA wire wrapping.  Module connection to the cable system must occur without impinging into the APA side tubes more than \SI{6}{\milli\meter}.   \\  \colhline
    
    Specification &  $<$\SI{6}{\milli\meter} \\   \colhline
    
    Rationale &     \\ \colhline
    Validation &   \\
   \colhline
\rowcolor{dunesky}
\newtag{SP-PDS-11}{ spec:pds-cablemate } & Name: Upper-Lower APA junction gap \\ 
    Description & The PD cabling system must allow for the upper and lower APAs to mate with a maximum \SI{6}{\milli\meter} gap between APA frame footer tubes.   \\  \colhline
    
    Specification &  $<$\SI{6}{\milli\meter} \\   \colhline
    
    Rationale &     \\ \colhline
    Validation &   \\
   \colhline
\rowcolor{dunesky}
\newtag{SP-PDS-12}{ spec:pds-location } & Name: Maintain detector location at LAr temperature.  \\ 
    Description & No absolute number is required for physics. The requirement is driven by engineering to ensure no damage occurs.   \\  \colhline
    
    Specification &  (Dave: what is required to avoid damage) \\   \colhline
    
    Rationale &     \\ \colhline
    Validation &   \\
   \colhline
\rowcolor{dunesky}
\newtag{SP-PDS-13}{ spec:pds-datarate } & Name: Data transfer from SP-PD to DAQ \\ 
    Description & <8 Gbps per DAQ input steady state per APA.  Burst rate TBD.     \\  \colhline
    
    Specification &  $<$\SI{8}{Gbps} \\   \colhline
    
    Rationale &     \\ \colhline
    Validation &   \\
   \colhline


\end{longtable}

% This file is generated, any edits may be lost.

\begin{longtable}{p{0.14\textwidth}p{0.13\textwidth}p{0.18\textwidth}p{0.22\textwidth}p{0.20\textwidth}}
\caption{Specifications for SP-CALIB \fixmehl{ref \texttt{tab:spec:SP-CALIB}}} \\
  \rowcolor{dunesky}
       Label & Description  & Specification \newline (Goal) & Rationale & Validation \\  \colhline

    \\ \rowcolor{dunesky} \newtag{SP-FD-1}{ spec:min-drift-field } & Name: Minimum drift field \\
    Description & The drift field in the TPC shall be greater than 250 V/cm, with a goal of 500 V/cm.   \\  \colhline
    Specification (Goal) &  $>$\,\SI{250}{ V/cm}  ( $>\,\SI{500}{ V/cm}$ ) \\   \colhline
    Rationale &   Lessens impacts of e-Ar recombination, e-lifetime, e-diffusion and space charge.  \\ \colhline
    Validation & ProtoDUNE  \\
   \colhline

    \\ \rowcolor{dunesky} \newtag{SP-FD-2}{ spec:system-noise } & Name: System noise \\
    Description & The total system noise seen by each wire should be no more than 1000 enc of noise. It is expected that random noise on the FE amplifier will be the dominant contribution to the total system noise.   \\  \colhline
    Specification &  $<\,\SI{1000}{enc}$ \\   \colhline
    Rationale &   Provides $>$5:1 S/N on induction planes for  pattern recognition and two-track separation.  \\ \colhline
    Validation & ProtoDUNE and simulation  \\
   \colhline

    \\ \rowcolor{dunesky} \newtag{SP-FD-3}{ spec:light-yield } & Name: Light yield \\
    Description & The light yield shall be sufficient for measuring event time (and total intensity) of events with visible energy above 200 MeV.  Goal is to make possible a 10\% energy measurement for events with a visible energy of 10 MeV.   \\  \colhline
    Specification (Goal) &  $>\,\SI{0.5}{pe/MeV}$  ( $>\,\SI{5}{pe/MeV}$ ) \\   \colhline
    Rationale &   Rejects nucleon decay backgrounds from cosmogenic events near cathode.  \\ \colhline
    Validation &   \\
   \colhline

    \\ \rowcolor{dunesky} \newtag{SP-FD-4}{ spec:time-resolution-pds } & Name: Time resolution \\
    Description & The time resolution of the photon detection system shall be less than 1 microsecond in order to assign a unique event time.   \\  \colhline
    Specification (Goal) &  $<\,\SI{1}{\micro\second}$  ( $<\,\SI{100}{\nano\second}$ ) \\   \colhline
    Rationale &   Enables \SI{1}{mm} position resolution for \SI{10}{MeV} SNB candidate events for instantaneous rate $<\,\SI{1}{m^{-3}ms^{-1}}$.  \\ \colhline
    Validation &   \\
   \colhline

   \newtag{SP-FD-5}{ spec:lar-purity }  & Liquid argon purity  &  $<$\,\SI{100}{ppt} \newline ($<\,\SI{30}{ppt}$) &  Provides $>$5:1 S/N on induction planes for  pattern recognition and two-track separation. &  Purity monitors and cosmic ray tracks \\ \colhline
    
    \\ \rowcolor{dunesky} \newtag{SP-FD-7}{ spec:misalignment-field-uniformity } & Name: Drift field uniformity due to component alignment \\
    Description & Misalignments of the various TPC components shall not introduce drift-field nonuniformities beyond those specified in the HVS requirements.   \\  \colhline
    Specification &  $<\,1\,$\% throughout volume \\   \colhline
    Rationale &   Maintains APA, CPA,  FC orientation and shape.  \\ \colhline
    Validation & ProtoDUNE  \\
   \colhline

   
  \newtag{SP-FD-9}{ spec:apa-wire-spacing }  & APA wire spacing  &  \SI{4.669}{mm} for U,V; \SI{4.790}{mm} for X,G &  Enables 100\% efficient MIP detection, \SI{1.5}{cm} $yz$ vertex resolution. &  Simulation \\ \colhline
    
    \\ \rowcolor{dunesky} \newtag{SP-FD-11}{ spec:hvs-field-uniformity } & Name: Drift field uniformity due to HVS \\
    Description & Design of TPC cathode and FC components shall ensure uniform field.  Production tolerances shall be set so as to maintain flatness of component surfaces and, by extension, the shape of the drift field volume.   \\  \colhline
    Specification &  $<\,\SI{1}{\%}$ throughout volume \\   \colhline
    Rationale &   High reconstruction efficiency.  \\ \colhline
    Validation & ProtoDUNE and simulation  \\
   \colhline

    \\ \rowcolor{dunesky} \newtag{SP-FD-13}{ spec:fe-peak-time } & Name: Front-end peaking time \\
    Description & The FE peaking time shall be set so as to optimize vertex resolution.    \\  \colhline
    Specification (Goal) &  \SI{1}{\micro\second}  ( Adjustable so as to see saturation in less than \SI{10}{\%} of beam-produced events ) \\   \colhline
    Rationale &   Vertex resolution; optimized for \SI{5}{mm} wire spacing.  \\ \colhline
    Validation & ProtoDUNE and simulation  \\
   \colhline

   
  \newtag{SP-FD-16}{ spec:det-dead-time }  & Detector dead time  &  $<\,\SI{0.5}{\%}$ &  Meet physics goals in timely fashion. &  ProtoDUNE \\ \colhline
    
    \\ \rowcolor{dunesky} \newtag{SP-FD-22}{ spec:data-rate-to-tape } & Name: Data rate to tape \\
    Description & The DAQ shall provide capability for triggering on events of interest in order to limit the total volume for events stored on tape.   \\  \colhline
    Specification &  $<\,\SI{30}{PB/year}$ \\   \colhline
    Rationale &   Cost.  Bandwidth.  \\ \colhline
    Validation & ProtoDUNE  \\
   \colhline

   
  \newtag{SP-FD-23}{ spec:sn-trigger }  & Supernova trigger  &  $>\,\SI{90}{\%}$ efficiency for SNB within \SI{100}{kpc} &  $>\,$90\% efficiency for SNB within 100 kpc &  Simulation and bench tests \\ \colhline
    
   
  \newtag{SP-FD-24}{ spec:local-e-fields }  & Local electric fields  &  $<\,\SI{30}{kV/cm}$ &  Maximize live time; maintain high S/N. &  ProtoDUNE \\ \colhline
    
   
  \newtag{SP-FD-25}{ spec:non-fe-noise }  & Non-FE noise contributions  &  $<<\,\SI{1000}\,e^- $ &  High S/N for high reconstruction efficiency. &  Engineering calculation and ProtoDUNE \\ \colhline
    
   
  \newtag{SP-FD-26}{ spec:lar-impurity-contrib }  & LAr impurity contributions from components  &  $<<\,\SI{30}{ppt} $ &  Maintain HV operating range for high live time fraction. &  ProtoDUNE \\ \colhline
    
   
  \newtag{SP-FD-27}{ spec:radiopurity }  & Introduced radioactivity  &  less than that from $^{39}$Ar &  Maintain low radiological backgrounds for SNB searches. &  ProtoDUNE and assays during construction \\ \colhline
    

   \newtag{SP-CALIB-1}{ spec:efield-calib-precision }  & Ionization laser electric field measurement precision  &  \SI{1}{\%} \newline ( $<$\SI{1}{\%} ) &  Electric field affects energy and position measurements. &  ProtoDUNE and external experiments. \\ \colhline
    
   \newtag{SP-CALIB-2}{ spec:efield-calib-coverage }  & Ionization laser electric field measurement coverage  &  $>$\SI{75}{\%} \newline ( \SI{100}{\%} ) &  The necessary coverage depends on how high a distortion can be reasonably expected. &  ProtoDUNE \\ \colhline
    
   \newtag{SP-CALIB-3}{ spec:efield-calib-granularity }  & Ionization laser \efield measurement  granularity  &  $~30\times 30\times 30~$\SI{}{\centi\metre\cubed} \newline ($10\times 10\times 10~$\SI{}{\centi\metre\cubed}) &  Minimum measurable region is set by the maximum expected distortion and position reconstruction requirements. &  ProtoDUNE \\ \colhline
    
   \newtag{SP-CALIB-4}{ spec:laser-position-precision }  & Laser beam position precision  &  \SI{0.5}{\milli\rad} \newline ( $<\,\SI{0.5}{\milli\rad}$ ) &  The necessary spatial precision does not need to be smaller than the APA wire gap. &  ProtoDUNE \\ \colhline
    
   \newtag{SP-CALIB-5}{ spec:neutron-source-coverage }  & Neutron source coverage  &  $>$\SI{75}{\%} \newline ( \SI{100}{\%} ) &  The coverage of the pulsed neutron system depends on the energy resolution requirements at low energy. &  Simulations \\ \colhline
    
   \newtag{SP-CALIB-6}{ spec:data-volume-laser }  & Ionization laser DAQ rate per year (per 10 kton)  &  $>$\SI{84}{TB/yr/10 kton} \newline ( $>$\SI{185}{TB/yr/10 kton} ) &  The laser data volume must allow the needed coverage and granularity. &  protoDUNE/simulations \\ \colhline
    
   \newtag{SP-CALIB-7}{ spec:data-volume-pns }  & Neutron source DAQ rate per year (per 10 kt)  &  $>\,\SI{84}{TB/yr/10 kt}$ \newline ( $>\,\SI{168}{TB/yr/10 kt}$ ) &  The pulsed neutron system must allow the needed coverage and granularity. &  Simulations \\ \colhline
    
   
  \newtag{SP-CALIB-8}{ spec:rate-gammas-source }  & Rate of 9 MeV capture gamma events in the proposed radioactive source  &  $<\,\SI{1}{\kilo\hertz}$ &  The source rate must be such that there is no more than one event per drift time. &  Lab tests \\ \colhline
    


\label{tab:specs:SP-CALIB}
\end{longtable}
% This file is generated, any edits may be lost.

% It defines macros which expand to corresponding
% specification values for subsystem SP-DAQ



\begin{longtable}{p{0.25\textwidth}p{0.7\textwidth}}   
\caption{Specification for SP-DAQ \fixmehl{ref \texttt{tab:specs:SP-DAQ}}} \\

\rowcolor{dunesky}
\newtag{SP-DAQ-1}{ spec:trigger-high-energy } & Name: Off-beam High-energy Trigger \\ 
    Description & The detector shall trigger on the visible energy of underground physics events from decays or interactions within the active volume with high efficiency.   \\  \colhline
    
    Specification &  $>$\SI{100}{\MeV} \\   \colhline
    
    Rationale &   Study of these events (atmospheric neutrinos, baryon-number-violating events) is part of the DUNE mission.  Cosmic rays are also essential for calibration.  \\ \colhline
    Validation & Refer to physics TDR. 100 MeV is an achievable parameter; lower thesholds are possible.  \\
   \colhline
\rowcolor{dunesky}
\newtag{SP-DAQ-2}{ spec:trigger-low-energy } & Name: Off-beam Low-energy Trigger \\ 
    Description & The detector shall be capable of triggering on the visible energy of single low energy neutrino interactions inside the active volume.   \\  \colhline
    
    Specification &  $>$\SI{10}{\MeV} \\   \colhline
    
    Rationale &   Study of these events (solar, supernova neutrinos) enables other physics studies and provides monitoring of detector.   \\ \colhline
    Validation & Refer to physics TDR. 10 MeV is an achievable parameter; lower thresholds are possible.  \\
   \colhline
\rowcolor{dunesky}
\newtag{SP-DAQ-3}{ spec:trigger-beam } & Name: Trigger for Beam \\ 
    Description & The detector shall trigger on the visible energy of beam interactions within the active volume with efficiency high enough that it has a sub-dominant impact on physics sensitivity.   \\  \colhline
    
    Specification &  $>$\SI{100}{\MeV} \\   \colhline
    
    Rationale &   Study of these events is the primary DUNE mission  \\ \colhline
    Validation & Techniques for doing this have been run succesfully in MINOS, NOvA and T2K. 100 MeV is an achievable parameter; lower thresholds are possible.  \\
   \colhline
\rowcolor{dunesky}
\newtag{SP-DAQ-4}{ spec:trigger-calibration } & Name: Trigger for Calibration \\ 
    Description & The detector shall provide triggers to and trigger on calibration stimuli and tag the data from these triggers as such   \\  \colhline
    
    Specification &   \\   \colhline
    
    Rationale &   Calibration is essential to attain required detector performance  \\ \colhline
    Validation &   \\
   \colhline
\rowcolor{dunesky}
\newtag{SP-DAQ-5}{ spec:trigger-snb } & Name: Trigger for Supernova Burst \\ 
    Description & A trigger shall be generated when a collection of signals is detected that constitute a candidate supernova burst with high galactic coverage, while meeting offline storage requirements and overall bandwidth limitations.   \\  \colhline
    
    Specification &   \\   \colhline
    
    Rationale &   Study of supernova bursts is part of the DUNE mision, this is one of the ways to collect such data  \\ \colhline
    Validation & Refer to physics TDR.  \\
   \colhline
\rowcolor{dunesky}
\newtag{SP-DAQ-6}{ spec:data-record } & Name: Physics Event Record \\ 
    Description & The DAQ shall merge data into a form suitable for offline analysis. Furthermore, tags shall be provided to allow the data collection conditions at the time and the livetime to be determined.   \\  \colhline
    
    Specification &   \\   \colhline
    
    Rationale &   Traditionally these things are referred to offline as an 'event' and a 'run'  \\ \colhline
    Validation &   \\
   \colhline
\rowcolor{dunesky}
\newtag{SP-DAQ-7}{ spec:daq-deadtime } & Name: DAQ Deadtime \\ 
    Description & The DAQ shall operate with deadtime that does not contribute significantly to overall loss of detector livetime.   \\  \colhline
    
    Specification &   \\   \colhline
    
    Rationale &   Zero deadtime makes physics analysis bookeeping for overall livetime much simpler, and allows acquisition of neutrino events even with accidentals from backgrounds such as radiologicals.  \\ \colhline
    Validation & Zero deadtime is an achievable inter-event deadtime but a small deadtime would not significantly compromise physics sensitivity.  \\
   \colhline


\end{longtable}
 
% This file is generated, any edits may be lost.

% It defines macros which expand to corresponding
% specification values for subsystem SP-CISC



\begin{longtable}{p{0.13\textwidth}p{0.33\textwidth}p{0.13\textwidth}p{0.17\textwidth}p{0.13\textwidth}}   
\caption{Specification for SP-CISC \fixmehl{ref \texttt{tab:specs:SP-CISC}}} \\

\rowcolor{dunesky}
  Label & Description  & Specification \newline (Goal) & Rationale & Validation \\  \colhline

\newtag{SP-CISC-1}{ spec:inst-noise }  & The instrumentation devices shall contribute no more than \SI{1000}{enc} of noise, with a goal of ALARA. This requirement is on total system noise;  &  $\ll\,\SI{1000}{enc}$ \newline ( ALARA ) &  Top level spec 25 &   \\ \colhline
    
    
\newtag{SP-CISC-2}{ spec:inst-efield }  & The maximum field near instrumentation devices should be $<\,\SI{30}{kV/cm}$ to avoid dielectric breakdowns.  &  $<\,\SI{30}{kV/cm}$ \newline ( $<\,\SI{15}{kV/cm}$ ) &   &   \\ \colhline
    
    
\newtag{SP-CISC-3}{ spec:elec-lifetime-prec }  & The precision on the measurement of the electron lifetime needs to sufficient to ensure $<$ 0.5\% uncertainty in charge readout.  &  $<\,$1.4\% ($<$4\%) \newline ( $<\,$1\% ) &   &   \\ \colhline
    
    
\newtag{SP-CISC-4}{ spec:elec-lifetime-range }  & The purity monitors inside the cryostat should be capable of measuring a lifetime range between 0 and 10 ms. The goal for the inline purity monitors is to measure a range of 0 to 30 ms for the drift electron lifetime.  &  \SIrange{0}{10}{ms} (\SIrange{0}{30}{ms}) \newline ( \SIrange{0}{10}{ms} (\SIrange{0}{30}{ms}) ) &   &   \\ \colhline
    
    
\newtag{SP-CISC-11}{ spec:temp-repro }  & The RMS of the distribution of independent temperature offsets between two sensors in successive immersions in LAr should be $<$ 5  mK  &  $<\,\SI{5}{mK}$ \newline ( \SI{2}{mK} ) &   &   \\ \colhline
    
    
\newtag{SP-CISC-14}{ spec:temp-stability }  & The thermometers should match precision requirement at all places, at all times  &  $<\,\SI{2}{mK}$ at all places and times \newline ( Match precision requirement at all places, at all times ) &   &   \\ \colhline
    
    
\newtag{SP-CISC-27}{ spec:camera-cold-coverage }  & The cold cameras are required to cover at least 80\% of the exterior of HV surfaces.  &  $>\,$80\% of HV surfaces \newline ( \num{100}\% ) &   &   \\ \colhline
    
    
\newtag{SP-CISC-51}{ spec:slowcontrol-alarm-rate }  & The total number of alarms/day seen by operators need to be less than 150.  &  $<\,$150/day \newline ( $<\,$50/day ) &   &   \\ \colhline
    
    
\newtag{SP-CISC-52}{ spec:slowcontrol-num-vars }  & This is the total number of variables monitored by slow controls from all subsystems of the detector.  &  $>\,\num{150000}$ \newline ( \SIrange{150000}{200000} ) &   &   \\ \colhline
    
    
\newtag{SP-CISC-54}{ spec:slowcontrol-archive-rate }  & Slow control quantities will need to archived at a rate that ranges from 0.02 Hz to 1 per few minutes, depending on the slow controls quantity.  &  \SI{0.02}{Hz} \newline ( Broad range \SI{1}{Hz} to \num{1} per few min. ) &   &   \\ \colhline
    
    


\end{longtable}  
% This file is generated, any edits may be lost.

\begin{longtable}{p{0.14\textwidth}p{0.13\textwidth}p{0.18\textwidth}p{0.22\textwidth}p{0.20\textwidth}}
\caption{Specifications for SP-TC \fixmehl{ref \texttt{tab:spec:SP-TC}}} \\
  \rowcolor{dunesky}
       Label & Description  & Specification \newline (Goal) & Rationale & Validation \\  \colhline

    \\ \rowcolor{dunesky} \newtag{SP-FD-1}{ spec:min-drift-field } & Name: Minimum drift field \\
    Description & The drift field in the TPC shall be greater than 250 V/cm, with a goal of 500 V/cm.   \\  \colhline
    Specification (Goal) &  $>$\,\SI{250}{ V/cm}  ( $>\,\SI{500}{ V/cm}$ ) \\   \colhline
    Rationale &   Lessens impacts of e-Ar recombination, e-lifetime, e-diffusion and space charge.  \\ \colhline
    Validation & ProtoDUNE  \\
   \colhline

    \\ \rowcolor{dunesky} \newtag{SP-FD-2}{ spec:system-noise } & Name: System noise \\
    Description & The total system noise seen by each wire should be no more than 1000 enc of noise. It is expected that random noise on the FE amplifier will be the dominant contribution to the total system noise.   \\  \colhline
    Specification &  $<\,\SI{1000}{enc}$ \\   \colhline
    Rationale &   Provides $>$5:1 S/N on induction planes for  pattern recognition and two-track separation.  \\ \colhline
    Validation & ProtoDUNE and simulation  \\
   \colhline

    \\ \rowcolor{dunesky} \newtag{SP-FD-3}{ spec:light-yield } & Name: Light yield \\
    Description & The light yield shall be sufficient for measuring event time (and total intensity) of events with visible energy above 200 MeV.  Goal is to make possible a 10\% energy measurement for events with a visible energy of 10 MeV.   \\  \colhline
    Specification (Goal) &  $>\,\SI{0.5}{pe/MeV}$  ( $>\,\SI{5}{pe/MeV}$ ) \\   \colhline
    Rationale &   Rejects nucleon decay backgrounds from cosmogenic events near cathode.  \\ \colhline
    Validation &   \\
   \colhline

    \\ \rowcolor{dunesky} \newtag{SP-FD-4}{ spec:time-resolution-pds } & Name: Time resolution \\
    Description & The time resolution of the photon detection system shall be less than 1 microsecond in order to assign a unique event time.   \\  \colhline
    Specification (Goal) &  $<\,\SI{1}{\micro\second}$  ( $<\,\SI{100}{\nano\second}$ ) \\   \colhline
    Rationale &   Enables \SI{1}{mm} position resolution for \SI{10}{MeV} SNB candidate events for instantaneous rate $<\,\SI{1}{m^{-3}ms^{-1}}$.  \\ \colhline
    Validation &   \\
   \colhline

   \newtag{SP-FD-5}{ spec:lar-purity }  & Liquid argon purity  &  $<$\,\SI{100}{ppt} \newline ($<\,\SI{30}{ppt}$) &  Provides $>$5:1 S/N on induction planes for  pattern recognition and two-track separation. &  Purity monitors and cosmic ray tracks \\ \colhline
    
   
  \newtag{SP-FD-27}{ spec:radiopurity }  & Introduced radioactivity  &  less than that from $^{39}$Ar &  Maintain low radiological backgrounds for SNB searches. &  ProtoDUNE and assays during construction \\ \colhline
    

\input{generated/req-SP-TC-01.tex}
\input{generated/req-SP-TC-02.tex}
\input{generated/req-SP-TC-03.tex}
   
  \newtag{SP-TC-4}{ spec:apa-storage-sd }  & APA stroage at logistics facility in SD  &  na &  Store APAs during lag between production and installation &  Agree upon space needs. \\ \colhline
    
   
  \newtag{SP-TC-5}{ spec:cleanroom-specification }  & Standard for ITF and installation cleanrooms  &  na &  Reduce dust (contains U/Th) to prevent induced radiological background in detector &  Monitor air purity \\ \colhline
    
   
  \newtag{SP-TC-6}{ spec:cleanroom-uv-filters }  & UV filter in ITF and installation cleanrooms for PDS sensor protection  &  na &  Prevent damage to PD coatings  &  Visual or spectrographic inspection \\ \colhline
    


\label{tab:specs:SP-TC}
\end{longtable} % aka iic or install

%%%%%%%%%%%%%%%%%%%%%%%%%%%%%%%
\subsection{Dual-phase}
\label{sec:tc-req-dp}

\fixme{not all the DP req tables have been provided yet}

\input{generated/req-longtable-DP-FD}

% This file is generated, any edits may be lost.

\begin{longtable}{p{0.14\textwidth}p{0.13\textwidth}p{0.18\textwidth}p{0.22\textwidth}p{0.20\textwidth}}
\caption{Specifications for DP-HV \fixmehl{ref \texttt{tab:spec:DP-HV}}} \\
  \rowcolor{dunesky}
       Label & Description  & Specification \newline (Goal) & Rationale & Validation \\  \colhline

    \\ \rowcolor{dunesky} \newtag{SP-FD-1}{ spec:min-drift-field } & Name: Minimum drift field \\
    Description & The drift field in the TPC shall be greater than 250 V/cm, with a goal of 500 V/cm.   \\  \colhline
    Specification (Goal) &  $>$\,\SI{250}{ V/cm}  ( $>\,\SI{500}{ V/cm}$ ) \\   \colhline
    Rationale &   Lessens impacts of e-Ar recombination, e-lifetime, e-diffusion and space charge.  \\ \colhline
    Validation & ProtoDUNE  \\
   \colhline

    \\ \rowcolor{dunesky} \newtag{SP-FD-2}{ spec:system-noise } & Name: System noise \\
    Description & The total system noise seen by each wire should be no more than 1000 enc of noise. It is expected that random noise on the FE amplifier will be the dominant contribution to the total system noise.   \\  \colhline
    Specification &  $<\,\SI{1000}{enc}$ \\   \colhline
    Rationale &   Provides $>$5:1 S/N on induction planes for  pattern recognition and two-track separation.  \\ \colhline
    Validation & ProtoDUNE and simulation  \\
   \colhline

    \\ \rowcolor{dunesky} \newtag{SP-FD-3}{ spec:light-yield } & Name: Light yield \\
    Description & The light yield shall be sufficient for measuring event time (and total intensity) of events with visible energy above 200 MeV.  Goal is to make possible a 10\% energy measurement for events with a visible energy of 10 MeV.   \\  \colhline
    Specification (Goal) &  $>\,\SI{0.5}{pe/MeV}$  ( $>\,\SI{5}{pe/MeV}$ ) \\   \colhline
    Rationale &   Rejects nucleon decay backgrounds from cosmogenic events near cathode.  \\ \colhline
    Validation &   \\
   \colhline

    \\ \rowcolor{dunesky} \newtag{SP-FD-4}{ spec:time-resolution-pds } & Name: Time resolution \\
    Description & The time resolution of the photon detection system shall be less than 1 microsecond in order to assign a unique event time.   \\  \colhline
    Specification (Goal) &  $<\,\SI{1}{\micro\second}$  ( $<\,\SI{100}{\nano\second}$ ) \\   \colhline
    Rationale &   Enables \SI{1}{mm} position resolution for \SI{10}{MeV} SNB candidate events for instantaneous rate $<\,\SI{1}{m^{-3}ms^{-1}}$.  \\ \colhline
    Validation &   \\
   \colhline

   \newtag{SP-FD-5}{ spec:lar-purity }  & Liquid argon purity  &  $<$\,\SI{100}{ppt} \newline ($<\,\SI{30}{ppt}$) &  Provides $>$5:1 S/N on induction planes for  pattern recognition and two-track separation. &  Purity monitors and cosmic ray tracks \\ \colhline
    
    \\ \rowcolor{dunesky} \newtag{SP-FD-11}{ spec:hvs-field-uniformity } & Name: Drift field uniformity due to HVS \\
    Description & Design of TPC cathode and FC components shall ensure uniform field.  Production tolerances shall be set so as to maintain flatness of component surfaces and, by extension, the shape of the drift field volume.   \\  \colhline
    Specification &  $<\,\SI{1}{\%}$ throughout volume \\   \colhline
    Rationale &   High reconstruction efficiency.  \\ \colhline
    Validation & ProtoDUNE and simulation  \\
   \colhline

   
  \newtag{SP-FD-12}{ spec:hv-ps-ripple }  & Cathode HV power supply ripple contribution to system noise  &  $<\,\SI{100}e^-$ &  Maximize live time; maintain high S/N. &  Engineering calculation, in situ measurement,   ProtoDUNE \\ \colhline
    
    \\ \rowcolor{dunesky} \newtag{SP-FD-17}{ spec:cathode-resistivity } & Name: Cathode resistivity \\
    Description & The cathode resistivity shall ensure that in the event of an HV discharge, the release of the large stored energy is spread out over time.    \\  \colhline
    Specification (Goal) &  $>\,\SI{1}{\mega\ohm/square}$  ( $>\,\SI{1}{\giga\ohm/square}$ ) \\   \colhline
    Rationale &   Detector damage prevention.  \\ \colhline
    Validation & ProtoDUNE  \\
   \colhline

   
  \newtag{SP-FD-24}{ spec:local-e-fields }  & Local electric fields  &  $<\,\SI{30}{kV/cm}$ &  Maximize live time; maintain high S/N. &  ProtoDUNE \\ \colhline
    

   \newtag{DP-HV-1}{ spec:hvdb-redundancy }  & Provide redundancy in HV distribution  &  $>\,\num{2}$ HVDB chain \newline (\num{12} HVDB chains) &  Ensure the HV connections to the detector &  ProtoDUNE and calculations \\ \colhline
    


\label{tab:specs:DP-HV}
\end{longtable}
% This file is generated, any edits may be lost.
\begin{footnotesize}
%\begin{longtable}{p{0.14\textwidth}p{0.13\textwidth}p{0.18\textwidth}p{0.22\textwidth}p{0.20\textwidth}}
\begin{longtable}{p{0.12\textwidth}p{0.18\textwidth}p{0.17\textwidth}p{0.25\textwidth}p{0.16\textwidth}}
\caption{DP PDS specifications \fixmehl{ref \texttt{tab:spec:DP-PDS}}} \\
  \rowcolor{dunesky}
       Label & Description  & Specification \newline (Goal) & Rationale & Validation \\  \colhline

   \newtag{DP-FD-3}{ spec:dp-light-yield }  & Light yield  &  $>\,\SI{1}{PE/MeV}$ (min), $>\,\SI{3}{PE/MeV}$ (avg) \newline ($>\,\SI{5}{PE/MeV}) &  Enable  fiducialization of NDK candidates anywhere in active volume with $>90\%$ flash reconstruction efficiency and $>90\%$ purity. Enable SNB triggering efficiency of $>90\%$ in galaxy with a fake trigger rate $<$1/month. &  Supernova and nucleon decay events in the FD with full simulation and reconstruction. \\ \colhline
    
   \newtag{DP-FD-4}{ spec:dp-time-resolution-pds }  & Time resolution  &  $<\,\SI{1}{\micro\second}$ \newline ($<\,\SI{100}{\nano\second}$) &  Enables \SI{1}{mm} position resolution for \SI{10}{MeV} SNB candidate events for instantaneous rate $<\,\SI{1}{m^{-3}ms^{-1}}$. &   \\ \colhline
    
   
  \newtag{DP-FD-15}{ spec:dp-lar-n-contamination }  & LAr nitrogen contamination  &  $<\,\SI{25}{ppm}$ &  Maintain \SI{0.5}{PE/MeV} PDS sensitivity required for triggering proton decay near cathode. &   \\ \colhline
    

   
  \newtag{DP-PDS-1}{ spec:dp-light-yield }  & Light yield  &  $>\,\SI{1}{PE/MeV}$ at anode, $>\,\SI{5}{PE/MeV}$ avg over  active vol &  Enable drift position determination of \dword{ndk} candidates. Enable \dword{pds}-based triggering on galactic \dwords{snb}. &  Full sim/reco of \dword{ndk}, \dword{snb} $\nu$ and radiological events. \\ \colhline
    
   \newtag{DP-PDS-2}{ spec:dp-time-resolution }  & Time resolution  &  $<\,\SI{1}{\micro\second}$ \newline ( $<\,\SI{100}{\nano\second}$ ) &  Enables \SI{1}{mm} pos resolution for \SI{10}{MeV} \dwords{snb} candidate events for instantaneous rate $<\,\SI{1}{m^{-3}ms^{-1}}$. &   \\ \colhline
    
   
  \newtag{DP-PDS-3}{ spec:dp-lar-nitrogen-contamination }  & LAr nitrogen contamination  &  $<\,\SI{3}{ppm}$ &  Higher contaminations signifcantly affect the no. of photons that reach the \dword{pmt}. &   \\ \colhline
    
   
  \newtag{DP-PDS-4}{ spec:hit-relative-timing }  & Relative timing accuracy among hits  &  $<\,\SI{100}{ns RMS}$ &  Enable effective clustering of \dword{pmt} signals based on relative hit timing information. &  Full sim/reco of \dword{ndk}, \dword{snb} $\nu$ and radiological events. \\ \colhline
    


\label{tab:specs:DP-PDS}
\end{longtable}
\end{footnotesize}
% This file is generated, any edits may be lost.

\begin{longtable}{p{0.14\textwidth}p{0.13\textwidth}p{0.18\textwidth}p{0.22\textwidth}p{0.20\textwidth}}
\caption{Specifications for DP-CISC \fixmehl{ref \texttt{tab:spec:DP-CISC}}} \\
  \rowcolor{dunesky}
       Label & Description  & Specification \newline (Goal) & Rationale & Validation \\  \colhline

    \\ \rowcolor{dunesky} \newtag{SP-FD-1}{ spec:min-drift-field } & Name: Minimum drift field \\
    Description & The drift field in the TPC shall be greater than 250 V/cm, with a goal of 500 V/cm.   \\  \colhline
    Specification (Goal) &  $>$\,\SI{250}{ V/cm}  ( $>\,\SI{500}{ V/cm}$ ) \\   \colhline
    Rationale &   Lessens impacts of e-Ar recombination, e-lifetime, e-diffusion and space charge.  \\ \colhline
    Validation & ProtoDUNE  \\
   \colhline

    \\ \rowcolor{dunesky} \newtag{SP-FD-2}{ spec:system-noise } & Name: System noise \\
    Description & The total system noise seen by each wire should be no more than 1000 enc of noise. It is expected that random noise on the FE amplifier will be the dominant contribution to the total system noise.   \\  \colhline
    Specification &  $<\,\SI{1000}{enc}$ \\   \colhline
    Rationale &   Provides $>$5:1 S/N on induction planes for  pattern recognition and two-track separation.  \\ \colhline
    Validation & ProtoDUNE and simulation  \\
   \colhline

    \\ \rowcolor{dunesky} \newtag{SP-FD-3}{ spec:light-yield } & Name: Light yield \\
    Description & The light yield shall be sufficient for measuring event time (and total intensity) of events with visible energy above 200 MeV.  Goal is to make possible a 10\% energy measurement for events with a visible energy of 10 MeV.   \\  \colhline
    Specification (Goal) &  $>\,\SI{0.5}{pe/MeV}$  ( $>\,\SI{5}{pe/MeV}$ ) \\   \colhline
    Rationale &   Rejects nucleon decay backgrounds from cosmogenic events near cathode.  \\ \colhline
    Validation &   \\
   \colhline

    \\ \rowcolor{dunesky} \newtag{SP-FD-4}{ spec:time-resolution-pds } & Name: Time resolution \\
    Description & The time resolution of the photon detection system shall be less than 1 microsecond in order to assign a unique event time.   \\  \colhline
    Specification (Goal) &  $<\,\SI{1}{\micro\second}$  ( $<\,\SI{100}{\nano\second}$ ) \\   \colhline
    Rationale &   Enables \SI{1}{mm} position resolution for \SI{10}{MeV} SNB candidate events for instantaneous rate $<\,\SI{1}{m^{-3}ms^{-1}}$.  \\ \colhline
    Validation &   \\
   \colhline

   \newtag{SP-FD-5}{ spec:lar-purity }  & Liquid argon purity  &  $<$\,\SI{100}{ppt} \newline ($<\,\SI{30}{ppt}$) &  Provides $>$5:1 S/N on induction planes for  pattern recognition and two-track separation. &  Purity monitors and cosmic ray tracks \\ \colhline
    
    \\ \rowcolor{dunesky} \newtag{SP-FD-15}{ spec:lar-n-contamination } & Name: LAr nitrogen contamination \\
    Description & The nitrogen contamination in the LAr shall remain below 25 ppm in order not to significantly affect the number of photons that reach the detectors (for both fast and late light components).   \\  \colhline
    Specification &  $<\,\SI{25}{ppm}$ \\   \colhline
    Rationale &   Maintain \SI{0.5}{PE/MeV} PDS sensitivity required for triggering proton decay near cathode.  \\ \colhline
    Validation & In situ measurment  \\
   \colhline

   
  \newtag{SP-FD-18}{ spec:cryo-monitor-devices }  & Cryogenic monitoring devices  &   &  Constrain uncertainties on detection efficiency, fiducial volume. &  ProtoDUNE \\ \colhline
    
   
  \newtag{SP-FD-25}{ spec:non-fe-noise }  & Non-FE noise contributions  &  $<<\,\SI{1000}\,e^- $ &  High S/N for high reconstruction efficiency. &  Engineering calculation and ProtoDUNE \\ \colhline
    

\input{generated/req-DP-CISC-01.tex}
   \newtag{DP-CISC-2}{ spec:inst-efield }  & Max. E field near instrumentation devices  &  $<\,\SI{30}{kV/cm}$ \newline ($<\,\SI{15}{kV/cm}$) &  Significantly lower than max field of 30 kV/cm per DUNE HV  &  3D electrostatic simulation \\ \colhline
    
   \newtag{DP-CISC-3}{ spec:elec-lifetime-prec }  & Precision in electron lifetime  &  $<\,$1.4\% \newline ($<\,$1\%) &  Required to accurately reconstruct charge per DUNE-FD Task Force report. &  ProtoDUNE and CITF \\ \colhline
    
   \newtag{DP-CISC-4}{ spec:elec-lifetime-range }  & Range in electron lifetime  &  \SIrange{0}{10}{ms} (\SIrange{0}{30}{ms}) \newline ( \SIrange{0}{10}{ms} (\SIrange{0}{30}{ms}) ) &  Slightly more than best values so far observed in other detectors. &  ProtoDUNE and CITF \\ \colhline
    
   \newtag{DP-CISC-11}{ spec:temp-repro }  & Precision: temperature reproducibility  &  $<\,\SI{5}{mK}$ \newline (\SI{2}{mK}) &  Allows validating CFD models that predict gradients less than 15 mK. &  ProtoDUNE and CITF \\ \colhline
    
   \newtag{DP-CISC-14}{ spec:temp-stability }  & Temperature stability  &  $<\,\SI{2}{mK}$ at all places and times \newline (Match precision requirement at all places, at all times) &  Measures temperature map with sufficient precision for the duration of thermometer operations. &  ProtoDUNE \\ \colhline
    
   \newtag{DP-CISC-27}{ spec:camera-cold-coverage }  & Cold camera coverage  &  $>\,$80\% of HV surfaces \newline (\num{100}\%) &  Enables detailed inspection of issues near HV surfaces. &  Calculated from location, validated in prototypes. \\ \colhline
    
\input{generated/req-DP-CISC-51.tex}
\input{generated/req-DP-CISC-52.tex}
   \newtag{DP-CISC-54}{ spec:slowcontrol-archive-rate }  & Archiving rate  &  \SI{0.02}{Hz} \newline (Broad range \SI{1}{Hz} to \num{1} per few min.) &  Archiving rate differs by variable, optimized to store important information &  ProtoDUNE \\ \colhline
    


\label{tab:specs:DP-CISC}
\end{longtable}


%%%%%%%%%%%%%%%%%%%%%%%%%%%%%%%%
\section{Risks}
\label{sec:fdsp-coord-risks}

\dword{dune} initiated a risk registry in 2018 (available
in~\citedocdb{6443}). This document includes consortium risks and
\dword{tc} risks. It includes a summary of the most significant
overall \dword{dune} risks.  This registry has been updated for the
\dword{tdr} and the full listing can be found in
Appendix~\ref{sec:fdsp-app-risk}.We  expect to update it
approximately yearly. The previous update occurred in early 2018
before \dword{protodune} was completed and the most recent update is for the
\dword{tdr}. Another update is planned for 2020. %Several risks associated with \dword{protodune} have been retired.  - repeated below
\dword{lbnf} and
\dword{dune}-US would like \dword{dune} to update and expand this risk
register to allow a \dword{mc} analysis of cost and schedule risks to
the \dword{us} project resulting from international \dword{dune}
risks. This request is under consideration as it may be useful for
other national projects as well.  Successfully operating
\dword{protodune} retired many \dword{dune} risks in
\dword{dune}. This includes most risks associated with the technical
design, production processes, \dword{qa}, integration, and
installation. Residual risks remain relating to design and production
modifications associated with scaling to \dword{dune}, mitigations to
known installation and performance issues in \dword{protodune},
underground installation at \dword{surf}, and organizational growth.

The highest technical risks include development of a system to
deliver \SI{600}{kV} to the \dword{dp} cathode; general delivery of the
required \dword{hv}; cathode and \dword{fc} discharge to the cryostat
membrane; noise levels, particularly for the \dword{ce}; 
number of dead channels; lifetime of components surpassing \dunelifetime{}; 
\dword{qc} of all components; verification of improved \dword{lem}
performance; verification of new cold  \dword{adc} and  \dword{coldata} performance;
argon purity; electron drift lifetime; \phel light yield;
incomplete calibration plan; and incomplete connection of design to
physics. Other significant risks include insufficient funding, optimistic
production schedules, incomplete plans for integration, testing and installation. 

One update to the risk registry since 2018 has been for \dword{tc}, to add %in which some
some risks  associated with  DUNE  integration and  installation  %have  been added 
(see \spchinstall{}, Table~9.2)

In addition to installation-related risks, \dword{tc} is developing its
own set of overall project risks not captured by conortia.  Key risks
for \dword{tc} to manage include the following:
\begin{enumerate}
\item Consortia leave too much scope unaccounted for and too much falls
  to  the \dword{comfund}.
\item Insufficient organizational systems are put into place to
  ensure that this complex, international mega-science project,
  including \dword{tc}, \dword{fnal} as host laboratory, \dword{surf}, \dword{doe}, and all international
  partners continue to work together successfully to ensure that 
  appropriate processes and services are provided for the success of
  the project.
\item Inability of \dword{tc} to obtain sufficient personnel resources to
  ensure that \dword{tc} can oversee and coordinate all project tasks.  While the \dword{us}, 
  as host country, has a special responsibility to \dword{tc}, personnel resources should
  be directed to \dword{tc} from each collaborating country. 
\end{enumerate}

The consortia have provided preliminary versions of risk analyses that
have been collected on the \dword{tc} webpage (\citedocdb{6443}). These have
been developed into an overall risk register that will be monitored
and maintained by \dword{tc} in coordination with the consortia. This
full set of risks can be found in 
Section~\ref{sec:fdsp-app-risk}.


%%%%%%%%%%%%%%%%%%%%%%%%%%%%%%%%
\section{Full DUNE Risks}
\label{sec:fdsp-app-risk}

%%%%%%%%%%%%%%%%%%%%%%%%%%%%%%%
\subsection{Single-phase}
\label{sec:tc-risks-sp}


% risk table values for subsystem SP-FD-APA
\begin{footnotesize}
%\begin{longtable}{p{0.18\textwidth}p{0.20\textwidth}p{0.32\textwidth}p{0.02\textwidth}p{0.02\textwidth}p{0.02\textwidth}}
\begin{longtable}{P{0.18\textwidth}P{0.20\textwidth}P{0.32\textwidth}P{0.02\textwidth}P{0.02\textwidth}P{0.02\textwidth}} 
\caption[APA risks]{APA risks (P=probability, C=cost, S=schedule) The risk probability, after taking into account the planned mitigation activities, is ranked as 
L (low $<\,$\SI{10}{\%}), 
M (medium \SIrange{10}{25}{\%}), or 
H (high $>\,$\SI{25}{\%}). 
The cost and schedule impacts are ranked as 
L (cost increase $<\,$\SI{5}{\%}, schedule delay $<\,$\num{2} months), 
M (\SIrange{5}{25}{\%} and 2--6 months, respectively) and 
H ($>\,$\SI{20}{\%} and $>\,$2 months, respectively). \fixmehl{ref \texttt{tab:risks:DP-FD-CISC}}} \\
\rowcolor{dunesky}
ID & Risk & Mitigation & P & C & S  \\  \colhline
RT-SP-APA-01 & Loss of key personnel & Implement succession planning and formal project documentation & L & L & M \\  \colhline
RT-SP-APA-02 & Delay in finalisation of APA frame design & Close oversight on prototypes and interface issues & L & L & M \\  \colhline
RT-SP-APA-03 & One additional pre-production APA may be necessary & Close oversight on approval of designs, commissioning of tooling and assembly procedures & L & L & L \\  \colhline
RT-SP-APA-04 & APA winder construction takes longer than planned & Detailed plan to stand up new winding machines at each facility & M & L & M \\  \colhline
RT-SP-APA-05 & Poor quality of APA frames and/or inaccuracy in the machining of holes and slots & Clearly specified requirements and seek out backup vendors & L & L & M \\  \colhline
RT-SP-APA-06 & Insufficient scientific manpower at APA assembly factories & Get institutional commitments for requests of necessary personnel in research grants & M & M & L \\  \colhline
RT-SP-APA-07 & APA production quality does not meet requirements & Close oversight on assembly procedures & L & M & M \\  \colhline
RT-SP-APA-08 & Materials shortage at factory & Develop and execute a supply chain management plan & M & L & L \\  \colhline
RT-SP-APA-09 & Failure of a winding machine - Drive chain parts failure & Regular maintenance and availability of spare parts & L & L & L \\  \colhline
RT-SP-APA-10 & APA assembly takes longer time than planned  & Estimates based on protoDUNE. Formal training of every tech/operator & L & M & M \\  \colhline
RT-SP-APA-11 & Loss of one APA due to an accident & Define handling procedures supported by engineering notes & M & L & L \\  \colhline
RT-SP-APA-12 & APA transport box inadequate & Construction and test of prototype transport boxes & L & L & M \\  \colhline
RO-SP-APA-01 & Reduction of the APA assembly time & Improvements in the winding head and wire tension mesurements & M & M & M \\  
 &  &  &  &  &  \\  \colhline

\label{tab:risks:SP-FD-APA}
\end{longtable}
\end{footnotesize}


% risk table values for subsystem SP-FD-HV
\begin{footnotesize}
%\begin{longtable}{p{0.18\textwidth}p{0.20\textwidth}p{0.32\textwidth}p{0.02\textwidth}p{0.02\textwidth}p{0.02\textwidth}}
\begin{longtable}{P{0.18\textwidth}P{0.20\textwidth}P{0.32\textwidth}P{0.02\textwidth}P{0.02\textwidth}P{0.02\textwidth}} 
\caption[HV risks]{HV risks (P=probability, C=cost, S=schedule) The risk probability, after taking into account the planned mitigation activities, is ranked as 
L (low $<\,$\SI{10}{\%}), 
M (medium \SIrange{10}{25}{\%}), or 
H (high $>\,$\SI{25}{\%}). 
The cost and schedule impacts are ranked as 
L (cost increase $<\,$\SI{5}{\%}, schedule delay $<\,$\num{2} months), 
M (\SIrange{5}{25}{\%} and 2--6 months, respectively) and 
H ($>\,$\SI{20}{\%} and $>\,$2 months, respectively).  \fixmehl{ref \texttt{tab:risks:SP-FD-HV}}} \\
\rowcolor{dunesky}
ID & Risk & Mitigation & P & C & S  \\  \colhline
RT-SP-HV-01 & Open circuit on the field cage divider chain & Component selection and cold tests. Varistor protection. & L & L & L \\  \colhline
RT-SP-HV-02 & Damage to the resistive Kapton film on CPA & Careful visual inspection of panel surfaces.  Replace panel if scratches are deep and long  & L & L & L \\  \colhline
RT-SP-HV-03 & Sole source for Kapton resistive surface; and may go out of production & Another potential source of resistive Kapton identified. Possible early purchase if single source. & M & L & L \\  \colhline
RT-SP-HV-04 & Detector components are damaged during shipment to the far site  & Spare parts at  LW. FC/CPA modules can be swapped and replaced from factories in a few days. & L & L & L \\  \colhline
RT-SP-HV-05 & Damages (scratches, bending) to aluminum profiles of Field Cage modules & Require sufficent spare profiles for substitution. Alternate: local coating with epoxy resin. & L & L & L \\  \colhline
RT-SP-HV-06 & Electric field uniformity is not adequate for muon momentum reconstruction  & Redundant components; rigorous screening. Structure based on CFD. Calibration can map E-field. & L & L & L \\  \colhline
RT-SP-HV-07 & Electric field is below goal during stable operations & Improve the protoDUNE SP HVS design to reduce surface E-field and eliminate exterior insulators. & M & L & L \\  \colhline
RT-SP-HV-08 & Damage to CE in event of discharge  & HVS was designed to reduce discharge to a safe level. Higher resistivity cathode could optimize. & L & L & L \\  \colhline
RT-SP-HV-09 & Free hanging frames can swing in the fluid flow  & Designed for flow using fluid model; Deformation can be calibrated by lasers or cosmic rays. & L & L & L \\  \colhline
RT-SP-HV-10 & FRP/ Polyethene/ laminated Kapton component lifetime is less than expected & Positive experience in other detectors. Gain experience with LAr TPC's; exchangeable feedthrough. & L & L & L \\  \colhline
RT-SP-HV-11 & International funding level for SP HVS too low & Cost reduction through design optimization. Effort to increase international collaboration. & M & M & M \\  \colhline
RT-SP-HV-12 & Underground installation is more labor intensive or slower than expected & SWF contingency, full-scale trial before installation. Estimates based on ProtoDUNE experience. & L & L & L \\  \colhline

\label{tab:risks:SP-FD-HV}
\end{longtable}
\end{footnotesize}


% risk table values for subsystem SP-FD-TPCELEC
\begin{footnotesize}
%\begin{longtable}{p{0.18\textwidth}p{0.20\textwidth}p{0.32\textwidth}p{0.02\textwidth}p{0.02\textwidth}p{0.02\textwidth}}
\begin{longtable}{P{0.18\textwidth}P{0.20\textwidth}P{0.32\textwidth}P{0.02\textwidth}P{0.02\textwidth}P{0.02\textwidth}} 
\caption[TPC electronics risks]{TPC electronics risks (P=probability, C=cost, S=schedule) The risk probability, after taking into account the planned mitigation activities, is ranked as 
L (low $<\,$\SI{10}{\%}), 
M (medium \SIrange{10}{25}{\%}), or 
H (high $>\,$\SI{25}{\%}). 
The cost and schedule impacts are ranked as 
L (cost increase $<\,$\SI{5}{\%}, schedule delay $<\,$\num{2} months), 
M (\SIrange{5}{25}{\%} and 2--6 months, respectively) and 
H ($>\,$\SI{20}{\%} and $>\,$2 months, respectively). \fixmehl{ref \texttt{tab:risks:SP-FD-TPCELEC}}} \\
\rowcolor{dunesky}
ID & Risk & Mitigation & P & C & S  \\  \colhline
RT-SP-TPC-001 & Cold ASIC(s) not meeting specifications & Multiple designs, use of appropriate design rules for operation in LAr & H  & M & L \\  \colhline
RT-SP-TPC-002 & Delay in the availability of ASICs and FEMBs & Increase pool of spares for long lead items, multiple QC sites for ASICs, appropriate measures against ESD, monitoring of yields & M & L & L \\  \colhline
RT-SP-TPC-003 & Damage to the FEMBs / cold cables during or after integration with the APAs & Redesign of the FEMB/cable connection, use of CE boxes, ESD protections, early integration tests & M & L & L \\  \colhline
RT-SP-TPC-004 & Cold cables cannot be run through the APAs frames & Redesign of APA frames, integration tests at Ash River and at CERN, further reduction of cable plant & L & L & L \\  \colhline
RT-SP-TPC-005 & Delay and/or damage to the TPC electronics components on the top of the cryostat & Sufficient spares, early production and installation, ESD protection measures & L & L & L \\  \colhline
RT-SP-TPC-006 & Interfaces between TPC electronics and other consortia not adequately defined & Early integration tests, second run of ProtoDUNE-SP with pre-production components & M & L & L \\  \colhline
RT-SP-TPC-007 & Insufficient number of spares & Early start of production, close monitoring of usage of components, larger stocks of components with long lead times & M & L & L \\  \colhline
RT-SP-TPC-008 & Loss of key personnel & Distributed development of ASICs, increase involved of university groups, training of younger personnel & H & L & M \\  \colhline
RT-SP-TPC-009 & Excessive noise observed during detector commissioning & Enforce grounding rules, early integration tests, second run of ProtoDUNE-SP with pre-production components, cold box testing at SURF & L & L & M \\  \colhline
RT-SP-TPC-010 & Lifetime of components in the LAr & Design rules for cryogenic operation of ASICs, measurement of lifetime of components, reliability studies & L & n/a & n/a \\  \colhline
RT-SP-TPC-011 & Lifetime of components on the top of the cryostat & Use of filters on power supplies, stockpiling of components that may become obsolete, design rules to minimize parts that need to be redesigned / refabricated & L & M & L \\  \colhline

\label{tab:risks:SP-FD-TPCELEC}
\end{longtable}
\end{footnotesize}
 % this is tpc elec

% risk table values for subsystem SP-FD-PD
\begin{footnotesize}
%\begin{longtable}{p{0.18\textwidth}p{0.20\textwidth}p{0.32\textwidth}p{0.02\textwidth}p{0.02\textwidth}p{0.02\textwidth}}
\begin{longtable}{P{0.18\textwidth}P{0.20\textwidth}P{0.32\textwidth}P{0.02\textwidth}P{0.02\textwidth}P{0.02\textwidth}} 
\caption[PD system risks]{PD system risks (P=probability, C=cost, S=schedule) The risk probability, after taking into account the planned mitigation activities, is ranked as 
L (low $<\,$\SI{10}{\%}), 
M (medium \SIrange{10}{25}{\%}), or 
H (high $>\,$\SI{25}{\%}). 
The cost and schedule impacts are ranked as 
L (cost increase $<\,$\SI{5}{\%}, schedule delay $<\,$\num{2} months), 
M (\SIrange{5}{25}{\%} and 2--6 months, respectively) and 
H ($>\,$\SI{20}{\%} and $>\,$2 months, respectively). \fixmehl{ref \texttt{tab:risks:SP-FD-PD}}} \\
\rowcolor{dunesky}
ID & Risk & Mitigation & P & C & S  \\  \colhline
RT-SP-PD -01 & Additional photosensors and engineering required to ensure PD modules collect enough light to meet system physics performance specifications. & Extensive validation of \dshort{xarapu} design to demonstrate they meet specification. & L & M & L \\  \colhline
RT-SP-PD-02 & Improvements to active ganging/front end electronics required to meet the specified 1~$\mu$s time resolution. & Extensive validation of photosensor ganging/front end electronics design to demonstrate they meet specification. & L & L & L \\  \colhline
RT-SP-PD-03 & Evolutions in the design of the photon detectors due to validation testing experience require modifications of the TPC elements at a late time. & Extensive validation of \dshort{xarapu} design to demonstrate they meet specification and control of PD/APA interface. & L & L & L \\  \colhline
RT-SP-PD-04 & Cabling for PD and CE within the \dshort{apa} frame or during the 2-APA assembly/installation procedure require additional engineering/development/testing. & Validation of PD/APA/CE cable routing in prototypes at Ash River. & L & L & L \\  \colhline
RT-SP-PD-05 & Experience with validation prototypes shows that the mechanical design of the PD is not adequate to meet system specifications. & Early validation of \dshort{xarapu} prototypes and system interfaces to catch problems ASAP. & L & L & L \\  \colhline
RT-SP-PD-06 & pTB WLS filter coating not sufficiently stable, contaminates \dshort{lar}. & Mechanical acceleration of coating wear.  Long-term tests of coating stability. & L & L & L \\  \colhline
RT-SP-PD-07 & Photosensors fail due to multiple cold cycles or extended cryogen exposure. & Execute testing program for cryogenic operation of photosensors including mutiple cryogenic immersion cycles. & L & L & L \\  \colhline
RT-SP-PD-08 & SiPM active ganging cold amplifiers fail or degrade detector performance. & Validation testing if photosensor ganging in multiple test beds. & L & L & L \\  \colhline
RT-SP-PD-09 & Previously undetected electro-mechanical interference discovered during integration. & Validation of electromechanical designin Ash River tests and at \dshort{pdsp2}. & L & L & L \\  \colhline
RT-SP-PD-10 & Design weaknesses manifest during module logistics-handling. & Validation of shipping packaging and handling prior to shipping.  Inspection of modules shipped to site immediately upon receipt. & L & L & L \\  \colhline
RT-SP-PD-11 & PD/CE signal crosstalk. & Validation in \dshort{pdsp}, \dshort{iceberg} and \dshort{pdsp2}. & L & L & L \\  \colhline
RT-SP-PD-12 & Lifetime of \dshort{pd} components outside cryostat. & Specification of environmental controls to mitigate detector aging. & L & L & L \\  \colhline

\label{tab:risks:SP-FD-PD}
\end{longtable}
\end{footnotesize}


% risk table values for subsystem SP-FD-CAL
\begin{longtable}{p{0.18\textwidth}p{0.20\textwidth}p{0.32\textwidth}p{0.02\textwidth}p{0.02\textwidth}p{0.02\textwidth}} 
\caption{Risks for SP-FD-CAL \fixmehl{ref \texttt{tab:risks:SP-FD-CAL}}} \\
\rowcolor{dunesky}
ID & Risk & Mitigation & P & C & S  \\  \colhline
RT-SP-CAL-01 & Inadequate baseline design & Early detection allows R\&D of alternative designs accommodated through multipurpose ports & L & M & M \\  \colhline
RT-SP-CAL-02 & Inadequate engineering or production quality & Dedicated small scale tests and full prototyping at ProtoDUNE; pre-installation QC & L & M & M \\  \colhline
RT-SP-CAL-03 & Laser impact on PDS & Mirror movement control to avoid direct hits; turn laser off in case of PDS saturation & L & L & L \\  \colhline
RT-SP-CAL-04 & Laser positioning system stops working & QC at installation time, redundancy in available targets, including passive, alternative methods & L & L & L \\  \colhline
RT-SP-CAL-05 & Laser beam misaligned & Additional (visible) laser for alignment purposes & M & L & L \\  \colhline
RT-SP-CAL-06 & The neutron anti-resonance is much less pronounced & Dedicated measurements at LANL and test at ProtoDUNE & L & L & L \\  \colhline
RT-SP-CAL-07 & Neutron activation of the moderator and cryostat & Neutron activation studies and simulations & L & L & L \\  \colhline
RT-SP-CAL-08 & Neutron yield not high enough & Simulations and tests at ProtoDUNE & L & M & M \\  \colhline
RT-SP-CAL-09 & Neutrons do not reach detector center & Alternative, movable design and simulations & L & L & L \\  \colhline

\label{tab:risks:SP-FD-CAL}
\end{longtable}

% risk table values for subsystem SP-FD-DAQ
\begin{longtable}{p{0.18\textwidth}p{0.20\textwidth}p{0.32\textwidth}p{0.02\textwidth}p{0.02\textwidth}p{0.02\textwidth}} 
\caption{Risks for SP-FD-DAQ \fixmehl{ref \texttt{tab:risks:SP-FD-DAQ}}} \\
\rowcolor{dunesky}
ID & Risk & Mitigation & P & C & S  \\  \colhline
RT-SP-DAQ-01 & Detector noise specs not met & ProtoDUNE experience with noise levels and provisions for data processing redundancy in DAQ system (high level filter) & M & M & M \\  \colhline
RT-SP-DAQ-02 & Externally-driven schedule change & Provisions for standalone testing and commissioning of production DAQ components, and schedule adjustment & H & L & M \\  \colhline
RT-SP-DAQ-03 & Lack of expert personnel & Resource-loaded plan for DAQ backed by institutional commitments, and schedule adjustment using float & L & L & H \\  \colhline
RT-SP-DAQ-04 & Power/space requirements exceed CUC capacity & Sufficient bandwidth to surface and move module 3/4 components to an expanded surface facility & L & H & L \\  \colhline
RT-SP-DAQ-05 & Excess fake trigger rate from instrumental effects & ProtoDUNE performance experience, and provisions for increase in event builder and high level filter capacity, as needed & M & M & M \\  \colhline
RT-SP-DAQ-06 & Calibration requirements exceed acceptable data rate & Provisions for increase in event builder and high level filter capacity, as neeed & M & M & M \\  \colhline
RT-SP-DAQ-07 & Cost/performance of hardware/computing excessive & Have prototyping and pre-production phases, reduce performance using margin or identify additional funds & L & H & L \\  \colhline
RT-SP-DAQ-08 & PDTS fails to scale for DUNE requirements & Hardware upgrade & L & L & M \\  \colhline
RT-SP-DAQ-09 & WAN network & Extensive QA and exercise of failure mode recovery in extended ProtoDUNE runs, and replacement of components & L & M & M \\  \colhline
RT-SP-DAQ-10 & Infrastructure & Performance specifications for external services, prior to construction. & M & H & H \\  \colhline

\label{tab:risks:SP-FD-DAQ}
\end{longtable} 

% risk table values for subsystem SP-FD-CISC
\begin{footnotesize}
%\begin{longtable}{p{0.18\textwidth}p{0.20\textwidth}p{0.32\textwidth}p{0.02\textwidth}p{0.02\textwidth}p{0.02\textwidth}}
\begin{longtable}{P{0.18\textwidth}P{0.20\textwidth}P{0.32\textwidth}P{0.02\textwidth}P{0.02\textwidth}P{0.02\textwidth}} 
\caption[CISC risks]{CISC risks (P=probability, C=cost, S=schedule) The risk probability, after taking into account the planned mitigation activities, is ranked as 
L (low $<\,$\SI{10}{\%}), 
M (medium \SIrange{10}{25}{\%}), or 
H (high $>\,$\SI{25}{\%}). 
The cost and schedule impacts are ranked as 
L (cost increase $<\,$\SI{5}{\%}, schedule delay $<\,$\num{2} months), 
M (\SIrange{5}{25}{\%} and 2--6 months, respectively) and 
H ($>\,$\SI{20}{\%} and $>\,$2 months, respectively).  \fixmehl{ref \texttt{tab:risks:SP-FD-CISC}}} \\
\rowcolor{dunesky}
ID & Risk & Mitigation & P & C & S  \\  \colhline
RT-SP-CISC-01 & Baseline design from ProtoDUNEs for an instrumentation device is not adequate for DUNE far detectors & Focus on early problem discovery in ProtoDUNE so any needed redesigns can start as soon as possible. & L & M & L \\  \colhline
RT-SP-CISC-02 & Swinging of long instrumentation devices (T-gradient monitors or PrM system) & Add additional intermediate constraints to prevent swinging. & L & L & L \\  \colhline
RT-SP-CISC-03 & High E-fields near instrumentation devices cause dielectric breakdowns in \dshort{lar} & CISC systems placed as far from cathode and FC as possible. & L & L & L \\  \colhline
RT-SP-CISC-04 & Light pollution from purity monitors and camera light emitting system & Use PrM lamp and camera lights outside PDS trigger window; cover PrM cathode to reduce light leakage. & L & L & L \\  \colhline
RT-SP-CISC-05 & Temperature sensors can induce noise in cold electronics & Check for noise before filling and remediate, repeat after filling. Filter or ground noisy sensors. & L & L  & L \\  \colhline
RT-SP-CISC-06 & Disagreement between lab and \em{in situ} calibrations for ProtoDUNE-SP dynamic T-gradient monitor & Investigate and improve both methods, particularly laboratory calibration. & M & L & L \\  \colhline
RT-SP-CISC-07 & Purity monitor electronics induce noise in TPC and PDS electronics. & Operate lamp outside TPC+PDS trigger window. Surround and ground light source with Faraday cage. & L & L & L \\  \colhline
RT-SP-CISC-08 & Discrepancies between measured temperature map and CFD simulations in ProtoDUNE-SP & Improve simulations with additional measurements inputs; use fraction of sensors to predict others   & L & L & L \\  \colhline
RT-SP-CISC-09 & Difficulty correlating purity and temperature in ProtoDUNE-SP impairs understanding cryo system. & Identify causes of discrepancy, modify design. Calibrate PrM differences, correlate with RTDs. & L & L & L \\  \colhline
RT-SP-CISC-10 & Cold camera R\&D fails to produce prototype meeting specifications \& safety requirements & Improve insulation and heaters. Use cameras in ullage or inspection cameras instead. & M & M & L \\  \colhline
RT-SP-CISC-11 & HV discharge caused by inspection cameras & Study E-field in and on housing and anchoring system. Test in HV facility. & L & L & L \\  \colhline
RT-SP-CISC-12 & HV discharge destroying the cameras & Ensure sufficient redundancy of cold cameras. Warm cameras are replaceable. & L & M & L \\  \colhline
RT-SP-CISC-13 & Insufficient light for cameras to acquire useful images & Test cameras with illumination similar to actual detector. & L & L & L \\  \colhline
RT-SP-CISC-14 & Cameras may induce noise in cold electronics & Continued R\&D work with grounding and shielding in realistic conditions. & L & L & L \\  \colhline
RT-SP-CISC-15 & Light attenuation in long optic fibers for purity monitors  & Test the max.\ length of usable fiber, optimize the depth of bottom PrM, number of fibers. & L & L & L \\  \colhline
RT-SP-CISC-16 & Longevity of purity monitors & Optimize PrM operation to avoid long running in low purity. Technique to protect/recover cathode. & L & L & L \\  \colhline
RT-SP-CISC-17 & Longevity: Gas analyzers and level meters may fail. & Plan for future replacement in case of failure or loss of sensitivity.  & M & M & L \\  \colhline
RT-SP-CISC-18 & Problems in interfacing  hardware devices (e.g. power supplies) with slow controls & Involve slow control experts in choice of hardware needing control/monitoring.
 & L & L & L \\  \colhline

\label{tab:risks:SP-FD-CISC}
\end{longtable}
\end{footnotesize}
 

% risk table values for subsystem SP-FD-INST
\begin{footnotesize}
%\begin{longtable}{p{0.18\textwidth}p{0.20\textwidth}p{0.32\textwidth}p{0.02\textwidth}p{0.02\textwidth}p{0.02\textwidth}}
\begin{longtable}{P{0.18\textwidth}P{0.20\textwidth}P{0.32\textwidth}P{0.02\textwidth}P{0.02\textwidth}P{0.02\textwidth}} 
\caption[Risks for SP-FD-INST]{Risks for SP-FD-INST (P=probability, C=cost, S=schedule) More information at \dword{riskprob}. \fixmehl{ref \texttt{tab:risks:SP-FD-INST}}} \\
\rowcolor{dunesky}
ID & Risk & Mitigation & P & C & S  \\  \colhline
RT-INST-01 & Personnel injury & Follow established safety plans. & M & L & H \\  \colhline
RT-INST-02 & Shipping delays & Plan one month buffer to store  materials locally. Provide logistics manual. & H & L & L \\  \colhline
RT-INST-03 & Missing components cause delays & Use detailed inventory system to verify availability of  necessary components.  & H & L & L \\  \colhline
RT-INST-04 & Import, export, visa issues  & Dedicated \dword{fnal} \dword{sdsd}division will expedite import/export and visa-related issues. & H & M & M \\  \colhline
RT-INST-05 & Lack of available labor  & Hire early and use Ash River setup to train \dword{jpo} crew. & L & L & L \\  \colhline
RT-INST-06 & Parts do not fit together & Generate \threed model, create interface drawings, and prototype detector assembly. & H & L & L \\  \colhline
RT-INST-07 & Cryostat damage & Use cryostat false floor and temporary protection. & L & L & M \\  \colhline
RT-INST-08 & Weather closes SURF & Plan for \dword{surf} weather closures & H & L & L \\  \colhline
RT-INST-09 & Detector failure during \cooldown & Cold test individual components then cold test \dword{apa} assemblies immediately before installation. & L & H & H \\  \colhline

\label{tab:risks:SP-FD-INST}
\end{longtable}
\end{footnotesize} 
%%
% risk table values for subsystem SP-FD-JPO
\begin{longtable}{p{0.15\textwidth}p{0.13\textwidth}p{0.13\textwidth}p{0.28\textwidth}p{0.06\textwidth}p{0.06\textwidth}p{0.06\textwidth}} 
\caption{Specification for SP-FD-JPO \fixmehl{ref \texttt{tab:specs:SP-FD-JPO}}} \\
\rowcolor{dunesky}
ID & Risk & Label & Mitigation & Prob ability & Cost Impact & Sched ule Impact \\  \colhline
RT-JPO-001 & Personnel injury & jpo-person-injury & Follow established safety plans. & M & L & H \\  \colhline
RT-JPO-002 & Shipping delays & jpo-shipping-delay & Plan one month buffer to store  materials locally. Provide logistics manual. & H & L & L \\  \colhline
RT-JPO-003 & Missing components cause delays & jpo-missing-components & Use detailed inventory system to verify availability of  necessary components.  & H & L & L \\  \colhline
RT-JPO-004 & Import, export, visa issues  & jpo-import-visa & Dedicated \dword{fnal} \dword{sdsd}division will expedite import/export and visa-related issues. & H & M & M \\  \colhline
RT-JPO-005 & Lack of available labor  & jpo-labor-avail & Hire early and use Ash River setup to train \dword{jpo} crew. & L & L & L \\  \colhline
RT-JPO-006 & Parts do not fit together & jpo-cannot-assemble & Generate \threed model, create interface drawings, and prototype detector assembly. & H & L & L \\  \colhline
RT-JPO-007 & Cryostat damage & jpo-cryostat-damage & Use cryostat false floor and temporary protection. & L & L & M \\  \colhline
RT-JPO-008 & Weather closes SURF & jpo-weather-delay & Plan for \dword{surf} weather closures & H & L & L \\  \colhline
RT-JPO-009 & Detector failure during \cooldown & jpo-cooldown-failure & Cold test individual components then cold test \dword{apa} assemblies immediately before installation. & L & H & H \\  \colhline

\label{tab:risks:SP-FD-JPO}
\end{longtable} % aka tc or iic

%%%%%%%%%%%%%%%%%%%%%%%%%%%%%%%
\subsection{Dual-phase}
\label{sec:tc-risks-dp}

For each risk, the risk probability, after taking into account the planned mitigation activities, is ranked as 
L (low $<\,$\SI{10}{\%}), 
M (medium \SIrange{10}{25}{\%}), or 
H (high $>\,$\SI{25}{\%}). 
The cost and schedule impacts are ranked as 
L (cost increase $<\,$\SI{5}{\%}, schedule delay $<\,$\num{2} months), 
M (\SIrange{5}{25}{\%} and 2--6 months, respectively) and 
H ($>\,$\SI{20}{\%} and $>\,$2 months, respectively).


% risk table values for subsystem DP-FD-CRP
\begin{footnotesize}
%\begin{longtable}{p{0.18\textwidth}p{0.20\textwidth}p{0.32\textwidth}p{0.02\textwidth}p{0.02\textwidth}p{0.02\textwidth}}
\begin{longtable}{P{0.18\textwidth}P{0.20\textwidth}P{0.32\textwidth}P{0.02\textwidth}P{0.02\textwidth}P{0.02\textwidth}} 
\caption[DP CRP Risks]{Risks for DP-FD-CRP (P=probability, C=cost, S=schedule) More information at \dshort{riskprob}. \fixmehl{ref \texttt{tab:risks:DP-FD-CRP}}} \\
\rowcolor{dunesky}
ID & Risk & Mitigation & P & C & S  \\  \colhline
RT-DP-CRP-01 & Poor quality of G10  frames and/or inaccuracy in the hole machining & Clearly specified requirements, followup and seek out backup vendors & L & L  & M \\  \colhline
RT-DP-CRP-02 & LEM production takes longer than expected & Define a production schedule allowing enough contingencies to limit the assembly impact   & L & L  & M \\  \colhline
RT-DP-CRP-03 & One of the CRP assembly site not ready on time & Close oversight on construction of tooling and preparation of assembly sites & L & M & H  \\  \colhline
RT-DP-CRP-04 & Materials shortage at production site & Develop and execute a supply chain management & M & L  & L \\  \colhline
RT-DP-CRP-05 & Failure of extraction grid winding machine & Regular maintenance and availability of spare parts & L & L  & L \\  \colhline
RT-DP-CRP-06 & CRP assembly takes longer time than planned & Estimates based on ProtoDUNE-DP. Formal training of every tech/operator at each site. & L & M & M \\  \colhline

\label{tab:risks:DP-FD-CRP}
\end{longtable}
\end{footnotesize}


% risk table values for subsystem DP-FD-HV
\begin{footnotesize}
%\begin{longtable}{p{0.18\textwidth}p{0.20\textwidth}p{0.32\textwidth}p{0.02\textwidth}p{0.02\textwidth}p{0.02\textwidth}}
\begin{longtable}{P{0.18\textwidth}P{0.20\textwidth}P{0.32\textwidth}P{0.02\textwidth}P{0.02\textwidth}P{0.02\textwidth}} 
\caption[DP HV risks]{Risks for DP-FD-HV (P=probability, C=cost, S=schedule) More information at \dshort{riskprob}. \fixmehl{ref \texttt{tab:risks:DP-FD-HV}}} \\
\rowcolor{dunesky}
ID & Risk & Mitigation & P & C & S  \\  \colhline
RT-DP-HV-01 & Broken resistors or varistors on voltage divider boards & Redundancy of resistors, varistors, and \dshorts{hvdb}.  & L & L & L \\  \colhline
RT-DP-HV-02 & \efield uniformity is not adequate for muon momentum reconstruction & Regularly map out field using a laser calibration sysem. & L & L & L \\  \colhline
RT-DP-HV-03 & \efield is below specification during stable operations & Improve purity by more aggressive filtering. & M & M & L \\  \colhline
RT-DP-HV-04 & Space charge from positive ions distorting the \efield beyond expectation & Minimize insulators facing cryostat wall ground. & M & M & L \\  \colhline
RT-DP-HV-05 & Damage to \dshort{ce} in event of discharge & Minimize the energy released in a short time using highly resistive connections. & L & L & L \\  \colhline
RT-DP-HV-06 & Energy stored in FC (in DP) is suddenly discharged & Delay energy discharge by connecting neighboring Al profiles with resistive sheaths.  & L & L & L \\  \colhline
RT-DP-HV-07 & Detector components are damaged during shipment to the far site & Make sufficient spares and increase the number of shipping boxes.  & L & L & L \\  \colhline
RT-DP-HV-08 & Damages (scratches, bending) to aluminum profiles of Field Cage modules & Make sufficient spares and increase the number of shipping boxes.  & L & L & L \\  \colhline
RT-DP-HV-09 & Bubbles from heat in PMTs or resistors cause HV discharge & A large area of cathode consists of high resistance rods, delaying the energy release.   & L & L & L \\  \colhline
RT-DP-HV-10 & Free hanging frames can swing in the fluid flow &  & L & L & L \\  \colhline
RT-DP-HV-11 & FRP/ Polyethene/ laminated Kapton component lifetime is less than expected &  & L & L & L \\  \colhline
RT-DP-HV-12 & Lack of collaboration effort on this HV system & Continue recruiting collaborators. & L & L & L \\  \colhline
RT-DP-HV-13 & International funding level for DP HVC too low & Employ cost saving measures and  recruit collaborators. & L & L & L \\  \colhline
RT-DP-HV-14 & Underground installation is more labor intensive or slower than expected & Increase labor contingency and refine labor cost estimates. Further improve installation procedure. & L & L & L \\  \colhline

\label{tab:risks:DP-FD-HV}
\end{longtable}
\end{footnotesize}


% risk table values for subsystem DP-FD-TPC
\begin{footnotesize}
%\begin{longtable}{p{0.18\textwidth}p{0.20\textwidth}p{0.32\textwidth}p{0.02\textwidth}p{0.02\textwidth}p{0.02\textwidth}}
\begin{longtable}{P{0.18\textwidth}P{0.20\textwidth}P{0.32\textwidth}P{0.02\textwidth}P{0.02\textwidth}P{0.02\textwidth}} 
\caption[DP TPC electronics risks]{Risks for DP-FD-TPC (P=probability, C=cost, S=schedule) More information at \dshort{riskprob}. \fixmehl{ref \texttt{tab:risks:DP-FD-TPC}}} \\
\rowcolor{dunesky}
ID & Risk & Mitigation & P & C & S  \\  \colhline
RT-DP-TPC-01 & Component obsolescence over the experiment lifetime & Monitor component stocks and procure an adequate number of spares at the time of production & L & M & L \\  \colhline
RT-DP-TPC-02 & Modification to the LRO FE electronics due to evolution in design of PD design & A strict and timely following of the evolution of DP PDS & L & L & M \\  \colhline
RT-DP-TPC-03 & Damage to electronics due to HV discharges or other causes & FE analog electronics is protected with TVS diodes. Electronics can be easily replaced. & L & L & L \\  \colhline
RT-DP-TPC-04 & Problems with FE card extraction due to insufficient overhead clearance & Addressed by imposing a clearance requirement on \dshort{lbnf} & L & L & L \\  \colhline
RT-DP-TPC-05 & Overpressure in the \dshorts{sftchimney} & The \dshorts{sftchimney} are equipped with overpressure release valves & L & L & L \\  \colhline
RT-DP-TPC-06 & Leak of nitrogen inside the \dshort{dpmod} via cold flange & Monitor chimney pressure for leaks and switch to argon cooling in case of a leak & L & L & L \\  \colhline
RT-DP-TPC-07 & Data flow increase due to inefficient compression caused by higher noise & Have a sufficiently large (a factor of \num{5}) margin in the available bandwidth & L & L & L \\  \colhline
RT-DP-TPC-08 & Damage to \dshort{utca} crates due to presence of water on the roof of the cryostat & \dshort{lbnf} requirement that the cryostat top remains dry & L & L & L \\  \colhline
RT-DP-TPC-09 & Clogging ventilation system of \dshort{utca} crates due to bad air quality & \dshort{lbnf} requirement that the air quality is comparable to a standard industrial environment & L & L & L \\  \colhline

\label{tab:risks:DP-FD-TPC}
\end{longtable}
\end{footnotesize}


% risk table values for subsystem DP-FD-PDS
\begin{longtable}{p{0.18\textwidth}p{0.20\textwidth}p{0.32\textwidth}p{0.02\textwidth}p{0.02\textwidth}p{0.02\textwidth}} 
\caption{Risks for DP-FD-PDS \fixmehl{ref \texttt{tab:risks:DP-FD-PDS}}} \\
\rowcolor{dunesky}
ID & Risk & Mitigation & P & C & S  \\  \colhline
RT-DP-PDS-01 & Insufficient light yield due to inefficient PDS design & Increase coverage of \dword{pmt} photo-cathodes and/or \dword{wls} reflector foils. & L & M & L \\  \colhline
RT-DP-PDS-02 & Poor coating quality for \dword{tpb} coated surfaces and \dword{lar} contamination by \dword{tpb} & Test quality and ageing properties of \dword{tpb} coating techniques. Improve techniques if needed. & L & L & L \\  \colhline
RT-DP-PDS-03 & \dword{pmt} channel loss due to faulty \dword{pmt} base design & Optimize clustering algorithms. Improve \dword{pmt} base design from analysis of possible failure modes in \dword{pddp}. & L & L & L \\  \colhline
RT-DP-PDS-04 & Bad \dword{pmt} channel due to faulty connection between \dword{hv}/signal cable and \dword{pmt} base & Optimize clustering algorithms. Test connectivity in \lntwo prior to installation. & L & L & L \\  \colhline
RT-DP-PDS-05 & \dword{pmt} signal saturation & Tune \dword{pmt} gain. In worst case, redesign \dword{fe} to adjust to analog input range of \dword{adc}. & M & L & L \\  \colhline
RT-DP-PDS-06 & Excessive electronics noise to distinguish \dword{lar} scintillation light & Measure noise levels during commissioning prior to \lar filling. Modify grounding, shielding, or power distribution schemes. & M & L & L \\  \colhline
RT-DP-PDS-07 & Availability of resources for work at the installation/integration site less than planned & Bring consortium members to the integration/installation site. & L & L & L \\  \colhline
RT-DP-PDS-08 & Damage of \dwords{pmt} during shipment to the experiment site & Use special packaging to avoid damage during shipment. Plan for 10\% spare  \dwords{pmt}. & L & L & L \\  \colhline
RT-DP-PDS-09 & Damage of optical fibers during installation & Install fibers last. Provide detailed installation procedures.   & L & L & L \\  \colhline
RT-DP-PDS-10 & Excessive exposure to ambient light of \dword{tpb} coated surfaces, resulting in degraded performance & Cover \dword{tpb} coated surfaces until cryostat closing. Include in installation procedure: minimize exposure to ambient light. & L & L & L \\  \colhline
RT-DP-PDS-11 & \dword{pmt} implosion during \dword{lar} filling & No mitigation necessary, per \dword{pmt} \SI{7}{bar} pressure rating and experience with similar \dwords{pmt} in  large liquid detectors. & L & L & L \\  \colhline
RT-DP-PDS-12 & Insufficient light yield due to poor \dword{lar} purity & Procure \dword{lar} with less than 3 ppm nitrogen. & M & L & L \\  \colhline
RT-DP-PDS-13 & \dword{pmt} channel or \dword{pds} sector loss due to failures in \dword{hv}/signal rack & Access/replace components easily outside cryostat and provide spares for all components for at least one \dword{hv}/signal rack. & L & L & L \\  \colhline
RT-DP-PDS-14 & Unstable response of the photon detection system over the lifetime of the experiment & Correct channel-level instabilities via light calibration system. Correct detector-level instabilities via cosmic-ray muon calibration data. & L & L & L \\  \colhline

\label{tab:risks:DP-FD-PDS}
\end{longtable}


% risk table values for subsystem DP-FD-CAL
\begin{footnotesize}
%\begin{longtable}{p{0.18\textwidth}p{0.20\textwidth}p{0.32\textwidth}p{0.02\textwidth}p{0.02\textwidth}p{0.02\textwidth}}
\begin{longtable}{P{0.18\textwidth}P{0.20\textwidth}P{0.32\textwidth}P{0.02\textwidth}P{0.02\textwidth}P{0.02\textwidth}} 
\caption[DP calibration risks]{Risks for DP-FD-CAL (P=probability, C=cost, S=schedule) More information at \dshort{riskprob}. \fixmehl{ref \texttt{tab:risks:DP-FD-CAL}}} \\
\rowcolor{dunesky}
ID & Risk & Mitigation & P & C & S  \\  \colhline
RT-DP-CAL-01 & Inadequate baseline design & Early detection allows R\&D of alternative designs accommodated through multipurpose ports. & L & M & M \\  \colhline
RT-DP-CAL-02 & Inadequate engineering or production quality & Dedicated small-scale tests and full prototyping at ProtoDUNE; pre-installation QC. & L & M & M \\  \colhline
RT-DP-CAL-03 & Laser impact on PDS & Mirror movement control to minimize direct hits; intelock to keep laser off while PMTs are on. & L & L & L \\  \colhline
RT-DP-CAL-04 & Laser positioning system stops working & QC at installation time, redundancy in available targets, including passive, alternative methods. & L & L & L \\  \colhline
RT-DP-CAL-05 & Laser beam misaligned & Additional (visible) laser for alignment purposes. & M & L & L \\  \colhline
RT-DP-CAL-06 & The neutron anti-resonance is much less pronounced & Dedicated measurements at LANL and test at ProtoDUNE. & L & L & L \\  \colhline
RT-DP-CAL-07 & Neutron activation of the moderator and cryostat & Neutron activation studies and simulations. & L & L & L \\  \colhline
RT-DP-CAL-08 & Neutron yield not high enough & Simulations and tests at ProtoDUNE; alternative, movable design & L & M & M \\  \colhline
RO-DP-CAL-09 & Laser beam is stable at longer distances than designed & tests at ProtoDUNE & M & H & L \\  \colhline

\label{tab:risks:DP-FD-CAL}
\end{longtable}
\end{footnotesize}


% risk table values for subsystem DP-FD-DAQ
\begin{footnotesize}
%\begin{longtable}{p{0.18\textwidth}p{0.20\textwidth}p{0.32\textwidth}p{0.02\textwidth}p{0.02\textwidth}p{0.02\textwidth}}
\begin{longtable}{P{0.18\textwidth}P{0.20\textwidth}P{0.32\textwidth}P{0.02\textwidth}P{0.02\textwidth}P{0.02\textwidth}} 
\caption[DP DAQ Risks]{Risks for DP-FD-DAQ (P=probability, C=cost, S=schedule) More information at \dshort{riskprob}. \fixmehl{ref \texttt{tab:risks:DP-FD-DAQ}}} \\
\rowcolor{dunesky}
ID & Risk & Mitigation & P & C & S  \\  \colhline
RT-DP-DAQ-01 & Detector noise specs not met & ProtoDUNE experience with noise levels and provisions for data processing redundancy in DAQ system; ensure enough headroom of bandwidth to FNAL. & L & L & L \\  \colhline
RT-DP-DAQ-02 & Externally-driven schedule change & Provisions for standalone testing and commissioning of production DAQ components, and schedule adjustment & L & L & L \\  \colhline
RT-DP-DAQ-03 & Lack of expert personnel & Resource-loaded plan for DAQ backed by institutional commitments, and schedule adjustment using float & L & L & H \\  \colhline
RT-DP-DAQ-04 & Power/space requirements exceed CUC capacity & Sufficient bandwidth to surface and move module 3/4 components to an expanded surface facility & L & L & L \\  \colhline
RT-DP-DAQ-05 & Excess fake trigger rate from instrumental effects & ProtoDUNE performance experience, and provisions for increase in event builder and high level filter capacity, as needed; headroom in data link to FNAL. & L & L & L \\  \colhline
RT-DP-DAQ-06 & Calibration requirements exceed acceptable data rate & Provisions for increase in event builder and high level filter capacity, as neeed; headroom in data link to FNAL. & L & L & L \\  \colhline
RT-DP-DAQ-07 & Cost/performance of hardware/computing excessive & Have prototyping and pre-production phases, reduce performance using margin or identify additional funds & L & L & L \\  \colhline
RT-DP-DAQ-08 & PDTS fails to scale for DUNE requirements & Hardware upgrade & L & L & L \\  \colhline
RT-DP-DAQ-09 & WAN network & Extensive QA and development of failure mode recovery and automation, improved network connectivity, and personnel presence at SURF as last resort. & L & M & M \\  \colhline
RT-DP-DAQ-10 & Infrastructure & Design with redundancy, prior to construction, and improve power/cooling system. & M & M & L \\  \colhline
RT-DP-DAQ-11 & Custom electronics manifacturing issues & Diversify the manifacturers used for production; run an early pre-production and apply stringent QA criteria. & L & M & M \\  \colhline

\label{tab:risks:DP-FD-DAQ}
\end{longtable}
\end{footnotesize}


% risk table values for subsystem DP-FD-CISC
\begin{footnotesize}
%\begin{longtable}{p{0.18\textwidth}p{0.20\textwidth}p{0.32\textwidth}p{0.02\textwidth}p{0.02\textwidth}p{0.02\textwidth}}
\begin{longtable}{P{0.18\textwidth}P{0.20\textwidth}P{0.32\textwidth}P{0.02\textwidth}P{0.02\textwidth}P{0.02\textwidth}} 
\caption[DP CISC risks]{Risks for DP-FD-CISC (P=probability, C=cost, S=schedule) More information at \dshort{riskprob}. \fixmehl{ref \texttt{tab:risks:DP-FD-CISC}}} \\
\rowcolor{dunesky}
ID & Risk & Mitigation & P & C & S  \\  \colhline
RT-DP-CISC-001 & Baseline design from ProtoDUNEs for an instrumentation device is not adequate for DUNE far detectors & Focus on early problem discovery in ProtoDUNE so any needed redesigns can start as soon as possible. & L & M & L \\  \colhline
RT-DP-CISC-002 & Swinging of long instrumentation devices (T-gradient monitors or PrM system) & Add additional intermediate constraints to prevent swinging. & L & L & L \\  \colhline
RT-DP-CISC-003 & High E-fields near instrumentation devices cause dielectric breakdowns in \dshort{lar} & CISC systems shielded and placed as far from cathode and FC as possible. & L & L & L \\  \colhline
RT-DP-CISC-004 & Light pollution from purity monitors and camera light emitting system & Use PrM lamp and camera lights outside PDS trigger window; cover PrM cathode to reduce light leakage. & L & L & L \\  \colhline
RT-DP-CISC-005 & Temperature sensors can induce noise in cold electronics & Check for noise before filling and remediate, repeat after filling. Filter or ground noisy sensors. & L & L  & L \\  \colhline
RT-DP-CISC-006 & Disagreement between lab and \em{in situ} calibrations for ProtoDUNE-SP dynamic T-gradient monitor & Investigate and improve both methods, particularly laboratory calibration. & M & L & L \\  \colhline
RT-DP-CISC-007 & Purity monitor electronics induce noise in TPC and PDS electronics. & Operate lamp outside TPC+PDS trigger window. Surround and ground light source with Faraday cage. & L & L & L \\  \colhline
RT-DP-CISC-008 & Discrepancies between measured temperature map and CFD simulations in ProtoDUNE-SP & Improve simulations with additional measurements inputs; use fraction of sensors to predict others   & L & L & L \\  \colhline
RT-DP-CISC-009 & Difficulty correlating purity and temperature in ProtoDUNE-SP impairs understanding cryo system. & Identify causes of discrepancy, modify design. Calibrate PrM differences, correlate with RTDs. & L & L & L \\  \colhline
RT-DP-CISC-010 & Cold camera R\&D fails to produce prototype meeting specifications \& safety requirements & Improve insulation and heaters. Use cameras in ullage or inspection cameras instead. & M & M & L \\  \colhline
RT-DP-CISC-011 & HV discharge caused by inspection cameras & Study E-field in and on housing and anchoring system. Test in HV facility. & L & L & L \\  \colhline
RT-DP-CISC-012 & HV discharge destroying the cameras & Ensure sufficient redundancy of cold cameras. Warm cameras are replaceable. & L & M & L \\  \colhline
RT-DP-CISC-013 & Insufficient light for cameras to acquire useful images & Test cameras with illumination similar to actual detector. & L & L & L \\  \colhline
RT-DP-CISC-014 & Cameras may induce noise in cold electronics & Continued R\&D work with grounding and shielding in realistic conditions. & L & L & L \\  \colhline
RT-DP-CISC-015 & Light attenuation in long optic fibers for purity monitors  & Test the max.\ length of usable fiber, optimize the depth of bottom PrM, number of fibers. & L & L & L \\  \colhline
RT-DP-CISC-016 & Longevity of purity monitors & Optimize PrM operation to avoid long running in low purity. Technique to protect/recover cathode. & L & L & L \\  \colhline
RT-DP-CISC-017 & Longevity: Gas analyzers and level meters may fail. & Plan for future replacement in case of failure or loss of sensitivity.  & M & M & L \\  \colhline
RT-DP-CISC-018 & Problems in interfacing  hardware devices (e.g. power supplies) with slow controls & Involve slow control experts in choice of hardware needing control/monitoring.
 & L & L & L \\  \colhline

\label{tab:risks:DP-FD-CISC}
\end{longtable}
\end{footnotesize}


% risk table values for subsystem DP-FD-INST
\begin{footnotesize}
%\begin{longtable}{p{0.18\textwidth}p{0.20\textwidth}p{0.32\textwidth}p{0.02\textwidth}p{0.02\textwidth}p{0.02\textwidth}}
\begin{longtable}{P{0.18\textwidth}P{0.20\textwidth}P{0.32\textwidth}P{0.02\textwidth}P{0.02\textwidth}P{0.02\textwidth}} 
\caption[DP module installation risks]{Risks for DP-FD-INST (P=probability, C=cost, S=schedule) More information at \dshort{riskprob}. \fixmehl{ref \texttt{tab:risks:DP-FD-INST}}} \\
\rowcolor{dunesky}
ID & Risk & Mitigation & P & C & S  \\  \colhline
RT-DPINST-01 & Personnel injury & Follow the safety rules in force. & L & L & H \\  \colhline
RT-DPINST-02 & Cryostat damage during installation & Use temporary protection for the corrugated membrane. & L & L & M \\  \colhline
RT-DPINST-03 & Detector components damage during transport/installation & Handling should be done by trained people, following the handling instructions provided by the consortia and in presence of a technical expert. & L & M & M \\  \colhline
RT-DPINST-04 & Detector components failure during test & Only trained people should test equipments. Spare components must be available at the warehouse facility. & L & L & M \\  \colhline
RT-DPINST-05 & Components interferences & 3D model, survey at the construction sites, and full scale assembly tests. & M & L & L \\  \colhline
RT-DPINST-06 & Shipping delays/missing parts & Buffer material and detector components at the warehouse facility and use inventory tools to follow the fundamental items. & L & L & L \\  \colhline
RT-DPINST-07 & Lack of specialised/trained manpower & Hire and train personnel. Plan with all the underground stakeholders enough in advance the required personnel. & L & L & L \\  \colhline

\label{tab:risks:DP-FD-INST}
\end{longtable}
\end{footnotesize}



%%%%%%%%%%%%%%%%%%%%%%%%%%%%%%%%
\section{Hazard Analysis Report (HAR)}
\label{sec:fdsp-har}

A key element of an effective \dword{esh} program is the hazard
identification process. Hazard identification allows production of a
list of hazards within a facility, so these hazards can be screened
and managed through a suitable set of controls.

The \dword{lbnf-dune} project completed a \dword{har}
to ensure that identified hazards are mitigated early in the
the design process.  The focus of the report is on process hazards,
not activity hazards that are typically covered in a job hazard
analysis.  The \dword{har} has been completed, identifying
hazards anticipated in the project's construction and operational
phases.

The hazard \dword{har} looks at the consequences of a hazard
to establish a pre-mitigation risk category. Proposed mitigation is
applied to hazards of concern to reduce risk and then establishes a
post-mitigation risk category.

As the \dword{dune} design matures, the \dword{har} will be
updated to ensure that all hazards are properly identified and
controlled through design and safety management system programs.  In
addition, some sections of the \dword{har} are used to meet
the safety requirements as defined in 10 CFR 851 and \dword{doe} Order
420.2C, Safety of Accelerator Facilities.  Table~\ref{tab:hazards}
summarizes these hazards.  The sections following the table describe
in more detail the hazards that are most applicable to \dword{dune}
activities and the design and operational controls used to mitigate
these hazards. The results of these evaluations confirm that the
potential risks from construction, operations, and maintenance are
acceptable. Individual activity-based \dword{ha} will be
developed for each work \dword{lbnf-dune} activity at
\dword{surf}.

\begin{longtable}{|p{0.35\textwidth}|p{0.28\textwidth}|p{0.28\textwidth}|}
  \caption[List of identified hazards]{List of identified hazards}
  \label{tab:hazards} \\  \toprowrule
  \rowtitlestyle   HA-1 (Construction) & HA-2 (Natural Phenomena) & HA-3 (Environmental)   \\ \toprowrule
  Site Clearing, Excavation, Mining, Tunneling (explosives), Vertical/Horizontal Conveyance Systems,
  Confined space, Heavy Equipment, Work at Elevations (steel, roofing), Material Handling (rigging)
  Utility interfaces, (electrical, steam, chilled water), Slips/trips/falls, Weather related conditions
  Scaffolding, Transition to Operations, Radiation Generating Devices &
  Seismic, Flooding, Wind, Lightning, Tornado &
  Construction impacts,
  Storm water discharge (construction and operations), Operations impacts, Soil and groundwater activation/contamination,
  Tritium contamination, Air activation, Cooling water activation (HVAC and Machine),
  Oils/chemical leaks or spills, Discharge/emission points (atmospheric/ground)\\ \colhline
  \rowtitlestyle HA-4 (Waste) & HA-5 (Fire) & HA-6 (Electrical)   \\ \toprowrule
  Construction Phase, Facility maintenance, Experimental Operations, Industrial, Hazardous, Radiological &
  Facility Occupancy Classification, Construction Materials, Storage, Flammable/combustible liquids,
  Flammable gasses, Egress/access, Electrical, Lightning, Welding/cutting/brazing work, Smoking  &
  Facility, Experimental, Job built Equipment, Low Voltage/High Current, High Voltage/High Power,
  Maintenance, Arc flash, Electrical shock, Cable tray overloading/mixed utilities, Exposed 110V,
  Stored energy (capacitors \& inductors), Be in contactors   \\ \colhline
  \rowtitlestyle   HA-7 (Mechanical) & HA-8 (Cryo/ODH) & HA-9 (Confined Space)   \\ \toprowrule
  Construction Tools, Machine Shop Tools, Industrial Vehicles, Drilling, Cutting, Grinding,
  Pressure/Vacuum Vessels and Lines, High Temp Equipment (Bakeouts) &
  Thermal, Cryogenic systems, Pressure, Handling and Storage,
  Liquid argon/nitrogen spill/leak, Use of inert gases (argon, nitrogen, helium), Specialty gases &
  Sumps, Utility Chases        \\ \colhline
  \rowtitlestyle   HA-11 (Chemical) & HA-14 (Laser) & HA-15 (Material Handling)   \\ \toprowrule
  Toxic, Compressed gas, Combustibles, Explosives, Flammable gases, Lead (shielding), Cryogenic &
  Alignment Laser, Testing and Calibration, Magnetic Fields, Calibration \& Testing &
  Overhead cranes/hoists, Fork trucks, Manual material handling, Delivery area distribution,
  Manual movement of materials, Hoisting \& Rigging, Lead, Beryllium Windows,Oils, Solvents, Acids,
  Cryogens, Compressed Gases   \\ \colhline
  \rowtitlestyle   HA-16 (Experimental Ops) &  &    \\ \toprowrule
  Electrical equipment, Water Hazard, Working from heights (scaffolding/lifts), Transportation of hazardous materials,
  Liquid Argon/Nitrogen, Chemicals (Corrosive, Reactive, Flammable), Elevations, Ionizing radiation,
  Ozone production, Slips, trips, falls, Machine tools/hand tools, Stray static magnetic fields, Research gasses (Inert, Flammable) &
  &   \\  
\end{longtable}

\subsection{Construction Hazards (LBNF-DUNE HA-1)}

The project will use the existing work planning and
control process for the laboratories along with a construction project safety and health
plan to communicate these policies and procedures as required by \dword{doe}
Order 413.3b. The installation and construction hazards
anticipated for the \dword{lbnf-dune} project include the following:
\begin{itemize}
\item Site clearing;
\item Excavation;
\item Installing vertical/horizontal conveyance systems;
\item Confined space;
\item Heavy equipment operation;
\item Work at elevation (erecting steel, roofing);
\item Material handling (rigging);
\item Utility interfaces (electrical, chilled water, ICW, natural gas);
\item Slips/trips/falls;
\item Weather related conditions;
\item Scaffolding;
\item Transition to operations; and
\item Devices generating radiation.
\end{itemize}

To reduce risks from construction hazards, \fnal will use engineered
and approved excavation and fall protection systems.  Heavy equipment
will use required safety controls. The \fnal construction safety
oversight program includes periodic evaluation of the construction
site and construction activities, \dword{ha} for all
subcontractor activities and frequent \dword{esh} communications at
the daily tool box meetings of subcontractors.

\subsection{Natural Phenomena (LBNF-DUNE HA-2)}

The \dword{lbnf-dune} design will be governed by the
International Building Code, 2015 edition; \dword{doe} Standard
(STD)-1020, 2016 edition, Natural Phenomena Hazard Analysis and Design
Criteria for \dword{doe} Facilities, guided the design in meeting the
natural phenomena hazard requirements.  The International Building
Code specifies design criteria for wind loading, snow loading, and
seismic events.

\dword{lbnf-dune} was determined to be a low-hazard,
performance category 1 facility according to the \dword{doe}
STD-1021-93. \dword{lbnf-dune} areas will contain small
quantities of activated, radioactive, and hazardous chemical
materials. Should a natural phenomenon hazard cause significant
damage, the impact will be mission-related and will not pose a hazard
to the public or the environment.

\subsection{Environmental Hazards (LBNF-DUNE HA-3)}

Environmental hazards from \dword{dune} include potentially releasing
chemicals to soil, groundwater, surface water, air, or sanitary sewer
systems that could, if not controlled, exceed regulatory limits.

\fnal maintains an environmental management system equivalent to ISO
14001, consisting of programs for protecting the environment, assuring
compliance with applicable environmental regulations and standards,
and avoiding adverse environmental impact through continual
improvement.  These programs are documented in the 8000 and 11000
series of chapters in the \dword{feshm}.  The environmental mitigation
plan also meets federal and state regulations.


\subsection{Waste Hazards (LBNF-DUNE HA-4)}

Waste-related hazards from \dword{dune} include the potential for
releasing waste materials (oils, solvents, chemicals, and radioactive
material) to the environment, injury to personnel, and reactive or
explosive event. Typical initiators will be transportation accidents,
incompatible materials, insufficient packaging or labeling, failure of
packaging, and a natural phenomenon.

During installation and \dword{dune} operation, we anticipate few
hazardous materials will be used. Such materials include paints,
epoxies, solvents, oils, and lead in the form of shielding. No current
or anticipated activities at \dword{dune} would expose workers to
levels of contaminants (dust, mists, or fumes) above regulatory
limits.

The \dword{esh}\&Q section industrial hygiene group and hazard control
technology team manage the program and guide collaborators subject to
waste-related hazards.  Their staff identify workplace hazards, help
identify controls, and monitor implementation. Industrial hygiene
hazards will be evaluated, identified, and mitigated as part of the
work planning and control hazard assessment process.

\subsection{Fire Hazards (LBNF-DUNE HA-5)}

Fire hazards have been evaluated and addressed to comply with
\dword{doe} Order 420.1C, Facility Safety, Chapter II and
\dword{doe}-STD 1066, Fire Protection Design Criteria.  The intent of
these documents is to meet \dword{doe}'s highly protected risk (HPR)
approach to fire protection.  In addition, the National Fire
Protection Association Standard 520, Standard on Subterranean Spaces,
was used in developing the basis for design related to fire
protection/life safety.

The combustible loads and the use of flammable and/or reactive
materials in the \dword{lbnf-dune} facility are controlled
following the International Building Code building occupancy
classification. Certain ancillary buildings outside the main structure
may be classified as higher hazard areas (Use Group H occupancy),
including the gas cylinder and chemical storage rooms, because they
hold more concentrated quantities of flammable or combustible
materials.  The control area concept used in the International
Building Code and National Fire Protection Association standards will
be followed for hazardous chemical use and storage areas to provide
the most flexibility and control of materials by allowing individual inventory
thresholds per control area.  The \dword{lbnf-dune} facility
will be equipped with fire detection systems and alarm systems that
will monitor water flow in case of fire suppression activation, as well
as monitor control valves and detection systems.

Audible/visual alarm notification devices will alert building
occupants.  Manual pull stations for the fire alarms will be installed
at all building exits.  Following National Fire Protection Association
90A, air handling systems will have photoelectric smoke detectors.
Smoke detection will be provided in areas with highly sensitive
electronic equipment.  Combinations of audible and visual alarm
notification devices will be set up throughout the underground
enclosures and service buildings to alert occupants. All fire alarm
signals will report through a centralized system at \dword{surf}.
Fire alarm and supervisory signals will be transmitted to internal and
external emergency responders using the campus reporting system.

While fixed fire protection systems afford an excellent level
of protection, additional strategies that include operational controls
that minimize combustible materials, adequately
fused power supplies, fire safety inspections, and operational
readiness reviews will be used to further reduce fire hazards
within the facility following \dword{doe} highly protected risk methods.

Experimental cabling will meet the requirements of the National Fire
Protection Association 70 and National Electrical Code, 2015 edition.
Preferred cables will be fire resistant, using appropriately
designated cable types for plenum or general-purpose cables.  When
there is a large investment in equipment for experiment power or
computer rack systems, or when equipment is custom-made (as opposed to
off-the-shelf commercial electronics), a device to detect faults or
smoke in the system will be provided.  This device will also shut
down the individual rack or racks when smoke or faults are detected. 


\subsection{Electrical Hazards (LBNF-DUNE HA-6)}

\dword{lbnf-dune} will have significant facility-related
systems and subsystems that produce or use high voltage, high current,
or high levels of stored energy, all of which can present electrical
hazards to personnel. Electrical hazards include electric shock and
arc flash from exposed conductors, defective and substandard
equipment, lack of training, or improper procedures.

\fnal has a well-established electrical safety program that
incorporates de-energizing equipment, isolation barriers, \dword{ppe}, 
and training. The cornerstone of the program is
the lockout/tagout following the \dword{feshm} Chapter 2100, \fnal
Energy Control Program (Lockout/Tagout).

Design, installation, and operation of electrical equipment will
comply with the National Electrical code (NFPA 70), applicable parts
of Title 29 Code of Federal Regulations, Parts 1910 and 1926, National
Fire Protection Association 70E, and \fnal electrical safety policies
documented in the \dword{feshm} 9000 series chapters. Equipment
procured from outside vendors or international in-kind partners will
be either certified by a nationally recognized testing laboratory,
conform to international standards previously evaluated and deemed
equivalent to USA standards, or inspected and accepted using \fnal's
electrical equipment inspection policies outlined in \dword{feshm}
9110, Electrical Utilization Equipment Safety.


\subsection{Noise/Vibration/Thermal/Mechanical (LBNF-DUNE HA-7)}

Hazards include overexposure of personnel to noise and vibrations as
specified by the American Conference of Governmental Industrial
Hygienists and US Occupational Safety and Health Administration
(OSHA), which set noise limits to avoid permanent hearing loss, also
known as permanent threshold shift. Vibration of equipment can
contribute to noise levels and could damage or interfere with
sensitive equipment.

\dword{lbnf-dune} will use a wide variety of equipment that
will produce a wide range of noise and vibration. Support equipment,
such as pumps, motors, fans, machine shops, and general HVAC all
contribute to point source and overall ambient noise levels. While
noise will typically be below the ACGIH and OSHA eight-hour time-weighted
average, certain areas with mechanical equipment could exceed that
criterion and will require periodic monitoring, posting, and use of
protective equipment. Ambient background noise is more a concern for
collaborator comfort, stress level, and fatigue.

The detector facilities use a wide variety of noisy equipment. Items such as pumps,
fans, and machine shop devices %,among others, 
are possible sources of
noise levels that might exceed the \fnal noise action
levels. \dword{feshm} Chapter 4140, Hearing Conservation, details
requirements for reducing noise and protecting personnel exposed to
excessive noise levels. Warning signs are posted wherever hazardous
noise levels may occur, and hearing protection devices are readily
available. Ways to reduce noise and vibration will be incorporated
into the \dword{lbnf-dune} design. These techniques include
using low-noise and low-vibration-producing equipment, especially for fans in
the HVAC equipment, isolating noise-producing equipment by segregating
or enclosing it, and using sound deadening materials on walls and
ceilings.

\subsection{Cryogenic/Oxygen Deficiency Hazard (LBNF-DUNE HA-8)}

The \dword{lbnf-dune} project will use large volumes of liquid
argon, nitrogen, and helium within the \dword{fscf}. Cryogenic
hazards could include \dword{odh} atmospheres due to failure of
the cryogenics systems, thermal (cold burn) hazards from cryogenic
components, and pressure hazards. Initiators could include the failure
or rupture of cryogenics systems from overpressure, failure of
insulating vacuum jackets, mechanical damage or failure, deficient
maintenance, or improper procedures.

Cryogenic liquids and gases are extremely dangerous to humans.
They can destroy tissue and damage materials and equipment past repair
by altering characteristics and properties (e.g., size, strength, and
flexibility) of metals and other materials.

Although cryogens are used extensively at \fnal, quantities that may
be used within a facility are strictly limited. Uses beyond defined
limits require \dword{odh} analyses and using
ventilation, \dword{odh} monitoring, or other controls.

Cryogenics systems are subject to formal project review, which includes
independent reviews by a subpanel of the Cryogenic Safety Subcommittee
following National Fire Protection Association Chapter 5032, Cryogenic
System Review. The members of this panel have relevant knowledge in
appropriate areas. They review the system safety documentation,
\dword{odh} analysis documentation, and the equipment before new
systems are permitted to begin the \cooldown process.

\fnal has developed and successfully deployed \dword{odh} monitoring
systems throughout the laboratory to support its current cryogenic
operations. The systems provide both local and remote alarms when
atmospheres contain less than 19.5\% oxygen by volume.

\fnal has a mature training program to address cryogenic safety
hazards. Key program elements include \dword{odh} training,
pressurized gas safety, and general cryogenic safety.


\subsection{Confined Space Hazards (LBNF-DUNE HA-9)}

Hazards from confined spaces could result in death or injury from
asphyxiation, compressive asphyxiation, smoke inhalation, or impact
with mechanical systems. Initiators would include failure of cryogenics
systems that are releasing liquid, gas, or fire, or failure of mechanical
systems. 

The \fnal confined-space program is outlined in \fnal Environmental
and Safety Manual Chapter 4230, Confined
Spaces. \dword{lbnf-dune} facilities will be incorporated into
this program. The emphasis at the \dword{lbnf-dune} design
phase will be to create the minimum number of confined spaces by
clearly articulating the definition of confined spaces to facility
designers to assure that such spaces have adequate egress, that mechanical
spaces are adequately sized, and, wherever possible, that no confined space
%created 
exists at all. During facility operations, the existing campus
confined-space program, along with appropriate labeling of confined
spaces, work planning and control, and entry permits will be used to
control access to these spaces.

\subsection{Chemical/Hazardous Materials Hazards (LBNF-DUNE HA-11)}

The \dword{dune} facility anticipates minimal use of chemical and hazardous
materials. Materials like paints, epoxies, solvents, oils, and lead
shielding may be used during construction and operation of the
facility. Exposure to these materials could result in injury;
exposure could also exceed regulatory limits. Initiators could be
experimental operations, transfer of material, failure of packaging,
improper marking or labeling, a reactive or explosive event, improper
selection of or lack of \dword{ppe}, or a
natural phenomenon.

\fnal maintains a database of hazardous chemicals in compliance with
the requirements imposed by 10 CFR 851 and \dword{doe} orders. In
addition to an inventory of chemicals at the facility, copies of each
manufacturer's safety data sheets (SDS) are maintained. Reviews of
conventional safety measures at the facilities show that using these
chemicals does not warrant special controls other than appropriate
signs, procedures, appropriate use of  \dword{ppe},
and hazard communication training. \dword{dune} will also supply
SDS documentation to the \dword{surf} \dword{esh}
department for all chemicals and hazardous materials that arrive on
site.

The industrial hygiene program, detailed in the \dword{feshm} 4000
series chapters, addresses potential hazards to workers using such
materials. The program identifies how to evaluate workplace hazards
when planning work and the controls necessary to either eliminate or
mitigate these hazards to an acceptable level.

Specific procedures are also in place for safe handling, storing,
transporting, inspecting, and disposing of hazardous materials. These
are contained in the \dword{feshm} 8000 and 10000 series chapters,
Environmental Protection and Material Handling and Transportation,
which describe how to comply with the standards set by the Code of
Federal Regulations, Occupational Safety and Health Standards, Hazard
Communication, Title 29 CFR, Part 1910.1200.


\subsection{Lasers \& Other Non-Ionizing Radiation Hazards (LBNF-DUNE HA-14)}
\label{sec:tc-esh-lasers}

Production and delivery of Class 3B and Class 4, near-infrared, UV,
and visible lasers must be completely contained in transport pipes or
designated enclosures for the Class 3B and Class 4 lasers, thus
creating a laser controlled area. (This will be in accordance with
\fnal \dword{feshm} chapter 4260.)  Establishing the laser controlled
area prevents areas around it from exceeding the maximum permissible
exposure as set by the \fnal laser safety officer.

\subsection{Material Handling Hazards (LBNF-DUNE HA-15)}

\dword{dune} will require a significant amount of manual and mechanical
material handling during the construction, installation, and operations
phases.  %The consequences of these 
These activities present hazards that include serious injury or
death to equipment operators and bystanders, damage to equipment and
structures, and interruption of the program.  Additional material
handling hazards from forklift and tow cart operations include injury
to the operator or personnel in the area and contact with equipment or
structures. Cranes and hoists will be used during fabrication,
testing, removal, and installation of equipment. The error precursors
associated with this type of work may include irregularly shaped loads,
awkward load attachments, limited space, obscured sight lines, and
poor communication.  The material or equipment being moved will
typically be one of a kind, expensive, or of considerable programmatic
value, and without dedicated lifting points or an obvious center
of gravity.

Lessons learned from across the \dword{doe} complex and OSHA have been
evaluated and incorporated into the \fnal material handling programs
documented in the \dword{feshm} 10000 series chapters.  The laboratory
limits personnel with access to mechanical material handling equipment
like cranes and forklifts to those who have successfully completed the
laboratory's training programs and demonstrated competence in
operating this equipment.


\subsection{Experimental Operations (LBNF-DUNE HA-16)}
Experimental activity undertaken at \dword{lbnf-dune} will be
fully reviewed under the \dword{orc} process
and by other experts as needed (e.g., representatives from electrical
safety, fire safety, environmental compliance, industrial hygiene,
cryogenic safety, and industrial safety), to identify and manage the
hazards of each experimental operation. The shift leader will ensure
that all safety reviews take place for each activity and that any
issues are appropriately addressed. The \dword{orc}  process will document
these reviews, covering the necessary controls and management approval
to proceed.

Typically, the \dword{orc}  process evaluates the scope of the proposed
experimental activity and identifies the hazards and controls to
mitigate them. The process ensures that collaborators are properly
trained, that qualified, hazardous material is kept to a minimum, that
engineering controls are deployed as a preferred mitigation, and that
\dword{ppe} is appropriate for the hazard.

