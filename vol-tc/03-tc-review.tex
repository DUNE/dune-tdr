\chapter{Reviews}
\label{vl:tc-review}

[Steve]

\dword{tc} is responsible for reviewing all stages of detector
development and works with each consortium to arrange reviews of the
design (conceptual design review (CDR), \dword{pdr} and \dword{fdr}),
production (\dword{prr} and \dword{ppr}) and \dword{orr} of their
system.  These reviews provide input for the TB to evaluate technical
decisions.  Review reports are tracked by \dword{tc} and provide
guidance as to key issues that will require engineering oversight by
the \dword{tc} engineering team. \Dword{tc} will maintain a calendar
of \dword{dune} reviews.

\Dword{tc} works with consortia leaders to review all detector designs,
with expectation of a \dword{pdr}, followed by a \dword{fdr}.  All
major technology decisions will be reviewed prior to down-select.  \Dword{tc}
may form task forces as necessary for specific issues that need more
in-depth review.


Start of production of detector elements can commence only after
successful \dwords{prr}. Regular production progress
reviews will be held once production has commenced. The \dwords{prr}
will typically include review of the production of \textit{Module 0}, the
first such module produced at the facility. \Dword{tc} will work with
consortium leaders for all production reviews.

\Dword{tc} is responsible to coordinate technical documents for the LBNC
Technical Design Review.

\section{Design Reviews}

Design review process (per DocDB-9664 and http://eshq.fnal.gov/manuals/feshm/)

\dword{tc} has empowered an Engineering Safety Committee consisting of
mechanical and electrical engineering experts from collaborating
institutions to develop the processes and procedures for evaluation of
engineering designs in terms of accepted international safety
standards.

\section{Production Reviews}

\section{Operations Reviews}

Operation readiness reviews (per http://eshq.fnal.gov/manuals/feshm/\#series2000)

\section{Review Tracking}

Tracking and control of review recommendations

