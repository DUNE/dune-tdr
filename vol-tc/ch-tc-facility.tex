???\chapter{Facility Description}
\label{vl:tc-facility}

The \dword{dune} detectors are located in the main underground campus at \dword{surf}. The main campus is located at the 4850 level between the Ross and Yates Shafts. Figure~\ref{fig:dune-underground} shows the overall underground campus at the \dword{surf} 4850 level. The following sections describe the facilites as thay are related to the \dword{dune} detectors.

\section{Underground Facilities and Infrastructure}
\label{sec:fdsp-coord-uderground-excavation}

The underground facilities include all the \dword{bsi} and utilities,
such as HVAC systems for underground environment control, chilled
water for cooling detector electronics and cryogenics compressors,
material handling equipment (monorails and bridge cranes), industrial
water systems and lighting required for constructing and operating
detectors.

Note: Above paragraph is from LBNF. This section needs more detail about overall excavation in terms that affect the detector. Where
does the power, HVAC, cooling water, etc go? What are the limits on DUNE? Add reference to BSI drawings?

Specfic comment from LBNC:
The document mentions that Building and Site Infrastructure (BSI) and utilities, such as HVAC systems etc. are the responsibility of LBNF. Is there a 
detailed interface document listing exactly what general services are installed (and when) by LBNF? Especially for some items that often fall between cracks, like doors, metallic structures (staircases, platforms???), fire partitions, fire-?????proofing of cable ducts after detector services installation etc. 



\section{Detector Caverns}
\label{sec:fdsp-coord-faci-caverns}


The underground facilities for the \dword{dune} detectors have two parallel
detector caverns with one central utility cavern between them. The
long direction of the caverns aligns approximately in the 
east west direction. Therefore, the two detector caverns are called the north
and south caverns.  The top of each detector is at the 4850 level, while the bottom of the cryostats are at the 4910 level.
\begin{dunefigure}[Underground campus]{fig:dune-underground}
  {Underground campus at the 4850 level.}
  \includegraphics[width=0.75\textwidth]{underground_campus_vertical.pdf}
\end{dunefigure}
Each detector cavern is \SI{144.5}{\meter} long, \SI{19.8}{\meter}
wide and \SI{27.95}{\meter} high, with one detector on the west side
and one on the east side. Between the two detector modules in each
cavern there is a \SI{12}{\meter} space for detector access, cryogenic pumps and valves.
. Access tunnels lie at the east and
west ends of each cavern, north and south side of north cavern and
north side of south cavern.

Detector one will be built in the east side of the north cavern and
detecor two will be built in the east side of the south cavern. This
is shown in Figure~\ref{fig:dune-full_assmebly}.
\begin{dunefigure}[\dword{dune} cryostat]{fig:dune-full_assmebly}
  {Placement of detector one and two in north and south caverns respectively}
  \includegraphics[width=0.85\textwidth]{LBNF_full_assembly.jpg}
\end{dunefigure}
The primary reasons for constructing the first detectors in the east sides of two caverns are:
\begin{itemize}
\item {\bf Personnel safety during construction and filling}: It is
  planned that detector one will be in the filling phase while detector
  two is in the construction phase. Therefore, construction of the the
  second detector has to take place in a totally separate cavern from
  filling of the first detector. In addition, the airflow in both
  caverns is in the west to east direction. Therefore, construction
  personnel who are primarily on the west side of the detectors will
  be on the upstream side of the airflow.
\item{\bf Access during construction}: The main access for bringing
  large items into both caverns is from the west side. It is,
  therefore, more advantageous that items that enter from the west are
  directly lowered and do not have to pass over detector construction
  site.
\end{itemize}
Both of the reasons above result in the same conclusion that the
detectors one and two are built on the east side of both caverns, and
that they are built in two separate caverns. It is, therefore, planned
that both caverns are ready for beneficial occupancy at the start of
respective detector construction phases.

It should be noted that both detectors will be built at the 4910
level. However, with the exception of the LAr circulation pumps, all
services are at the 4850 level.

The \dword{cuc} is \SI{190}{\meter} long, \SI{19.3}{\meter}
wide, and \SI{10.95}{\meter} high; it will house cryogenic equipment, \dword{daq}
room and mechanical and electrical services for all four
detectors.

\section{Cryostat}
\label{sec:fdsp-coord-cryostat}

Each detector is housed inside a cryostat. Each cryostat is made of
(from outside to inside): external steel structure, a warm membrane
plate, insulation and a cold membrane. Figure~\ref{fig:dune-cryostat}
shows the overall construction. Each cryostat will have a vertical
\dword{tco} \SI{13.43}{\meter} high and \SI{2.68}{\meter}
wide, at one end to allow for installation of detector components. The
opening will be closed after the majority of detector installation is
complete and before filling. After the \dword{tco} closing, the last of the
detector installation will be completed. After final detector
installation and equipment removal through the roof openings, the
cryostat will be closed for purge, cooldown and filling.
\begin{dunefigure}[\dword{dune} cryostat]{fig:dune-cryostat}
  {Overall construction and dimensions of the \dword{dune} cryostat.}
  \includegraphics[width=0.85\textwidth]{cryostat.pdf}
\end{dunefigure}

\section{Cryogenics}
\label{sec:fdsp-coord-cryogenics}

Note: get some more information from David M.


The detector cryogenics system supplies \dword{lar} and provides
circulation, re-condensation and purification. The cryogenic system
components are housed inside the \dword{cuc}, on top of each detector module
and between the two detector modules in each cavern. The cryogenic system comprises
\begin{itemize}
\item {\bf Cryogenic infrastructure}: This includes \dword{lar} and LN$_2$ receiving
  facilities on the surface, nitrogen refrigeration systems (both
  above ground and underground), LN$_2$ buffer storage
  underground, piping to interconnect equipment (LN$_2$, GN$_2$ and GAr),
  components in the detector cavern and the \dword{cuc} and process control/support
  equipment.
\item {\bf Proximity cryogenics}: This includes reliquefaction 
  and purification subsystems for the argon (both gas and liquid), associated
  instrumentation and monitoring equipment and \dword{lar} piping to
  interconnect equipment and components in the detector cavern and the
  \dword{cuc}. The proximity cryogenics are split into three areas: in the
  \dword{cuc}, on top of the mezzanine, as shown in Figure~\ref{fig:detector_mezzanines},
  and on the side of the cryostat where \dword{lar} circulation pumps are installed.
\item {\bf Internal cryogenics}: This includes \dword{lar} and GAr distribution
  systems inside the cryostat, as well as features to cool the
  cryostat and the detector uniformly.
\end{itemize}
Figure~\ref{fig:dune-cryogenics} shows the process flow diagram of the
\dword{lbnf} cryogenic system. Only one cryostat is shown.
\begin{dunefigure}[Cryogenics system]{fig:dune-cryogenics}
  {Overall process flow diagram of the cryogenic system showing one
    cryostat only; other cryostats are the same.}
  \includegraphics[width=0.85\textwidth]{LBNF_PFD_180909.pdf}
\end{dunefigure}


\section{Detector and Cavern Integration}
\label{sec:fdsp-coord-det-cav-integ}
The following figures and explanations show the
integration of the detector modules within the cavern. The interface
boundaries are not specifically pointed out in each figure, but they
follow the scheme as represented in Figure~\ref{fig:integration_nodes}.

Figure~\ref{fig:detector_cavern} shows one detector module in the
cavern. In this figure, the cryogenics equipment and racks on top of
the detector are visible. The \dword{lar} recirculation pumps can also be seen
on the lower level.
\begin{dunefigure}[Overall view of the detector in cavern.]{fig:detector_cavern}
  {Overall view of a detector module within the cavern.}
  \includegraphics[width=0.8\textwidth]{LBNF-Cryostat-NW_Iso_c.png}
\end{dunefigure}

Figure~\ref{fig:detector_ew_elevation} shows the east to west
elevation view of the detector in the cavern. The services from the
\dword{cuc} enter the cavern through a passage visible on the left.
\begin{dunefigure}[East-west elevation view of Detector]{fig:detector_ew_elevation}
  {East-west elevation view of one detector module in the cavern.}
  \includegraphics[width=0.9\textwidth]{LBNF-Cryostat-South_Elevation_in_Cavern_c.png}
\end{dunefigure}

Figure~\ref{fig:detector_ns_elevation} shows the north to south
elevation view of the detector in the cavern. The services entering
from the \dword{cuc} are visible on the right.
\begin{dunefigure}[North-south elevation view of detector]{fig:detector_ns_elevation}
  {North-south elevation view of one detector in the cavern.}
  \includegraphics[width=0.7\textwidth]{LBNF-Cryostat-West_Elevation_in_Cavern_c.png}
\end{dunefigure}

Figure~\ref{fig:detector_mezzanines} shows the elevation view of the
top of cryostat showing mezzanines, cryogenic equipment, and
electronic racks.
\begin{dunefigure}[Elevation view of top of cryostat showing mezzanines, cryogenic
    equipment and electronic racks.]{fig:detector_mezzanines}
  {Elevation view of top of cryostat showing mezzanines, cryogenic
    equipment, and electronic racks.}
  \includegraphics[width=0.85\textwidth]{LBNF-Cryostat-West_Elevation-Cryo_c.png}
\end{dunefigure}
The cryogenics are installed on a mezzanine supported from
the cavern roof and cavern wall. Cryogenic distribution lines are
routed under the mezzanine. Local control rooms for the
cryogenic equipment are on the mezzanine.

Detector electronics are installed in short racks close to
feedthroughs and in taller racks installed on a separate electronics
mezzanine shown on the left of Fig.~\ref{fig:detector_mezzanines}.
This will allow easy access for maintenance and reduce complexity on
top of the detector.

\section{Detector Grounding}
\label{sec:fdsp-coord-faci-grounding}


The grounding strategy provides the detectors with an independent
isolated ground to minimize any environmental electrical noise that
could couple into the detector readout electronics either conductively or
through emitted electromagnetic interference.

The detectors will be placed at the 4910 level of \surf. The
electrical conductivity of the various rock masses are unknown but
should have extremely poor and inconsistent conductive
properties. Ensuring adequate sensitivity of the detectors requires a
special ground system that will isolate the detectors from all other
electrical systems and equipment, minimize the influence of inductive
and capacitive coupling, and eliminate ground loops. The grounding
infrastructure should reduce or eliminate ground currents through the
detector that would affect detector sensitivity, maintain a low
impedance current path for equipment short circuit and ground fault
currents, and ensure personnel safety by limiting any potential for
equipment-to-equipment and equipment-to-ground contact.

The infrastructure grounding plan of the underground facilities is fully described in EDMS 2095975.  The plan is to have a separate Detector Ground for each of the four detectors which is ?isolated? from the rest of the facility.  The Detector Ground will primarily be comprised of the steel containment vessel, cryostat membrane and connected readout electronics.  The facility ground is constructed out of two interconnected grounding structures; these are the Cavern Ground and the Ufer grounds which are described below.  For safety reasons, a saturable inductor will connect the Detector ground to the rest of the facility. 

Figure~\ref{fig:dune-grounding} shows the areas of construction for the Cavern and Ufer grounds..
\begin{dunefigure}[Overall \dword{dune} grounding structure]{fig:dune-grounding}
  {Overall \dword{dune} facility grounding structure incorporated in cavern.}
  \includegraphics[width=0.85\textwidth]{SURF_Grounding.pdf}
\end{dunefigure}
Grounding structure definitions include:
\begin{enumerate}
 \item Cavern Ground consisting of overlapping welded wire mesh
   supported by rock bolts and covered with shotcrete. The
   \dword{lbnf}/\dword{dune} Cavern Ground includes all walls and
   crown areas above the 4850 level in the north and south detector
   caverns and their associated central access drifts, as well as tin-plated
   copper bus bars that run the length of the detector vessels
   on each side along the cavern walls and are mounted external to the
   shotcrete.  The Cavern Ground structure
\begin{enumerate}
 \item spans the full length of the cavern from the west end access
   drift entrance through the mid-chamber to the east end access drift
   entrance;
 \item spans the full width of the cavern from the 4850 level sill
   (top of the detector vessels and mid-chamber floor) on both sides
   up and across the crown of the cavern;
 \item includes mid chamber walls to the 4910 level;
 \item includes the east and west end walls of the cavern, from the
   4850 level to the crown.
\end{enumerate}
 \item \dword{ufer} ground consisting of metal rebar embedded in
   concrete floors. The \dword{lbnf}/\dword{dune} \dword{ufer} ground
   system includes the concrete floors in the cavern mid-chambers,
   center access drifts, and central utility cavern. The Cavern and
   \dword{ufer} Grounds will be well bonded electrically to construct
   a single facility ground isolated from detector ground.
 \item Detector Ground consists of the steel containment vessel
   enclosing the cryostat and all metal structures attached to or
   supported by the detector vessel.
\end{enumerate}


To ensure safety, a safety ground with one or more saturated inductors
will be installed between the Detector Ground and the electrically
bonded \dword{ufer} and Cavern Grounds which form the facility ground.
Figure~\ref{fig:dune-grounding_figure} illustrates the use of the
safety ground. The safety ground inductors saturate with flux under
low-frequency high currents and present minimal impedance to that
current.  Thus, an AC power fault current would be shunted to the
facility ground and provide a safe grounding design. At higher
frequencies and lower currents, such as coupled noise currents, the
inductor provides an impedance to this current, restricting its flow
between grounded metal structures. The desired total impedance between
the detector ground structure and the cavern/\dword{ufer} ground
structure should be a minimum of \SI{10}{Ohms} at \SI{10}{MHz}.

\begin{dunefigure}[Simplified detector grounding]{fig:dune-grounding_figure}
  {Simplified detector grounding scheme.}
  \includegraphics[width=0.5\textwidth]{Simplified_Grounding_Figure.pdf}
\end{dunefigure}

As stated above, the Detector Ground exists only in the area of the steel containment vessel enclosing the cryostat and all metal structures attached to or supported by the detector vessel.  All signal cables which run between the detector and the DAQ Underground Processing Room in the Central Utilities Chamber will be fiber optic.  All connections to the cryogenic plant which will be on the facility ground will be isolated from the cryostat with dielectric breaks.  A conceptual drawing showing the isolation of the cryostat is presented below in Figure~\ref{fig:dune-grounding_scheme}

\begin{dunefigure}[Detector grounding schematic]{fig:dune-grounding_scheme}
  {Schematic of detector grounding system.}
  \includegraphics[width=0.75\textwidth]{DUNE_Grounding_figure_vertical.PNG}
\end{dunefigure}
The construction of the facility ground provides a low impedance path for return currents of the facility services, such as cryogenic pumps, and noise coupling from facility services will be greatly reduced or eliminated.  The experiment has also been carefully designed such that facility return currents will not flow under the cryostats.

The cryostat itself is treated as a Faraday cage.  Any connections coming from facilities outside of actual detector electronics are electrically isolated from the cryostat.  For detector electronics, specific rules for signal and power cables penetrating the cryostat are given in EDMS 2095958.  





\section{Detector Power}
\label{sec:fdsp-coord-faci-power}

The DUNE detectors are constrained in both size and power consumption.  Requirements were given to conventional facilities to set limits on both the size of the cavern excavation and the power consumption of the electronics.  These limits have driven other factors, such as the cooling design and DUNE experiments must stay within established power requirements.

The conventional facilities will supply a 1000 KVA transformer for each cavern.  Each cavern will host two DUNE detectors.  Power from this initial transformer will be de-rated with no more than 75% of total power available at the electrical distribution panels.  We plan for a maximum consumption of 360 KW per detector.

For the single phase detector, the power will be largely distributed to a number of detector racks which will sit on a detector rack mezzanine above the cryostat.  Each of the racks will receive a 30 Amp 120 Volt supply and there will be a maximum of 80 racks.  

The cold electronics readout dissipates 306 W per APA. The LV power supplies have a controller which adds approximately 35 W per
APA and has an efficiency of approximately 85 percent. This leads to about 400 Watts
per APA, or a total load of 60 KW per detector module. The APA wire-bias power supplies
have a maximum load of 465 W per set of six APAs, for a total budget of around 12 KW. Cooling
fans and heaters near the feedthroughs will use a nominal amount of power, so the overall power
budget for cold electronics is expect to be less than 75 KW.

The PDS electronics is based on the Mu2e electronics, which reports a total power load of approximately
6 kW. DUNE plans a power budget of 8 kW because of cable drops and power supply inefficiencies.
The PDS electronics present a significantly lower power load than the alternate solution, used
in ProtoDUNE-SP, of using SSP modules at a power load of approximately 72 KW per detector
module.
Each of the approximately 80 detector racks will have fan units, Ethernet switches, rack protection,
and slow controls modules, adding a load of about 500 W per rack, for a total of 40 KW.
Twenty-five racks are reserved for cryogenics instrumentation with a per-rack load conservatively
estimated at 2 KW, for a total of 50 KW.
The detector module will thus use an estimated 173 KW of power. The higher-load SSP alternative
for the PDS would increase this to 237 KW. The highest estimate is approximately 67% of our available power.

Estimates from the dual phase electronics is approximately 160 KW and also fits well within the planned maximum of 360 KW.

For the single phase detector, the power will be largely distributed to a number of detector racks which will sit on a detector rack mezzanine above the cryostat.  Each of the racks will receive a 30 Amp 120 Volt supply and there will be a maximum of 80 racks.  

The other area where DUNE requires power underground is in the DAQ room.  There the power budget is determined by the available 750 KVA transformer.  We plan on having 480KW available to be distributed to a maximum of 60 racks.  Each water-cooled rack will have approximately 8KW available.  

\section{Data Fibers}
\label{sec:fdsp-coord-faci-fibers}


The \dword{dune} experiment requires a number of fiber 
optic pairs to run between the surface and the 4850 level.  A
total of 96 fiber pairs, which includes both \dword{dune} and \dword{lbnf}
needs, will be supplied through redundant paths with bundles of 96 pairs
coming down both the Ross and Yates shafts.  The individual fibers are specified to allow for transmission of 100 Gbps.  The fiber path from surface to underground is documented in drawing U1-FD-T-301.  Include this drawing???

The redundant fiber cable runs of 96 fiber pairs are received in existing communications enclosures at the 4850L entrances of the Ross and Yates shafts.  The two 96 pair fiber bundles are next routed to a Communication Distribution Room (CDR).  The fibers will be received and terminated in an optical fiber rack. Two additional network racks will be used to route the fiber data between the surface and underground.

The plan is to use only the Ross shaft set of 96 fiber pairs with the Yates set being redundant.  The network switches will allow for switchover is case of catastrophic failure of fibers in the Ross shaft.  Plans are being formed to periodically test the redundant Yates path and verify its viability.

From the CDR, 6 fiber pairs are routed to provide general network connections to the detector caverns and central utilty cavern.

The remaining 90 fiber pairs are all routed to the underground DAQ room for use by the DUNE experiment and LBNF. The fibers are reserved as follows:
* 15 pairs for DUNE data per detector ? total 60 pairs
* 1 pair for slow controls per detector ? total 4 pairs
* 2 pairs reserved for \dword{gps}
* 6 pairs for FSCF
* 4 pairs for LAr cryogenics
* 4 pairs for LN2 cryogenics
The set of reserved fiber pairs total to 80, the remaining 10 pairs are available for assignment or as spares.
	
   



\section{Central Utility Cavern Control and DAQ Rooms}
\label{sec:fdsp-coord-cuc-daq}

The \dword{cuc} contains various cryogenic equipment and the \dword{daq} and Control
Room for the \dword{dune} experiment.  The cryogenic system and areas are described in Section~\ref{sec:fdsp-coord-cryogenics} . Both the control and DAQ rooms are in the west
end of the \dword{cuc} (see Figure~\ref{fig:dune-cuc}).  
\begin{dunefigure}[DAQ and control rooms in CUC]{fig:dune-cuc}
  {Location of underground DAQ and control rooms in the CUC.}
  \includegraphics[width=0.85\textwidth]{Location_Underground_DAQ_Control_Rooms.pdf}
\end{dunefigure}
Figure~\ref{fig:dune-daq} shows the layout and suggested outfitting of the rooms.
\begin{dunefigure}[DAQ and control rooms]{fig:dune-daq}
  {Underground DAQ and control rooms layout.}
  \includegraphics[width=0.85\textwidth]{Preliminary_Layout_DAQ_Control_Rooms.pdf}
\end{dunefigure}


The Control Room is an underground space of approximately 18 x 48 feet and serves multiple purposes.   It provides a meeting or work space where
five to ten people can sit with laptops during system commissioning.
It is not truly a control room from which the
experiment will be run.  The DUNE experiment will have a remote control room located at Fermilab?s main campus.  Personnel involved in commissioning the experiment are expected to be working in the Detector Caverns, not the CUC.  If DAQ equipment need to be serviced, an experience and trained group of technicians will be able to work with remote collaborators to address issues in the DAQ room.

Additionally, the Control Room provides the required workstation for
monitoring of Fire and Life Safety, and the building management system.  These are facility services for which DUNE is not directly responsible for, however, the experiment will need to interface with these systems.  One example of this interface would be the reporting of any smoke detected within a detector rack.

Lastly, the cryogenic team requires a space allocation within the Control Room of two racks and two work benches for the technicians who monitor the cryogenic systems.  Checking with David/Mark on long term operations plan ? no answer yet.
       
The \dword{daq} room is approximately 26 x 56 feet and will contain between 52 to 60 racks that will be
used for fiber optic cable distribution, networking, \dword{dune}
\dword{daq} and the requirements of conventional facilities.  The current design, shown in Figure YYY, shows a total of 60 racks are possible.

\begin{dunefigure}[DAQ room layout]{fig:dune-DAQ_layout}
  {Proposed rack layput in DAQ room.}
  \includegraphics[width=0.85\textwidth]{Prop_DAQ_room_layout.png}
\end{dunefigure}
  

A quantity of 48 racks are reserved exclusively for \dword{daq}.  Two additional racks are required for optical fiber distribution and network connection to the surface.

Conventional facilities will supply the \dword{daq} room with cooling
water, an 18~inch raised floor, lighting, HVAC, dry fire protection
and dedicated 750kVA transformer.  \Dword{tc} will be responsible for
the final outfitting and layout of the room.  This includes the layout
and installation of water cooled racks, design and installation of
piping that carries cooling water to the racks, the AC power
distribution to the racks and installation of cable trays.

\section{Surface Rooms}
\label{sec:fdsp-coord-surf-rooms}

The \dword{dune} experiment requires space on the surface for a small number of DAQ, networking, and fiber optic distribution racks.  Space is also allocated for cryogenics.  The surface cryogen building and operations will be described in section XXX (David M.).

The \dword{daq} consortium requires a surface computer room with eight
racks and a minimum of 50kVA of power.  They also require connection
to the optical fibers running to the 4850L via the Ross and Yates shafts as well as to the Energy Sciences Network (ESnet).

The surface DAQ, networking equipment, and fiber distribution racks will be placed in a new Main Communications Room (MCR) in Ross Dry Building.  The MCR is approximately 628 square feet and will be completed as part of the LBNF project, with seven racks installed for the conventional facilities and space allocated for eight racks provided by the experiment.  The seven racks allocated for conventional facilities will include networking and fiber optic distribution.  The eight racks allocated to the experiment will contain computer servers, disk buffer and some network connections.  Power and cooling will be provided as part of the LBNF project.

\begin{dunefigure}[Surface rooms layout]{fig:dune-surface_layout}
  {Ross area surface rooms layout.}
  \includegraphics[width=0.85\textwidth]{Surface_rooms_layout.png}
\end{dunefigure}




\section{\dword{dune} Detector Safety System}
\label{sec:fdsp-coord-det-safety}

STILL WORKING ON RE-WRITING THIS SECTION

The \dword{ddss} functions to protect experimental equipment.  The
system must detect abnormal and potentially harmful operating
conditions.  It must recognize when conditions are not within the
bounds of normal operating parameters and automatically take
pre-defined protective actions.


The detector safety system must communicate to the \surf \dword{firus}
and the \dword{dune} slow controls system, which monitors detector
status.  Working together through communication links, these three
systems can monitor the status of the experiment, protect equipment
and provide life safety. Figure~\ref{fig:dune-DDSS} indicates how
these systems interact. For addtional information regarding the
facility saftey systems see
Section~\ref{sec:fdsp-coord-host_facility_services}.
\begin{dunefigure}[\dword{ddss}]{fig:dune-DDSS}
  {\dword{ddss} block diagram.}
  \includegraphics[width=0.85\textwidth]{DSS_Block_Diagram.pdf}
\end{dunefigure}


The \dword{ddss} will be implemented through hardware interlocks and a
\dword{plc}.  Listed below are some of \dword{dune} experimental
conditions that require intervention by the \dword{ddss}:
\begin{enumerate}
 \item A drop in the \dword{lar} level.  This condition requires a hardware
   interlock on the liquid level.  If the level drops below a
   pre-determined level, the drift high voltage must be automatically 
   shut off to prevent equipment damage.  Slow controls would be
   alerted through normal monitoring.
 \item Smoke or a temperature/humidity increase above normal operating
   levels. This is detected inside a rack or near an instrumented
   feedthrough.  If either of these conditions is detected, local
   power must be switched off. If smoke is detected, a
   dedicated line will alert \dword{firus}.
 \item A water leak detected near energized equipment in the \dword{daq}
   underground data processing room.  Water leak detectors 
   report to the \dword{ddss} \dword{plc} and a decision will be made to either
   issue an alert or immediately shut power down to the room depending
   on the alert level.  This condition would also be reported
   to \dword{firus}.
\end{enumerate}



\section{LBNF/SURF Safety System}
\label{sec:fdsp-coord-surf-safety}

