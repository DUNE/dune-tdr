\chapter{Technical Coordination and Management at Facilities}
\label{vl:tc-facility_mgmt}

\dword{dune} \dword{tc} manages the teams at the integration sites and
at \surf. These teams consist of personnel from the consortia
(e.g., students and postdocs, among others) and from \dword{tc} (e.g., technicians and
engineers). \fixme{This list needs a name.} 
\begin{enumerate}
 \item Technical coordination of activities \surf: This section
   includes what \dword{tcoord} provides and interaction with consortia work
   at \surf.
 \item Technical coordination of activities \dword{itf}: This section includes
   what TC provides and how it interfaces with consortia work at \dword{itf}.
 \item Technical coordination of activities at Ash River installation
   test site: This section includes what \dword{tcoord} provides and how it
   interfaces with consortia work at Ash River. 
\end{enumerate}



Two management teams are responsible for coordinating the
detector installation and integration both underground and at the
surface. The group responsible for activities in the underground areas
is referred to as the \dword{uit}.  On the
surface, the \dword{sit} coordinates the Logistics
Center, where all \dword{dune} shipping and receiving is initiated, and the
\dword{itf}, where the 
\dword{apa}s are integrated.  These installation teams include management, safety
officer, riggers, and equipment operators, all needed to move detector
components at the surface and underground. \surf and the \dword{cmgc} \fixme{The abbreviation should be capitalized but not the phrase.} are
responsible for moving the components from the headframe to the detector hall.

These two teams must work closely with the \dword{lbnf} logistics team at
\surf, \dword{cmgc}, the different consortia, and technical coordination to
ensure that all components required for the infrastructure and
detectors are well organized and scheduled for delivery. During
civil construction the shaft is controlled by the \dword{cmgc}; they manage all loads
underground.  The Logistics Center and \dword{itf} are
needed approximately 2 years before beneficial occupancy of the North
Cavern.  During this period, a buffer of \dword{apa}s must be developed 
and  the installation of the underground infrastructure must begin. This
increase in manpower helps develop a trained and organized team on the
surface that then merges with the team underground.


\section{Surface Facilities}

Because no materials or equipment can be  shipped directly to the Ross or
Yates headframe, the Logistics Center is used for both short and
long-term storage and re-packaging before anything is shipped underground.
The function of the Logistics Center is shown in
Figure~\ref{fig:orgchart_lc}.
\begin{dunefigure}[Logistics concept.]{fig:orgchart_lc}
  {Organization chart for the Logistics Center.}
  \includegraphics[width=0.65\textwidth]{logistics.png}
\end{dunefigure}
Everything is inventoried and
entered into the hardware database at the Logistics Center, and the Logistics Center is 
responsible for trucking between the \dword{itf} and \surf and for dunnage coming
back.  \dword{tpc} \fixme{The first letter should be capitalized in the previous phrase.} components that must go to the \dword{itf} arrive at the Logistics Center, are logged into the inventory
system, and then are transported to the \dword{itf} as they are needed. Once the integration is
complete \dword{apa} pairs are transported back to the Logistics Center ready
to be shipped underground.  The team of riggers and equipment operators
are trained professionals that move all the \dword{tpc} components and
infrastructure but also are trained to assist as needed with integration. The organization of the Logistics Center and \dword{itf} are shown in Figure~\ref{fig:orgchart_itf}.
\begin{dunefigure}[Org Chart for the Logistics Center and Integration and Testing Facility.]{fig:orgchart_itf}
  {Organization chart for the Logistics Center and Integration and Testing
    Facility.}  \includegraphics[width=0.65\textwidth]{lc.png}
\end{dunefigure}

\subsection{Logistics Center}

The Logistics Center is a shipping and handling facility with a
smaller crew than \dword{itf}, but it must operate 20 hours a day, 5 days a week.
This will require 2 rigging/operator crews that will be managed by a
joint management team working in concert with the \dword{itf} that includes 
a manager, two deputy supervisors (from the Logistics Center and \dword{itf}), an
administrative assistant, and a safety officer working under the direction
of Fermilab Safety.  Each shift would have a lead worker who
directs the crew of $\sim$5 FTE \fixme{FTE has not been defined and is not in the glossary.} responsible for the rigging and
equipment operation for the shipping and receiving.  Their tasks would
also include using the inventory system to track all materials
coming in and out of the Logistics Center. Some repackaging will be
required to fit materials in the cage or on a slung load.  Management must work directly with the \dword{lbnf} logistics team at \surf and
\dword{cmgc} so that loads to the headframe arrive as scheduled.  The headframe has
room for only 1 or 2 trucks, so proper scheduling is
critical.

\subsection{Integration and Testing Facility}

The Integration and Testing Facility is where the integration of the
\dword{apa}, \dword{ce}, and \dword{pd} are completed. While the
consortia does most of this technical work and testing,  a team of riggers and equipment operators must still move the \dword{apa}s
on and off the trucks and around the cleanroom.  The  \dword{apa}s are
built at a rate that means the \dword{itf} must start integration approximately 2
years before the detector is assembled.  This will also serve as a
training period for \dword{uit} staff as things
start to progress underground.  Estimated work force at the \dword{itf} is a
deputy supervisor, a lead worker, and a crew of 3 to 4 riggers and equipment
operators.  This crew also helps as needed with the integration of the
\dword{apa}s to minimize wasted time.

The Logistics Center and \dword{itf} management work directly with the
consortia to ensure that the components are in the proper place when
needed using the inventory system.
  

\section{Underground Facilities}

\surf/\dword{cmgc} will move the \dword{tpc} materials from the
headframe to the three caverns; once underground, the materials are handled by the \dword{uit}. The responsibilities of the 
\dword{uit} is more complex because of the material
movement involved:
\begin{itemize}
 \item North, Central, and South Caverns: \dword{tpc} components and
   infrastructure materials can be stored in the available cavern
   space during the construction of detectors 1 and 2.  The \dword{uit} team will
   use forklifts, carts, and the cavern crane to move
   materials into the clean room.
 \item Inside the \dword{sas} and clean room: The main tasks are manipulating \dword{apa}
   onto the different \dword{apa} towers and \dword{cpa} component
   assembly.  The team operates the shuttle beam and crane, opens and closes the
   cold boxes, and moves materials with forklift, carts, and motorized
   pallet jacks.
 \item Inside the Cryostat: The \dword{uit} team installs the \dword{dss} and detector
   subfloor; they then move the completed \dword{apa} and \dword{cpa} pairs into
   position.  They also deploy the \dword{fc}.
\end{itemize}

These teams work four 10-hour day and afternoon shifts.  Depending on
available cage trips for shift movement, lead workers,
safety officer, and management would have overlapping shifts to insure 
proper transfer of knowledge.  Typically, this would happen at the daily
toolbox safety meeting and work assignment updates.  The safety
officer on each shift would be responsible for any discussion of safety
during the meeting but also for training all workers, including those from
the consortia. The safety officer management goes directly to the \surf
and \dword{lbnf} ES\&H \fixme{This abbreviation has not been defined and is not in the glossary. If this is the only time the abbreviation is used, or if it is used only a few times, do not abbreviate it.} to minimize any conflict of interest although 
all \dword{uit} and consortia personnel have the right to stop
work for any safety issues.

\subsection{Underground Facilities Management}

Figure~\ref{fig:orgchart_uit} shows management of the \dword{uit}. 
\begin{dunefigure}[Org Chart for the \dword{uit}.]{fig:orgchart_uit}
  {Organization chart for \dword{dune} \dword{uit}.}
  \includegraphics[width=0.5\textwidth]{uit.png}
\end{dunefigure}
\dword{uit}and consists of three levels:
\begin{itemize}
  \item Underground Installation Manager: The \dword{uit} manager oversees
    both shifts but also covers for the day or night shift deputy
    manager if needed.  This manager is the contact person working directly
    with the logistics manager and \surf/\dword{cmgc} to organize cage
    trips for materials needed underground, attending all high-level
    meetings as required, and submitting weekly reports.  The \dword{uit} manager schedules and
    organizes the work along with the consortia to ensure they have
    the rigging crews at the appropriate times.
  \item Day/Night Deputy Installation Supervisors: The deputy
      supervisors are working managers, trained
      riggers and equipment operators who can fill in as needed.  They
      are trained in all installation procedures, working
      directly with the consortia to keep the project 
      on schedule.  If the lead worker is sick or on vacation, the deputy supervisors fill in. Communication among the deputy supervisors between
      shifts is critical for a smooth shift change.
  \item Lead Workers: The lead workers typically are the main
    equipment operators and direct the individual teams, which typically comprise 2 or 3
    riggers or equipment operators.  The lead workers are trained in all
    installation procedures and help the consortia as needed for
    each task.  Two lead workers per shift shift as
    needed between the main cavern, materials \dword{sas}, clean room, and
    inside and on top of the cryostat.
  \item Riggers and Equipment Operators: Riggers and equipment
    operators are trained in both roles; they
    can run the cranes, forklifts, and other equipment as needed.  They
    work in two person teams at a minimum, more if a
    task requires it. Typical \dword{apa} movement requires at least three workers and a
    spotter. They are also well trained in all installation
    procedures and can help as needed.
\end{itemize}

\section{Ash River}

The trial assembly work at Ash River has three major phases of \dword{dune}
mechanical tests to confirm designs, particularly if the design has been modified from \dword{protodune}.  The \dword{nova} Far Detector Laboratory is managed by
the University of Minnesota and is funded through an
operations grant from Fermilab.  We follow both university and Fermilab
safety regulations, whichever is more stringent.  The university code
officials approve all building permits, which must include engineered
drawings signed by an engineer registered in Minnesota. All hazard
analyses and work procedure documents are approved by a joint safety
committee with members drawn from both the University of Minnesota and Fermilab and may include specialists as needed.

The work at Ash River has three main goals: verify that
the \dword{dune} \dword{tpc} can be installed as safely and efficiently as possible; complete a set of procedural documents as the basis of
documentation for work underground; and serve as a training center for new
hires at \surf so underground procedures are well understood.

There is a full time staff of five FTEs \fixme{This abbreviation does not appear in the glossary. I've seen it only once, so it may not be a necessary abbreviation} at Ash River. A manager,
deputy manager and 3 experienced technicians.  One of the technicians
is also our safety officer and is the chairperson of the joint safety
committee.


Phase 0: This work has started now.  A vertical cable test of two
full-scale APA side tubes have been mounted to a column in the
lab. Using a complete cable bundle we will test how well the conduit
system works and what modifications need to be made. We have also
completed a series of different mounting schemes for the ground plane
on the field cage.

Phase 1: We will construct an APA tower using a steel frame large
enough to hold a commercial stair scaffold in the middle as can be
seen in Figure ~\ref{fig:ashriver1}.
\begin{dunefigure}[View of the Ash River Phase 1 assembly.]{fig:ashriver1}
  {Phase 1 Trial Assembly APA Tower at the NOvA Far Detector Lab in Ash River.}
  \includegraphics[width=0.65\textwidth]{phase1.jpg}
\end{dunefigure}
This APA tower will be designed so we can
use it for joining the top and bottom APAs together and all the
cabling. Procedures for removing the cable conduits and replacing
failed Photon Detectors will also be developed.

Phase 2: A more complex structure, which mocks up the shuttle
beam/crane in the cleanroom will be needed. A schematic can be seen in
Figure~\ref{fig:ashriver2}.
\begin{dunefigure}[View of the cleanroom for the \dword{dsp} detector \#1 showing the shuttle beam system]{fig:ashriver2}
  {View of the cleanroom for the \dword{dsp} detector \#1 showing the shuttle beam system}
  \includegraphics[width=0.65\textwidth]{phase2.jpg}
\end{dunefigure}
This is a very similar to the existing
design of the shuttle beam in the cryostat.  As we proceed with the
phase 1 tests, we will adapt the cleanroom rail system or cryostat
rail system to allow full movement of the APA and CPA pairs. We would
decide how best to allow us to do a full-scale test of the rest of the
installation procedures using this structure including:
\begin{itemize}
 \item Installation of the Detector Support System (DSS)
 \item Transfer of TPC component through the TCO
 \item Moving APA and CPA pairs into ``final'' position and deploying the field cages
 \item Cabling of the APA through the cryostat penetrations
 \item Installation of the end walls
 \item Potentially deployment of the Dual Phase detector
\end{itemize}
