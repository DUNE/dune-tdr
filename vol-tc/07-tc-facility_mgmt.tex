\chapter{Technical Coordination and Management at Facilities}
\label{vl:tc-facility_mgmt}

[Bill]

This needs to include how we manage teams from consortia
e.g. students, equipment, etc. We need to say what is provided by TC,
e.g. techs, engineers, tools, etc.
\begin{enumerate}
 \item Technical coordination of activities SURF: This section includes what TC provides and how it interfaces with consortia work at SURF.
 \item Technical coordination of activities ITF: This section includes what TC provides and how it interfaces with consortia work at ITF.
 \item Technical coordination of activities at Ash River installation
   test site: This section includes what TC provides and how it
   interfaces with consortia work at Ash River (Should this go into
   APA? Or into Single Phase Installation and Integration?)
\end{enumerate}

Installation Coordination and Support (also called simply
\textit{Installation}) is responsible for coordinating the detector
installations, providing detector installation support and providing
installation-related infrastructure. The installation group management
responsibility is shared by a scientific lead and a technical lead
that report to the Technical Coordinator. The group responsible for
activities in the underground areas is referred to as the
\dword{uit}. The \dword{itf} group, which delivers equipment to the
Ross Shaft, and the \dword{uit}, which receives the equipment
underground, need to be in close communication and work closely
together.

Underground installation is in general responsible for coordinating
and supporting the installation of the \dwords{detmodule} and
providing necessary infrastructure for installation of the
experiment. While the individual consortia are responsible for the
installation of their own detector equipment, it is essential that the
detector installation be planned as a whole and that a single group
coordinates the installation and adapts the plans throughout the
installation process. The \dword{uit} has responsibility for overall
coordination of the installations. In conjunction with each consortium
the \dword{uit} makes the installation plan that describes how the
\dwords{detmodule} are to be installed. The installation plan is used
to define the spaces and infrastructure that will be needed to install
them.

The installation plan will also be used to define the interfaces
between the Installation group and the consortium deliverables.

\subsection{UIT Infrastructure}

The installation scope includes the infrastructure needed to install
the \dword{fd} such as the cleanroom, a small machine shop, special
cranes, scissor lifts, and access equipment.  Additional equipment
required for installation includes: rigging equipment, hand tools,
diagnostic equipment (including oscilloscopes, network analyzers, and
leak detectors), local storage with some critical supplies and some
personal protective equipment (PPE). The \dword{uit} will also provide
trained personnel to operate the installation infrastructure. The
consortia will provide the detector elements and custom tooling and
fixtures as required to install their detector components.
