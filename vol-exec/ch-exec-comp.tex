%\ifdefined\isfinal\documentclass[final]{dune}\else\documentclass{dune}\fi
%\pdfoutput=1            % must be in first 5 lines so arXiv finds it
%\graphicspath{ {graphics/} {executive-summary/graphics/}{generated/} }
%% * <aheavey@fnal.gov> 2018-04-06T17:26:26.700Z:
%%
% ^.
%% stuff before document begins.  

% Most packages are required through dune.cls.  Put hyper-ref related here to assure it's last.
%%% xr won't play nice
% \usepackage{xr-hyper}
\usepackage[pdftex,bookmarks,hidelinks]{hyperref}

\graphicspath{ {graphics/} }

% make some "if"s for DP and SP.
% They should be set true inside DP or SP volume or chapter mains.
% set try like \dptrue or \sptrue.
% they can be referenced with \ifdp or \ifsp and terminated with \fi.
\newif\ifdp
\newif\ifsp




% This holds definitions of macros to enforce consistency in names.

% This file is the sole location for such definitions.  Check here to
% learn what there is and add new ones only here.  

% also see units.tex for units.  Units can be used here.

%%% Common terms

% Check here first, don't reinvent existing ones, add any novel ones.
% Use \xspace.

%%%%% Anne adding macros for referencing TDR volumes and annexes Apr 20, 2015 %%%%%
\def\expshort{DUNE\xspace}
\def\dune{\expshort}
\def\explong{The Deep Underground Neutrino Experiment\xspace}

%\def\thedocsubtitle{LBNF/DUNE Technical Design Report (DRAFT)}
\def\thedocsubtitle{Deep Underground Neutrino Experiment (DUNE)} 
%\def\tdrtitle{Technical Proposal}
\def\tdrtitle{DRAFT Technical Design Report}


% All volume titles and numbers in one place.
\def\voltitleexec{Executive Summary\xspace}
\def\volnumberexec{1}

\def\voltitlephysics{DUNE Physics\xspace}
\def\volnumberphysics{2}

\def\voltitlesp{Single-Phase Far Detector Module\xspace}
\def\volnumbersp{3}

\def\voltitledp{Dual-Phase Far Detector Module\xspace}
\def\volnumberdp{4}

\def\voltitletc{Technical Coordination\xspace}
\def\volnumbertc{5}

\def\voltitlend{Near Detector\xspace} % not part of TDR, a chapter goes in Exec Summ
\def\volnumbernd{6}

\def\voltitleswc{Software and Computing\xspace} % not part of TDR, a chapter goes in Exec Summ
\def\volnumberswc{7}

% see~\refsec{exec}{2.3}
\newcommand{\refsec}[2]{Volume~\csname volnumber#1\endcsname \xspace Section~#2}
% see~\refch{exec}{2}
\newcommand{\refch}[2]{Volume~\csname volnumber#1\endcsname \xspace Chapter~#2}
% see Table~\refinch{exec}{1.2}
\newcommand{\refinch}[2]{#2 in Volume~\csname volnumber#1\endcsname \xspace}

\newcommand{\bigo}[1]{\ensuremath{\mathcal{O}(#1)}}


% Things about oscillation
%
\newcommand{\numu}{\ensuremath{\nu_\mu}\xspace}
\newcommand{\nue}{\ensuremath{\nu_e}\xspace}
\newcommand{\nutau}{\ensuremath{\nu_\tau}\xspace}

\newcommand{\anumu}{\ensuremath{\bar\nu_\mu}\xspace}
\newcommand{\anue}{\ensuremath{\bar\nu_e}\xspace}
\newcommand{\anutau}{\ensuremath{\bar\nu_\tau}\xspace}

\newcommand{\dm}[1]{\ensuremath{\Delta m^2_{#1}}\xspace} % example: \dm{12}

\newcommand{\sinst}[1]{\ensuremath{\sin^2\theta_{#1}}\xspace} % example \sinst{12}
\newcommand{\sinstt}[1]{\ensuremath{\sin^22\theta_{#1}}\xspace}  % example \sinstt{12}

\newcommand{\deltacp}{\ensuremath{\delta_{\rm CP}}\xspace}   % example \deltacp
\newcommand{\mdeltacp}{\ensuremath{\delta_{\rm CP}}}   %%%%%%%%%%  <--- missing something; what's the m for?

\newcommand{\nuxtonux}[2]{\ensuremath{\nu_{#1} \to \nu_{#2}}\xspace}  % example \nuxtonux23 (no {...} )
\newcommand{\numutonumu}{\nuxtonux{\mu}{\mu}}
\newcommand{\numutonue}{\nuxtonux{\mu}{e}}
% Add chi sqd MH?  avg delta chi sqd?

% atmospheric neutrinos and PDK
\newcommand{\ptoknubar}{\ensuremath{\rightarrow K^+ \overline{\nu}}\xspace}
\newcommand{\ptoepizero}{\ensuremath{p^+ \rightarrow e^+ \pi^0}\xspace}



% Isotopes - stay here
\def\argon40{${}^{40}$Ar}       
\def\Ar39{$^{39}$Ar}
\def\Cl40{$^{40}$Cl}
\def\K40{$^{40}$K}
\def\B8{$^{8}$B}
\newcommand\isotope[2]{\textsuperscript{#2}#1} % use as, e.g.,: \isotope{Si}{28}

% Parameters common to SP DP
\def\ndfromtarget{\SI{574}{\meter}\xspace} % ND from target
\def\fdfiducialmass{\SI{40}{\kt}\xspace}
\def\driftvelocity{\SI{1.6}{\milli\meter/\micro\second}\xspace} % same for sp and dp?
\def\lartemp{\SI{88}\,K\xspace}
\def\larmass{\SI{17.5}{\kt}\xspace} % full mass in cryostat
\def\cryostatht{\SI{17.8}{\meter}\xspace} % outer height of cryostat (Jim Stewart 5/2/19)
\def\cryostatlen{\SI{65.8}{\meter}\xspace} % length of cryostat (Jim Stewart 5/2/19)
\def\cryostatwdth{\SI{18.9}{\meter}\xspace} % width of cryostat (Jim Stewart 5/2/19)
\def\nominalmodsize{\SI{10}{kt}\xspace} % nominal module size 10 kt
\def\dunelifetime{\SI{20}{years}\xspace} % nominal operational life time of DUNE experiment
\def\pipiibeampower{\SI{1.2}{MW}\xspace} 
\def\cooldown{cool-down\xspace} % standardize w/ or w/o space or hyphen

% Parameters SP
\def\spmaxfield{\SI{500}{\volt/\centi\meter}\xspace} % SPfield strength
\def\spactivelarmass{\SI{10}{\kt}\xspace} % active mass in cryostat
\def\spmaxdrift{\SI{3.52}{\m}\xspace}
\def\tpcheight{\SI{12.01}{\meter}\xspace} % height of SP TPC, APA, CPA and of DP TPC
\def\sptpclen{\SI{58.17}{\meter}\xspace} % length of SP TPC, APA, CPA
\def\apacpapitch{\SI{2.32}{\meter}\xspace} % pitch of SP CPAs and APAs
\def\spfcmodlen{\SI{3.5}{\m}} % length of SP FC module
\def\spnumch{\num{384000}\xspace} % total number of APA readout channels 
\def\spnumpdch{\num{6000}\xspace} % total number of PD readout channels 
\def\planespace{\SI{4.75}{\milli\meter}\xspace}
\def\sptargetdriftvoltpos{\SI{180}{kV}\xspace} % target drift voltage - positive
\def\coldbox{cold box\xspace} % standardize w/ or w/o space or hyphen
\def\Coldbox{Cold box\xspace} % standardize w/ or w/o space or hyphen
\def\endwall{end wall\xspace} % standardize w/ or w/o space

% Parameters DP
\def\dpactivelarmass{\SI{12.096}{\kt}\xspace} % active mass in cryostat
\def\dpfidlarmass{\SI{10.643}{\kt}\xspace} % fiducial mass in cryostat
\def\dpmaxdrift{\SI{12}{\m}\xspace} % max drift length
\def\dptpclen{\SI{60}{\meter}\xspace} % length of TPC
\def\dptpcwdth{\SI{12}{\meter}\xspace} % width of TPC
\def\dpswchpercrp{\num{36}\xspace} % number of anode/lem sandwiches per CRP 
\def\dpnumswch{\num{2880}\xspace} % total number of anode sandwiches in module
\def\dptotcrp{\num{80}\xspace} % total number of CRPs in module
\def\dpchpercrp{\num{1920}\xspace} %  channels per CRP
\def\dpnumcrpch{\num{153600}\xspace} % total number of CRP channels in module
\def\dpchperchimney{\num{640}\xspace} %  channels per chimney  --CRP channels?
\def\dpnumpmtch{\num{720}\xspace} % number of PMT channels
\def\dpstrippitch{\SI{3.125}{\milli\meter}\xspace} % pitch of anode strips
\def\dpnumfcmod{\num{244}\xspace} % number of FC modules
\def\dpnumfcres{\num{240}\xspace} % number of FC resistors
\def\dpnumfcrings{\num{60}\xspace} % number of FC rings
\def\dpnominaldriftfield{\SI{500}{\volt/\cm}\xspace} % nominal drift voltage per cm
\def\dptargetdriftvoltpos{\SI{600}{\kV}\xspace} % target drift voltage - positive
\def\dptargetdriftvoltneg{\SI{-600}{\kV}\xspace} % target drift voltage - negative

% Nominal readout window time
%% SP has 2.25ms drift time.  The readout is 2*dt + 20%*dt extra.
\def\spreadout{\SI{5.4}{\ms}\xspace}
%% DP has 7.5 ms drift time.  The same (over generous) rule gives 16.5ms
\def\dpreadout{\SI{16.5}{\ms}\xspace}
% Supernova Neutrino Burst buffer and readout window time
\def\snbtime{\SI{100}{\s}\xspace}
% interesting amount of time we might have SNB neutrinos but not yet
% enough to trigger.
\def\snbpretime{\SI{10}{\s}\xspace}
% SP SNB dump size. MUST KEEP THIS MANUALLY IN SYNC 1.5 TB/s * \snbtime
\def\spsnbsize{\SI{45}{\TB}\xspace}

% keep these three numerically in sync
\def\offsitepbpy{\SI{30}{\PB/\year}\xspace}
\def\offsitegbyteps{\SI{1}{\GB/\s}\xspace}
\def\offsitegbps{\SI{8}{\Gbps}\xspace}
\def\surffnalbw{\SI{100}{\Gbps}\xspace}



% New from Anne March/April 2018
%physics terms
\newcommand{\efield}{E field\xspace}
\newcommand{\Lbl}{Long-baseline\xspace}
\newcommand{\rms}{RMS\xspace} % Might want this small caps?
\newcommand{\threed}{3D\xspace}
\newcommand{\twod}{2D\xspace}



% Top-level requirements and specifications
% 1 Minimum drift field
\def\mindriftfield{\SI{250}{\volt/\cm}\xspace}
\def\mindriftfieldgoal{\SI{500}{\volt/\cm}\xspace}
% 2 FE elec noise
\def\elecnoisefe{ \num{1000} enc\xspace}
% 3 light yield
\def\lightyield{\SI{0.5}{pe/\MeV}\xspace}
\def\lightyieldgoal{\SI{5}{pe/\MeV}\xspace}
% 4 time resolution
\def\timeres{\SI{1}{\micro/\second}\xspace}
\def\timeresgoal{\SI{100}{ns}\xspace}
% 5 LAr purity
\def\larpurity{\SI{100}{ppt}\xspace}
\def\larpuritygoal{\SI{30}{ppt}\xspace}
% 6 APA gaps
\def\apagapsame{\SI{15}{\milli\m}\xspace}
\def\apagapdiff{\SI{30}{\milli\m}\xspace}
% 7 drift field uniformity (from component positioning)
\def\fielduniformity{\SI{1}{\%}\xspace}
% 8a APA collection wire angle
\def\apacollwireangle{$\SI{0}{^\circ}$\xspace}
% 8b APA induction wire angle
\def\apainducwireangle{$\pm\SI{35.7}{^\circ}$\xspace}
% 9a APA wire pitch - U,V
\def\uvpitch{\SI{4.669}{\milli\meter}\xspace}
% 9b APA wire pitch - X, G
\def\xgpitch{\SI{4.790}{\milli\meter}\xspace}
% 10 APA wire position tolerance
\def\wirepitchtol{$\pm$\SI{0.5}{\milli\meter}\xspace}
% 11 drift field uniformity (from HVS)
\def\fielduniformityhv{\SI{1}{\%}\xspace}
% 12 HV PS ripple contrib to noise
\def\hvripplenoise{\SI{100}{enc}\xspace}
% 13 FE peaking time
\def\fepeaktime{\SI{1}{\micro\second}\xspace}
% 14 signal saturation level (SP)
\def\spsignalsat{\num{500000} electrons\xspace}
% 15 LAr N contamination
\def\nitrogencontam{\SI{25}{ppm}\xspace}
% 16 detector dead time
\def\deadtime{\SI{0.5}{\%}\xspace}
% Engineering
% 17 Cathode resistivity
\def\cathodemegohm{\SI{1}{\mega\ohm/square}\xspace}
\def\cathodegigohm{\SI{1}{\giga\ohm/square}\xspace}
% 
%\def\{\xspace}
% 19 ADC sampling frequency
\def\samplingfreq{\SI{2}{\mega\hertz}\xspace}
% 20 ADC dynamic range
\def\adcdynrange{\num{12} bits}  %{3000}:\num{1}\xspace}
\def\adcdynrangegoal{\num{13} bits} %{4070}:\num{1}\xspace}
% 21 CE power consumption (SP)
\def\cepower{\SI{50}{mW/channel}\xspace}
% 22 data to tape
\def\dataratetotape{\SI{30}{PB/year}\xspace}
% 23 SNB trigger
\def\snbtriggereff{90\% efficiency\xspace}
%\def\snbtriggervisenergy{90\% efficiency \xspace}
% 24 local E fields
\def\localefield{\SI{30}{\kV/\cm}\xspace}
% 25 non-FE noise contributions
\def\elecnoisenonfe{$\ll$ \SI{1000}{enc}\xspace}
% 26 impurity contrib from components
\def\larpuritycomps{$\ll$ \SI{30}{ppt}\xspace}
% 28 dead channels
\def\deadchannels{\SI{1}{\%}\xspace}

\newcommand{\fdth}{feedthrough\xspace} % ok not in gloss
\newcommand{\phel}{photoelectron\xspace} % ok not in gloss
\newcommand{\frfour}{FR-4\xspace} % used in gloss and sp hv chap


% The following from phys ch-bsm 1/3/19 (was in their cls file)
\newcommand{\lsim}{{\;\raise0.3ex\hbox{$<$\kern-0.75em\raise-1.1ex\hbox{$\sim$}}\;}}
\newcommand{\gsim}{{\;\raise0.3ex\hbox{$>$\kern-0.75em\raise-1.1ex\hbox{$\sim$}}\;}}
\newcommand{\beq}{\begin{equation}}
\newcommand{\eeq}{\end{equation}}
\newcommand{\bea}{\begin{eqnarray}}
\newcommand{\eea}{\end{eqnarray}}
\newcommand{\DF}{\Delta_{4}}
\mathchardef\minus="002D
\newcommand{\dk}[1]{\textcolor{red}{#1}}
\newcommand{\dkc}[1]{\textbf{\textcolor{red}{(#1 --DK)}}}
\newcommand{\dd}[1]{\textcolor{blue}{#1}}

%Milestones (from Eric's talk Mar 12, 2019} https://indico.fnal.gov/event/20149/contribution/0/material/slides/1.pdf
\newcommand{\startpduneiispinstall}{March 2021\xspace}% Start of ProtoDUNE-II (SP) Installation: March 2021
\newcommand{\startpduneiidpinstall}{March 2022\xspace}%Start of ProtoDUNE-II (DP) Installation: March 2022
\newcommand{\sdlwavailable}{April 2022\xspace}%South Dakota Logistics Warehouse Available: April 2022
\newcommand{\cucbenocc}{October 2022\xspace}% Beneficial Occupancy of Cavern 1/CUC: October 2022
\newcommand{\accesscuccountrm}{April  2023\xspace}% CUC Counting Room Accessible: April 2023
\newcommand{\accesstopfirstcryo}{January 2024\xspace}%Top of Far Detector #1 Cryostat Accessible: January 2024
\newcommand{\startfirsttpcinstall}{August 2024\xspace}%Start of Far Detector #1 TPC Installation: August 2024
\newcommand{\firsttpcinstallend}{May 2025\xspace}% End of Far Detector #1 TPC Installation: May 2025
\newcommand{\accesstopsecondcryo}{January 2025\xspace}%Top of Far Detector #2 Accessible: January 2025
\newcommand{\startsecondtpcinstall}{August 2025\xspace}%Start of Far Detector #2 TPC Installation: August 2025
\newcommand{\secondtpcinstallend}{May 2026\xspace}%End of Far Detector #2 TPC Installation: May 2026

%Mike Kordosky: the command below is used to refer to
% planned entries in the requirements table.
\newcommand{\rrt}[1]{\ifthenelse{\equal{#1}{}}{[RT:TBD]}{[RT:#1]}}

%%%%%%% Everything below here is DEPRECATED (4/30/19 AH) %%%%%%%%%%

% Names of expts or detectors -- all these go into glossary DEPRECATED
% in use
\newcommand{\cherenkov}{Cherenkov\xspace}  %nonaccel
\newcommand{\kamland}{KamLAND\xspace} %cisc dp
\newcommand{\superk}{Super--Kamiokande\xspace} %nonaccel, bsm
\newcommand{\hyperk}{Hyper--Kamiokande\xspace} %nonaccel
\newcommand{\microboone}{MicroBooNE\xspace} %used in cisc sp, hv, daq
\newcommand{\minerva}{MINER$\nu$A\xspace} %tools/meth, nuosc11
\newcommand{\nova}{NO$\nu$A\xspace} %lots
\newcommand{\lariat}{LArIAT\xspace} % calib,dppds
\newcommand{\argoneut}{ArgoNeuT\xspace} %nonaccel
%not in use
\newcommand{\kkande}{Kamiokande\xspace}  
\newcommand{\miniboone}{MiniBooNE\xspace}
\newcommand{\numi}{NuMI\xspace}
\newcommand{\larnd}{LAr ND\xspace}

% usage not checked for the rest 4/30/19
% Random -- all these go into glossary DEPRECATED
\newcommand{\lartpc}{LArTPC\xspace}
\newcommand{\globes}{GLoBES\xspace}
\newcommand{\larsoft}{LArSoft\xspace}
\newcommand{\snowglobes}{SNOwGLoBES\xspace}
\newcommand{\docdb}{DUNE DocDB\xspace}
\newcommand{\lbl}{long-baseline\xspace} %DEPRECATED

% also have in glossary; Glossary also has CERN, PSL, other big labs, etc. DEPRECATED
\newcommand{\fnal}{Fermilab\xspace} 
\newcommand{\surf}{SURF\xspace} 
\newcommand{\bnl}{BNL\xspace}
\newcommand{\anl}{ANL\xspace}
 
 %detectors and modules
% also have in glossary; THE FOLLOWING NINE TERMS ARE DEPRECATED 4/30/19
\newcommand{\detmodule}{detector module\xspace}
\newcommand{\dual}{DP\xspace}
\newcommand{\Dual}{DP\xspace}
\newcommand{\single}{SP\xspace}
\newcommand{\Single}{SP\xspace}
\newcommand{\dpmod}{DP detector module\xspace}
\newcommand{\spmod}{SP detector module\xspace}
\newcommand{\lar}{LAr\xspace}
\newcommand{\lntwo}{LN$_2$\xspace}  %used in sp-tpcelec 

%detector components SP and DP -- need to be in gloss; THESE 14 ITEMS DEPRECATED:
\newcommand{\dss}{DSS\xspace}
\newcommand{\hv}{high voltage\xspace}
\newcommand{\fcage}{field cage\xspace}
\newcommand{\fc}{FC\xspace}
\newcommand{\fcmod}{FC module\xspace}  %%%   don't need?
\newcommand{\topfc}{top FC\xspace}
\newcommand{\botfc}{bottom FC\xspace}
\newcommand{\ewfc}{endwall FC\xspace}
\newcommand{\pdsys}{PD system\xspace}
\newcommand{\phdet}{photon detector\xspace}
\newcommand{\sipm}{SiPM\xspace}
\newcommand{\pmt}{PMT\xspace}
\newcommand{\pwrsupp}{power supply\xspace}
\newcommand{\pwrsupps}{power supplies\xspace}
% This holds definitions of macros to enforce consistency in units.

% This file is the sole location for such definitions.  Check here to
% learn what there is and add new ones only here.  

% also see defs.tex for names.


% see
%  http://ctan.org/pkg/siunitx
%  http://mirrors.ctan.org/macros/latex/contrib/siunitx/siunitx.pdf

% Examples:
%  % angles
%  \ang{1.5} off-axis
%
%  % just a unit
%  \si{\kilo\tonne}
%
%  % with a value:
%  \SI{10}{\mega\electronvolt}

%  range of values:
% \SIrange{60}{120}{\GeV}

% some shorthand notation
%\DeclareSIUnit \MBq {\mega\Bq}
\DeclareSIUnit \s {\second}
\DeclareSIUnit \MB {\mega\byte}
\DeclareSIUnit \GB {\giga\byte}
\DeclareSIUnit \TB {\tera\byte}
\DeclareSIUnit \PB {\peta\byte}
\DeclareSIUnit \Mbps {\mega\bit/\s}
\DeclareSIUnit \Gbps {\giga\bit/\s}
\DeclareSIUnit \Tbps {\tera\bit/\s}
\DeclareSIUnit \Pbps {\peta\bit/\s}
\DeclareSIUnit \kton {\kilo\tonne} % changed  back to kton
\DeclareSIUnit \kt {\kilo\tonne}
\DeclareSIUnit \Mt {\mega\tonne}
\DeclareSIUnit \eV {\electronvolt}
\DeclareSIUnit \keV {\kilo\electronvolt}
\DeclareSIUnit \MeV {\mega\electronvolt}
\DeclareSIUnit \GeV {\giga\electronvolt}
\DeclareSIUnit \m {\meter}
\DeclareSIUnit \cm {\centi\meter}
\DeclareSIUnit \in {\inchcommand}
\DeclareSIUnit \km {\kilo\meter}
\DeclareSIUnit \kV {\kilo\volt}
\DeclareSIUnit \kW {\kilo\watt}
\DeclareSIUnit \MW {\mega\watt}
\DeclareSIUnit \MHz {\mega\hertz}
\DeclareSIUnit \mrad {\milli\radian}
\DeclareSIUnit \year {year}
\DeclareSIUnit \POT {POT}
\DeclareSIUnit \sig {$\sigma$}
\DeclareSIUnit\parsec{pc}
\DeclareSIUnit\lightyear{ly}
\DeclareSIUnit\foot{ft}
\DeclareSIUnit\ft{ft}
\DeclareSIUnit \ppb{ppb}
\DeclareSIUnit \ppt{ppt}
\DeclareSIUnit \samples{S}

\sisetup{inter-unit-product = \ensuremath{{}\cdot{}}}
%\def\ktyr{\si[inter-unit-product=\ensuremath{{}\cdot{}}]{\kt\year}\xspace}
\newcommand{\ktyr}{\si{\kt\year}\xspace}
%\def\Mtyr{\si[inter-unit-product=\ensuremath{{}\cdot{}}]{\Mt\year}\xspace}
\newcommand{\Mtyr}{\si{\Mt\year}\xspace}
%\def\msr{\si[inter-unit-product=\ensuremath{{}\cdot{}}]{\meter\steradian}\xspace}
\newcommand{\msr}{\si{\meter\steradian}\xspace}
%\def\ktMWyr{\si[inter-unit-product=\ensuremath{{}\cdot{}}]{\kt\MW\year}\xspace}
\newcommand{\ktMWyr}{\si{\kt\MW\year}\xspace}
% used for hyphen, obsolete now: \newcommand{\SIadj}[2]{\SI[number-unit-product = -]{#1}{#2}}
% change command definition Nov 2017 in case people copy e.g., \ktadj from CDR text.
% E.g., \ktadj{10} now renders the same as \SI{10}{\kt}
\newcommand{\SIadj}[2]{\SI{#1}{#2}}

% Adjective form of some common units (Nov 2107 changed to be same as normal form, no hyphen)
% "the 10-kt detector"

\newcommand{\ktadj}[1]{\SIadj{#1}{\kt}}
% "the 1,300-km baseline"
\newcommand{\kmadj}[1]{\SIadj{#1}{\km}}
% "a 567-keV endpoint"
\newcommand{\keVadj}[1]{\SIadj{#1}{\keV}}
% "Typical 20-MeV event"
\newcommand{\MeVadj}[1]{\SIadj{#1}{\MeV}}
% "Typical 2-GeV event"
\newcommand{\GeVadj}[1]{\SIadj{#1}{\GeV}}
% "the 1.2-MW beam"
\newcommand{\MWadj}[1]{\SIadj{#1}{\MW}}
% "the 700-kW beam"
\newcommand{\kWadj}[1]{\SIadj{#1}{\kW}}
% "the 100-tonne beam"
\newcommand{\tonneadj}[1]{\SIadj{#1}{\tonne}}
% "the 4,850-foot depth beam"
\newcommand{\ftadj}[1]{\SIadj{#1}{\ft}}
%

% Mass exposure, people like to put dots between the units
% \newcommand{\ktyr}[1]{\SI[inter-unit-product=\ensuremath{{}\cdot{}}]{#1}{\kt\year}}
% must make usage of \ktyr above consistent with this one before turning on

% Beam x mass exposure, people like to put dots between the units
\newcommand{\ktmwyr}[1]{\SI[inter-unit-product=\ensuremath{{}\cdot{}}]{#1}{\kt\MW\year}}

\newcommand{\ndmuidbeamrate}{\SI[round-mode=places,round-precision=1]{20.0}{\hertz}\xspace}

\newcommand{\cosmicmuondatayear}{\SI[round-mode=places,round-precision=0]{20.4289491035}{\tera\byte}\xspace}

\newcommand{\tpcdriftdistance}{\SI[round-mode=places,round-precision=2]{3.6}{\meter}\xspace}

\newcommand{\gdaqchannelnumberfactor}{\num[round-mode=places,round-precision=1]{1.1}\xspace}

\newcommand{\daqsamplerate}{\SI[round-mode=places,round-precision=1]{2.0}{\mega\hertz}\xspace}

\newcommand{\beamspillcycle}{\SI[round-mode=places,round-precision=1]{1.2}{\second}\xspace}

\newcommand{\tpcapaperdriftcell}{\num[round-mode=places,round-precision=2]{0.75}\xspace}

\newcommand{\snbdatavolumezs}{\SI[round-mode=places,round-precision=0]{2.4e-05}{\giga\byte}\xspace}

\newcommand{\beamreprate}{\SI[round-mode=places,round-precision=2]{0.833333333333}{\hertz}\xspace}

\newcommand{\ndrockmuonrate}{\SI[round-mode=places,round-precision=1]{10.0}{\hertz}\xspace}

\newcommand{\snbdataratefs}{\SI[round-mode=places,round-precision=1]{17.5226192958}{\mega\byte\per\second}\xspace}

\newcommand{\dunefsreadoutsize}{\SI[round-mode=places,round-precision=1]{24.8832}{\giga\byte}\xspace}

\newcommand{\betarate}{\SI[round-mode=places,round-precision=1]{11.16}{\mega\hertz}\xspace}

\newcommand{\chargezsthreshold}{\SI[round-mode=places,round-precision=1]{0.5}{\MeV}\xspace}

\newcommand{\tpcmodulemass}{\SI[round-mode=places,round-precision=0]{10.0}{\kilo\tonne}\xspace}

\newcommand{\dunenumberchannels}{\num[round-mode=places,round-precision=0]{1536000.0}\xspace}

\newcommand{\chargemaxsignalnoiseratio}{\num[round-mode=places,round-precision=0]{74.0}\xspace}

\newcommand{\ndecalmass}{\SI[round-mode=places,round-precision=0]{93.0}{\tonne}\xspace}

\newcommand{\cosmicmuondatarate}{\SI[round-mode=places,round-precision=1]{647.36816}{\kilo\byte\per\second}\xspace}

\newcommand{\dunenumberapas}{\num[round-mode=places,round-precision=0]{600.0}\xspace}

\newcommand{\dunefsreadoutrate}{\SI[round-mode=places,round-precision=1]{4.608}{\tera\byte/\second}\xspace}

\newcommand{\ndecalbeamrate}{\SI[round-mode=places,round-precision=1]{18.6}{\hertz}\xspace}

\newcommand{\cosmicmuoneventsize}{\SI[round-mode=places,round-precision=1]{2.5}{\mega\byte}\xspace}

\newcommand{\tpcapapermodule}{\num[round-mode=places,round-precision=0]{150.0}\xspace}

\newcommand{\nddatarate}{\SI[round-mode=places,round-precision=1]{1.0048}{\mega\byte\per\second}\xspace}

\newcommand{\chargehethreshold}{\SI[round-mode=places,round-precision=1]{10.0}{\MeV}\xspace}

\newcommand{\beamhighdatayear}{\SI[round-mode=places,round-precision=0]{21.9254891928}{\giga\byte}\xspace}

\newcommand{\betaratedensity}{\SI[round-mode=places,round-precision=3]{0.279}{\hertz\per\kilo\gram}\xspace}

\newcommand{\cosmicmuoncorrection}{\num[round-mode=places,round-precision=2]{1.36}\xspace}

\newcommand{\snbeventratetpc}{\SI[round-mode=places,round-precision=0]{100.0}{\hertz}\xspace}

\newcommand{\snbcloserfactor}{\num[round-mode=places,round-precision=0]{100.0}\xspace}

\newcommand{\snbeventsize}{\num[round-mode=places,round-precision=0]{48.0}\xspace}

\newcommand{\beamrunfraction}{\num[round-mode=places,round-precision=3]{0.667}\xspace}

\newcommand{\beamdatayearfs}{\SI[round-mode=places,round-precision=0]{218.230533073}{\tera\byte}\xspace}

\newcommand{\tpcfulllength}{\SI[round-mode=places,round-precision=1]{58.0}{\meter}\xspace}

\newcommand{\snbcanddatayearzs}{\SI[round-mode=places,round-precision=0]{200.88}{\giga\byte}\xspace}

\newcommand{\snbdataratehighzs}{\SI[round-mode=places,round-precision=0]{2.4e-05}{\tera\byte/\second}\xspace}

\newcommand{\ndeventrate}{\SI[round-mode=places,round-precision=1]{50.24}{\hertz}\xspace}

\newcommand{\dunefsreadoutsizeyear}{\SI[round-mode=places,round-precision=1]{145.414314891}{\exa\byte}\xspace}

\newcommand{\tpcfullwidth}{\SI[round-mode=places,round-precision=1]{14.5}{\meter}\xspace}

\newcommand{\daqchannelsperapa}{\num[round-mode=places,round-precision=0]{2560.0}\xspace}

\newcommand{\tpcapaheight}{\SI[round-mode=places,round-precision=1]{6.0}{\meter}\xspace}

\newcommand{\ndbytesperchannel}{\SI[round-mode=places,round-precision=0]{5.0}{\byte}\xspace}

\newcommand{\ndmuidmass}{\SI[round-mode=places,round-precision=0]{100.0}{\tonne}\xspace}

\newcommand{\ndmuidchannels}{\num[round-mode=places,round-precision=0]{165888.0}\xspace}

\newcommand{\gdaqtrigsamplespersnhit}{\num[round-mode=places,round-precision=0]{48.0}\xspace}

\newcommand{\beamdataratefs}{\SI[round-mode=places,round-precision=0]{6.915456}{\mega\byte\per\second}\xspace}

\newcommand{\ndbeameventratedensity}{\SI[round-mode=places,round-precision=1]{0.2}{\hertz\per\tonne}\xspace}

\newcommand{\tpcapawidth}{\SI[round-mode=places,round-precision=1]{2.3}{\meter}\xspace}

\newcommand{\cosmicmuonrate}{\SI[round-mode=places,round-precision=3]{0.258947264}{\hertz}\xspace}

\newcommand{\ndsstbeamrate}{\SI[round-mode=places,round-precision=1]{1.6}{\hertz}\xspace}

\newcommand{\tpcdrifttime}{\SI[round-mode=places,round-precision=2]{2.25}{\milli\second}\xspace}

\newcommand{\snbeventsizefs}{\SI[round-mode=places,round-precision=1]{46.08}{\tera\byte}\xspace}

\newcommand{\chargeminsignalnoiseratio}{\num[round-mode=places,round-precision=0]{15.0}\xspace}

\newcommand{\beameventsizecompressed}{\SI[round-mode=places,round-precision=1]{0.1}{\mega\byte}\xspace}

\newcommand{\ndecalchannels}{\num[round-mode=places,round-precision=0]{52224.0}\xspace}

\newcommand{\ndcosmicmuonrate}{\SI[round-mode=places,round-precision=2]{0.04}{\hertz}\xspace}

\newcommand{\ndbeameventrate}{\SI[round-mode=places,round-precision=1]{40.2}{\hertz}\xspace}

\newcommand{\gdaqronetriggeredphysicsdatarate}{\SI[round-mode=places,round-precision=0]{0.8448}{\byte\per\second}\xspace}

\newcommand{\gdaqunit}{\SI[round-mode=places,round-precision=0]{0.825}{\byte}\xspace}

\newcommand{\gdaqbetahighrateAPA}{\SI[round-mode=places,round-precision=0]{500.0}{\kilo\hertz}\xspace}

\newcommand{\dunenumbermodules}{\num[round-mode=places,round-precision=0]{4.0}\xspace}

\newcommand{\daqreadoutchannelsamples}{\num[round-mode=places,round-precision=0]{10800.0}\xspace}

\newcommand{\beameventsize}{\SI[round-mode=places,round-precision=1]{2.5}{\mega\byte}\xspace}

\newcommand{\tpcdriftvelocity}{\SI[round-mode=places,round-precision=1]{1.6}{\milli\meter\per\micro\second}\xspace}

\newcommand{\gdaqronebeamphysicsdatarate}{\SI[round-mode=places,round-precision=0]{41250.0}{\kilo\byte\per\second}\xspace}

\newcommand{\gdaqronedatarate}{\SI[round-mode=places,round-precision=0]{41.5153208448}{\mega\byte\per\second}\xspace}

\newcommand{\ndsstmass}{\SI[round-mode=places,round-precision=0]{8.0}{\tonne}\xspace}

\newcommand{\beamfsdatarate}{\SI[round-mode=places,round-precision=0]{436.461066147}{\peta\byte\per\year}\xspace}

\newcommand{\snbreadouttime}{\SI[round-mode=places,round-precision=1]{10.0}{\second}\xspace}

\newcommand{\gdaqronebetadatarate}{\SI[round-mode=places,round-precision=1]{41.25}{\mega\byte\per\second}\xspace}

\newcommand{\snbcanddataratezs}{\SI[round-mode=places,round-precision=0]{6365.63904105}{\byte\per\second}\xspace}

\newcommand{\gdaqnoncosmicoverheadfactor}{\num[round-mode=places,round-precision=1]{1.5}\xspace}

\newcommand{\gdaqronelowenergyphysicsdatarate}{\SI[round-mode=places,round-precision=0]{7.92}{\kilo\byte\per\second}\xspace}

\newcommand{\gdaqcrrateAPA}{\SI[round-mode=places,round-precision=4]{0.0004}{\hertz}\xspace}

\newcommand{\beamonfraction}{\num[round-mode=places,round-precision=0]{0.0030015}\xspace}

\newcommand{\betadatayear}{\SI[round-mode=places,round-precision=0]{52.8262940816}{\peta\byte}\xspace}

\newcommand{\betainspillyear}{\SI[round-mode=places,round-precision=0]{158.558121686}{\tera\byte}\xspace}

\newcommand{\gdaqtrigcompression}{\num[round-mode=places,round-precision=0]{2.0}\xspace}

\newcommand{\gdaqhighrateAPAs}{\num[round-mode=places,round-precision=0]{100.0}\xspace}

\newcommand{\daqdriftsperreadout}{\num[round-mode=places,round-precision=1]{2.4}\xspace}

\newcommand{\snbdatavolumehighzs}{\SI[round-mode=places,round-precision=0]{0.00024}{\tera\byte}\xspace}

\newcommand{\ndsstchannels}{\num[round-mode=places,round-precision=0]{215040.0}\xspace}

\newcommand{\gdaqroneSNphysicsdatarate}{\SI[round-mode=places,round-precision=0]{257.4}{\kilo\byte\per\second}\xspace}

\newcommand{\beamhighdatarate}{\SI[round-mode=places,round-precision=2]{0.694791666667}{\kilo\byte\per\second}\xspace}

\newcommand{\dunefsreadoutsizeminute}{\SI[round-mode=places,round-precision=1]{276.48}{\tera\byte}\xspace}

\newcommand{\daqchannelspermodule}{\num[round-mode=places,round-precision=0]{384000.0}\xspace}

\newcommand{\gdaqtrigsamplesperbetahit}{\num[round-mode=places,round-precision=0]{24.0}\xspace}

\newcommand{\snbcandrate}{\SI[round-mode=places,round-precision=0]{12.0}{\per\year}\xspace}

\newcommand{\dunedetectormass}{\SI[round-mode=places,round-precision=0]{40.0}{\kilo\tonne}\xspace}

\newcommand{\gdaqAPAchannelspercosmic}{\num[round-mode=places,round-precision=0]{2560.0}\xspace}

\newcommand{\ndeventsize}{\num[round-mode=places,round-precision=0]{4000.0}\xspace}

\newcommand{\snbdatayearfs}{\SI[round-mode=places,round-precision=0]{552.96}{\tera\byte}\xspace}

\newcommand{\dunefsreadoutsizesecond}{\SI[round-mode=places,round-precision=1]{4.608}{\tera\byte}\xspace}

\newcommand{\betainbeamyear}{\SI[round-mode=places,round-precision=0]{79.279060843}{\giga\byte}\xspace}

\newcommand{\gdaqlphighrateAPA}{\SI[round-mode=places,round-precision=0]{200.0}{\hertz}\xspace}

\newcommand{\betaeventsize}{\num[round-mode=places,round-precision=1]{100.0}\xspace}

\newcommand{\betadatarate}{\SI[round-mode=places,round-precision=1]{1.674}{\giga\byte\per\second}\xspace}

\newcommand{\betareadoutsize}{\SI[round-mode=places,round-precision=0]{150.0}{\byte}\xspace}

\newcommand{\snbcandeventsizezs}{\SI[round-mode=places,round-precision=1]{16.74}{\giga\byte}\xspace}

\newcommand{\daqreadouttime}{\SI[round-mode=places,round-precision=1]{5.4}{\milli\second}\xspace}

\newcommand{\cosmicmuonflux}{\SI[round-mode=places,round-precision=2]{5.66e-09}{\hertz\per\centi\meter\squared}\xspace}

\newcommand{\snbdataratezs}{\SI[round-mode=places,round-precision=0]{2.4e-06}{\giga\byte\per\second}\xspace}

\newcommand{\tpcfullheight}{\SI[round-mode=places,round-precision=1]{12.0}{\meter}\xspace}

\newcommand{\gdaqsnhighrateAPA}{\SI[round-mode=places,round-precision=0]{6500.0}{\hertz}\xspace}

\newcommand{\beameventoccupancy}{\num[round-mode=places,round-precision=4]{0.0005}\xspace}

\newcommand{\beamrate}{\SI[round-mode=places,round-precision=0]{8770.19567714}{\per\year}\xspace}

\newcommand{\daqbytespersample}{\SI[round-mode=places,round-precision=1]{1.5}{\byte}\xspace}

%% generated file, do not edit% This file is generated, any edits may be lost.

% It defines macros which expand to corresponding
% specification values for subsystem SP-FD


\def\splightyield{$>\,\SI{0.5}{pe/MeV}$}
\def\spfepeaktime{\SI{1}{\micro\second}}
\def\sptimeresolutionpds{$<\,\SI{1}{\micro\second}$}
\def\spdetdeadtime{$<\,\SI{0.5}{\%}$}
\def\spapawireangles{\SI{0}{\degree} for collection wires, \SI{35.7}{\degree} for induction wires}
\def\spadcnumberofbits{\num{12} bits}
\def\spapawirepostolerance{$\pm\,\SI{0.5}{mm}$}
\def\spdeadchannels{$<\,\SI{1}{\%}$}
\def\splocalefields{$<\,\SI{30}{kV/cm}$}
\def\spcathoderesistivity{$>\,\SI{1}{\mega\ohm/square}$}
\def\spradiopurity{ALARA}
\def\splarncontamination{$<\,\SI{25}{ppm}$}
\def\spmindriftfield{$>$\,\SI{250}{ V/cm}}
\def\splarpurity{$<$\,\SI{100}{ppt}}
\def\spcryomonitordevices{}
\def\spapagaps{$<\,\SI{15}{mm}$ between APAs on same support beam; $<\,\SI{30}{mm}$ between APAs on different support beams}
\def\spspsignalsaturation{\num{500000} electrons}
\def\sphvpsripple{$<\,\SI{100}{enc}$}
\def\spadcsamplingfreq{$\sim\,\SI{2}{\mega\hertz}$}
\def\spdataratetotape{$<\,\SI{30}{PB/year}$}
\def\splarimpuritycontrib{$<<\,\SI{30}{ppt} $}
\def\spcepowerconsumption{$<\,\SI{50}{ mW/channel} $}
\def\spnonfenoise{$<<\,\SI{1000}{enc} $}
\def\spsystemnoise{$<\,\SI{1000}{enc}$}
\def\spmisalignmentfielduniformity{$<\,1\,$\% throughout volume}
\def\spapawirespacing{\SI{4.669}{mm} for U,V; \SI{4.790}{mm} for X,G}
\def\sphvsfielduniformity{$<\,\SI{1}{\%}$ throughout volume}
\def\spsntrigger{$>\,\SI{90}{\%}$ efficiency for SNB within \SI{100}{kpc}}

% This file is generated, any edits may be lost.

% It defines macros which expand to corresponding
% specification values for subsystem SP-APA


\def\spapaunitsize{\SI{6.0}{m} tall $\times$ \SI{2.3}{m} wide}
\def\spapabadchannels{$<$1\%, with a goal of $<$0.5\%}
\def\spapaframeplanarity{$<$\SI{5}{mm}}
\def\spapabiasvoltage{The setup, including boards, must hold 150\% of max operating voltage.}
\def\spapawiretension{\SI{5}{N} $\pm$ \SI{1}{N}}
\def\spapaactivearea{Maximize total active area.}

% This file is generated, any edits may be lost.

% It defines macros which expand to corresponding
% specification values for subsystem SP-HV


\def\sphvconnectionredundancy{Two-fold}
\def\sppowersupplystability{$>\,\SI{95}{\%}$ uptime}

% This file is generated, any edits may be lost.

% It defines macros which expand to corresponding
% specification values for subsystem SP-PDS


\def\spapainstall{$>\,\SI{1}{\milli\meter}$}
\def\spenvhumiditylimit{$<\,\SI{50}{\%}$ RH at \SI{70}{\degree F}}
\def\spenvlightexposure{No exposure to sunlight. All other unfiltered sources: $<\,\num{30}$ minutes integrated across all exposures}
\def\splighttightness{Cryostat light leaks responsible for $<\,\SI{10}{\%}$  of data transferred from PDS to DAQ}
\def\spmechcompatibility{$>\,\SI{1}{\milli\meter}$}
\def\spmechdeflection{$<\,\SI{5}{\milli\meter}$}
\def\sppdscable{$<\,\SI{6}{\milli\meter}$}
\def\sppdscablemate{\SI{0}{\milli\meter} separation mechanically allowed}
\def\sppdscleanroomassbly{Class \num{100000} clean assembly area}
\def\sppdsclearance{$>\,\SI{0.5}{\milli\meter}$}
\def\sppdsdarkrate{$<\,\SI{1}{kHz}$}
\def\sppdsdatarate{$<\,\SI{8}{Gbps}$}
\def\sppdsdynamicrange{$<\,\SI{20}{\%}$}
\def\sppdssignaltonoise{$>\,\num{4}$}
\def\spspatiallocalization{$<\,\SI{2.5}{\meter}$}

% This file is generated, any edits may be lost.

% It defines macros which expand to corresponding
% specification values for subsystem SP-DAQ


\def\sptriggercalibration{}
\def\sptriggerlowenergy{$>$\SI{10}{\MeV}}
\def\spdaqdeadtime{}
\def\sptriggerhighenergy{$>$\SI{100}{\MeV}}
\def\spdatarecord{}
\def\sptriggerbeam{$>$\SI{100}{\MeV}}
\def\sptriggersnb{}

% This file is generated, any edits may be lost.

% It defines macros which expand to corresponding
% specification values for subsystem SP-CISC


\def\sptemprepro{$<\,\SI{5}{mK}$}
\def\spslowcontrolarchiverate{\SI{0.02}{Hz}}
\def\spcameracoldcoverage{$>\,$80\% of HV surfaces}
\def\sptempstability{$<\,\SI{2}{mK}$ at all places and times}
\def\spinstnoise{$\ll\,\SI{1000}{enc}$}
\def\speleclifetimeprec{$<\,$1.4\% ($<$4\%)}
\def\spslowcontrolnumvars{$>\,\num{150000}$}
\def\spslowcontrolalarmrate{$<\,$150/day}
\def\spinstefield{$<\,\SI{30}{kV/cm}$}
\def\speleclifetimerange{\SIrange{0}{10}{ms} (\SIrange{0}{30}{ms})}



% For abbrev/acronym lists, see also init.tex for entry point of acronyms.tex.
% \usepackage[intoc]{nomencl}
% \makenomenclature
% \renewcommand{\nomname}{Acronyms, Abbreviations and Terms}
% \setlength{\nomlabelwidth}{0.2\textwidth}

% For glossaries, needs to be loaded *after* hyperref to get clickable links.
% See dune-words.tex for detailed explanation.

% http://mirrors.ctan.org/macros/latex/contrib/glossaries/glossaries-user.pdf

% \usepackage[acronyms,toc]{glossaries}
\usepackage[toc]{glossaries}
\makeglossaries


% for terms with acronyms
\newcommand{\dshort}[1]{\glsentrytext{#1}}
\newcommand{\dshorts}[1]{\glsentryshortpl{#1}}
\newcommand{\dlong}[1]{\glsentrylong{#1}}
\newcommand{\dlongs}[1]{\glsentrylongpl{#1}}

% force the "first time" behavior
\newcommand{\dfirst}[1]{\glsfirst{#1}}
\newcommand{\dfirsts}[1]{\glsfirstplural{#1}}

\newcommand{\dword}[1]{\gls{#1}}
\newcommand{\dwords}[1]{\glspl{#1}}
\newcommand{\Dword}[1]{\Gls{#1}}
\newcommand{\Dwords}[1]{\Glspl{#1}}


% use this to define terms that do NOT have acronyms.
% \newduneword{label}{term}{description}
\newcommand{\newduneword}[3]{
    \newglossaryentry{#1}{
        text={#2},
        long={#2},
        name={\glsentrylong{#1}},
        first={\glsentryname{#1}},
        firstplural={\glsentrylong{#1}\glspluralsuffix},
        description={#3}
    }
}

% use this to define terms that DO have acronyms.
%                 1      2     3       4 
% \newduneabbrev{label}{abbrev}{term}{description}
\newcommand{\newduneabbrev}[4]{
  \newglossaryentry{#1}{
    text={#2},
    long={#3},
    shortplural={{#2}\glspluralsuffix},
    longplural={{#3}\glspluralsuffix{}},
    name={\glsentrylong{#1}{} (\glsentrytext{#1}{})},
    first={\glsentryname{#1}},
    firstplural={#3\glspluralsuffix{} (\glsentrytext{#1}\glspluralsuffix{})},
    description={#4}
  }
}

%% If plural needs special spelling besides adding an "s"
%                 1      2     3       4        5
% \newduneabbrev{label}{abbrev}{term}{terms}{description}
\newcommand{\newduneabbrevs}[5]{
  \newglossaryentry{#1}{
    text={#2},
    long={#3},
    plural={#4},
    shortplural={{#2}\glspluralsuffix},
    longplural={#4},
    name={\glsentrylong{#1}{} (\glsentrytext{#1}{})},
    first={#3 (#2)},
    firstplural={#4 (\glsentrytext{#1}\glspluralsuffix{})},
    description={#5}
  }
}


% tres meta
\newduneword{dword}{DUNE Word}{A term in the DUNE lexicon}

%%%%%%     START ADDING WORDS, IN ALPHABETICAL ORDER IF POSSIBLE!    %%%%%%%%
\newduneword{nasa}{NASA}{National Aereonautics and Space Administration}

%near detector
\newduneabbrev{nd}{ND}{near detector}{Refers to the detector(s) %or more  generally the experimental site
 installed close to the neutrino source at \fnal }

%far detector
\newduneabbrev{fd}{FD}{far detector}{Refers to the \fdfiducialmass fiducial mass DUNE detector %or more generally the experimental 
  to be installed at the far site at \surf in
  Lead, SD, to be composed of four \nominalmodsize modules}

%single-phase
\newduneabbrev{sp}{SP}{single-phase}{Distinguishes one of the DUNE far detector technologies by the fact that it operates using argon in its liquid phase only
  %Distinguishes one of the four  \SI{10}{\kton} \glspl{detmodule} of the DUNE far detector by the  fact that it operates using argon in just its liquid phase}
  }

%dual-phase
\newduneabbrev{dp}{DP}{dual-phase}{Distinguishes one of the DUNE far detector technologies by the fact that it operates using argon %Distinguishes one of the four \SI{10}{\kton} \glspl{detmodule} of the DUNE far detector by the fact that it operates using argon 
 in both gas and liquid phases}

%photon detection system
\newduneabbrev{pds}{PDS}{photon detection system}{The detector 
  subsystem sensitive to light produced in the \lar }

%high voltage system
\newduneabbrev{hvs}{HVS}{high voltage system}{The detector 
  subsystem that provides the TPC drift field }

%time projection chamber
\newduneabbrev{tpc}{TPC}{time projection chamber}{The portion of each
  DUNE \gls{detmodule} that records ionization electrons
  after they drift away from a cathode through the \lar, and also
  through gaseous argon in a \dual module. 
  The activity is recorded by digitizing the waveforms of current
  induced on the anode as the distribution of ionization charge passes by
  or is collected on the electrode}

%anode plane assembly
\newduneabbrevs{apa}{APA}{anode plane assembly}{anode plane assemblies}{A unit of the \single
  detector module containing the elements sensitive to ionization in the \lar. 
  It contains two faces each of three planes of wires, and interfaces to the cold
  electronics and photon detection system} 

\newduneabbrev{awg}{AWG}{American wire gauge} {U.S. standard set of non-ferrous wire conductor sizes}

\newduneabbrev{ufer}{Ufer}{Concrete Encased Electrode} {U.S. National Electrical Code grounding method refered to as Concrete Encased Electrode}

%charge readout
\newduneabbrev{cro}{CRO}{charge readout}{The system for detecting
  ionization charge distributions in a \dual detector module}

%light readout
\newduneabbrev{lro}{LRO}{light readout}{The system for detecting
  scintillation photons in a \dual  detector module}

%safe high voltage
\newduneabbrev{shv}{SHV}{safe high voltage}{Type of bayonet mount
connector used on coaxial cables that has additional insulation 
compared to standard BNC and MHV connectots that makes it safer
for handling high voltage by preventing accidental contact with the
live wire connector in an unmated connector or plug}

%front-end
\newduneabbrev{fe}{FE}{front-end}{The front-end refers a point that is
  ``upstream'' of the data flow for a particular subsystem. 
  For example the front-end electronics is where the cold electronics
  meet the sense wires of the TPC and the front-end DAQ is where the
  DAQ meets the output of the electronics}

\newduneabbrev{daqrou}{RU}{DAQ readout units}{The first element in the data flow of the DAQ.}

\newduneabbrev{cots}{COTS}{commercial off-the-shelf}{Items, typically hardware such as 
computers, which may be purchased whole, without any custom design or fabrication and 
thus at normal consumer prices and availability.}

\newduneabbrev{i2c}{I2C}{Inter-Integrated Circuit}{I$^2$C or I2C is a synchronous, 
multi-master, multi-slave, packet switched, single-ended, serial computer bus widely used 
for attaching lower-speed peripheral ICs to processors and microcontrollers in short-distance, 
intra-board communication.} %leave upper case

\newduneabbrev{spi}{SPI}{Serial Peripheral Interface}{The Serial Peripheral Interface is a 
synchronous serial communication interface specification used for short distance 
communication, primarily in embedded systems.}%leave upper case

\newduneabbrev{miso}{MISO}{master in slave out}{The Master In Slave Out is a logic
signal on the \gls{spi} bus on which the data from the slave are transmitted once
a request from the master is received.} %leave upper case

\newduneabbrev{mosi}{MOSI}{master out slave in}{The Master Out Slave In is a logic
signal on the \gls{spi} bus on which the data from the master is transmitted.} %leave upper case

\newduneabbrev{uart}{UART}{Universal Asynchrous Receiver/Transmitter}{A universal 
asynchronous receiver-transmitter is a computer hardware device for asynchronous 
serial communication in which the data format and transmission speeds are configurable.}%leave upper case

\newduneword{cr}{CR}{Capacitance-Resistance} %leave upper case

\newduneword{dc}{DC}{direct coupling} % I think these are ok lower case (anne)

\newduneword{ac}{AC}{capacitive coupling}  % I think these are ok lower case (anne)

\newduneabbrev{pll}{PLL}{Phase-Locked Loop}{A phase-locked loop is a control system that generates an
output signal whose phase is related to the phase of an input signal.}  %leave upper case

\newduneword{fifo}{FIFO}{First-In-First-Out} % leave in upper case

\newduneword{tsmc}{TSMC}{Taiwan Semiconductor Manufacturing Company}

\newduneword{saci}{SACI}{\gls{slac} \gls{asic} Control Interface}

\newduneword{qfp}{QFP}{Quad Flat Package} % leave in upper case

\newduneabbrev{ams}{AMS}{analog and mixed signal}{Verilog-AMS is a derivative of the Verilog hardware description language that includes analog and mixed-signal extensions (AMS) in order to define the behavior of analog and mixed-signal systems.}

\newduneabbrev{hepa}{HEPA}{High Efficiency Particulate Air}{The High Efficiency Particulate Air filters are a type of air filter that remove \num{99.97}\% of particles that have a size greater han or equal to \SI{0.3}{$\mu$m}.}  % leave in upper case

\newduneabbrev{uvm}{UVM}{universal verification methodology}{The Universal Verification Methodology is a standardized methodology for verifying integrated circuit designs.}   % leave in upper case

\newduneword{lhc}{LHC}{Large Hadron Collider}

\newduneabbrev{lsb}{LSB}{Least Significant Bit}{The bit with the lowest numerical value in a binary number}

\newduneabbrev{ldo}{LDO}{low-dropout regulator}{A low-dropout or LDO regulator is a DC linear voltage regulator that can regulate the output voltage even when the supply voltage is very close to the output voltage.}

%analog digital converter
\newduneabbrev{adc}{ADC}{analog-to-digital converter}{A sampling of a voltage
  resulting in a discrete integer count corresponding in some way to
  the input}

\newduneabbrev{inl}{INL}{integral non-linearity}{A commonly used measure of performance in \gls{adc}s. It is the deviation between the ideal input threshold value and the measured threshold level of a certain output code.}

\newduneabbrev{dnl}{DNL}{differential non-linearity}{A commonly used measure of performance in \gls{adc}s. The DNL error is defined as the difference between an actual step width and the ideal value of one \gls{lsb}.}

\newduneword{pnp}{PNP}{Type of bipolar junction transistor consistning of a
layer of N-doped semiconductor sandwiched between two layers of P-doped material.}

\newduneword{spice}{SPICE}{SPICE
(``Simulation Program with Integrated Circuit Emphasis'') is a general-purpose, 
open-source analog electronic circuit simulator. It is a program used in integrated 
circuit and board-level design to check the integrity of circuit designs and to 
predict circuit behavior.}

%data acquisition
\newduneabbrev{daq}{DAQ}{data acquisition}{The data acquisition system
  accepts data from the detector FE electronics, buffers
  the data, performs a \gls{trigdecision}, builds events from the selected
  data and delivers the result to the offline \gls{diskbuffer}}

%detector module
\newduneword{detmodule}{detector module}{The entire DUNE far detector is
  segmented into four modules, each with a nominal \SI{10}{\kton}
  fiducial mass}

%detector unit
\newduneword{detunit}{detector unit}{A \gls{submodule} may be partitioned
  into a number of similar parts. 
  For example the single-phase TPC \gls{submodule} is made up of APA
  units}

%signal over noise ratio
\newduneword{snr}{\mbox{S/N}}{signal over noise ratio}

%secondary DAQ buffer
\newduneword{diskbuffer}{secondary DAQ buffer}{A secondary
  \dshort{daq} buffer holds a small subset of the full rate as
  selected by a \gls{trigcommand}. 
  This buffer also marks the interface with the DUNE Offline}

%online  monitoring
\newduneabbrev{om}{OM}{online monitoring}{Processes that run inside
  the DAQ on data ``in flight,'' specifically before landing on the
  offline disk buffer, and that provide feedback on the operation of
  the DAQ itself and the general health of the data it is marshalling}

%data quality monitoring
\newduneabbrev{dqm}{DQM}{data quality monitoring}{Analysis of the raw
  data to monitor the integrity of the data and the performance of the
  detectors and their electronics. This type of monitoring may be
  performed in real time, within the \gls{daq} system, or in later
  stages of processing, using disk files as input}

%DAQ dump buffer
\newduneword{dumpbuffer}{DAQ dump buffer}{This \dshort{daq} buffer
  accepts a high-rate data stream, in aggregate, from an associated
  \gls{submodule} sufficient to collect all data likely relevant to
  a potential \gls{snb}}

%Global Trigger Logic
\newduneabbrev{etl}{ETL}{external trigger logic}{Trigger processing
  that consumes \gls{detmodule} level \gls{trignote} information
  and other global sources of trigger input and emits
  \gls{trigcommand} information back to the \glspl{mtl}}

%trigger notification
\newduneword{trignote}{trigger notification}{Information provided by
  \gls{mtl} to \gls{etl} about \gls{trigdecision} its processing}

%trigger primitive
\newduneword{trigprimitive}{trigger primitive}{Information derived by
  the DAQ \gls{fe} hardware that describes a region of space (e.g.,
  one or several neighboring channels) and time (e.g., a contiguous set
  of ADC sample ticks) associated with some activity}

%external trigger candidate
\newduneword{externtrigger}{external trigger candidate}{Information
  provided to the \gls{mtl} about events external to a
  \gls{detmodule} so that it may be considered in forming
  \glspl{trigcommand}}

%Out-of-band trigger command dispatcher
\newduneabbrev{daqoob}{OOB dispatcher}{out-of-band trigger command
  dispatcher}{This component is responsible for dispatching a \gls{snb} dump
  command to all \glspl{daqfer} in the \gls{detmodule}}

%module trigger logic
\newduneabbrev{mtl}{MTL}{module trigger logic}{Trigger processing
  that consumes \gls{detunit} level \gls{trigcommand} information
  and emits \glspl{trigcommand}. 
  It provides the \gls{etl} with \glspl{trignote} and receives back any
  \glspl{externtrigger}}

%octant
\newduneword{octant}{octant}{Any of the eight parts into which 4$\pi$
  is divided by three mutually perpendicular axes. 
  In particular in referencing the value for the mixing angle
  $\theta_{23}$}

%sub-detector ??? %%%%%%%%%%%%%%%%%%%      Why not ``subdet''? (Anne)  %%%%%% ???????
\newduneword{submodule}{subdetector}{A detector unit of granularity less
  than one \gls{detmodule} such as the TPC of either a \single or \dual 
  module}

%trigger candidate
\newduneword{trigcandidate}{trigger candidate}{Summary information derived
  from the full data stream and representing a contribution toward
  forming a \gls{trigdecision}}

%trigger command
\newduneword{trigcommand}{trigger command}{Information derived from
  one or more \glspl{trigcandidate}  that directs elements of the
  \gls{detmodule} to read out a portion of the data stream}

%trigger command message
\newduneabbrev{tcm}{TCM}{trigger command message}{A message flowing
  down the trigger hierarchy from global to local context.  Also see \gls{tpm}}

\newduneabbrev{mlt}{MLT}{module level trigger}{add def}

%trigger decision
\newduneword{trigdecision}{trigger decision}{The process by which
  \glspl{trigcandidate} are converted into \glspl{trigcommand}}

%trigger primitive message
\newduneabbrev{tpm}{TPM}{trigger primitive message}{A message flowing
  up the trigger hierarchy from local to global context.  Also see \gls{tcm}}

\newduneabbrev{ipc}{IPC}{inter-process communication}{A system for software elements to exchange information between threads, local processes or across a data network.  An IPC system is typically specified in terms of protocols  composed of message types and their associated data schema.}

\newduneword{daqdispre}{discovery and presence}{As used in the context of the \gls{ipc}, a system which provides mechanisms for a node on a communication network to learn of the existence of peers and their identity (discovery) as well as determine if they are currently operational or have become unresponsive (presence).}

\newduneabbrev{pubsub}{PUB/SUB}{publish-subscribe communication pattern}{An \gls{ipc} communication pattern where one element, the publisher, sends data to all connected elements, the subscribers.  Each subscriber may connect to multiple publishers.  A variant is PUB/SUB with topics where a subscriber may register an identifier, the topic, to limit the information received to just an associated subset.}

%event builder
\newduneabbrev{eb}{EB}{event builder}{A software agent servicing one
  \gls{detmodule} by executing \glspl{trigcommand} by reading out
  the requested data}

\newduneabbrev{daqdfo}{DFO}{dataflow orchestration}{The process by which trigger commands are executed in parallel and asynchronous manner by the back-end output subsystem of the DAQ.}

% yes, this is a silly name
\newduneabbrev{daqfbi}{FBI}{front-end buffer interface}{The process which provides read-only access to data residing in a front-end's buffers to processes on the network.}

%cluster on board
\newduneabbrev{cob}{COB}{cluster on board}{An ATCA motherboard housing four RCEs}

%reconfigurable computing element
\newduneabbrev{rce}{RCE}{reconfigurable computing element}{Data processor located outside of the cryostat on a \gls{cob} which contains \gls{fpga}, RAM and SSD resources, responsible for buffering data, producing trigger primitives, responding to triggered requests for data and sinking \gls{snb} dumps}

%bump on wire
\newduneabbrev{bow}{BOW}{Bump On Wire}{A working name for the front-end readout computing elements used in the nominal DAQ design to interface the \dual  crates to the DAQ front-end computers}

%advanced telecommunication computing architecture
\newduneabbrev{atca}{ATCA}{Advanced Telecommunications Computing
  Architecture}{An advanced computer architecture specification developed for the telecommunications, military, and aerospace industries that incorporates the latest trends high-speed interconnect technologies, next-generation processors, and improved reliability, availability and serviceability} 

%Micro Telecommunications Computing Architecture
\newduneabbrev{utca}{$\mu$TCA}{Micro Telecommunications Computing Architecture}{The computer architecture specification followed by the crates that house charge and light readout electronics in the dual-phase module} 

%user datagram protocol
\newduneabbrev{udp}{UDP}{user datagram protocol}{A simple,
  connectionless Internet protocol that supports data integrity
  checksums, requires no handshaking, and does not guarantee packet delivery}

%advanced meazzanine card
\newduneabbrev{amc}{AMC}{advanced mezzanine card}{Holds digitizing
  electronics and lives in \gls{utca} crates}

%radio frequency
\newduneabbrev{rf}{RF}{radio frequency}{Electromagnetic emissions
  that are within the (radio) frequency band of sensitivity of the detector
  electronics}

%field programmable gate array
\newduneabbrev{fpga}{FPGA}{field programmable gate array}{An
integrated circuit technology that allows the hardware to be reconfigured to
execute different algorithms after its manufacture and deployment}

\newduneabbrev{fmc}{FMC}{FPGA mezzanine card}{Boards holding FPGA and other integrated circuitry which attach to a motherboard.}

%Front-End Link eXchange
\newduneabbrev{felix}{FELIX}{Front-End Link eXchange}{A
  high-throughput interface between front-end and trigger electronics
  and the standard PCIe computer bus}

%DAQ partition
\newduneword{daqpart}{DAQ partition}{A cohesive and
 coherent collection of DAQ hardware and software working together to trigger and read out some portion of one detector module; it consists of an integral number of \glspl{daqfrag}. 
 Multiple DAQ partitions may operate simultaneously, but each instance operates independently}

%front-end computer
\newduneabbrev{fec}{FEC}{front-end computer}{The portion of one
  \gls{daqpart} that hosts the \gls{daqdr}, \gls{daqbuf} and
  \gls{daqds}.  It hosts the \gls{daqfer} and corresponding portion of the \gls{daqbuf}.}

%DAQ front-end fragment
\newduneword{daqfrag}{DAQ front-end fragment}{The portion of one
  \gls{daqpart} relating to a single \gls{fec} and corresponding to an
  integral number of \glspl{detunit}.  See also \gls{datafrag}}

%data fragment
\newduneword{datafrag}{data fragment}{A block of data read out from a single \gls{daqfrag} that
span a contiguous period of time as requested by a \gls{trigcommand}}

%DAQ front-end redout
\newduneabbrev{daqfer}{FER}{DAQ front-end readout}{The portion of a
  \gls{daqfrag} that accepts data from the detector electronics and
  provides it to the \gls{fec}.}

%DAQ data receiver
\newduneabbrev{daqdr}{DDR}{DAQ data receiver}{The portion of the
  \gls{daqfrag} that accepts data from the \gls{daqfer}, emits
  trigger candidates produced from the input trigger primitives, and
  forwards the full data stream to the \gls{daqbuf}}

%primary DAQ buffer
\newduneabbrev{daqbuf}{primary buffer}{DAQ primary buffer}{The portion
  of the \gls{daqfrag} that accepts full data stream from the
  corresponding \gls{detunit} and retains it sufficiently long for it
  to be available to produce a \gls{datafrag}}

%data selector
\newduneword{daqds}{data selector}{The portion of the \gls{daqfrag}
  that accepts \glspl{trigcommand} and returns the corresponding
  \gls{datafrag}.  Not to be confused with \gls{daqdsn}.}

\newduneword{daqdsn}{data selection}{The process of forming a trigger decision for selecting a subset of detector data for output by the DAQ from the content of the detector data itself.  Not to be confused with \gls{daqds}.}

\newduneabbrev{daqros}{RO}{DAQ readout sub-system}{The sub-system of the DAQ for accepting and buffering data input from detector electronics.}

\newduneabbrev{daqdss}{DS}{DAQ data selection sub-system}{The sub-system of the DAQ responsible for forming a trigger decision based on a portion of the input data stream.  The majority subset of the \gls{daqtrs}.}
\newduneabbrev{daqtrs}{TS}{DAQ trigger sub-system}{The sub-system of the DAQ responsible for forming a trigger decision.}
\newduneabbrev{daqbes}{BE}{DAQ back-end sub-system}{The portion of the DAQ which is generally toward its output end.  It is responsible for accepting and executing trigger commands and marshaling the data they address to output storage buffers.}
\newduneabbrev{daqtss}{TSS}{DAQ timing and synchronization sub-system}{The portion of the DAQ which provides for timing and synchronization to various components.}


%front-end mother board
\newduneabbrev{femb}{FEMB}{front-end mother board}{Refers a unit of
  the \gls{sp} cold electronics that contains the front-end amplifier
  and ADC ASICs covering 128 channels}

%application-specific integrated circuit
\newduneword{asic}{ASIC}{application-specific integrated circuit}

%low voltage
\newduneword{lv}{LV}{low voltage}

%iceberg
\newduneword{iceberg}{ICEBERG}{Integrated Cryostat and Electronics Built for Experimental Research Goals:
a new double wall cryostat build at Fermilab and installed in the Proton Assembly Building meant for 
liquid argon detector R\&D and for testing of DUNE detector components}

%coldadc
\newduneword{coldadc}{ColdADC}{newly developed 16-channels ASIC providing analog to digital conversion}

%coldata
\newduneword{coldata}{COLDATA}{a 64-channels control and communications ASIC}
%\newduneabbrev{coldata}{COLDATA}{a 64-channel control and communications ASIC}{A key component of the 128-channel \gls{femb} that provides a control and communication interface between cold \gls{lvds} channels and warm electronics external to the cryostat}

%cryo
\newduneword{cryo}{CRYO}{integrated ASIC including front-end circuitry providing signal amplification and pulse shaping, analog to digital conversion, and control and communication functionalities for 64 channels}

%liquid argon application-specific integrated circuit
\newduneword{larasic}{LArASIC}{A 16-channels front-end ASIC that provides signal amplification and pulse shaping}

%complementary metal-oxide-semiconductor
\newduneword{cmos}{CMOS}{Complementary metal-oxide-semiconductor}

%equivalent noise charge
\newduneword{enc}{ENC}{equivalent noise charge}

%equivalent number of bits
\newduneword{enob}{ENOB}{effective number of bits}

%dynamic range enhancement
\newduneword{dre}{DRE}{dynamic range enhancement}

%successive approximation register
\newduneword{sar}{SAR}{successive approximation register}

%protodune
\newduneword{protodune}{ProtoDUNE}{Either of the two DUNE prototype detectors constructed at CERN. % and operated in  a CERN test beam (expected fall 2018). 
  One prototype implements \single and the other \dual technology. }

\newduneword{protodune2}{ProtoDUNE-2}{The second run of the ProtoDUNE detector}

%the single-phase ProtoDUNE detector
\newduneword{pdsp}{ProtoDUNE-SP}{The \single ProtoDUNE detector}

%the dual-phase ProtoDune detector
\newduneword{pddp}{ProtoDUNE-DP}{The \dual ProtoDUNE detector}

%WA105 dual-phase demonstrator
\newduneword{wa105}{WA105 DP demonstrator}{The \SI[product-units=power]{3x1x1}{m} WA105 dual-phase prototype detector at CERN}

%data aquisition event block --- includes dirty word: "event"
\newduneword{rawevent}{DAQ event block}{The unit of data output by the
  DAQ. 
  It contains trigger and detector data spanning a unique, contiguous
  time period and a subset of the detector channels.}

%solid-state disk
\newduneabbrev{ssd}{SSD}{solid-state disk}{Any storage device that
  may provide sufficient write throughput to receive, both collectively and
  distributed, the sustained full rate of data from a \gls{detmodule}
  for many seconds}

%high-level trigger --- fixme: this needs improvement
\newduneabbrev{hlt}{HLT}{high-level trigger}{This is actually a filter applied to data which has been triggered and aggregated in order to further reduce or characterize it.}

%particle identification
\newduneabbrev{pid}{PID}{particle ID}{Particle identification}

%readout window
\newduneword{readout window}{readout window}{A fixed, atomic and
  continuous period of time over which data from a \gls{detmodule}, in
  whole or in part, is recorded. 
  This period may differ based on the trigger that initiated the
  readout}

%zero-suppression
\newduneabbrev{zs}{ZS}{zero-suppression}{Used to delete some portion of a
  data stream that does not significantly deviate from zero or
  intrinsic noise levels. 
  It may be applied at different granularity from per-channel to per
  \gls{detunit}}

%run control ------- fixme: maybe another sentence
\newduneabbrev{rc}{RC}{run control}{The system for configuring,
  starting and terminating the DAQ}

\newduneabbrev{daqccm}{CCM}{DAQ control, configuration and monitoring sub-system}{A system for controlling, configuring and monitoring other systems in particular those that make up the DAQ where the CCM encompasses \gls{rc}.}

\newduneword{daqrun}{DAQ run}{A period of time over which relevant data taking conditions and DAQ configuration are asserted to be unchanged. 
  Multiple DAQ runs may occur simultaneously when multiple \glspl{daqpart} are active. 
  This term should not be confused with DUNE experiment or beam ``runs'' which typically span many DAQ runs.}
\newduneword{daqrunnum}{DAQ run number}{A monotonic increasing count which uniquely and globally identifies a \gls{daqrun}.}

%supernova neutrino burst  
\newduneabbrev{snb}{SNB}{supernova neutrino burst}{A prompt 
  increase in the flux of low-energy neutrinos emitted in the first few seconds of a core-collapse supernova.  It can also refer to a trigger command type that may be due to an SNB,
  or detector conditions that mimic its interaction signature}

%supernova burst and low energy
\newduneabbrev{snble}{SNB/LE}{supernova neutrino burst and low
  energy}{Supernova neutrino burst and low-energy physics program}

%supernova early warning system
\newduneabbrev{snews}{SNEWS}{SuperNova Early Warning System}{A global
  supernova neutrino burst trigger formed by a coincidence of SNB
  triggers collected from participating experiments}

%one pulse per second signal
\newduneabbrev{pps}{1PPS signal}{one-pulse-per-second signal}{An
  electrical signal with a fast rise time and that arrives in real
  time with a precise period of one second}

%spill location system
\newduneabbrev{sls}{SLS}{spill location system}{A system residing at
  the DUNE far detector site that provides information, possibly
  predictive, indicating periods of time when neutrinos are being
  produced by the \fnal Main Injector beam spills}

%warm interface board
\newduneabbrev{wib}{WIB}{warm interface board}{Digital electronics
  situated just outside the \single cryostat that receives digital data
  from the FEMBs over cold copper connections and sends it to the RCE
  FE readout hardware}

\newduneabbrev{gps}{GPS}{Global Positioning System}{A satellite-based system that provides a highly accurate \gls{pps} which may be used to synchronize clocks and determine the location.}

\newduneabbrev{ntp}{NTP}{Network Time Protocol}{A networking protocol which allows synchronizing of clocks to within a few \si{\milli\second} of a time standard on a local network and within a few tens of \si{\milli\second} over the Internet.} 

\newduneabbrev{irig}{IRIG}{inter-range instrumentation group}{A standards body which defined a time code standard for transferring timing information.}

%network interfce controller
\newduneabbrev{nic}{NIC}{network interface controller}{Hardware for controlling the interface to a communication network.  Typically, one that obeys the Ethernet protocol}

%warm interface electronics crate
\newduneabbrev{wiec}{WIEC}{warm interface electronics crate}{Crates mounted on the signal flanges that contain the warm interface boards}

%power and timing cards
\newduneabbrev{ptc}{PTC}{power and timing cards}{Cards that provide further processing and distribution of the signals entering and exiting the \single cryostat}
\newduneabbrev{ptb}{PTB}{power and timing backplane}{Backplane used to connect the \gls{wib}s and the \gls{ptc}s on the \gls{wiec}. Also connects the \gls{ce} flange on the cryostat penetration.}

%silicon photomultipler
\newduneabbrev{sipm}{SiPM}{silicon photomultiplier}{A solid-state
  avalanche photodiode sensitive to single \phel signals}

%cryogenic instrumentation and slow control
\newduneabbrev{cisc}{CISC}{cryogenic instrumentation and slow controls}{A DUNE
  consortium responsible for the cryogenic instrumentation and slow controls components}

%FTE
\newduneword{fte}{FTE}{Full Time Equivalent. A unit of labor
  for the project. One year of work from one person.}

%consortium
\newduneword{consortium}{consortium}{A unit of organization in the
  DUNE project focused on one major component of the far detector}

%art 
\newduneword{art}{\textit{art}}{A software framework implementing an
  event-based execution paradigm} %http://art.fnal.gov/

%sequential access via metadata  
\newduneabbrev{sam}{SAM}{sequential
  access via metadata}{A data-handling system to store and retrieve
  files and associated metadata, including a complete record of the
  processing that has used the files}

%art data aquisition
\newduneword{artdaq}{\textit{art}DAQ}{A general-purpose \fnal data acquisition framework for distributed data combination and processing. 
It is designed to exploit the parallelism that is possible with
modern multi-core and networked computers}

%big-O notation
\newduneword{order}{$\mathcal{O}(n)$}{of order $n$}

%\newduneword{3d}{3D}{3 dimensional (also 1D, 2D, etc.)} % not phys 
%(Use defs.tex version}
%.........

%beamline
\newduneword{beamline}{beamline}{A sequence of control and monitoring devices used for the formation of a directed collection of particles}
%conceptual design report
\newduneabbrev{cdr}{CDR}{conceptual design report}{A formal project
  document %required by funding agencies
   that describes the experiment
  at a conceptual level}

%conventional facilities
\newduneabbrev{cf}{CF}{conventional facilities}{Pertaining to
  construction and operation of buildings or caverns and conventional infrastructure}

%charge parity
\newduneabbrev{cp}{CP}{charge parity}{Product of charge and parity
  transformations}

%product of charge, parity and time-reversal
\newduneword{cpt}{CPT}{product of charge, parity
  and time-reversal transformations}

%charge-parity symmetry violation
\newduneabbrev{cpv}{CPV}{charge-parity symmetry violation}{Lack of
  symmetry in a system before and after charge and parity
  transformations are applied}

%us department of energy
\newduneword{doe}{DOE}{U.S. Department of Energy}

\newduneabbrev{fra}{FRA}{Fermi Research Alliance}{A joint partnership of the University of Chicago and the Universities Research Association (URA) that manages and operates Fermilab on behalf of the \gls{doe}}

\newduneword{us}{USA}{United States of America}

%deep underground neutrino experiment
\newduneword{dune}{DUNE}{Deep Underground Neutrino Experiment}

%environment, safety and health
\newduneword{esh}{ES\&H}{Environment, Safety and Health}

\newduneabbrev{ppe}{PPE}{personnel protective equipment}{Equipment worn to minimize exposure to hazards that cause serious workplace injuries and illnesses}

\newduneword{odh}{ODH}{Oxygen Deficiency Hazard}

\newduneword{feshm}{FESHM}{Fermilab Environmental and Safety Manual}

%fine-grained tracker
\newduneabbrev{fgt}{FGT}{fine-grained tracker}{A near detector module ??}

%far site conventional facilities
\newduneabbrev{fscf}{FSCF}{far site conventional facilities}{The
  \gslpl{cf} at the DUNE far detector site, \surf}
  
%near site conventional facilities
\newduneabbrev{nscf}{NSCF}{near site conventional facilities}{The
  \gslpl{cf} at the DUNE near detector site, \fnal}

%grand unified theory
\newduneabbrevs{gut}{GUT}{grand unified theory}{grand unified theories}{A class of theories that unifies the electro-weak and strong forces}

%liquid argon
\newduneabbrev{lar}{LAr}{liquid argon}{The liquid phase of argon}

%liquid argon time-projection chamber
\newduneabbrev{lartpc}{LArTPC}{liquid argon time-projection chamber}{A
  class of detector technology that forms the basis for the DUNE far
  detector modules. 
  It typically entails observation of ionization activity by
  electrical signals and of scintillation by optical signals}

%long-baseline
\newduneabbrev{lbl}{LBL}{long-baseline}{Refers to the distance between the 
  neutrino source  and the far detector.  It can also refer to the distance between the near and far detectors. 
  The ``long'' designation is an approximate and relative distinction. For DUNE, this distance  (between \fnal and \surf) is approximately \SI{1300}{km}}

%long-baseline neutrino facility
\newduneabbrev{lbnf}{LBNF}{Long-Baseline Neutrino Facility}{The
  organizational entity responsible for developing the neutrino beam, the cryostats
  and cryogenics systems, and the conventional facilities for DUNE}
\newduneabbrev{lbnf-dune}{LBNF/DUNE}{\gls{lbnf} and \gls{dune} project}{The overall global project, including \gls{lbnf} and \gls{dune}}

\newduneabbrev{lbnc}{LBNC}{Long-Baseline Neutrino Committee}{The
  organizational entity responsible for overseeing the \gls{lbnf} and \gls{dune}  projects. The committee reports to the \fnal director}

\newduneabbrev{ncg}{NCG}{Neutrino Cost Group}{The
  organizational entity responsible for reviewing and scrutinizing the \gls{dune}  project cost. The committee reports to the \fnal director}

%mass hierarchy
\newduneabbrev{mh}{MH}{mass hierarchy}{Describes the separation
  between the mass squared differences related to the solar and
  atmospheric neutrino problems}

%fnal main injector
\newduneabbrev{mi}{MI}{\fnal Main Injector}{An accelerator at
  \fnal that provides a beam of high-energy protons that upon
  striking a target produce secondaries that decay to provide the
  neutrinos directed toward the DUNE far detector}

%protons on target
\newduneabbrev{pot}{POT}{protons on target}{Typically used as a unit
  of normalization for the number of protons striking the neutrino
  production target}

%quality assurance
\newduneabbrev{qa}{QA}{quality assurance}{The set of actions taken to provide confidence that quality requirements are fulfilled, and to detect and correct poor results}

%quality control
\newduneabbrev{qc}{QC}{quality control}{An aggregate of activities (such as design analysis and inspection for defects) performed to ensure adequate quality in manufactured products}

%standard model
\newduneabbrev{sm}{SM}{Standard Model}{Refers to a theory describing
  the interaction of elementary particles}

%technical design report
\newduneabbrev{tdr}{TDR}{technical design report}{A formal project
  document %required by funding agencies 
  that describes the experiment at a technical level}

%interim design report
\newduneabbrev{tp}{IDR}{interim design report}{An intermediate
milestone on the path to a full \gls{tdr}} % changed from ``technical proposal'' 6/6/2018

%%%%%%%%%%%%% PROJECT AND PHYSICS VOLUME list for acronyms below %%%%%%%%%%%%
\newduneabbrev{ckm}{CKM matrix}{Cabibbo-Kobayashi-Maskawa
  matrix}{Refers to the matrix describing the mixing between mass and
  weak eigenstates of quarks}

\newduneabbrev{cl}{CL}{confidence level}{Refers to a probability
  used to determine the value of a random variable given its
  distribution}

\newduneabbrev{pmns}{PMNS}{Pontecorvo-Maki-Nakagawa-Sakata}{A type of matrix that describes the mixing between mass and weak eigenstates of
  the neutrino}


%%%%%%%%%%%%%%%%%.....................

%%%%%%%%%%%%% PROJECT AND DETECTORS VOLUME list for acronyms below %%%%%%%%%%%%

% fixme: should not have degenerate definition.  This also should be an abbrev.
%\newduneword{blm}{BLM}{(in Volume 4) beamline measurement (system); (in Volume 3) beam loss monitor} (not used -- yet! Anne 10 May 2018)
%omit trailing period in newduneabbrev(s) and newduneword, glossaries package will append the end of sentence period - Ddm

\newduneabbrevs{cpa}{CPA}{cathode plane assembly}{cathode plane assemblies}{The component of the \single detector module that provides the drift HV cathode}

\newduneabbrev{fc}{FC}{field cage}{The component of a LArTPC that contains and shapes the applied \efield}

\newduneword{cpafc}{CPA/FC}{A pair of \gls{cpa} panels and the top and bottom \gls{fc} portions that attach to the pair; an intermediate assembly for installation into the \gls{spmod} }

\newduneabbrev{topfc}{top FC}{top field cage}{The horizontal portions of the \single FC on the top}

\newduneabbrev{botfc}{bottom FC}{bottom field cage}{The horizontal portions of the \single FC on the bottom}

\newduneabbrev{ewfc}{endwall FC}{endwall field cage}{The vertical portions of the \single FC near the wall}

\newduneabbrev{gp}{GP}{ground plane}{An electrode that is held to be
  electrically neutral relative to Earth ground voltage}

  \newduneword{gg}{ground grid}{An electrode that is held to be
  electrically neutral relative to Earth ground voltage  ?? fix def or can we just use gp def?}


\newduneabbrev{alara}{ALARA}{as low as reasonably
  achievable}{Typically used with regard management of radiation
  exposure but may be used more generally. It means making every
  reasonable effort to maintain e.g., exposures, to as far below the
  limits as practical, consistent with the purpose for that the
  activity is undertaken}

\newduneabbrev{ecal}{ECAL}{electromagnetic calorimeter}{A detector
  component that measures energy deposition of traversing particles}

\newduneabbrev{hv}{HV}{high voltage}{Generally describes a voltage
  applied to drive the motion of free electrons through some media}

% can also use in the text: \dword{sp} \dword{detmodule} 
\newduneword{spmod}{SP module}{single-phase detector module}
\newduneword{dpmod}{DP module}{dual-phase detector module}
%\newduneword{dsp}{DUNE-SP}{The \single DUNE detector} % No they are det modules!
%\newduneword{ddp}{DUNE-DP}{The \dual DUNE detector}% No they are det modules!

\newduneabbrev{tcoord}{TC}{technical coordinator}{Responsible for organization of
the project effort; is a member of the \gls{dune} management team}

\newduneabbrev{rcoord}{RC}{resource coordinator}{Responsible for financing of
the project effort; is a member of the \gls{dune} management team}

\newduneabbrev{tc}{TCN}{technical coordination}{Responsible for overall integration 
of the detector elements and successful execution of the detector
construction project; areas of responsibility include 
general project oversight, systems engineering, \gls{qa} 
and safety}

\newduneabbrev{exb}{EB}{executive board}{The highest level DUNE
  decision making body for the collaboration}

\newduneabbrev{tb}{TB}{technical board}{The DUNE organization responsible for
  evaluating technical decisions}

\newduneabbrev{rrb}{RRB}{resource review board}{The organization of DUNE funding agencies responsible for funding decisions}


%%%%%%%%%%%%% PHYSICS AND DETECTORS VOLUME list for acronyms below %%%%%%%%%%%%

\newduneabbrev{cc}{CC}{charged current}{Refers to an interaction
  between elementary particles where a charged weak force carrier
  ($W^+$ or $W^-$) is exchanged}


\newduneabbrev{dis}{DIS}{deep inelastic scattering}{Refers to
  interaction of an elementary charged particle with a nucleus in an
  energy range where the interaction can be modeled as being with
  individual nucleons}

\newduneabbrev{fsi}{FSI}{final-state interactions}{Refers to
  interactions between elementary or composite particles subsequent to
  the initial, fundamental particle interaction, such as may occur as
  the products exit a nucleus}

\newduneword{geant4}{Geant4}{A
  software toolkit for the simulation of the passage of particles
  through matter using Monte Carlo methods}

\newduneabbrev{genie}{GENIE}{Generates Events for Neutrino Interaction
  Experiments}{Software providing an object-oriented neutrino
  interaction simulation resulting in kinematics of the products of
  the interaction}

\newduneabbrev{mc}{MC}{Monte Carlo}{Refers to a method of numerical
  integration that entails the statistical sampling of the integrand
  function. 
  Forms the basis for some types of detector and physics simulations}

\newduneabbrev{qe}{QE}{quasi-elastic}{Refers to interaction between
  elementary particles and a nucleus in an energy range where the
  interaction can be modeled as occurring between constituent quarks
  of one nucleon and resulting in no bulk recoil of the resulting
  nucleus}

%%%%%%%%%%%%%%%%%%%%%%%%% PROJECT VOLUME list for acronyms below %%%%%%%%%%%%%%%

\newduneabbrev{mou}{MoU}{memorandum of understanding}{A document
  summarizing an agreement between two or more parties}

\newduneabbrev{pip2}{PIP-II}{Proton Improvement Plan II}{A plan for
  improving the protons on target delivered for the DUNE neutrino
  production beam. 
  This is revision two of this plan and is planned to be followed by a PIP-III}

\newduneabbrev{sdsta}{SDSTA}{South Dakota Science and Technology
  Authority}{The legal entity that manages the Sanford Underground
  Research Facility, \surf, in Lead, S.D}
  
\newduneabbrev{sdsd}{SDSD}{Fermilab South Dakota Services Division}{A Fermilab division responsible providing host laboratory functions at the far site}

\newduneabbrev{firus}{FIRUS}{Facility Information Reporting Utility System}
 {The safety system at \surf}

\newduneabbrev{bsi}{BSI}{Building and Site Infrastructure}
 {The work package for outfitting of the \dword{lbnf} underground infrastructure.}



\newduneabbrev{wbs}{WBS}{work breakdown structure}{An organizational
  project management tool by which the tasks to be performed are
  partitioned in a hierarchical manner}

%%%%%%%%%%%%%%%%%%%%%%%%% PHYSICS VOLUME list for acronyms below %%%%%%%%%%%%%%%
\newduneabbrev{br}{BR}{branching ratio}{A fractional probability for a
  decay of a composite particle to occur into some specified set or
  sets of products}
\newduneword{bsm}{BSM}{beyond the standard model}

\newduneabbrev{dm}{DM}{dark matter}{The term given to the unknown
  matter or force that explains measurements of motion of galaxies
  that are otherwise inconsistent with the amount of mass associated
  with observed amount of photon production}
  
  \newduneabbrev{bdm}{BDM}{dark matter}{Add definition}

\newduneabbrev{cern}{CERN}{European Organization for Nuclear
Research}{In French, Organisation europ\'{e}enne pour la recherche
nucl\'{e}aire, derived from Conseil Europ`{e}en pour la Recherche
Nucl\`{e}aire, the leading particle physics laboratory in Europe and home
to the ProtoDUNEs}


\newduneabbrev{dsnb}{DSNB}{Diffuse Supernova Neutrino Background}{The
  term describing the pervasive, constant flux of neutrinos due to all
  past supernova neutrino bursts}

\newduneabbrev{espp}{ESPP}{European Strategy for Particle Physics}{The
European Strategy for Particle Physics is the cornerstone of Europe's
decision-making process for the long-term future of the
field. Mandated by the CERN Council, it is formed through a broad
consultation of the grass-roots particle physics community, it
actively solicits the opinions of physicists from around the world,
and it is developed in close coordination with similar processes in
the US and Japan in order to ensure coordination between regions and
optimal use of resources globally.}

\newduneword{gar}{GAr}{gaseous argon}
\newduneabbrev{gartpc}{GArTPC}{gaseous argon time-projection
chamber}{A possible technology choice for the \gls{nd}.}


\newduneabbrev{globes}{GLoBES}{General Long-Baseline Experiment
  Simulator}{A software package for simulating energy spectra of
  neutrino flux, interaction and measured (after application of some
  model of a detector response) energy spectra}

\newduneword{snowglobes}{SNOwGLoBES}{SuperNova
Observatories with \gls{globes}. From the official description~\cite{snowglobes}: 
SNOwGLoBES is public software for computing interaction rates and distributions of observed quantities for supernova burst neutrinos in common detector materials. The intent is to provide a very simple and fast code and data package which can be used for tests of observability of physics signatures in current and future detectors, and for evaluation of relative sensitivities of different detector configurations. The event estimates are made using available cross-sections and parameterized detector responses. Water, argon, scintillator and lead-based configurations are included. The package makes use of GLoBES front-end software. SNOwGLoBES is not intended to replace full detector simulations; however output should be useful for many types of studies}


% are these really used anywhere?
\newduneword{l/e}{L/E}{length-to-energy ratio}
\newduneword{lri}{LRI}{long-range interactions}
%\newduneword{solarmass}{$M_{\odot}$}{solar mass}

\newduneabbrev{nc}{NC}{neutral current}{Refers to an interaction
  between elementary particles where a neutrally charged weak force carrier
  ($Z^0$) is exchanged}

\newduneabbrev{nh}{NH}{normal hierarchy}{Refers to the neutrino mass
  eigenstate ordering whereby the sign of the mass squared difference
  associated with the atmospheric neutrino problem is positive}

\newduneabbrev{ih}{IH}{inverted hierarchy}{Refers to the neutrino mass
  eigenstate ordering whereby the sign of the mass squared difference
  associated with the atmospheric neutrino problem is negative}

\newduneabbrev{msw}{MSW}{Mikheyev-Smirnov-Wolfenstein effect}{Explains
  the oscillatory behavior of neutrinos produced inside the sun as
  they traverse the solar matter}

\newduneabbrev{nsi}{NSI}{nonstandard interactions}{A general class of
  theory of elementary particles other than the Standard Model}



\newduneabbrev{pfive}{P5}{Particle Physics Project Prioritization
Panel}{The Particle Physics Project Prioritization Panel (P5) was a
subpanel of the High Energy Physics Advisory Panel (HEPAP). It completed
its Report, a ten-year strategic plan for high energy physics in the
U.S., in 2014. This report included a recommendation that ``host a world-leading neutrino
program that will have an optimized set of short- and long-baseline neutrino oscillation experiments, and its long-term focus
is a reformulated venture referred to here as the Long Baseline
Neutrino Facility (LBNF)''}

\newduneword{sme}{SME}{Standard-Model Extension}

\newduneabbrev{susy}{SUSY}{supersymmetry}{Theoretical symmetry between a fermion and a boson}

\newduneabbrev{wimp}{WIMP}{weakly-interacting massive particle}{A
  hypothesized particle that may be a component of dark matter}

%%%%%%%%%%%%%%%%%%%%%%%%% DETECTORS VOLUME list for acronyms below %%%%%%%%%%%%%%%

\newduneabbrev{ce}{CE}{cold electronics}{Refers to readout electronics that operate at cryogenic temperatures}

\newduneabbrev{crp}{CRP}{charge-readout plane}{In the \dual technology, a  collection of
  electrodes in a planar arrangement placed at a particular voltage
  relative to some applied \efield such that drifting electrons
  may be collected and their number and time may be measured}

\newduneabbrev{dram}{DRAM}{dynamic random access memory}{A computer memory technology}

% these should be abbrev and maybe combined.
\newduneword{fermilab}{\fnal}{Fermi National Accelerator Laboratory (in Batavia, IL, the Near Site)}
\newduneword{fnal}{\fnal}{Fermi National Accelerator Laboratory (in Batavia, IL, the Near Site)}
%should probably get rid of either fnal or fermilab here.
\newduneword{bnl}{BNL}{Brookhaven National Laboratory (in Upton, NY)}
\newduneword{slac}{SLAC}{SLAC National Accelerator Laboratory (in Menlo Park, CA)}
\newduneword{lbnl}{LBNL}{Lawrence Berkeley National Laboratory (in Berkeley, CA)}
\newduneword{anl}{ANL}{Argonne National Laboratory (in Lemont, IL)}
\newduneabbrev{fs}{FS}{full stream}{Relates to a data stream that has not undergone selection, compression or other form of reduction}

\newduneabbrev{lem}{LEM}{large electron multiplier}{A micro-pattern detector suitable for use in ultra-pure argon vapor; LEMs consist of copper-clad PCB boards with sub-millimeter-size holes through which electrons undergo amplification}
%\newduneabbrev{lem}{LEM}{large electron multiplier}{A technique of electron multiplication via avalanche in sub-millimeter-size holes in a
%millimeter-thick printed circuit board where electric fields are understood}
\newduneabbrev{lng}{LNG}{liquefied natural gas}{Pertaining to natural gas in its liquid phase}

% this is pretty generic and isn't referenced.
%\newduneword{mesh}{mesh screen}{A fine mesh screen, glued directly to
%  the steel frame on both sides of each APA in the single-phase TPC,
% creates a uniform ground layer beneath the wire planes}

\newduneabbrev{mip}{MIP}{minimum ionizing particle}{Refers to a
  momentum traversing some medium such that the particle is losing
  near the minimum amount of energy per distance traversed} % some \mip and some \dword{mip}. If time, rectify. ??

%\newduneword{abc}{MTS}{Materials Test Stand}
%\newduneword{muid}{MuID}{muon identifier (detector)}

%\newduneword{abc}{OPERA}{Oscillation Project with Emulsion-Racking Apparatus (experiment at LNGS)}
%\newduneword{abc}{NND}{(used only in Volume 4 Chapter 7) near neutrino detector, same as ND}
%\newduneword{abc}{OD}{outer diameter}

% there is also pds
\newduneabbrev{pd}{PD}{photon detector}{Refers to the detector
  elements involved in measurement of number and arrival times of
  optical photons produced in a detector module} %volume}

\newduneabbrev{pmt}{PMT}{photomultiplier tube}{A device that makes use
  of the photoelectric effect to produce an electrical signal from the
  arrival of optical photons}

\newduneabbrev{ppm}{ppm}{parts per million}{A number equal to $10^{-6}$}
\newduneabbrev{ppb}{ppb}{parts per billion}{A number equal to $10^{-9}$}
\newduneabbrev{ppt}{ppt}{parts per trillion}{A number equal to $10^{-12}$}

\newduneword{rio}{RIO}{reconfigurable input output}

\newduneword{rpc}{RPC}{resistive plate chamber}
\newduneword{s/n}{S/N}{signal-to-noise (ratio)}
\newduneword{ssp}{SSP}{SiPM signal processor}
\newduneword{sbn}{SBN}{Short-Baseline Neutrino program (at \fnal)}
\newduneword{stt}{STT}{straw tube tracker}
%\newduneword{abc}{SURF (also Sanford Lab)}{Sanford Underground Research Facility (in Lead, SD, the Far Site)}
\newduneword{tr}{TR}{transition radiation}
%\newduneword{abc}{W}{Watt (also mW, kW, MW) }
%\newduneword{abc}{WA105}{Single-Phase LArTPC and the Long Baseline Neutrino Observatory Demonstration}
\newduneword{wire board}{wire board}{At the head end of the APA in the \single TPC, stacks of electronics boards referred to as ``wire boards'' are arrayed to anchor the wires.  They also provide the connection between the wires and the cold electronics} %?? long for a word. ??

\newduneabbrev{wls}{WLS}{wavelength shifting}{A material or process by
  which incident photons are absorbed by a material and photons are
  emitted at a different, typically longer, wavelength}
  
\newduneabbrev{tpb}{TPB}{tetra-phenyl butadiene}{A type of wavelength shifting material}

\newduneabbrev{ptp}{PTP}{p-terphenyl}{A type of wavelength shifting material}

\newduneabbrev{sft}{SFT}{signal feedthrough}{A cryostat penetration allowing for the passage of cables or other extended parts}
\newduneabbrev{sftchimney}{SFT chimney}{signal feedthrough chimney}{In the \dual technology, a volume above the cryostat penetration used for a signal feedthrough}


\newduneword{catiroc}{CATIROC}{charge and time integrated readout chip}

\newduneabbrev{wr}{WR}{White Rabbit}{A component of the timing system that forwards clock signal and time-of-day reference data to the master timing unit}

\newduneabbrev{mch}{MCH}{MicroTCA Carrier Hub}{An network switching device}

%\newduneabbrev{wrmch}{WR-MCH}{White Rabbit \gls{utca} Carrier Hub}{add def} %??
\newduneabbrev{wrmch}{WR-MCH}{White Rabbit \gls{utca} Carrier Hub}{A card mounted in \gls{utca} crate that recieves time syncronization information and trigger data packets over \gls{wr} network and disributes them to the \gls{amc} over \gls{utca} backplane} 

\newduneabbrev{wrtsn}{WR-TSN}{White Rabbit TimeStamping Node}{A unit on the \gls{wr} network that timestamps the trigger signals and sends out trigger data packets to \gls{wrmch}}

\newduneword{cmp}{CMP}{configuration management plan}

\newduneword{qap}{QAP}{quality assurance plan} %{A project management device for planning \gls{qa}}

\newduneword{ieshp}{IESHP}{integrated environmental, safety and health plan}%{Refers to the LBNF/DUNE project planning instrument}

\newduneword{dmp}{DMP}{data management plan} %{A project management device to state how the experimental data will be managed}

\newduneword{qam}{QAM}{quality assurance manager} %{The manager of \gls{qa} for the LBNF/DUNE project}

\newduneabbrev{dss}{DSS}{detector support system}{The system used to support the \single detector within the cryostat}

\newduneabbrev{ddss}{DDSS}{\gls{dune} detector safety system}{The system used to manage key aspects of detector safety}

\newduneabbrev{itf}{ITF}{integration and test facility}{A facility where various detector components will be tested prior to installation}

\newduneabbrev{lc}{LC}{logistics center}{A facility where \gls{lbnf} and \gls{dune} components will be received and transhipped to \gls{surf}}

\newduneabbrev{tco}{TCO}{temporary construction opening}{An opening in the side of a cryostat through which detector elements are brought into the cryostat; utilized during construction and installation}

\newduneabbrev{surf}{SURF}{Sanford Underground Research Facility}{The laboratory in South Dakota where \gls{lbnf} will be constructed}

\newduneabbrev{sit}{SIT}{surface installation team}{An organizational unit responsible for logistics and integration in South Dakota}

\newduneabbrev{uit}{UIT}{underground installation team}{An organizational unit responsible for installation in the underground area at the \surf site}

\newduneabbrev{cmgc}{CMGC}{construction manager/general contractor}{The organizational unit responsible for management of the construction of conventional facilities at the underground area at the \surf site}

\newduneword{cdrev}{conceptual design review}{A project management device by which a conceptual design is reviewed} % anne changed - was CDR which has another meaning; see cdr.
\newduneabbrev{pdr}{PDR}{preliminary design review}{A project management device by which an early design is reviewed} % do we want this for `review' not `report'?
\newduneabbrev{fdr}{FDR}{final design review}{A project management device by which a final design is reviewed}
\newduneabbrev{prr}{PRR}{production readiness review}{A project management device by which the production readiness is reviewed}
\newduneabbrev{irr}{IRR}{installation readiness review}{A project management device by which the plan for installation is reviewed}
\newduneabbrev{orr}{ORR}{operational readiness review}{A project management device by which the operational readiness is reviewed}
\newduneabbrev{ppr}{PPR}{production progress review}{A project management device by which the progress of production is reviewed}
\newduneabbrev{edms}{EDMS}{engineering document management system}{A computerized system deveolped at CERN by which documents, drawings and models are managed}

\newduneword{wrgm}{WR grandmaster}{White Rabbit grandmaster}


%%%%% Software and computing %%%%

\newduneword{larsoft}{\larsoft}{Liquid Argon Software (\larsoft),  a shared base of physics software across \lartpc experiments}
\newduneword{nova}{\nova}{The \nova off-axis neutrino oscillation experiment at \fnal }
\newduneword{minerva}{\minerva}{The \minerva neutrino cross sections experiment at \fnal }
\newduneword{microboone}{\microboone}{The \lartpc-based \microboone neutrino oscillation experiment at \fnal }
\newduneword{sbnd}{SBND}{The Short-Baseline Near Detector experiment at \fnal}
\newduneword{nexo}{nEXO}{Enriched Xenon Observatory}
\newduneword{argoneut}{ArgoNeuT}{The ArgoNeuT test-beam experiment and \gls{lar} \gls{tpc} prototype at \fnal}
\newduneword{icarus}{ICARUS}{add def}
\newduneword{atlas}{ATLAS}{add def}
\newduneword{lbne}{LBNE}{add def}
\newduneword{lbno}{LBNO}{add def}


\newduneword{wirecell}{wire-cell}{A tomographic automated \threed neutrino event reconstruction method for \lartpc{}s}
\newduneword{ftslite}{F-FTS-lite}{Light-weight version of the \fnal File Transfer system used for rapid data transfers out of the online systems}
\newduneword{fts}{FTS}{File Transfer System developed at \fnal to catalog and move data to permanent storage}

%%% new ones that I haven't categorized (Anne)
\newduneword{35t}{35 ton prototype}{The 35 ton prototype cryostat and \gls{sp} detector built at \fnal before the \gls{protodune} detectors}

\newduneabbrev{cuc}{CUC}{central utility cavern}{The central underground cavern  containing utilities such as central cryogenics and other systems, and the underground data center and control room}
\newduneabbrev{cfd}{CFD}{computational fluid dynamics}{High performance computer-assisted modeling of fluid dynamical systems}
\newduneword{vuv}{VUV}{vacuum ultra-violet}
\newduneword{tallbo}{TallBo}{A cylindrical cryostat at \fnal primarily used for developing scintillation light collection technologies for \lartpc detectors}

\newduneword{root}{ROOT}{add definition}

\newduneabbrev{eos}{EOS}{EOS}{The XRootD based distributed file system developed by CERN}
\newduneabbrev{ehn1}{EHN1}{Experiment Hall North One}{Location at CERN of the ProtoDUNE experiments}
\newduneword{led}{LED}{Light-emitting diode}
\newduneword{rtd}{RTD}{Cryostat level and temperature monitoring devices}
\newduneword{swc}{SWC}{Software \& Computing}
\newduneabbrev{las}{LAS}{LEM-anode Sandwich}{In the \dual technology, a \gls{lem} and its corresponding anode are mounted together in a module called a LEM-anode sandwich}

\newduneword{roi}{ROI}{region of interest}
\newduneword{hpc}{HPC}{high-performance computing facilities; generally computing facilities emphasizing parallel computing with aggregate power of more than a teraflop}


\newduneword{comfund}{common fund}{The shared resources of the collaboration}
\newduneabbrev{ims}{IMS}{integrated master schedule}{A project management device consisting of linked tasks and milestones}

\newduneword{hvdb}{HVDB}{dual-phase HV divider board}
\newduneword{sas}{SAS}{A pass-through chamber used to ensure safe transfer of materials, avoiding contamination in both directions}

\newduneword{fea}{FEA}{finite element analysis}

\newduneword{fss}{FSS}{field shaping strips}
\newduneword{lvds}{LVDS}{low-voltage differential signaling}



%electrostatic discharge
\newduneword{esd}{ESD}{electrostatic discharge}%{ESD is the sudden flow of electricity between two electrically charged objects caused by contact, an electrical short, or dielectric breakdown. ESD can cause failure of electronic components such as integrated circuits}

\newduneabbrev{rp}{RP}{resistive panel}{Resistive panels form the constant potential surfaces for a \gls{spmod} \gls{cpa}; they are composed of a thin layer of carbon-impregnated Kapton and laminated to both sides of a \frfour sheet. }

\newduneword{uhmwpe}{UHMWPE}{ultra-high molecular weight polyethylene}

\newduneword{cts}{CTS}{Cryogenic Test System}
\newduneword{plc}{PLC}{programmable logic controller}

\newduneword{mppc}{MPPC}{\SI{6}{mm}$\times$\SI{6}{mm} Multi-Pixel Photon Counters produced by Hamamatsu\texttrademark{} Photonics K.K.}

\newduneabbrev{sfp}{SFP}{small form-factor pluggable (SFP)}{a particular standard for optical transceivers}

\newduneabbrev{minipod}{MiniPOD}{miniature parallel optical device}{a family of types of multi-channel optical transceivers}

\newduneword{ccc}{CCC}{configuration change command}
\newduneword{act}{ACT}{activation time stamp}
\newduneword{lcm}{LCM}{light calibration module}
\newduneword{lpm}{LPM}{light pulser module}
\newduneword{dac}{DAC}{digital-to-analog converter}
\newduneword{sarapu}{S-ARAPUCA}{ARAPUCA design with different wavelength shifter coatings on both faces of the dichroic filter window(s) of the cell}
\newduneword{xarapu}{X-ARAPUCA}{ARAPUCA design with different wavelength shifter coating on only the external face of the dichroic filter window(s) but with a wavelength shifter doped plate inside the cell}
\newduneword{feb}{FEB}{front-end board}

%\newduneword{gdml}{GDML}{needs def}
\newduneword{lsnd}{LSND}{needs def}

\newduneword{cvn}{CVN}{convolutional visual network}
\newduneword{pandora}{Pandora}{needs def - software development kit...}

%Lisa added
\newduneabbrev{pma}{PMA}{Projection Matching Algorithm}{A reconstruction algorithm that combines \twod reconstructed objects to form a \threed representation}
\newduneabbrev{bdt}{BDT}{Boosted Decision Tree}{A method of multivariate analysis}
\newduneabbrev{cnn}{CNN}{Convolutional Neural Network}{A deep learning technique most commonly applied to analyzing visual imagery}
\newduneword{pdg}{PDG}{Particle Data Group}

% from CISC
\newduneword{pci}{PCI}{Peripheral Component Interconnect}

\newduneword{labview}{LabVIEW}{Laboratory Virtual Instrument Engineering Workbench is a system-design platform and development environment for a visual programming language from National Instruments}

\newduneword{pcb}{PCB}{printed circuit board}

\newduneword{crio}{cRIO}{Compact Reconfigurable Input Output}

\newduneword{dcs}{DCS}{Distributed Communications System}

\newduneword{opc-ua}{OPC-UA}{OPC  Unified Architecture is a machine to machine communication protocol for industrial automation developed by the OPC Foundation. OPC stands for Object Linking and Embedding for Process Control}

\newduneword{cabangle}{Cabibbo angle}{add definition- from nu osc 01 or 02}
\newduneword{valor}{VALOR}{add def (from nu-osc 03)}
\newduneword{cafana}{CAFAna}{add def (from nu-osc 03)}
\newduneabbrev{pca}{PCA}{principal component analysis}{add def (from nu-osc 03)}
\newduneword{numi}{NuMI}{a set of facilities, collectively called ``Neutrinos at the Main Injector.''  The NuMI neutrino beamline target system converts an intense proton beam into a focused neutrino beam}
\newduneword{gibuu}{GiBUU}{Giessen Boltzmann-Uehling-Uhlenback Project; a unified theory and transport framework in the MeV and GeV energy regimes for elementary reactions on nuclei }
\newduneword{crpa}{CRPA}{add def (from nu-osc 05)}
\newduneword{rpa}{RPA}{add def (from nu-osc 05)}
\newduneword{t2k}{T2K}{T2K (Tokai to Kamioka) is a long-baseline neutrino experiment in Japan studying neutrino oscillations }
\newduneword{mptdet}{MPT detector}{multipurpose tracking detector}
\newduneword{hpg}{HPG}{high-pressure gas? add definition- from nu osc 06}
\newduneword{tdc}{TDC}{add def nu osc 07}
\newduneword{lariat}{LArIAT}{add def (from nu-osc 09)}
\newduneword{captain}{CAPTAIN}{add def (from nu-osc 09)}
\newduneword{dayabay}{Daya Bay}{add def (from nu-osc 10)}
\newduneword{nuwro}{NuWro}{neutrino interaction generatoradd def (from nu-osc 11)}
\newduneword{neut}{NEUT}{neutrino interaction generator, add def (from nu-osc 11)}
\newduneword{minos}{MINOS}{add def}
\newduneword{mpt}{MPT}{add def}


\newduneabbrev{efig}{EFIG}{Experimental Facilities Interface Group}{The body responsible for the required high-level coordination between the \gls{lbnf} and \gls{dune} projects.}
\newduneword{ashriver}{Ash River}{The Ash River, Minnesota, USA \gls{nova} experiment far site, used as an assembly test site for \gls{dune}} 
\newduneabbrev{ipd}{PI-DIR}{Project Integration Director}{Responsible for integration and installation of \gls{lbnf} and \gls{dune} deliverables in South Dakota. Manages the \gls{jpo}}
\newduneabbrev{jpo}{JPO}{Joint Project Office}{The office formed from members of the \gls{lbnf} project and \gls{dune} 
\gls{tc} teams to direct integration and installation of the \gls{fd} modules. Its functions include global project configuration and integration, installation planning and
coordination, scheduling, safety assurance, technical review planning
and oversight, development of partner agreements, and financial
reporting}

\newduneword{ifbeam}{IFbeam}{Database that stores beamline information
indexed by timestamp}

\newduneabbrev{marley}{MARLEY}{Model of Argon Reaction Low Energy
Yields}{Developed at UC Davis, MARLEY is the first realistic model of
neutrino electron interactions on argon for enegies less than 50
MeV. This includes the energy range important for supernova burst
neutrinos and also solar 8--Boron neutrinos}

\newduneabbrev{es}{ES}{elastic scattering}{Events in which a neutrino
elastically scatters off of another particle}


\newduneword{cno}{CNO}{The CNO cycle (for carbon–nitrogen–oxygen) is
one of the two known sets of fusion reactions by which stars convert
hydrogen to helium, the other being the proton–proton chain reaction
(pp-chain reaction). In the CNO cycle, four protons fuse, using
carbon, nitrogen, and oxygen isotopes as catalysts, to produce one
alpha particle, two positrons and two electron neutrinos}

\newduneabbrev{sdwf}{SDWF}{South Dakota Warehouse Facility}{Warehousing operations in South Dakota responsible for receiving LBNF/DUNE goods and coordinating shipping to the Ross shaft.}

\newduneabbrev{wms}{WMS}{warehouse management system}{Commercial software package used to track shipments and interface to freight forwarders. This includes a database for shipping.}

\newduneabbrev{dcdb}{DCDB}{DUNE construction database}{Database used by DUNE to track the history and testing of all parts of the detectors.}

\newduneabbrev{aup}{AUP}{acceptance for use and possession}{Beneficial occupancy of the underground areas for LBNF and DUNE.}

%\newduneabbrev{wec}{WEC}{Warm Electronics Crate}{Electronics crate mounted on the feedthru Tees which hold the warm interface boards and power timing cards.} %get rid of this - weic

\newduneabbrev{sno}{SNO}{Sudbury Neutrino Observatory}{The Sudbury
Neutrino Observatory was a detector built 6800 feet under ground, in
INCO's Creighton mine near Sudbury, Ontario, Canada. SNO was a
heavy-water Cherenkov detector designed to detect neutrinos produced
by fusion reactions in the sun}

\newduneword{sk}{Super-Kamiokande}{From the official website~\cite{skwebsite}: Super-Kamiokande is a large water
Cherenkov detector.  The Super-Kamiokande detector consists of a
stainless-steel tank,39.3 m diameter and 41.4 m tall,
filled with 50,000 tons of ultra pure water. About 13,000
photo-multipliers are installed on the tank wall. The detector is
located at 1,000 meter underground in the Kamioka-mine, Hida-city,
Gifu, Japan}

\newduneabbrev{id}{ID}{inner diameter}{Inner diameter of a tube.}

\newduneabbrev{od}{OD}{outer diameter}{Outer diameter of a tube.}


\newduneabbrev{rms}{RMS}{root mean square}{The square root of the arithmetic mean of the squares of a set of values, used as a measure of the typical magnitude of a set of numbers, regardless of their sign.}

\newduneabbrev{orc}{ORC}{operational readiness clearance}{Final safety approval prior to the start of operation.}

\newduneabbrev{gsc}{GSC group}{global safety coordination group}{Evaluates applicable codes and standards including international code equivalency for the design, assembly, and installation of the Far Detector.}

\newduneabbrev{ha}{HA}{hazard analysis}{A first step in a process to assess risk; the result of hazard analysis is the identification of different types of hazards}
\newduneword{har}{HAR}{Hazard Analysis Report}
\newduneabbrev{tap}{TAP}{trip action plan}{add def}
\newduneword{em}{EM}{emergency management}
\newduneword{ert}{ERT}{emergency rescue team}

% from DP-PDS --begin
\newduneabbrev{ndk}{NDK}{nucleon decay}{The hypothetical, baryon number violating decay of a proton or a bound neutron into lighter particles}

\newduneabbrev{emi}{EMI}{electromagnetic interference}{Disturbance generated by an external source that affects an electrical circuit by electromagnetic induction, electrostatic coupling, or conduction}

\newduneabbrev{pe}{PE}{photoelectron}{An electron ejected from the surface of a material by the photoelectric effect}

\newduneabbrev{spe}{SPE}{single photoelectron}{A single photoelectron}

\newduneabbrev{fwhm}{FWHM}{full width at half maximum}{Width of a distribution measured between those points at which the distribution is equal to half of its maximum amplitude}

\newduneabbrev{gdml}{GDML}{geometry description markup language}{Application-indepedent geometry description format based on XML}

\newduneabbrev{xml}{XML}{extensible markup language}{A markup language that defines a set of rules for encoding documents in a format that is both human-readable and machine-readable}

\newduneabbrev{crt}{CRT}{cosmic ray tagger}{Detector external to the TPC designed to tag TPC-traversing cosmic ray particles}

\newduneabbrev{sn}{SN}{supernova}{Event that occurs upon the death of certain types of stars}

\newduneabbrev{wg}{WG}{working group}{A group of persons working together to achieve specified goals}

\newduneabbrev{ctsf}{CTSF}{Coating, Testing and Storage Facility}{A facility where the photodetectors of the \dual \gls{pds} will be coated, tested and stored.}

% from DP-PDS --end


% from Schellman

\newduneword{rucio}{Rucio}{Data management system originally developed
by ATLAS but now open-source and shared across HEP}
\newduneabbrev{doma}{DOMA}{Data Organization, Management, and
Access}{Data Organization, Management, and Access efforts through the
HEP Software Foundation}
\newduneabbrev{hsf}{HSC}{High Energy Physics Software Foundation}{High
Energy Physics Software Foundation}
\newduneabbrev{wlcg}{WLCG}{Worldwide LHC Computing Grid}{Worldwide LHC
Computing Grid}
\newduneabbrev{osg}{OSG}{Open Science Grid}{Open Science Grid}
\newduneabbrev{sci}{SCI}{Scientific Computing Infrastructure}{Proposed
extension of the infrastructure component of \gls{wlcg} to other
experiments}
\newduneabbrev{csc}{CSC}{Computing and Software Consortium}{DUNE
Computing and Software Consortium}

\newduneword{dirac}{DIRAC}{Computing workflow management designed for
LHCb and now used by many HEP experiments}

% from DP-HV --start
\newduneword{frp}{FRP}{fiber-reinforced plastic}
\newduneabbrev{hdpe}{HDPE}{add full name}{add def}

\newduneword{hvps}{HVPS}{\gls{hv} power supply}
\newduneword{aisi}{AISI}{American Iron and Steel Institute}
\newduneword{ific}{IFIC}{Instituto de Fisica Corpuscular (in Valencia, Spain)}
\newduneabbrev{rsds}{RSDS}{radioactive source deployment system}{Proposed calibration system based on the deployment of
radioactive sources inside the \gls{dune} cryostat.}
\newduneword{2p2h}{2p2h}{two particle, two hole}
\newduneabbrev{duneprism}{DUNE-PRISM}{\gls{dune} Precision Reaction-Independent Spectrum Measurement}{a mobile near detector that can perform measurements over a range of angles off-axis from the neutrino beam direction in order to sample many different neutrino energy distributions}
\newduneword{arcube}{ArgonCube}{The name of the core part of the \gls{dune} \gls{nd}, a \gls{lartpc}}

\newduneabbrev{citf}{CITF}{cryogenic instrumentation test facility}{A facility at \fnal with small ($<\,\SI{1}{ton}$) to intermediate ($\sim\,\SI{1}{ton}$) volumes of instrumented, purified TPC-grade \lar, used for testing devices intended for use in \gls{dune}}

\newduneabbrev{3dst}{3DST}{3D scintillator tracker}{The core part of the 3D projection scintillator tracker spectrometer}
\newduneabbrev{3dsts}{3DST-S}{3D scintillator tracker spectrometer}{The 3D projection scintillator tracker spectrometer}
\newduneword{mpd}{MPD}{multi-purpose detector}
\newduneword{hpgtpc}{HPgTPC}{high-pressure gaseous argon \gls{tpc}}
\newduneword{src}{SRC}{short-range correlated nucleon-nucleon interactions}
\newduneword{larpix}{LArPix}{ \gls{asic} pixelated charge readout for a \gls{tpc} }
\newduneword{arclt}{ArCLight}{a light detector \gls{arcube} effort}
\newduneword{fhc}{FHC}{forward horn current ($\numu$ mode)}
\newduneword{rhc}{RHC}{reverse horn current ($\overline{\nu}_{\mu}$ mode)}
\newduneword{mwpc}{MWPC}{multi-wire proportional chamber}
\newduneword{na61}{NA61}{CERN hadron production experiment}
\newduneword{pdnd}{ProtoDUNE-ND}{a prototype \gls{dune} \gls{nd}}
\newduneword{ccqe}{CCQE}{charged current quasielastic interaction} 
\newduneabbrev{roc}{ROC}{readout chamber}{readout chamber for gaseous argon \gls{tpc}}
\newduneabbrev{iroc}{IROC}{inner readout chamber}{inner (radial) readout chamber for gaseous argon \gls{tpc}}
\newduneabbrev{oroc}{OROC}{outer readout chamber}{outer (radial) readout chamber for gaseous argon \gls{tpc}}

\newduneword{lux}{LUX}{add def}
\newduneword{mjdemo}{Majorana Demonstrator}{add def}
\newduneword{lz}{LZ}{add def}
\newduneword{mu2e}{Mu2e}{add def}
\newduneword{pdsp2}{ProtoDUNE-2}{A second test run in the singe-phase ProtoDUNE test stand at CERN, acting as a validation of the final single-phase detector design}
\newduneword{osha}{OSHA}{Occupational Safety and Health Administration (USA Department of Labor) formed by the Occupational Safety and Health Act of 1970.}
\newduneabbrev{pns}{PNS}{pulsed neutron source}{Calibration system based
on neutron capture gamma showers spread out in the whole detector.}

\newduneabbrev{fv}{FV}{fiducial volume}{Detector volume within the \gls{tpc},
that is selected for physics analysis, through cuts on reconstructed event position}

\newduneword{p6}{P6}{framework used to plan and status the resource-loaded schedule of activities associated with the USA contributions to \gls{lbnf} and \gls{dune} }
\newduneword{evms}{EVMS}{earned value management system}


\newduneword{core}{CORE}{add def}
\newduneabbrev{ahj}{AHJ}{Authority Having Jurisdiction}{add def}
\newduneword{cte}{CTE}{coefficient of thermal expansion}

\newduneword{opc}{OPC}{Open Platform Communications is a series of standards and specifications for industrial telecommunication.} 
\newduneword{scada}{SCADA}{Supervisory Control and Data Acquisition}
\newduneword{ln}{LN}{liquid nitrogen}
\newduneabbrev{lapd}{LAPD}{Liquid Argon Purity Demonstrator}{Cryostat at Fermilab for long-term studies requiring a large volume of argon}

\newduneabbrev{pab}{PAB}{Proton Assembly Building}{Home of several \gls{lar} facilities at Fermilab}
\newduneword{tvs}{TVS}{transient voltage suppression}
\newduneword{hep}{HEP}{high energy physics}
\newduneword{sc}{SC}{scientific computing}  % check, this may be wrong (ask Heidi)
\newduneword{cms}{CMS}{Compact Muon Solenoid experiment at CERN}
\newduneword{alice}{ALICE}{A Large Ion Collider Experiment, at CERN}
\newduneword{gpib}{GPIB}{general purpose interface bus}


% Heidi S.  - this leads to duplicates so commented it out 
%\input{common/glossary-anne}

% Do this here to make use of the experiment name and title macro
\hypersetup{
    pdftitle={\expshort TDR \thedocsubtitle},
    pdfauthor={\expshort Collaboration},
    final=true,
    colorlinks=false,
    linktocpage=true,
    linkbordercolor=blue,
    citebordercolor=green,
    urlbordercolor=magenta,
    filecolor=black,
    pdfpagemode=UseOutlines,
    pdfborderstyle={/S/U},  
}





% define a local dword for later incorporation

\newcommand{\ldword}[1]{{\bf{[#1]}}\todo{define word#1}}
\newcommand{\ldshort}[1]{{\bf{[#1]}}\todo{define abbr #1}}
\newcommand{\ignore}[1]{}
\newcommand{\lcite}[1]{\cite{#1}}
\renewcommand\thedoctitle{\voltitleexec} % defined in common/defs.tex
%newcommand\thevolumenumber{1}
%\def\titleextra{\includegraphics[width=0.55\textwidth]{energy_nu_no}} -- change image file

%\begin{document}

%%%%%%%%%%%%%%%%%%%%  REMOVE ABOVE %%%%%%%%%%%%%%%%%
%\newduneword{rucio}{Rucio}{Data management system originally developed by ATLAS but now open-source and shared across HEP}
%\newduneabbrev{doma}{DOMA}{Data Organization, Management, and Access}{Data Organization, Management, and Access efforts through the HEP Software Foundation}
%\newduneabbrev{hsf}{HSC}{High Energy Physics Software Foundation}{High Energy Physics Software Foundation}
%\newduneabbrev{wlcg}{WLCG}{Worldwide LHC Computing Grid}{Worldwide LHC Computing Grid}
%\newduneabbrev{osg}{OSG}{Open Science Grid}{Open Science Grid}
%\newduneabbrev{sci}{SCI}{Scientific Computing Infrastructure}{Proposed extension of the infrastructure component of \dword{wlcg} to other experiments}
%\newduneabbrev{csc}{CSC}{Computing and Software Consortium}{DUNE Computing and Software Consortium}


\chapter{Computing in DUNE}
\label{ch:exec-comp}
%
%\fixme{Heidi, this outline may be overkill for the exec summary; it may be a good structure for the computing CDR volume, then pared down for inclusion here. My 2 cents! -Anne}
%%%%%%%%%%%%%%%%%%%%%%%%%%%%%%%%%%%%%%%%%%%%%%%%%%%%%%%%%%%
%\section{To remove: just examples}
%\label{sec:exec-comp-1}
%
%Sample figure to copy and edit, Figure~\ref{fig:map}:
%
%\begin{dunefigure}[DUNE collaboration global map]{fig:mhexec}{The international DUNE
%collaboration. Countries with DUNE membership are shown in orange.}
%\includegraphics[width=0.9\textwidth]{global-retouched.jpg}  
%\label{fig:map}
%\end{dunefigure}
%
%Sample table to copy and edit, Table~\ref{tab:execosctable}:
%
%\begin{dunetable}[Required exposures to reach oscillation physics
%  milestones]{lcc}{tab:execosctable}{The exposure in mass (kt) $\times$ proton beam power
%    (MW) $\times$ time (years) and calendar years assuming the staging plan described in this chapter needed to reach certain oscillation physics
%    milestones. The numbers are for normal hierarchy using the NuFit 2016 best fit values of the known oscillation parameters.  }
%Physics milestone & Exposure  & Exposure \\ \rowtitlestyle
%  & (\ktMWyr{}) & (years)  \\ \toprowrule 
%  $1^\circ$ $\theta_{23}$ resolution ($\theta_{23} = 42^\circ$) & 29  &  1\\ \colhline
%  CPV at $3\sigma$ ($\delta_{\rm CP} = -\pi/2$)  & 77 &  3\\ \colhline
%  \dword{mh} at  $5\sigma$ (worst point) & 209 & 6 \\ \colhline
%  $10^\circ$ $\delta_{\rm CP}$ resolution ($\delta_{\rm CP} = 0$) & 252 & %5 
%  6.5 \\ \colhline
%  ($\sin^2 2 \theta_{13} = 0.084 \pm 0.003$) &  &  \\  
%\end{dunetable}
%
%%%%%%%%%%%%%%%%%%%%%%%%%%%%%%%%%%%%%%%%
\section{Executive Summary}
\label{ch:exec-comp-es}

The DUNE  collaboration consists of 178 institutions from 32 countries, including 15 European nations and CERN. The experiment is in preparation now with commissioning of the first 17kT Liquid Argon TPC  module expected over the period 2024-2026 and a long data taking run growing to 4 modules from 2026-2036 and beyond.  An active prototyping program is already in place with a short test beam run in 2018 at CERN.  These tests used  a 700T, 15,360 channel prototype TPC with single-phase readout.  Tests of a similar sized dual-phase detector are scheduled for mid-2019.   The DUNE experiment has already  benefited greatly from these initial tests.  The collaboration has recently formed a formal Computing Consortium, with significant participation by European Institutions and interest from groups in Asia to work on common software and computing development and to formalize resource contributions.

The consortium resource model benefits from existing Open Science Grid\dshort(OSG)  and \dword{wlcg} infrastructure developed for the LHC and broader HEP community.  DUNE, through  the ProtoDUNE-SP effort, is already using global resources for simulation and the analysis of ProtoDUNE-SP data.  Multiple European sites are part of this resource pool and are making significant contributions to the ProtoDUNE single and dual phase programs.  We expect this global computing consortium to grow and evolve as we move towards data from the full DUNE detectors in the middle of the next decade.

The DUNE science program is expected to produce raw data volumes similar in scale to the data volumes that current LHC Run-2 experiments have already recorded.  Baseline predictions for the DUNE data, dependent on actual detector performance and noise levels, are $\ge 30$ PB of raw data per year.  These data, with simulations and derived analysis samples, need to be available to all collaborating institutions.  We anticipate that institutions worldwide will play an important role both as contributors and end-users of storage and CPU resources for DUNE.

To enable these resource contributions in cooperation with the LHC and other communities, we plan to utilize common computing layers for infrastructure access and use common tools to ease integration of facilities with both the DUNE and LHC computing ecosystems.  We will use common data storage methodologies to establish large high-availability data lakes worldwide  and to collaborate with the broader HEP community in developing other common tools.


HEP has considerable infrastructure in place for international computing collaboration thanks to the LHC program.  Additional large non-LHC experiments including LSST, SKA, DUNE and HyperK will be entering operation over the next decade and will need to utilize and expand this model for international cooperation.  This work on organizing the broader HEP community, is being formalized through the HEP Software Foundation(\dshort{hsf})\lcite{Alves:2017she}.  The HSF is an organization of interested parties working to use the extensive knowledge we have gained over the past two decades and the needs that experiments will have over the next two decades, to develop a sustainable computing landscape for the HEP community.  The HSF white papers and roadmaps place emphasis on common tools and infrastructure as the underpinnings of this landscape.

DUNE's computing strategy leverages heavily this model of common tools and infrastructure and features efforts in data movement and storage, job control and monitoring, accounting and authentication which both utilized and contribute to this global community.   DUNE recognizes that other large-scale experiments have similar needs and will encounter similar issues which drive worldwide cooperation on common tools as the most cost-effective path fulfilling the scientific missions of the experiments.  Already in the R\&D phases of DUNE computing there pilot programs that use this model.  Most recently in the realm of data management and storage, a collaboration between Fermilab, CERN, Rutherford Appleton Laboratory and other academic institutions in the United Kingdom's, is investigating the adaptation and use of the {\it \dword{rucio}}\cite{Barisits:2019fyl} data management systems to serve as the core data management system for DUNE.

Examples of this proto-culture of international collaboration within DUNE were demonstrated during the 2018 test beam run of the ProtoDUNE Single-Phase (SP) detector.  The ProtoDUNE run was a valuable live test of this model.  During this run the Single Phase detector situated at CERN produced raw data at rates exceeding 2GB/s.  These data were transferred and stored to the archive facilities at CERN and Fermilab and replicated at sites in the UK and Czech Republic.

In total, 1.8 PB of raw data were produced during the 10 week test beam run, mimicking within a factor of two, actual expected data rates and volumes from the initial running of the far detector complex .  The prototype run was used to examine and test the scalability of existing and proposed computing infrastructure and to establish operational experience within the institutions which have expressed interest in the development and construction of the DUNE computing environment.  The technical design presented here builds heavily on the measurements and information gained from the ProtoDUNE experience.   These measurements are treated as proofs of concept for many of the systems and their behavior is extrapolated to the projected levels needed for the full DUNE experiment. 

In addition to traditional HEP computational strategies which have been prototyped with the ProtoDUNE experience, the format and organization of DUNE's data as large multi-dimensional arrays, open the possibility of treating the data with computing paradigms similar to those developed for astrophysical image data.  These image processing techniques heavily leverage advances in machine learning and pattern recognition.  More over these techniques benefit from the computing resources that are available at High Performance Computing facilities (Supercomputers) where massively parallel and computation heavy algorithms can be mapped onto and scale better with the hardware toplogies than they do with traditional batch farms.  This field machine learning and HPC oriented analysis is currently very active with active advances coming from the current generation of liquid argon experiments such as ArgoNeut, MicroBooNE, SBND and ICARUS.  DUNE is poised to benefited greatly from this work and to expand on it as part of its computing and analysis model.

In summary, DUNE's computing strategy is to be {\bf global}, working with partners worldwide, and {\bf collaborative}, as almost all of the computational challenges we face are faced by similar experiments. 
 
%%%%%%%%%%%%%%%%%%%%%%%%%%%%%%%%%%%%%%%%
\section{Overview}
\label{ch:exec-comp-ovr}
The mission of the computing consortium is to facilitate the acquisition, processing and analysis of both detector data and supporting simulations for the DUNE experiment.  This mission must extend over all of primary physics drivers for the experiment, and must do so in a cost effective and secure manner. The computing and software consortium \dshort{csc} provides the bridge between the multiple online \dword{daq} and monitoring systems and the different physics groups who develop high-level algorithms and analysis techniques to perform measurements with the DUNE data and simulation. The S\&C works with collaborating institutions to identify and provide computational and storage resources.  They provide the software and computing infrastructure, in the form of analysis frameworks, data catalogs, data transport systems, databases infrastructure, code distributions mechanisms and other supporting services that are essential for recording and analyzing the data and simulation. 

The computing consortium works with national agencies and major laboratories to negotiate use and allocation of computing resources.  This work includes support for near term and R\&D efforts such as the ProtoDUNE runs and extends to the design, development and deployment of the DUNE computing model and its requisite systems.
These designs include the evaluation of major software infrastructure systems, including workload management and data management system, to determine their suitability for satisfying the DUNE physics requirements.   These evaluations are aimed at identifying opportunities for adopting or adapting existing technologies and engaging in collaborative ventures with HEP experiments outside of DUNE. 

In this context, the DUNE computing needs are modest in the context of the projected rates and needs for the high luminosity LHC experiments.  Initial calculation and extrapolations place the data volumes for the DUNE far detector program on par with the volumes produced during LHC Run II.  However the  beam structure, event sizes and analysis methodologies make DUNE very unlike the collider experiments in terms of event processing needs and projected computational budgets.  In addition, the large DUNE event sizes present a novel technical challenge when data processing and analysis are mapped onto  current and planned computing facilities.  Neutrino oscillation analysis and parameter extraction also present novel computational challenges.   Some of these challenges may require significant effort to adapt to the global computing resources that will be available to the experiment.  These global resources are projected to be both heterogenenous with regards to computational capabilities (featuring CPU, GPU and other advanced technologies) and diverse in terms of topological architectures and provisioning models.  The DUNE computing consortium will need to address these issues of diversity and architectural inclusion to fully exploit the global resources that will be available in the 2026+ era, and to enable all collaborators to access the data and perform the scientific mission of the experiment.  

\section{Data types and volumes}

The maximum data rate that the DUNE far detector modules can produce for a single readout is computed from the detector channel counts, base sampling rates, and digitization resolution.  This is then scaled by the expected triggering rates for beam induced activity along with other sources of trigger initiation (i.e. cosmic ray activity, radiological signals, calibration information, etc...) to compute the expected upper limit on the data rates and volumes that can be expected from the detector.  Separately estimates of the effects of lossless data compression and zero suppression schemes have been estimated and in some cases demonstrated with the ProtoDUNE single phase data.  These techniques are highly dependent  on the structure of the data and in particular the structure of the noise or other sources of background that may be observed in the final DUNE detector modules.  As a result the compression/suppression estimates are given as ranges based on detector performance assumptions.

The parameters described in Table~\ref{tab:exec-comp-bigpicture} display the key characteristics of the DUNE far-detector data stream.  A bandwidth specification has been established which provides an operational envelope within which the data acquisition, networking, and software and computing tasks should  operate. This bandwidth envelope has been set to  30~PB of data/yr.  This corresponds to a time-averaged egress bandwidth  of 950 MB/s (7.6 Gbit/s).  As an uncompressed event stream this would equate to 0.15 evt/s, rising to 0.6 evt/s under a target compression of 4:1 of single phase module readout.    The reduction from full streaming data rates to these rates will be implemented in the DAQ/trigger level of the experiment. 
Those decisions are the purview of the collaboration scientists and the data acquisition design with feedback from computing on what is possible.  The ProtoDUNE experience has provided invaluable information to feed back to the experiment design. 

\subsection{Far detector}

The computing model needs to be able to handle a wide range of data inputs from the far detectors, as documented in more detail in docdb-9240\lcite{bib:docdb9240}.

\begin{itemize}
\item Supernova triggers which would have an uncompressed size of 138 TB for a 30 second readout of all channels in a 4-module single-phase detector at a likely rate of 1/month.  
\item Beam neutrino interactions within a single detector module with an uncompressed size of $\approx$ 6.2 GB.  Beam neutrinos arrive at a rate of up to 1 Hz but most do not result in measurable interactions.
\item Atmospheric neutrino interactions, nucleon decay and other lower energy processes confined to a subset of a detector module with a low threshold largely driven by radiological backgrounds.
\item Cosmic ray and rock muons at a rate of around 4,500/day/module.
\item Other calibration systems 
\end{itemize}

The estimates in docdb-9240, with conservative estimates for increased needs for low level data taking during commissioning, have led to a negotiated upper limit of 30 PB/year data volume as a standard for both the trigger and data acquisition and computing groups to work towards. 


\begin{dunetable}[Useful quantities for computing estimates]{lrr}{tab:exec-comp-bigpicture}{Useful quantities for computing estimates}%\rowtitlestyle
Quantity&Value&Explanation\\ 
\hline
{\bf Far Detector Beam:}\\
Single APA readout &41.5 MB& Uncompressed 5.4 ms\\
APAs per module& 150&\\
Full module readout &6.22  GB& Uncompressed 5.4 ms\\
Beam rep. rate&\beamreprate&Untriggered\\
CPU time/event&600-1,200 sec&from MC/ProtoDUNE\\
Memory footprint&2-4 GB&ProtoDUNE experience\\
\hline
{\bf Supernova:}\\
Single channel readout &90.0 MB& Uncompressed 30 s\\
Four module readout&138.2 TB& Uncompressed 30 s\\
Trigger rate&1  per month&(assumption)\\
%Yearly rates nd
%Reduction with roi.  
%CPU time/ event for reconstruction
%Reduction for analysis
%Users 
\end{dunetable}

\subsection{Near Detector}
In addition, a near detector of reasonable size will have multiple neutrino interactions/beam spill leading to a need to read out at the full beam rate of 0.8-1.2 Hz.
The near detector will have fewer channels and better signal/noise discrimination but much higher readout rates.  While the details of the detector design are still unknown, we assume data volumes of similar size to the far detector (30PB/year) in our planning.

\subsection{Simulation}
The bulk of data collected is likely to be backgrounds, with real beam interaction events in the far detector numbering in the thousands/year, not millions. Thus the size of simulation samples is likely to be less than that of the unprocessed raw data considered above.  Lower energy events are either very rare or can be simulated in sub-volumes of the whole detector.  As a result, while simulation will be an important part of the experiment, it is not expected to dominate data volumes as it does in many experiments.  

However, simulation inputs such as flux files, overlay samples and shower libraries pose a special problem as they must be distributed to simulation jobs carefully.  Proper simulation requires that these inputs be delivered in an unbiased fashion. This can be technically difficult in a widely distributed environment and will require thoughtful design. 

\subsection{Analysis}

Analysis formats have not yet been fully defined.  We anticipate that most analysis samples will be orders of magnitude smaller than the raw data.  However, as they are idiosyncratic to particular analyses and in fact particular users,  producing and cataloging them will be sociologically difficult. 
Likely there will be a mix of official samples -  produced by physics groups and distributed through a common catalog and file transfer mechanisms - and small user samples on local disk. 


\section{ProtoDUNE-SP as an example}
\label{ch:exec-comp-proto-SP}
The first ProtoDUNE single phase run at CERN in late 2018 has already led to a small-scale test of a global computing model.  In the following we will describe the ProtoDUNE data design and the lessons learned from our experience. Much of this carries over into planning for full far-detector operations. 

\subsection{Introduction}

The ProtoDUNE Single Phase detector ran at CERN in the np04 beamline from September to November of 2018. Since then, studies with cosmic rays have continued. Prior to that run there were several data challenges at high rate to validate the data transfer mechanisms. 

\subsection{Data Challenges}

ProtoDUNE performed a series of data challenges, starting in late 2017.  Simulated data were passed through the full chain from the event builder machines to tape store at CERN and Fermilab at rates of up to 2 GB/s.  These studies allowed optimization of the network and storage elements well before the start of data taking.
We note that the full DUNE far detector writing 30 PB/year would produce data at rates similar to, or less than, those demonstrated in the 2018 data challenges. 

\subsection{Commissioning and Physics Operations}

The first phase of operations was commissioning of the detector readout systems while the argon reached full purity.  Data were taken with cosmic rays and beam during the commissioning period. Once high Argon purity had been achieved, physics data were  taken with beam through October and half of November. Normal trigger rates were approximately 25 Hz but tests were done at rates of up to 100 Hz. Since the beam run ended, cosmic-ray data continues to be taken with varying detector conditions, such as modified high voltage and purity, and new readout schemes. 

%\subsection{Data Quality Monitoring}

%\todo{DQM}

\subsection{Data volumes}
The single-phase ProtoDUNE detector consists of a \dshort{tpc} with  6 Anode Plane Assemblies (\dshort{apa}), photon detectors (\dshort{pd}s) and a Cosmic Ray Tagger (\dshort{crt}). In addition, the np04 beamline is instrumented with hodoscopes and Cerenkov counters to generate beam triggers. Random triggers  were generated at lower rates to collect unbiased cosmic ray information. The data volume from the test beam run was dominated by readout of the \dshort{TPC}.  Each \dshort{apa} has 2,560 channels and reads out 12 bit ADC values at 2 MHz.   The nominal readout window during beam running was  3 ms to match the drift time at the full voltage of 180 kV that was maintained for most of the run.  The size of the \dshort{tpc} data without compression was thus 138 MB/event, not including headres.  The uncompressed event size including all TPC information and \dword{crt} and \dword{pd} data was 170-180 MB. Compression was implemented before the October beam physics run, lowering the total size per event from around 180 MB to 75 MB.  

\begin{dunetable}[Data volumes]{lrr}{tab:exec-comp-pd-volumes}{Data volumes  recorded by ProtoDUNE-SP as of December 2018.}
Type  & Events & Size\\ %\rowtitlestyle
Raw Beam&8.08 M& 520 TB \\
Raw Cosmics&3.46 M& 271 TB\\
Commissioning&3.86 M& 388 TB\\
Pre-commissioning&13.89 M&641 TB\\
\end{dunetable}

Events were written out in raw files of size 8 GB with each containing of order 100 events. The beam was live for two 4.5 s spills every 32 s beam cycle and data were taken at  rates of up to 50 Hz (typically 25 Hz) leading to compressed DC rates out of the detector of 400-800MB/sec.  Each beam cycle could therefore produce 1-4  8 GB output files.  In earlier running with uncompressed data, and during an April data challenge, transfer rates of up to 2GB/s were demonstrated over substantial periods. 

Beam stopped on November 12 but cosmic ray studies of the detector continue, some with an increased time window of 7.5 ms to collect more complete tracks each readout.  This raises the compressed event size to around 170 MB.


\subsection{ProtoDUNE-SP data streams}
The ProtoDUNE-SP data consist of multiple sources in addition to the TPC data. One of the major challenges for the offline computing systems is merging of these multiple streams into a coherent whole for analysis.  Table \ref{tab:exec-comp-pd-sources} lists the data sources used and their granularity. 

\begin{dunetable}[Data sources]{lrr}{tab:exec-comp-pd-sources}{Data sources  }
Type & indexed by & destination\\
TPC  & run/event & event data\\
Photon Detector data & run/event & event data\\
Cosmic Ray Tagger & run/event & event data\\
Beamline devices & timestamp & beam database\\
Detector conditions & timestamp & slow controls database\\
DAQ configuration & run & files/elisa logbook\\
Run quality & run & human generated spreadsheets\\
Data quality & run/event/time & Data Quality web application\\
File metadata & file & \dword{sam} file database\\
\end{dunetable}

Information about the detector conditions, \dword{daq} configuration and run quality is spread across a number of sources and must be collected and then boiled down into the quantities relevant for offline data analysis.  For example, the Slow Controls system logs detector conditions continually.  Offline analysis needs to know about these data with coarser granularity and then have algorithms capable of using that information. A full conditions database transfer mechanism is being developed but was not available during the run.  As a result, with the exception of beamline information, coarse information is currently added to the \dword{sam} file catalog run by run to allow files with given operating conditions to be easily identified and retrieved. Beam data is stored in the \dword{ifbeam}
database and connected to event data via time stamps.

\subsection{Reconstruction of ProtoDUNE-SP data}
Thanks to substantial previous effort by the 35T prototype, \dword{microboone} and the Liquid Argon TPC community, high quality algorithms were already in place to reconstruct the TPC  data.  As a result, a first pass reconstruction of the ProtoDUNE-SP data with beam triggers was completed by early December, less than a month after the end of data taking.



\subsection{Data preparation}

Before pattern recognition, data from the ProtoDUNE detector is
unpacked and copied to a standard format within the art framework based on ROOT derived objects. 
The format is also used in detector simulation events.
This reformatted raw data includes the waveform for each channel, consisting of 6,000-15,000,  12-bit
, 0.5 $\mu$sec samples. 

The first step in reconstruction is data preparation with the goal of
converting each ADC waveform into a calibrated charge waveform with
signals proportional to charge. At the end of data preparation, regions of interest (ROIs), i.e. blocks of contiguous samples where
useful signals appear, are identified and the data outside these regions are discarded.

%Perhaps most important, the bipolar signals in induction wires are made unipolar.
%Also the electronics shaping is replaced with the Gaussian shaping expected in the
%next stage of processing.
%Relative channel-to-channel and absolute calibration is applied to account for the
%responses of the amplifiers and ADCs.
%And attempts are made to remove noise and mitigate ADC distortions.

The sequence is described more fully in docdb-12349\lcite{bib:docdb12349} and in the Methods section of the Physics Volume but is summarized here:

\begin{enumerate}
\item Each waveform is unpacked into integers.
\item Pedestals are determined per event/per channel from the most common ADC value. 
\item Pedestals and calibrations are applied. %\label{local:ped}
\item Bad channels, sticky bits and other know hardware problems are corrected or removed.
\item Signal undershoot that creates a long negative tail is removed. 
\item The waveforms  are deconvoluted.  In the first processing this was done with simple 1-D  convolution for a single wire.  A 2-D method of deconvolving a detailed detector electrostatic field map, originally developed for MicroBooNE\lcite{Adams:2018dra}, is in now available for ProtoDUNE use and will be used in the future reconstruction passes.  It properly undoes the long-range induction effects while keeping efficiency high and bias low.  The deconvolution Fourier transforms the waveform, replaces the  bipolar field response function with a unipolar function, applies a low pass filter to remove high frequency noise and then transforms back.




\item Finally regions of interest are defined where the signal exceeds a given threshold and time slices well outside the \dshort{roi} are dropped, leading to significant reduction in the size of the remaining data. These data feed into the reconstruction algorithms for further pattern recognition. %\label{local:roi}
\end{enumerate}




%\subsection{Signal processing}
Figures~\ref{fig:ch-exec-comp-chtraw}-\ref{fig:ch-exec-comp-chtroi} illustrate the transformation of TPC data  during data
preparation.

\begin{figure}[t]
\includegraphics[width=\textwidth,angle=0]{comp-evd_twq-proj_5449_20926_raw.png}
\caption{
Example of pedestal-subtracted data for one of the ProtoDUNE  wire planes.  The top pane shows the ADC values in a V (induction) plane with the x-axis being channel number and the y-axis, time slice. The bottom pane shows the bipolar pulses induced on one channel. 
}
\label{fig:ch-exec-comp-chtraw}
\end{figure}




%\begin{figure}[t]
%  \includegraphics[width=\textwidth]{ccomp-evd.twq-proj.5449.20926.recon..png}
%\caption{
%Same as Fig.~\ref{fig:ch-exec-comp-chtraw} except after hit correction, tail removal and deconvolution.
%}
%\label{fig:ch-exec-comp-chtdco}
%\end{figure}

\begin{figure}[t]
  \includegraphics[width=\textwidth]{comp-evd_twq-proj_5449_20926_decon.png}
\caption{
Same as Fig.~\ref{fig:ch-exec-comp-chtraw} except after calibration, cleanup, deconvolution and ROI finding. 
}
\label{fig:ch-exec-comp-chtroi}
\end{figure}

%%%%%%%%%%%%%%%%%%%%%%%%%%%%
%\todo{Statement about timing and memory for this phase}

\subsection{Computational characteristics of data preparation and deconvolution }
Decoding for ProtoDUNE-SP is currently done with all 6 APAs in memory. As each 3 ms APA readout consists of over 15M 16-bit values, decompression and conversion to floating point results in substantial memory expansion.  Decoding and deconvolution of 6 APAs with 3 ms readout fits within a normal 2 GB memory/core footprint but the 7.5 ms readout window used in some cosmic ray studies requires a correspondingly larger memory footprint. As electrical signals are correlated between channels within an APA wire plane, but not between planes, processing each wire plane (3/APA) independently reduces the memory footprint.  These changes are being implemented.


However,  while subdividing the detector into wire planes solves the memory problems for short readouts it is  not a viable solution for the long readouts expected for supernova events. We are still exploring the best strategy for dealing with these much larger ($\times 10,000$)time windows. The DAQ group is already testing 1 second ($300 \times$ longer time window) readouts of small numbers of channels.  These are being used as tests of optimal models for data segmentation.  Section \ref{ch:exec-comp-mod} describes the start of a bottoms-up collaboration with the \dshort{daq} consortium on an optimal data model for the full DUNE detectors. 

\subsection{Further reconstruction}
The downstream pattern recognition steps starting with \dword{roi} are described further in the Tools and Methods chapter of the Physics Volume.  
Full reconstruction of ProtoDUNE-SP interactions, with beam particles and of order 20-40 cosmic rays per readout window took 600-1200 sec/event.
Table  \ref{tab:comp-raw-data-size} shows the input datasize for a typical beam event, dominated by around 71 MB of TPC waveform information. Table  \ref{tab:comp-reco-data-size} shows the size of different reconstructed objects, still dominated by around 10 MB of reduced TPC hit information,  while \ref{tab:comp-reco-data-time} shows the reconstruction time breakdown.  This event had a 3 ms readout window.  The input size and reconstruction time scale reasonably linearly with the readout window.  

\subsection{Reconstruction characteristics}

The data preparation phase can be segmented by detector component, for example into wire planes within a APA.  The operations performed in signal processing require few decisions to be made but do include operations such as fast-Fourier transforms and deconvolution.  These operations are well suited for GPU and parallel processing. There is active exploration into multi-threading processing for all data preparation algorithms.


Once ROI's have been identified, several 3-D reconstruction packages are used. For the first reconstruction pass in November, the  \dword{pandora}\cite{Acciarri:2017hat}, \dword{wirecell}\cite{wirecell} and \dword{pma}\cite{ref:PMA}  frameworks were used with results described in the Physics volume.   Table \ref{tab:comp-reco-data-time} indicates that they are comparable in terms of CPU time used.   Deep Learning techniques based on image pattern recognition algorithms are also being developed. Many of these algorithms can be adapted to run on HPC's, but probably different architectures that would be optimal for the data preparation phase. 

All of these algorithms are currently being run on conventional unix CPU's using \dword{osg}/\dword{wlcg} grid computing  infrastructure. 



\begin{dunetable}[Compressed size/event for Raw data - 7 GeV beam data with a 3 ms time window]{rrl}{tab:comp-raw-data-size}{Compressed size/event for Raw data - 7 GeV beam data with a 3 ms time window}
  Size in Bytes&Fraction&Data Product Name\\
\hline
44,155.47&0.576&RCE DAQ Fragments\\
27,952.64&0.364&FELIX DAQ Fragments\\
4,586.82&0.06&Photon Detector DAQ Fragments\\
5.72&0&CTB DAQ Fragments\\
0.17&0&DAQ Timing Fragments\\
0.09&0&Trigger Results\\
\hline
76,703.25 & 1.0 & Total\\
\end{dunetable}

\begin{dunetable}
[Compressed size/event for Reconstructed data - 7 GeV beam data]
{rrl}
{tab:comp-reco-data-size}
{Compressed size for Reconstructed data - 7 GeV beam events}
Size in kBytes&Fraction&Data Product Name\\
4,218,536&0.185&recob::Wires\\
2,236,432&0.098&recob::Hits(hitpdune) \\
2,102,520&0.092&recob::Hits(gaushit)\\
2,052,796&0.090&recob::Hits(linecluster)\\
2,020,575&0.089&artdaq::Photon Detector DAQ Fragments\\
1,532,502&0.067&raw::OpDetWaveforms\\
1,018,088&0.045&anab::Calorimetrys(pandoracalo)\\
873,797&0.038&recob::Tracks(pandoraTrack)\\
806,513&0.035&anab::Calorimetrys(pmtrackcalo)\\
555,775&0.024&recob::SpacePoints(pandora)\\
479,599&0.021&recob::Tracks(pmtrack)\\
414,824&0.018&raw::OpDetWaveforms\\
391,791&0.017&recob::Hitrecob::Trackrecob::TrackHitMetaart::Assns\_pmtrack.\\
379,553&0.017&recob::SpacePoints\_pmtrack\_\_DecoderandReco.\\
310,021&0.014&recob::SpacePoints\_reco3d\_pre\_DecoderandReco.\\
260,143&0.011&recob::Hitrecob::SpacePointvoidart::Assns\_hitpdune\\
250,175&0.011&recob::Hitrecob::SpacePointvoidart::Assns\_reco3d\\
229,711&0.01&recob::SpacePoints\_reco3d\_noreg\\
218,874&0.01&recob::SpacePoints\_reco3d\\
200,618&0.009&recob::Hitrecob::SpacePointvoidart::Assns\_pandora\\
2,407,376&0.106&Smaller Objects\\
\hline
22,759,597&1.000&Total\\
\end{dunetable}

\begin{dunetable}[Algorithm timing for 7 GeV beam event]{lrr}{tab:comp-reco-data-time}{Algorithm timing for 7 GeV beam events.  Smaller processes not shown for clarity. A 10 event job used 2.7 GB of memory to do this reconstruction.}
  Processing step&Average CPU time, sec\\
  RootInput(read)&0.2\\
  %decode:timingrawdecoder:TimingRawDecoder&0.0\\
  %decode:ssprawdecoder:SSPRawDecoder&0.3\\
  PDSPTPCRawDecoder&15.9\\
  %decode:crtrawdecoder:CRTRawDecoder&0.0\\
  %decode:ctbrawdecoder:PDSPCTBRawDecoder&0.0\\
  BeamEvent&0.7\\
  DataPrepModule&89.1\\
  Deconvolution&113.3\\
  %decode:digitwire:EventButcher&0.8\\
  GausHitFinder&20.7\\
  SpacePointSolver&18.0\\
  DisambigFromSpacePoints&23.0\\
  LineCluster&3.9\\
  StandardPandora&93.2\\
  LArPandoraTrackCreation&12.3\\
  LArPandoraShowerCreation&5.2\\
  Calorimetry&5.8\\
  %decode:pandorapid:Chi2ParticleID&0.0\\
  PMAlgTrackMaker&142.0\\
  PMCalorimetry&6.1\\
  %decode:pmtrackpid:Chi2ParticleID&0.0\\
  %decode:ophitInternal:OpHitFinder&0.0\\
  %decode:ophitExternal:OpHitFinder&0.0\\
  %decode:opflashInternal:OpFlashFinder&0.0\\
  %decode:opflashExternal:OpFlashFinder&0.0\\
  %[art]:TriggerResults:TriggerResultInserter&0.0\\
  %end_path:out1:RootOutput&0.0\\
  RootOutput(write)&3.3\\
  Total&503.7\\
\end{dunetable}





 




\subsection{Processing Infrastructure for Reconstruction and Simulation}
\label{ch-comp-processing}
DUNE makes use of computing resources internationally through the Open Science Grid and the parallel infrastructure set up for WLCG in Europe.  In 2018, significant effort was put into integrating European sites into the DUNE reconstruction and simulation processing with very positive results.  
Figure \ref{fig:ch-exec-comp-cpupie} shows the distribution of production jobs worldwide in November and December 2018 during the main reconstruction pass.  FNAL and CERN as the host laboratories made the largest contributions but significant resources were also made available from the UK through integration with GridPP and in the Czech republic through FZU and at IN2P3 in France. 

\begin{figure}[htp]
\centering
%\subfloat[]{
%\includegraphics[height=2.5in]{comp-dunepro_pdsp_keepup.png}%
\includegraphics[height=4in]{graphics/comp-vo-summary.png}
%}
%\vspace{1cm}
%\subfloat[]{
%\includegraphics[height=2.3in]{comp-dunepro_mcc11.png}%
%\includegraphics[height=2.3in]{comp-dunepro_mcc11_legend.png}
%}
\caption{CPU wall-time for November 2018 during ProtoDUNE-SP reconstruction showing multiple site contributions.  The major contributions were from FNAL, CERN, many UK institutions and FZU.}
\label{fig:ch-exec-comp-cpupie}
\end{figure}

\begin{dunetable}
[Data storage  and CPU needs for reconstruction of ProtoDUNE test beam data]
{llrrrr}{tab:exec-comp-needs}{Data storage and CPU needs for reconstruction of ProtoDUNE-SP test beam data taken in 2018 and projections for 2019-2021.  We assume two copies of raw data are stored and that each event is reconstructed twice.  Analysis and simulation are estimated to be of order the same CPU use as reconstruction based on the 2018 experience.}%\rowtitlestyle
Detector& value &
2018&
2019&
2020&
2021\\
&&As built\\
\hline
SP&
Events, M&
15.1&
13.0&
6.5&
40.5\\
&
Raw data, TB&
1047&
2239&
1120&
2799\\
&
Reco data, TB&
2094&
4479&
2239&
5599\\
&
CPU, MH&
5.0&
4.3&
2.2&
13.5\\
\hline
DP&
Events, M&
0.0&
101.1&
56.2&
119.9\\
&
Raw data, TB&
0&
809&
449&
1799\\
&
Reco data, TB&
0&
1617&
899&
3598\\
&
CPU, MH&
0.0&
33.7&
18.7&
40.0\\
\hline
total&
Events, M&
15.1&
114.0&
62.6&
160.4\\
&2x
Raw data, TB&
2094&
6096&
3138&
9197\\
&
Reco data, TB&
2094&
6096&
3138&
9197\\
total&Storage, TB
4188&
12193&
6276&
18394\\
&
Reco CPU, MH&
5.0&
38.0&
20.9&
53.5\\
&
Analysis CPU, MH&
5.0&
40.0&
40.0&
40.0\\
Total&CPU, MH&
10.0&
78.0&
60.9&
93.5\\
\end{dunetable}

\subsection{Lessons learned}

\begin{itemize}
    \item Data and simulation challenges led to a reasonably mature and robust model for acquiring, storing and cataloging the main data stream. 
    \item The experiment was able to integrate multiple existing grid sites and make use of substantial opportunistic resources.  This allowed initial processing of data within one month of the end of the run.
    \item Substantial but successful effort went into signal processing. 
    \item Prototype infrastructure was in place for provisioning, authentication and authorization, Data Management, networking, file catalog, and workflow management. 
    \item Reconstruction algorithms are not perfect but sufficient for studies of detector performance and calibration. 
    \item Beam information was successfully integrated into the processing through the \dword{ifbeam} database.
    \item Auxiliary information from, for example, slow controls was not integrated into processing due to lack of personpower.  This led to dependence on hand input of running conditions by shift personnel and offline incorporation of that information into the data catalog. 
\end{itemize}

Overall, the ProtoDUNE-SP data taking and processing was a success but overly dependent on doing things ``by hand'' as automated processes were not always in place. Considerable effort will be needed needed for integration of detector conditions, data migration, workflow systems and integration of HPCs with multi-threaded and vectorized software.

\subsection{Near future}

Table \ref{tab:exec-comp-needs} summarizes the known resource usage for 2018 and projections for 2019-2021.  The collaboration has requested a substantial test beam run for both single and dual phase detectors in 2021.  The \dshort{csc} views this run as a first production test for the full DUNE computing infrastructure. 





%%%%%%%%%%%%%%%%%%%%%%%%%%%%%%%%%%%%%%%%
\section{Data Model for the Far and Near Detectors}
\label{ch:exec-comp-mod}

%%%%%%%%%%%%%%%%%%%%
\subsection{Introduction}
\label{ch:exec-comp-mod-int}
In parallel with the ProtoDUNE-SP data, a joint Data Model task force was formed by the DAQ and Computing Consortia to lay a framework for the full near and far DUNE detectors. 
The Data Model task force grappled with the problems of efficiently triggering, reading out and storing data from an enormous detector on multiple time scales.

They defined major concepts.

\begin{description}

\item{Configuration:} set of parameters that define the persisted, expected detector state. Globally, this corresponds to a desirable state for the detector, capable of providing data of physics or calibration quality. Each component of the detector may have its own configuration.
 
\item{Run:} Period of time over which data has been collected across some set of desired components in a consistent configuration.
 
\item{Subrun:} Period of time within a run over which data has been collected across some set of desired components in a consistent configuration. The set of desired components in a subrun must be a subset of the desired components for a run, and is the set of components over which data is expected.
 
(Time-based rollovers of runs and subruns may be automatic. Differences of subrun and run due to configuration or changes in the desired components will be tracked by the DAQ, and may be either manual or automatic.)
 
\item{Trigger:} data from the desired components in that subrun over a readout window. This would typically be centered around a trigger time, and is what is recorded by the DAQ. The readout window may be subdivided into frames as determined by the DAQ.
 
\item{Event:} subset of a trigger isolated in time and space containing an independent interaction in the detector. Events may overlap in space or time, within the same trigger. This is generally determined by the offline, based on reconstruction of data in a frame.

\end{description}

These definitions are intended to allow triggering, recording and reconstruction of interactions in subsets of the detector. While the whole detector (or time window) can produce enormous amounts of data, any individual interaction is expected to span a reasonably short time and spatial volume. A data model that can isolate individual interactions  allows efficient storage and reconstruction of interactions. 


The main data stream will be augmented by beam, slow controls, \dword{daq} configuration and calibration information. 

This work continues and informs  the  joint  calibration, \dshort{csc} and \dshort{daq} designs.


\section{Consortium Organization}

%The DUNE computing and software consortium was formed in October 2018.  
%%%%%%%%%%%%%%%%%%%%
%\subsection{Requirements}
%\label{ch:exec-comp-mod-req}


%%%%%%%%%%%
%\subsubsection{How Physics Drives}
%\label{ch:exec-comp-mod-req-phys-drv}


%%%%%%%%%%%
%\subsubsection{Oscillation Analyses}
%\label{ch:exec-comp-mod-req-osc}


%%%%%%%%%%%
%\subsubsection{Cross Sections}
%\label{ch:exec-comp-mod-req-xsec}


%%%%%%%%%%%
%\subsubsection{Other Drivers}
%\label{ch:exec-comp-mod-req-oth-drv}


%%%%%%%%%%%%%%%%%%%%
%\subsection{Interfaces with Other Projects}
%\label{ch:exec-comp-mod-intfc}


%%%%%%%%%%%
%\subsubsection{DAQ}
%\label{ch:exec-comp-mod-intfc-daq}


%%%%%%%%%%%
%\subsubsection{Calibration}
%\label{ch:exec-comp-mod-intfc-calib}

%Why is this cutting off here on the page?

%%%%%%%%%%%
%\subsubsection{Physics}
%\label{ch:exec-comp-mod-intfc-phys}

%Why is this cutting off here on the page?

%%%%%%%%%%%%%%%%%%%%
%\subsection{Use cases}
%\label{ch:exec-comp-mod-use}


%%%%%%%%%%%
%\subsubsection{Discussion of detector options (SP,DP,ND? )}
%\label{ch:exec-comp-mod-use-opt}


%%%%%%%%%%%
%\subsubsection{Data acquisition}
%\label{ch:exec-comp-mod-use-daq}


%%%%%%%%%%%
%\subsubsection{Data Quality}
%\label{ch:exec-comp-mod-use-dq}


%%%%%%%%%%%
%\subsubsection{Reconstruction}
%\label{ch:exec-comp-mod-use-reco}


%%%%%%%%%%%
%\subsubsection{Calibration}
%\label{ch:exec-comp-mod-use-calib}


%%%%%%%%%%%
%\section{Simulation}
%\label{ch:exec-comp-mod-use-sim}


%%%%%%%%%%%
%\section{Analysis}
%\label{ch:exec-comp-mod-use-anls}


%%%%%%%%%%%%%%%%%%%%
%\subsection{Existing Infrastructure}
%\label{ch:exec-comp-mod-infr}


%%%%%%%%%%%
%\subsubsection{sam/enstore/eos/castor}
%\label{ch:exec-comp-mod-infr-stor}


%%%%%%%%%%%
%\subsubsection{Grid}
%\label{ch:exec-comp-mod-infr-gr}


%%%%%%%%%%%
%\subsubsection{Databases}
%\label{ch:exec-comp-mod-infr-db}


%%%%%%%%%%%%%%%%%%%%
%\subsection{ProtoDUNE Experience}
%\label{ch:exec-comp-mod-pdune}


%%%%%%%%%%%%%%%%%%%%
%\subsection{Evolving Infrastructure}
%%\label{ch:exec-comp-mod-evlv}
%Why is this cutting off here on the page?

%%%%%%%%%%%
%\subsubsection{rucio}
%\label{ch:exec-comp-mod-evlv-ruc}
%Why is this cutting off here on the page?

%%%%%%%%%%%
%\subsubsection{Load Management}
%\label{ch:exec-comp-mod-evlv-load}


%%%%%%%%%%%
%\subsubsection{?.}
%\label{ch:exec-comp-mod-evlv-}


%%%%%%%%%%%%%%%%%%%%
%\subsection{Novel Architectures}
%\label{ch:exec-comp-mod-nov}

%%%%%%%%%%%
%\subsubsection{HPC}
%\label{ch:exec-comp-mod-nov-hpc}

%%%%%%%%%%%%%%%%%%%%
%\subsection{Authentication}
%\label{ch:exec-comp-mod-auth}


%%%%%%%%%%%%%%%%%%%%%%%%%%%%%%%%%%%%%%%%
%\section{Software}
%\label{ch:exec-comp-sw}


%%%%%%%%%%%%%%%%%%%%
%\subsection{Introduction}
%\label{ch:exec-comp-sw-int}


%%%%%%%%%%%%%%%%%%%%
%\subsection{Existing Packages}
%\label{ch:exec-comp-sw-int-pkg}


%%%%%%%%%%%
%\subsubsection{GEANT4}
%\label{ch:exec-comp-sw-int-gnt}


%%%%%%%%%%%
%\subsubsection{ROOT}
%\label{ch:exec-comp-sw-int-root}
%hy is this cutting off here on the page?

%%%%%%%%%%%
%\subsubsection{art}
%\label{ch:exec-comp-sw-int-art}


%%%%%%%%%%%%%%%%%%%%
%\subsection{Evolving Packages}
%\label{ch:exec-comp-sw-evpkg}


%%%%%%%%%%%
%\subsubsection{LArSoft}
%\label{ch:exec-comp-sw-evpkg-larsoft}


%%%%%%%%%%%
%\subsubsection{Wirecell}
%\label{ch:exec-comp-sw-evpkg-wcell}


%%%%%%%%%%%
%\subsubsection{GENIE}
%\label{ch:exec-comp-sw-evpkg-genie}


%%%%%%%%%%%%%%%%%%%%
%\subsection{DUNE-specific Software}
%\label{ch:exec-comp-sw-evpkg-spec}


%%%%%%%%%%%%%%%%%%%%
%\subsection{Novel Architectures}
%\label{ch:exec-comp-sw-novarch}

%%%%%%%%%%%
%\subsubsection{Machine Learning}
%\label{ch:exec-comp-sw-novarch-mach}
%Why is this cutting off here on the page?
%%%%%%%%%%%
%\subsubsection{Vectorization}
%\label{ch:exec-comp-sw-novarch-vec}

%%%%%%%%%%%%%%%%%%%%
%\subsection{Development Environment}
%\label{ch:exec-comp-sw-devenv}

%%%%%%%%%%%
%\subsubsection{Environment Specifications}
%\label{ch:exec-comp-sw-devenv-spec}


%%%%%%%%%%%
%\subsubsection{Code and Configuration Management}
%\label{ch:exec-comp-sw-devenv-mgmt}


%%%%%%%%%%%
%\subsubsection{Validation}
%\label{ch:exec-comp-sw-devenv-val}


%%%%%%%%%%%%%%%%%%%%
%\subsection{Training and Communication}
%\label{ch:exec-comp-sw-train}
%Why is this cutting off here on the page?

%%%%%%%%%%%%%%%%%%%%
%\subsection{Lessons from ProtoDUNE}
%\label{ch:exec-comp-sw-lessons}


%%%%%%%%%%%%%%%%%%%%%%%%%%%%%%%%%%%%%%%%
\subsection{Resources and Governance}
\label{ch:exec-comp-gov}

The Computing and Software effort is now a DUNE Consortium.  Docdb 12751 \lcite{bib:docdb12751} describes the governance structure for the Consortium.  The Consortium coordinates effort across the collaboration but funding comes from collaborating institutions, laboratories and national funding agencies. 

The consortium has an overall Consortium Leader. The Consortium Leader is responsible for the sub-system deliverables and represents the consortium to the overall DUNE collaboration.

In addition there  are Technical Leads to act as the overall project managers for the consortium. The Technical Leads report to the overall Consortium Leader.
Computing has both a Host Laboratory Technical Project Lead, responsible for coordination with the DUNE Project and host lab and an External Technical Lead responsible for coordination with other entities.
It is anticipated that at least one of the three leadership roles will be held by a non-US scientist. 
As with other DUNE consortia, the consortium is responsible for assigning a provisional division of institutional
responsibilities for computing resources, deliverables and operations, amongst the participating institutions. This division of responsibilities must account for the resources that are likely to be available. The internally agreed division of responsibilities needs to be presented to the Technical Board, which will then make a recommendation to the collaboration management for approval.



\subsection{Scope of the Consortium}
The Computing and Software Consortium (\dword{csc}) is mainly concerned with the infrastructure and resources for offline computing.  Algorithm development resides within the Physics groups while online systems at experimental sites are governed by the Data Acquisition and Cryogenics Instrumentation and Slow Controls Consortia. These groups coordinate closely to assure that the full chain of data acquisition, processing and analysis works. Formal interfaces with these groups are described in Docdb 7123 (DAQ)\lcite{bib:docdb7123} and Docdb 7126 (CISC)\lcite{bib:docdb7126}.

The consortium operates at two levels: at the hardware level, where generic resources can be provided as in kind contributions to the collaboration, and at the human level, where individuals and groups contribute to the development of common software infrastructure. 

\subsection{Hardware}
As noted above, the collaboration has already made use of substantial global resources through the \dword{wlcg} and \dword{osg} grid mechanisms. As the Consortium evolves, institutions and collaborating nations will be asked to make formal pledges of resources (both CPU and storage) and those resources will be accounted for and considered in-kind contributions to the collaboration.
As illustrated above, several international partners are already making substantial contributions. We are currently integrating additional large national facilities. Most resources are currently opportunistic but Fermilab and CERN have committed several thousand cores and several PB of disk and the UK reserves 10\% of GridPP resources for non-LHC experiments, an allocation that DUNE has already benefited from.

\begin{dunetable}
[DUNE S+C Consortium Members as of February 20 19]{lll}{tab:exec-comp-consortium}{DUNE S+C Consortium Members as of February 2019, -- indicates sites not yet integrated into production computing. }%\rowtitlestyle
Institution& Country \\%& Integrated\\
KISTI&Korea\\%&--\\
TIFR  & India \\%& in process \\
Nikhef&NL\\%&Yes\\
Edinburgh&UK\\%&Yes\\
GridPP&UK\\%&\\%Yes \\
Manchester&UK\\%&Yes\\
RAL/STFC&UK\\%&Yes\\
Argonne&USA\\%&--\\
BNL&USA\\%&Yes\\
Cincinnati&USA\\%& Yes\\
Colorado State&USA\\%& Yes\\
CU Boulder&USA\\%&Yes\\
Fermilab&USA\\%& Yes \\
Florida &USA\\%& Yes\\
LBNL&USA\\%&Yes\\
Minnesota&USA\\%&Yes\\
Northern Illinois Univ.\\%&USA& --\\
Notre Dame&USA\\%&Yes\\
Oregon State University&USA\\%&Yes\\
Tennessee&USA\\%&--\\
Texas, Austin&USA\\%&--\\
\end{dunetable}

\subsection{People}

The \ldshort{csc} has (or will have) subgroups for the following areas.  Highlights of some of the ongoing projects are detailed in subsequent sections. 

\begin{itemize}
    \item 
Collaborative Tools
\item Data Storage and Management
\item Databases 
\item Production and Processing 
\item Workflow Management
\item Data Quality Monitoring 
\item Software Release Management 
\item Core Software led by a Software Architect
\item Advanced Architectures
\item Algorithm liaisons
\item Networking
\end{itemize}
%%%%%%%%%%%%%%%%%%%%
\subsection{Cooperative Work with Other Collaborations}
\label{ch:exec-comp-gov-coop}

The HEP computing community has come together to form an HEP Software Foundation (HSF)\lcite{Alves:2017she} which, through working groups, workshops and white-papers is guiding the next generation of shared HEP software.  DUNE's time-scale, where we are in the planning and evaluation phase, is almost perfect for us to contribute to and benefit from these efforts.  Our overall strategy for computing infrastructure is to carefully evaluate existing and proposed field-wide solutions, to participate in their design where they are useful and to only build our own solutions where the common solutions do not fit and additional joint development is not feasible.   This section describes some of these common activities. 



\subsection{\dword{larsoft} for event reconstruction}

The \dword{larsoft} \lcite{Snider:2017wjd} reconstruction package is shared by a collaboration of LAr neutrino experiments.  MicroBooNE, SBND, DUNE and others share in the development of a common core software framework with customization for each experiment. The existence of this software suite and prior efforts by other experiments is what made the rapid reconstruction of the ProtoDUNE-SP data possible.  DUNE will be a major contributor to  the future evolution of this package, in particular in introducing full multi-threading to allow parallel reconstruction of parts of large events in anticipation of the very large events expected from the full detector. 



\subsection{Relation to WLCG/OSG}
The Worldwide LHC Computing Grid organization (\dshort{wlcg})\lcite{Bird:2014ctt}, which currently combines the resource and infrastructure missions for the LHC experiments, has proposed a future governance structure that splits the dedicated resource provision for LHC experiments from the general middleware infrastructure used to access those resources.  This Scientific Computing Infrastructure (\dshort{sci}) is already used by many other experiments worldwide.  In a white paper submitted to the European Strategy Group in December 2018\lcite{bib:BirdEUStrategy}, a formal Scientific Computing Infrastructure organization is proposed. As part of the transition to that structure the DUNE collaboration has been provisionally invited to join the WLCG with observer status and participate in the Grid Management Board. The goal of our participation is to have input into the technical decisions on global computing infrastructure while contributing to that infrastructure. 

Areas of collaboration are described in the following subsections. 



\subsubsection{Rucio Development and Extension}

 \dword{rucio}
\cite{Barisits:2019fyl}
is a data management system originally developed by the ATLAS collaboration and now an open-source project.  DUNE has chosen to use \dword{rucio} for our large scale data movement.  In the short term it is being combined with the \dword{sam} data catalog used by Fermilab experiments.  DUNE collaborators at FNAL and in the UK are actively collaborating in the \dword{rucio} project, adding value for DUNE but also the wider effort.


There is a global \dword{rucio} team which now includes Fermilab and BNL staff, DUNE and CMS collaborators,  in addition to the core developers on ATLAS who wrote it initially.  Consortium members have started collaborative work on several projects.   These include (a) making object stores (such as Amazon S3 and compatible utilities) work with Rucio.  There is a large object store in the UK on which DUNE has a sizable allocation.  (b) Monitoring  and administration of the \dword{rucio} system, leveraging the Landscape system at Fermilab.  (c) Designing a  data description engine that can be used as a replacement for the SAM system we currently use.

Rucio has already been shown to be a powerful and useful tool for moving defined datasets from point A to point B.  Our initial observation is that \dword{rucio} is a good solution for file localization but is missing the detailed tools for data description and granular dataset definition available in the current \dword{sam} system.  The rapidly varying conditions in the test beam have highlighted a need for a sophisticated data description database interfaced to \dword{rucio}'s location functions. 

In addition,   LHC experiments such as ATLAS and CMS work with disk stores and tape stores that are independent of each other.  This is different than the dCache model used by most Fermilab experiments in which most of dCache is a caching frontend for a tape store.  Efficient integration of caching into the \dword{rucio} model will be an important component for DUNE unless  we can afford to have most data on disk to avoid staging.

\subsubsection{Testing of New Storage Technologies and Interfaces}

There is currently a Data Organization, Management, and Access (\dword{doma}) taskforce working in the larger HEP community\lcite{Berzano:2018xaa}
 in which several DUNE collaborators are involved. There are task forces for authorization, caching, third party copy, hierarchical storage, and quality of service. All of these are of interest to DUNE as they will determine the long-term standards for common computing infrastructure in the field. 
In particular, the authorization issues have significant impact on DUNE and are covered in subsection \ref{ch-comp-auth}.


\subsubsection{Data Management and Retention Policy Development}



There is a data life cycle built into the DUNE data model.  Obsolete samples such as old simulations and histograms and old commissioning data do not have to be maintained indefinitely.  
We are organizing the structure of lower storage so that the various retention types are stored separately for easy deletion when necessary.  

\subsubsection{Authentication and Authorization Security and Interoperability}\label{ch-comp-auth}

Within the next 2-3 years we expect the global HEP community to make significant changes in the methods of authentication and authorization of computing and storage. 
Over that period, DUNE needs to collaborate with the US and European HEP computing communities on improved authentication methods  that will allow secure but transparent access to storage and other resources such as logbooks and code repositories.  The current model where individuals need to authenticate through different mechanisms for access to US and European resources is already a roadblock to efficient integration  of personnel and storage. 
Current efforts to expand the trust realm between CERN and Fermilab should allow single sign on for each to provide access to the other lab.





%\todo{\verbatim{Add reference to http://wlcg-docs.web.cern.ch/wlcg-docs/technical_documents/HEP-Computing-Evolution.pdf}}



\subsubsection{Evaluations of other important infrastructure}

The DUNE S+C effort is still evaluating major infrastructure components, notably databases and workflow management systems.

For databases\cite{Laycock:2019ynk}, the Fermilab Conditions Database is being used for the first run of ProtoDUNE but the Belle II\cite{Ritter:2018jxh} system supported by BNL is also being considered for subsequent runs. 

For workflow management, we are evaluating \dword{dirac}\cite{Falabella:2016waj} and plan to investigate PANDA \cite{Megino:2017ywl} in comparison with the current GlideInWMS, HT Condor, and POMS solution that was successfully used for the 2018 ProtoDUNE campaigns.
Both of \dword{dirac} and PANDA are used by multiple LHC and non-LHC experiments and are already being integrated with \dword{rucio}. 


\section{Conclusion}

The DUNE Software and Computing efforts have already undergone a substantial test with the successful run of ProtoDUNE-SP, including demonstration of data movement to storage at 2GB/s, reconstruction with high quality algorithms of the full test beam sample and the start of analysis of the multiple PB of reconstructed data. 

The Consortium is now working with the global HEP computing community to evaluate and specify modern infrastructure that will serve the needs of DUNE and the rest of the community.  We plan to collaborate wherever possible with other experiments where we have common technical challenges. However, the extremely large but simple events generated by Liquid Argon TPC's, even with short readouts, present a unique challenge. 

Over the next two years our major activities  will be  thorough reviews of available and potential tools, continued growth of collaborations and acquisition of the resources necessary to launch this large suite of ambitious projects. 
%%%%%%%%%%%%%%%%%%%%
%\subsection{Resource Needs}
%\label{ch:exec-comp-gov-res}

%%%%%%%%%%%
%\subsubsection{Hardware}
%\label{ch:exec-comp-gov-res-hw}

%%%%%%%%%%%
%\subsubsection{Personnel}
%\label{ch:exec-comp-gov-res-hum}

%%%%%%%%%%%%%%%%%%%%
%\subsection{Contribution Models}
%\label{ch:exec-comp-gov-contrib}
%Why is this cutting off here on the page?

%%%%%%%%%%%%%%%%%%%%
%\subsection{Technical Decision Governance}
%\label{ch:exec-comp-gov-tech}

%%%%%%%%%%%%%%%%%%%%
%\subsection{Resource Decision Governance}
%\label{ch:exec-comp-gov-resdec}

%%%%%%%%%%%%%%%%%%%%
%\subsection{Project Management}
%\label{ch:exec-comp-gov-pm}


%%%%%%%%%%%%%%%%%%%%%%%%%%%%%%%  REMOVE THESE %%%%%%%%%%%%%
%\printglossaries
%\end{document}

%\ignore{
%  REFERENCES 
%
%@article{Snider:2017wjd,
%      author         = ``Snider, E. L. and Petrillo, G.'',
%      title          = ``{LArSoft: Toolkit for Simulation, Reconstruction and
%                        Analysis of Liquid Argon TPC Neutrino Detectors}'',
%      booktitle      = ``{Proceedings, 22nd International Conference on Computing
%                        in High Energy and Nuclear Physics (CHEP2016): San
%                        Francisco, CA, October 14-16, 2016}'',
%      journal        = ``J. Phys. Conf. Ser.'',
%      volume         = ``898'',
%      year           = ``2017'',
%      number         = ``4'',
%      pages          = ``042057'',
%      doi            = ``10.1088/1742-6596/898/4/042057'',
%      reportNumber   = ``FERMILAB-CONF-17-052-CD'',
%      SLACcitation   = ``%%CITATION = 00462,898,042057;%%''
%}
%
%REFERENCES
%
%@article{Snider:2017wjd,
%      author         = ``Snider, E. L. and Petrillo, G.'',
%      title          = ``{LArSoft: Toolkit for Simulation, Reconstruction and
%                        Analysis of Liquid Argon TPC Neutrino Detectors}'',
%      booktitle      = ``{Proceedings, 22nd International Conference on Computing
%                        in High Energy and Nuclear Physics (CHEP2016): San
%                        Francisco, CA, October 14-16, 2016}'',
%      journal        = ``J. Phys. Conf. Ser.'',
%      volume         = ``898'',
%      year           = ``2017'',
%      number         = ``4'',
%      pages          = ``042057'',
%      doi            = ``10.1088/1742-6596/898/4/042057'',
%      reportNumber   = ``FERMILAB-CONF-17-052-CD'',
%      SLACcitation   = ``%%CITATION = 00462,898,042057;%%''
%}
%DQM
%
%https://docs.dunescience.org/cgi-bin/private/ShowDocument?docid=10551
%
%DOMA
%
%@article{Berzano:2018xaa,
%      author         = ``Berzano, Dario and others'',
%      title          = ``{HEP Software Foundation Community White Paper Working
%                        Group -- Data Organization, Management and Access (DOMA)}'',
%      year           = ``2018'',
%      eprint         = ``1812.00761'',
%      archivePrefix  = ``arXiv'',
%      primaryClass   = ``physics.comp-ph'',
%      reportNumber   = ``HSF-CWP-2017-04, FERMILAB-PUB-18-671-CD'',
%      SLACcitation   = ``%%CITATION = ARXIV:1812.00761;%%''
%}
%
%
%%%% contains utf-8, see: https://inspirehep.net/info/faq/general#utf8
%%%% add \usepackage[utf8]{inputenc} to your latex preamble
%
%@article{Laycock:2019ynk,
%      author         = ``Bracko, Marko and Clemencic, Marco and Dykstra, Dave and
%                        Formica, Andrea and Govi, Giacomo and Jouvin, Michel and
%                        Lange, David and Laycock, Paul and Wood, Lynn'',
%      title          = ``{HEP Software Foundation Community White Paper Working
%                        Group Ð Conditions Data}'',
%      year           = ``2019'',
%      eprint         = ``1901.05429'',
%      archivePrefix  = ``arXiv'',
%      primaryClass   = ``physics.comp-ph'',
%      reportNumber   = ``FERMILAB-PUB-19-044-CD'',
%      SLACcitation   = ``%%CITATION = ARXIV:1901.05429;%%''
%}
%
%@article{Calafiura:2018rwe,
%      author         = ``Calafiura, Paolo and others'',
%      editor         = ``Hegner, Benedikt and Kowalkowski, Jim and Sexton-Kennedy,
%                        Elizabeth'',
%      title          = ``{HEP Software Foundation Community White Paper Working
%                        Group - Data Processing Frameworks}'',
%      year           = ``2018'',
%      eprint         = ``1812.07861'',
%      archivePrefix  = ``arXiv'',
%      primaryClass   = ``physics.comp-ph'',
%      reportNumber   = ``HSF-CWP-2017-08, FERMILAB-PUB-18-693-CD'',
%      SLACcitation   = ``%%CITATION = ARXIV:1812.07861;%%''
%}
%
%@article{Berzano:2018xaa,
%      author         = ``Berzano, Dario and others'',
%      title          = ``{HEP Software Foundation Community White Paper Working
%                        Group -- Data Organization, Management and Access (DOMA)}'',
%      year           = ``2018'',
%      eprint         = ``1812.00761'',
%      archivePrefix  = ``arXiv'',
%      primaryClass   = ``physics.comp-ph'',
%      reportNumber   = ``HSF-CWP-2017-04, FERMILAB-PUB-18-671-CD'',
%      SLACcitation   = ``%%CITATION = ARXIV:1812.00761;%%''
%}
%
%
%%%% contains utf-8, see: https://inspirehep.net/info/faq/general#utf8
%%%% add \usepackage[utf8]{inputenc} to your latex preamble
%
%@article{Bellis:2018hej,
%      author         = ``Bellis, Matthew and others'',
%      title          = ``{HEP Software Foundation Community White Paper Working
%                        Group Ð Visualization}'',
%      year           = ``2018'',
%      eprint         = ``1811.10309'',
%      archivePrefix  = ``arXiv'',
%      primaryClass   = ``physics.comp-ph'',
%      reportNumber   = ``HSF-CWP-2017-15, FERMILAB-PUB-18-710-CD'',
%      SLACcitation   = ``%%CITATION = ARXIV:1811.10309;%%''
%}
%
%@article{Hildreth:2018tsn,
%      author         = ``Hildreth, M. D. and others'',
%      title          = ``{HEP Software Foundation Community White Paper Working
%                        Group - Data and Software Preservation to Enable Reuse}'',
%      year           = ``2018'',
%      eprint         = ``1810.01191'',
%      archivePrefix  = ``arXiv'',
%      primaryClass   = ``physics.comp-ph'',
%      reportNumber   = ``HSF-CWP-2017-06, FERMILAB-FN-1060-CD'',
%      SLACcitation   = ``%%CITATION = ARXIV:1810.01191;%%''
%}
%
%@article{Albertsson:2018maf,
%      author         = ``Albertsson, Kim and others'',
%      title          = ``{Machine Learning in High Energy Physics Community White
%                        Paper}'',
%      booktitle      = ``{Proceedings, 18th International Workshop on Advanced
%                        Computing and Analysis Techniques in Physics Research
%                        (ACAT 2017): Seattle, WA, USA, August 21-25, 2017}'',
%      journal        = ``J. Phys. Conf. Ser.'',
%      volume         = ``1085'',
%      year           = ``2018'',
%      number         = ``2'',
%      pages          = ``022008'',
%      doi            = ``10.1088/1742-6596/1085/2/022008'',
%      eprint         = ``1807.02876'',
%      archivePrefix  = ``arXiv'',
%      primaryClass   = ``physics.comp-ph'',
%      reportNumber   = ``FERMILAB-PUB-18-318-CD-DI-PPD'',
%      SLACcitation   = ``%%CITATION = ARXIV:1807.02876;%%''
%}
%
%@article{Berzano:2018krv,
%      author         = ``Berzano, Dario and others'',
%      title          = ``{HEP Software Foundation Community White Paper Working
%                        Group - Training, Staffing and Careers}'',
%      collaboration  = ``HEP Software Foundation'',
%      year           = ``2018'',
%      eprint         = ``1807.02875'',
%      archivePrefix  = ``arXiv'',
%      primaryClass   = ``physics.ed-ph'',
%      reportNumber   = ``HSF-CWP-2017-02'',
%      SLACcitation   = ``%%CITATION = ARXIV:1807.02875;%%''
%}
%
%@article{Bauerdick:2018qjx,
%      author         = ``Bauerdick, Lothar and others'',
%      editor         = ``Neubauer, Mark S.'',
%      title          = ``{HEP Software Foundation Community White Paper Working
%                        Group - Data Analysis and Interpretation}'',
%      collaboration  = ``HEP Software Foundation'',
%      year           = ``2018'',
%      eprint         = ``1804.03983'',
%      archivePrefix  = ``arXiv'',
%      primaryClass   = ``physics.comp-ph'',
%      reportNumber   = ``HSF-CWP-2017-05, FERMILAB-FN-1057-CD-PPD'',
%      SLACcitation   = ``%%CITATION = ARXIV:1804.03983;%%''
%}
%
%@article{Apostolakis:2018ieg,
%      author         = ``Apostolakis, J and others'',
%      editor         = ``Elvira, V and Harvey, J'',
%      title          = ``{HEP Software Foundation Community White Paper Working
%                        Group - Detector Simulation}'',
%      collaboration  = ``HEP Software Foundation'',
%      year           = ``2018'',
%      eprint         = ``1803.04165'',
%      archivePrefix  = ``arXiv'',
%      primaryClass   = ``physics.comp-ph'',
%      reportNumber   = ``HSF-CWP-2017-07, FERMILAB-FN-1054-CD'',
%      SLACcitation   = ``%%CITATION = ARXIV:1803.04165;%%''
%}
%
%@article{Albrecht:2018iur,
%      author         = ``Albrecht, Johannes and others'',
%      title          = ``{HEP Community White Paper on Software Trigger and Event
%                        Reconstruction}'',
%      year           = ``2018'',
%      eprint         = ``1802.08638'',
%      archivePrefix  = ``arXiv'',
%      primaryClass   = ``physics.comp-ph'',
%      reportNumber   = ``FERMILAB-PUB-18-071-CD'',
%      SLACcitation   = ``%%CITATION = ARXIV:1802.08638;%%''
%}
%
%@article{Albrecht:2018zgl,
%      author         = ``Albrecht, Johannes and others'',
%      title          = ``{HEP Community White Paper on Software Trigger and Event
%                        Reconstruction: Executive Summary}'',
%      year           = ``2018'',
%      eprint         = ``1802.08640'',
%      archivePrefix  = ``arXiv'',
%      primaryClass   = ``physics.comp-ph'',
%      reportNumber   = ``FERMILAB-PUB-18-072-CD'',
%      SLACcitation   = ``%%CITATION = ARXIV:1802.08640;%%''
%}
%
%@article{Couturier:2017cgq,
%      author         = ``Couturier, Benjamin and others'',
%      title          = ``{HEP Software Foundation Community White Paper Working
%                        Group - Software Development, Deployment and Validation}'',
%      year           = ``2017'',
%      eprint         = ``1712.07959'',
%      archivePrefix  = ``arXiv'',
%      primaryClass   = ``physics.comp-ph'',
%      reportNumber   = ``HSF-CWP-2017-13'',
%      SLACcitation   = ``%%CITATION = ARXIV:1712.07959;%%''
%}
%
%@article{Alves:2017she,
%      author         = ``Albrecht, Johannes and others'',
%      title          = ``{A Roadmap for HEP Software and Computing R\&D for the
%                        2020s}'',
%      collaboration  = ``HEP Software Foundation'',
%      journal        = ``Comput. Softw. Big Sci.'',
%      volume         = ``3'',
%      year           = ``2019'',
%      number         = ``1'',
%      pages          = ``7'',
%      doi            = ``10.1007/s41781-018-0018-8'',
%      eprint         = ``1712.06982'',
%      archivePrefix  = ``arXiv'',
%      primaryClass   = ``physics.comp-ph'',
%      reportNumber   = ``HSF-CWP-2017-01, HSF-CWP-2017-001,
%                        FERMILAB-PUB-17-607-CD'',
%      SLACcitation   = ``%%CITATION = ARXIV:1712.06982;%%''
%}
%
%
%
%
%
%
%
%
%
%
%
%
%}% end ignore
