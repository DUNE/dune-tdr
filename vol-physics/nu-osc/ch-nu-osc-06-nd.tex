\section{Near Detector and Uncertainties}\label{sec:nu-osc-06}\label{sec:physics-lbnosc-ND}
%{\it Assigned to:} {\bf Chris Marshall} with contributions from Mike Kordosky and Steven Manly and also from Kendall Mahn and Mike Wilking and Justo Martin-Albo.

%This section describes assumptions in the Near Detector simulation and ``reconstruction". Some thought has to go into the connection with the ND CDR and any other ND description in the TDR.



%\begin{itemize}
%    \item The ND concept
%    \item Simulations and ``Reconstruction"
%    \item Event Selections
%    \item Samples
%    \item  Detector Response Systematic Uncertainties
%    \item Connection to Flux and Cross Section Systematic Uncertainty
%    \item DUNE PRISM
%\end{itemize}

\subsection{The Near Detector concept}
\label{sec:ndconcept}

The DUNE near detector (ND) will be located 574 m from the neutrino target at Fermilab and approximately 60 m underground. The hall is oriented at 90 degrees with respect to the beam axis to facilitate measurements at both on-axis and off-axis locations. The detector concept consists of a liquid argon time-projection chamber (LAr TPC) functionally coupled to a magnetized multi-purpose tracker (MPT). The MPT includes a high-pressure gaseous argon (HPG) TPC, surrounded by an electromangetic calorimeter (ECal), and a magnet. This section describes the basic near detector concept, as well as the specific assumptions made in the oscillation sensitivity analysis presented in this chapter. Optimization studies and prototyping are ongoing, and the actual detector may be somewhat different from what is described. More details on the design can be found in Chapter~\ref{ch:ndexecutivesummary}.

The LAr TPC is a modular detector with pixelated readout, based on the concept being studied by the Argon Cube collaboration~\cite{ArgonCube}, and is the most upstream component of the ND. Each module is itself a LAr TPC with two anode planes and a central cathode. The active dimensions are $1 \times 3 \times 1$~m ($x \times y \times z$), where the $z$ direction is $6^{\circ}$ upward from the neutrino beam, and the $y$ direction points upward. Charge drifts in the $\pm x$ direction, with a maximum drift distance of 50 cm for electrons created in the center of a module. The module design is described in detail in Ref.~\cite{ArgonCube}. The full LAr detector consists of an array of modules in a single cryostat. The minimum active size for full containment of hadronic showers is $3 \times 4 \times 5$~m. High-angle muons can also be contained by extending the width to 7 m. For this analysis, 35 modules are arranged in an array 5 modules deep in the $z$ direction and 7 modules across in $x$ so that the total active dimensions are $7 \times 3 \times 5$~m. The total active LAr volume is $105 m^{3}$, corresponding to a mass of 147 tons.

The anode planes are tiled with readout pads, such that the $yz$ coordinate is given by the pad location and the $x$ coordinate is given by the drift time, and the three-dimensional position of an energy deposit is uniquely determined. A dedicated, low-power readout ASIC is being developed, which will enable single-pad readout without analog multiplexing~\cite{LArPix}. The module walls orthogonal to the anode and cathode are lined with a photon detector that is sensitive to scintillation light produced inside the module, called ArCLight~\cite{ArCLight}. The detector is optically segmented, and tiled so that the vertical position of the optical flash can be determined with $\sim$30 cm resolution. It is therefore possible to isolate flashes to a volume of roughly 0.3 m$^{3}$, and associate them to a specific neutrino interaction even in the presence of pile-up. The neutrino interaction time, $t_{0}$, is determined from the prompt component of the scintillation light. Absent a scintillation signal, the pulsed nature of the beam gives $\sim$1.6 cm position resolution in the drift direction.

% exact dimensions of PV in simulation are 273cm radius, 518cm axis
The MPT sits immediately downstream of the LAr TPC so that in the on-axis position, the beam center crosses the exact center of both the LAr and HPG active volumes. The HPG TPC is cylindrical, approximately 5 m in diameter and 5 m long, and will reuse existing readout chambers from the ALICE experiment. It is divided into two drift regions by a central cathode, and filled with a 90/10 Ar/CO$_{2}$ gas mixture, pressurized to 10 atmospheres. The gas TPC is described in detail in Ref.~\cite{gasTPC}. \fixme{This is a private DocDB that should be made into a public note} It is surrounded by a cylindrical pressure vessel and a highly granular, tiled ECal. The ECal is composed of a series of absorber layers, followed by arrays of scintillator tiles. Each tile is read out by a SiPM mounted to a 1 mm thick printed circuit board. The ECal design is described in Ref.~\cite{CALICEecal}. The entire MPT is inside a magnetic field with a central field strength of 0.4 T. The reference magnet design is a conventional dipole, described in the DUNE CDR~\cite{cdr-vol-4}. Solenoid and superconducting coil designs are also being considered.

For this study, the pressure vessel is 3 cm titanium. The ECal tiles are $10 \times 10 \times 5$~mm, with a 2 mm copper absorber. The inner-most 10 ECal layers are inside the pressure vessel, giving excellent angular resolution to photon-induced showers, which generally do not convert in the gas. An additional 20 ECal layers are positioned outside the pressure vessel to contain energy from these showers. The magnet is a solenoid with an inner radius of 320 cm and total length along the cylinder axis of 780 cm. The yoke is cut out in the upstream barrel to minimize the passive material between the two TPC detectors.

\subsection{Liquid argon simulation and parameterized reconstruction}
\label{sec:larndsimreco}

Neutrino interactions are simulated in the active volumes of the LAr and HPG TPCs described in Section~\ref{sec:ndconcept}. The neutrino flux prediction is described in Section~\ref{ch:osc-bm-nus}.%\label{sec:nu-osc-05}
Interactions are simulated with the GENIE event generator using the model configuration described in Section~\ref{sec:nu-osc-05}. The propagation of neutrino interaction products through the detector volumes is simulated using a Geant4-based model. Pattern recognition and reconstruction software has not yet been developed for the near detector. Instead, we perform a parameterized reconstruction based on true energy deposits in active detector volumes as simulated by Geant4.

The LAr fiducial region is a central $6 \times 2 \times 3$~m$^{3}$~volume with a 50 cm active buffer on the sides and upstream end, and a 150 cm active buffer on the downstream end. Hadronic energy is estimated from energy deposits in the active LAr volume only. A veto region is defined as the outer 30 cm of the active volume on all sides. Events with more than 30 MeV total energy deposit in the veto region are excluded from analysis, as this energy near the detector edge suggests leakage, resulting in poor energy reconstruction. Even with the containment requirement, events with large shower fluctuations to neutral particles can still be very poorly reconstructed. Neutrons, in particular, are largely unreconstructed energy.

Electrons are reconstructed calorimetrically in the liquid argon. The radiation length is 14 cm in LAr, so for fiducial interactions there are between 10 and 30 radiation lengths between the vertex and the edge of the TPC. As there is no magnetic field in the LAr TPC region, electrons and positrons cannot be distinguished.

Muons with kinetic energy greater than $\sim$1 GeV typically exit the LAr. An energetic forward-going muon will pass through the ECal and into the gaseous TPC, where its momentum and charge are reconstructed by curvature. For these events, it is possible to differentiate between $\mu^{+}$ and $\mu^{-}$ event by event. Muons that stop in the LAr or ECal are reconstructed by range. Exiting muons that do not match to the HPG TPC are not reconstructed, and events with these tracks are rejected from analysis. The asymmetric transverse dimensions of the LAr volume make it possible to reconstruct wide-angle muons with some efficiency. High-angle tracks are typically lost when the $\nu-\mu$ plane is nearly parallel to the $y$ axis, but are often contained when it is nearly parallel to the $x$ axis. 

The charge of stopping muons in the LAr volume cannot be determined. However, the wrong-sign flux is predominantly concentrated in the high-energy tail, where leptons are likelier to be forward and energetic. In FHC mode, the wrong-sign background in the focusing peak is negligibly small, and $\mu^{-}$ is assumed for all stopping muon tracks. In RHC mode, the wrong-sign background is larger in the peak region. Furthermore, high-angle leptons are generally at higher inelasticity, which enhances the wrong-sign contamination in the contained muon sample. To mitigate this, a Michel electron is required. The wrong-sign $\mu^{-}$ captures on Ar with 75\% probability, effectively suppressing the relative $\mu^{-}$ component by a factor of four.

Events are classified as either $\nu_{\mu}$ CC, $\bar{\nu}_{\mu}$ CC, $\nu_{e}$+$\bar{\nu}_{e}$ CC, or NC. True muons and charged pions are evaluated as potential muon candidates. The track length is determined by following the true particle trajectory until it hard scatters or ranges out. The particle is classified as a muon if its track length is at least 1 m, and the mean energy deposit per centimeter of track length is less than 3 MeV. The mean energy cut rejects tracks with detectable hadronic interactions. The minimum length requirement imposes an effective threshold on true muons of about 200 MeV kinetic energy, but greatly suppresses potential NC backgrounds with short, non-interacting charged pions.

True electrons are reconstructed with an ad-hoc efficiency that is zero below 300 MeV, and rises linearly to unity between 300 and 700 MeV. Neutral-current backgrounds arise from photon and $\pi^{0}$ production. Photons are misreconstructed as electrons when the energy deposit per centimeter in the first few cm after conversion is less than 4 MeV. This is typically for Compton scatters, and can also occur due to a random downward fluctuation in the $e^{+}e^{-}$ dE/dx. The conversion distance must also be small so that no visible gap can be identified. We consider a photon gap to be clear when the conversion distance is greater than 2cm, which corresponds to at least three pad widths. For $\pi^{0}$ events, the second photon must also be either less than 50 MeV, or have an opening angle to the first photon less than 10 mrad. It is possible for CC $\nu_{\mu}$ events to be reconstructed as CC $\nu_{e}$ when the muon is too soft and a $\pi^{0}$ fakes the electron.

LAr events are classified as $\nu_{\mu}$ CC, $\bar{\nu}_{\mu}$ CC, $\nu_{e}$ + $\bar{\nu}_{e}$ CC, or NC. Charged-current events are required to have exactly one reconstructed lepton of the apropriate flavor. The muon-flavor samples are separated by reconstructed charge, but the electron sample is combined because the charge cannot be determined. The neutral-current sample includes all events with zero reconstructed leptons.

Events with exiting tracks that do not enter the HPG TPC are rejected. These are predominantly muon CC, where the muon momentum cannot be determined. Events with more than 30 MeV of visible hadronic energy in the veto region are also excluded.

\begin{dunefigure}[ND reconstruction]{fig:NDreco}
{Left: Reconstructed vs. true neutrino energy for $\nu_{\mu}$ CC events with either contained or MPT-matched muon and ell-contained hadronic shower. The bin at zero true energy corresponds to events that are not true $\nu_{\mu}$ CC, which are mostly neutral current with a charged pion faking the muon track, hence the lower energy spectrum. Center: Detector acceptance for $\nu_{\mu}$ CC events as a function of muon transverse and longitudinal momentum. Right: Acceptance as a function of hadronic energy; the black line is for the full fiducial volume while the red line is for a $1 \times 1 \times 1$~m$^{3}$ volume in the center, and the blue curve is the expected distribution of hadronic energy given the DUNE flux.}
 \includegraphics[width=0.3\textwidth]{true_reco_Ev.pdf}
 \includegraphics[width=0.3\textwidth]{pL_pT_eff.pdf}
 \includegraphics[width=0.3\textwidth]{Ehad_eff.pdf}
\end{dunefigure}

\begin{dunefigure}[ND selected samples]{fig:recoEvcats}
{Reconstructed neutrino energy for events classified as $\nu_{\mu}$ CC, $\nu_{e}$ CC, and NC in FHC mode. The colors correspond to true neutrino flavor.}
 \includegraphics[width=0.3\textwidth]{recoE_muCC.pdf}
 \includegraphics[width=0.3\textwidth]{recoE_eCC.pdf}
 \includegraphics[width=0.3\textwidth]{recoE_NC.pdf}
\end{dunefigure}

Selection in an off-axis location of the LAr detector is similar to the on-axis case. The event rate, after selection, is shown as a function of off-axis position in Table~\ref{table:evrates_LAR}.  The predictions are POT-scaled to an illustrative year-long run plan that takes 50\% of the available POT on-axis spreads the remaining beam time equally among  twelve off-axis positions. \fixme{Update text to run plan you want to see current approximate number  of positions?} 

\subsection{Gaseous argon simulation and parameterized reconstruction}
\label{sec:garndsimreco}

\fixme{Tanaz should write this}

\todo{Update numbers for table: LAr  event rates in slices of off-axis position.}

\begin{table}
\begin{tabular}{ l l || c c | c || c c | c | c || c }
\multirow{3}{*}{Offset} & \multirow{3}{*}{$10^{19}$POT} & \multicolumn{7}{c||}{CCInc} & NCInc \\
\cline{3-10}
& & \multicolumn{3}{c||}{$\mu$ contained} & \multicolumn{3}{c|}{$\mu$ exit, $\textrm{T}_{\mu}^\textrm{\tiny exit} > 50 \textrm{MeV}$} & \multirow{2}{*}{$\nu_\textrm{e}$} & \multirow{2}{*}{$\nu_{\mu}$} \\
& & $\nu_{\mu}$ & $\epsilon_{\nu_{\mu},\textsc{cc}}$ & $\bar{\nu}_{\mu}$/$\nu_{\mu}$ & $\nu_{\mu}$ & $\epsilon_{\nu_{\mu},\textsc{cc}}$ & $\bar{\nu}_{\mu}$/$\nu_{\mu}$ & & \\ \hline
 0~m  &  55  & 6.6E5 & 3\% & 1\% & 5.3E6 & 22\% & 3\% & 6.2E4 & 1.8E6 \\
 3~m  &  4.58  & 5.5E4 & 3\% & 1\% & 4.1E5 & 22\% & 3\% & 5.0E3 & 1.4E5  \\
 6~m  &  4.58  & 5.8E4 & 4\% & 1\% & 3.0E5 & 22\% & 4\% & 4.3E3 & 1.1E5 \\
 9~m  &  4.58  & 6.0E4 & 7\% & 2\% & 1.9E5 & 22\% & 4\% & 3.4E3 & 7.5E4 \\
 12~m  &  4.58  & 5.9E4 & 12\% & 3\% & 1.1E5 & 22\% & 5\% & 2.5E3 & 5.2E4 \\
 15~m  &  4.58  & 5.4E4 & 18\% & 3\% & 6.2E4 & 20\% & 6\% & 2.2E3 & 3.7E4 \\
 18~m  &  4.58  & 4.6E4 & 22\% & 4\% & 3.8E4 & 18\% & 8\% & 1.7E3 & 2.7E4 \\
 21~m  &  4.58  & 3.9E4 & 27\% & 5\% & 2.5E4 & 17\% & 9\% & 1.4E3 & 2.1E4 \\
 24~m  &  4.58  & 3.1E4 & 30\% & 6\% & 1.7E4 & 16\% & 9\% & 1.2E3 & 1.6E4 \\
 27~m  &  4.58  & 2.6E4 & 32\% & 7\% & 1.2E4 & 15\% & 10\% & 9.8E2 & 1.3E4 \\
 30~m  &  4.58  & 2.1E4 & 33\% & 7\% & 9.6E3 & 16\% & 12\% & 8.3E2 & 1.0E4 \\
 33~m  &  4.58  & 1.7E4 & 35\% & 8\% & 7.5E3 & 15\% & 13\% & 7.6E2 & 8.3E3 \\
 36~m  &  4.58  & 1.2E4 & 35\% & 8\% & 6.1E3 & 16\% & 15\% & 6.7E2 & 6.6E3 \\
\hline
\hline
\multicolumn{2}{c||}{Totals} & $\nu_{\mu}$ & --- & $\bar{\nu}_{\mu}$ & $\nu_{\mu}$ & --- & $\bar{\nu}_{\mu}$ & $\nu_\textrm{e}$ & $\nu_{\mu}$ \\ \hline
 All  &  110  & 1.1E6 & --- & 1.6E4 & 6.5E6 & --- & 2.2E5 & 8.7E4 & 2.3E6 \\
\end{tabular}

\caption{The selected event rates for a year-long, neutrino-mode run plan. The wrong sign fraction, intrinsic electron neutrino and neutral current event rates are also shown. In all cases, the hadronic containment cut is applied, and the (anti-)muon neutrino events are separated into two samples depending on the containment topology of the final state muon.}
\label{table:evrates_LAR}
\end{table}

The efficiency to select events in an off-axis position varies in a well-defined way based on muon range, angle and hadronic system. Interactions that occur near the edge of the LAr fiducial region produce muons that are more likely to exit the detector volume with sufficient kinetic energy than those occurring in the center of a detector; at increasingly larger off-axis positions, the lower average energy of the neutrino beam produces more contained muons. Even though the neutrino energy spectra varies significantly as a function of off-axis angle, the efficiency for containing the hadronic system only depends on the fractional energy transferred to the hadronic system, as expected. \fixme{Mike: Please clean up this text}

For this document, rates of the LAr are presented in Table~\ref{table:evrates_LAR} are separated based on whether or not the muon is contained (and there is an MPT). Regardless, the table shows that it will be possible to collect sufficient statistics in all off-axis positions with a run plan in which 50\% of the data is collected with the detector in the on-axis position. 

\subsection{Near Detector Systematic Uncertainties}

Uncertainties are evaluated on the muon and hadron acceptance, as well as on the energy reconstruction. The detector acceptance for muons and hadrons is shown in Figure~\ref{fig:NDreco}. Inefficiency at very low lepton energy is due to events being misreconstructed as neutral current, which can also be seen in Figure~\ref{fig:recoEvcats}. For high energy, forward muons, the inefficiency is only due to events near the edge of the fiducial volume where the muon misses the MPT. At high transverse momentum, muons begin to exit the side of the LAr active volume, except when they happen to go along the 7m axis. The acceptance is sensitive to the modeling of muons in the detector. An uncertainty is estimated based on the change in the acceptance as a function of muon kinematics.

Inefficiency at high hadronic energy is due to the veto on more than 30 MeV deposited in the outer 30cm collar of the active volume. Rejected events are typically poorly reconstructed due to low containment, and the acceptance is expected to decrease at high hadronic energy. Similar to the muon reconstruction, this acceptance is sensitive to detector modeling, and an uncertainty is evaluated based on the change in the acceptance as a function of true hadronic energy.

Energy scale uncertainties arise due to the precision of the detector calibration. Overall calibration uncertainties are implemented, as well as particle response uncertainties that are different for muons, charged hadrons, neutrons, and electromagnetic showers. Some are partially correlated with the far detector, for example the muon energy in liquid argon. Others are treated as fully uncorrelated, for example the curvature-based MPT muon reconstruction. Observed neutron energy is treated as uncorrelated due to the anticipated challenges of associating neutrons. The full list of energy scale uncertainties is given as Table~\ref{tab:NDenergyScaleSysts}.

%% Please check it compiles before moving to dunetable! :p
\begin{table}
\begin{tabular}{|c|c|c|c|}
    \hline
    Systematic & Particle(s) & ND effect & FD effect \\
    \hline
    Muon LAr  & $\mu$ & 1\% (contained $\mu$ only) & 1\% \\
    Muon GAr ND & $\mu$ & 1\% & 0 \\
    Electromagnetic & $e$ & 2\% & 2\% \\
    Charged hadron Correlated & $P, \pi^{\pm}$ & 5\% & 5\% \\
    Charged hadron ND &  $p, \pi^{\pm}$ & 1\% & 0 \\
    Charged hadron FD &  $p, \pi^{\pm}$ & 0 & 1\% \\
    Neutron ND & $n$ & 20\% & 0 \\
    Neutron FD & $n$ & 0 & 20\% \\
    $\pi^{0}$ Correlated & $\pi^{0}$ & 5\% & 5\% \\
    $\pi^{0}$ ND & $\pi^{0}$ & 2\% & 0 \\
    $\pi^{0}$ FD & $\pi^{0}$ & 0 & 2\% \\
    \hline
\end{tabular}
\caption{Near detector energy scale systematic uncertainties.}
\label{tab:NDenergyScaleSysts}
\end{table}

\subsection{Role of Near detector in flux and cross section systematic uncertainty assessment}
% Connection to Flux and Cross Section Systematic Uncertainty

The selected charged-current on- and off-axis samples are fit in two-dimensional bins of neutrino energy, $E_{\nu} = E_{lep} + E_{had}$, and inelasticity, $y = E_{had}/E_{\nu}$ to constrain nuisance parameters associated to the flux and interaction model. Due to significant correlations in the flux at the on-axis position and  cross section physics (same target material) the on-axis event samples are expected to reduce the overall flux and cross section uncertainty by approximately a factor of two. The off-axis event samples will have comparable constraint, and so can serve as a test of the interaction model and/or flux model. The off-axis positions are sensitive to changes of the focusing optics (e.g. horn current or position), and studies demonstrate that the off-axis positions event rates are valuable to diagnose incomplete aspects of the cross section model.

In addition to the analysis presented here, DUNE will employ a range of additional techniques and samples to further constrain the flux and cross section uncertainties: \begin{itemize}
\item Neutrino-electron elastic scattering is a rare process with a well know cross section. A dedicated selection, following work with a comparable flux%Citation: https://inspirehep.net/record/1411311
, can be used to reduce the flux uncertainties.
\item Low-nu method?
\item Alternate projections of the data, such as Ref~%http://inspirehep.net/record/1410087
or exclusive selections separated by final state particles, may be
 used to constrain cross section physics.
\end{itemize}

\subsubsection{Neutrino-electron elastic scattering}
\label{sec:nu+e}

Measurements of neutrino-nucleus scattering are sensitive to the product of the flux and cross section, both of which are uncertain. This can lead to a degeneracy between flux and cross section nuisance parameters in the oscillation fit, and result in significant anti-correlations, even when the uncertainty on the diagonal component is small. One way to break this degeneracy is by including a sample for which the a priori cross section uncertainties are very small. 

Neutrino-electron scattering is a pure-electroweak process with calculable cross section at tree level. The final state consists of a single electron, subject to the kinematic limit 

\begin{equation}
1 - \cos \theta = \frac{m_{e}(1-y)}{E_{e}},
\end{equation}

where $\theta$ is the angle between the electron and incoming neutrino, $E_{e}$ and $m_{e}$ are the electron mass and total energy, respectively, and $y = T_{e}/E_{\nu}$ is the fraction of the neutrino energy transferred to the electron. For DUNE energies, $E_{e} \gg m_{e}$, and the angle $\theta$ is very small, such that $E_{e}\theta^{2} < 2m_{e}$.

The overall flux normalization can be determined by counting $\nu e \rightarrow \nu e$ events. Events can be identified by searching for a single electromagnetic shower with no other visible particles. Backgrounds from $\nu_{e}$ charged-current scattering can be rejected by looking for large energy deposits near the interaction vertex, which are evidence of nuclear breakup. Photon-induced showers from neutral-current $\pi^{0}$ events can be distinguished from electrons by the energy profile at the start of the track. The dominant background is expected to be $\nu_{e}$ charged-current scattering at very low $Q^{2}$, where final-state hadrons are below threshold, and $E_{e}\theta^{2}$ happens to be small. The background rate can be constrained with a control sample at higher $E_{e}\theta^{2}$, but the shape extrapolation to $E_{e}\theta^{2} \rightarrow 0$ is uncertain at the 10-20\% level.

For the DUNE flux, approximately 100 events per year per ton of fiducial mass ar expected with electron energy above 0.5 GeV. For a LAr TPC mass of 25 tons, this corresponds to 2500 events per year, or 12500 events in the full 5-year FHC run, assuming the ND stays on axis. Given the very forward signal, it may be possible to expand the fiducial volume to enhance the rate. The statistical uncertainty on the flux normalization from this technique is expected to be $\sim$1\%.



