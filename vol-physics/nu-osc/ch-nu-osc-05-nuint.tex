\section{Neutrino Interactions and Uncertainties}\label{sec:nu-osc-05} %\label{sec:physics-lbnosc-nuint}


\subsection{Interaction Model Summary}


The goal of parameterizing the neutrino interaction model uncertainties is to provide a framework for considering how these uncertainties affect the oscillation analysis at the \dword{fd}, and for considering how constraints at the \dword{nd} can limit those uncertainties.
 
The model developed for this purpose generally factorizes the neutrino interaction on nuclei into an incoherent sum of hard scattering neutrino interactions with the single nucleons in the nucleus. The effect of the nucleus is implemented as initial and final state interaction effects, with some (albeit few) nucleus-dependent hard scattering calculations. Schematically, we express this concept as $\text{Scattering Process} = \text{Initial State} \otimes \text{Nucleon Interaction} \otimes \text{Final State Propagation}$.

The initial state effects relate to the description of the momentum and position distributions of the nucleons in the nucleus, kinematic modifications to the final state (such as separation energy, or sometimes described as a binding energy), and Coulomb effects.   The concept of binding energy reflects the idea that the struck nucleon may be off the mass-shell inside the nucleus.
Final state interactions refer to the propagation and interaction of hadrons produced in the nucleon interaction through the nucleus. The \dword{fsi} alter both the momentum and energy of the recoiling particles produced in the final state, and may also alter their identity and multiplicity in the case of inelastic reinteractions (e.g., in a nucleus a hadron may be absorbed, rescattered, or create a secondary hadron).  The \dword{fsi} model implemented in the \dword{genie}, \dword{nuwro}, and \dword{neut} neutrino interaction generators is a semi-classical cascade model. In particular, \dword{genie}'s $hA$ model is a single step scaled model, based on hadron-nucleus and hadron-nucleon scattering data and theoretical corrections. % \dword{gibuu}, another available generator, uses a full particle transport model, including support for off-shell hadrons in the nuclear medium.

Generators vary in their attempts to accurately model the largely undetected final state ``spectator'' nuclear system.  The nuclear system can carry away significant undetected momentum---hundreds of MeV is not unusual---in the form of one or more heavy, non-relativistic particles.  These particles typically carry off very little kinetic energy; however they can absorb on the order of tens of MeV of energy from the initial state from breakup or excitation of the target nucleus.  This energy and momentum will typically be invisible to the detector.


The factorization outlined above is not present in all parts of the model.  Most modern generators include ``2p2h'' (two particle, two hole) interactions that model meson exchange processes and scattering on highly correlated pairs of nucleons in the nucleus.  These interactions are often implemented as another process that incorporates both hard scattering and initial state effects in processes that create multiple final state nucleons, with a different prescription for different nuclei.
Neutrino scattering on atomic electrons and the coherent production of pions (which scatters off the entire nucleus) also
do not follow this factorization.

The interaction model and its variations are implemented in the  \dword{genie} generator.  The fixed version of \dword{genie} used for this report, v2.12.10\footnote{At the time of the development of this model for interactions and their uncertainties, initial pieces of \dword{genie} 3 had just recently been released (October 2018) and reweighting and documentation followed after this. The timing made it impractical to use \dword{genie} 3 for this work.}, will not contain all of the possible cross section variations which need to be modeled.  Therefore, the variations in the cross sections to be considered are implemented as some combination of: \dword{genie} weighting parameters (sometimes referred to as ``\dword{genie} knobs''), {\em ad hoc} weights of events that are designed to parameterize uncertainties or cross section corrections currently not implemented within \dword{genie}, and discrete alternative model comparisons, achieved through alternative generators, alternative \dword{genie} configurations, or custom weightings. For the studies presented in this chapter we have  identified classes of uncertainties that are intended to span a representative range of alternative models such as those found in other generators. 

In this work, two example alternative models are used directly to evaluate additional uncertainties in the case where the assumptions about the near detector are relaxed. These studies are described in Section~\ref{sec:ndimpact}. The first is based on the \dword{nuwro} generator and the second is designed to produce the same on-axis visible energy distributions as the nominal model, but with a different relationship between true neutrino energy and visible energy.


\subsection{Interaction Model Uncertainties}

The interaction uncertainties are divided into seven roughly exclusive groups: (1) initial state uncertainties, (2) hard scattering uncertainties and nuclear modifications to the quasielastic process, (3) uncertainties in multinucleon (2p2h) hard scattering processes, (4) hard scattering uncertainties in pion production processes, (5) uncertainties governing other, higher $W$ and neutral current processes, (6) final state interaction uncertainties, (7) neutrino flavor dependent uncertainties. Uncertainties are intended to reflect current theoretical freedom, deficiencies in implementation, and/or current experimental knowledge.  
There are constraints on nuclear effects because of measurements on lighter targets, however for the argon nuclear target some additional sources of uncertainty are identified.  We also discuss cases where the parameterization is limited or simplified.

\subsubsection{Initial State Uncertainties}
The default nuclear model in \dword{genie} is a modified global Fermi gas model of the nucleons in the nucleus.  There are significant deficiencies that are known in global Fermi gas models. These include a lack of consistent incorporation of the tails that result from correlations among nucleons, the lack of correlation between location within the nucleus and momentum of the nucleon, and an incorrect relationship between momentum and energy of the off-shell, bound nucleon within the nucleus. \dword{genie} modifies the nucleon momentum distribution empirically to account for short-range correlation effects, which populates tails above the Fermi cutoff, but the other deficiencies persist. Alternative initial state models, such as spectral functions~\cite{Benhar:1994hw,Nieves:2004wx}, the mean field model of GiBUU~\cite{Gallmeister:2016dnq}, or continuum random phase approximation (CRPA) calculations~\cite{Pandey:2014tza} may provide better descriptions of the nuclear initial state~\cite{Sobczyk:2017mts}. 

\subsubsection{Quasielastic uncertainties}
The primary uncertainties considered in quasielastic interactions are the axial form factor of the nucleon and nuclear screening---from the so-called  \dword{rpa} calculations---of low momentum transfer reactions.

%MAQE Blurb
The axial form factor uncertainty has been historically described with a single parameter uncertainty with the dipole form by varying $M_A$, and we will continue this for these studies.  Unfortunately, this framework overconstrains the form factor at high $Q^2$, and an alternative parameterization based on the $z$-expansion has been proposed as a replacement~\cite{Meyer:2016oeg}.  However, this parameterization is multi-dimensional and poses problems for the analysis framework of this study which factorizes all $N$-dimensional variations out into $N\times{}1$-dimensional analysis bin response functions. For some multi-dimensional parameterizations, this simplification is an adequate approximation, e.g., the BeRPA described below. 

One part of the Nieves et al.\cite{Nieves:2011pp,Gran:2013kda} description of the $0\pi$ interaction on nuclei includes RPA, used to sum the $W^\pm$ self-energy terms. In practice, this modifies the 1p1h/Quasi-Elastic cross-section in a non-trivial way. The calculations from Nieves et al. have associated uncertainties presented in~\cite{nieves_uncert}, 
\fixme{could it be \cite{Valverde:2006zn}, Theoretical uncertainties on quasielastic charged-current neutrino-nucleus cross sections?} which were evaluated as a function of $Q^2$~\cite{sanchez-private}. In 2018, \minerva and \nova parameterized the central value and uncertainty in $(q_0, q_3)$ using RPA uncertainties as parameterized in~\cite{RikRPA}, whereas T2K used central values and uncertainties in $Q^2$ only. Here we use T2K's 2017/8 parameterization of the RPA effect~\cite{Abe:2018wpn}
%\fixme{We have three T2K refs for 2018: could it be \cite{Abe:2018wpn}, Search for CP Violation in Neutrino and Antineutrino
%                        Oscillations by the T2K Experiment with $2.2\times10^{21}$
%                        Protons on Target?  or \cite{Abe:2018pwo} Characterization of nuclear effects in muon-neutrino
%                        scattering on hydrocarbon with a measurement of
%                        final-state kinematics and correlations in charged-current
%                        pionless interactions at T2K? or \cite{Vladisavljevic:2018prd}, Constraining the T2K Neutrino Flux Prediction with 2009
%NA61/SHINE Replica-Target Data? }
due to its simplicity. The shape of the correction and error is parameterized with a Bernstein polynomial up to $Q^2=1.2\text{ GeV}^2$ which switches to a decaying exponential. The BeRPA (Bernstein RPA) function has three parameters controlling the polynomial ($A, B, C$), where the parameters control the behavior at increasing $Q^2$ and a fourth parameter $E$ controls the high $Q^2$ tail.

The axial form factor parameterization we use is known to be inadequate.  However, the convolution of BeRPA uncertainties with the limited axial form factor uncertainties do provide more freedom as a function of $Q^2$, and the two effects likely provide adequate freedom for the $Q^2$ shape in quasielastic events.

\subsubsection{$\boldsymbol{2p2h}$ uncertainties}
We start with the Nieves et al.\ or ``Valencia'' model~\cite{Nieves:2011pp,Gran:2013kda} 
%\fixme{ \cite{Gran:2013kda} Neutrino-nucleus quasi-elastic and 2p2h interactions up
%                        to 10 GeV? It's the only one with 2p2h in the title.}
for multinucleon ($2p2h$) contributions to the cross section.  However, \minerva has shown directly~\cite{Rodrigues:2015hik}, and \nova indirectly, that this description is missing observed strength on carbon.  As a primary approach to the model, we add that missing strength to a number of possible reactions.  We then add uncertainties for energy dependence of this missing strength and uncertainties in scaling the $2p2h$ prediction from carbon to argon.

The extra strength from the ``\minerva tune'' to $2p2h$ is applied in $(q_0,q_3)$ space (where $q_0$ is energy transfer from the leptonic system, and $q_3$ is the magnitude of the three momentum transfer) to fit reconstructed \minerva CC-inclusive data~\cite{Rodrigues:2015hik} in $E_\text{avail}$\footnote{$E_\text{avail}$ is calorimetrically visible energy in the detector, roughly speaking total recoil hadronic energy, less the masses of $\pi^\pm$ and the kinetic energies of neutrons} and $q_3$.  Reasonable fits to \minerva's data are found by attributing the missing strength to any of $2p2h$ from $np$ initial state pairs, $2p2h$ from $nn$ initial state pairs, or $1p1h$ or quasielastic processes.  The default tune uses an enhancement of the $np$ and $nn$ initial strengths in the ratio predicted by the Nieves model, and alternative systematic variation tunes (``MnvTune'' 1-3) attribute the missing strength to the individual hypotheses above. Implementation of the ``MnvTune'' is based on weighting in true $(q_0,q_3)$. The weighting requires \dword{genie}'s Llewelyn-Smith $1p1h$ and Valencia $2p2h$ are used as the base model. To ensure consistency in using these different tunes as freedom in the model, a single systematic parameter is introduced that varies smoothly between applying the $1p1h$ tune at one extreme value to applying the $nn$ tune at the other extreme via the default tune which is used as the central value. The $np$ tune is neglected in this prescription as being the most redundant, in terms of missing energy content of the final state, of the four discrete hypotheses.

The rates for $1p1h$ and $2p2h$ processes could be different on argon and carbon targets.  There is little neutrino scattering data to inform this, but there are measurements of short-ranged correlated pairs from electron scattering on different nuclei~\cite{Colle:2015ena}.  These measurements directly constrain $2p2h$ from short range correlations, although the link to dynamical sources like meson exchange current processes (MEC) is less direct. Interpolation of that data in $A$ (Nucleon number) suggests that scaling from carbon relative to the naive $\propto A$ prediction for $2p2h$ processes would give an additional factor of $1.33\pm 0.13$ for $np$ pairs, and $0.9\pm 0.4$ for $pp$ pairs.
\dword{genie}'s prediction for the ratio of $2p2h$ cross-sections in $\text{Ar}^{40}/\text{C}^{12}$ for neutrinos varies slowly with neutrino energy in the DUNE energy range: from $3.76$ at $1$~GeV to $3.64$ at $5$~GeV. The ratio for antineutrino cross sections is consistent with $3.20$ at all DUNE energies. Since the ratio of $A$ for $\text{Ar}^{40}/\text{C}^{12}$ is $3.33$, this is consistent with the ranges suggested above by the measured $pp$ and $np$ pair scaling.  A dedicated study by the SuSA group using their own theoretical model for the relevant MEC process also concludes that the transverse nuclear response (which drives the $\nu-A$ MEC cross section) ratio between $\text{Ca}^{40}$ (the isoscalar nucleus with the same $A$ as $\text{Ar}^{40}$) and $\text{C}^{12}$ is $3.72$ \cite{Amaro:2017eah}. We vary \dword{genie}'s Valencia model based prediction, including the \minerva tune, for $2p2h$ by $\sim 20\%$ to be consistent with the correlated pair scaling values above. This is done independently for neutrino and antineutrino scattering.

The \minerva tune may be $E_\nu$ dependent. \minerva separated its data into an $E_\nu<6$~GeV and an $E_\nu>$6~GeV piece, and sees no dependence with a precision of better than $10\%$~\cite{Rodrigues:2015hik}.  The mean energy of the $E_\nu<6$~GeV piece is roughly $\left< E_\nu\right>\approx 3$~GeV.  In general, an exclusive cross-section will have an energy dependence $\propto \frac{A}{E_\nu^2}+\frac{B}{E_\nu}+C$~\cite{llewelyn-smith}; therefore, unknown energy dependence may be parameterize by an {\em ad hoc} factor of the form $1/\left(1+ \frac{A^{'}}{E_\nu^2 }+\frac{B^{'}}{E_\nu}\right)$.  The \minerva constraints suggest $A^{'}<0.9$~GeV$^2$ and $B^{'}<0.3$~GeV.  The variations for neutrinos and antineutrinos could be different since this is an effective modification. Ideally this energy dependent factor would only affect the \minerva tune, but practically, because of analysis framework limitations already discussed, this is not possible. As a result, this energy dependent factor is applied to all true $2p2h$ events.

\subsubsection{Single pion production uncertainties}
\dword{genie} uses the Rein-Sehgal model for pion production. Tunes to $D_2$ data have been performed, both by the \dword{genie} collaboration itself and in subsequent re-evaluations~\cite{Rodrigues:2016xjj}; we use the latter tune as our base model. For simplicity of implementation, the `v2.8.2 (no norm.)' results are used here. %Ideally the `v2.8.2 (RES)' would have been used, but the practical effect of this choice on the quoted sensitivities is expected to be small.

\minerva single pion production data~\cite{Altinok:2017xua,McGivern:2016bwh,Eberly:2014mra} indicates disagreement at low $Q^2$ which may correspond to an incomplete nuclear model for single pion production in the generators. A similar effect was observed at MINOS~\cite{Adamson:2014pgc} 
%\fixme{bad citation; no suggestions}
and \nova implements a similar correction in analyses~\cite{nova_2018}.
A fit to \minerva data~\cite{StowellThesis} measured a suppression parameterized by
\begin{align}
R(Q^2<x_3) & = \frac{R_{2} (Q^2-x_1)(Q^2-x_3)}{(x_2-x_1)(x_2-x_3)} \notag\\
             & + \frac{(Q^2-x_1)(Q^2-x_2)}{(x_3-x_1)(x_3-x_2)}  \\
W(Q^{2}) &= 1-(1-R_{1})(1-R(Q^{2}))^{2}
\end{align}
where $R_{1}$ defines the magnitude of the correction function at the intercept, $x_{1}=0.0$. $x_{2}$ is chosen to be $Q^2=0.35\text{ GeV}^2$ so that $R_{2}$ describes the curvature at the center point of the correction. The fit found $R_1\approx0.3$ and $R_2\approx0.6$. The correction is applied to events with a resonance decay inside the nucleus giving rise to a pion, based on \dword{genie} event information.

An improved Rein-Sehgal-like resonance model has recently been developed~\cite{minoo} which includes a non-resonant background in both $I=\frac{1}{2}$ and $I=\frac{3}{2}$ channels and interference between resonant-resonant and resonant-non-resonant states. 
It also improves on the Rein-Sehgal model in describing the outgoing pion and nucleon kinematics using all its resonances.
A template weighting in $(W, Q^2, E_\nu)$ is implemented to cover the differences between the two models as a systematic uncertainty. The weighting also suppresses \dword{genie} non-resonant pion production events (deep inelastic scattering events with $W<1.7\text{ GeV}$) as the new model already includes the non-resonant contribution coherently. 
The weighting is only applied to true muon-neutrino charged-current resonant pion production interactions.

Coherent inelastic pion production measurements on carbon are in reasonable agreement with the \dword{genie} implementation of the Berger-Sehgal model~\cite{Mislivec:2017qfz}.  The process has not been measured at high statistics in argon. While coherent interactions provide a very interesting sample for oscillation analyses, they are a very small component of the event rate and selections will depend on the near detector configuration. Therefore we do not provide any evaluation of a systematic uncertainty for this extrapolation or any disagreements between the Berger-Sehgal model and carbon data.

\subsubsection{Other hard scattering uncertainties}
\nova oscillation analyses~\cite{nova_2018} have found the need for excursions beyond the default \dword{genie} uncertainties to describe their single pion to deep inelastic scattering (DIS) transition region data.  Following suit, we drop \dword{genie}'s default ``Rv[n,p][1,2]pi'' knobs and instead implement separate, uncorrelated uncertainties for all perturbations of 1, 2, and $\geq 3$ pion final states, CC/NC, neutrinos/anti-neutrinos, and interactions on protons/neutrons, with the exception of CC neutrino 1-pion production, where interactions on protons and neutrons are merged, following \cite{Rodrigues:2016xjj}. This leads to 23 distinct uncertainty channels ([3 pion states] x [n,p] x [nu/anti-nu] x [CC/NC] - 1), all with a value of 50\% for $W \leq 3$ GeV.  For each channel, the uncertainty drops linearly above $W = 3$ GeV until it reaches a flat value of 5\% at $W = 5$ GeV, where external measurements better constrain this process.

\subsubsection{Final state interaction uncertainties}\label{sec:fsi}
\dword{genie} includes a large number of final state uncertainties to its $hA$ final state cascade model which are summarized in Table~\ref{table:HadTranspKnobs}.  These uncertainties have been validated in neutrino interactions primarily on light targets such as carbon, but there is very little data available on argon targets.
The lack of tests against argon targets is difficult to address directly because there are many possible \dword{fsi} processes that could be varied.



\subsubsection{Neutrino flavor dependent uncertainties}
The cross sections include terms proportional to lepton mass, which are significant contributors at low energies where quasielastic processes dominate.  Some of the form factors in these terms have significant uncertainties in the nuclear environment.  Ref.~\cite{Day-McFarland:2012} ascribes the largest possible effect to the presence of poorly constrained second-class current vector form factors in the nuclear environment, and proposes a variation in the cross section ratio of $\sigma_\mu/\sigma_e$ of $\pm 0.01/{\rm\textstyle Max}(0.2~{\rm\textstyle GeV},E_\nu)$ for neutrinos and $\mp 0.018/{\rm\textstyle Max}(0.2~ {\rm\textstyle GeV},E_\nu)$ for anti-neutrinos.  Note the anticorrelation of the effect in neutrinos and antineutrinos.

In addition, radiative Coulomb effects may also contribute, which for T2K is of order $\pm5$ MeV shifts in reconstructed lepton momentum.  Like the second class current effect in the cross section, it flips sign between neutrinos and antineutrinos and is significant only at low energies.  This effect is not implemented herein.

Finally, some electron neutrino interactions occur at four momentum transfers where a corresponding muon neutrino interaction is kinematically forbidden, therefore the nuclear response has not been constrained by muon neutrino cross section measurements.  This region at lower neutrino energies has a significant overlap with the Bodek-Ritchie tail of the Fermi gas model. There are significant uncertainties in this region, both from the form of the tail itself, and from the lack of knowledge about the effect of RPA and $2p2h$ in this region. The allowed phase space in the presence of nonzero lepton mass is $E_\nu-\sqrt{\left( E_\nu-q_0\right) ^2-m_l^2}\leq q_3\leq E_\nu+\sqrt{\left( E_\nu-q_0\right) ^2-m_l^2}$. Here, a 100\% variation is allowed in the phase space present for $\nu_e$ but absent for $\nu_\mu$.

A similar prescription cannot applied for differences between interactions of $\nu_\mu$ and $\nu_\tau$ because the $\tau$ mass scale is of the same order of magnitude as the neutrino energies, and is thus a leading effect. No specific uncertainties were developed for $\nu_\tau$ interactions as there is little theoretical guidance.


\subsection{Listing of Interaction Model Uncertainties}

The complete set of interaction model uncertainties includes \dword{genie} implemented uncertainties 
(Tables~\ref{table:NuXSecKnobs}, and \ref{table:HadTranspKnobs}), 
and new uncertainties developed for this effort (Table~\ref{tab:nuintsystlist}) which represent uncertainties beyond those implemented in the \dword{genie} generator.  

\begin{table}[ptb]
\center
\global\long\def\arraystretch{1.75}
\scalebox{0.9}{
\begin{tabular}{llr|l}
\hline
$x_{P}$  & Description of $P$  & $P_\textsc{cv}$ & $\delta{P}/{P}$  \tabularnewline
\hline
&\textbf{Quasielastic}&&\tabularnewline
$x_{M_{A}}^{CCQE}$ & Axial mass for CCQE & & ${}^{+0.25}_{-0.15}$~GeV \tabularnewline
$x_{VecFF}^{CCQE}$  & Choice of CCQE vector form factors (BBA05 $\leftrightarrow$ Dipole)  &  & N/A \tabularnewline
$x_{kF}^{CCQE}$  & Fermi surface momentum for Pauli blocking &   & $\pm$30\% \tabularnewline
&\textbf{Low $\mathbf{W}$}&&\tabularnewline
$x_{M_{A}}^{CCRES}$ & Axial mass for CC resonance & 0.94 & $\pm$0.05~GeV \tabularnewline
$x_{M_{V}}^{CCRES}$  & Vector mass for CC resonance &  & $\pm$10\% \tabularnewline
$x_{\eta\ BR}^{\Delta Decay}$  & Branching ratio for $\Delta\rightarrow\eta$ decay &   & $\pm$50\% \tabularnewline
$x_{\gamma\ BR}^{\Delta Decay}$  & Branching ratio for $\Delta\rightarrow\gamma$ decay &   & $\pm$50\% \tabularnewline
$x_{\theta_{\pi}^{\Delta Decay}}$  & $\theta_{\pi}$ distribution in decaying $\Delta$ rest frame (isotropic $\rightarrow$ RS) &   & N/A \tabularnewline
&\textbf{High $\mathbf{W}$}&\tabularnewline
$x_{A_{HT}^{BY}}^{DIS}$  & $A_{HT}$ higher-twist param in BY model scaling variable $\xi_{w}$  &   & $\pm$25\% \tabularnewline
$x_{B_{HT}^{BY}}^{DIS}$  & $B_{HT}$ higher-twist param in BY model scaling variable $\xi_{w}$  &  & $\pm$25\% \tabularnewline
$x_{C_{V1u}^{BY}}^{DIS}$  & $C_{V1u}$ valence GRV98 PDF correction param in BY model  &  & $\pm$30\% \tabularnewline
$x_{C_{V2u}^{BY}}^{DIS}$  & $C_{V2u}$ valence GRV98 PDF correction param in BY model  &  & $\pm$40\% \tabularnewline
&\textbf{Other neutral current}&&\tabularnewline
$x_{M_{A}}^{NCEL}$  & Axial mass for NC elastic  &  & $\pm$25\% \tabularnewline
$x_{\eta}^{NCEL}$  & Strange axial form factor $\eta$ for NC elastic  &  & $\pm$30\%  \tabularnewline
$x_{M_{A}}^{NCRES}$  & Axial mass for NC resonance & &$\pm$10\% \tabularnewline
$x_{M_{V}}^{NCRES}$  & Vector mass for NC resonance & &$\pm$5\% \tabularnewline
&\textbf{Misc.}&&\tabularnewline
$x_{FZ}$  & Vary effective formation zone length &  & $\pm$50\% \tabularnewline
\hline
\end{tabular}
}% scalebox
\\[2pt]
\caption[Neutrino interaction cross-section systematic parameters considered in GENIE]
{Neutrino interaction cross-section systematic parameters considered in \dword{genie}. \dword{genie} default central values and uncertainties are used for all parameters except $x_{M_{A}}^{CCRES}$. Missing \dword{genie} parameters were omitted where uncertainties developed for this analysis significantly overlap with the supplied \dword{genie} freedom, the response calculation was too slow, or the variations were deemed unphysical.
}
\label{table:NuXSecKnobs}
\end{table}


\begin{table}[ptb]
\center

\global\long\def\arraystretch{1.75}
% enlarge line spacing
\begin{tabular}{llll}
\hline
$x_{P}$  & Description of $P$  & $\delta{P}/{P}$  & \tabularnewline
\hline
% $x_{mfp}^{N}$  & Nucleon mean free path (total rescattering probability)  & $\pm$20\%  & \tabularnewline
$x_{cex}^{N}$  & Nucleon charge exchange probability  & $\pm$50\%  & \tabularnewline
$x_{el}^{N}$  & Nucleon elastic reaction probability  & $\pm$30\%  & \tabularnewline
$x_{inel}^{N}$  & Nucleon inelastic reaction probability  & $\pm$40\%  & \tabularnewline
$x_{abs}^{N}$  & Nucleon absorption probability  & $\pm$20\%  & \tabularnewline
$x_{\pi}^{N}$  & Nucleon $\pi$-production probability  & $\pm$20\%  & \tabularnewline
% $x_{mfp}^{\pi}$  & $\pi$ mean free path (total rescattering probability)  & $\pm$20\%  & \tabularnewline
$x_{cex}^{\pi}$  & $\pi$ charge exchange probability  & $\pm$50\%  & \tabularnewline
$x_{el}^{\pi}$  & $\pi$ elastic reaction probability  & $\pm$10\%  & \tabularnewline
$x_{inel}^{\pi}$  & $\pi$ inelastic reaction probability  & $\pm$40\%  & \tabularnewline
$x_{abs}^{\pi}$  & $\pi$ absorption probability  & $\pm$20\%  & \tabularnewline
$x_{\pi}^{\pi}$  & $\pi$ $\pi$-production probability  & $\pm$20\%  & \tabularnewline
\hline
\end{tabular}\\[2pt] \caption[Intra-nuclear hadron transport systematic parameters implemented in GENIE]
{The intra-nuclear hadron transport systematic parameters implemented in \dword{genie} with associated uncertainties considered in
this work. Note that the 'mean free path' parameters are omitted for both N-N and $\pi$-N interactions as they produced unphysical variations in observable analysis variables. Table adapted from Ref~\cite{Andreopoulos:2015wxa}.
% https://arxiv.org/pdf/1510.05494.pdf
}


\label{table:HadTranspKnobs}
\end{table}

Table~\ref{tab:nuintsystlist} separates the interaction model parameters into three categories based on their treatment in the analysis:
\begin{itemize}
\item Category 1:  On-axis near detector data is expected to constrain these parameters; the uncertainty is implemented in the same way in near and far detectors.  All \dword{genie} uncertainties (original or modified) are all treated as Category 1.
\item Category 2: These uncertainties are implemented in the same way in near and far detectors, but on-axis data alone is not sufficient to constrain these parameters. We use two sub-categories. The first category (2A) corresponds to interaction effects which may be difficult to disentangle from detector effects. A good example of this is the $E_b$ parameter, which may be degenerate with the energy scale of the near detector.  This may be constrained with electron scattering and dedicated studies carefully selected samples of near detector data, but would be difficult to constrain with inclusive near detector samples. The second category (2B) corresponds to parameters that can be constrained by off-axis samples, described in Section~\ref{sec:nu-osc-06}.
\item Category 3: These uncertainties are implemented only in the far detector.  Examples are $\nu_e$ and $\overline{\nu}_e$ rates which are small and difficult to precisely isolated from background at the near detector.  Therefore, near detector data is not expected to constrain such parameters.
\end{itemize}


\begin{table}[ptb]
\scalebox{0.9}{
\begin{tabular}{lccc}\hline
Uncertainty & Mode & Description & Category  \\  \hline \hline
BeRPA & 1p1h/QE & RPA/nuclear model suppression &  1  \\  \hline
% $E_b$ & 1p1h/QE & Shift in nuclear model removal energy  &  2A   \\  \hline %2
MnvTune1 & 2p2h & Strength into (nn)pp only &  1 \\  \hline
% MnvaTune2 & 2p2h  & Strength into np pairs only &  1 \\  \hline
MnvTuneCV & 2p2h & Strength into 2p2h  &  1 \\  \hline
MnvTune2 & 1p1h/QE & Strength into 1p1h  &  1 \\  \hline
ArC2p2h & 2p2h Ar/C scaling & Electron scattering SRC pairs & 1 \\ \hline
$E_{2p2h}$ & 2p2h & 2p2h Energy dependence  &  2B \\  \hline
Low $Q^2$ $1\pi$ & RES & Low $Q^2$ (empirical) suppression &   1 \\  \hline
MK model & $\nu_\mu$ CC-RES & alternative strength in W &   1 \\  \hline
CC Non-resonant $\nu\rightarrow\ell+1\pi$ & $\nu$ DIS & Norm. for $\nu+n/p\rightarrow\ell+1\pi$ (\it{c.f.}\cite{Rodrigues:2016xjj}) & 1 \\ \hline
Other Non-resonant $\pi$ & $N\pi$ DIS & Per-topology norm. for $1<W<5$ GeV. & 1 \\ \hline
$E_{avail}/q_0$ & all & Extreme \dword{fsi}-like variations  &  2B \\
\hline
Modified proton energy & all & 20\% change to proton E &  2B \\
\hline
$\nu_\mu\to\nu_e$ & $\nu_e$/$\overline{\nu}_e$ & 100\% uncertainty in $\nu_e$ unique phase space &  3 \\ \hline
$\nu_e$/$\overline{\nu}_e$ norm & $\nu_e$,$\overline{\nu}_e$ & Ref.~\cite{Day-McFarland:2012}  &  3 \\
\hline
\hline
\end{tabular}
} % scalebox
\caption[List of extra interaction model uncertainties in addition to those provided by GENIE]{List of extra interaction model uncertainties in addition to those provided by GENIE.}
\label{tab:nuintsystlist}
\end{table}

Finally, there are a number of tunes applied to the default model, to represent known deficiencies in \dword{genie}'s description of neutrino data, and these are listed in Table~\ref{table:NuXSecKnobs_Central}.

% Central values tunes
% BeRPA, MACCRES, Nonresonant background 1pi, MINERvA 2p2h tune
\begin{table}[ptb]
\center
\global\long\def\arraystretch{1.75}
\scalebox{0.9}{
\begin{tabular}{ll|l}
\hline
$x_{P}$  & Description of $P$  & $P_\textsc{cv}$  \tabularnewline
\hline
&\textbf{Quasielastic} & \tabularnewline
BeRPA & Random Phase Approximation tune & $A: 0.59$ \tabularnewline
& $A$ controls low $Q^2$, $B$ controls low-mid $Q^2$ & $B: 1.05$ \tabularnewline
& $D$ controls mid $Q^2$, $E$ controls high $Q^2$ fall-off & $D: 1.13$ \tabularnewline
& $U$ controls transition from polynomial to exponential & $E: 0.88$ \tabularnewline
& & $U: 1.20$ \tabularnewline
&\textbf{2p2h}&\tabularnewline
MINERvA 2p2h tune & $q0,q3$ dependent correction to 2p2h events&\\
&\textbf{Low $\mathbf{W}$ single pion production} & \tabularnewline
$x_{M_{A}}^{CCRES}$ & Axial mass for CC resonance in \dword{genie}& $0.94$ \tabularnewline
Non-res CC1$\pi$ norm. & Normalization of CC1$\pi$ non-resonant interaction & $0.43$  \tabularnewline
\hline
\end{tabular}
}% scalebox
\\[2pt]
\caption[Neutrino interaction cross-section systematic parameters that receive a central-value tune]{Neutrino interaction cross-section systematic parameters that receive a central-value tune}
\label{table:NuXSecKnobs_Central}
\end{table}

