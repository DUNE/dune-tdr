\section{Neutrino Interactions and Uncertainties}\label{sec:nu-osc-05} \label{sec:physics-lbnosc-nuint}
%{\it Assigned to:} {\bf Kevin McFarland} with contributions from Kendall and the DuneReweight group. This section has a description of event generator used, justify choices of model and a description of Dune Reweight and treatment of the uncertainties.  This will be organized by reaction type or effect, preferably with overarching categories as subsections. 

\subsection{Interaction Model Summary}
  
The model proposed herein generally factorizes the neutrino
interaction on nuclei into an incoherent sum of ``hard scattering'' neutrino interactions with the single nucleons in the nucleus. The effect of the nucleus is implemented as
initial and final state interaction effects, with some (albeit few) nucleus-dependent hard scattering calculations. Schematically, we express this concept as $\text{Scattering Process} = \text{Initial State} \otimes \text{Nucleon Interaction} \otimes \text{Final State Propagation}$.


By ``initial state'' effects, we usually mean a description of the momentum and position distributions of the nucleons in the nucleus, kinematic modifications to the final state (such as removal energy, sometimes described as a ``binding energy''), and Coulomb effects.   The concept of ``binding energy'' indicates that the struck nucleon may be off the mass-shell inside the nucleus.  
``Final state interactions'' (FSI) refer to the propagation and interaction of hadrons produced in the nucleon interaction through the nucleus. The FSI alter both the momentum and energy of the recoiling particles produced in the final state, and may also alter their identity and multiplicity in the case of inelastic reinteractions (e.g. in a nucleus a hadron may be absorbed, rescattered, or create a secondary hadron).  The FSI model is implemented in the GENIE, NuWro, and NEUT neutrino interaction generators as a semi-classical cascade model (or in the case of GENIE's $hA$ model a single step scaled model), based on hadron-nucleus and hadron-nucleon scattering data and theoretical corrections.  GiBUU, another available generator, uses a full particle transport model, including support for off-shell hadrons in the nuclear medium.

Generators vary in their attempts to accurately model the largely undetected final state ``spectator'' nuclear system.  The nuclear system can carry away significant undetected momentum ---hundreds of MeV is not unusual--- in the form of one or more heavy, non-relativistic particles.  These particles typically carry off very little kinetic energy; however they can absorb of order tens of MeV of energy from the initial state from breakup or excitation of the target nucleus.  Most of this energy and momentum will typically be invisible to the detector. 
\fixme{Add citations for spectator nucleus if possible}

The factorization outlined above is not present in all parts of the model.  Most modern generators include ``2p2h'' (two particle, two hole) interactions which model meson exchange processes and scattering on highly correlated pairs of nucleons in the nucleus.  These processes are often implemented as another process which incorporates both hard scattering and initial state effects in processes that create multiple final state nucleons, with a different prescription for different nuclei.
Neutrino scattering on atomic electrons and the coherent production of pions (which scatters off the entire nucleus) also
do not follow this factorization. 

The interaction model and its variations are implemented in the  GENIE generator.  The fixed version of GENIE used for this report, v2.12.{\bf\textcolor{red}{??}}\footnote{At the time of the development of this report, GENIE 3 had just recently been released (October 15, 2018).  However, at the time of that release, there was no documentation of the tunes, and the code required to implement uncertainties had also not been released; therefore, it was impossible to use GENIE 3 for this work.}\todo{Update GENIE version}, will not contain all of the possible cross section variations which need to be modeled.  Therefore, the variations in the cross sections to be considered are implemented as some combination of: GENIE weighting parameters (sometimes referred to as "GENIE knobs"), {\it ad hoc} weights of events which are designed to parameterize uncertainties or cross section corrections currently not implemented within GENIE, and as discrete alternative model comparisons, achieved through alternate generators, alternate GENIE configurations, or custom weightings. 

Discrete alternative model comparisons are valuable, but are used sparingly in this work due to finite computational resources and ease of interpretation. The approach taken here was to identify classes of uncertainties and provide generous uncertainties intended to span a reasonable range of alternate models. Using the alternate model as mock data either provides a closure test, or a mechanism by which to inflate uncertainties. Most of the alternate models in alternate generators, GiBUU, NEUT and NuWro,  or alternate configurations of GENIE are expected to be covered by the (weighted) uncertainties described here. 

It is not guaranteed that the alternate models will be covered by the chosen systematic uncertainty, but analysis  of alternate models can be computationally slow and interpretation not easy; there may be artificial reasons (e.g. a given generator's implementation choice ) which can create unexpected effects. 

For this work, we chose one particular discrete alternate model as a source of uncertainty, inspired by differences in models but which has a well defined implementation and interpretation; it is described in Section~\ref{sec:fsi}. The most problematic feature of an interaction model for any oscillation analysis is if the model uncertainties are incomplete in the association between reconstructed observables and true neutrino energy. A special mock data sample was created which preserved the same observables for on-axis samples as the default sample, but which significantly modifies the observable-energy relationship.
%This isolates one of the fundamental issues, which is that all tests of the relationship to true neutrino energy via observables have so far been largely indirect and averaged over a broad 

%\fixme{ KM outline:}
%- Future work on DUNE will develop better machinery for alternate generators; this is a computational and technical challenge faced by many experiments; 
%-- LBNC may not accept "computing limitations" so trying to finesse this
%- Alternate generators provide a closure test of the weighting scheme and/or a new uncertainty
%-- NEUT/NuWro variations expected to be covered by current parameterization, but generally, 
%-- the uncertainties generally span the availible GENIE alternate configurations. (non-reweightable knobs that are  implemented, e.g., formation zone and other hadronization uncertainties-- not quite)
%--But, there can be artificial reasons why a particular generator-generator comparison may not reflect error (e.g. bug in implmentatin). We can't generally know how to interpret the results, so they are not considered here.

%- One single important case-- and this is the most worrysome case which the alternate generators embody:  This mock data sets modify the reconstructed to true neutrino energy relationship, while preserving the outgoing final state particle kinematics. This isolates one of the fundamental issues, which is that all tests of the relationship to true neutrino energy via observables have so far been largely indirect and averaged over a broad flux.
\subsection{Interaction Model Uncertainties}

The interaction uncertainties are divided into seven roughly exclusive groups: (1) initial state uncertainties, (2) hard scattering uncertainties and nuclear modifications to the quasielastic process, (3) uncertainties in multinucleon (2p2h) hard scattering processes, (4) hard scattering uncertainties in pion production processes, (5) uncertainties governing other, higher $W$ and neutral current processes, (6) final state interaction uncertainties, (7) neutrino flavor dependent uncertainties.  We discuss each in turn.  Uncertainties are intended to reflect current theoretical freedom, deficiencies in implementation, and/or current experimental knowledge. Where the parameterization is limited or simplified is also discussed. 

\subsubsection{Initial State Uncertainties}
The default nuclear model in GENIE is a modified global Fermi gas model of the nucleons in the nucleus.  There are significant deficiencies that are known in global Fermi gas models; these include a lack of consistent incorporation of the tails that result from correlations among nuclei, the lack of correlation between location within the nucleus and momentum of the nucleon, an incorrect relationship between momentum and energy of the off-shell, bound nucleon within the nucleus. GENIE does empirically modify the nucleon momentum distribution to account for short-range correlation effects, which populates tails above the Fermi cutoff, but the remainder of these deficiencies persist.   Alternate initial state models, such as spectral functions~\cite{Benhar:1994hw,Nieves:2004wx}, the mean field model of GiBUU~\cite{Gallmeister:2016dnq}, or CRPA calculations~\cite{Pandey:2014tza} may provide better descriptions of the nuclear initial state~\cite{Sobczyk:2017mts}.  It is, however, generally impossible to weight predictions from one initial state model to another in order to compare predictions, and such studies would have to be done with alternate Monte Carlo simulations.

An important Fermi gas parameter which enters directly into neutrino energy reconstruction is the binding or removal energy, $E_B$.  This parameter directly affects the transfer of energy to the nucleus.  It cannot be implemented by event weights within the GENIE Fermi gas model, and is implemented by a "lateral" shift of lepton energies as a function of neutrino energy and lepton energy, following T2K's implementation. For GENIE, the recommendation from Ref.~\cite{bodek_eb} is to set $E_b$ to $17.8\pm 3$~MeV in neutrino interactions and $21.8\pm 3$~MeV in antineutrino interactions.  Note that these values significant differ from the GENIE default.


\subsubsection{Quasielastic uncertainties}
The primary uncertainties in quasielastic interactions that are considered are the axial form factor of the nucleon and nuclear screening (from the so-called RPA calculations) of low momentum transfer reactions.

%MAQE Blurb
%The axial form factor uncertainty has been historically described with a single parameter uncertainty with the dipole form by varying $M_A$, and we will continue this for current DUNE TDR studies.  However, it is now understood that this framework overconstrains the form factor at high $Q^2$, and an alternate parameterization based on the $z$-expansion has been proposed as a replacement~\cite{Meyer:2016oeg}.  This parameterization is available in GENIE and will be used for future studies.
%Z-exp blurg
The axial form factor uncertainty has been historically described with a single parameter uncertainty with the dipole form by varying $M_A$.  However, it is now understood that this framework overconstrains the form factor at high $Q^2$, and an alternate parameterization based on the $z$-expansion has been proposed as a replacement~\cite{Meyer:2016oeg}.  The $z$-expansion parameterization, available in GENIE, will be used for these studies.

One part of the Nieves et al.\cite{nieves1,nieves2} description of the $0\pi$ interaction on nuclei includes the Random Phase Approximation, used to sum the $W^\pm$ self-energy terms. In practice, this modifies the 1p1h/QE cross-section in a non-trivial way. The calculations from Nieves et al. have associated uncertainties presented in \cite{nieves_uncert}, which were evaluated as a function of $Q^2$ by Federico Sanchez. In 2018, MINERvA and NOvA parameterized the central value and uncertainty in $(q_0, q_3)$ using Rik Gran's RPA uncertainties\cite{RikRPA}, whereas T2K in 2016 corrected in $(Q^2, E_\nu)$ without uncertainty, and in 2017/8 used central values and uncertainties in $Q^2$ only.
For this model we use T2K's 2017/8 parameterization of the RPA effect\cite{t2k_2018}. The shape of the correction and error is parameterized with a Bernstein polynomial up to $Q^2=1.2\text{ GeV}^2$ and then switches to a decaying exponential. The BeRPA (Bernstein RPA) function has three parameters controlling the polynomial ($A, B, C$), where the parameters control the behaviour at increasing $Q^2$. A fourth parameter $E$ controls the high $Q^2$ tail. 

\subsubsection{$\boldsymbol{2p2h}$ uncertainties}
For these studies, we chose to start with the Nieves et al.\ or "Valencia" model~\cite{nieves1,nieves2} for multinucleon ($2p2h$) contributions to to the cross section.  However, MINERvA has shown directly~\cite{Rodrigues:2015hik}, and NOvA indirectly, that this description is missing observed strength on carbon.  As a primary approach to the model, we add that missing strength to a number of possible reactions.  We then add additional uncertainties for energy dependence of this missing strength and uncertainties in scaling the $2p2h$ prediction from carbon to argon.

The extra strength from the "MINERvA tune" to $2p2h$ is applied in $(q_0,q_3)$ space (where $q_0$ is energy transfer from the leptonic system, and $q_3$ is the magnitude of the three momentum transfer) to fit reconstructed \minerva CC-inclusive data~\cite{Rodrigues:2015hik} in $E_{avail}$\footnote{$E_{avail}$ is calorimetrically visible energy in the detector, roughly speaking total recoil hadronic energy, less the masses of $\pi^\pm$ and the kinetic energies of neutrons} and $q_3$.  Reasonable fits to MINERvA's data are found by attributing the missing strength to any of $2p2h$ from np initial state pairs, $2p2h$ from nn initial state pairs, or $1p1h$ or quasielastic processes.  The default tune uses an enhancement of the np and nn initial strengths in the ratio predicted by the Nieves model, and alternate systematic variation tunes ("MnvTune" 1-3) attribute the missing strength to the individual hypotheses above.
Implementation of the MINERvA tune is based on weighting in true $(q_0,q_3)$. Because the implementation is based on weighting, it requires GENIE's 1p1h and GENIE's Valencia model $2p2h$ prediction.  
%The added cross sections, averaged over MINERvA's "low energy" beam, in regions of $(q_0,q_3)$ are shown in Figures~\ref{fig:mnvTunes} and \ref{fig:mnvTunesNubar}. 


%\begin{dunefigure}[Added differential $d^2\sigma/dq_0dq_3$ cross sections for $\nu$s averaged over the MINERvA flux]{fig:mnvTunes}{The added differential $d^2\sigma/dq_0dq_3$ cross sections for neutrinos averaged over the MINERvA flux in units of $10^{-38}/cm^2/GeV^2$ for the default \minerva tune and the three systematic variations.  Each is in good agreement with the \minerva "low recoil" data in \minerva's neutrino~\cite{Rodrigues:2015hik} beam. \todo{Fix titles/label/update MnvTunes plots 1/2}}
%\includegraphics[width=\textwidth]{graphics/Fudges.png}
%\end{dunefigure}

%\begin{dunefigure}[Added differential $d^2\sigma/dq_0dq_3$ crosssections for $\anu$s averaged over the MINERvA flux]{fig:mnvTunesNubar}
%{The added differential $d^2\sigma/dq_0dq_3$ cross sections averaged over the MINERvA flux for antineutrinos in units of $10^{-38}/cm^2/GeV^2$ for the default \minerva tune and the three systematic variations.  Each is in good agreement with the \minerva "low recoil" data in \minerva's antineutrino~\cite{Gran:2018fxa} beam. \todo{Fix titles/label/update MnvTunes plots 2/2}}
%\includegraphics[width=\textwidth]{graphics/antiFudges.png}
%\end{dunefigure}

 The rates for 1p1h and 2p2h processes could be different on argon and carbon targets.  There is little neutrino scattering data to inform this, but there are measurements of short-ranged correlated pairs from electron scattering on different nuclei~\cite{Colle:2015ena}.  These measurements directly constrain $2p2h$ from short range correlations, although the link to dynamical sources like meson exchange current processes (MEC) is less direct.  Interpolation of that data in $A$ suggests that scaling from carbon relative to the naive $\propto A$ prediction for 2p2h processes would give an additional factor of $1.33\pm 0.13$ for $np$ pairs, and $0.9\pm 0.4$ for $pp$ pairs.
GENIE's prediction for the ratio of 2p2h cross-sections in $\text{Ar}^{40}/\text{C}^{12}$ for neutrinos varies slowly with neutrino energy in the DUNE energy range: from $3.76$ at $1$~GeV to $3.64$ at $5$~GeV. The ratio for antineutrino cross sections is consistent with $3.20$ at all DUNE energies. Since the ratio of $A$ for $\text{Ar}^{40}/\text{C}^{12}$ is $3.33$, this is consistent with the ranges suggested above by the measured $pp$ and $np$ pair scaling.  A dedicated study by the SuSA group using their own theoretical model for the relevant MEC process also concludes that the transverse nuclear response (which drives the $\nu-A$ MEC cross section) ratio between $\text{Ca}^{40}$ (the isoscalar nucleus with the same $A$ as $\text{Ar}^{40}$) and $\text{C}^{12}$ is $3.72$ \cite{Amaro:2017eah}.   We conclude that we should vary GENIE's Valencia model based prediction, including the MINERvA tune, for $2p2h$ by $\sim 20\%$ to be consistent with the correlated pair scaling values above independently for neutrino and antineutrino scattering. 

 The MINERvA tune may be $E_\nu$ dependent. MINERvA separated its data into an $E_\nu<6$~GeV and an $E_\nu>$6~GeV piece, and sees no dependence with a precision of better than $10\%$~\cite{Rodrigues:2015hik}.  The mean energy of the $E_\nu<6$~GeV piece is roughly $\left< E_\nu\right>\approx 3$~GeV.  In general, an exclusive cross-section will have an energy dependence $\propto \frac{A}{E_\nu^2}+\frac{B}{E_\nu}+C$~\cite{llewelyn-smith}; therefore, unknown energy dependence may be parametrized by an {\em ad hoc} factor of the form $1/\left( \frac{A^{'}}{E_\nu^2 }+\frac{B^{'}}{E_\nu}\right)$.  The MINERvA constraints suggest $A^{'}<0.9$~GeV$^2$ and $B^{'}<0.3$~GeV.  The variations for neutrinos and antineutrinos could be different since this is an effective modification.

\subsubsection{Single pion production uncertainties}
GENIE uses the Rein-Sehgal model for pion production, \fixme{including isotropic Delta++ decay - correct statement?} Tunes to $D_2$ data have been performed, both by the GENIE collaboration itself and in subsequent re-evaluations~\cite{Rodrigues:2016xjj}; we use the latter tune as our base model.

\minerva single pion production data~\cite{Altinok:2017xua,McGivern:2016bwh,Eberly:2014mra} indicates disagreement at low $Q^2$ which may correspond to an incomplete nuclear model for single pion production in the generators. A similar effect was observed at MINOS\cite{minos_pi_q2}. and \nova implements a correction for similar indications in its data~\cite{nova_2018}.
A fit to MINERvA data~\cite{StowellThesis} measured a suppression parameterized by
\begin{align}
R(Q^{2}<x_m) & = 
\frac{R_{2} (x-x_1)(x-x_m)}{(x_2-x_1)(x_2-x_m)} \notag\\
&+ \frac{(x-x_1)(x-x_2)}{(x_m-x_1)(x_m-x_2)},  \\
W(Q^{2}) &= 1-(1-R_{1})(1-R(Q^{2}))^{2}
\end{align}
\noindent where $x_m$ is the chosen maximum cut off above which no events are suppressed and $(x_1,R_1)$ and $(x_2,R_2)$ are the chosen points defining the shape of the suppression function.
The points $x_1$ and $x_2$ were set to $0.0~\text{GeV}^2$ and $0.35~\text{GeV}^2$ respectively, and fitted values of $R_1\approx0.3$ and $R_2\approx0.6$. The correction is applied to events with a resonance decay inside the nucleus giving rise to a pion, based on GENIE event information. 

An improved Rein-Seghal-like resonance model has recently been developed\cite{minoo} which includes a non-resonant background in both $I_{1/2}$ and $I_{3/2}$ channels and interference between resonant-resonant and resonant-non-resonant states. %Additionally, it uses the full resonance continuum to calculate the pion-nucleon ejection kinematics, which the Rein-Sehgal model only uses $\Delta(1232)$ for. 
The largest differences with GENIE's Rein-Sehgal implementation are in the outgoing pion and nucleon kinematics and the hadronic mass ($W$) spectrum. 
%The anti-neutrino cross-section is between 200-400\% larger than for the Rein-Sehgal model, in contention with data from MINERvA\cite{McGivern:2016bwh}.
%Implementation: Template weighting in $(W, Q^2, E_\nu)$ or $(q_0, q_3, E_\nu)$. Uncertainty: Excursion set to cover difference between the MK model and current Rein-Sehgal model.
We implement a template weighting in $(W, Q^2, E_\nu)$ space to cover the differences between the two models as a systematic uncertainty.

Coherent inelastic pion production measurements on carbon are in reasonable agreement with the GENIE implementation of the Berger-Sehgal model~\cite{Mislivec:2017qfz}.  The process has not been measured at high statistics in argon. While coherent interactions provide a very interesting sample for oscillation analyses, they are also a very small component of the event rate and selections will depend on the near detector configuration. Therefore, for the studies here, we do not provide any evaluation of a systematic uncertainty for this extrapolation or any disagreements between the Berger-Sehgal model and carbon data.  

\subsubsection{Other hard scattering uncertainties}

\nova oscillation analyses\cite{nova_2018} have found the need for excursions beyond the default GENIE uncertainties in order to describe their transition region data.  We implement separate 50\% uncertainties for $\geq 3$ pion final states for neutral current and charged current processes.
%Luke: What do we do about other hard scattering events? Either let B-Y do it all (do we ignore B-Y uncertainties on events with ? 2 pions or let them compound?) or, retain something like the old fully correlated DIS norm uncertainty for all DIS events > W_cutoff that have Npi < 2. Either way, I think a sentence here would be good.

\subsubsection{Final state interaction uncertainties}\label{sec:fsi}
GENIE includes a large number of final state uncertainties to its $hA$ final state cascade model which are summarized in Table~\ref{table:HadTranspKnobs}.  These uncertainties have been validated in neutrino interactions primarily on light targets such as carbon, but there is very little data available on argon targets.  
This concern about the lack of validation on Argon targets is difficult to address directly because there are many possible FSI processes that could be varied, but it would nevertheless be a mistake not to put some degree of freedom in the model to address this.

One way to incorporate the effect of FSI uncertainties is to parameterize them in terms of the primary experimental observable effect. The DUNE detectors seek to keep all observed final states as signal samples, so it is natural to consider the effect of a variation on the energy reconstruction.  The DUNE detector will measure a recoil quantity which will need to be mapped to the true energy loss from the leptonic system, or $q_0$.  In the absence of sophisticated particle flow calorimetry\footnote{Such a sophisticated particle flow reconstruction will eventually follow but is not yet available at this stage of the experimental development} a good approximation to the measured recoil energy is the $E_{avail}$ described above, which is the recoil energy, less the rest mass of charged pions and the kinetic energy of any produced neutrons.  FSI will alter the ratio $E_{avail}/q_0$ by changing the fraction of recoil energy in neutrons and charged pions through event weighting while conserving the total cross section.
The variation will look at how FSI alters $E_{avail}/q_0$ in carbon, and apply an arbitrary 50\% fraction of this shift to the default prediction on argon.  Event by event weights as a function of $(q_3,q_0,E_{avail}/q_0)$ will be calculated to conserve the total cross section in each $(q_3,q_0)$ bin while shifting the first and second moments.

The fake data set considered here is generated by scaling down the energy deposits due to protons by 20\%, under the assumption that this missing energy is instead carried by some other, unobserved particles, such as neutrons:
\begin{equation}
E_{\rm proton}^{dep} \rightarrow E_{\rm proton}^{\prime dep} = 0.8 \times E_{\rm proton}^{dep} \, .
\end{equation}

This propagates to the reconstructed neutrino energy as:
\begin{equation}
E_{\rm rec} \rightarrow E_{\rm rec}^{\prime} = E_{\rm rec} - 0.2 \times E_{\rm proton}^{dep} \, .
\end{equation}

The modified proton energy systematic uncertainty is treated as a discrete alternate model, to test the entire chain of oscillation analysis robustness to alternate assumed energy dependence in the interaction model. The mock data is prepared by the following prescription. A given events proton energy is reduced by scaling down the energy deposits by 20\%; this assumes that the missing energy is instead carried by some other, unobserved particles, such as neutrons. Finally, a weighting function is applied to recover agreement with the nominal on-axis sample detector observables.


\subsubsection{Neutrino flavor dependent uncertainties}
The cross sections include terms proportional to lepton mass terms, which are significant contributors only at low energies where quasielastic processes dominate.  Some of the form factors that enter in these terms have significant uncertainties in the nuclear environment.  Ref.~\cite{Day-McFarland:2012} ascribes the largest possible effect to the presence of poorly constrained second-class current vector form factors in the nuclear environment, and proposes a variation in the cross section ratio of $\sigma_\mu/\sigma_e$ of $\pm 0.01/{\rm\textstyle Max}(0.2~{\rm\textstyle GeV},E_\nu)$ for neutrinos and $\mp 0.018/{\rm\textstyle Max}(0.2~ {\rm\textstyle GeV},E_\nu)$ for anti-neutrinos.  Note that anticorrelation of the effect in neutrinos and antineutrinos.    

In addition, radiative Coulomb effects may also contribute, which for T2K is of order $\pm5$ MeV shifts in reconstructed lepton momentum.  This shift is not implemented.  Like the second class current effect in the cross section, it flips sign between neutrinos and antineutrinos and is significant only at low energies.  This effect is not currently implemented.

Finally, some electron neutrino interactions occur in a space in momentum and energy transfer $(q_0,q_3)$ where there are no muon neutrino interactions and therefore the nuclear response has not been constrained by muon neutrino measurements.  This region at lower neutrino energies has a significant overlap with the Bodek-Ritchie tail of the Fermi gas model.  There are significant uncertainties in this region, both from the mail of the tail itself, and from the lack of knowledge about the effect of RPA and 2p2h in this region.  The allowed phase space in the presence of non-zero lepton mass is $E_\nu-\sqrt{\left( E_\nu-q_0\right) ^2-m_l^2}\leq q_3\leq E_\nu+\sqrt{\left( E_\nu-q_0\right) ^2-m_l^2}$.  A 100\% variation allowed in the phase space present for $\nu_e$ but absent for $\nu_\mu$. \\ 

\subsubsection{Summary of interaction model uncertainties}

The complete set of interaction model uncertainties includes suggested GENIE default uncertainties (Tables~\ref{table:NuXSecKnobs}, %\ref{table:HadronzKnobs},
\ref{table:HadTranspKnobs}),  modified uncertainties (Table~\ref{}), or new uncertainties developed for this effort (Table~\ref{tab:nuintsystlist}). 

\begin{table}[htb]
\center
\global\long\def\arraystretch{1.75}
\scalebox{0.9}{
\begin{tabular}{llr|l}
\hline 
$x_{P}$  & Description of $P$  & $P_\textsc{cv}$ & $\delta{P}/{P}$  \tabularnewline
\hline 
&\textbf{Quasielastic}&&\tabularnewline
$x_{ZExpA[1\textrm{--}4]}^{CCQE}$ & Four free $z$-expansion parameters & Ref.\ref{Meyer:2016oeg} & \tabularnewline
$x_{VecFF}^{CCQE}$  & Choice of CCQE vector form factors (BBA05 $\leftrightarrow$ Dipole)  &  & N/A \tabularnewline
$x_{kF}^{CCQE}$  & Fermi surface momentum for Pauli blocking &   & $\pm$30\% \tabularnewline
&\textbf{Low $\mathbf{W}$}&&\tabularnewline
$x_{Norm.}^{CCRES}$ & Normalization factor for CC resonance &  115 & $\pm$7\% \tabularnewline
$x_{M_{A}^{Shape}}^{CCRES}$ & Axial mass for CC resonance (shape effect only) & 0.94 & $\pm$0.05~GeV \tabularnewline
%$x_{R_{bkg}^{\nu}}^{CC1\pi}$  & Non-resonant $\nu$ $CC1\pi$ normalization (Called DIS Norm. in \ref{Rodrigues:2016xjj}) & 43\% & $\pm$4\% \tabularnewline
$x_{M_{V}^{Shape}}^{CCRES}$  & Vector mass for CC resonance &  & $\pm$10\% \tabularnewline
$x_{\eta\ BR}^{\Delta Decay}$  & Branching ratio for $\Delta\rightarrow\eta$ decay &   & $\pm$50\% \tabularnewline
$x_{\gamma\ BR}^{\Delta Decay}$  & Branching ratio for $\Delta\rightarrow\gamma$ decay &   & $\pm$50\% \tabularnewline
$x_{\theta_{\pi}^{\Delta Decay}}$  & $\theta_{\pi}$ distribution in decaying $\Delta$ rest frame (isotropic $\rightarrow$ RS) &   & N/A \tabularnewline
&\textbf{High $\mathbf{W}$}&\tabularnewline
$x_{A_{HT}^{BY}}^{DIS}$  & $A_{HT}$ higher-twist param in BY model scaling variable $\xi_{w}$  &   & $\pm$25\% \tabularnewline
$x_{B_{HT}^{BY}}^{DIS}$  & $B_{HT}$ higher-twist param in BY model scaling variable $\xi_{w}$  &  & $\pm$25\% \tabularnewline
$x_{C_{V1u}^{BY}}^{DIS}$  & $C_{V1u}$ valence GRV98 PDF correction param in BY model  &  & $\pm$30\% \tabularnewline
$x_{C_{V2u}^{BY}}^{DIS}$  & $C_{V2u}$ valence GRV98 PDF correction param in BY model  &  & $\pm$40\% \tabularnewline
$x_{x_{F}^{AGKY1\pi}}^{DIS}$  & Vary the $x_{F}$ distribution for 1$\pi$ DIS produced by AGKY &  & $\pm$20\% \tabularnewline
$x_{p_{T}^{AGKY1\pi}}^{DIS}$  & Vary the $p_{T}$ distribution for 1$\pi$ DIS produced by AGKY &  & $\pm$3\% \tabularnewline
&\textbf{Other neutral current}&&\tabularnewline
$x_{M_{A}}^{NCEL}$  & Axial mass for NC elastic  &  & $\pm$25\% \tabularnewline
$x_{\eta}^{NCEL}$  & Strange axial form factor $\eta$ for NC elastic  &  & $\pm$30\%  \tabularnewline
$x_{Norm.}^{NCRES}$ & Normalization factor for NC resonance neutrino production &  & $\pm$20\% \tabularnewline
$x_{M_{A}^{Shape}}^{NCRES}$  & Axial mass for NC resonance (shape effect only) & &$\pm$10\% \tabularnewline
$x_{M_{V}^{Shape}}^{NCRES}$  & Vector mass for NC resonance (shape effect only) & &$\pm$5\% \tabularnewline
&\textbf{Misc.}&&\tabularnewline
$x_{FZ}$  & Vary effective formation zone length &  & $\pm$50\% \tabularnewline
\hline 
\end{tabular}
}% scalebox
\\[2pt] 
\caption{Neutrino interaction cross-section systematic parameters considered
in GENIE. 
%\fixme{For some of the above parameters there are two weighting implementations: One which includes the full effect of the systematic (shape + normalization) and one which includes only its effect on the shape of observable distributions (maintains normalization). -- Can Clarence/Dan resolve this and remove any extra parameters we don't need from the table?}
Table adapted from Ref~\cite{}% https://arxiv.org/pdf/1510.05494.pdf
%Note that some systematics have overlapping effects so care is needed to avoid double counting.
}
\label{table:NuXSecKnobs} 
\end{table}


\begin{table}[htb]
\center

\global\long\def\arraystretch{1.75}
% enlarge line spacing
\begin{tabular}{llll}
\hline 
$x_{P}$  & Description of $P$  & $\delta{P}/{P}$  & \tabularnewline
\hline 
$x_{mfp}^{N}$  & Nucleon mean free path (total rescattering probability)  & $\pm$20\%  & \tabularnewline
$x_{cex}^{N}$  & Nucleon charge exchange probability  & $\pm$50\%  & \tabularnewline
$x_{el}^{N}$  & Nucleon elastic reaction probability  & $\pm$30\%  & \tabularnewline
$x_{inel}^{N}$  & Nucleon inelastic reaction probability  & $\pm$40\%  & \tabularnewline
$x_{abs}^{N}$  & Nucleon absorption probability  & $\pm$20\%  & \tabularnewline
$x_{\pi}^{N}$  & Nucleon $\pi$-production probability  & $\pm$20\%  & \tabularnewline
$x_{mfp}^{\pi}$  & $\pi$ mean free path (total rescattering probability)  & $\pm$20\%  & \tabularnewline
$x_{cex}^{\pi}$  & $\pi$ charge exchange probability  & $\pm$50\%  & \tabularnewline
$x_{el}^{\pi}$  & $\pi$ elastic reaction probability  & $\pm$10\%  & \tabularnewline
$x_{inel}^{\pi}$  & $\pi$ inelastic reaction probability  & $\pm$40\%  & \tabularnewline
$x_{abs}^{\pi}$  & $\pi$ absorption probability  & $\pm$20\%  & \tabularnewline
$x_{\pi}^{\pi}$  & $\pi$ $\pi$-production probability  & $\pm$20\%  & \tabularnewline
\hline 
\end{tabular}\\[2pt] \caption{Neutrino-induced hadronization and resonance-decay systematic parameters and intranuclear hadron transport systematic parameters implemented in GENIE with associated uncertainties considered in
this work. Table adapted from Ref~\cite{}% https://arxiv.org/pdf/1510.05494.pdf 
}


\label{table:HadTranspKnobs} 
\end{table}

Table~\ref{tab:nuintsystlist} separates the interaction model parameters into three categories based on their treatment in the analysis:
\begin{itemize}
\item Category 1:  On-axis near detector data is expected to constrain these parameters; the uncertainty is implemented in the same way in near and far detectors.
\item Category 2: These uncertainties are implemented in the same way in near and far detectors, but on-axis data alone is not sufficient to constrain these parameters. We use two sub-categories. The first category (2A) corresponds to interaction effects which may be difficult to disentangle from detector effects. The best example of this is the $E_b$ parameter, which may be degenerate with the energy scale of the near detector; this may be constrained with electron scattering and/or thoughtful alternate projections of near detector data (Section~\ref{sec:nu-osc-06}). The second category (2B) corresponds to parameters where off-axis samples, described in Section~\ref{sec:nu-osc-06}, will likely constrain these effects significantly.
%An observed constraint from the near detector on-axis position alone should be viewed with suspicion and be scrutinized for why it provides constraint. Previous studies of conventional near detectors indicate that constraint of (true energy) systematic uncertainties is difficult or artificial. 
%It is expected additional developments in the ND group (e.g. DUNE PRISM~\cite{docdbDUNEPRISM}\fixme{Need to define off-axis angles here/make more general as term not introduced yet.}, scintillator detector neutron information) may provide a reasonable mechanism to constrain these parameters. If we find no reasonable mechanism or artificial constraint, we would iterate with ND groups to adjust the treatment. As an example, a workaround would be to force these parameters to be uncorrelated between near and far detector.
\item Category 3: Uncertainty should be implemented only in the far detector (e.g. as $\nu_e$ and $\overline{\nu}_e$ rates are  small at the near detector).  Near detector data is not expected to constrain such parameters.
\end{itemize}
All GENIE uncertainties (original or modified) are all treated as Category 1.

%\afterpage{
%\begin{landscape}
%\mbox{}\vfill
\begin{table}[ptb]
\scalebox{0.9}{
\begin{tabular}{lccc}\hline
Uncertainty & Mode & Description & Category  \\  \hline \hline
BeRPA & 1p1h/QE & RPA/nuclear model suppression &  1  \\  \hline 
$E_b$ & 1p1h/QE & Shift in nuclear model removal energy  &  2A   \\  \hline %2
MnvaTune1 & 2p2h & Strength into (nn)pp only &  1 \\  \hline 
MnvaTune2 & 2p2h  & Strength into np pairs only &  1 \\  \hline 
MnvaTune3 & 1p1h/QE+2p2h & Strength into 1p1h vs. 2p2h  &  1 \\  \hline 
ArC2p2h & 2p2h Ar/C scaling & Electron scattering SRC pairs & 1 \\ \hline
$E_{2p2h}$ & 2p2h & 2p2h Energy dependence  &  2B \\  \hline 
Low $Q^2$ $1\pi$ & RES & Low $Q^2$ (empirical) suppression &   1 \\  \hline 
MK model & RES & Alternate strength in W &   1 \\  \hline
CC Non-resonant $\nu\rightarrow\ell+1\pi$ & $\nu$ DIS & Norm. for $\nu+n/p\rightarrow\ell+1\pi$ (\it{c.f.}\ref{Rodrigues:2016xjj}) & 1 \\ \hline
Other Non-resonant $\pi$ & $N\pi$ DIS & Per-topology norm. for $1<W<5$ GeV. & 1 \\ \hline
$E_{avail}/q_0$ & all & Extreme FSI-like variations  &  2B \\ 
\hline
Modified proton energy & all & 20\% change to proton E &  2B \\ 
\hline
$\nu_\mu\to\nu_e$ & $\nu_e$/$\overline{\nu}_e$ & 100\% uncertainty in $\nu_e$ unique phase space &  3 \\ \hline
$\nu_e$/$\overline{\nu}_e$ norm & $\nu_e$,$\overline{\nu}_e$ & Ref.~\cite{Day-McFarland:2012}  &  3 \\ 
\hline
\hline
\end{tabular} 
} % scalebox
\caption{List of extra interaction model uncertainties in addition to those provided by GENIE.}
\label{tab:nuintsystlist}
\end{table} 
%\vfill
%\end{landscape}
%}

%\subsubsection{Uncertainties not accounted for in this framework}
%% KM - Merged with introduction which seems natural to raise caveats. Other cases here are described well in the text.

%As previously noted, there are cross section variations which are not easily accounted for in the framework that has been developed for the DUNE TDR.  Alternate initial state models are extremely difficult to study, and in many cases not implemented within the GENIE framework.  Alternate final state interaction models do exist within GENIE, but one can not reweight between final state interaction models.

%Another category of systematic uncertainty not considered are ones that are, in principle, easily handled within this framework, but are not ready for the TDR.  Those include studies of Coulomb corrections, particularly as they relate to flavor dependent uncertainties, use of the $z$-expansion to describe weak form factors of nucleons, and uncertainties on coherent inelastic pion production which is a subleading process.

%Table~\ref{tab:mds} lists a set of possible mock data studies which would address some of these missing systematic uncretainties. Ideally, the suite of uncertainties provided in the proposed model should appropriately cover these scenarios. However, if bias or decreased coverage of oscillation parameters is observed outside of the coverage of the reweightable systematics, then the final contours should be appropriately adjusted. 

%\begin{table}
%\begin{tabular}{lcc}\hline 
%\hline
%NEUT SF & QE+FSI & SF with different FSI, no 2p2h  \\  \hline
%NuWro & all+FSI &  Similar underlying model with different FSI  \\  \hline 
%%GENIE & G18\_10a &  Alternate GENIE models \\ \hline
%GENIE hN2015 & all & Same base model as production with alternate FSI model \\ \hline
%z-expansion & QE & Alternate non-dipole form factors   \\  \hline
%DUNE PRISM & all & Probes $E_{rec}$-$E_{true}$ response function.   \\ 
%\end{tabular} 
%\caption{List of alternate models to be tested in the analysis (``mock data studies'')}
%\label{tab:mds}
%\end{table} 