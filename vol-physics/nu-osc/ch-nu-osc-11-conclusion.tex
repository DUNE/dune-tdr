\section{Conclusion}
\label{sec:physics-lbnosc-conclude}

The studies presented in this chapter are based on full, end-to-end simulation, reconstruction, and event selection of \dword{fd} Monte Carlo and parameterized analysis of \dword{nd} Monte Carlo. Detailed uncertainties from flux, the neutrino interaction model, and detector effects have been included in the analysis. Sensitivity results are obtained using a sophisticated, custom fitting framework. These studies demonstrate that DUNE will be able to achieve its primary physics goals of measuring \deltacp to high precision, unequivocally determining the neutrino mass ordering, and making precise measurements of the oscillation parameters governing long-baseline neutrino oscillation. It has also been demonstrated that accomplishing these goals relies upon accumulated statistics from a well-calibrated, full-scale FD,  operation of a 1.2-MW beam upgraded to 2.4~MW, and detailed analysis of data from a highly capable ND.

DUNE will be able to establish the neutrino mass ordering at the 5$\sigma$ level for 100\% of \deltacp values after between two and three years. CP violation can be observed with 5$\sigma$ significance after about 7 years if \deltacp = $-\pi/2$ and after about 10 years for 50\% of \deltacp values. CP violation can be observed with 3$\sigma$ significance for 75\% of \deltacp values after about 13 years of running. For 15 years of exposure, \deltacp resolution between five and fifteen degrees are possible, depending on the true value of \deltacp. The DUNE measurement of \sinstt{13} approaches the precision of reactor experiments for high exposure, allowing measurements that do not rely on an external \sinstt{13} constraint and facilitating a comparison between the DUNE and reactor \sinstt{13}  results, which is of interest as a potential signature for beyond the standard model physics. DUNE will have significant sensitivity to the $\theta_{23}$ octant for values of \sinst{23} less than about 0.47 and greater than about 0.55.

These measurements will make significant contributions to completion of the standard three-flavor 
mixing picture and guide theory in understanding if there are new symmetries in the neutrino sector or whether there is a relationship between the generational structure of quarks and leptons. Observation of CP violation in neutrinos would be an important step in understanding the origin of the baryon asymmetry of the universe. Precise measurements made in the context of the three-flavor paradigm may also yield inconsistencies that point us to physics beyond the standard three-flavor model. 