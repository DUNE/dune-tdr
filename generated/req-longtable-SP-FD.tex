% This file is generated, any edits may be lost.

% It defines macros which expand to corresponding
% specification values for subsystem SP-FD



\begin{longtable}{p{0.25\textwidth}p{0.7\textwidth}}   
\caption{Specification for SP-FD \fixmehl{ref \texttt{tab:specs:SP-FD}}} \\

\rowcolor{dunesky}
\newtag{SP-FD-1}{ spec:min-drift-field } & Name: Minimum drift field \\ 
    Description & The drift field in the TPC shall be greater than 250 V/cm, with a goal of 500 V/cm.   \\  \colhline
    Specification (Goal) &  $>$\,\SI{250}{ V/cm}  ( $>\,\SI{500}{ V/cm}$ ) \\   \colhline
    
    Rationale &   Limits impacts of electron-ion recombination (on particle ID via $dE/dx$ versus range), reduces effect of finite electron lifetime on S/N ratio (with implications on tracking and calorimetry), and limits electron diffusion and to a lower degree space charge effects.  \\ \colhline
    Validation & ProtoDUNE will demonstrate if the present HVS design allows reaching the nominal electric field in the drift volume.  Initial data taking will be with the maximum obtainable electric field setting, but additional studies at lower fields to study the effect on particle ID will also be targeted. Detector simulation will take advantage of the experimental data collected with ProtoDUNE.   Additional runs collected at lower field settings will allow for more fine tuning of the models.   \\
   \colhline
\rowcolor{dunesky}
\newtag{SP-FD-2}{ spec:system-noise } & Name: System noise \\ 
    Description & The total system noise seen by each wire should be no more than 1000 enc of noise, with a goal of ALARA. It is expected that random noise on the FE amplifier will be the dominant contribution to the total system noise.   \\  \colhline
    Specification (Goal) &  $<\,\SI{1000}{enc}$  ( ALARA ) \\   \colhline
    
    Rationale &   The noise specification is driven by pattern recognition and two-track separation.  Studies suggest that a minimum of ~5/1 signal to noise ratio on individual wire measurements allows for sufficient reconstruction performance. Converting this signal to noise ratio to ENC on an induction plane signal for a MIP with minimum signal yields the figure of system-noise. The corresponding ratio on the collection wires is  approximately 13/1.  \\ \colhline
    Validation & ProtoDUNE will demonstrate the noise level achievable by the current electronics and validate the grounding and shielding rules being used. Simulation will quantify how physics reach is correlated to electronics noise for selected physics channels.  \\
   \colhline
\rowcolor{dunesky}
\newtag{SP-FD-3}{ spec:light-yield } & Name: Light yield \\ 
    Description & The light yield shall be sufficient for measuring event time (and total intensity) of events with visible energy above 200 MeV.  Goal is to make possible a 10\% energy measurement for events with a visible energy of 10 MeV.   \\  \colhline
    Specification (Goal) &  $>\,\SI{0.5}{pe/MeV}$  ( $>\,\SI{5}{pe/MeV}$ ) \\   \colhline
    
    Rationale &   Minimal requirement is based on rejecting background to nucleon decay events occuring near the cathode, for which the produced photons need to travel furthest to reach the photon detectors.  \\ \colhline
    Validation &   \\
   \colhline
\rowcolor{dunesky}
\newtag{SP-FD-4}{ spec:time-resolution-pds } & Name: Time resolution \\ 
    Description & The time resolution of the photon detection system shall be less than 1 microsecond in order to assign a unique event time.   \\  \colhline
    Specification (Goal) &  $<\,\SI{1}{\micro\second}$  ( $<\,\SI{100}{\nano\second}$ ) \\   \colhline
    
    Rationale &   Based on the minimal energy deposition (10 MeV), spatial separation (\SI{1}{m}), and temporal separation (\SI{1}{ms}) for which one wants to assign a unique event time. Time resolution of \SI{1}{\micro\second} is required to have position resolution along the drift direction of about \SI{1}{mm}.  \\ \colhline
    Validation &   \\
   \colhline
\rowcolor{dunesky}
\newtag{SP-FD-5}{ spec:lar-purity } & Name: Liquid argon purity \\ 
    Description & The LAr purity shall remain lower than 100 ppt (with a goal of < 30 ppt). This corresponds to e- lifetime of 3 (10) ms.   \\  \colhline
    Specification (Goal) &  $<$\,\SI{100}{ppt}  ( $<\,\SI{30}{ppt}$ ) \\   \colhline
    
    Rationale &   Liquid argon purity directly impacts the number of electrons received at the APA collection wires and hence the S/N.  \\ \colhline
    Validation & Liquid argon purity will be measured via dedicated purity monitors and through different analysis techniques.  We will have the ability to operate at different purity levels to study its effects on track reconstruction and HVS stability.  \\
   \colhline
\rowcolor{dunesky}
\newtag{SP-FD-6}{ spec:apa-gaps } & Name: Gaps between APAs  \\ 
    Description & The gap size between APAs shall minimize loss of fiducial volume and distortion of charge collection.   \\  \colhline
    Specification (Goal) &  $<\,\SI{15}{mm}$ between APAs on same support beam; $<\,\SI{30}{mm}$ between APAs on different support beams  ( ALARA ) \\   \colhline
    
    Rationale &   Allowing gaps to open up between some APAs will reduce the overall shrinkage of the detector.  This also simplifies construction and installation of the detector support structure. Use of electron diverters should minimize impact of gaps, based on simulation.  \\ \colhline
    Validation & Effect of gaps between APAs (and effectiveness of installed electron diverters) on track reconstruction will be studied using the ProtoDUNE data.  \\
   \colhline
\rowcolor{dunesky}
\newtag{SP-FD-7}{ spec:misalignment-field-uniformity } & Name: Drift field uniformity due to component alignment \\ 
    Description & Misalignments of the various TPC components shall not introduce drift-field nonuniformities beyond those specified in the HVS requirements.   \\  \colhline
    Specification (Goal) &  $<\,1\,$\% throughout volume  ( ALARA ) \\   \colhline
    
    Rationale &   DSS should provide support such that the APA, CPA and FC can maintain rectangular parallelepiped under LAr conditions. Specifically the rails should remain parallel to each other at a fixed distance throughout the detector volume  \\ \colhline
    Validation & Space charge in ProtoDUNE will complicate the analysis of the field uniformity. Local effects close to the field cage electrodes could be in principle be disentagled, exploiting through-going muon tracks.   \\
   \colhline
\rowcolor{dunesky}
\newtag{SP-FD-8}{ spec:apa-wire-angles } & Name: APA wire angles \\ 
    Description & The wire angles shall be set such that each induction wire crosses each collection wire only once, reducing ambiguities.   \\  \colhline
    
    Specification &  \SI{0}{\degree} for collection wires, \SI{35.7}{\degree} for induction wires \\   \colhline
    
    Rationale &   This starts from the constraint that wire readout electronics sits on top of APA to minimize dead space between APAs.  \\ \colhline
    Validation & No validation required; this is a geometrical calculation.  \\
   \colhline
\rowcolor{dunesky}
\newtag{SP-FD-9}{ spec:apa-wire-spacing } & Name: APA wire spacing \\ 
    Description & The wire spacing shall be chosen as a compromise between good S/N (coarser pitch) and good vertex resolution (finer pitch).   \\  \colhline
    
    Specification &  \SI{4.669}{mm} for U,V; \SI{4.790}{mm} for X,G \\   \colhline
    
    Rationale &   S/N consistent with 100\% hit reconstruction efficiency for MIPs.  Spacing fine enough to provide \SI{1.5}{cm} vertex resolution in $y-z$ plane.  \\ \colhline
    Validation & No validation required; this is a design choice. Monte Carlo studies have been performed showing no significant difference in sensitivity to CP violation for smaller wire spacing.  \\
   \colhline
\rowcolor{dunesky}
\newtag{SP-FD-10}{ spec:apa-wire-pos-tolerance } & Name: APA wire position tolerance \\ 
    Description & The wire position tolerance shall be set so as to ensure uniform wire plane transparency and required reconstruction precision on dE/dx.   \\  \colhline
    Specification (Goal) &  $\pm\,\SI{0.5}{mm}$  ( ALARA ) \\   \colhline
    
    Rationale &   Tolerance is on both the spacing between adjacent wires in each plane and the spacing between the wire planes.  \\ \colhline
    Validation &   \\
   \colhline
\rowcolor{dunesky}
\newtag{SP-FD-11}{ spec:hvs-field-uniformity } & Name: Drift field uniformity due to HVS \\ 
    Description & Design of TPC cathode and FC components shall ensure uniform field.  Production tolerances shall be set so as to maintain flatness of component surfaces and, by extension, the shape of the drift field volume.   \\  \colhline
    Specification (Goal) &  $<\,\SI{1}{\%}$ throughout volume  ( ALARA ) \\   \colhline
    
    Rationale &   Non-uniformity of \efield affects 3D reconstruction due to introduction of non-constant electron drift velocity, including residual transverse components with respect to the nominal drift direction.  \\ \colhline
    Validation & Space charge in ProtoDUNE will complicate the analysis of the field uniformity. Local effects close to the field cage electrodes could be in principle be disentagled, exploiting through-going muon tracks. Simulation will be used to determine fiducial volume cuts, with input from the ProtoDUNE data.  \\
   \colhline
\rowcolor{dunesky}
\newtag{SP-FD-12}{ spec:hv-ps-ripple } & Name: Cathode HV power supply ripple contribution to system noise \\ 
    Description & Power supply ripple shall be adequately attenuated to guarantee that its contribution to the overall system electronics noise  is negligible.   \\  \colhline
    Specification (Goal) &  $<\,\SI{100}{enc}$  ( ALARA ) \\   \colhline
    
    Rationale &     \\ \colhline
    Validation & Effect of HV system on the baseline electronics noise should be easily measured by switching on and off the HV power supply.  Equivalent full circuit simulation of the coupling between the HVS and the FE electronics will be performed to compare against ProtoDUNE data and validate the FD design.  \\
   \colhline
\rowcolor{dunesky}
\newtag{SP-FD-13}{ spec:fe-peak-time } & Name: Front-end peaking time \\ 
    Description & The FE peaking time shall be set so as to optimize vertex resolution.    \\  \colhline
    Specification (Goal) &  \SI{1}{\micro\second}  ( Adjustable so as to see saturation in less than \SI{10}{\%} of beam-produced events ) \\   \colhline
    
    Rationale &   Driven by the need for optimal vertex resolution, which is itself determined from single-track resolution and the ability of the reconstruction algorithms to separate two nearby tracks.  Value of 1 us is based on 5mm spacing between anode planes. Reconstruction performance depends on anode wire spacing, anode-to-cathode spacing, bias voltages, and distance between anode planes (as well as electron drift lifetime).  \\ \colhline
    Validation & ProtoDUNE vertex reconstruction in multi-track events will demonstrate the validity of the requirement. Simulation studies could illuminate the trade-off between shaping time and noise if the new electronics is able to achieve lower noise at longer shaping times.  \\
   \colhline
\rowcolor{dunesky}
\newtag{SP-FD-14}{ spec:sp-signal-saturation } & Name: Signal saturation level \\ 
    Description & \   \\  \colhline
    
    Specification &  \num{500000} electrons \\   \colhline
    
    Rationale &   The largest signals correspond to events with multiple protons produced in the primary event vertex, in particular, when the trajectories of one or more of those particles are parallel to the wire, causing the charge over a long path length to be collected within a short time period.    \\ \colhline
    Validation &   \\
   \colhline
\rowcolor{dunesky}
\newtag{SP-FD-15}{ spec:lar-n-contamination } & Name: LAr nitrogen contamination \\ 
    Description & The nitrogen contamination in the LAr shall remain below 25 ppm in order not to significantly affect the number of photons that reach the detectors (for both fast and late light components).   \\  \colhline
    Specification (Goal) &  $<\,\SI{25}{ppm}$  ( ALARA ) \\   \colhline
    
    Rationale &   Production quenching  prevents most late light from being collected at levels of 10ppM and above.  The attenuation length in liquid argon with 50 ppM of Nitrogen contamination is roughly 3m, which starts to signifcantly affect the number of photons that reach the detectors (for both fast and late light components).  \\ \colhline
    Validation & Nitrogen contamination levels can be measured both from gas analyzers and from studies of the effective shortening of the slow component of generated scintillation light orginating from nitrogen quenching.  Comparison of the two will inform light simulation models.  \\
   \colhline
\rowcolor{dunesky}
\newtag{SP-FD-16}{ spec:det-dead-time } & Name: Detector dead time \\ 
    Description & The down time of the detector should be such that data taking interruptions affecting all active cryostats are kept below $<$0.5\% with a goal of ALARA   \\  \colhline
    Specification (Goal) &  $<\,\SI{0.5}{\%}$  ( ALARA ) \\   \colhline
    
    Rationale &   In order to collect the required physics data, the detector must operate stably over long time periods. The specification is driven by the risk of missing a supernova burst if all operating cryostats are offline.  \\ \colhline
    Validation & Operation of the ProtoDUNE detector will provide information on the stability of the HVS over time.    \\
   \colhline
\rowcolor{dunesky}
\newtag{SP-FD-17}{ spec:cathode-resistivity } & Name: Cathode resistivity \\ 
    Description & The cathode resistivity shall ensure that in the event of an HV discharge, the release of the large stored energy is spread out over time.    \\  \colhline
    Specification (Goal) &  $>\,\SI{1}{\mega\ohm/square}$  ( $>\,\SI{1}{\giga\ohm/square}$ ) \\   \colhline
    
    Rationale &   This prevents damage to detector components. The goal resistivity is > \SI{1}{\giga\ohm}/sq  \\ \colhline
    Validation & In ProtoDUNE the resistivity of the CPA panels  is in the MW/sq range due to the detector being operated on the surface.  There will be opportunities to test if the front-end electronics is adeguately protected against discharges (hopefully at the end of beam operations).  Existing discharge simulations can be tuned based on ProtoDUNE data to better validate far detector designs.  \\
   \colhline
\rowcolor{dunesky}
\newtag{SP-FD-18}{ spec:cryo-monitor-devices } & Name: Cryogenic monitoring devices \\ 
    Description & CISC shall provide sufficient instrumentation  to validate the fluid flow model, which constrains the level of purity inhomogeneities within the liquid.    \\  \colhline
    
    Specification &   \\   \colhline
    
    Rationale &     \\ \colhline
    Validation & All cryogenic instrumentation devices will be validated in ProtoDUNEs as same designs are being used for DUNE. The current CFD simulation will be validated/tuned using ProtoDUNE instrumentation data.  \\
   \colhline
\rowcolor{dunesky}
\newtag{SP-FD-19}{ spec:adc-sampling-freq } & Name: ADC sampling frequency \\ 
    Description & The ADC sampling frequency shall be set so as to extract maximal information without unecessarily increasing data rate.   \\  \colhline
    
    Specification &  $\sim\,\SI{2}{\mega\hertz}$ \\   \colhline
    
    Rationale &   This value is chosen to match \SI{1}{\micro\second} shaping time (the approximate Nyquist requirement).  \\ \colhline
    Validation & No validation required; this is a design choice.  \\
   \colhline
\rowcolor{dunesky}
\newtag{SP-FD-20}{ spec:adc-number-of-bits } & Name: Number of ADC bits \\ 
    Description & The ADC shall digitize the charge deposited on the wires with12 bits precision.   \\  \colhline
    Specification (Goal) &  \num{12} bits  ( \num{13} bits ) \\   \colhline
    
    Rationale &   The lower end of the ADC dynamic range is driven by the requirement that the
ADC digitization does not contribute to the total electronics noise. The upper end of the ADC dynamic range is defined by the previous requirement on the signal saturation level. Combining these two requirements with the specification for the
total electronics noise results in a the need for a 12 bits digitization.  \\ \colhline
    Validation & No validation required; this is a design choice.  \\
   \colhline
\rowcolor{dunesky}
\newtag{SP-FD-21}{ spec:ce-power-consumption } & Name: Cold electronics power consumption  \\ 
    Description & The CE power consumption shall remain below 50 mW/channel.  Goal is ALARA.   \\  \colhline
    Specification (Goal) &  $<\,\SI{50}{ mW/channel} $  ( ALARA ) \\   \colhline
    
    Rationale &   The CE power consumption must be low enough to prevent the occurrence of local hot spots that could cause bubbling in the LAr, potentially leading to HV discarges.    \\ \colhline
    Validation &   \\
   \colhline
\rowcolor{dunesky}
\newtag{SP-FD-22}{ spec:data-rate-to-tape } & Name: Data rate to tape \\ 
    Description & The DAQ shall provide capability for triggering on events of interest in order to limit the total volume for events stored on tape.   \\  \colhline
    
    Specification &  $<\,\SI{30}{PB/year}$ \\   \colhline
    
    Rationale &     \\ \colhline
    Validation &   \\
   \colhline
\rowcolor{dunesky}
\newtag{SP-FD-23}{ spec:sn-trigger } & Name: Supernova trigger \\ 
    Description & The DAQ architecture shall provide a mechanism for triggering on galactic supernova bursts and recording neutrino interactions associated with those bursts over a 30 second period, with a goal of 100 seconds. During this period, sufficient information must be recorded to allow reconstruction of $>$90\% of neutrino interactions.   \\  \colhline
    
    Specification &  $>\,\SI{90}{\%}$ efficiency for SNB within \SI{100}{kpc} \\   \colhline
    
    Rationale &   Most models of SNB show structure in the neutrino flux for up to 30s and there is potential for interesting measurements to be made up to 100s. 90\% triggering efficiency for a typical burst out to 100 kpc provides sensitivity to SNB in our galaxy and several small nearby galaxies.  \\ \colhline
    Validation &   \\
   \colhline
\rowcolor{dunesky}
\newtag{SP-FD-24}{ spec:local-e-fields } & Name: Local electric fields \\ 
    Description & The integrated detector design shall minimize potential pathways for HV discharges.   \\  \colhline
    Specification (Goal) &  $<\,\SI{30}{kV/cm}$  ( ALARA ) \\   \colhline
    
    Rationale &   HV discharges limit detector livetime and have the potential for damaging detector components. Keeping the local fields to this value is necessary in order to reach the desired drift fields. The minimum \efield requirement is based on the minimum drift-field goal of \SI{500}{V/cm}.  \\ \colhline
    Validation &   \\
   \colhline
\rowcolor{dunesky}
\newtag{SP-FD-25}{ spec:non-fe-noise } & Name: Non-FE noise contributions \\ 
    Description & All non-FE noise contributions shall be much lower than the targeted system noise level, with a goal of ALARA   \\  \colhline
    Specification (Goal) &  $<<\,\SI{1000}{enc} $  ( ALARA ) \\   \colhline
    
    Rationale &   It is expected that system noise will be dominated by random noise on the front-end.    \\ \colhline
    Validation &   \\
   \colhline
\rowcolor{dunesky}
\newtag{SP-FD-26}{ spec:lar-impurity-contrib } & Name: LAr impurity contributions from components \\ 
    Description & Contributions to LAr contamination from detector components, through outgassing or other processes, shall remain << 30 ppt  (ALARA) so as to avoid significantly increasing the nominal level of contamination.   \\  \colhline
    Specification (Goal) &  $<<\,\SI{30}{ppt} $  ( ALARA ) \\   \colhline
    
    Rationale &     \\ \colhline
    Validation &   \\
   \colhline
\rowcolor{dunesky}
\newtag{SP-FD-27}{ spec:radiopurity } & Name: Introduced radioactivity \\ 
    Description & Introduced radioactivity shall be less than that from 39Ar.   \\  \colhline
    Specification (Goal) &  ALARA  ( ALARA ) \\   \colhline
    
    Rationale &   Materials used in construction of detector components should not significantly increase the radiological background beyond nominal levels.  \\ \colhline
    Validation &   \\
   \colhline
\rowcolor{dunesky}
\newtag{SP-FD-28}{ spec:dead-channels } & Name: Dead channels \\ 
    Description & Detector components shall be sufficiently reliable so as to ensure that dead channels do not exceed 1\% over the lifetime of the experiment.   \\  \colhline
    Specification (Goal) &  $<\,\SI{1}{\%}$  ( ALARA ) \\   \colhline
    
    Rationale &   Detector needs to operate over a 20-30 year liftime with components that will be inaccessible post-installation, so components must be robust against damage during installation/cooldown and have long-term stability.  \\ \colhline
    Validation & ProtoDUNE will identify infant mortality rates for the current detector components.  Additional cosmic-ray operation in 2019 will validate longer-term stability.  \\
   \colhline


\end{longtable}
