
% specification values for subsystem SP-FD-APA
\begin{longtable}{p{0.02\textwidth}p{0.18\textwidth}p{0.08\textwidth}p{0.27\textwidth}p{0.04\textwidth}p{0.28\textwidth}} 
\caption{Specification for SP-FD-APA \fixmehl{ref \texttt{tab:specs:SP-FD-APA}}} \\
\rowcolor{dunesky}
ID & Title & Category & Explanation & Risk Level & Mitigation \\  \colhline
1 & Winder modifications do not provide enough reduction in time for APA assembly. & APA Schedule & It is not possible to substantially reduce the manufacturing time of an APA (presently about 3 months) by middle of 2019 because of limited engineering resources, complexity of the winding problems, tight manufacturing tolerances. & M & Allocate enough engineering resources asap to address modifications to the winder. Plan for enough assembly lines to produce the total number of APAs within the required time.  \\  \colhline
2 & Cabling for PD and CE can not be accommodated within the 2-APA assembly/installation procedure. & APA Technical & Larger tube sizes and slots for the cables may weaken APA structure, or the cabling procedure of the APA may not be compatible with the installation operation. & M & Allocate enough engineering resources and proceed as soon as possible with the design modifications, to provide enough time to examine different solutions.  \\  \colhline
3 & Evolution in the design of the photon detectors require modifications of the APA frame at a late time & APA Technical & The design of the prototype frame for the DUNE FD will take into account the requirements of the PD consortium. Additional requests may come at a later time. & M & Interface activities with the Photon Detector consortium should be given a very high priority. A deadline for requesting modifications to the APA frame should be set by the DUNE Project.  \\  \colhline
4 & Experience with protoDUNE shows an unacceptable large number of dead channels. & APA Technical & We are planning for a channel failure percentage of less than 1 percent due to mechanical problems, with a goal of less than 0.5 percent. ProtoDUNE results may point to a larger failure rate. & M & Allocate enough engineering/scientific resources to understand the cause of the problem. Improve QA/QC procedures and possibly some aspects of the design.  \\  \colhline
5 & Experience with ProtoDUNE shows that the mechanical design of the APA is not adequate & APA Technical & Measurable change in wire tension after the cold test, wire breakage, loose wires, warping of the frame & L & Cold test in protoDUNE should provide an answer. Perform a cold test on the prototype APA for DUNE FD.  \\  \colhline
6 & Experience with ProtoDUNE shows problems in reconstruction efficiency of events in LAr & APA Technical & Wire spacing inadequate, difficulty in reconstruction of 3D events, reconstruction efficiency not uniform across the APA, presence of border effects & L & Develop as realistic as possible simulation and reconstruction. Analyze ProtoDUNE data asap.  \\  \colhline

\label{tab:risks:SP-FD-APA}
\end{longtable}