%rjw 11/28/18 Remove from Overview section out as a separate file
\section{Design Validation}
\label{sec:fdsp-pd-validation}
%\metainfo{\color{blue} Content: Segreto/Cancelo/Mualem/Djurcic}

%\fixme{Update the following to reflect protodune, measurement plans in Brazil, ICEBERG and what else is needed to be done prior to final design. Much of this is currently in the Design section. I have added section headings rjw 11/32/18.}

This section summarizes the most important sets of measurements, completed, % to date and those that are 
ongoing or planned, that validate the \dword{pds} design, along with the \dword{mc} simulation that indicates how well the \dword{pd} detector performance satisfies %matches to 
the physics requirements.

\fixme{From 1.1.5.1 Anne}

DUNE investigated numerous \dword{pd} photon collector module options prior to the formation of the \single \dword{pd} consortium, of which we selected four for further development. 
Two of the designs, \dword{sarapu}\footnote{\textit{Arapuca} is the name of a simple trap for catching birds originally used by the Guarani people of Brazil.} (\textbf{S}tandard-) and \dword{xarapu} (e\textbf{X}tended-), use a relatively new concept that is scalable and may enable %has the potential for the  
significantly better performance than the other approaches. Functionally, ARAPUCA is %functionally
 a light trap that captures wavelength-shifted photons inside boxes with highly reflective internal surfaces until they are eventually detected by \dwords{sipm}.  The two other designs are based on the use of wavelength-shifters and long plastic light guides coupled to \dwords{sipm} at the ends. Their performance meets the basic physics requirements but with only a small safety margin and they are not easily scalable within the geometric constraints of the \dword{spmod}. \dword{xarapu} is an evolution of the original ARAPUCA concept that combines the desirable characteristics of both a light guide and a photon trap. 
Figure~\ref{fig:3dtpc-pd} shows a \threed model of the \single TPC %with a zoom in to 
and a detail of the anode plane where the three photon collector technologies developed to full-scale prototypes for \dword{pdsp} are illustrated % visible; for illustration 
the \dword{spmod} will %contain 
use just \dword{xarapu}.

Though the ARAPUCA concept is relatively recent -- it was first proposed in 2015 -- 
%and accepted for installation in \dword{pdsp} in mid-2016 
a series of prototypes on an evolving design has resulted in detection efficiency measurements ranging from  \num{0.4}\% to \num{1.8}\%, demonstrating the potential for substantially higher performance than the light guide designs that were also considered. \dword{mc} simulations show that %detection efficiencies at the level of several per cent could be reasonably reached with improvements to the basic design. 
improvements to the basic design could lead to detection efficiencies at the level of several per cent. 
The concept also naturally allows for finer spatial segmentation along the detector modules than is the case for the light guide designs. 
\fixme{along a module? or within a module? (Anne)}
The baseline design, the \dword{xarapu} detailed in Section~\ref{ssec:fdsp-pd-pc-arapuca}, is an evolution of the ARAPUCA concept that combines the light trapping capability of \dword{sarapu} with the collection and transport capabilities of light guides. 
\fixme{end from 1.1.5.1 anne}

\subsection{Small-Scale \dword{sarapu} Prototypes}

%\metainfo{Machado/Segreto}

\subsubsection{Proof-of-Concept Measurements}
\label{sec:proof-principle}
\fixme{in some of these you say when and where the measurements were made, but not in others. Suggest all or none. Anne}

ARAPUCA prototypes with different configurations have been tested in \lar at multiple facilities. In each case, the first wavelength shift of \SI{127}{nm} scintillation photons down to \SI{350}{nm} that can pass through the filter substrate was performed by p-TerPhenyl (pTP) evaporated onto the outside of a dichroic filter window. 

The first prototype was made of PTFE \fixme{define} with internal dimensions of \SI{3.5}{cm} $\times$ \SI{2.5}{cm} $\times$ \SI{0.6}{cm}, a window formed from a dichroic filter of  dimensions \SI{3.5}{cm} $\times$ \SI{2.5}{cm} and a wavelength cut-off at \SI{400}{nm}. 
TetraPhenyl-Butadiene (\dword{tpb}) was evaporated onto the internal side of the filter where it absorbs the shifted \SI{350}{nm} photons and re-emits around \SI{430}{nm}. \fixme{around?}Trapped light was detected by a single \SI{0.6}{cm} $\times$ \SI{0.6}{cm} SensL \dword{sipm} mod C60035\footnote{http://sensl.com/products/c-series/}.
The device was installed inside a vacuum-tight stainless-steel cylinder closed by two CF100 flanges. The cylinder was deployed inside a \lar open bath, vacuum pumped down to a pressure around  10$^{-6}$~\si{mbar} and then filled with one liter of ultra-pure \lar\footnote{Argon 6.0, less than \SI{1}{ppm} total residual contamination.}. 
Scintillation light emission was produced by an alpha source\footnote{A $^{238}$U-Al alloy in the form of a metallic foil, with alpha particle emission of 4.267 MeV.} installed in front of the ARAPUCA immersed in \lar. Signals were read out through an Aquiris\footnote{Aquiris High-Speed Digitizer products; http://www.acqiris.com/.} PCI board and stored on a computer.
%Figure \ref{LNLS_test} shows ARAPUCA and the cryogenic system on the Toroidal Grating Monochromator (TGM) beamline at the Brazilian Synchrotron Light Laboratory (LNLS). 
%rjw 12/1/18 since there will be photos of the x-arapuca test facility, we can skip these older ones.
%Figure \ref{LNLS_test} shows photographs of the ARAPUCA and cryogenic system 
%%in the Toroidal Grating Monochromator (TGM)  beamline of 
%at the Brazilian Synchrotron Light Laboratory (LNLS). 

%\begin{dunefigure}[ARAPUCA test at the Brazilian Synchrotron Light Laboratory.]{LNLS_test}
%{ARAPUCA test at the Brazilian Synchrotron Light Laboratory} 
%	\includegraphics[height=6cm]{pds-tgm_1} \quad
%	\includegraphics[height=6cm]{pds-tgm_30}\quad
%	\includegraphics[height=6cm]{pds-tgm_0}
%\end{dunefigure}

The detection efficiency of the ARAPUCA was calculated by determining the number of detected photoelectrons corresponding to the end point of the $\alpha$ spectrum and comparing that with the expected number of photons impinging on the acceptance window for that % particular 
energy value (\SI{4.267}{MeV}).  This calculation depends only on known properties of \lar and on the solid angle subtended by the ARAPUCA window. A detection efficiency at the level of 1.8\%  was measured, consistent with \dword{mc} expectations for this configuration~\cite{Marinho:2018doi}.


%%%%%%%%%%%%%%%%%%%%%%%%%%%%%%%%%
\subsubsection{\dword{tallbo}}
%\metainfo{New content: Gustavo}

%%%%%%%  
%The TallBo facility at Fermilab provides a \SI{450}{liter} capacity cryostat with \SI{56}{cm} inner diameter and up to a \SI{183}{cm} liquid depth that accommodates  up to three different PD modules with dimensions \todo{how different?} close to the real ones.

% Remove the following since it is superseded by more recent measurements. 
%Several prototypes were tested under cryogenic conditions at Fermilab \dword{tallbo}.  
%The first, performed in mid-2016 at the Proton Assembly Building (PAB) at the ScENE cryogenic test facility, had dimensions of  \SI{5.0}{cm}$\times$\SI{5.0}{cm}$\times$\SI{1.0}{cm} with a dichroic window of \SI{5.0}{cm}$\times$\SI{5.0}{cm} with a cut-off of \SI{400}{nm} which was deposited with pTP and \dword{tpb}. However, in this case, two of sensL \dwords{sipm} mod C60035 were installed inside the box.  The ARAPUCA was again deployed inside a vacuum-tight cryostat filled with ultrapure \lar. An $^{241}$Am alpha source was positioned in front of the window of the device \SI{5}{cm} from its center. The efficiency of the ARAPUCA was estimated taking into account that the alpha particles from this source have a  monochromatic energy of about \SI{5.4}{MeV}. 
%The estimated efficiency in this case was approximately 1\%, a factor two below the expected value; this is attributed to the sub-optimal quality and uniformity of the pTP and \dword{tpb} films, and to the lack of reflectivity of the inner PTFE surfaces in this early prototype.
%\fixme{was the "sub-optimal quality" based on a visual inspection or other determination?}

%A set of tests was performed at the beginning of 2017 using the \dword{tallbo} facility at Fermilab, which is large enough to  test several devices at a time. 
Eight %different 
ARAPUCA cells with filters from different manufacturers, different reflectors, and different dimensions were tested in early 2017 using the \dword{tallbo} facility at Fermilab.  
%Per Gustavo 12/4/18 
The ARAPUCA cells each contained two \SI{6}{mm} $\times$ \SI{6}{mm} SensL \dwords{sipm}. 
%Scintillation light was again produced by alpha particles emitted by an $^{241}$Am  source mounted on a holder that could be moved with an external manipulator in order to place it in front of each prototype. The detection efficiencies of these ARAPUCAs ranged from 0.4\% to 1.0\%.
An $^{241}$Am source, mounted on a movable holder that allowed placement in front of each prototype, provided the scintillation light. The detection efficiencies of these ARAPUCAs ranged from 0.4\% to 1.0\%.
%\fixme{Which type and how many \dwords{sipm} were in these ARAPUCAs?}

%The most recent measurements were performed in \dword{tallbo} at the end of 2017 with an array of eight ARAPUCAs together with two reference bars (double-shift light guide design). 
In late 2017 we measured an array of eight ARAPUCAs together with two reference bars of a double-shift light guide design. An array of counters on opposite sides of the cryostat formed a cosmic ray trigger . The so-called ``hi-low'' trigger selected through-going muons at a few degrees away from vertical.
\fixme{rjw what was the range of angles for the Hi-Low CR selection?}
%\fixme{Gustavo to include updates from recent TallBo analysis.}
% from Gustavo; edited down by rjw
%\subsection{Efficiency analysis}
%\textit{\it TallBo Measurement Efficiency analysis}

%Once we have selected the dataset, we perform the efficiency analysis.
We define the efficiency as the ratio between the number of measured photons and the estimated number of photons arriving at each ARAPUCA:
%\begin{equation}
$\mathcal{R}_{i}=\frac{PE_i}{PH_i}$.
%\end{equation} 
In addition to this metric, the ratio between the sum of the number of photoelectrons over all ARAPUCAs divided by the sum of the number of expected photons landing (PH) on them is also calculated. 
\fixme{sum of the number?}
%\begin{equation}
$\mathcal{R}_{TOT}=\frac{\sum_{i=1}^8PE_i}{\sum_{i=1}^8PH_i}$
%\end{equation} 
Three kinds of analyses were performed, finding similar results:
(1)  log-normal fit; (2) robust statistic; (3) strong correlation between the ARAPUCAs
%\end{itemize}
\fixme{clarify bootstrap and the correlation method}
Figure~\ref{fig:pds-tallbo-arapuca} displays the median values found with the first two analyses and the mean of the mean value obtained with the bootstrap procedure.
The purple points come from the log-normal fit, the light blue are for the robust analysis and in green the ones obtained with the bootstrap procedure.

The $x$-axis represents the ARAPUCA number, (number 0 is relative to the total ratio) and the $y$-axis gives the efficiency. A similar analysis is made for the low-low configuration,\fixme{??} replacing the energy deposit value with \SI{3}{MeV/cm}. A correction to this value is needed because the muon energy increases for more horizontal events.

The TallBo experiment showed that ARAPUCAs with only four \dwords{mppc} %with 
covering a total area of \SI{1.44}{$cm^2$} and a photon collection area of \SI{80}{$cm^2$} achieved close to 1\% absolute efficiency (the range of values for the eight ARAPUCA cells is 0.68\%--0.85\%). The effective ARAPUCA gain is about 4.5 times the photosensor area.

\begin{dunefigure}[ARAPUCA TallBo measurements.]
 {fig:pds-tallbo-arapuca}
 {Hi-low configuration values found with three analyses. The purple points come
from the log-normal fit, the light blue points are from the robust analysis and the green ones from
the bootstrap procedure. Number 0 is relative to the total ratio, the others are the ARAPUCA
numbers.}
%\includegraphics[angle=0,width=8.4cm,height=6cm]{pds-tallbo-arapuca-trepl.png}
\includegraphics[angle=0,height=4cm]{pds-tallbo-arapuca-trepl.png}
\end{dunefigure}


\subsubsection{Photosensors and Active Ganging}
\label{sec:pds-valid-ganging}
%\metainfo{Content: Gustavo}
%\fixme{11/29/18? Zelimir added this subsection, with Gustavo's material}

As described in Section~\ref{sec:pds-design-ganging}, the active ganging of \dwords{sipm}  aims to increase the active photo-detecting area while keeping the number of readout channels at a reasonable number. 
%Even when the ARAPUCA has a gain in detector sensitive area, the internal reflectivity of the ARAPUCA is not 100\%, hence more than a single photosensor per ARAPUCA is needed to keep a good photon collection efficiency. 
Several active ganging detectors were designed and tested during 2017-2018. 
The systems were based on an active summing node working in the \lar near the photosensors. Several incarnations of the cold summing node were designed and tested using SensL and Hamamatsu \dwords{sipm} \fixme{\dwords{mppc}?}  %and also 
as were several types of operational amplifiers (OpAmps). 
Some of these designs were tested, validated and successfully contributed to the March and November 2017 TallBo experiments.  
We describe here only the most recent design that demonstrated that 48 Hamamatsu \dwords{mppc} in the baseline design can be ganged together on a single differential output with excellent signal performance, low noise and low power dissipation.

Figure~\ref{fig:fig-pds-gang-1}(left) shows a matrix array of 72 \dwords{mppc} organized as 12 rows of six \dwords{mppc} each. 
The six \dwords{mppc} per row are connected in parallel, giving a total output capacitance of \SI{7.8}{nF}. The 12 rows are connected to the summing node of an OpAmp THS4131, as sketched in Figure~\ref{fig:fig-pds-gang-1}(right). 
Since the DUNE baseline design is based on 48 \dwords{mppc}/ARAPUCA module, only eight rows of six were used for the tests. 
The performance of the cold summing electronics was done by illuminating the \dword{mppc} array with an LED and digitizing the output with a high-speed oscilloscope and with the \dword{ssp} readout electronics (see Section~\ref{sec:ssp-protodune-electronics}).
As shown in Figure~\ref{fig:fig-pds-gang-2}, the mean signal shows a rise time of \SI{60}{ns} and a recovery time of \SI{660}{ns}, well within the DUNE PD specifications. %requirements.
%\fixme{Need reference/citation for \dword{ssp} electronics - is there an arXiv or docDB number?}

\begin{dunefigure}[Photosensors signal ganging scheme.]
 {fig:fig-pds-gang-1}
 {Summing board with a total of 72 \dwords{mppc} used to demonstrate the optimal combination of passive and active ganging with 48 Hamamatsu \SI{6}{mm}$\times$\SI{6}{mm} \dwords{mppc} (left).  Schematics of summing circuit with an OpAmp THS4131 (right).}
\includegraphics[height=6cm]{graphics/pds_gang_fig1.jpg}
\includegraphics[height=6cm]{graphics/pds_gang_fig2.png}
\end{dunefigure}

\begin{dunefigure}[Photosensors signal ganging with 48 \dwords{sipm}.]
 {fig:fig-pds-gang-2}
 {Waveform signal from 48 \dwords{mppc}/ARAPUCA module, summed with OpAmp THS4131, and digitized with the \dword{ssp} frontend board.}
\includegraphics[height=4.5cm]{graphics/pds-gang-rise_time.jpg}
\end{dunefigure}
\fixme{anne to here}

\begin{dunefigure}[Photoelectron peaks from 48 ganged \dwords{mppc}.]
 {fig:fig-pds-gang-3}
 {Photoelectron peaks from 48 ganged \dwords{mppc}, collected in liquid nitrogen at 45 Volt bias (left) and at 47 Volt bias (right).}
\includegraphics[height=4.5cm]{graphics/pds-gang48-45v.jpg}
\includegraphics[height=4.5cm]{graphics/pds-gang48-47v.jpg}
\end{dunefigure}
\fixme{Need larger text on the figures}

Figure~\ref{fig:fig-pds-gang-3} shows a histogram of light collection in the array for a bias voltage equivalent to 2 volts above the mean breakdown voltage. Since there are 48 \dwords{mppc} in the array, and a single common bias, there is a spread in the allowed gains. Even when the Hamamatsu \dwords{mppc} have a small spread in breakdown voltages, that spread is enough to smear the peaks in the histogram. It is worth noting the difference between noise and gain spread. The circuit noise can be measured as FWHM or RMS around the 0 PE signal (3.5 ADC counts in the figure). The first PE peak is at 33 ADC counts, resulting in an optimal signal to noise ratio of about 10.

Since the breakdown voltage of the \dwords{mppc} is specified by the manufacturer, for the DUNE experiment the gain spread can be reduced by picking groups of 48 \dwords{mppc} with similar breakdown values for each module. The differential output of the cold electronics impedance is matched to the readout electronics and able to reject more than \SI{60}{dB} of common mode noise. This is particularly important since the \dwords{mppc} and output wiring are inside a high voltage TPC. The timing properties of the 48 ganged electronics were also measured 
in LAr using a Am-241 alpha source. 
Figure~\ref{fig:fig-pds-gang-4} shows the time walk for a constant discrimination threshold which, 
%Figure~\ref{fig:fig-pds-gang-5} shows that, 
as expected, is not a linear function of the signal height. The error distribution, which is not Gaussian, has a FWHM of \SI{80}{ns}. This value is well within the DUNE specification (Table~\ref{tab:spec:time-resolution}).

\begin{dunefigure}[Time walk from 48 ganged \dwords{mppc}.]
 {fig:fig-pds-gang-4}
 {Time walk from 48 ganged \dwords{mppc} measured at the constant discrimination threshold with \dword{ssp} board.}
%\includegraphics[angle=0,width=8.4cm,height=4cm]{graphics/pds_gang_fig6.jpg}
\includegraphics[height=4.5cm]{graphics/pds-gang-time_walk.jpg}
\includegraphics[height=4.5cm]{graphics/pds_gang_fig8.jpg}
\end{dunefigure}

%Figure~\ref{fig:fig-pds-gang-5}  shows that, as expected, the time-walk is not a linear function of the signal height. The error distribution is not Gaussian. 
%The FWHM is 80 ns. This value meets DUNE physics requirements.

%\begin{dunefigure}
% {fig:fig-pds-gang-4}
% {Distribution of the time walk from from 48 \dwords{mppc}/ARAPUCA module.}
%\includegraphics[angle=0,height=4cm]{graphics/pds_gang_fig8.jpg}
%\end{dunefigure}

%Primarily this section is a summary of the active ganging studies, but is also effectively a report of the \dwords{mppc}.

%% Short section on Mu2e electronics with the active ganging studies 
% From Josh S 11/28/18
%\subsection{Mu2e Electronics Validation Tests, Completed and Planned}
\textit{Mu2e Electronics Validation}
%\fixme{Gustavo/Josh need to blend this Mu2e text with the active ganging studies text.}
The Mu2e electronics have undergone a number of end-to-end warm and cold tests to demonstrate single-photon sensitivity in various parallel/series ganging and \dword{sipm}/\dword{mppc} configurations. Here we summarize the results with the 72-\dword{mppc} active ganging array described in Section~\ref{sec:pds-valid-ganging}
%(Vb = 47.2 V; each of 12~rows of 6x6 mm$^2$ Hamamatsu S13360-6050VE \dwords{mppc} in %parallel, with total capacitance of 7.8 nF per row) 
at liquid nitrogen temperatures. 
%This active ganging system, discussed in Section~\ref{}, is designed to mimic the arrangement envisioned for DUNE, although the DUNE \dword{pd} active ganging scheme will feature 48~\dwords{mppc} rather than 78. 
A balun is used to convert from the differential \dword{mppc} array output to the single-ended \dword{feb}. The data represent triggers in time with an LED flasher, with samples taken every 12.5~ns for the length of the readout window ($\sim$3~$\mu$s in this case). Figure~\ref{fig:pds-board-balun-adc} shows the system used and a histogram of the maximum ADC value during each trigger window. The first peak above zero represents noise and the second peak represents a one-PE signal. The signal to noise from these tests was measured to be 4, determined by the ratio of the single photon peak (20 ADC, after subtracting the noise peak) to the spread in the noise ($\sigma_{noise}$ = 5~ADC) for direct comparison to the \dwords{ssp} (S/N = 5).

\begin{dunefigure}[Readout of 72-\dword{mppc} active ganging array with the Mu2e electronics readout board.]
{fig:pds-board-balun-adc}
{The Mu2e electronics readout board was used to read out a 72-\dword{mppc} active ganging array (V$_b$ = \SI{47.2}{V}) (left). The maximum ADC results are shown, with the first and second peaks representing 0 and 1~photoelectron signals (right).}\includegraphics[height=4.8in]{pds-board-balun-adc} 
\vspace{-7.0cm}
\end{dunefigure}


%%%%%%%%%%%%%%%%%%%%%%%%%%%%%%%%%%%%%%%%%%%%%%%
% rjw 11/23/18 add section headings for the major sources of ARAPUCA information to come.
%>>>> ProtoDUNE
%%%%%%%%%%%%%%%
\subsection{\dword{pdsp}}
%\metainfo{\color{blue} Content: Mualem}
%LMM made some transitional changes.  Maybe good enough for now.}

The most comprehensive set of data on the \dword{sarapu} will come from the fully instrumented modules in the \dword{pdsp} experiment~\cite{Abi:2017aow} that completed first beam running in November \num{2018}. 
Since \dword{pdsp} will remain filled with \lar for much of the CERN long-shutdown we will also have a long-term cold test of full-scale \dword{pd} modules for the first time so it may be possible to quantify any deterioration in their performance.
%such as the loss of \dword{tpb} from the coating noted previously. 
%More broadly, aging effects in various detectors technologies, such as scintillator and \dwords{pmt}, are well documented, and knowledge of such effects is required at the design stage so that the photon detection performance will meet minimum requirements for the whole life of the experiment.


Three photon collector designs are present in \dword{pdsp}: \num{29} double-shift guides, \num{29} dip-coated guides, and two \dword{sarapu} arrays. 
The TPC provides precise reconstruction in \threed of the track of any ionizing event inside the active volume and matching the track with the associated light signal will enable an accurate comparison of the relative detection efficiencies of the different \dword{pd} modules. 
The large number of modules and independent channels that record each event can be used to constrain the propagation parameters of the \lar that regulate \dword{vuv} light propagation in the simulation and are poorly determined in the literature. %such as Rayleigh scattering length
In principle, absolute calculations are also possible using \dword{mc} simulations.
The absolute precision of this approach may be limited by the precision of the constraints on the parameters, but will result in a consistent simulation constrained by actual measurements. 

%LMM I think I outlined the solution in the paragraph now. 
%\fixme{we need to describe how much this limitation affects our confidence in the overall design}
%A plan will be developed to address this limitation.
%\fixme{we need to describe the plan...}

%\fixme{Is there a standard reference for ProtoDUNE?} % Yes \cite{Abi:2017aow}

The first large-scale implementation of an \dword{sarapu} module, in \dword{pdsp}, is composed of an array of sixteen ARAPUCA cells each one acting as an individual detector element. 
%The \dword{pdsp} ARAPUCA design collects light from one side of the box through an optical window formed  
%by a dichroic filter deposited with a layer of pTP\footnote{p-TerPhenyl,  supplier: Sigma-Aldrich\textregistered.}
%wavelength shifter on the external surface that shifts the incident \dword{vuv} light to a near-visible frequency able to pass through the filter plate to the interior of the box.  
The \dword{sarapu} design collects light from one side of the box through an optical window formed by a dichroic filter deposited with a layer of pTP\footnote{p-TerPhenyl, supplier: Sigma-Aldrich\textregistered.}
% https://www.sigmaaldrich.com/catalog/substance/pterphenyl230309294411.} 
wavelength shifter on the external surface. This shifts the incident \dword{vuv} light to a near-visible frequency that is able to pass through the filter plate to the interior of the box.  

In the \dword{pdsp} version of \dword{sarapu}, the inner surface of the box opposite the window houses an array of \dwords{sipm} that covers a small fraction of the area of the window (2.8-5.6\%), surrounded by a foil of a highly reflective material coated with a second wavelength shifter, \dword{tpb}. The \dword{tpb} converts the light passing through the filter to a wavelength that is reflected by the filter. It has been shown in simulation and in prototypes that a large fraction of these trapped photons, reflecting from the filter and the lined walls of the box, will eventually fall on a \dword{sipm} and be detected.

%Two arrays of \dword{sarapu} modules have been operated inside \dword{pdsp} to test the devices in a large-scale experimental environment and allow direct comparison of their performance with the light guide designs. 
%\fixme{Describe location of the arapucas.}
The \dword{pdsp} holds two \dword{sarapu} arrays. The first is installed in the \dword{apa} \#3 in the fourth position from the top.  This is near the level the beam particles enter this drift volume in order to illuminate it and the surrounding modules with a significant amount of light from each beam particle interaction.
The second  is installed in the \dword{apa} \#6 in the 6th position from the top.  This module does not see significant light from beam events, but it is also surrounded by light guide modules, signals from these will help to constrain the light that the module is detecting from each interaction.
 
\fixme{ Figure needs to be better, maybe different color histos on the same scale, could make it a smaller picture with the same information displayed, but with legible axes etc. } 
\begin{dunefigure}[Raw pulse height detected in ARAPUCA for different energies.]{fig:arapuca_beamph}
{Integrated raw pulse height in ADC count summed from all channels in an ARAPUCA module exposed to the beam at various beam energies. No selection on particle types has been performed.}
	\includegraphics[angle=0,width=\columnwidth]{pds-ARAPUCA-beamPH.pdf}
\end{dunefigure}

Each \dword{pdsp} \dword{sarapu} module array is composed of sixteen cells where each cell is an \dword{sarapu} box with dimensions of \SI{8}{cm}$\times$\SI{10}{cm}; half of the cells have twelve \dwords{sipm} installed on the bottom side of the cell and  half have six \dwords{sipm}. The \dwords{sipm} have active dimensions \SI{0.6}{cm}$\times$\SI{0.6}{cm} and account for 5.6\% (\num{12} \dwords{sipm}) or \num{2.8}\% (\num{6} \dwords{sipm}) of the area of the window (\SI{7.8}{cm}$\times$\SI{9.8}{cm}).
The \dwords{sipm}  are passively ganged together, so that only one readout channel is needed for each ARAPUCA grouping of \num{12} \dwords{sipm} (the boxes with six \dwords{sipm} are ganged together to form \num{12}-\dword{sipm} units) so a total of \num{12} channels is required per array. 
%Studies are underway to investigate active ganging that would permit combining signals from multiple boxes, as required to reduce the number of electronics channels and cables under the working assumption that the \single \dword{pds} is restricted to four readout channels per \dword{pd} module. 
The total width of a module is \SI{9.6}{cm}, while the active width of an \dword{sarapu} is \SI{7.8}{cm}, the length is the same as the light guide modules (\SI{210}{cm})\footnote{Since \dword{pdsp} was constructed, the slot opening in the \dword{apa} opening for \dword{pd} module installation has been enlarged allowing for a module with larger collection area}.
An \dword{sarapu} array during assembly is shown in Figure~\ref{fig:sarapuca_array_prod} during the assembly; the array installed in \dword{pdsp} is shown in Figure~\ref{fig:arapuca-protodune}. If the \dword{sarapu} cells achieved the same detection efficiency as earlier prototypes (1.8\%), the effective area of an \dword{sarapu} module will be approximately \SI{23}{cm$^2$}.

\begin{dunefigure}[Photo of ARAPUCA module prototype assembly.]{fig:sarapuca_array_prod}
{ProtoDUNE ARAPUCA module being assembled in a class 100,000 clean area.  Front face of assembled module (left) shows the 16 coated dichroic filter plates.  Assembly photos show the reflective rear side (top right) and inner coated surface (right bottom) of Vikuiti\texttrademark\ reflective foils.  Note the cutouts in foil for \dword{mppc} active area.}
	\includegraphics[angle=90,width=0.75\columnwidth]{pds-pdune-arapuca-assby}
\end{dunefigure}

%\begin{dunefigure}[\dword{sarapu} array during assembly.]{fig:sarapuca_array_prod}
%{\dword{sarapu} array during assembly.}
%{\dword{sarapu} array for \dword{pdsp} during assembly. (\dwords{sipm} are visible in the sixteen cells before the installation of reflecting foils, coated filter windows, and readout cabling) (Left); ARAPUCA array in \dword{pdsp} (Right).} 
%\includegraphics[height=8cm]{pds-arpk-apa3_pd.jpg} 
%\end{dunefigure}

\begin{dunefigure}[Full-scale \dword{sarapu} array installed in \dword{pdsp}.]{fig:arapuca-protodune}
{\dword{sarapu} array installed in \dword{pdsp}.} 
\includegraphics[height=7cm]{pds-arpk-apa3_pd.jpg} 
\end{dunefigure}

%\begin{dunefigure}[\dword{sarapu} array in \dword{pdsp}.]{fig:arapuca_array}
%{ARAPUCA array in \dword{pdsp}dword{pdsp}.} 	
%\includegraphics[height=8cm]{pds-arpk-apa3_pd.jpg} 
%\end{dunefigure}

\subsubsection{\dword{pdsp} Electronics}
\label{sec:ssp-protodune-electronics}

%\metainfo{Content: Djurcic}
%\fixme{Need a good reference/citation for \dword{ssp} - is there one? Is it described ( well enough) in the ProtoDUNE TDR?}


%As already described above, the \dword{pd} development has matured to the point where three different \dword{pd} light-collector designs are prepared and deployed with \dword{pdsp} experiment at CERN. The experiment started operation in September 2018 and collected samples of test-beam data, cosmic muons events, and the data from the \dword{pd} calibration and monitoring system. The analysis is underway to provide validation of \dword{pd} readout and light-collection designs, and to inform design and selection of above concepts for DUNE Far Detector.


The electronics readout of the \dword{pdsp} \dword{pds} was provided by a sophisticated custom-designed system, the \dword{sipm} Signal Processor (\dwords{ssp})~\citedocdb{3126} that was also used extensively for most of the earlier prototyping studies and photosensor testing.
A passive signal summing scheme with three \dwords{sipm} summed together was chosen for the light guides (four \dword{ssp} channels for each bar) and groups of twelve \dwords{sipm} are passively summed for the two ARAPUCA modules (12 \dwords{ssp} channels per module).
The unamplified analog signals from the \dwords{sipm} are transmitted directly to outside the cryostat for processing and digitization. A custom module, 
called the \dwords{sipm} Signal Processor (\dword{ssp}), receives the \dwords{sipm} signals outside the cryostat. An \dword{ssp} consists of 12 readout channels packaged in 
a self-contained 1U module. The ProtoDUNE \dword{pds} is read out with the total of 288 \dwords{ssp} channels. 
Twenty-four custom \dword{sipm} Signal Processor (\dword{ssp}) units were produced to read out the 58 light guide and two ARAPUCAs photon collector arrays.
%\dword{ssp} at the top of \dword{pdsp} cryostat are shown in Figure~\ref{fig:fig-pds-readout}, right. 
Each channel contains a fully-differential voltage amplifier and a \num{14}-bit, \num{150}-MSPS \dword{adc} that digitizes the \dword{sipm} signal waveforms.

%A dedicated \dword{pd} readout system with a high-performant electronics \dword{fe} was developed for the \dword{pdsp} experiment, as schematically presented in Figure~\ref{fig:fig-pds-readout}, left. 
%For \dword{pdsp}, a passive signal summing scheme with three \dwords{sipm} summed together was chosen for the light guides (4 \dword{ssp} channels for each bar) and groups of twelve \dwords{sipm} are passively summed for the two ARAPUCA modules (12 \dwords{ssp} channels per module).
%The un-amplified analog signals from the \dwords{sipm} are transmitted directly to outside the cryostat for processing and digitization. A custom module, called the \dwords{sipm} Signal Processor (\dword{ssp}), receives the \dwords{sipm} signals outside the cryostat. An \dword{ssp} consists of 12 readout channels packaged in a self-contained 1U module. ProtoDUNE \dword{pds} is red-out with the total of 288 \dwords{ssp} channels. 
%Twenty-four custom \dword{sipm} Signal Processor (\dword{ssp}) units were produced to read out the 58 light guide and 2 ARAPUCAs photon collector arrays.
%\dword{ssp} at the top of \dword{pdsp} cryostat are shown in Figure~\ref{fig:fig-pds-readout}, right. Each channel contains a fully-differential voltage amplifier and a \num{14}-bit, \num{150}-MSPS \dword{adc} that digitizes the \dword{sipm} signal waveforms.
%The \dword{fe} amplifier is configured as fully-differential, and receives the \dwords{sipm} signals into a termination resistor that matches the characteristic impedance of the signal cable. 

%\begin{dunefigure}[\dword{pdsp} \dword{pd} module readout.]
% {fig:fig-pds-readout}
% {Block diagram of the ProtoDUNE \dword{pd} readout module (left figure). Photon-detector \dword{fe} electronics operational 
%at ProtoDUNE (right figure).}
%\includegraphics[angle=0,width=8.4cm,height=6cm]{pds-fig-e-3.png}
%\includegraphics[angle=0,width=8.4cm,height=6cm]{protodune_readout.png}
%\end{dunefigure}
\fixme{The triggering here is not really correct for how it was done at protodune.  The module continually digitizes (always) and stores waveforms above threshold in a circular buffer.  On an external trigger, it will send a waveform and header for the time period of the trigger and any stored waveforms or subsequent waveforms above threshold for the configured readout window. For protoDUNE it always send header and waveform, and in all cases if it sends a waveform it will send a header. }

In the standard mode of operation, the module performs waveform capture, using either an external or internal trigger. In the latter case the module self-triggers to capture only waveforms with an amplitude greater than a specified threshold. In \dword{pdsp} the photon readout is configured to read waveforms when triggered by a beam event, or to provide header information when self-triggered by cosmic muons.
The header portion summarizes pulse amplitude, integrated charge, and time-stamp information of events. The \dword{ssp} for \dword{pdsp} uses \si{Gb} Ethernet 
communication implemented over an optical interface. The \SI{1}{Gb/s} Ethernet supports full TCP/IP protocol.  Advanced on-board FPGAs are utilized for signal processing (ARTIX-7) and trigger/timing communication (Zynq).

%The module includes a separate 12-bit high-voltage DAC for each channel to provide bias to each \dword{sipm}. Currently there are two DAC options:  one with a voltage range of 0V to 30V, used with 11 the sensL \dwords{sipm} (17 of the 24 \dword{ssp} units); and the other with a range 0V to 60V for use with the 12 Hamamatsu \dwords{mppc} (seven of the 24 \dword{ssp} units).
%The \dword{ssp}  provides a trigger output signal from internal discriminators in firmware based on programmable coincidence logic, with a standard ST fiber interface to the central trigger board (CTB).
%Input signals are provided to CTB from the beam instrumentation, the \dwords{ssp}, and the beam TOF system. The CTB receives timing information from the \dword{pdsp} timing system and the CTB trigger inputs are distributed to the experiment via the timing system.
%To that end, the \dword{ssp} implements the timing receiver/transmitter endpoint hardware to receive trigger inputs and clock signals from the timing system.

%Other readout components, aside from the \dword{fe} electronics (\dwords{ssp}) included design and implementation of signal feed-through, selection and fabrication.

\subsubsection{\dword{pdsp} \dword{pds} Measurements}
\label{sec:protodune-results}

\fixme {lmm ProtoDUNE analysis strategy outline.   These enumerated section below will get incorporated into the text as strategy becomes clearer as analysis progresses}
\fixme{rjw 12/2/18 For now just I think it is better to have just a few words and some of the most relevant results, ideally with a reference to a separate document with all the details :-) (I kept your original text in the file)}

\dword{pdsp} provides several distinct sets of data for understanding \dword{pds} performance.  
There are data sets with cosmic triggering, randomly, or in coincidence the CRT modules.
There are beam data sets with triggers determined by the beam instrumentation.
There are calibration module data sets, with triggers in coincidence or free running with a configured light pulse. The cosmic ray data and beam data include the TPC and/or CRT information to provide strong constraints on the particle trajectory that will allow us to determine the exposure of each module on an event-by-event basis.  The additional data sets are useful for calibration, and understanding background rates due to other processes.  Beam events also appear to be useful for calibration, as they provide very low level light signals to the modules in the beam-left segment of \dword{pdsp} where the beam particles were not seen directly.

The analysis is ongoing but here we summarize some relevant initial results:
\fixme{Summarize the key results from ProtoDUNE with emphasis on ARAPUCA and \dword{mppc}.  We should also summarize lightguide  Denver points out, for \dword{xarapu} simulation the results from the IU bars feed into the simulation of the filter/light-trapping from ARAPUCA.}
%\fixme{{\bf For this draft } it would be good to have at least some plots indicating that we see light in the ARAPUCAs and there are no obvious major problems.}
% included ARAPUCA above, will work on a figure as indicated below.

\fixme{Include a figure with calibration module pulses seen in different modules, highlighting ARAPUCA and \dword{mppc} channel response.  Maybe just \dword{apa} 5 and 6 with just a few SensL devices, but both an ARAPUCA and many \dword{mppc} bars}

%\begin{itemize}
%\item Random Cosmic events - broad distribution mostly vertical
%\item CRT Cosmic events - broad x-y dist mostly horizontal
%\item Beam events - well localized, controlled energy
%\end{itemize}

%There are 3 sets of data with distinct features, in all cases the TPC and/or CRT will provide strong constraint on the detailed particle trajectory for measuring exposure of each module on an event by event basis.
%\begin{itemize}
%\item Random Cosmic events - broad distribution mostly vertical
%\item CRT Cosmic events - broad x-y dist mostly horizontal
%\item Beam events - well localized, controlled energy
%\end{itemize}
%Utilize these sources for their unique features.  Initially one can simply select similar events for all modules.  For example, only select tracks that are within 50cm of the cathode, and vertical.  This will strongly limit any dependence on the modeling of light propagation, as the distance will be well controlled.  In this case, relative comparisons could be performed with minimal dependence on any correction factors, geometric or propagation dependent.

%\begin{enumerate}
%\item Calibrate each type of module/sensor (depends on sensor/ganging multiple)  Calibration module may play a large role here, but uncertain at the moment, most have single PE peaks already for calibration
%\item Look at Cosmic pulse height or integral  distributions (can be in parallel)
%\subitem See if stable for given module run to run, and within runs (beginning and end, ...)
%\subitem If yes, see how much modules of same type/readout vary by location 1vs3vs5vs7vs9, and 2vs4vs6vs8vs10... where technology is the same
%\subitem if not well defined, limit trajectories until it is.
%\subitem    if slow enough change, use to predict light in each module (something like distribution peak or mean or 1/2 maximum vs. position) 
%\subitem compare based on module types.  (only one or 2 measurements for ARAPUCA, so they would just be relative comparison of another type of module at that location.
%\item  Using tracking from CRT and/or TPC reconstruct event by event light distribution, and use to determine model parameters, photons/cm, Rayleigh scattering length, ...  (will take a fair amount of data, I suspect)
%\subitem using above, should determine pe/MeV (or pe/cm) for each type of module and at each location in the detector.
%\subitem Beam events can also be used to calculate absolute parameters, but only after determining propagation parameters of \lar 
%\end{enumerate}


\subsection{Single Cell \dword{xarapu} Measurements}
\label{sec:xarapuca-unicamp}
%\metainfo{Content: Segreto/Machado}
\fixme{First draft by Warner.  Should be edited and corrected by Segreto/Machado}

The first tests of an \dword{xarapu} cell were made in a cryostat at Unicamp in November 2018. This test cell structure was designed to allow for operation as an \dword{sarapu} or an \dword{xarapu}, with single- or double-sided readout in both configurations.  This flexibility will allow relative and absolute measurements of performance in the same cryostat and so provide a crucial step to validating the baseline design selection.

Building on the experience with the \dword{pdsp} prototypes, the frames for the test cell were fabricated from FR-4 G-10 in a configuration very similar to that planned for the \dword{fd}, but with some small modifications necessitated by the requirements for holding a single-window. The overall dimensions of the cell are \SI{123}{cm}$\times$\SI{100}{cm}$\times$\SI{15.6}{cm}. Figure~\ref{fig:xarapuca-cell} shows an exploded design drawing and the completed cell. 

\begin{dunefigure}[\dword{xarapu} test cell.]{fig:xarapuca-cell}
{\dword{xarapu} test cell:  Assembled cell (left); exploded model(right).  Note that exploded components can be duplicated on the back side (not shown) for double-sided test cell} 
	\includegraphics[angle=270, origin=c, width=7cm]{pds-x-arapuca-test-cell-image-2}
	\includegraphics[height=6cm]{pds-unicamp-single-cell-arapuca-exploded}
\end{dunefigure}

The dichroic window(s) for the prototype are the same size as one of the six windows in a \dword{fd}-design ARAPUCA supercell: \SI{10.0}{cm}$\times$\SI{7.8}{cm}.  The pTP coating on the exterior of the filter plate was deposited using an in-house deposition system at Unicamp. Figure~\ref{fig:xarapuca-plates} shows photos of two coated plates and the film coating facility. 

The Wavelength shifting plate in the \dword{xarapu} configuration is made from Eljen EJ-286 blue WLS plate, with dimensions \SI{9.3}{cm}$\times$\SI{7.8}{cm}$\times$\SI{0.3}{cm}.  The side walls of the test cell are lined with Vikuiti reflector, with cutouts at the positions of the photosensors.

\begin{dunefigure}[Coated dichroic filters and vacuum coating system.]{fig:xarapuca-plates}
{Coated dichroic filter plates (left) and Unicamp thin film coating facility (right).} 
	\includegraphics[angle=90, height=6cm]{pds-coated-filter-plates-unicamp}\quad
	\includegraphics[height=7cm]{pds-unicamp-coating-machine}
\end{dunefigure}

The photosensors in the test cell are of the current baseline design:  \SI{0.6}{cm}$\times$\SI{0.6}{cm} Hamamatsu S13360-6050VE \dwords{mppc}.  The photosensors are arranged in the same configuration as in the baseline design, with four \dwords{mppc} (passively ganged) mounted to two sides of the test cell, with positioning relative to the WLS plates and dichroic filters identical to the baseline design.  In a departure from the baseline design, the two passively ganged groups of 4 \dwords{mppc} are read out separately--no active ganging circuit is implemented for these tests.

The test cell is immersed in an \dword{lar} bath in a two-liter test dewar.  The test cell is held in a mounting structure together with the alpha source used to generate scintillation light. Figure~\ref{fig:xarapu-teststand} shows photos of the test cryostat and the test cell in the support structure; the alpha source holder is visible through the windows. 

\begin{dunefigure}[\dword{xarapu} test stand.]{fig:xarapu-teststand}
{Test cryostat at Unicamp (left) and \dword{xarapu} test cell mounting structure (right).  Note the alpha test source in holder.} 
	\includegraphics[height=6cm]{pds-unicamp-test-cryostat} \quad
	\includegraphics[height=7cm]{pds-unicamp-sample-cell-holder}
\end{dunefigure}  

As in the ARAPUCA proof-of-principle tests detailed in Section~\ref{sec:proof-principle}, scintillation light emission is produced by an alpha source (U-Al alloy) in the form of a metallic foil, with alpha particle emission of \SI{4.267}{MeV} mounted in front of the \dword{xarapu} immersed in \lar. Signals are read out through an Aquiris\footnote{Aquiris High-Speed Digitizer products; http://www.acqiris.com/.} PCI board.

Monte Carlo simulations of the predicted performance of this detector have been made, and will be compared to results observed during testing
\fixme{references to figures?}
\begin{dunefigure}[MC Simulation of \dword{xarapu} test cell.]{fig:xarapu-test-cell-mc-data}
{MC Simulation of \dword{xarapu} test cell: Alpha spectrum (left) and number of photons per alpha (right).} 
	\includegraphics[height=10cm]{pds-x-arapuca-test-cell-preliminary-mc-results}
    \vspace{-3cm}
\fixme{add MC results plots--  DWW screenshot added.  Need proper plots from Ana and Ettore}
\end{dunefigure}

Initial testing of the test cell in double-sided \dword{xarapu} mode began in November 2018.  Initial results are promising:  scintillation light from the alpha source has been detected at very encouraging levels, and similar light levels were detected from both sides of the detector.


\begin{dunefigure}[Data from \dword{xarapu} test cell.]{fig:xarapu-test-cell-data}
{Data from \dword{xarapu} test cell: xxxx(left) and number of photons per alpha (right).} 
%	\includegraphics[height=10cm]{pds-x-arapuca-test-cell-preliminary-mc-results}
\fixme{Eagerly awaiting  plots from Ana and Ettore!}
\end{dunefigure}


[Note in 1st draft:  In the coming weeks, absolute detection efficiencies for the double sided readout \dword{xarapu} configuration will be extracted using data collected in these initial tests.  Additional other tests will be performed, including single sided \dword{xarapu} tests (with a Vikuiti reflective foil installed on the non-viewing side of the test cell in place of one of the dichroic window), as well as single- and double-sided \dword{sarapu} configurations, allowing for a complete suite of comparisons of detector performance.]



% 
%%%%%% rjw edits, ZD edits

\subsection{ICEBERG Test Stand}
\label{sec:iceberg-teststand}
%\fixme{summary of what PD is going into ICEBERG needed and schedule for when we might expect \dword{xarapu} results}
%\fixme{Description of ICEBERG (described in other chapters? electronics?) and the arapucas that will be in.}
%fixme{Iceberg section added here.  Comments?}
%\fixme{Integrate the mention of Mu2e electronics text from Josh pasted at the end of this section.}

The ICEBERG test-stand is a small-scale TPC using reduced-size \dword{fd} \dword{apa} and cathode designs constructed primarily to provide a platform for DUNE \dword{ce} testing at \dword{fnal}. 
The test stand consists of an approximately \SI{94.7}{cm} $\times$ \SI{79.9}{cm} anode plane assembly, with an approximately \SI{30}{cm} drift length from a cathode plane on each side.  
It can accommodate up to two almost 1/2-length \dword{pd}\footnote{In order to use existing \dword{apa} components from \dword{pdsp} as a cost-saving measure, however, slight modifications were made to the \dword{apa} frame resulting in \dword{pd} modules \SI{50}{mm} shorter than final modules will be, requiring slight modifications to the \dword{pd} module design.} in a mounting structure nearly identical to the final DUNE FD configuration, allowing for testing of \dword{pd} prototype performance, electrical connections, and interfaces with the \dword{ce} and \dword{apa} systems. 
In addition, it will be used to allow comparisons between Mu2e-based warm electronics and \dword{pdsp} \dword{ssp} system, as well as testing newer versions of photosensors active ganging circuit designs.

%The photos are nice but not directly germane to the PD tests.
%Figure~\ref{fig:fig-iceberg} shows the ICEBERG TPC. 

The ICEBERG facility will enable the primary validation of the \dword{xarapu} design prior to a full-scale test envisioned at a future \dword{pdsp} run in 2020. 
%It is envisioned that up to three test cycles will be performed prior to the DUNE technical design report submission.
At least two test campaigns are planned for the ICEBERG TPC with PD modules:  An initial test run in December 2018 incorporated one full-length \dword{sarapu} supercell and one full-length \dword{xarapu} supercell.  Both of these supercells incorporate single-sided readout, allowing comparisons of \dword{sarapu} and \dword{xarapu} performance to summer and fall 2018 \dword{pdsp} measurements.

%\begin{dunefigure}[zd more ICEBERG figures all together.]
%{fig:fig-iceberg}
%{ICEBERG cryostat (left figure). ICEBERG signal feed-through (right figure). }
%\includegraphics[angle=0,width=7cm,height=5cm]{ICEBERG_0.png}
%\includegraphics[angle=0,width=7cm,height=5cm]{ICEBERG.png}
%\end{dunefigure}

 \begin{dunefigure}[ICEBERG TPC Model and assembled \dword{apa}.]
 {fig:fig-pds-iceberg-tpc}
 {Software solid model of ICEBERG TPC (left), and assembled ICEBERG \dword{apa} (right).  Note the \dword{pd} module mounting rails, which are vertical in this image but horizontal in operation. The centrally-mounted \dword{apa} allows for testing of double-sided readout photon detector modules.}
\includegraphics[angle=0,width=8.4cm,height=6cm]{pds-iceberg-tpc}
\includegraphics[angle=0,width=8.4cm,height=6cm]{pds-iceberg-apa.pdf}
\end{dunefigure}

%Since the \dword{apa} for the ICEBERG test stand is nearly half the width of a full far-detector \dword{apa} frame, including full-size PD insertion slots (\SI{136}{mm}$\times$\SI{25}{mm}) and electrical connections.  


\begin{dunefigure}[Single-supercell ICEBERG PD module.]
 {fig:fig-pds-iceberg-tpc}
 {Software solid model of single-supercell ICEBERG PD module (left) and fabricated components during assembly (right).  Note connector board (green) in right photo is mounted to the \dword{apa} frame prior to wire wrapping.}
\includegraphics[angle=0,width=8.4cm,height=6cm]{pds-iceberg-run-1-module.pdf}
\includegraphics[angle=0,width=8.4cm,height=6cm]{graphics/pds-iceberg-module-assembly-photo.pdf}
\end{dunefigure}

A second campaign is planned for winter/spring 2019, incorporating two \dword{sarapu} and two \dword{xarapu} supercells, partially occluded in the frame due to the limitations in \dword{apa} size mentioned above.  One supercell of each kind will be single-sided and one double-sided, allowing for additional comparisons of \dword{pd} technologies.

Along with the \dword{pds} technology studies, the test stand will  provide further testing and validation of the \dword{pds} Mu2e-based electronics system, including a side-by-side comparison with the \dword{pdsp} \dword{ssp} electronics readout. In addition, concurrent data taking with the TPC and light collection system will allow us to study TPC-induced noise on the \dword{pd}, \dword{pd}-induced noise on the TPC, grounding scheme configuration, controller-\dword{daq} and controller-\dword{feb} interfaces, bandwidth and rates issues, online and offline \dword{pd}-TPC interfaces, zero-suppression techniques, firmware development, accepting and producing triggers, and, in general, will inform possible upgrade paths for the system. 


%%%%%%%%%%%%%%%%%%%%%%%%%%%%%%%%%%%%%%%%%%%%%%%%
% Provided by dj 11/26/18
\subsection{Calibration and Monitoring}
\label{sec:fdsp-pd-validation-candm}

%\metainfo{Content: Djurcic}

All major components of the \dword{spmod} \dword{pds} calibration and monitoring system have been designed, fabricated, tested, and operated in \dword{pdsp}.
Figure~\ref{fig:pds_calmon_fig2} shows the hardware components of the system.

 \begin{dunefigure}[\dword{pdsp} UV calibration and monitoring system.]
 {fig:pds_calmon_fig2}
 {The figures show the hardware components of the \dword{pdsp} calibration and monitoring system.}
 \includegraphics[angle=0,width=11.4cm,height=9cm]{graphics/pds-calmon-fig2.png}
\end{dunefigure}

Although at a longer wavelength than \lar scintillation light, the UV light from the system exercises the full chain of measurement steps initiated by a physics event in the \dword{detmodule}, starting from the wavelength %-shifter 
conversion, photon capture in the \dword{xarapu}, photon %sensor 
detection and the \dword{fe} electronics readout.
Figure~\ref{fig:pds_calmon_fig3} shows typical double waveforms recorded by an \dword{pdsp} \dword{ssp} module as a response to calibration system
light pulses illuminating %one of ARAPUCA 
an \dword{sarapu} channel. Here, the calibration modules and \dwords{ssp} are triggered externally through the trigger and timing system to efficiently collect calibration light pulses  and minimize rate of in-time cosmic ray muons.

 \begin{dunefigure}[\dword{pdsp} ARAPUCA response to UV calibration and monitoring system.]
 {fig:pds_calmon_fig3}
 {Double waveforms recorded by \dword{pdsp} \dword{ssp} module as a response to calibration system light pulses collected by an \dword{sarapu} channel.}
%\fixme{Zelimir: Can the plot be made with a smaller y-axis scale to reduce the white space?}
\includegraphics[height=4cm]{graphics/pds-calmon-example.png}
\end{dunefigure}

The \dword{pdsp} data set has been collected and the data analysis is underway.
Goals of the analysis are to verify that the \dword{cpa} includes % there is 
an optimal distribution of light diffusers for the \dword{spmod};  %at the \dword{cpa} for DUNE; use the system 
to evaluate gain and timing resolution; to perform relative comparisons of photon channels;
and to characterize and monitor stability of the \dword{pds} over the duration of \dword{pdsp}. The results will inform the design of an optimal
\dword{pds} calibration and monitoring plan for the \dword{spmod}.%DUNE.



%\subsection{Other Key Measurements}

%The consortium has made use of many facilities to develop the \dword{pds} photon collector and electronics and to test photosensors candidates that led up to the larger tests at \dword{pdsp} and ICEBERG. We summarize here some of the most significant measurements.

%that will allow testing of smaller scale prototypes of the modules (or sections of them) and facilities for precision optical measurements and cryogenic testing of photosensors.
%These include: Fermilab, Colorado State University and Universidade Estadual de Campinas, Indiana University, Northern Illinois University, University of Iowa, Syracuse University, Institute of Physics in Prague, INFN Milano Bicocca, and INFN Bologna.


%\subsubsection{CERN cold box}
%%% rjw 11/23/18 superseded by the actual protodune operation?
%A critical issue for large experiments are the interfaces between the subsystems. \dword{pd} modules and interfaces with the \dword{apa} system and cold electronics will be conducted using cryogenic gaseous nitrogen in cold box studies at CERN, using a test stand developed for testing of \dword{pdsp} components prior to installation into the detector.  A full-scale \dword{pdsp} \dword{apa} has been fabricated,  and will be instrumented with \dword{ce} and \dwords{pd}, allowing the interfaces to be carefully studied.


%%%%%%%%%%%%%%%%%%%%%%%%%%%%%%%%%%%%%%%%%%%%%%%%%%%%%%%%%%%%%%%%%%%%%%%%%%%%%%%%%%%%%%%%%%
% rjw 11/25/18 Add the Simulation section here as a part of the design validation
%%%%%%%%%%%%%%%%%%%%%%%%%%%%%%%%%%%%%%%%%%%%%%%%%%%%%%%%%%%%%%%
% rjw 11/25/18 Make subsection of the Validation section 
% 1/9/19 Move back to just after the Introduction.
% rjw 4/7/19 move to appendix per TB
%
\subsection{Simulation}
\label{sec:fdsp-pd-simphys}
%\metainfo{(Length: \dword{tdr}=50 pages, TP=20 pages)}
%\metainfo{\color{blue} Content: Conveners}
% Provided by Alex H. 15mar18
%\metainfo{Content: Himmel}

%Content update by AH nov 2018
%edits by rjw nov 2018

\fixme{AIH: LBNC "Overall for Section 1.1, I think the technical requirements are clearly explained, but the driving motivations are not."}
\forlbnc{The requirements all flow down from the highest level physics goals of the experiment. However, it's somewhat difficult to attach each detector property directly to the physics goals, so these ``physics deliverables'' act as an intermediary. In this chapter we take these as given, and show how we meet them, without needing to get into the detailed physics of different nucleon decay channels or supernova models here. To help make this connection more explicit, we have added references to relevant sections of the physics TDR where these requirements are defined.}

The broad performance specifications for the \dword{pds} are determined by a series of physics deliverables addressing the major physics goals of DUNE: nucleon decay searches, supernova burst neutrinos, and beam neutrinos. Detailed subdetector specifications, such as light yield of the light collectors, are determined using a full simulation, reconstruction, and analysis chain developed for the \larsoft framework. 
%\fixme{I'm confused: the goals determine the needed sim/reco/analysis work which determines the requirements, which in turn determine the deliverables? Or the goals determine the deliverables, which determine the sim/reco/analysis work, which determines the requirements? Anne}
%The major physics goals of DUNE -- nucleon decay searches, \dword{snb} neutrinos, and beam neutrinos -- determine a series of physics deliverables.  The \dword{pds} requirements flow from the physics needs, determined using a full simulation, reconstruction, and analysis chain developed for the \larsoft framework. 

%The goal is to evaluate the performance in physics deliverables for each of the photon collector designs under consideration. The metrics evaluated will include efficiency for determining the time of the event ($t_0$), timing resolution, and calorimetric energy resolution for three physics samples: \dword{snb} neutrinos, nucleon decay events, %\footnote{The most relevant sample is actually the \emph{background} to nucleon decay events. However, efficiently simulating background that can mimic nucleon decays is challenging since they can be quite rare topologies, so it is easier to simulate the nucleon decay signal which should be representative of the background.}, 
%and beam neutrinos. However, the development of analysis tools to take advantage of this full simulation chain is fairly recent, so this proposal will only include one test case: $t_0$-finding efficiency for \dword{snb} neutrinos versus the effective area of the photon collectors (see Section~\ref{sssec:photoncollectors}).


\subsubsection{Simulation and Reconstruction Steps} 
\label{subsec:fdsp-pd-simphys-sim}

The first step in the simulation specific to the \dword{pds} is the simulation of the production of light and its transport within the volume to the \dwords{pd}. Argon is a strong scintillator, producing \SI{24000}{$\gamma$s/MeV} at our nominal drift field. Even accounting for the efficiency of the \dwords{pd}, it is prohibitive to simulate every optical photon with \dword{geant4} in every event. So, prior to the full event simulation, the detector volume is voxelized and many photons are produced in each voxel. The fraction of photons from each voxel reaching each photosensor is called the visibility, and these visibilities are recorded in a 4-dimensional library.
%rjw 3/19 removed per LBNC (akin to the photon maps used in the \dword{dpmod} simulation described in \voltitledp~Chapter~5).
%\fixme{This reference to DP chapter 5 is hardwired - check this is the correct chapter number.}
This library includes Rayleigh scattering length ($\lambda=$ \SI{60}{cm}\cite{Grace:2015yta}), absorption length ($\lambda=$ \SI{20}{m}), and the measured collection efficiency versus position of the double-shift light-guide bars. There is significant uncertainty on the scattering length in the literature, so the value chosen is conservatively chosen at the low end of those reported. With these optical properties, there is a factor of 20 difference in total amount of light collected between events right in front of the photon detectors and those on the far side of the drift volume \SI{3.6}{m} away. When a particle is simulated, at each step it produces charge and light. The light produced is distributed onto the various \dwords{pd} using the photon library as a look-up table and the 30\% early (\SI{6}{ns}) plus 70\% late (\SI{1.5}{$\mu$s}) scintillation time constants are applied. Transport time of the light through the \lar is not currently simulated, but is under development. It is not expected to make a significant difference in the studies presented here.

\fixme{AIH: (last sentence previous para) LBNC  no simulation yet. Can we give a projected impact??}

\forlbnc{The more precise timing resolution is not expected to make much impact on the physics shown here. Having it may open up some new possible applications (nucleon decay identification using fine time structure, for example), but those types of applications are still quite speculative so were not included here.}

The second step is the simulation of the sensor and electronics response. For the studies shown here, the SensL \dword{sipm} and \dword{sipm} signal processor (\dword{ssp}) readout electronics used for \dword{pd} development and in \dword{pdsp} is assumed (see Section~\ref{sec:fdsp-pd-pde}). However, a range of \dword{s/n} and dark rates are considered in order to set requirements on the needed performance of the electronics.
%Waveforms are produced on each channel by adding an \dword{sipm} single-\phel response shape for each true photon. In addition, other characteristics of the \dword{sipm} are included such as dark noise, and crosstalk, based on data from device measurements. 
%) From Alex 4/17/18 - We do include afterpulsing as well, at rates based on the sensl sipm's tested in Hawaii. You can just add it to the list of things included in the electronics simulation. 
Crosstalk (where a second cell avalanches when a neighbor is struck by a photon generated internal to the silicon) is introduced by adding a second \phel \num{16.5}\% of the time when an initial \phel is added to the waveform. Additional uncorrelated random noise is added to the waveform with an RMS of %\SI{2.6}{ADC} (or approximately 
\SI{0.1}{\phel{}}. The response of the \dword{ssp} self-triggering algorithm, based on a leading-edge discriminator, is then simulated to determine if and when a \SI{7.8}{$\mu$s} waveform will be read out, or in the case of the simulation, %it will be 
stored and passed on for later processing.

\fixme{AIH: LBNC "where do these numbers come from (for added cross-talk, noise)? Are they “typical values”  - conservative values?"}
\forlbnc{These are typical values based on various lab measurements of devices, though they may not all be from the exact same device. However, studies of a wide range of S/N and dark rates showed little impact on physics, though low S/N/ does typically lead to higher data rates.}

The third step is reconstruction, which proceeds in three stages. The first is a ``hit finding'' algorithm that searches for peaks on individual waveforms channel-by-channel, identifying the time (based on the time of the first peak) and the total amount of light collected (based on the integral until the hit goes back below threshold). The second step is a ``flash finding'' algorithm that searches for coincident hits across multiple channels. All the coincident light is collected into a single object that has an associated time (the earliest hit), an amount of light (summed from all the hits), and a position on the plane of the \dwords{apa} ($y$-$z$) that is a weighted average of the positions of the photon collectors with hits in the flash. %\footnote{Currently, the flash reconstruction does not consider the positions of the hits, only their times. This will need to be updated in the future when we simulate the full-sized \dword{spmod} but for now we are working in a small test geometry that acts as a crude simulation of this kind of constraint.}. 
The final step is to ``match'' the flash to the original event by taking the largest flash within the allowed drift time that is within \SI{240}{cm} in the $y$-$z$ plane. Since the TPC reconstruction is still in active development, especially for low-energy events, we match to the true event %\dword{mc} 
vertex of the event in the analyses presented here. This is a reasonable approximation since the position resolution of the TPC will be significantly better than that of the \dword{pds}. 

These tools (or subsets of them) are then used to evaluate how the performance of the \dword{pds} affects the following set of physics deliverables.

\subsubsection{Nucleon Decay}
\label{subsec:fdsp-pd-simphys-ndk}

Nucleon decays are rare events, so excluding backgrounds is of the utmost importance. Since some backgrounds can be generated by cosmic rays passing outside the active detector area, setting a fiducial volume to exclude such events is critically important.

\textit{Fiducialization with \tzero}

\fixme{AIH: The following LBNC comment was specific to nucleon decay but is more generally in the way we use the term "physics deliverable" - "it's not clear why this is the “physics deliverable” - the physics deliverable is presumably - “we want to have sensitivity to x nucleon decay”. That then drives that we have to be able to see 99\% of such decays in the detector, which in turn drives the light yield requirements."}

\forlbnc{This is a concrete instantiation of the requirement laid out in the Nucleon Decay chapter of the Physics TDR. We have added an explicit reference to this section.}

The physics deliverable: the \dwords{pd} must be able to determine \tzero with approximately \SI{1}{\micro s} resolution (SP-FD-4: time resolution) for events with visible energy greater than \SI{200}{MeV} throughout the active volume, and do so with $>99\%$ efficiency (SP-FD-3: light yield), as described in Section~\ref{subsec:nonaccel-ndk-requirements}. This energy regime is relevant for nucleon decay and atmospheric neutrinos. The time measurement is needed for event localization for optimal energy resolution and rejection of entering backgrounds. 
This resolution is required for comparable spatial resolution to the TPC along the drift direction.

\fixme{is efficiency averaged over the entire detector? Or just in the dimmest regions? The CPA is an important reference point because it is the farthest point from the PDS?}

\forlbnc{This is light yield at the CPA, the dimmest region of the detector. We have added additional explanation to the caption to make this point clear.}

\begin{dunetable}[PDS efficiency for nucleon decay events.]
{ccc}
{tab:pds-ndk}
{Efficiency for tagging nucleon decay events with the \dword{pds} at the \dword{cpa}, the dimmest region of the detector, which \SI{3.6}{m} from the \dwords{pd}, shown for range of light yields (LY) at that position. Also shown is the total collection efficiency required for that light yield with the simulated scattering length, \SI{60}{cm}.)}
\dword{cpa} Light yield (PE/MeV) & Collection Efficiency  (\%) & Efficiency at the \dword{cpa} (\%) \\
%(PE/MeV) & (\%) & (\%) \\ 
\toprowrule
0.09 & 0.24   & $93.8 \pm 0.4$ \\ \colhline
0.28 & 0.75  & $97.7 \pm 0.4$ \\ \colhline
0.33 & 0.88  & $98.4 \pm 0.2$ \\ \colhline
0.50 & 1.3  & $98.9 \pm 0.2$ \\ 
\end{dunetable}


The physics here feeds down to a requirement on the minimum light yield (SP-FD-3: light yield), determined by measuring how often the correct flash was not assigned to nucleon decay 
events\footnote{The most relevant sample is actually the \textit{background} to nucleon decay events. However, efficiently simulating background that can mimic nucleon decays is challenging since they can be quite rare topologies. It is therefore easier to simulate the nucleon decay signal that should be representative of the background.} 
in the dimmest region of the detector, near the \dword{cpa}. A minimum light yield of \SI{0.5}{PE/MeV} is required to meet the requirement of 99\% efficiency, as shown in Table~\ref{tab:pds-ndk}. 

A light collector with 1.3\% collection efficiency (defined as the probability that a photon reaching the surface of the light collector will be recorded as a photo-electron) achieves this light yield with the simulated \SI{60}{cm} scattering length. This efficiency is equivalent to having \SI{23}{cm^2} of active area per module with 100\% efficiency. At this scattering length there is a factor of 20 difference in light yield between the brightest and dimmest regions of the detector, so techniques to improve light yield (discussed in the Appendix to this chapter) would reduce the inefficiency still further.


\subsubsection{Supernova Neutrinos}
\label{subsec:fdsp-pd-simphys-snb}

Supernova bursts are also rare events, though here the event is made up of many interactions instead of a single interaction. For distant supernovae (at the far side of the Milky Way or in the Large Magellanic Cloud), the top priority is to ensure that the detector can identify a burst when it happens and trigger the detector readout. For nearby supernovae, triggering will not be a challenge, and instead the goal is to record as much information as possible about the burst.

\textit{Burst Triggering}
\fixme{AIH: LBNC "this is closer to a physics deliverable in my mind - “we have to be able to see SNBs in our galaxy and the LMC”. It's not clear, however, why the PDS trigger needs similar performance to the TPC trigger."}

\forlbnc{Missing a supernova would be a serious problem for DUNE. So, we require redundant triggering abilities. By having two highly capable triggering systems in each single phase module, we dramatically reduce the chance of missing a supernova, even if one system or the other is unable to trigger because of some adverse detector conditions (for example: poor purity or on-going calibration). In order to make this point clear, we are changing to text to give a specific target for efficiency, before discussing similarity to the TPC.}

The physics deliverable: the \dword{pds} must be able to trigger on \dwords{snb} in our galaxy with $\sim100\%$ efficiency and the Large Magellanic Cloud with high efficiency with a false positive rate of less than one per month. This deliverable is most important for distant supernovae where the most important requirement is that we trigger and record the data. If both the \dword{pds} and TPC triggers have good efficiency, they can provide redundancy against one another or be combined to increase efficiency or lower the background rate. The once-per-month false positive rate is determined by limits in data handling.

\fixme{AIH: (last sentence previous para) in my reading of the DAQ, I got no sense of a limit? What do PDS know that we don't??}

\forlbnc{This was a benchmark value given to us when developing trigger plans, which I believe is more related to total data volume limitations than DAQ, so if we can do some relatively fast analysis on the burst we can likely tolerate a higher false positive rate, but without having any other concrete guidance we decided to stick with this conservative benchmark.}

The \dword{pds} trigger performance was studied for a plausible but challenging signal: a supernova burst in the Large Magellanic Cloud, which we conservatively assumed would produce only 10 signal events in the far detector. The trigger efficiency was studied with variations in light yield, dark rate, and signal-to-noise ratio, keeping the requirement from the DAQ that the fake rate be held to less than one per month. The burst trigger efficiency for 10 supernova neutrino events in one \SI{10}{kTon} module (a relatively pessimistic prediction for a supernova in the LMC), was found to be approximately $80\%$, and it is relatively insensitive to all these parameters for average light yield $>$\SI{7}{PE/MeV} (equivalent to 0.9\% collection efficiency with the simulated optical properties), dark rate $<$\SI{1}{kHz}, and signal-to-noise $>3$. The uncorrelated noise from dark rate and low signal-to-noise was easily excluded from trigger primitives by the clustering scheme, and the increased light yield makes both backgrounds and signal brighter together so performance stays basically constant. Thus this physics deliverable, while important, does not constrain any detector requirements.


\textit{TPC Energy Measurement and Time Resolution with \tzero}

\fixme{AIH: LBNC "same comment on “physics deliverable” - this is a technical requirement driven by a higher physics deliverable that isn't listed here. I'll stop making these comments from here on out."}
\forlbnc{We believe discussing the details of supernova physics is out-of-scope for the PDS chapter, so we are adding a reference to the SNB physics chatper where the benefits of increased energy resolution are discussed.}
The physics deliverable: the \dwords{pd} must be able to provide \tzero determination with approximately \SI{1}{\micro s} resolution (SP-FD-4: time resolution) for at least 60\% of the neutrinos in a typical \dword{snb} energy spectrum. The \tzero's are used in concert with the TPC-reconstructed event in two ways: to correct for the attenuation of the charge signal as a function of how far the charge drifts through the TPC and to provide more precise absolute event times for resolving short time features in the \dword{snb} neutrino event rate. This deliverable is important primarily for nearby supernovae where the number of events is large enough that time and energy resolution will be the limiting factors in extracting physics, as described in \textcolor{red}{Section XXXXX}. 

\begin{dunefigure}[Supernova neutrino energy resolution from the TPC for different PD performance assumptions.]
{fig:pds-snb-driftcor}
{The energy resolution for supernova neutrino events when reconstructed by the TPC and drift corrected with varying assumptions on the performance of the \dwords{pds}. The options considered range from drift correction for no events (black), to 60\% of events (blue), to 100\% of events (red).
}
  \includegraphics[width=0.6\textwidth]{graphics/pds-snb-drift-corr}
 \end{dunefigure}

The 60\% \tzero tagging requirement comes from two studies of a typical \dword{snb} neutrino spectrum under varying \dword{pd} performance assumptions: the resolution of the energy reconstructed with the TPC and drift-corrected using the time from the \dwords{pd}, and the observability of the in-fall `notch' in the \dword{snb} event time distribution. Both studies show significant improvement when going from no \dwords{pd} to a system which has a collection efficiency of at least 0.25\% (equivalent to \SI{0.5}{PE/MeV} for 60\% of the detector volume), but only marginal improvements past that point. The light yield required here is sufficiently low that this deliverable does not set any detector requirements.


\textit{Calorimetric Energy}

Physics deliverable: the \dword{pds} should be able to provide a calorimetric energy measurement for low-energy events, like \dwords{snb}, complementary to the TPC energy measurement. 
Improving the energy resolution will enable us to extract the maximum physics from a \dword{snb} (see \textcolor{red}{Section~XXXXXX}), and with the goal to achieve energy resolution comparable to the TPC, we can take full advantage of the anti-correlation between the emission of light and charge signals imposed by the conservation of energy. In addition, this requirement allows the photon detection system to provide redundancy if a supernova occurs during adverse detector conditions. If the argon purification system is offline, the photon signal is significantly less sensitive to electronegative impurities , and if the drift field is low, the reduced charge signal can be partially recovered by increased light.
\fixme{AIH: Calorimetric energy, simulations in Fig 1.2. Do these results come from just one simulation. Is there more than one simulation? Is there agreement between simulations?}
\forlbnc{The physics studies are all based on one simulation implemented in LArSoft. The lines on this plot were made with different assumptions about the efficiency of the detectors, but come from the same underlying simulation.}
\fixme{AIH: LBNC "Fig. 1.2 - why didn't you try 23 cm$^2$ as a case, which was the area driven above? i feel that the physics case here overall is still not clear enough to me.  The first paragraph here states that the PDS should provide complementary energy info to the TPC to “extract maximum physics” but again, there's no actual physics plot that states what that 'maximum' physics is for a resolution of 0.3 vs. 0.4.  That study might be in the Physics TDR but it needs to exist and be referenced."}
\forlbnc{The MC runs at different light yields are fairly costly, and the range of values we were interested in testing has changed over time (the nucleon decay studies were done before the supernova studies). However, it is straightforward to interpolate between light yields. As with the other comments along these lines, we are adding cross-references to the physics chapters rather than trying to address the details of supernova physics here in the photon detector chapter.}
\begin{dunefigure}[Supernova neutrino energy resolution from the PDS for different light yields.]
{fig:pds-snb-calo}
{The energy resolution (determined from the distribution widths of the fraction of difference between reconstructed and true  to true neutrino energy for simulated events) for supernova neutrino events when reconstructed directly through \dword{pds} calorimetry for a range of light yields, represented by different colors. The red line labeled \textit{Physics} 
shows the energy smearing inherent to the neutrino interactions and thus serves as a theoretical minimum resolution. The black line shows the energy resolution achieved by the \dword{tpc}, defined in a similar way. The performance improves significantly up until approximately \SI{20}{PE/MeV} where the \dword{pds} and \dword{tpc} give comparable resolution below approximately \SI{7}{MeV}.
%The behavior vs. true and reco energy looks different because some higher energy neutrinos lose otherwise visible energy into neutrons, causing the poorly resolved higher true energy events to all feed down to lower reconstructed energies. 
}
  \includegraphics[width=0.6\columnwidth]{graphics/pds-snb-res-vs-true.pdf}
  %\includegraphics[width=0.48\columnwidth]{graphics/pds-snb-res-vs-reco.pdf}
 \end{dunefigure}

The calorimetric energy performance was studied for supernova burst neutrino events simulated in the far detector for a range of different detector performance assumptions. The energy reconstruction was simple, correcting the total observed amount of photons for the average number of photons expected per MeV as a function of position along the drift direction. Events were required to be well away from the side walls to avoid any possible edge effects. The energy resolution vs. true energy is shown in Fig.~\ref{fig:pds-snb-calo}. There is a significant benefit to achieving a photon detector with an average light yield of \SI{20}{PE/MeV}, where the \dword{pds} and \dword{tpc} have comparable resolution for the lowest energy ($<$\SI{7}{MeV}) supernova neutrinos. Past this light yield, the improvement appears to plateau in this analysis. This physics deliverable thus sets a requirement, FD-SP-3: light yield, of \SI{20}{PE/MeV} averaged over the active volume.

Note, that while options which can improve the uniformity of the detector are not required to achieve required resolution, they are likely to improve the calorimetric energy reconstruction above and beyond total light yield. A detector which is more uniform will be easier to calibrate, and the impact of uncertainties on the optical parameters of the liquid argon will be reduced. This effect is potentially important for supernova neutrinos, and certainly more important for the beam neutrino events described in the next section. In addition, for Xe-doping specifically, speeding up the late light will allow for flashes which are narrower in time, reducing the amount of radiological contamination mixed in with the signal, which is of particular importance with these relatively small signals.



\subsubsection{Beam Neutrinos}
\label{subsec:fdsp-pd-simphys-beam}

The \dword{pds} is not needed for fiducializing beam neutrino events since the pulsed beam will provide sufficient precision to place the interactions in space. However, the \dwords{pd} can potentially contribute to the energy measurement and the better timing resolution can help identify Michel electrons from muon and pion decay.


\textit{\it Calorimetric Energy}

Physics deliverable: the \dword{pds} should be able to provide a calorimetric energy measurement for high-energy events, like neutrinos from the LBNF beam, complementary to the TPC energy measurement.
Neutrino energy is an observable critical to the success of the oscillation physics program (see \textcolor{red}{Section XXXXX}), and a second independent measurement can provide a cross-check that reduces systematic uncertainties or directly improves resolution for some types of events. In order to provide a meaningful cross-check, the resolution and uncertainty of the \dword{pds} measurement must be comparable to the calorimetric resolution of the TPC. The limit on this measurement will likely come from how well the efficiency of the detector and the optical properties of the argon can be determined (both must be known to approximately 5\% to have a comparable measurement of electron shower energy), which define a program of measurements between now and the operation of the detector rather than requirements on the system itself. The requirement that does flow down from this is that the dynamic range of the system be sufficient to allow for accurate measurement of the amount of light reaching the \dword{pds}. 

\fixme{AIH: LBNC "again, where does this 20\% number come from? Why is 20\% ok but not 40\%?  although i suppose in the end it doesn't matter since it doesn't saturate anyway according to your sims."}
\forlbnc{Unfortunately, we only have a fairly rough estimate of how much saturation we will be able to correct for since this analysis is still on-going. If we find that the saturation is an issue with the X-ARAPUCA design and 12-bit ADCs, it is possible to upgrade the electronics to 14-bit where saturation will, indeed, be negligibly rare for beam events.}
Some amount of saturation is tolerable since it can be corrected for using the pulse shape or the neighboring unsaturated channels. However, if the saturation is too large, and too many channels are saturated, the corrections become difficult, so we require that no more than $20\%$ of beam neutrino events have saturating channels (SP-PDS-16: dynamic range), consistent with but looser than the \dword{tpc} requirement of $10\%$.

We studied the likelihood of channels saturating by simulating beam neutrino events in the far detector. The likelihood of saturation depends on the digitization frequency, the dynamic range, and the collection efficiency of the detector design. Assuming the baseline electronics, a 12-bit and \SI{80}{MHz} digitizer, we find the likelihood of saturation vs. average light yield shown in Table~\ref{tab:pds-dynamicrange}. 

\begin{dunetable}[Fraction of beam events with channels that saturate]
{ccc}
{tab:pds-dynamicrange}
{The fraction of beam events which have saturating \dword{pds} channels for different light yields, and the corresponding PDS collection efficiencies.}
Avg. Light Yield (PE/MeV) & Collection Efficiency (\%) & Saturation Fraction (\%) \\ \toprowrule
6 & 0.88  & 6 \\ \colhline
13 & 1.8   & 13 \\ \colhline
21 & 2.6   & 20 \\ \colhline
28 & 3.5   & 24 \\ 
\end{dunetable}


\textit{\it Michel Electron Tagging}

Physics deliverable: the \dword{pds} should be able to identify events with Michel electrons from muon and pion decays.
The identification of Michel electrons can improve background rejection for both beam neutrinos and nucleon decay searches. 
Some Michel electrons are difficult to identify with the TPC since they appear simultaneous within the time resolution of the TPC and co-linear with their parent. However, because the \dword{pds} can observe the fine time structure of events in the detector, it can identify Michel electrons that appear separated in time from the main event. While DUNE-specific studies of Michel electron tagging have not been performed, the LArIAT experiment has demonstrated that Michel electrons can be identified and studied using photon signals. 

%\fixme{Add LArIAT reference once it exists}


%%%%%%%%%%%%%%%%%%%%%%%%%%%%%%%%%%%%%%%%%%%%%%%%%%%%%%%%%%%%%%%%%%%%%%%%%%%%%%%%%%%%%%%%%%