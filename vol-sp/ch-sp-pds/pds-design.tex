% This section reworked to have all the major PD subsystems as sections. 

%%%%%%%%%%%%%%%%%%%%%%%%%%%%%%%%%%%%%%%%%%%%%%%%%%%%%%%%%%%%%%%
%\section{Photon Detection System Design}
%\label{sec:fdsp-pd-design}
%\metainfo{(Length: \dword{tdr}=50 pages, TP=20 pages)}
%\metainfo{\color{blue} Content: Warner, WG Conveners}

% Everything in this section is about the 10kt module now
%\subsection{ARAPUCA Configuration in DUNE \SI{10}{kt} }
%\label{sssec:arapuca-dune}

%%%%%%%%%%%%%%%%%%%%%%%%%%%%%%%%%%%
\section{The Light Collectors}
%\label{ssec:fdsp-pd-pc-arapuca}
\label{sec:fdsp-pd-lc}


% (rjw) The reworking was done on 1/17/19 -- I haven't had chance to do a careful read through after the recovery yet (1/25/19) \fixme{rjw 1/12/19 This section needs some reworking after Anne's reorg. 1/21 has reworking been done? anne}%\{moved top four pgraphs + op princ graphic to overview. Anne 1/19}

%\subsection{\dword{xarapu}} 
%\label{sssec:x-arapuca}

%\fixme{\color{blue}Ettore/Dave/Bob: This section needs some reworking after Anne's reorg. with an Overview section.  rjw 1/12/19}
%\fixme{\color{blue}Dave: I made some changes but you should go over it after reading through the 1.2.5 Design Overview to avoid too much repetition (you can refer back to it by way of introduction). rjw 1/16/19}

The \dword{xarapu}, adopted as the baseline design, is an evolution of the ARAPUCA concept that further improves the collection efficiency, while retaining the same working principle, mechanical form factor and active photosensitive coverage. 
 In the original ARAPUCA concept, two wavelength-shifters were coated on either side of the dichroic filter window. 
 In contrast, the \dword{xarapu} replaces the inner surface coating with a wavelength-shifter doped polyvinyl toluene (PVT)  light guide\footnote{Eljen EJ-286\texttrademark{}.} occupying a portion of the cell volume, with the silicon photosensor readout mounted along the narrow sides of the cell, as illustrated in Figure~\ref{fig:pds-x-arapuca-cell}. The model shown is a single cell design used for prototypes that allows for photons to enter from either face, however one window can be replaced with an opaque reflecting surface for sensitivity through just one face.

Photons entering the lightguide plate are absorbed and wavelength-shifted with high efficiency and some fraction (those incident on the plate surface at greater than the critical angle) are transported to the readout via total internal reflection. The \dword{lar} gaps between the plate and the surfaces of the cavity ensure the discontinuity of the refractive index that contributes to effective trapping of the photons (n$_{plate}$=1.58 and n$_{LAr}$=1.24 for the wavelengths emitted by the plate).
Those exiting the plate reflect off the filter or other highly reflecting surfaces of the cell, with some fraction eventually incident on a \dword{sipm}, as in a standard ARAPUCA cell.
\dword{xarapu} is thus effectively a hybrid solution between the \dword{sarapu} and the wavelength-shifting light guide concepts implemented in \dword{pdsp}.

%\fixme{\color{blue}Dave: A Google search for Eljen EJ-296 identifies it (on the Eljen site) as a scintillator paint - is there a reference for the doped plate (and the plate material)? }

%\fixme{DWW:  Yes--  should be EJ-286, which chows up on their page as WLS plate.  Fixed in footnote}

This solution minimizes the number of reflections on the internal surfaces of the cell and thus minimizes the probability of photon loss. Simulations suggest that this modification will lead to a significant increase of the collection efficiency. 
%to approximately 60\%, allowing the photon detection efficiency -- including the \dword{sipm} response.
%\fixme{Bob--  this last sentence needs fixing.  Perhaps just un-comment the line just after this note?}
%-- to approach 20\% in principle.  
Results from prototype measurements are presented in Sections~\ref{sec:xarapuca-unicamp} and \ref{sec:iceberg-teststand} and are consistent with the simulations.

 \begin{dunefigure}[\dshort{xarapu} conceptual model]{fig:pds-x-arapuca-cell}
{Simplified conceptual model depicting a \dword{xarapu} cell design sensitive to light from both sides: assembled cell (left),  exploded view (right). The yellow plates represent the dichroic filters (coated on their outside surfaces with \dword{ptp} \dword{wls}), the pale blue plate represents the wavelength shifting plate, and the photosensors are visible on the right side of the cell. The size and aspect ratio of the cells can be adjusted to match the spatial granularity required for a \dword{pd} module. 
%The cell is shown laying on its side; it stands vertically when mounted in an \dword{apa}.
} 
 % \vspace{-2.5cm}
 %two-sided x-arapuca 4/14/18 
  \includegraphics[height=.25\textheight]{pds-x-arapuca-cell}
  \includegraphics[height=.25\textheight]{pds-x-arapuca-exploded-view}
\end{dunefigure}
%\fixme{\color{blue}Dave: figure \ref{fig:pds-x-arapuca-cell} needs bigger text and labels need to match nomenclature used in text. anne}
%\fixme{Addressed nomenclature in label.  Will add component labels 1-15-19 DWW}
%\fixme{\color{blue}Dave: The text in Figs. 1.8 and 1.9 needs enlarging too}
%\fixme{modifying descriptive text below}
%In the \dword{xarapu} design, Figure~\ref{fig:pds-x-arapuca-cell}, the inner shifter coating/lining over the reflective walls of the box is replaced by a thin wavelength-shifting light guide slab inside the box, of the same dimensions of the acceptance filter window and parallel to it. The \dword{sipm} arrays are installed vertically on the sides of the box, parallel to the light guide thin ends. 
%In the \dword{xarapu} design, Figure~\ref{fig:pds-x-arapuca-cell}, a single acrylic wavelength shifting light guide (Eljen EJ-296) 

The  \dword{pd} module designed for the DUNE \dword{spmod}, illustrated in Figure~\ref{fig:pds-x-arapuca-full-module}, consists of four supercells each containing a rectangular 
light guide inside the cell positioned behind an array of six dichroic filters that form the entrance window.  
% \fixme{Anne proposes to remove: This thin wavelength-shifting slab replaces the inner shifter coating over the reflective walls in the \dword{pdsp} \dword{sarapu} design. }
This design is easily configurable to detect light from just one side, as required for the side \dword{apa}s, or from both sides for the central \dword{apa}s. 

For dual-sided \dword{xarapu} modules, dichroic filters are placed on both sides of the cell facing the drift volumes.  In the case of the single-sided device, the back side of the cell has a layer of highly reflective Vikuiti\footnote{3M Vikuiti\texttrademark\  ESR - http://multimedia.3m.com/mws/media/193294O/vikuiti-tm-esr-application-guidelines.pdf} to act as %the rear
a  reflector.  In both cases, the \dword{sipm} arrays are installed on two of the narrow sides of the cell perpendicular to the windows, parallel to and up against the light guide thin ends. Half of the \dword{sipm} active detection area collects photons from the light guide, a quarter of the area on either side of the guide is free to collect the fraction of photons reflected off the cell walls and windows. 
%\{check -Anne added this} - amended rjw

The basic mechanical design of the \dword{xarapu}-based \dword{pd} modules  
is similar to that of the two prototypes produced for \dword{pdsp}. Modifications to the prototype design include:  mechanical changes to allow for single-sided or dual-sided readout, increase in light collection area made possible by larger slots in the \dword{apa}, and modifications to the cabling and connector plan required to move the \dword{pd} cables out of the \dword{apa} side tubes, while reducing the cable requirements to one cat-6 cable per \dword{pd} module.


%{figure of full PD module (with outside dimensions) here}

\begin{dunefigure}[An \dshort{xarapu} module indicating four supercells.]
{fig:pds-x-arapuca-full-module}
{\dword{xarapu} module overview. A module, which spans the width of an \dword{apa}, includes 24
 \dword{xarapu} cells, grouped into a set of four supercells of six cells each. In the center, active ganging PCBs collect the signals and mechanically connect the supercells.}
% tweaked rjw \{check anne's update to figure caption~\ref{fig:pds-x-arapuca-full-module}}
 %two-sided x-arapuca 4/14/18
   \includegraphics[width=17cm]{pds-design-full-module-dimensioned-2}
  %\vspace{-2.5cm} 
\end{dunefigure}

%\fixme{Figure 1.8 is the same as figure 1.4.  I wonder if we should remove 1.4?  DWW.  I removed the earlier use of this figure 1-15}
%\fixme{\color{blue}DWW:  I made some modifications--  I think I removed the need to do so 12-15-19.  Anne added that figure in the Overview - need to ask her if repeating the figure is kosher, if not where should it go, Overview or Design.}

An \dword{xarapu} module is assembled in a bar-like configuration with external dimensions inside the APA frame of \SI{2092}{mm}$\times$\SI{118}{mm}$\times$\SI{23}{mm},  allowing insertion between the wire planes through each of the ten slots (five on each side) in an \dword{apa}. In addition, there is a header block \SI{5}{mm}(long)$\times$\SI{135}{mm}(wide) at the insertion side of the module used to fix the module inside the APA frame, bringing the maximum length to \SI{2097}{mm} and the maximum width to \SI{135}{mm}.
The module contains four \dword{xarapu} supercells, each with six dichroic filter-based optical windows (for the single-sided readout) or twelve windows (double-sided readout) with an exposed area of \SI{78}{mm}$\times$\SI{93}{mm}.  
The total widow area for each (single-sided) supercell \dword{xarapu} is \SI{43524}{mm$^2$}.
The internal dimensions of a supercell are approximately \SI{488}{mm}$\times$\SI{100}{mm}$\times$\SI{8}{mm}. A \dword{wls} plate (Eljen EJ-286) of dimensions \SI{487}{mm}$\times$\SI{93}{mm}$\times$\SI{3.5}{mm} is centered in the supercell midway between the dichroic windows. 

The thickness of \SI{3.5}{mm} for the plate is chosen to allow the almost complete absorption of the photons wavelength-shifted by the \dword{ptp} and to ensure the nominal conversion efficiency. This thickness allows to have a \SI{2}{mm} \dword{lar} gap on both sides of the plate, which prevents any physical contact of the surfaces even considering the tolerances on materials' thicknesses and plate flatness.   

\begin{dunefigure}[Exploded \dshort{xarapu} supercell]{fig:pds-x-arapuca-exploded-Detail}
{Detailed exploded view of \dword{xarapu} supercell. Note that components are designed to be cut from FR-4 G-10 sheets to simplify fabrication.}
 % \vspace{-2.5cm}
 %two-sided x-arapuca 4/14/18
   \includegraphics[height=.55\textheight]{pds-exploded-supercell-assembly-r3}
 % \includegraphics[height=.25\textheight]{pds_design_dimensioned_cross_section}
\end{dunefigure}

To reduce production costs and simplify fabrication, most of the \dword{pd} components are designed to be water-jet cut from sheets of FR-4 G-10 material, with minimal post-cutting machining required (mostly the tapping of pre-cut holes).  The current design contains many small fasteners; we will investigate replacing some of the fasteners with epoxy lamination of cut sheets where appropriate and cost effective.

The \dwords{sipm} are mounted to PCBs called ``photosensor mounting boards,'' which are positioned on the long sides of the supercell.  
Six \dwords{sipm} are mounted to a single photosensor mounting board.  All six are electrically connected in parallel (``passively ganged'').

%\fixme{Clarify PCBs vs photosensor mounting board vs passive photosensor ganging (in figure). Anne}
%\fixme{Addressed 1-14-19.  DWW}

 Before mounting into the \dword{xarapu} module the boards are tested at room and LN2 temperatures. 
 %It is anticipated that production of the boards for the \dwords{spmod} will be done outside the USA, and South American institutions will optimize the design.
 %\{latin am or south - pick one} 
 Each supercell uses eight photosensor mounting boards, each with six \dwords{sipm} (Figure~\ref{fig:mounting-board-routing-board}~(top)), %mounting boards are used per supercell 
 to accommodate the 48 \dwords{sipm}.  The ganged signal outputs from these boards are connected to traces in signal routing boards at the edge of the \dword{pd} module. These signal routing boards also act as mechanical elements in the design, mechanically joining the supercells and providing for rigidity.  The routing boards PCBs are four-layer boards, \SI{1046}{mm}$\times$\SI{23}{mm}$\times$\SI{1.5}{mm}.

%\fixme{check left and right and angle in figure}
 \begin{dunefigure}[\dshort{xarapu} SiPM mounting and signal routing boards]
 {fig:mounting-board-routing-board}
{Model of photosensor mounting board (top) and signal routing PCB (bottom) for \dword{xarapu} module.  Six Hamamatsu \dwords{mppc} are passively ganged and the ganged signals transmitted along the routing board to the active ganging circuits in the center of the module.}
 % \vspace{-2.5cm}
%  \includegraphics[angle=90,height=7cm]{pds-photosensor-mounting-board-r4}
%  \includegraphics[angle=90,height=7cm]{pds-trace-routin
\includegraphics[height=7cm]{pds-photosensor-mounting-board-r4}
\includegraphics[height=7cm]{pds-trace-routing-pcb-r2}
\end{dunefigure}
The passively ganged signals are then routed through these boards to an active-ganging PCB at the center of the module, where all eight passively ganged signals from a single supercell are actively ganged into one output channel (Figure~\ref{fig:mounting-board-routing-board}~(bottom)). This summed output from a single supercell is then connected to a single twisted pair in the Cat-6 readout cable for the module.  The active ganging PCBs (one per supercell, four per module) are positioned in the module so that they are located inside the central \dword{apa} mechanical support tube when fully installed.

%\{?? add pictures of module center showing active ganging PCBs outside \dword{apa}, and buried in central tube} leave out for now

%remove the of standard arapucas used for protodune
%\begin{dunefigure}[\dword{pdsp} ARAPUCA modules during assembly.]{fig:arap-prod01}
%{\dword{pdsp} ARAPUCA modules during assembly prior to installation of the dichroic filters; \dwords{sipm} and TPB coated reflector are visible.}
%  \includegraphics[height=8cm]{pds-arapuca-protodune02-apr18}
% this photo no longer used so removed from Overleaf graphics folder 2/15/19 
%\end{dunefigure}

The  internal surface on the lateral sides of the cell are lined with the Vikuiti\texttrademark\ adhesive-backed dielectric mirror foil
% rjw this link was provided on an earlier page as a footnote.
%\footnote{3M Vikuiti\texttrademark\  ESR - http://multimedia.3m.com/mws/media/193294O/vikuiti-tm-esr-application-guidelines.pdf}
that has been laser cut with openings at the locations of the \dwords{sipm}.  In the case of the single-sided readout the dichroic filter windows on the non-active side of the cell are replaced by a blank FR-4 G-10 sheet, lined in the cell interior with a Vikuiti\texttrademark\ reflector foil.
%\fixme{ETTORE--  CAN YOU PLEASE ADD TEXT ABOUT WHERE THEY HAVE BEEN USED BEFORE?}
Vikuiti\texttrademark\ foils have been extensively used in the WArP\footnote{http://warp.lngs.infn.it/} and \dword{lariat}\footnote{https://lariat.fnal.gov/} experiments where they had good optical performance and there were no reported issues related to adhesion of the film or dissolution of the wavelength shifter in LAr.

In DUNE, we have demonstrated that the foils adhere very strongly to FR4 G10 surfaces cleaned following the PD standard cleaning procedures.  Tests by the PD group have demonstrated adhesion is maintained through multiple cryogenic (LN2)/warm thermal cycles.  The mechanical design provides additional mechanical constraints on the Vikuiti\texttrademark\ sheets after module assembly, so the foils will be held in place mechanically even if the adhesive fails.  Samples of the adhesive have been used in other experiments with no negative impact on the \dword{lar} purity observed.  Samples will be tested in the FNAL materials test stand, and in \dword{iceberg}, \dword{sbnd}, and \dword{pdsp2} to confirm that the adhesive does not negatively impact \dword{lar} purity or detector performance.


%\forlbnc{LBNC: LAr flow in/out the superccells addressed here.}

To allow for air to vent out of the cell and \dword{lar} to completely fill the cell during the detector fill, holes are provided at the end of each supercell (four holes total, top and bottom of the cell when mounted in the APAs).

%these are visible in Figure~\ref{fig:arap-prod01}, which shows an ARAPUCA during assembly prior to installation of the optical windows.

%The backplane \dword{sipm} boards for the \dword{pdsp} modules were designed at CSU and produced by an external USA vendor\footnote{Advanced Circuits Inc.; www.4pcb.com.}; the \dwords{sipm} were soldered on the boards using a reflow oven at CSU. Before mounting into the ARAPUCA module they were tested at room and LN2 temperatures. It is anticipated that the production of the boards for  \dwords{spmod} will be done outside the USA. %move to Brazil.
 %and the design will be optimized by South American institutions in collaboration with CSU.
%\{want to specify Brazil?} - leave as SA

%\fixme{Adding in text to mention Brazilian filter manufacturer}

The optical window(s) of each supercell are dichroic filters with a cut-off at \SI{400}{nm}. While the filters used for the \dword{pdsp} prototypes have been acquired from Omega Optical Inc.\footnote{http://www.omegafilters.com/}, Opto Eletronica S.A.\footnote{http://www.opto.com.br/} (in Brazil) is our current primary candidate vendor for DUNE production filters.  Opto is a well-established company with a long history of involvement in research optical components for harsh environments and large thermal gradients (including camera optics for satellite photography).  We plan an extensive suite of testing of their filters at \dword{unicamp}, in \dword{iceberg}, and in \dword{pdsp2}. Other vendors are also being investigated\footnote{ASHAI -Japan, Andover-USA, and Edmunds Optics-USA}.

%\fixme{Added text about Opto qualifications to respond to Mont concern  DWW Done}

% rjw 12/2/18 the following moved to Prod and Assembly
%Prior to coating, the filters are cleaned according to the procedures given by the manufacturer using isopropyl alcohol. Since the most likely vector for scratching/damaging the coating is dragging contaminated wipes across the surface, new clean lint free wipes are used for each  cleaning pass on the surface. Clean filters are then baked at 100$^\circ$C for \SI{12}{hours}.    
   
The filters are coated on the external side facing the \lar active volume with \dword{ptp}\footnote{p-TerPhenyl, supplier: Sigma-Aldrich\textregistered.}.  The coatings for the \dword{pdsp} modules have been made at the thin film facility facility at Fermilab using a vacuum evaporator. Each coated filter was dipped in \dword{ln} to check the stability of the evaporated coating at cryogenic temperature. 

%\forlbnc{LBNC: Where coatings will be done addressed here.}

For the \dword{fd}, filter coatings will be done by the vacuum deposition facility at \dword{unicamp} (see Section~\ref{sec:xarapuca-unicamp}).

%For \dword{pdsp} \dword{pd} production the evaporation process will be performed at \dword{unicamp} in Brazil, where a large vacuum evaporator with an internal diameter of one meter is now available. The  conversion efficiency of the film deposited on the filters or on the Vikuiti\texttrademark\ foils will be measured with a dedicated set-up that will use the \SI{127}{nm} light produced by a VUV monochromator.
%\{Bob:  I moved this to the assembly section  Dave: will this be done on every foil? Is this a design issue or QA/QC?}

%%%%%%%%%%%%%%%%%%%%%%%%%%%%%%%%%%%%%%%%%%%%%%%%%%%%%%%%%%%%%%%%%%%%%%%%%%%%
% rjw Moved the section on prototype measurements to the Validation section
%%%%%%%%%%%%%%%%%%%%%%%%%%%%%%%%%%%%%%%%%%%%%%%%%%%%%%%%%%%%%%%%%%%%%%%%%%%%


%================================================================

%\{Full light guides options sections removed - replace with summary, results and rationale for arapuca as baseline unless already addressed previously.}
%rjw 10/7/18 remove light-guide sections
%\section{Photon Collector: Dip-Coated Light Guides}
%\label{ssec:fdsp-pd-pc-bar1}
%\metainfo{\color{red} \bf Content needed: (4 pages) Toups}
%================================================================

%%%%SILICON PHOTOSENSORS %%%%%%%%%%%%%%%%%%%%%%%%%%%%%%%%%%%%%%%%%%%%%%%%%%%

\section{Silicon Photosensors}
\label{sec:fdsp-pd-ps}

%\fixme{rjw 1/13/19 The following moved from what was originally a design considerations section...but converted by Anne to an overview, so this was too much detail. Moved the summary that was here in the Design section to there. Need to condense to a shorter narrative and get to the bottom line.} 

The physics goals, the design of the light collectors, and the trigger and data acquisition system constraints determine the suite of specifications for the silicon photosensors such as the number of devices, spectral sensitivity, dynamic range, triggering threshold and rate, and zero-suppression threshold. An initial survey of commercial products and a 12-month period of R\&D indicated that the performance characteristics of devices from several vendors effectively meet the \dword{pds} needs. 
However, a key additional requirement is to ensure the mechanical and electrical integrity of these devices in a cryogenic environment. Catalog devices for most vendors are certified for operation only down to \num{-40}$^\circ$C and though one candidate device performed well initially, after an unadvertised production process change a large fraction cracked when submerged in \dword{lar}\footnote{SensL MicroFC-60035C-SMT}. This highlighted the need to be in close communication with vendors in the 
\dword{sipm} design, fabrication, and packaging certification stages to ensure 
robust and reliable long-term operation in a cryogenic environment. 

Nearly one thousand of several types of \dwords{sipm} are used in the \dword{pdsp} \dword{pd}\footnote{\dword{pdsp} \dword{pds} uses 516 SensL MicroFC-60035C-SMT, 288 Hamamatsu MPPC 13360-6050CQ-SMD with cryogenic packaging, and 180 Hamamatsu MPPC 13360-6050VE.}, providing an excellent test bed for evaluation and monitoring of \dword{sipm} performance in a realistic environment over a period of months. Results from \dword{pdsp} are summarized in Section~\ref{sec:fdsp-pd-validation}.

The baseline \dword{pds} design has \num{192} \num{6}$\times$\SI{6}{mm$^2$} \dwords{mppc} per \dword{pd} module with groups of \num{48} \dwords{mppc} electrically ganged into four electronics readout channels. This leads to a total of \num{288000} \dwords{mppc} per \dword{spmod}. 

Two entities have expressed interest to engage with the consortium with an explicit intent to provide a product specifically for cryogenic operation: (1) Hamamatsu Photonics K.K., a large well-known commercial vendor in Japan, and (2) \dword{fbk}\footnote{https://www.fbk.eu/en/}
%Fondazione Bruno Kessler (FBK) 
in Italy. 
\dword{fbk} is an experienced developer of solid state photosensors that typically licenses its technology; it is partnering with the DarkSide\footnote{http://darkside.lngs.infn.it/} collaboration to develop devices with specifications very similar to DUNE's.  Table~\ref{tab:photosensors} summarizes the key characteristics of the baseline device, Hamamatsu S13360, and two other devices from Hamamatsu and FBK that are under consideration.

%\fixme{\color{blue} Francesco: Can we show some performance info from  FBK devices here?}

While the devices from Hamamatsu have been tested extensively by the consortium, those from \dword{fbk} are relatively new to us. The technologies they have developed that are suitable for the needs of DUNE are the NUV-HD-SF (``standard field'') and NUV-HD-LF (``low field'')~\cite{Gola:2019idb}. In particular, the LF technology (see Table~\ref{tab:photosensors}) offers the lowest dark current rate and has been successfully employed for the DarkSide experiment. NUV-HV-SF sensors developed by \dword{fbk} specifically for DUNE have been tested in Milano (Italy), CSU (CO, US), and NIU (IL, US). The sensors were characterized both at room and cryogenic temperatures (\SI{77}{K}) and underwent more than \num{50} thermal cycles. The tests confirmed the nominal performance of the photosensors and proved the reliability of the sensors at low temperature. Extensive thermal tests and characterization of sensors in the NUV-HD-LF technology are in progress.   


%\forlbnc{LBNC: Text added to address concern on photosensors downselect timescale by establishing a milestone for the decision.}
%\fixme{Photosensor working group--  can you add some text about the downselect process here}
The milestone for photosensor selection for the first \dword{spmod} is early 2021.  Though a baseline photosensor that meets the requirements has been identified, the addition of experienced INFN groups to the \dword{pds} effort has enabled us to pursue the promising \dword{fbk} option in way that was not possible previously.  We are carrying out targeted investigations on the performance, cost, and production capability to establish the viability of the alternatives for all or part of the sensors required for either the first or subsequent \dwords{spmod}. Two photosensor types (one from each vendor) will be selected in early 2020 to be used in \Dword{pdsp2}.

As described in Section~\ref{sec:fdsp-pd-lc}, the size and sensitivity of currently available \dwords{sipm} requires that multiple devices are needed for each \dword{xarapu} cell. The spatial granularity of each device is much smaller than required for DUNE so,
along with limitations on the number of readout channels, it is required that the signal output of the \dwords{sipm} must be electrically ganged. The terminal capacitance of the sensors strongly affects the \dword{s/n} when devices are ganged in parallel, which led to a design that passively gangs several sets of \dwords{sipm} in parallel, which are then summed with active components, as described in Section~\ref{sec:pds-design-ganging}.


%{ rjw 11/25/18 reduced table to just Hamamatsu and up to two FBK device -- values to be inserted}
\begin{dunetable}[Candidate photosensors characteristics]
{p{0.18\textwidth}p{0.18\textwidth}p{0.18\textwidth}p{0.18\textwidth}}
{tab:photosensors}
{Candidate Photosensors Characteristics.}
	                      &Hamamatsu (Baseline)   & Hamamatsu-2    & FBK                 \\ \toprowrule
Series part \#            & S13360                &     S14160         & NUV-HD-LF         \\ \colhline
V$_{\rm br}$ (typical)    & 50 V to 52 V          &   36 V to 38 V & 31 V to 33 V                \\ \colhline
V$_{\rm op}$ (typical)    & V$_{\rm br}$+\SI{3}{V}             &   V$_{\rm br}$+\SI{2.5}{V} & V$_{\rm br}$+\SI{3}{V}                \\ \colhline
Temperature dependence of V$_{\rm br}$  & \SI{54}{mV/K}&  \SI{35}{mV/K}& \SI{25}{mV/K}   \\ \colhline
Gain~at V$_{\rm op}$(typical)   & $1.7\times10^6$     &      $2.5\times10^6$ &  $0.75\times10^6$          \\ \colhline
Pixel size                & 50 $\mu$m             &       50 $\mu$m    & 25 $\mu$m            \\ \colhline
Size                      & 6 mm x 6 mm           &     6 mm x 6 mm    & 4 mm x 4 mm            \\ \colhline
Wavelength                & 320 to 900 nm         &     280 to 900 nm  & 280 to 700 nm            \\ \colhline
PDE peak wavelength       & \SI{450}{nm}         &      \SI{450}{nm}     & \SI{450}{nm}           \\ \colhline
PDE at peak                & 40\%                  &        50\%        & 50\%            \\ \colhline
DCR at \si{0.5}{PE}               & < \SI{50}{\kilo\hertz\per\square\milli\meter}      & < \SI{100}{\kilo\hertz\per\square\milli\meter}   & < \SI{25}{\kilo\hertz\per\square\milli\meter}               \\ \colhline
Crosstalk                 & <~3\%				  &      <~7\%          & <~3\%             \\ \colhline
Afterpulsing              &                       &                &                 \\ \colhline
Terminal capacitance      & \SI{35}{\pico\farad\per\square\milli\meter}          &   \SI{55}{\pico\farad\per\square\milli\meter}     &      \SI{50}{\pico\farad\per\square\milli\meter}         \\ \colhline
Lab experience            & Mu2e and DUNE prototypes      &                &     Darkside  \\         
\end{dunetable}


%%%%ELECTRONICS %%%%%%%%%%%%%%%%%%%%%%%%%%%%%%%%%%%%%%%%%%%%%%%%%%%

\section{Electronics}
\label{sec:fdsp-pd-pde}
%\metainfo{\color{red}\bf  Content: Djurcic/Franchi/Moreno/Spitz/Toups}

The electronic readout system for the \dword{pds} must (1) collect and process electrical signals from \dword{sipm}s reading out the light collected by the \dwords{xarapu}, (2) provide an interface with the trigger and timing systems supporting data reduction and classification, and (3) transfer data to offline storage for physics analysis. Figure~\ref{fig:pds-electronics_signalpath} provides a simple overview of the signal path and key elements. 
%\forlbnc{LBNC:  Replaced figure with bigger font}
\begin{dunefigure}[Overview of the \dshort{pds} signal path]
 {fig:pds-electronics_signalpath}
 {Overview of the \dword{pds} signal path.}
\includegraphics[width=15cm]{graphics/pds-signal-routing-diagram-r3.pdf}
\end{dunefigure}


%As specified in Table~\ref{tab:spec:time-resolution}, the readout system must tab:specs:SP-PDS
As specified in the requirements Table~\ref{tab:specs:SP-PDS}, the readout system must enable the $t_0$ measurement of non-beam events; this capability will also enhance beam physics by recording interaction time of events within 
beam spill more precisely to help separate against potential cosmic background interactions. A highly capable readout system was developed for use with \dword{pdsp} and prototype development as described in Section~\ref{sec:valid-pdsp}. However, a more cost-effective waveform digitization system developed for the \dword{mu2e} experiment has been identified and selected as the baseline choice for the \dword{pd} system. 

%Physics simulation studies are currently underway to determine if pulse-shape discrimination will be required, which would provide the capability to record both prompt and delayed components of scintillation light (characteristic times of \SI{6}{ns} and \SI{1.3}{$\mu$s}), the latter consisting mostly of single photoelectrons and thus place stringent requirements on \dword{s/n} performance. 
% *** 12/3/18 Per Josh S  - the mu2e system can do this so can use for this purpose. No need to reference the studies.


%Selection of the ganging option will include passive or active solutions, where the active 
%circuitry may require cold components such as an amplifier in the \lar volume. Design options with active cold components will need 
%to address issues of power dissipation and potential risks of single-point failures of multi-channel devices inside the cryostat.
%In the case of passive ganging, analog signals are transmitted outside of the cryostat for processing and digitalization. 
%Successful demonstrations of passive ganging at \lar temperatures have been made for groups of four and twelve 6x6 mm Micro-FC-60035C-SMT C series, and groups of 2, 4, 8, and 12  Hamamatsu \dwords{mppc} (S13360-6050PE) at  \num{25}$^\circ$C, - \num{70}$^\circ$C and \SI{77}{K}. 
%Active ganging has been demonstrated for an array of 12 sensL \num{4}$\times$\num{4} arrays of \SI{3}{mm}$\times$\SI{3}{mm} sensL C-series \dwords{sipm} (48 in all) and  72 \dwords{sipm} mounted in a hybrid combination of passive and active ganging using \SI{6}{mm}$\times$\SI{6}{mm} \dwords{mppc} with a low noise operational amplifier--this design combines 12 active branches into the op-amp, where each branch has six \dwords{mppc} in a parallel passive-ganging configuration.

\subsection{SiPM Signal Ganging}
\label{sec:pds-design-ganging}
%{Zelimir added this subsection}

%Light collector techniques require electrical summing (ganging) of the \dwords{sipm} in order to maximize the active area of the photosensors while keeping the channel count reasonably low. 
The ganging of electrical signals from \dword{sipm} arrays is implemented to minimize the electronics channel count while maintaining adequate redundancy and granularity, as well as to improve the readout system performance.  
Technical factors that affect performance of the ganging system are the characteristic capacitance of the \dword{sipm} and the number of \dwords{sipm} connected together, which together dictate the \dword{s/n} and affect the system performance and design considerations.

We have demonstrated a feasible purely passive summing scheme with twelve Hamamatsu \dword{mppc} sensors now operational in \dword{pdsp}. For optimal performance in DUNE, we have shown that an ensemble of 48 Hamamatsu \SI{6}{mm}$\times$\SI{6}{mm} \dwords{mppc} can be summed into a single channel by a combination of passive and active ganging (see Section~\ref{sec:pds-valid-ganging}).  In this scheme, an amplifier is used to adjust the \dword{mppc} output signal level to the input of an \dword{adc}; the active summing is realized with an OpAmp THS4131. This combination of passive and active ganging with cold signal summing and amplification, illustrated in Figure~\ref{fig:fig-pds-6x8gang}, is the baseline for the \dword{pds}.

\begin{dunefigure}[SiPM signal summing board circuit]
 {fig:fig-pds-6x8gang}
 {\dword{sipm} signal summing board circuit: 6 passive  x 8 active, 48 \dwords{sipm} total.}
\includegraphics[height=12cm]{graphics/pds_gang_fig2.png}
\end{dunefigure}
%\fixme{Gustavo - he will try to improve resolution on this or replace with a drawing.}

%\fixme{\color{blue}Gustavo: Need a figure with the circuit showing the DUNE 6 passive x 8 active configuration - GUSTAVO WILL PROVIDE}

%In addition to the baseline scheme, a parallel effort is underway to investigate further optimization including development of detailed simulation and testing of prototype boards in South America. One scheme being investigated simulates the signals of 12 passively ganged \dwords{sipm} passed through a charge integrator or a charge amplifier transimpedance model into a summing stage. The simulation flow and the simulated response are shown in 
%Figure~\ref{fig:fig-pds-gang-jorge-1}(left) shows design scheme used in the simulation flow presented in 
%Figure~\ref{fig:fig-pds-gang-jorge-1}. Figure~\ref{fig:fig-pds-gang-jorge-2} shows the circuit designed based on these studies to be used with the \dword{pd} in the \fnal ICEBERG test stand (Section~\ref{sec:iceberg-teststand}).

%\begin{dunefigure}[Active summing board design simulation flow.]
% {fig:fig-pds-gang-jorge-1}
% {Active summing board design simulation flow (left) and simulated response of the circuit with 48 \dwords{sipm}(right).}
% {The active summing board design scheme (left) used in the simulation flow (right).}
%\includegraphics[angle=0,width=8.4cm,height=6cm]{graphics/pds-gang-jorge-1.png}
%\includegraphics[angle=0,width=10.4cm,height=5cm]{graphics/pds-gang-jorge-2.png}
%\includegraphics[angle=0,height=5cm]{graphics/pds-gang-jorge-2.png}
%\includegraphics[height=5cm]{graphics/pds-gang-jorge-3.png}
%\end{dunefigure}

%\begin{dunefigure}[Active ganging circuit to be used in the ICEBERG test stand.]
% {fig:fig-pds-gang-jorge-2}
 %{Active ganging circuit to be used in the ICEBERG test stand.}
%\includegraphics[angle=0,width=8.4cm,height=6cm]{graphics/pds-gang-jorge-3.png}
%\includegraphics[angle=0,width=8.4cm,height=5cm]{graphics/pds-gang-jorge-4.png}
%\end{dunefigure}

%The development framework presented above presents a very powerful tool in the design of the electronics. It shows that is possible to gang 48 \dwords{sipm} and distinguish single photon signals with less than 1 $\mu$s width.
%.For single photons there is no significant difference between both models in duration of the pulse and S/N ratio. Preliminary simulations indicates that for 10 photons we don’t see any difference either.
%The \dword{s/n} ratio obtained is about \SI{8}{dB}, with all noise effects, including thermal and dark noise contributions.

%\{add more in validation section?}
%5.The optimal sampling rate obtained is >≈ 20 MSPS.
%6.The design of the board for the ICEBERG test stand is ready and in process of fabrication. It includes both designs in the same board, that can be easily exchanged.

%This solution is now used as the baseline summing option for DUNE Far Detector.
%next steps?:This board will also measure the photoelectron collection efficiency when the \dwords{sipm} are coated with \dword{tpb} as a reference for ARAPUCA measurements with a similar ganging level (the summing  board is the same size as the \dword{pdsp} ARAPUCA backplane to facilitate the comparisons). 

%Charge processing requires a charge preamplifier ideally located within the cold environment, so the design must take into consideration the failure risks and the power dissipated into the environment.

%{rjw 11/23/2018: Here should be a description of the Gustavo's and Paraguay groups ganging scheme options (results are in the validation section) - can we identify the FNAL scheme as baseline for the purpose of the TDR draft? Perhaps in the section add the circuit design and drawing of the ganging circuit board that could go in a 10kt production module.}
%{Zelimir added ganging from Gustavo, and from Jorge base on Jorge's 30\% review slides.}

%Typically, arrival time and total charge are the key parameters to be obtained from a detector. Extraction of these parameters 
%is possible using analog or digital systems. Charge preamplifiers will be connected to the output of the detector to integrate current producing a charge proportional output. 

% => this to next steps: In both systems, performance parameters related  to sampling rate, number of bits, power requirements, signal to noise ratio, and interface requirements should be evaluated to arrive to selected solution.  
%Pulse shapes can be fully analyzed to improve detection of a new physics but it will have an important impact on the digitalization frequency.

%**********************************************************************************


%%%%%%%%%%%%%%%%%%%%%%%%%%%%%%%%%%%%%%%%%%%%%%%%%%%%%%%%%%%%%%%%
\subsection{Front-end Electronics Baseline Design}
\label{sec:electronics}
%\{11/28/18  Josh S. text/figures on Mu2e}
%\fixme{revise following - the abbreviation SSP appears before the definition and reference. Specific institutions are not mentioned elsewhere...}

%revised text rjw 6/25/19 Give the reference to SSP here.
The front-end electronics for the development and prototype stages of the  \dword{pds}, including ProtoDUNE-SP, was provided by a custom-designed \dword{sipm} Signal Processor (\dword{ssp}, see \citedocdb{3126}). This system was highly configurable and provided detailed information on the photosensor signal, which allowed a thorough understanding of the photon system performance.
For the much larger DUNE \dword{fd}, a system is required that meets the performance requirements yet optimizes the cost.
%Following the successful design, fabrication, operation and performance of \dfirst{ssp} readout in \dword{pdsp}, and with initial high-quality beam and cosmic ray day data collected by the \dword{pdsp} photon system, we are further cost-optimizing the readout electronics.  
To this end, we have developed a solution based on lower sampling rate commercial ultrasound \dword{asic} chips rather than digitizers based on flash \dwords{adc} used in the \dword{ssp}. Inspiration for this cost-effective front-end comes from the system developed for the \dword{mu2e} experiment cosmic ray tagger readout system.
Both \dword{ssp} and the new design are used in the photon detector validation process summarized in Section~\ref{sec:fdsp-pd-validation} to allow direct comparison.

%\forlbnc{LBNC: Text has been added about electronics development plan.}

Development of the readout electronics to date has been primarily by US groups. 
However, since fabrication of the DUNE readout electronics will be conducted by a collaboration of Latin American institutions (including groups in Peru, Colombia, Paraguay and Brazil), further development is being performed by these groups, with support from the US groups.  Engineers from several of the groups will meet at Fermilab in the summer of 2019 to continue the development of the system. The \dword{iceberg} facility at Fermilab will provide a realistic system test of the prototypes.  Pre-production \dword{mu2e}-based electronics readout will also be tested in the \dword{pdsp2} run at \dword{cern} in 2021-2022.


%Development of the readout electronics has been led by US groups, including Argonne National Labs (the custom-designed \dword{sipm} Signal Processor~\citedocdb{3126}, \dword{ssp}) development, demonstrating the concept of wave-form digitization of \dword{sipm} signals over approximately \SI{30}{m} cable lengths with single-PE resolution), and a collaboration of Fermilab and the University of Michigan (demonstrating a low-cost modification to this readout plan, using low-cost commercial ADCs and FPGAs).  
%Development efforts to date have included the 35T and \dword{pdsp} (for the SSP system) and \dword{iceberg} (for the \dword{mu2e}-based readout system).  Since fabrication of the readout electronics will be conducted by a collaboration of Latin American institutions, including groups in Peru, Colombia, Paraguay and Brazil (with support from US groups in an advisory roll), further development of this system will involve a collaboration of all these groups.  Engineers from several of these groups will meet at Fermilab in the summer of 2019 to continue the development of the system, with testing focused at the \dword{iceberg} test facility.  Final pre-production \dword{mu2e}-based electronics readout will also be tested in the \dword{pdsp2} run at \dword{cern} in 2021-2022.

\subsubsection{Front-end Board and Controller}

The readout and digitization of the signals from the active summing board described in Section~\ref{sec:pds-design-ganging} will rely on a set of front-end board (\dword{feb}) readout electronics boards and controller boards, originally designed for the \dword{mu2e} experiment~\cite{bib:mu2e_tdr}
at Fermilab. As discussed in Section~\ref{sec:pds-valid-ganging}, preliminary results indicate that the active-summing board and \dword{mu2e} electronics \dword{feb} combination will perform well together and, in general, meet the readout requirements for the experiment. The 64~channel \dword{feb} design carried over from \dword{mu2e} can be seen in Figure~\ref{fig:feb}. The board has a number of notable features, discussed below. Most importantly,  
the board is designed to utilize commercial, off-the-shelf parts only, and is therefore quite inexpensive compared to other designs. In particular, for the digitization it uses the low-noise, high-gain, and high-dynamic-range commercial \dwords{adc} used in ultrasonic transducers. 

The \dword{feb} is the centerpiece of the baseline readout electronics system.  
The current 64~channel\footnote{This text assumes 64 channels/\dword{feb} when presenting the \dword{feb} and controller. However, we envision 40~channels/\dword{feb} in the final design, corresponding to a single \dword{apa} as described in Section~\ref{subsec:pds-fe-next}.} \dword{feb} relies on commercial ultrasound chips\footnote{Texas Instruments\texttrademark{} 12~bit, 80~MS/s; AFE5807.},
%fixme{check if "next steps" section still there after rearrangement}  
with programmable anti-alias filters and gain stages, to read out the \dword{mppc} signals from the active ganging boards inside the \dword{pd} modules. The board currently takes HDMI inputs, with four channels per input.  Each of the eight ultrasound chips on an \dword{feb} handles eight channels (\SI{120}{mW} per channel) of data using a low-noise preamplifier, a programmable gain amplifier, and a programmable low-pass filter. The information is buffered with a total of \SI{1}{GB} DDR SDRAM, divided in four places, and a set of Spartan~6\texttrademark~\dwords{fpga} are used for parallelizing the serial \dword{adc} data, zero suppression, and timing. Each of the four \dwords{fpga} on a board, corresponding to 16 channels, handles two \dword{adc} chips with an available \SI{256}{MB} DDR SDRAM. 

After digitization, the data from each \dword{feb}, in the form of pulses (time-stamp and pulse height) is sent to a master controller via Ethernet, which aggregates the signals from 24 \dword{feb}s, or $64\times24=1536$ channels. The 24 \dword{feb}s corresponding to a single controller will come in sets of 12, with each set of 12 \dword{feb}s referenced to a single chassis as shown in %. A picture and schematic of the controller is seen in 
Figure~\ref{fig:controller}. A trigger decision (e.g., accelerator timing signal) can be produced and/or received by the controller and, depending on the decision, each event's digital information is sent to the controller and then to \dword{daq} computers for processing and storage. The controller-to-\dword{daq} connection will rely on a fiber connection, although an Ethernet-based controller output option is available.

% {Since we cannot use this feature (next paragraph) is it worth mentioning? would it need to be removed by the redesigned \dword{feb}? Is that step scheduled?} - remove the follow para

% In addition to its digitization and decision-making abilities, the current \dword{feb} has a number of other notable features. For example, each \dword{feb} contains an onboard Cockcroft-Walton (CW), which is available to generate bias voltage for the \dwords{mppc}. Notably, however, the CW is not currently considered an option for the summing board as it isn't stable enough to bias the differential amplifier. The CW is capable of providing $\approx$+75 V of bias. The board can also apply $\approx$-3~V to the \dword{mppc} anode, allowing the possibility of 78~V. In addition to the \dword{mppc} bias, the \dword{feb}s allow for an on-board current measurement (\SI{100}{pA} resolution) for producing IV curves of \dwords{mppc}. The \dword{feb}s can also be used in stand-alone mode, without a controller, via a telnet interface, which is useful for R\&D purposes and debugging. 
 
\subsubsection{Bandwidth, readout rates, and zero suppression}

\dword{daq} system and data storage limitations impose constraints on the  data flow from the front-end electronics system. 
For example, if it were necessary to read out a \SI{5.5}{\micro\second} waveform in order to include more of the longer time constant scintillation light component, the \SI{80}{MS/s}, 12-bit \dword{adc} device would produce a 5.3~kbit waveform. For an envisioned dark count (DC) rate of 250~Hz/channel, this corresponds to a data transfer rate of \SI{53}{Mb/s}/\dword{apa} (1~\dword{apa}=40 channels) or \SI{6.6}{MB/s} \dword{feb}-to-controller DC rate. This rate approaches the crucial bottleneck in the electronics readout system with a maximum rate of \SI{10}{MB/s} (per \dword{feb}). However, zero-suppression techniques and multi-channel coincidence/threshold requirements at the \dword{feb} firmware level can be used to significantly mitigate this issue, noting that each on-\dword{feb} \dword{fpga} handles 16 channels. 

The design is flexible enough to accommodate modest changes in system requirements, such as the suppression factor determined by parameters like the readout window length and limits on the overall trigger rate. 
Firmware and zero-suppression technique development is in progress and can easily adapt to the physics and calibration requirements of the \dword{pd}.
In addition to its bandwidth and DC rate readout capabilities, the system can also  manage a highly-coincident event in which a large number (or all) channels fire at once. For example, even a highly unlikely ``all-detector'' event featuring 6000~channels firing at once (corresponding to \SI{4}{MB} event size) could be handled by the controller's 24-board write speed of \SI{150}{MB/s}. 

The baseline electronics readout system performance is consistent with the \dword{daq} interface specification of \SI{8}{Gb/s} per connection, given that
each \dword{feb} signal corresponds to a maximum of \SI{10}{Mb/s} (\SI{240}{Mb/s} total).  

%\fixme{AIH?: (previous sentence) LBNC It seems that the PDS has a lot more knowledge of the DAQ specifications than I recall from the DAQ TDR!}

%\forlbnc{With respect to question on DAQ specs: 8 Gbps/APA was the data rate limit we were given in conversation with the DAQ. We cannot speak to this specification being missing from the DAQ TDR.}

\subsubsection{Power, grounding, and rack schemes} 

Figure~\ref{fig:grounding_power} shows the grounding, power, and data link schemes for the system. The \dword{feb}s are powered via power-over-Ethernet (\SI{600}{mA}, \SI{48}{V} supply) from the controller. One Cat-6 cable from the controller to each \dword{feb} handles the signal and power simultaneously. The reference planes of the controller and \dword{feb} are isolated on both sides. The grounding scheme calls for each set of twelve \dword{feb}s referenced to a single chassis, with each chassis and corresponding controller on detector ground and the \dword{daq}, connected to each controller via fiber, on building ground. 
 
The rack space and power consumption required by the system assume
a total of 6000 channels with 40 channels/\dword{feb}. This system requires 13 chassis (12 \dword{feb}/chassis) at 6U each and seven controllers (controlling 24 \dword{feb} each) at 1U each; these can be accommodated in just over two 42U capacity racks. The power supply on a controller is \SI{700}{W}, with each \dword{feb} taking \SI{20}{W}. 
 

\begin{dunefigure}[\dshort{pds} 64-channel front-end board]
 {fig:feb}
 {Photograph of the 64-channel \dword{pds} front-end board (\SI{80}{MS/s}, \SI{12}{bit} ADC) (left); schematic of the front-end board (right).}
\includegraphics[height=4.8in]{graphics/pds-feb-tdr.pdf} 
\vspace{-6.3cm}
\end{dunefigure}

\begin{dunefigure}[\dshort{pds} front-end electronics controller module]
 {fig:controller}
 {The front (left-bottom) and back views of the controller module (left-top), which is capable of accepting signals from 24 \dword{feb}s; schematic of the controller (right).}
\includegraphics[height=5.0in]{graphics/pds-controller.pdf} 
\vspace{-5.5cm}
\end{dunefigure}

\begin{dunefigure}[\dshort{pds} front-end electronics grounding scheme]
 {fig:grounding_power}
 {Grounding scheme with 1 chassis, containing 12 \dword{feb}s, a controller module, and a \dword{daq} PC, as an example (left); power and data link arrangement of the \dword{feb} and controller (right).}
\includegraphics[height=4.8in]{graphics/pds-grounding-power.pdf} 
\vspace{-7.1cm}
\end{dunefigure}


%\begin{figure}[h]
%\begin{centering}
%\includegraphics[height=4.8in]{graphics/pds-grounding-power.pdf} 
%\vspace{-7.1cm}
%\caption{(Left) The envisioned grounding scheme with 1 chassis, containing 12 \dword{feb}s, a controller module, and a \dword{daq} PAC, as an example. The power and data link arrangement of the \dword{feb} and controller.}
%\label{fig:grounding_power}
%\end{centering}
%\end{figure}
%%%%%%%%%%%%%%%%%%%%%%%%%%%%%%%%%%%%%%%%%%%%%%%%%%%%%%%%%%%%%%%%%%%%%%%

\subsection{Electronics Next Steps}
\label{subsec:pds-fe-next}
%{Zelimir moved sections around 11/29/2018, and also made a pass through the text below.}
%{updated by Toups 11/28/2018}

%Toups start - 11/28/2018
Although the \dword{feb}s developed for the Mu2e cosmic ray veto and proposed here for use in DUNE have
demonstrated the capacity to read out an array of \dwords{mppc} with an adequate \dword{s/n} ratio to provide sensitivity to single photons, there are a number of optimization and development tasks that are being pursued:  

\begin{enumerate}
\item To better match the 40 \dword{pdsp} channels per \dword{apa}, the system presented here assumes that only 40 out of the 64 channels on the existing Mu2e \dword{feb} are populated with active electronics.  A prototype board will test this configuration and validate the associated cost model.

\item The Mu2e warm readout electronics use last-generation (Xilinx\texttrademark{} Spartan-6) \dwords{fpga} and other components that have since been superseded by newer devices.  Design and prototyping work will 
incorporate newer \dwords{fpga} (Xilinx Spartan-7 or Artix-7) into the electronics,
improving their performance and maintainability over the lifetime of the DUNE experiment. The Artix-7 \dwords{fpga} have been implemented in the \dword{ssp} readout system used in \dword{pdsp} and therefore the expertise with these system components has been established. 

\item Results from the ICEBERG test stand can determine whether there are sufficient logic resources in the \dwords{fpga} to meet a broad range of possible \dword{daq} requirements expected from the warm readout electronics. To that end the low-cost front-end solution will be compared to existing 14-bit, 150 MS/s \dword{ssp} readout.  Straightforward zero suppression schemes that can be implemented on the Mu2e board with the current Spartan-6 \dword{fpga} will be tested with respect to potential \dword{daq} data rate limitations.  However, increases in the number of logic cells can be accommodated by switching to more capable, but still pinout-compatible, devices within the same Xilinx \dword{fpga} family as discussed above.

\item It may be desirable to increase the dynamic range of the \dwords{adc} used on the \dword{feb}s in order to achieve desired physics goals related to the energy resolution of beam neutrino events.  To this end, we plan to investigate replacing the TI AFE5807 ultrasound chip with the TI AFE5808 ultrasound chip, which is pinout-compatible but incorporates a 14-bit \dword{adc}.  Ultimately, a prototype board will incorporate all relevant optimizations and improvements.

\end{enumerate}
%Toups end - 11/28/2018
%Although the requirements for the electronics system are not all fully established, it not expected that the system requires novel high-risk techniques and can be developed and fabricated well within the schedule for the \dword{pds}.
The final requirements for the system will be informed by analysis of the data from the readout system implemented in \dword{pdsp} and subsequent data from ICEBERG expected in summer 2019.

%\forlbnc{LBNC: Text has been added here with respect to when decisions are needed on this.}

Additional testing of the system will continue through \dword{pdsp2} operations. The specifications for the readout electronics system will be reconsidered based on that experience and established before the \dword{pd} final design review (see the high-level schedule in Section~\ref{sec:fdsp-pd-org-cs}). 

%production readiness review prior to beginning final readout electronics production, as shown in the high-level schedule.

%As identified in Section~\ref{sec:fdsp-pd-ps}, the most important near term R\&D program will be to optimize the ganging scheme including choice of \dword{sipm} and cable types. 
%The first objective is to demonstrate that an ensemble of \numrange{48}{72} Hamamatsu \SI{6}{mm}$\times$\SI{6}{mm} \dwords{mppc} can be summed into a single channel by a combination of passive and active ganging. This board will also measure the photoelectron collection efficiency when the \dwords{sipm} are coated with \dword{tpb} as a reference for ARAPUCA measurements with a similar ganging level (the summing  board is the same size as the \dword{pdsp} ARAPUCA backplane to facilitate the comparisons).
%Charge processing requires a charge preamplifier ideally located within the cold environment, so the design must take into consideration the failure risks and the power dissipated into the environment.

%The timing resolution, minimum threshold and dynamic range requirements for the system are dictated by the physics requirements. These are well known for the higher energy physics (>\SI{200}{MeV}) but, as noted elsewhere in this document, are still evolving for lower energy. Currently,  a timing resolution of 1$\mu$s is called for and the sampling rate and number of sample bits is estimated based on this. For this task some digital process such as a sample interpolation may be proposed, enhancing the recorded raw sample time precision.
%The light sensitivity and the dynamic range requirement will determine the number of bits and the sample rate required by either waveform or charge collection methods. In both cases, the signal to noise ratio and the power consumption must be estimated.  
%With this data from \dword{pdsp} and the ganging studies, the choice between waveform readout and integrated charge readout will be made taking into account \dword{daq} readout and trigger requirements and, in general, the physics requirements of the experiment. 
% rjw 12/3/18 Per email from Matt Toups - previous paragraph not needed
%{rjw Is this still an open question for the baseline?}


%%%%%%%%%%%%%%%%%%%%%%%%%%%%%%%%%%%%%%%%%%%%%%%%%
\section{Calibration and Monitoring}
\label{sec:fdsp-pd-CandM}
%\metainfo{\color{red}\bf  Content: Djurcic}
%\fixme{rjw 7/16/19 Waiting for information from Zelimir. Want to add more detail such as the number/size/spacing of diffusers, coverage of light on the APA, eng drawing of diffusers on teh CPA (as in a ProtoDUNE docDB note)}
%\fixme{rjw 7/17/19 incorporating info from Zelimir} 
% 7/17/19 Some details added by rjw from ZD's CERN workshop talk and an email

%{New section on the Calibration and Monitoring system design here - provided by dj 11/25/18 - thanks!} 
%{Note there is a section in the Production and Assembly needed too.}

%%%%%%%%%%%%%%%%%%%%%%%%%%%%%%%%%%%
% provided by dj 11/25/18
% Note: The section on protodune experience in the file is moved to the pds-validation section
%\section{Calibration and Monitoring}
%\label{sec:fdsp-pd-calib}

%\subsection{Introduction}

%\fixme{\color{blue}Zelimir: need a figure showing the design elements. - rjw}
%add this in the next round rjw  1/18/19 \fixme{Need a figure showing the design elements. - rjw}
%\fixme{\color{blue}Zelimir: specify cpa panel, plane, etc. There isn't just a cpa. - Anne}
%Warm components include controlled pulsed-UV source (\SIrange{245}{280}{nm}) and warm optics. These warm components will interface with \dword{cisc} and \dword{daq} subsystems.
%describe Cali Module. Describe what calibrations could be done
%\forlbnc{LBNC: LCM and LPM descriptions added.}
%\fixme{Bob--  I edited this to include some of what Zelimir sent us.  You might want to shorten it again.}

%\begin{dunefigure}[\dshort{pdsp} \dshort{pd} calibration and monitoring system]
% {fig:pds_calmon_fig1}
% {Schematic of the \dword{pdsp} \dword{pd} calibration and monitoring system.\fixme{Replace figure}}
%\includegraphics[angle=0,width=8.4cm,height=6cm]{graphics/pds-calmon-fig1.png}
%\end{dunefigure}
%\fixme{rjw 7/18/19 - add justification vis a vis muon rate per text from Zelimir}

Calibration and monitoring is an essential component of the \dword{pds}.
The primary system is a pulsed UV-light source that will allow calibration of the \dword{pd} gain, linearity, and timing resolution, and to monitor the stability of the photon response of the system over time.
In many experiments, a pulsed light system is a valuable well-defined, controllable light source for monitoring but for (near) surface detectors it is often just a supplement to using tracked cosmic ray muons, which provide a much closer replica of the signal from events of interest. 
However, at DUNE the muon rate per individual photon detector will be very low and insufficient to monitor changes in the system response. In this situation, the pulsed system will play an essential role in achieving and maintaining the \dword{pd} performance required for neutrino calorimetry. 
This system will also be a valuable detector commissioning tool prior to sealing the cryostat, in the cool-down phase, and during the \lar fill.
Other complementary calibration systems, such as radioactive sources, are described in the calibration chapter (\spchcalib). 

The system design is almost identical to that deployed in \dword{pdsp}, as described in Section~\ref{sec:fdsp-pd-validation-candm}; the primary differences are is the number of diffusers, the lengths of the optical fibers, and the addition of a monitoring diode.

The system hardware consists of both warm and cold components but has no active components within the cryostat. The active component consists of a 1U rack mount light calibration module (\dword{lcm}) located outside the cryostat. The \dword{lcm} generates UV (\SIrange{245}{280}{nm}) pulses that propagate through a quartz fiber-optic cable to diffusers at the \dword{cpa} that distribute the light uniformly across the photon detectors mounted within the \dword{apa}. 
It consists of an \dword{fpga}-based control logic unit coupled to an internal \dword{led} pulser module (\dword{lpm}) and an additional bulk power supply. 
The \dword{lpm} has multiple digital outputs from the control board to control the pulse amplitude, pulse multiplicity, repetition rates, and pulse duration; programmable \dwords{dac} control the \dword{lpm} pulse amplitude. \dword{adc} channels internal to the \dword{lcm} are used to read out a reference photodiode used for pulse-by-pulse monitoring of the \dword{led} light output. The output of the monitoring diode is available for normalizing the response of the \dwords{sipm} in the detector to the monitoring pulse. 


\begin{dunefigure}[Calibration system diffuser locations on the SP CPA]
 {fig:pds-calmon-cpa-diffusers}
 {Schematic of a complete SP cathode plane ($\SI{60}{m}\times\SI{12}{m}$) showing the locations of the calibration and monitoring system diffusers. Each diffuser illuminates a region of about $\SI{4}{m}\times\SI{4}{m}$ (indicated by the squares) on APAs \SI{3.6}{m} away.}
\includegraphics[angle=0,width=0.9\textwidth]{graphics/pds-calibration-cpa}
\end{dunefigure}

Quartz fibers, \SIrange{10}{30}{m} long, are used to transport light from the optical feedthrough (at the cryostat top) through the \dword{fc} ground plane, and through \dword{fc} strips to the \dword{cpa} top frame. 
These fibers are then optically connected to diffusers located on the \dword{cpa} panels using fibers that are \SIrange{2}{10}{m} long. 
The diffusers, \SI{2.5}{cm} in diameter, are mounted on the cathode plane panels acting as light sources to illuminate \dwords{pd} embedded in the \dword{apa}s. There will be \num{45} diffusers uniformly distributed across each of the SP module cathode planes facing APAs, as indicated in 
Figure~\ref{fig:pds-calmon-cpa-diffusers}. Each diffuser will illuminate an area of approximately $\SI{4}{m}\times\SI{4}{m}$ on the APAs that are \SI{3.6}{m} away. 

The diffusers reside at the \dword{cpa} potential, so the \dword{hv} system places a requirement on the fiber electrical resistance to protect the cathode from experiencing electrical breakdown along this path. This requirement is easily met by the fibers. 
%The fibers are excellent electrical insulators, mitigating any concern of breakdown along this path.  
%The optical feedthrough is a part of the cryostat interface.

As demonstrated in \dword{pdsp}, the system performs the required calibration and monitoring tasks with minimal impact on the TPC design and function. 
