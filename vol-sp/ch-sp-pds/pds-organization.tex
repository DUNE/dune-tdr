%%%%%%%%%%%%%%%%%%%%%%%%%%%%%%%%%%%%%%%%%%%%%%%%%%%%%%%%%%%%%%%%%%%%
\section{Organization and Management}
\label{sec:fdsp-pd-org}
%\metainfo{\color{red}\bf  Content: Segreto/Warner}

The single-phase photon detector consortium has and will continue to benefit from the contributions of many institutions and facilities in Europe and North and South America.  To help guide the interactions of these many contributors, the consortium is divided into six working groups, each with two or three conveners, to direct its activities.  % The structure of the organization is detailed below.

%%%%%%%%%%%%%%%%%%%%%%%%%%%%%%%%%%%
\subsection{Consortium Organization}
\label{sec:fdsp-pd-org-consortium}

The \single \dword{pds} consortium follows the typical organizational structure of DUNE consortia:
\begin{itemize}
\item A consortium lead
provides overall leadership for the effort, and attends meetings of the DUNE Executive and Technical Boards.
\item A technical lead 
provides technical support to the consortium lead, attends the Technical Board and other project meetings, oversees the project schedule and \dword{wbs}, and oversees the operation of the project working groups.  
%In the case of the \dword{pds}, the technical lead is supported by a deputy technical lead.
\end{itemize}

\begin{dunefigure}[PDS consortium organization chart]{fig:pds-org-chart}
{\dword{pds} consortium organization chart.}

	\includegraphics[angle=90,height=12cm]{pds-consortium-org-chart}
	
\end{dunefigure} 



Below the leadership, the consortium is divided up into six working groups, each led by two or three working group conveners as shown in Figure~\ref{fig:pds-org-chart}. 

%\{Note that we have added the P-DUNE analysis working group}

% the \dword{pds} Management Table of Organization.  
Each working group is charged with one primary area of responsibility within the consortium, and the conveners report directly to the technical lead regarding those responsibilities.  As the consortium advances to a more detailed \dword{wbs} and project schedule, it is envisioned that each working group will be responsible for one section of those documents.

The working group conveners are appointed by the \dword{pds} project lead and technical lead, and the structure may evolve as the consortium matures and additional needs are identified. 

%%%%%%%%%%%%%%%%%%%%%%%%%%%%%%%%%%
\subsection{Project Planning Assumptions}
\label{sec:fdsp-pd-org-assmp}

\fixme{I think this sub-section is redundant and can be removed.  It is all still here if you want to resurrect it.}
%\metainfo{\color{blue} Content: Segreto/Warner}

%Plans for the \dword{pds} consortium are based on the overall schedule for the DUNE \dword{fd}. In particular, the \dword{apa} schedule defines the time window for the completion of the final development program for the light collectors: A final down-select to a baseline light collector option and \dword{fe} electronics has been made for the TDR.  Due to the opportunities presented by our new Italian colleagues and their close connection with FBK, we may maintain an alternate photosensor option up to the pre-production review in September of 2020, but all other systems must be defined prior to the \dword{tdr}.

%The \dword{pds} modules will undergo final assembly and testing at the \dword{pds} assembly facility at UNICAMP, with an initial  assembly rate of approximately twenty modules per week, accelerating to forty modules per week in the second half of module fabrication.  

%The modules will be shipped from the fabrication facilities to the project storage facility, 
%to be integrated along with the \dword{ce} into the \dword{apa} frames and cold tested in a cryogenic test facility.  We plan for an initial rate of two \dwords{apa} per week, with the possibility of accelerating to four \dwords{apa} per week as production lessons are learned.  \Dword{pds} personnel will be present at the integration facility to oversee the installation and testing.

%Meeting this timeline requires that the development of the ARAPUCA system be aggressively pursued throughout 2018, with a goal of testing near-final prototypes in the late fall of 2018 and allowing technology comparisons between the ARAPUCA and the light guide technologies in winter of 2019.

%Additional development efforts will focus on:

%\begin{itemize}
%\item identifying and selecting reliable cryogenic photosensor (\dword{sipm}) candidates,
%\item reducing cost and optimizing performance of \dword{fe} electronics, and 
%\item solidifying \dword{pds} performance requirements from additional physics simulation efforts.
%\end{itemize}

%We assume that apart from these items, where rapid development is still required, most of the detector components to be delivered by the \dword{pd} consortium will require only minor changes relative to the \dword{pdsp} components. For this reason, modifications of these other detector components will be delayed until 2019, which will also help with the funding profile. Exceptions will be made for further development in test stands with regard to cabling studies, and for the interface engineering required to ensure satisfactory integration of the \dword{pd} with the \dword{apa} and \dword{ce}  systems.

%%%%%%%%%%%%%%%%%%%%%%%%%%%%%%%%%%%
%\subsection{WBS and Responsibilities}
%\label{sec:fdsp-pd-org-wbs}
%\metainfo{\color{blue} Content: Warner/Mualem}

%%%%%%%%%%%%%%%%%%%%%%%%%%%%%%%%%%
\subsection{High-Level Schedule}
\label{sec:fdsp-pd-org-cs}
%\metainfo{\color{blue} Content: Warner/Mualem}

\fixme{This is a standard table template for the TDR schedules.  It contains overall FD dates from Eric James as of March 2019 (orange) that are held in macros in the common/defs.tex file so that the TDR team can change them if needed. Please do not edit these lines! Please add your milestone dates to fit in with the overall FD schedule. And fix table captions and label, please. (Anne)}


\begin{dunetable}
[Consortium X Schedule]
{p{0.65\textwidth}p{0.25\textwidth}}
{tab:Xsched}
{Consortium X Schedule}   
Milestone & Date (Month YYYY)   \\ \toprowrule
Technology Decision Dates &      \\ \colhline
Final Design Review Dates &      \\ \colhline
Start of module 0 component production for ProtoDUNE-II &      \\ \colhline
End of module 0 component production for ProtoDUNE-II &      \\ \colhline
\rowcolor{dunepeach} Start of \dword{pdsp}-II installation& \startpduneiispinstall      \\ \colhline
\rowcolor{dunepeach} Start of \dword{pddp}-II installation& \startpduneiidpinstall      \\ \colhline
 \dword{prr} dates &      \\ \colhline
Start of  (component 1) production  &      \\ \colhline
Start of (component 2) production  &      \\ \colhline
Start of  (component 3) production  &      \\ \colhline
\rowcolor{dunepeach}South Dakota Logistics Warehouse available& \sdlwavailable      \\ \colhline
\rowcolor{dunepeach}Beneficial occupancy of cavern 1 and \dword{cuc}& \cucbenocc      \\ \colhline
\rowcolor{dunepeach} \dword{cuc} counting room accessible& \accesscuccountrm      \\ \colhline
\rowcolor{dunepeach}Top of \dword{detmodule} \#1 cryostat accessible& \accesstopfirstcryo      \\ \colhline
End of  (component 1) production  &      \\ \colhline
... & ...                       \\ \colhline
\rowcolor{dunepeach}Start of \dword{detmodule} \#1 TPC installation& \startfirsttpcinstall      \\ \colhline
\rowcolor{dunepeach}End of \dword{detmodule} \#1 TPC installation& \firsttpcinstallend      \\ \colhline
\rowcolor{dunepeach}Top of \dword{detmodule} \#2 accessible& \accesstopsecondcryo      \\ \colhline
 \rowcolor{dunepeach}Start of \dword{detmodule} \#2 TPC installation& \startsecondtpcinstall      \\ \colhline
\rowcolor{dunepeach}End of \dword{detmodule} \#2 TPC installation& \secondtpcinstallend      \\ \colhline

last item & ...                         \\
\end{dunetable}



%The high-level schedule for the photon detector consortium through submission of the \dword{tdr} at the end of Q2 in FY19 is detailed in Figure~\ref{fig:pds-sched-to-tdr} and the pre-\dword{tdr} key milestones are listed in Table~\ref{fig:pds-pretdrkeymilestones}.

%\begin{dunefigure}[\dword{pd}S consortium schedule through to the \dword{tdr}.]{fig:pds-sched-to-tdr}
%{Photon Detector System consortium schedule through to the \dword{tdr}.}
 %%\includegraphics[width=1.0\columnwidth]{pds-sched-to-tdr.pdf}
%\end{dunefigure}

%\fixme{Dave W:  Will be updated in March 2019 per Eric.} Is on Dave's todo list. Comment added to the caption.

{\bf [Note to reviewers:  This section is a placeholder containing dates from the technical proposal known to be inaccurate.  Correct dates will be inserted into the final revision to be presented in July.]}

\begin{dunetable}[Pre-\dword{tdr} key milestones. Will be updated in March 2019.]
{ll}
{fig:pds-pretdrkeymilestones}
{Pre-\dword{tdr} key milestones. Will be updated in March 2019.}
Milestone													&	Date	       \\ \toprowrule
Preliminary \dword{pd} technology selection criteria determined				&	03/21/18	\\ \colhline
Results from final prototype light collector studies available			&	02/21/19	\\ \colhline
Final \dword{pd} technology selection criteria available						&	02/21/19	\\ \colhline
Down-select to primary (and potential alternate) light collector technology	&	02/22/19	\\ \colhline
Submit initial \dword{tdr} draft for internal review							&	03/29/19	\\ 
\end{dunetable}

High-level post-\dword{tdr} milestones are listed in Table~\ref{fig:pds-posttdrkeymilestones}.

%LMM fixed detector -> Module below and 10 kt -> \SI{10}{kt}

\begin{dunetable}[Post-\dword{tdr} key milestones.]
{ll}
{fig:pds-posttdrkeymilestones}
{Post-\dword{tdr} key milestones.}
Milestone											&	Date	       \\ \toprowrule
\dword{pd} pre-production review(s) complete					&	03/2020 	\\ \colhline
Initial \dword{pd} module fabrication begins						&	09/2020	\\ \colhline
Final \dword{pd} production review based on initial production \dword{qa}		&	02/2021	\\ \colhline
First  \dword{pd} modules delivered for installation				&	05/2021	\\ \colhline
Installation into \dwords{apa} begins							&	06/2021     \\ \colhline
\dword{pd} fabrication complete (first \dword{spmod})			&	07/2023	\\ 
\end{dunetable}

\subsection{High-Level Cost Summary}

\fixme{New cost table template to come early April. Anne}

{\bf [Note to reviewers:  This section is a placeholder containing tables which include costs known to be inaccurate.  Correct costs will be inserted prior to submission of the final revision in July.]}

In the fall of 2018 we completed an initial cost estimate for fabrication of \dword{pd} modules for one 10kt \dword{dune} detector.  The estimates were based on ProtoDUNE costs, modified as necessary for an \dword{xarapu} design.  Vendor quotations were used for most of the major components.  The biggest uncertainties in fabrication costs center around dichroic filter procurement and coating, which are currently based on initial contacts with Brazilian filter firms.   Samples have not yet been procured and tested.

%\fixme {DWW: Placeholder numbers (DWW)  NOT ACCURATEcurrently in table.  Will update in April 2019} Comment added to the caption.
\begin{dunetable}
[Single Phase Photon detector System Cost Summary.]
{p{0.7\textwidth}p{0.2\textwidth}}
{tbl:sppdcostsumm}
{Single Phase Photon Detector System Cost Summary. Will update in April 2019.}
Item & Core Cost (k\$ US) \\ \toprowrule
Design, Engineering and Development & \num{1.0} \\ \colhline
Production Setup & \num{1.0} \\ \colhline
Production & \num{1.0} \\ \colhline
Integration & \num{1.0}\\ \colhline
Installation & \num{1.0} \\ 

\end{dunetable}

%\fixme{Stolen from CE!  Need to change to pd-specific!DWW:  Labor placeholder table.  Will update in April 2019. }; on Dave's todo list. Comment added to the caption.

\begin{dunetable}
[Personnel needs for the \dword{pd} consortium. Will update in April 2019.]
{lrrrrr}
{tab:SPCE:personnel}
{Personnel needs (in FTE--years) for the construction of the \dword{pd} detector 
components, their integration and installation for different job categories and 
in different project phases. Will update in April 2019.}
Component & Students & Postdocs & Scientists & Engineers & Technicians \\
Management & & & & & \\ \colhline
Physics and simulations & & & & & \\ \colhline
\rowcolor{dunetablecolor}
\multicolumn{6}{c}{Design, Engineering and R\&D} \\ \toprowrule
Light collectors & & & & &  \\ \colhline
Photosensors & & & & &  \\ \colhline
Electronics, cabling, monitoring & & & & & \\ \colhline
Integration \& installation tooling & & & & & \\ \colhline
\rowcolor{dunetablecolor}
\multicolumn{6}{c}{Production Setup} \\ \toprowrule
Light collectors & & & & &  \\ \colhline
Photosensors & & & & &  \\ \colhline
Electronics, cabling, monitoring & & & & & \\ \colhline
Integration \& installation tooling & & & & & \\ \colhline
\rowcolor{dunetablecolor}
\multicolumn{6}{c}{Production} \\ \toprowrule
Light collectors & & & & &  \\ \colhline
Photosensors & & & & &  \\ \colhline
Electronics, cabling, monitoring & & & & & \\ \colhline
Integration & & & & & \\ \colhline
Installation & & & & & \\ 
\end{dunetable}
