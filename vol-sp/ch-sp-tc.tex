\chapter{Installation}
\label{ch:sp-tc}

This chapter covers all the work and infrastructure required to install an SP detector module. 
Before getting into the details, we first provide some reminders of the scale of the task, beginning with the two facts that drive all others: A DUNE FD module is enormous, with outer cryostat dimensions length$\times $width$\times $height$=$ $62\times 19\times 18$ m$^{3}$; and every piece of the FD module must travel $1500$ \si{m} down the Ross shaft to the 4850 level of SURF and be transported to the detector caverns.

For the \dword{tpc}, 150 \dword{apa}s, each $6.0$ m high and $2.3$ m wide and  weighing $600$ kg with $3500$ strung sense and shielding wires, must be taken down the shaft as special \textquotedblleft slung loads\textquotedblright and moved to the area outside the \dword{dune} cryostat. 
The \dword{apa} are moved into a pre-prepared length$\times $width$\times $height$=$ $30\times 19 \times 17\ {\rm or}\ 10$ m$^{3}$ clean room where they are first be outfitted  with photon detector units and passed through a series of qualification tests.
Here two APAs are linked into a vertical 12 m high double unit, connect to readout electronics, receive a cold-test in place, and then move into the cryostat to be connected at the proper location on the previously installed detector support structure (DSS) and cabled up with the feedthroughs. 
In parallel the field cages that define the \dword{tpc} active volume must be installed with all their high voltage connections, along with  elements of the cryogenic instrumentation and slow control (CISC) and calibration systems.

After twelve months of detector component installation, which follows twelve months of detector infrastructure installation, the cryostat closes (with the last installation steps occurring in confined space accessed through a narrow human access port). 
Following leak checks, final electrical connection tests, and installation of the neutron calibration source, the process of filling the cryostat with $17,000,000$ kg of LAr begins.

From this terse summary, it is clear that installation requires meticulous planning and execution of thousands of tasks by well-trained teams of technicians, riggers, and detector specialists. 
High level requirements for these tasks are spelled out in Table \ref{tab:specs:just:SP-TC} and the text that follows. 
In all the planning and future work the requirement that subsumes all others in the installation process is safety safety. DUNE's goal is zero accidents resulting in personal injury, damage to detector components, or harm to the environment.

% This file is generated, any edits may be lost.

\begin{longtable}{p{0.14\textwidth}p{0.13\textwidth}p{0.18\textwidth}p{0.22\textwidth}p{0.20\textwidth}}
\caption{Specifications for SP-TC \fixmehl{ref \texttt{tab:spec:SP-TC}}} \\
  \rowcolor{dunesky}
       Label & Description  & Specification \newline (Goal) & Rationale & Validation \\  \colhline

\input{generated/req-SP-TC-01.tex} % 1
\input{generated/req-SP-TC-02.tex} % 2
\input{generated/req-SP-TC-03.tex} % 3
   
  \newtag{SP-TC-4}{ spec:apa-storage-sd }  & APA stroage at logistics facility in SD  &  na &  Store APAs during lag between production and installation &  Agree upon space needs. \\ \colhline
     % 4
   
  \newtag{SP-TC-5}{ spec:cleanroom-specification }  & Standard for ITF and installation cleanrooms  &  na &  Reduce dust (contains U/Th) to prevent induced radiological background in detector &  Monitor air purity \\ \colhline
     % 5
   
  \newtag{SP-TC-6}{ spec:cleanroom-uv-filters }  & UV filter in ITF and installation cleanrooms for PDS sensor protection  &  na &  Prevent damage to PD coatings  &  Visual or spectrographic inspection \\ \colhline
     % 6


\label{tab:specs:just:SP-TC}
\end{longtable}

Installation of the DUNE FD presents a multitude of hazards that includes  manipulation heavy loads in the tight spaces of the mine and detector module,  working at considerable heights above the floor, repeated utilization of large volumes of cryogens, multiple tests with high voltage, commissioning of a class IV laser system, and deployment of a high activity neutron source. Mitigation of these hazards begins with the strong professional on-site ES\&H\ teams of the Fermilab South Dakota Services Division (SDSD) and the host SURF lab.
All installation team members, both at the surface and underground, will undergo rigorous formal safety training that will be update at daily intervals. Any team member can stop work at any time for safety purposes. Further details of the overall DUNE safety plan are provided in Chapter 9 of the Technical Coordination Volume of the TDR. In addition, each section of this chapter provides further details on the evolving safety plan for installation. This plan has been informed by the successful safety experience of SURF with other underground experiments (e.g., LUX, Majorana Demonstrator, LZ), DUNE members in executing projects at other underground locations (e.g., MINOS at Soudan, Minnesota, USA), at other locations remote from major international laboratories (e.g., Daya Bay, China and No$\nu $a Far Detector (Ash River, Minnesota, USA), and at the home laboratories of both Fermilab and CERN.


% risk table values for subsystem SP-FD-JPO
\begin{longtable}{p{0.15\textwidth}p{0.13\textwidth}p{0.13\textwidth}p{0.28\textwidth}p{0.06\textwidth}p{0.06\textwidth}p{0.06\textwidth}} 
\caption{Specification for SP-FD-JPO \fixmehl{ref \texttt{tab:specs:SP-FD-JPO}}} \\
\rowcolor{dunesky}
ID & Risk & Label & Mitigation & Prob ability & Cost Impact & Sched ule Impact \\  \colhline
RT-JPO-001 & Personnel injury & jpo-person-injury & Follow established safety plans. & M & L & H \\  \colhline
RT-JPO-002 & Shipping delays & jpo-shipping-delay & Plan one month buffer to store  materials locally. Provide logistics manual. & H & L & L \\  \colhline
RT-JPO-003 & Missing components cause delays & jpo-missing-components & Use detailed inventory system to verify availability of  necessary components.  & H & L & L \\  \colhline
RT-JPO-004 & Import, export, visa issues  & jpo-import-visa & Dedicated \dword{fnal} \dword{sdsd}division will expedite import/export and visa-related issues. & H & M & M \\  \colhline
RT-JPO-005 & Lack of available labor  & jpo-labor-avail & Hire early and use Ash River setup to train \dword{jpo} crew. & L & L & L \\  \colhline
RT-JPO-006 & Parts do not fit together & jpo-cannot-assemble & Generate \threed model, create interface drawings, and prototype detector assembly. & H & L & L \\  \colhline
RT-JPO-007 & Cryostat damage & jpo-cryostat-damage & Use cryostat false floor and temporary protection. & L & L & M \\  \colhline
RT-JPO-008 & Weather closes SURF & jpo-weather-delay & Plan for \dword{surf} weather closures & H & L & L \\  \colhline
RT-JPO-009 & Detector failure during \cooldown & jpo-cooldown-failure & Cold test individual components then cold test \dword{apa} assemblies immediately before installation. & L & H & H \\  \colhline

\label{tab:risks:SP-FD-JPO}
\end{longtable}


As part of the \dword{dune} design process the detector components and the TPC have been prototyped at various stages. The \dword{protodune}-SP prototype which was assembled from full scale dune components has recently been completed and has taken data. 
This process has been extremely important in planning the far detector installation. A detailed list of lessons learned from \dword{protodune}-SP construction and installation was compiled.\cite{bib:docdb8255} 
These lessons learned and other experience from the team planning the installation was used to develop a list of \textquotedblleft risks \textquotedblright for the \dword{dune} installation. 
The risk register documents what could be improved from the \dword{protodune} installation and what people felt could go wrong in the \dword{dune} installation. After the list of risks are generated plans to prevent the risks from occurring or minimizing the impacting to the project are formulated as mitigation strategies. These plans to improve what was done in \dword{protodune} or prevent difficulties at \dword{dune} are then built into the \dword{dune} installation plan. The list of risks associated with the \dword{dune} installation are shown in Table \ref{tab:INSTALL-risks}. The risks in Table \ref{tab:INSTALL-risks} are considered the highest impact risks where a dedicated mitigation strategy is required. All the lessons learned from \dword{protodune} will be factored into the detailed installation planning.

This  installation chapter is divided into three main sections describing the main divisions of work. First the logistics section describes how material will be delivered to the South Dakota region and then forwarded to the Ross headframe on the \dword{surf} site for transport underground. A warehouse facility is used for buffering materials prior to transport to \dword{surf}, inventorying shipments, consolidating packages, and coordinating with the  \dword{cf}-\dword{cmgc}. The second section describes all the infrastructure needed to install and operated the detector. This includes the cleanroom for installation and its contents but also infrastructure like racks, cable trays, storage facilities and machining facilities. The third section describes the actual installation process itself which is divided into 3 phases: the \dword{cuc} setup phase, the installation setup phase, and the detector installation phase. 


\section{Logistics}
\label{sec:fdsp-tc-log}


% orig 2nd draft Transporting equipment and people underground is one of the more challenging aspects of the \dword{lbnf}/\dword{dune} endeavor. Access underground goes through the mile long Ross shaft, which is now undergoing renovation. The shaft is outfitted with a single cage for people and materials and two skips that are needed to remove the rock underground. Planning for using the cage is important to making \dword{lbnf}/\dword{dune} a success. Given the enormous cost of the \dword{cf} contracts and the large cost of any inefficiencies in construction, the overall coordinator of the Ross Shaft for \dword{lbnf}/\dword{dune} will be the \dword{cf} \dword{cmgc}. Both \dword{lbnf} and \dword{dune} will also have a large number of contractors, institutions, and scientists who will need to bring equipment and materials underground. \dword{lbnf}/\dword{dune} will establish a logistics organization in South Dakota near \dword{surf} to facilitate the flow of material and people to the underground area. This organization will be responsible for receiving all goods for \dword{lbnf} (except CF) and \dword{dune},for coordinating the transport of this material underground with the \dword{cf}-\dword{cmgc}, and for coordinating personnel usage of the Ross cage with the \dword{cf}-\dword{cmgc}. 

Transport is one of the more challenging aspects of the \dword{lbnf} and \dword{dune} endeavor.  Access to the underground installation area for \dword{lbnf} and \dword{dune} personnel, equipment and materials will be provided by a single shaft, the mile-deep Ross Shaft, which is outfitted with a hoist that controls a cage for transporting people and materials.  
%
Given the enormous cost of the \dword{cf} contracts and the costs of inefficiencies in construction, scheduling use of the shaft is important to making \dword{lbnf} and \dword{dune} successful. A logistics organization is needed to ensure that all deliveries to the Ross Headframe arrive according to schedule. The \dword{lbnf} \dword{cf} \dword{cmgc} will coordinate overall usage of the Ross Shaft for the project. 

\dword{lbnf} and \dword{dune} will establish a logistics organization and lease a warehouse facility within a two-hour drive of  \dword{surf} to facilitate the flow of people and material to the Ross Headframe.  It is expected that this facility, referred to as the \dword{sdwf}, will include warehouse space, personnel and a \dword{wms} system for inventory.  A facility has not yet been selected. 
Most \dword{lbnf} and \dword{dune} material will be shipped to the \dword{sdwf}; \dword{cf} material, and likely cryogenics equipment, are exceptions and will ship directly to \dword{surf}. The logistics  organization will be responsible for (1) receiving and inventorying all  goods shipped to the \dword{sdwf}, and (2) coordinating with the \dword{cf}-\dword{cmgc}  to transport this material to the Ross Headframe in a just-in-time manner. Figure~\ref{fig:logistics-material-flow} shows a high-level overview of the material flow to the Ross Headframe.


%\begin{dunetable}
%[Logistics Specifications]
%{ll}
%{tab:table-Log-Req}
%{Table of logistics specifications.}
%Specifications &  \\ \toprowrule
%Material Handling & Comply with the SURF Material %Handling Specification \\ \colhline
%CMGS coordination & Provide CMGC with two-week notice %of shipments to SURF \\ \colhline
%Stage DUNE Shipments & Provide a one-month local %buffer of DUNE materials \\
%\colhline
%\dword{apa} Storage & Provide storage space for 150 %\dword{apa}s in a clean environment \\
%\colhline
%Inventory & Provide an inventory system accessible to %the collaboration \\
%\end{dunetable}
 %\fixme{Specifications table needs to be converted to official format}
 
% Freight is delivered to the \dword{sdwf}, except for possibly the cryogenic equipment. The materials are then transported in a just-in-time fashion to the Ross headframe where they are brought underground. 
 %For the detector, the \dword{apa}, electronics and \dword{pd} components %will %also be are shipped to the \dword{itf} where they are assembled and then shipped %back  to \dword{sdwf}.
 
\begin{dunefigure}[Material flow diagram for logistics ]{fig:logistics-material-flow}
  {Material flow diagram for the \dword{lbnf} and \dword{dune} logistics.}
 \includegraphics[width=\textwidth]{logistics-material-flow}
\end{dunefigure}

%%%%%%%%%%%%%%%%%%%%%%%%%%%%  
\subsection{Logistics Planning}
\label{sec:fdsp-tc-logPln}

\dword{lbnf} and \dword{dune} logistics oversees transportation of the cryostat (steel, foam, and membrane), the cryogenics system, the detector, and all related infrastructure not provided by %facilities. 
the \dword{cf}. \dword{lbnf} specifically oversees the cryostat and cryogenics system, which %will not be 
are  discussed in detail in %this 
the \dword{lbnf} \dword{tdr}; 
\fixme{add ref} because \dword{lbnf} material dominates the logistics, we present a summary. % is required. 
The %cryostat 
steel structure for each cryostat requires %bringing 
roughly 1,800 individual steel pieces, % underground, 
some of which weigh up to \SI{7.5}{t}, as well as \SI{125}{t} of bolts to assemble the steel pieces. The internal structure for each, which includes the foam insulation and the thin stainless steel membrane, %will 
requires transporting roughly 4,000 boxes, 
 each roughly 1.5 $\times$ 3.5 $\times$ 1.2 m$^3$. The plan for cryostat installation, at present, calls for all components to be warehoused at the \dword{sdwf} before installation begins. %This means that t
This facility will need to have %The logistics operation will  therefore need 
roughly $\SI{5,000}{m^2}$ of  space available to the logistics operation approximately two years before installation of the first \dword{detmodule} begins. By the time detector components start arriving, most of the cryostat boxes will have been removed from the \dword{sdwf}, leaving ample space for the detector and cryogenics components. 
%\fixme{above it says cryogen probably goes straight to headframe???}
Additional space may be required if the boxes for the second cryostat arrive before  \dword{detmodule} \#1 installation is complete; several buildings of the required size are available in the area around \dword{surf}. % if expansion is required.

\begin{dunefigure}
[Simplified model of the Ross Cage]
{fig:fdsp-tc-Cage}
{Simplified Ross Cage model and Specifications.}
\parbox{1.5in}{\includegraphics[width=0.32\textwidth]{graphics/Cage-view.pdf}}
\qquad\hspace{60pt}
\begin{minipage}{0.5\textwidth}%
\begin{tabular}{p{3.4cm}p{3.4cm}}        
\multicolumn{2}{c}{Ross Cage Specifications}\\ \toprowrule
Inside height & 3.6 m\\ \colhline
Inside depth  & 3.7 m \\ \colhline
Inside width  & 1.38 m \\ \colhline
Weight limit  &  5,897 kg \\ \colhline
Round trip \newline time & 17 min \newline (incl. unloading) \\ \colhline
\end{tabular}
\end{minipage}
\end{dunefigure}

%All material brought underground must conform to the \dword{surf} Facility Access Specification \cite{bib:docdb328}. This document defines the limitations on dimensions and weights for all materials to be transported underground.  The most important limitations, which are described in detail in the specification document, relate to the Ross shaft and Ross cage. It is possible to bring material down the shaft underneath the cage as a slung load, but this is a much slower process and requires careful planning, detailed procedures, and review. The \dword{dune} \dword{apa}, for example, requires this special handling because they are too tall to fit in the cage. Most material should be brought underground inside the cage. Figure \ref{fig:fdsp-tc-Cage} shows an image of the new Ross cage and Table \ref{tab:table-Ross-Cage} summarizes its parameters. The round trip travel time for the Ross cage is 17 minutes and this is dominated by loading and unloading time. The time needed to load, lower, and unload any slung load is more than an hour round trip as each step is much longer. 
The \dword{surf} Facility Access Specification~\cite{bib:docdb328} defines the limitations on dimensions and weights for all materials to be transported underground, the most stringent of which are set by the Ross shaft and cage. It is possible to bring material down the shaft underneath the cage as a slung load, but this is a much slower process and requires careful planning, detailed procedures, and review. The \dword{dune} \dword{apa}s, for example, require this special handling because they are too tall to fit in the cage. Most material will be brought underground inside the cage. Figure \ref{fig:fdsp-tc-Cage} illustrates the new Ross cage and summarizes its parameters.  The round trip travel time for the Ross cage is 17 minutes (actual travel time is \num{3.6} minutes each way), dominated by loading and unloading time.  Slung loads will require more than an hour round trip.

%\fixme{sorry, I did not keep the orig pgraph here}

The Ross headframe has no loading dock, so careful coordination is required. All materials transported to it must arrive on a flatbed or curtain-sided chassis, where a forklift can unload  the items. The logistics team coordinates all deliveries to the headframe, and the \dword{cf}-\dword{cmgc} coordinates all transport from there down the shaft.  Most material will be delivered first to the \dword{sdwf}, where a central inventory system will capture data about the shipments.  All deliveries, either from this warehouse or direct to the Ross headframe, require (1) coordination with the logistics team, and (2) minimum two weeks prior notice, per an advance delivery plan.  The logistics team will provide a shipping manual \fixme{ref -reserve a docdb?} to \dword{dune} institutions. The institutions must provide shipping data and consign cargo according to the guidelines so that the logistics team can monitor progress. 



In \dword{pdsp}, delays in shipping and customs resulted in up to three weeks delay in the arrival of some parts, which necessitated significant re-planning of the installation work. To prevent this from becoming a much larger problem in \dword{dune}, we plan a minimum one month buffer of materials. This buffer will allow advance planning for the underground work, %can be planned well in advance, knowing 
with confidence that all materials will be available as needed. %This will require that sufficient space be available in the warehouse and underground at \dword{surf} to house the material buffer. Many small parcels will arrive at the warehouse from different sources. 
Sufficient space must be made available in the warehouse and in the underground area  to house this material. % buffer. Many small parcels will arrive at the warehouse from different sources.
The \dword{sdwf} staff will de-consolidate or consolidate arriving cargo into larger boxes and crates, as needed, for  delivery to %the \dword{itf} or 
\dword{surf}, following established % \dword{itf} or \dword{surf} 
delivery plans, to make the most efficient use of available trucks and the Ross Shaft. %hoist. 

\begin{comment}
\end{comment}
\begin{dunefigure}[Underground space needs during installation setup]{fig:fdsp-tc-setup}
  {CAD image showing the empty half of the north cavern as used during the installation setup phase of the first \dword{detmodule}.  Half of this empty space will be used for the cryostat work and half for storage of the detector infrastructure. The material shown outside the cavern must be stored in the \dword{sdwf}.}
\includegraphics[width=.9\textwidth]{Material-Setup}
\end{dunefigure}


To discover how much space is needed for storage and how much hoist time must be dedicated to \dword{dune}, a detailed inventory of all detector equipment and \dword{dune} infrastructure is needed. The list of %all the 
materials has been solicited from all consortia and technical coordination. The entries in the inventory spreadsheet are organized as ``loads'' for the Ross shaft where a load is a crate or set of boxes that will be transported underground in one trip, either in the %hoist
cage or as a slung load~\cite{bib:docdb8426}. 
Information captured in the load spreadsheet includes the number of %hoist 
trips, type of trip (slung load or cage), package dimensions, weight and type of package (crate, pallet, box, or carton). 

The load list at present predicts 1,600 hoist trips and approximately two  months of cage time, most of which is spread over one year. Installation %operation 
(see Figure \ref{fig:high-level-schedule}) for the \dword{spmod} %detector 
will span two years, so we divide the logistics planning into three  %several 
phases: % is prudent.  The load information is divided into 
(1) the \dword{cuc} setup phase, (2) the installation setup phase, and (3) the detector installation phase. For each phase, a model was generated to show how much material can be stored underground outside the work area and how much material must be stored on the surface. These models set the space requirements for the logistics on the surface. The phase with the largest amount of material to transport is the installation setup phase.  Figure \ref{fig:fdsp-tc-setup} shows the model of the underground area and the required boxes for surface storage for the first third of the setup. This represents the first month of installation setup and shows that roughly 1,000 m$^2$ of warehouse space will be needed for \dword{dune} at this time.  The %warehouse 
\dword{sdwf} will also need space to store up to 150 \dwords{apa}, %in addition to the space needed to receive and ship equipment underground. This 
adding %an additional 
700 m$^2$ to the 1,000 m$^2$. %\ to needed warehouse area. 


%%%%%%%%%%%%%%%%%%%%%%%%%%%%
%\subsection{Logistics Quality Assurance and Quality Control}
\subsection{Logistics Quality Control}
\label{sec:fdsp-tc-log-qaqc}


%\dword{protodune} was an extremely useful exercise in general, but we can draw only a few conclusions about \dword{dune} logistics %because 
%since shipping to Europe differs from shipping to South Dakota and different staff will be responsible for receiving and transport. %the CERN receiving and transport divisions will not be used for \dword{dune}. 
The \dword{protodune} experience offers a couple of significant lessons regarding logistics.

%The full list of %lessons learned is found in the 
%ProtoDUNE-SP lessons learned is in~\cite{bib:docdb8255}. %spreadsheets. 
%The most important lessons learned from \dword{protodune} logistics are: % listed below.
\begin{enumerate}
%\item Lack of a central inventory system made it impossible to track shipments.
%\item Delays in shipping meant that the installation work could not be planned and parts were installed as they arrived. 
\item A central inventory system is essential for tracking  shipments.
\item It is important to avoid delays in shipping because they prevent installation work from  proceeding as planned. 
\end{enumerate}

%To address these issues, a central inventory system will be implemented at the logistics warehouse facility, and a minimum one month material buffer will be required from the consortia in South Dakota.
The central inventory system  implemented at the \dword{sdwf}  and the minimum one-month material buffer are the plans we have in place to prevent repetition of the schedule problems we experienced with \dword{pdsp}.   The full list of lessons learned from \dword{pdsp} is in~\cite{bib:docdb8255}. 

We do not foresee any component testing at the \dword{sdwf}, %in the warehouse, 
so the scope of the \dword{qc} work there is limited to two functions: %A \dword{qa}/\dword{qc} component, however, will be required at receiving. 
%However, a critical \dword{qc} check there %at the logistics facility 
%will 
The facility will ensure that all materials %to be shipped to the Ross headframe 
will fit in the Ross cage, or %. Moreover, 
if a slung load is needed, %the facility will confirm 
that the necessary procedures are in place and approved before any material is transported to the Ross headframe. % to \dword{surf}. 
It will also %The other primary \dword{qc} function performed at the logistics facility is to 
inventory all received shipments. % described below.

%The contribution in kind model of this project makes logistics and inventory control as well as gathering of the relevant construction data extremely complex. Therefore, the logistics (inventory) control and collection of scientific data  must be controlled by independent systems. 
%The logistics supply chain will be controlled by the contributors' freight forwarding system until supplies arrive at the \dword{sdwf}, which has yet to be defined/established.  \dword{sdwf} will be the ultimate point of capture for all \dword{lbnf}/\dword{dune} parts/equipment except possibly for the cryogenic system, given the contractual requirements. The inventory process at the \dword{sdwf}, the \dword{itf} and the \dword{surf} receiving at Ross Shaft will be controlled by one \dword{wms}. 

The contribution-in-kind model of this project complicates logistics, inventory control and the gathering of the relevant construction data. Therefore, the inventory control and the collection of component testing data must be controlled by independent systems. 
Until materials arrive at the \dword{sdwf} or \dword{surf} (if directly shipped), the contributors' freight forwarding system will control the logistics supply chain, which has yet to be defined.  \dword{sdwf} will be the ultimate point of capture for all the materials, except possibly for the cryogenics system, given its special contractual requirements. A central \dword{wms} will control the inventory process at both the \dword{sdwf} and the \dword{surf} receiving at the Ross headframe. 

%The \dword{wms} will provide basic receiving, inventory control, and shipping status for all parts and equipment delivered to \dword{sdwf}. That will include pre-assembled equipment, which will enter as new parts from \dword{itf} as created by the integration work. The \dword{qc}/\dword{qa}, manufacturing, and other relevant data required by the \dword{dune} collaborators will be stored in a separate, as yet undefined and undeveloped, \dword{dcdb}.  The \dword{dcdb} is independent of the \dword{wms} system, and the relevant contributing consortia are responsible for transferring the required data before shipment from supplier to the \dword{dcdb}.  The \dword{wms} database will provide the relevant logistics data to the \dword{dcdb}. The form of data transfer is not yet determined. All \dword{qc}/\dword{qa}, test, and other relevant manufacturing data will be directly input into \dword{dcdb} and will be the responsibility of the different contributing consortia. \dword{dune} must provide a \dword{qc}/\dword{qa} process for all parts/equipment received at the warehouse after being inventoried. That \dword{qc}/\dword{qa} data must be transferred directly to \dword{dcdb} by \dword{dune}.  The \dword{dcdb} will be an integral part of the logistics, assembly, and \dword{qc}/\dword{qa} system. It must provide the \dword{itf} shipping (supply) and assembly reports and create the new equipment denomination for the \dword{wms} to register.  The \dword{dcdb} will document the \dword{itf} sub-assembly process in its entirety.
The \dword{wms} will control basic receiving, inventory control, and shipping status for all components, parts, and equipment delivered to the \dword{sdwf}.  %including equipment  pre-assembled at the \dword{itf}, 
\fixme{I think we don't need anymore: ``which will be entered  as new parts created by the integration work.''} The \dword{qc},  manufacturing, and other %relevant 
data required by the \dword{dune} collaborators will be stored in a separate  %and undeveloped, 
\dword{dcdb} (the system is not yet determined).  


The \dword{dcdb} will be an integral part of the logistics, assembly, and \dword{qc} system.% It must provide the shipping (supply) %and assembly reports and create the new equipment denomination for the \dword{wms} to register.  The \dword{dcdb} will document the \dword{itf} sub-assembly process in its entirety. 
This database will contain all information that needs to be archived, including the shipping reports. 

The consortia are responsible for entering all \dword{qc},  %dword{qa}, 
test, and %other 
relevant manufacturing data directly into \dword{dcdb}.  The \dword{wms} database (available only during the period that we lease the \dword{sdwf}) will provide the associated logistics data to the \dword{dcdb} (the form of data transfer is not yet determined).  The \dword{qc} and shipping data flow is shown in Figure~\ref{fig:logistics-data-and-mat-flow}.

\dword{dune} must provide a post-inventory \dword{qc} %/\dword{qa} 
process to follow for all %parts and equipment 
items at the \dword{sdwf}. 
\fixme{Who at the warehouse executes the QC and adds the data to the DCDB?}

\begin{dunefigure}[QC and shipping data flow diagram for logistics ]{fig:logistics-data-and-mat-flow}
  {QC and shipping data flow diagram for the \dword{lbnf} and \dword{dune} logistics.}
 \includegraphics[width=\textwidth]{logistics-data-and-mat-flow}
\end{dunefigure}

%The new sub-assembled items will be inventoried in \dword{wms} as new items during the warehouse receiving process. 
The \dword{jpo} installation management team will provide %be responsible for providing 
a shipping (supply) report to the \dword{sdwf} %for scheduling of 
to schedule each shipment of parts and equipment to \dword{surf}. 
These shipments %from \dword{sdwf} to \dword{surf} 
will be inventoried upon receipt %as received 
at the Ross headframe %\dword{surf} 
in the \dword{wms}. 
Once the items are underground, the \dword{dune} installation team has the responsibility to  %must transfer 
input their % relevant 
\dword{qc}, %/\dword{qa}, 
test, and installation status data to the \dword{dcdb}. % directly.

\fixme{This looks like a summary of the use of the WMS, and it doesn't seem complete. Can we just end this section here, and drop the rest? Anne} To capture all relevant construction and logistics data on parts and equipment, logistics information will %should 
follow this process:

\begin{itemize}
\item The consortia enter data related to %any 
shipments to the \dword{sdwf} into the \dword{wms}. 
\item \fixme{missing the sdwf step}
\item The receiving team at \dword{surf} inventories the shipments from \dword{sdwf} % to  \dword{surf} will be inventoried as received at \dword{surf} 
in the \dword{wms}. The \dword{cmgc}, \dword{lbnf}, \dword{sdsd}, and \dword{dune} will hand off this responsibility among one another at different stages of construction. 
\item Two weeks in advance of any shipment to the Ross headframe, the \dword{jpo} installation management team %will be responsible for 
provides a %shipping (supply) 
report to the \dword{sdwf} detailing the items to ship. %   for scheduling shipments %of parts/equipment to \dword{surf} .
\end{itemize}
\fixme{anne 3/14 - check item 3 above -correct?}

%%%%%%%%%%%%%%%%%%%%%%%%%%%%
\subsection{Logistics Safety}
\label{sec:fdsp-tc-log-safety}

\fixme{talk with Mike A, Niehoff, Bill Miller}

%The \dword{lbnf}/\dword{dune} logistics facility is operated by \dword{sdsd}  as a Fermilab facility, but because of the international connections, we also follow CERN HSE, Fermilab ES\&H, and \dword{surf} ES\&H regulations.  Work is in progress to combine the three into a coherent list of codes and requirements. The \dword{dune} Project ES\&H Coordinator has overall ES\&H oversight responsibility for the \dword{dune} Project.  This person coordinates any activities and facilitates the resolution of any issues that cut across various divisions and institutions and subject to the requirements of the \dword{doe} Workers Safety and Health Program, Title 10, Code Federal Regulations (CRF) Part 851 (10 CFR 851). These requirements are promulgated through the Fermilab Directors Policy Manual and Fermilab ES\&H manual (FESHM), which align with the \dword{surf} ES\&H Manual.  Using the NOvA Far Detector Laboratory as a guideline for remote facilities, several other key documents guide the Logistics Center Safety Program.  The Building Safety Plan combines all building specific documents in a single folder:
The \dword{sdwf}  is operated by \dword{sdsd}  as a Fermilab facility, but because of the international collaboration, we  follow \dword{esh} regulations from CERN and \dword{surf} in addition to Fermilab's.  Work is in progress to combine the three into a coherent list of codes and requirements. The \dword{dune} Project \dword{esh} Coordinator has overall \dword{esh} oversight responsibility for the \dword{dune} Project.  This person coordinates any \dword{esh} activities and facilitates the resolution of any issues that are subject to the requirements of the \dword{doe} Workers Safety and Health Program, Title 10, Code Federal Regulations (CRF) Part 851 (10 CFR 851), and that cut across various divisions \fixme{divisions of what?} and institutions. These requirements are promulgated through the Fermilab Director's Policy Manual \fixme{ref} and Fermilab \dword{esh} manual (FESHM\cite{feshm}), which aligns with the \dword{surf} \dword{esh} manual.  Using the \dword{nova} Far Detector Laboratory as a guideline for remote facilities, several other key documents guide the Logistics Center Safety Program.  The Building Safety Plan \fixme{ref} combines all building specific documents in a single folder:

\begin{enumerate}
\item	Fire Safety and Building Emergency Evacuation Plan, which includes the fire evacuation plan, fire safety plan,  lockdown plans, and the site plan;
\item	Hazard Analysis document, which describes all typical hazards and their mediation %including 
procedures; 
\item	%SDS: 
Safety Data Sheets (SDS), 
\item	Respiratory Plan, as required for chemical or ODH hazards, and 
\item	Training Program, which covers required certifications and  training records.
\end{enumerate}

The current Technical Coordination Facilities Management Plan \fixme{ref} specifies a joint safety officer for %both 
the \dword{itf} and logistics facilities. \fixme{ no ITF, how to rewrite?}This safety officer facilitates training, writes hazard analysis documents, runs weekly safety meetings, and keeps documentation records on materials-handling equipment and personnel. \fixme{what aspect of personnel? sounds a bit 1984ish. }


%%%%%%%%%%%%%%%%%%%%%%%%%%%%
%\subsection{Cost and Schedule} %, and Risk Analysis}  MOVED TO NEW FILE






\section{Infrastructure}
\label{sec:fdsp-tc-infr}


%%%%%%%%%%%%%%%%%%%%%%%%%%%%
%\subsection{Introduction}
%\label{sec:fdsp-tc-infr-intro}

%The infrastructure needed to install the \dword{fd} includes the \dword{dss}, the electronics mezzanine on the cryostat roof (including racks), cable trays, underground cleanroom with related installation equipment, piping inside the cryostat, and \coldbox{}es. The major infrastructure provided by the installation group is described below.  Separate sub-sections are included for the \dword{dss}, the cryostat roof infrastructure, cryostat internal infrastructure, cleanroom, and cryogenics and \coldbox{}es.

%In addition to the equipment described below, many other items will be needed: a small machine shop, scissor lifts, rigging equipment, hand tools, diagnostic equipment (including oscilloscopes, network analyzers, and leak detectors), local storage with some critical supplies and \dword{ppe}.  

The infrastructure needed to install the \dword{spmod} includes the \dword{dss}, the electronics mezzanine on the cryostat roof (including racks), cable trays, an underground cleanroom with appropriate installation equipment, piping inside the cryostat, and \coldbox{}es. The major infrastructure provided by the installation group is described below.  

Many other items will be needed: a small machine shop, scissor lifts, rigging equipment, hand tools, diagnostic equipment (including oscilloscopes, network analyzers, and leak detectors), local storage with some critical supplies, and \dword{ppe}.  
\fixme{items provided by whom?}


%%%%%%%%%%%%%%%%%%%%%%%%%%%%
%\subsection{Detector Support System}
\subsection{Detector Support System}
\label{sec:fdsp-tc-infr-dss}


The \dword{dss} provides the structural support for the detector inside the cryostat.  
It also provides the necessary infrastructure inside the cryostat to move the detector elements into place during assembly. 
The \dword{dss} is a new design, quite different from the \dword{pdsp} \dword{dss}. The detector elements supported by the \dword{dss} include the \dwords{ewfc}, the \dword{apa}s, and the \dword{cpa}s with top and bottom \dword{fc} panels. 
The nominal load of the detector elements both dry (in air) and wet (in \dword{lar}) are shown in Table \ref{tab:installation-DSS-load}. 
The weights listed are the current design weights.  
The \dword{dss}, however, is designed to accommodate significant design changes, even if the detector weight  doubles the \dword{dss} would still meet the design code requirements.  
Deformations would increase due to any increase in loads, and this effect would be evaluated if needed.

\begin{dunetable}
[DSS Loads]
{l|c|cc|cc}
{tab:installation-DSS-load}
{The expected dry and wet stratic loads for the DSS.}
%\multicolumn{2}{c}{} &  \multicolumn{4}{|c}{Dry Weight}\\ \toprowrule
& &  \multicolumn{4}{|c}%{Dry Weight}
{Weight before fill (Dry)}\\ \toprowrule
& & \multicolumn{2}{c|}{Unit Weight} & \multicolumn{2}{c|}{Total Weight}  \\ \colhline

Detector Component &\# Units& (kg)&(lbs) & (kg) &(lbs)\\ \colhline
\dword{dss} & 1 &NA&NA& 12318  & 27100 \\ 
\colhline
\dword{apa} (Installed \dword{apa} pair, no cables)& 75&1184 &2604 &88768  &195290\\ 
\colhline
\dword{cpa} & 100& 233 & 513 & 23331 & 51327 \\ 
\colhline
Top or Bottom \dword{fc} module (FC TB)& 400&149 & 328	 & 59679 & 131294\\ 
\colhline
\dword{ce} Cables &750& 182 & 400 & 13636 & 30000\\
\colhline
\dword{ewfc}  & 8	&904 &	1989  & 7234 & 15914\\ 
\colhline
{\bf Total} &  & & & 204966 &	450925\\ 
\colhline
\toprowrule

\rowtitlestyle & &  \multicolumn{4}{c}{Weight after fill (Wet)}\\
\toprowrule
\dword{dss} (not in liquid) & 1 & NA & NA & 12318 & 27100 \\ 
\colhline
\dword{apa} (Installed \dword{apa} pair/No cables)&75& &0 & 0 &0\\ 
\colhline
\dword{cpa} & 100& 45 & 99 & 4520 & 9943 \\ 
\colhline
Top or Bottom \dword{fc} module (FC TB)& 400 & 68 & 150	& 27359 & 60191 \\ 
\colhline
\dword{ce} Cables & 75 & & & 13636& 30000 \\
\colhline
\dword{ewfc}  & 8 & 283& 	622& 2263 & 4978\\  
\colhline
{\bf Total} &  & & &60096	 &132211 \\ 
\colhline
\end{dunetable}


The \dword{dss} shown in Figure~\ref{fig:DSS} consists of 5 rows of I-beams inside the detector which support the 5 rows of \dword{apa} and \dword{cpa}. 
The I-beams themselves are supported from the cryostat outer steel structure through a series of vertical supports or mechanical \fdth{}s also shown in Figure~\ref{fig:DSS}. 
The \dword{dss} constrains the location of the detector inside the cryostat and also accommodates the detector elements' movement and contraction during cooling. The design of the \dword{dss} sets the overall structure of the \dword{detmodule} since only after the elements are mounted to the \dword{dss} and connected do they make a unified mechanical structure. 
During installation the detector components are moved along the I-beams using both simple and motorized trolleys. 
The end of the \dword{dss} nearest the TCO is also designed as a switchyard. An additional set of north south beams allow a short section of the I-beam rail to be shifted between the 5 rows of \dword{dss} beams which correspond to the five alternating rows of detector elements  (\dword{apa}-\dword{cpa}-\dword{apa}-\dword{cpa}-\dword{apa}).  
With this the \SI{12}{m} tall detector elements can enter the cryostat on an I-beam through the TCO, be loaded on the short switch yard beam, moved to the required row of \dword{dss} and then be pushed into position. 
\fixme{add reference to TC vol Ch 7 fig 7.6}

\begin{dunefigure}[\threed model of the \dword{dss} ]{fig:DSS}
  {\threed model of the \dword{dss} showing the entire
  structure on the left along with one \dword{apa} row and one
  \dword{cpa}-\dword{fc} row at each end. The right panel is a zoomed image
  showing the connections between the vertical supports and the
  horizontal I-beams.}
\includegraphics[width=.49\textwidth]{DSS-1.pdf}
 \includegraphics[width=.49\textwidth]{DSS-2.pdf}
\end{dunefigure}


% \textquotedblleft cryostat crossing tubes\textquotedblright\   that penetrate the cryostat insulation.  \ref{fig:crossingtube}


The \dword{dss} is designed to meet the following  requirements:
\begin{itemize}
 \setlength\itemsep{1mm}
\setlength{\parsep}{1mm}
\setlength{\itemsep}{-5mm}
% \small
\item Support the weight of the detector;
\item Accommodate cryostat roof movement during filling, testing, and operation;
\item Accommodate variation in \fdth locations and
  variation in the flange angles due to installation tolerances and
  loading on the warm structure;
\item Accommodate shrinkage of the detector and \dword{dss} from ambient
  temperature to \dword{lar} temperature;
\item Define the positions of the detector components relative to each other; 
\item Provide electrical connection to the cryostat ground and remain electrically isolated from the detector;
\item Allow support penetrations to be purged with gaseous argon to prevent contaminants from diffusing back into the liquid; 
\item Ensure that the instrumentation cabling does not interfere with the \dword{dss};
\item Consist entirely of components that can  
be installed through the \dword{tco};
\item Meet AISC-360 codes. % or appropriate codes required at \dword{surf}; \fixme{and/or?}
\item Meet seismic requirements one mile underground at \dword{surf};
\item Consist entirely of materials compatible %for 
with operation in ultrapure \dword{lar};
\item Ensure that beams are either completely submerged in \dword{lar} or completely in the ullage;  
\item Ensure that detector components are not less than \SI{400}{mm} from the membrane flat surface;
\item Ensure that the supports do not interfere with the cryostat I-beam structures;
\item Ensure that the detector's lower \dword{gp} is lies over the cryogenic piping and that the tops of the \dword{dss} beams are submerged in \dword{lar} while leaving a \SI{4}-\SI{5}{\%} ullage at the top of the cryostat; 
\item Include the infrastructure necessary to move the \dword{apa} and
  \dword{cpa}-\dword{fc} assemblies from outside the cryostat through the
  \dword{tco} and to the correct position.
\end{itemize}

Each row of the \dword{dss} consist of a series of \num{10} \SI{6.4}{m}-long
W10$\times$26 stainless steel I-beam sections for a total of \num{50} I-beam segments for the 5 rows. The length of the beam segments was chosen to be a multiple of the \SI{1.6}{m} pitch of the major cryostat beams which allows the regular placement of the support \fdth across the cryostat roof. With a W10$\times$26 I-beam and \SI{6.4}{m} between the supports the beam deflections due to the loads was kept below 5mm. 
Each I-beam is suspended on both ends by the mechanical \fdth{}s that penetrate the cryostat roof. 
During cool down  each I-beam shrinks while the mechanical supports outside the cryostat remain fixed  causing gaps to form between \dword{apa}s that are adjacent but supported on separate beams.
\dword{apa}s that are supported on the same beam will not have gaps develop because both the beam and \dword{apa} frames are stainless steel so they shrink together.
The gap between two adjacent \dword{dss} beams after cooldown will be \SI{17}{mm} which is considered acceptable. 
Increasing the the beam length beyond \SI{6.4}{m} was not considered as the deformation of the I-beam under load would increase, the gap between \dword{apa} on adjacent beams would increase, and the difficulty of installing the beams would increase.
\fixme{need max deflection number from Vic}


\begin{dunefigure}[DSS vertical support \fdth]{fig:DSS-Support}
  { Drawing of the \dword{dss} vertical support \fdth. The detector load is carried by the \SI{25}{mm} inner support rod. The outer lateral support tube prevents swinging during installation.  The \fdth mounts to the cryostat crossing tube which is an integral part of the cryostat. Not shown here are the short vacuum chambers for the CISC instrumentation which can be found on Figure \ref{fig:CISC-feedthru}.}
\includegraphics[width=.85\textwidth]{graphics/DSS-Support.pdf}
\end{dunefigure}


The \dword{dss} I-beams are supported on both ends from a vertical support \fdth shown in Figure \ref{fig:DSS-Support}. A \SI{25}{mm} solid stainless steel rod, which is threaded at both ends, runs down the center of the \fdth and carries the detector load. The support  rod connects on the bottom end to a clevis which is then pinned to the \dword{dss} beams shown in Figure \ref{fig:DSS-lateral-support}. At the top the rod bolts to an X-Y table sitting on the top Conflat Flange that allows a lateral adjustment of $\pm$\SI{2.5}{cm} (\SI{1}{in}). A swivel washer is used in the bolted connection to the X-Y table to allow the support rod to swing freely. The bolted connection also allows the \dword{dss} I-beams to be adjusted vertically. The vacuum seal is established at the top with a bellows between the rod and the top flange. The top flange of the \dword{dss} support \fdth is a Conflat flange that connects to the cryostat crossing tube's mating flange. The crossing  tube is welded to the cryostat roof and the top flange is mechanically supported from the cryostat's  \SI{1.1}{m} tall support I-beams. The cryostat crossing tubes are shown in Figure \ref{fig:crossingtube}.

During installation the detector components will be pushed along the \dword{dss} I-beams which will place a lateral load on the \dword{dss} support structure. 
A \SI{15.2}{cm} (\SI{6}{in}) \dword{od}  tube is welded to the top flange of the \dword{dss} \fdth . 
This lateral support tube  extends through the cryostat insulation and has a clamping collar at the bottom which is used to fix the I-beam support clevises in position during installation. 
The bottom of the lateral support tube is seen in Figure \ref{fig:DSS-lateral-support}. 
The long bolts press on the flat sides of the clevis to fix the support rod's location. 
There is a nominal \SI{10}{mm} gap between the OD of the support tube and the \dword{id} of the clearance tube in the cryostat. 
The clevis can be positioned anywhere inside the \SI{15.2}{cm} tube.




\begin{dunefigure}[DSS support for lateral loads ]{fig:DSS-lateral-support}
  {Left panel shows how the central support rod is locked in postion during detector installation. The outer  \SI{15.2}{cm} (\SI{6}{in}) tube is used to fix the support clevis in position. The right panel shows the system as it is connected to the I-Beam.}
\includegraphics[width=.75\textwidth]{graphics/dss-lateral-support.pdf}
\end{dunefigure}

After the detector has been installed all restraints on the clevis are released to allow motion as the detector contracts during cool down.  The two support rods that support each \dword{dss} beam will contract and move toward each other by 13.1 mm along the axis of the detector.  
The drift distance will shrink by 7.4 mm caused by the contraction of the field cages.  The detector is symmetric in the drift direction around the center \dword{apa}.  The drifts on either side of the center \dword{apa} will  shrink toward the center while the center \dword{apa} remains unmoved.  This results in the \dword{cpa}s moving 7.4 mm toward the center and the outer \dword{apa}s moving 14.8 mm (2$\times$7.4 mm) toward the center.  The hanging rod is designed to have a range of motion of 15 mm in the drift direction to accommodate this shrinkage.




Detector components are installed using a shuttle beam system as
illustrated in Figure~\ref{fig:shuttle}.  
The last two columns of \fdth{}s (western-most) support temporary beams that run
north-south, perpendicular to the main \dword{dss} beams.  
A shuttle beam has trolleys mounted to it and transverses 
north-south until it aligns with the required row of \dword{dss} beams.  
The last \dword{apa} or \dword{cpa} in a row is supported by the shuttle beam, which is bolted directly to the \fdth{}s once it is in place.  
As the last \dword{cpa} or \dword{apa} in each row is installed, the north-south beams are removed.

\begin{dunefigure}[\threed models of the shuttle beam end of the \dword{dss}]{fig:shuttle}
  {\threed models of the shuttle beam end of the \dword{dss}. The figures show how an \dword{apa}
is translated into position using the north-south beams until it lines up with the correct
row of I-beams.}
\includegraphics[width=.49\textwidth]{/Shuttle-1.pdf}
 \includegraphics[width=.42\textwidth]{shuttle-2.pdf}
\end{dunefigure}

A mechanical stops  prevents trolleys
from passing the end of the shuttle beam unless it is aligned with a
corresponding \dword{dss} beam.  The shuttle beam and each detector component are
moved using a motorized trolley.  A commercially available motorized
trolley will be modified as needed for the
installation. 




\begin{dunefigure}[Prototype of the motorized DSS trolley ]{fig:DSS-trolley}
  {Prototype of the motorized DSS trolley that will push the APA and CPA along the I-beams and through the switchyard.}
\includegraphics[width=.49\textwidth]{graphics/DSS-trolley.pdf}
\end{dunefigure}



A mock-up of the shuttle system will be constructed to test the
mechanical interlock and drive systems for the shuttle beam
for each \dword{detmodule}.  Tests will be conducted to evaluate the level of
misalignment between beams that can be tolerated and the amount of
positional control that can be achieved with the motorized trolley. We plan to construct a full scale prototype of a section of the  switchyard and perform tests at floor level. Later, the test program will be expanded at Ash River, where a full scale installation test will be performed. This is described in 
Section~\ref{sec:fdsp-tc-infr-qaqc}.


%%%%%%%%%%%%%%%%%%%%%%%%%%%%
\subsection{Cryostat Roof Infrastructure}
\label{sec:fdsp-tc-infr-cryo-roof}



\begin{dunefigure}[Mezzanine and electronics racks]{fig:mezzanine}
  {The electronics racks sit on the \dword{dune} electronics mezzanine as shown. The top image is a view from above the detector looking at the racks from the side. In this view the cavern and cryogenics mezzanine are hidden. The bottom view is from the end of the cryostat looking over the roof. Here, the access stairs to the mezzanine are shown.}
 \includegraphics[width=\textwidth]{mezzanine.pdf}
\end{dunefigure}

\begin{dunefigure}[Electronics rack contents]{fig:rack-build1}
  {The nominal contents of the electronics racks on the mezzanine is shown. Each rack is configured to consume less than 3.5 \si{kW}. }
 \includegraphics[width=.8\textwidth]{graphics/rack-build1.png} %pdf} It had too much room around it (was at .98)
\end{dunefigure}

The top image in Figure \ref{fig:mezzanine} shows the \dword{dune} electronics mezzanine with the 42U racks placed on top. 
\fixme{42U defined somewhere?} 
During the initial design steps, it became clear that the constraints placed on the rack location by the many \dword{dss} support feedthroughs, the electronics feedthrough, and the I-beams themselves make distributing the racks on the roof very challenging. 
By constructing a fixed mezzanine for the electronics 
above the cryostat at the same height as the cryogenic mezzanine, the electronics feedthroughs are kept clear. 
This configuration also makes working on the electronics much easier because there are no local obstacles and all the racks are in one place.

The racks are %on 
at detector ground, so the mezzanine, % is 
also at detector 
ground, % where it 
can simply be bolted to the cryostat I-beams. 
%As 
Figure \ref{fig:mezzanine} (top) shows 16 groups of five racks each %are 
on the mezzanine for a total of 80 racks. 
The heat load inside the detector racks will %should 
dissipate through air flow generated by cooling fans.  The major heat load resides with the cold electronics and photon electronics located near the cryostat feedthroughs.  If rack-mounted fans do not provide sufficient cooling,  \dword{cf} will provide sufficient water cooling capacity at the entrance to the North Cavern to accommodate the maximum heat load. 

%Twenty-five racks will be needed for cold electronics low voltage power and twenty-five racks are available for \dword{apa} wire bias voltage, \dword{pd} power and miscellaneous additional cold electronics, \dword{pd}, and  \dword{apa} electroncis modules. The remaining 30 racks will be available for slow control, calibration, and other miscellaneous electrical equipment. Small 12U high mini-racks will also be placed near the electronics feedthroughs for the \dword{pd} readout electronics and optical patch panels. If this is not enough, then additional racks can be placed on the cryostat roof. The present rack build for this layout is shown in Figure \ref{fig:rack-build1}. The modules inside the racks are distributed to keep power consumption for each rack below 3.5 \si{kW}.The racks to be installed will be 42U high, so we will have significant extra rack space.  If added electronics are needed for the \dword{ce}, however, we would need to install the modules in a \dword{pd} rack to stay under the power limit per rack.

Of the 80 racks, \dword{ce} \dword{lv} power require \num{25}, and another 25 will be made available collectively for  \dword{apa} wire bias voltage, \dword{pd} power and miscellaneous additional \dword{ce}, \dword{pds}, and  \dword{apa} electronics modules. The remaining 30 will be available for slow control, calibration, and other electrical equipment. Small 12U high mini-racks will  be placed near the electronics feedthroughs for the \dword{pd} readout electronics and optical patch panels. If this is not enough, additional racks can be placed on the cryostat roof. The present rack build for this layout is shown in Figure~\ref{fig:rack-build1}. 
\fixme{is `rack build' a common term? I would use `configuration'. anne} The modules \fixme{modules is used for so many things - can we use `items' here?} inside the racks are distributed to keep power consumption for each rack below \SI{3.5}{kW}.The racks are 42U high, which provides significant extra rack space.  If the \dword{ce} requires additional electronics, however, the items would need to go in a \dword{pd} rack to respect the per-rack power limit.

The 12U high mini-racks near the feedthrough flanges will be relatively empty because the \dword{pd} readout should need only approximately 2U in height while the \dword{ce} patch panel needs less than 1 U. The mini-racks are shown in the lower panel of Figure \ref{fig:mezzanine}; %fig:rack-build1}; 
they are the gray rectangles near the electronics crosses.

The north-south cable trays \fixme{where is north-south defined relative to the cryostats?  Can we add ``transverse to the beam?} running from the electronics mezzanine to the electronics feedthrough are routed under the floor of the cryostat roof next to the %cryostat 
I-beams. This keeps the roof reasonably clear, so that equipment can be transported across it. %the cryostat roof as needed. 
The gap between the web of the I-beams is \SI{1.2}{m}, such that % a 
installation of a \SIrange{200}{300}{mm} wide cable tray %can be installed while still leaving 
leaves enough space %to stand on the roof and 
for people to work on the electronics crates. 
The cable trays between the \dword{cuc} and the electronics mezzanine will run along the west end of the cryostat under the floor.  \fixme{which floor?}
We estimate only half of the \SI{1.6}{m} space is needed, so the cable tray quantity could in principal be doubled if necessary. 

The flooring material for the \dword{spmod} %should 
will be similar to the \SI{25}{mm} thick plywood used at \dword{pdsp}. %; it will be used for the final \dword{dune} detector. The flooring material 
It must be easy to cut so as to fit around many obstacles and pipes on the roof, it must be light enough to lift up to allow access under the floor, and it must support the load of a person %or 
and a small cart. 
We will investigate %what type 
fire-retardant options available %exist 
in the USA, %and will solicit 
with input the \dword{fnal} fire life-safety group. 

Air filters for the cleanroom and inside the cryostat will also be placed on the cryostat roof. The present plan is to place fan filter units near the manholes on the east end of the cryostat. Initial calculations indicate sufficient airflow is possible to support one air exchange per hour inside the cryostat. The air handling system has yet to be designed in detail.


\begin{dunefigure}[Cryostat crossing tube design]{fig:crossingtube}
  {Draft drawing of the cryostat crossing tubes. The hatched region is the cryostat insulation.}
\includegraphics[width=.85\textwidth]{crossingtube}
\end{dunefigure}
\fixme{In figure \ref{fig:crossingtube} say what units are. mm? What are u and v? Clarify!}
 
The cryostat crossing tubes are among the most critical \fixme{challenging? Everything is critical, no?} components of the roof infrastructure. %These are the vacuum components that penetrate the cryostat roof and connect to the cold cryostat membrane. 
These  vacuum components penetrate the cryostat roof and connect to the cold cryostat membrane.The top flange of the crossing tube supports \fixme{either?} the electronics feedthrough or the detector support flanges and must be directly tied to the steel I-beams for support. %Accurately placing the crossing tubes and installing them true to vertical is important, ensuring the interface to the cryostat membrane can be made. 
Accurate placement and true vertical installation of the crossing tubes is important to ensure proper interfacing to the cryostat membrane. A draft assembly drawing of the crossing tube is shown in Figure \ref{fig:crossingtube}. The tube consists of a \SI{464}{mm} long stainless steel pipe with a \SI{1}{cm} \fixme{thick? and cm or mm?} wall. One end of the thin stainless steel tube is welded to the cryostat membrane, %at one end of the tube 
and the other to custom conflat flanges. % are welded to the other end. 
The %1 \si{cm} thick 
tube is welded to the steel roof plates.  \fixme{The cylindrical surface? which part?}


%To ensure that the crossing tubes are adequately cleaned during the initial gaseous argon (\dword{gar}) purge and that air is removed from the tubes, each crossing tube has a small side port connected to a network of pipes on the roof of the cryostat. During the initial \dword{gar} purge, argon gas is withdrawn from each port and analyzed to assess progress and determine when the system is ready to be cooled down. The large number of crossing tubes, about 250 in the \dword{sp} configuration, means the content from each port cannot be analyzed independently of all others. Five streams, each one collecting gas from about 50 crossing tubes are connected independently to the gas analyzers. Gas impurities accumulate in the ullage. During steady state operations, \dword{gar} from the crossing tubes will be analyzed to ensure that impurities are adequately removed from the \dword{gar}.    If additional cleaning of the ullage is necessary, the \dword{gar} withdrawn from all ports, or a smaller set of them, can be sent to the condenser, re-condensed, and purified in the liquid phase along with the rest of the \dword{lar}. To detect any possible leaks that could develop in the room temperature feedthroughs over time, simple O$^2$\ sensors monitor the return gas for traces of oxygen, which would indicate a leak.
 Each of the 250 crossing tubes has a small side port connected to a network of pipes on the roof of the cryostat. To ensure adequate cleaning and removal of air from the  tubes during the initial \dword{gar} purge,  \dword{gar} is withdrawn from each port and analyzed to assess progress and determine when the system is ready to be cooled down.  Five streams, each collecting gas from 50 crossing tubes, are connected independently to the gas analyzers. The collection and analysis of the gas will continue during steady state operations to ensure that impurities, which accumulate in the ullage,  are adequately removed from the \dword{gar}.    If additional cleaning of the ullage is necessary, the \dword{gar} from all or a subset of ports can be sent to the condenser, re-condensed, and purified along with the rest of the \dword{lar}. Simple O$_2$\ sensors monitor the return gas for traces of oxygen, which would indicate development of  a leak in the room temperature feedthroughs.

Each detector has a power budget of \SI{500}{kVA}. The total power budget available for use by detector electronics is derated to \SI{400}{kW} at the power distribution panels.  The \dword{ce} imposes the largest power load.  

The \dword{ce} dissipates \SI{306}{W} per \dword{apa}.  %The \dword{lv} power supplies have a controller that is about \SI{35}{W} per \dword{apa} and has an efficiency of approximately 85 percent.  
The \dword{lv} power supplies' controller imposes about \SI{35}{W} per \dword{apa} and has an efficiency of approximately 85 percent. This leads to an approximately  \SI{400}{W} load per \dword{apa}, or a total load of  \SI{60}{kW} per \dword{detmodule}.  The \dword{apa} wire-bias power supplies have a maximum load of  \SI{465}{W} per set of six \dword{apa}s, for a total budget of around  \SI{12}{kW}.   Cooling fans and heaters near the feedthroughs will use a nominal amount of power, so the overall power budget for \dword{ce} %should 
is expect to be less than \SI{75}{kW}.
\fixme{check my use of `imposes' vs `requires' vs `present' vs whatever for loads}

%The \dword{pds} electronics, based on the Mu2e electronics, report a power load of approximately 6KW.  A power budget of 8KW should be used because of cable drops and  power supply inefficiencies.  Please note that these \dword{pd} electronics are a significantly lower power load than the alternate \dword{protodune} solution of using \dword{ssp} modules at a power load of approximately 72 KW per detector.
The \dword{pds} electronics is based on the Mu2e electronics, which reports a power load of approximately  \SI{6}{kW}.  \dword{dune} plans a power budget of  \SI{8}{kW} because of cable drops and  power supply inefficiencies.  The \dword{pds} electronics present a significantly lower power load than the alternate solution, used in \dword{pdsp}, of using \dword{ssp} modules at a power load of approximately  \SI{72}{kW} per \dword{detmodule}.

Each of the approximately 80 detector racks will have fan units, Ethernet switches, rack protection, and slow controls modules, adding a load of about \SI{500}{W} per rack, for a total of \SI{40}{kW}.

%A number of racks are reserved for cryogenics instrumentation.  The load of these 25 racks is conservatively estimated at 2 KW per rack for a total of 50 KW.
Twenty-five racks are reserved for cryogenics instrumentation with a per-rack load conservatively estimated at \SI{2}{kW}, for a total of \SI{50}{kW}. 

The \dword{detmodule} will thus use  an estimated \SI{173}{kW} of power. The higher-load \dword{ssp} alternative for the \dword{pds} would increase this to \SI{237}{kW}.  These numbers provide a safety factor of about two on our power estimates relative to available power.


%%%%%%%%%%%%%%%%%%%%%%%%%%%%
\subsection{Cryostat Internal Infrastructure}
\label{sec:fdsp-tc-infr-cryo-int}


%%%%Internal Cryogenics%%%%
\label{sec:fdsp-tc-internal-cryo}

The internal cryogenics comprises three sets of pipe distribution networks and two sets of sprayers. All pipes enter the cryostat from the top; some go all the way down to the floor, and others remain in the ceiling. On the floor are
\begin{itemize}
\setlength\itemsep{1mm}
\setlength{\parsep}{1mm}
\setlength{\itemsep}{-5mm}
\item \textbf{GAr distribution}: a set of pipes flowing GAr. These pipes are used only at the beginning to remove air that fills the cryostat. They have either a longitudinal slit or calibrated holes to distribute GAr uniformly along the length of the cryostat. Computational fluid dynamics simulations show that air will be removed from the system as long as GAr is flowing in at the right speed, calculated and experimentally verified as \SI{1.2}{m/hr}.


\item \textbf{\dword{lar} distribution}: two sets of pipes flowing \dword{lar}. These pipes are used to fill the cryostat and, during steady state operations, to return the \dword{lar} from the purification system. The pipes have calibrated holes to return the \dword{lar} uniformly throughout the length of the cryostat. This is very important for uniform purity. Four pumps circulate the \dword{lar}. Initially, all of pumps operate at once to achieve purity, but once the target purity is achieved, only one or two pumps remain in service. Two sets of pipes are needed to adequately distribute the \dword{lar} over this broad range of flow rates.
\end{itemize}

On the ceiling are

\begin{itemize}
\setlength\itemsep{1mm}
\setlength{\parsep}{1mm}
\setlength{\itemsep}{-5mm}
\item \textbf{Cool down sprayers}: Two sets of cool down sprayers are distributed along the long sides of each cryostat. One set distributes \dword{lar} using liquid sprayers that generate a conical profile of small droplets of liquid. The other set of sprayers distributes GAr to move the \dword{lar} droplets inside and cool down the detector and cryostat uniformly. These sprayers are being tested in \dword{pddp}. They are a variation of those implemented in \dword{pdsp}.
\end{itemize}

Figure~\ref{fig:internal-cryo-3D} shows the current layout of the internal cryogenics. 
%The current drawing of the internal cryogenics is presented in Figure~\ref{fig:internal-cryo-drawing}. 
The GAr pipes are in red, the \dword{lar} pipes in blue.

\begin{dunefigure}[Cryogenic piping inside the cryostat ]{fig:internal-cryo-3D}
  {Layout of the internal cryogenics.}
\includegraphics[width=.98\textwidth]{graphics/Internal-Piping-3D.pdf}
\end{dunefigure}

%\begin{dunefigure}[Drawing of the cryogenic %piping inside the cryostat %]{fig:internal-cryo-drawing}
%  {Drawing of the internal cryogenics.}
%\includegraphics[angle=90,width=.98\textwidth]{%graphics/Internal-pipes-HQ.pdf}
%\includegraphics[angle=90,height=.98\textheight%]{graphics/Internal-pipes-HQ.pdf}
%\end{dunefigure}
%\fixme{Please reduce size of Internal-pipes-HQ.pdf}


Other infrastructure inside the cryostat includes the cryostat false floor, the UV filtered lighting, and the battery-operated scissor lifts. 

The false floor %s not yet designed because the 
requirements are not yet fully defined. %The false floor 
It must support the load of the scissor lift used to work on the electronic cabling on the inside of the cryostat roof and allow the scissor lift to get close enough to the \dword{apa}s to work comfortably at the top. 
It  must be laid out so the panels can be removed in sections just before the equipment is installed,  %The space between the lowest point of 
except for the \dword{apa}s;  there is not enough room between the bottom of the \dword{apa}s %modules 
and the floor to allow removal. %is too small to allow the flooring to be removed once the \dword{apa} is in place. %However, t
%The floor panels in front of the \dword{apa} must % %Careful layout will be needed. 

The cryostat lighting, using UV-filtered \dword{led} lamps, is expected to be fairly simple. Options for the lighting will be developed during the Ash River testing. Floor-mounted lights with task lighting will be investigated. If needed, lighting can also be mounted to the \dword{dss} and removed as the detector is installed.

The plan is to use a commercially available battery-operated scissor lift with a 12 \si{m} reach. Tests at Ash River will verify the stability of the lift at the 12 \si{m} height. If the lift is determined to be suitable, then the remaining issue to resolve is how to install and remove it from the cryostat. The commercially available scissor lifts are too wide to fit through the \dword{tco} near the floor where the center post protrudes above floor level. Custom lifting equipment will be needed to insert the lifts into the cryostat. At the end of the installation process, dismantling the last lift may be necessary to remove it from the cryostat.

% clear the figure buffer before starting the next section
\clearpage

%%%%%%%%%%%%%%%%%%%%%%%%%%%%
\subsection{Cleanroom Infrastructure}
\label{sec:fdsp-tc-infr-comm}

\begin{dunefigure}[Installation Cleanroom layout]{fig:cleanroom-layout}
  {Two views of the installation cleanroom. The walls are shown as semi-transparent so the equipment inside can be seen. The left view is from the west looking east and shows the materials airlock (orange outline) and changing room (yellow outline). The cleanroom proper is behind the them and the cryostat is at the back. The second image on the right is a view looking north. It shows the outline of the cleanroom in green. The cryostat is on the left of the cleanroom and the airlocks on the right.} 
\includegraphics[width=1.0\textwidth]{cleanroom-layout}
\end{dunefigure}
\fixme{for fig \ref{fig:cleanroom-layout}: walls are semi-transparent IN THE FIGURE, not for real ( I added `shown as'). Directions are confusing -left, right, west, north.  Also would be nice to add a person to show scale.}

The assembly of the detector sub-components into \SI{12}{m} tall full assemblies %modules 
must be done underground directly outside the cryostat. The full %12 \si{m} long 
assemblies are too large and fragile to be brought down the shaft, and the only place with enough vertical space to assemble them is in front of the cryostat itself. A combination of contractors and the lead worker and rigger teams will do the infrastructure work; they will also assist in detector assembly. The cleanliness requirement for all work on the components inside the detector means that all work must be done in an ISO-8 cleanroom, so a very large cleanroom must be built outside the cryostat containing all the necessary infrastructure to assemble the %detector \dword{tpc} elements. 
\dword{lartpc} components. Figure \ref{fig:cleanroom-layout} illustrates %provides an image of 
the conceptual design of the installation cleanroom. The left figure is an end view of the cleanroom, showing the materials airlock 
\fixme{at the back (east end) ?} 
and the changing room. The right image is a side view showing the cleanroom on the left and the airlock on the right.

The plan is to bring all detector elements into the cleanroom through large roll up doors in the side wall of the airlock. The materials will be moved using using  a battery operated forklift and electric pallet jacks. The airlock will be  13 \si{m} wide 10 \si{m} deep and 17 \si{m} tall. This is large enough to hold the \dword{apa} assembly tower described below with enough extra space to move large objects inside. An access hatch is planned for the roof, allowing the cavern bridge crane to be used in the central area of the airlock. The crane is needed to manipulate the \dword{apa} modules in the airlock; the plan calls for assembly of the double-high \dword{apa} pairs there. Also shown on the right of the figure is the changing room for the cleanroom. The changing room is 3 \si{m} wide and 10 \si{m} deep. The dimensions were chosen to allow 20-30 people to gown up for the cleanroom within a reasonable time. The requirements for work in an ISO-8 cleanroom are a cleanroom lab coat, clean shoes, and nets for hair and beards.  This will be augmented with a clean hard hat and gloves for safety reasons. 

The outline of the cleanroom proper is shown on the right in Figure \ref{fig:cleanroom-layout}. The cleanroom dimensions are 16 \si{m} wide 10 \si{m} deep and 17 \si{m} tall. The cleanroom lies primarily under the north-south bridge and in the 3.4 m space between the bridge and the cryostat. %Here operations that must be performed are the 
Cabling of the \dword{apa}, cold testing of the \dword{apa}, and assembly of the cathode \dword{fc} modules will take place here. The tower for the \dword{apa} cabling, the \coldbox{}es for testing, and the switchyard for moving the 12m tall objects are described below. All areas of the airlock and cleanroom will be outfitted with UV filtered lights. In addition to this cleanroom, the inside of the cryostat will also be effectively a cleanroom.  By filtering the air and forcing it  into the cryostat at the east end, the clean air will flow through the cryostat and out through the cleanroom and airlock. This will keep the inside of the cryostat at least at ISO-8. The construction process for the cleanroom is still at the conceptual level so what is shown is a steel frame structure where panels can be mounted. %The size is substantial and the occupancy significant, so the cleanroom should have electrical outlets, UV filtered lighting, and fire protection. 
Given the substantial size and the significant occupancy, the cleanroom will have electrical outlets, UV filtered lighting, and fire protection. 

\fixme{need to check all dimensions and update the figure}

\begin{dunefigure}[APA assembly tower]{fig:apa-assemble-frame}
  {The \dword{apa} assembly tower with the \dword{apa} assembly frame attached. The transport rails at the top of the figure are used to move the assembled \dword{apa} into the cleanroom. The work decks allow access to the \dword{apa} from multiple levels. }
\includegraphics[width=.5\textwidth]{apa-assemble-frame}
\end{dunefigure}

\fixme{need to update the figure}

The \dword{apa} assembly tower is a six story tall stair tower used to assemble the individual \dword{apa} modules into the final 12 \si{m} tall pair. 
The tower is designed to so that the stair landings will allow access to the sides and top of the \dword{apa} as required for assembly. 
The \dword{apa} assembly tower also provides support for the the \dword{apa} assembly fixture provided by the \dword{apa} consortia. 
The \dword{apa} assembly fixture is the tooling needed to hold the upper and lower \dword{apa} during assembly and to bring the two modules together so they can be connected. 

Figure \ref{fig:apa-assemble-frame} shows an image of the \dword{apa} assembly tower. 
In this figure, the \dword{apa} assembly frame is shown mounted on the assembly tower, and an \dword{apa} transport box is on the airlock floor in front of the tower. 
At the top of the \dword{apa} assembly tower is a set of I-beam rails used to move the \dword{apa} pair into the cleanroom through a sliding door shown in brown. 
The final trolleys will be used from this step forward. 
Care will be taken in the \dword{apa} assembly steps to remove only a minimal amount of the dust protection from the \dword{apa} because manipulating the \dword{apa} must be performed using the cavern bridge crane, which will require an opening in the roof. 
Once the \dword{apa}s are assembled, the bridge crane is no longer needed, and the roof opening is closed. Air is then circulated until the required purity is met. The tower shown in Figure \ref{fig:apa-assemble-frame} has landings that will meet safety codes, but the outer steel frame to support the landings has not yet been designed.


\begin{dunefigure}[APA cabling tower]{fig:apa-cable-tower}
  {The \dword{apa} cabling tower. The two reels of cable for the electronics are shown. The work decks allow access to the \dword{apa} from multiple levels. }
\includegraphics[width=.5\textwidth]{apa-cable-tower}
\end{dunefigure}

%Inside the cleanroom is another stair tower used for cabling and the initial tests of the \dword{ce}. 
Inside the cleanroom  another stair tower is used for cabling the \dword{apa}s and the initial tests of the \dword{ce}. Due to the time required for cabling and testing of the electronics in \dword{protodune}, % show that 
two workstations are needed. The \dword{apa} cabling tower is shown in Figure \ref{fig:apa-cable-tower}. The \dword{apa} cabling tower is a stair tower similar to the \dword{apa} assembly tower, where access is provided to the sides of the \dword{apa} at many levels, and a dedicated lift provides access on the outer face. The cable is brought up to the top of the platform on spools using the switchyard described below. The \dword{apa}s are moved next to the stair tower work faces, also using the switchyard.  Sufficient space is needed at the top levels to allow people to work on the cabling and test the electronics. The size and the layout of the top level of the tower will be optimized based on input from the future prototyping tests %to be done 
at Ash River. 

\begin{dunefigure}[Cleanroom material transport system]{fig:cleanroom-switchyard}
  {The switchyard inside the cleanroom and airlock. The beams of the switchyard are highlighted in red. Fixed beams running east to west (the \dword{detmodule}'s long axis) are used to move the The \dword{apa} and The \dword{cpa} to a dedicated work location while a bridge crane running north-south is used to transport the units between the different fixed beams. }
\includegraphics[width=.75\textwidth]{cleanroom-switchyard}
\end{dunefigure}

Once the \dword{apa} pair or the \dword{cpa} panels are assembled, the units will be moved around the cleanroom and airlock using a dedicated rail-based switchyard (I-beam and trolley). The switchyard is illustrated in Figure \ref{fig:cleanroom-switchyard}; %in the center of the figure is 
the bridge crane, which travels north-south, is in the center. Several bridge beams driven by electric trolleys %will 
move on the runway beams of the crane.  By aligning the bridge beams with a set of fixed beams supported from the cleanroom roof, the \dword{apa} and \dword{cpa} can be transferred from the fixed beams to the bridge crane and %then 
moved to different locations in the cleanroom. Figure~\ref{fig:cleanroom-switchyard} shows a single fixed beam in front of the \dword{apa} assembly tower in the airlock and seven locations on the right where equipment can be moved during assembly and testing. The work stations are on the two faces of the \dword{apa} cabling tower, the two beams through the \dword{tco} \fixme{are?} used to move the finished detector into the cryostat, and the three I-beams  \fixme{are?} used to move the \dword{apa} into the \coldbox{}es in the northeast corner of the cleanroom. An additional fixed beam must be added over the cabling tower to allow hoisting the cable reels %to be hoisted
to the top. 

%%%%%%%%%%%%%%%%%%%%%%%%%%%%
\subsection{Cryogenics and \coldbox{}es}
\label{sec:fdsp-tc-infr-cryo}

\fixme{anne to here 3/11}


After an \dword{apa} pair is fully assembled and cabled but before installation, it is thermally cycled  % before being installed in the cryostat. %To do this, three tall but narrow cryostats are needed that can be cooled as close to LAr temperature as reasonably possible. These cryostats are referred to as \coldbox{}es and require a dedicated cryogenics system. 
in a tall narrow cryostat, called a \coldbox{}, shown in Figure~\ref{fig:install-coldbox}). To test \dword{apa}s at a rate necessary to keep up with the installation plan, we will use three identical \coldbox{}es in the cleanroom. The \coldbox{}es require a dedicated cryogenics system that cools them  close to \dword{lar} temperature. 


%The \coldbox{}es used to thermally cycle the assembled \dword{apa} pairs before they are installed in the cryostat are shown in Figure~\ref{fig:install-coldbox}. To test \dword{apa}s at a rate necessary to keep up with the installation plan, three \coldbox{}es are needed inside the cleanroom. Each \coldbox has external dimensions of 14.0 \si{m} by 3.2 \si{m} by 1.3 \si{m} (H $\times$ L $\times$ W). Three layers of 100 \si{mm} thick foam insulation are used for the thermal isolation. This gives an internal dimensions of 13.4 \si{m} x 2.6 \si{m} x 0.7 \si{m} (H $\times$ L $\times$ W). Inside each \coldbox, a rail section similar to the ones used in the \dword{tco} as well as other fixed rails in the cleanroom will be mounted.This will allow the \dword{apa} to be pushed into the \coldbox{}es using the cleanroom switchyard and trolleys. A support base will be needed under the \coldbox{}es to adjust the height to mate with the cleanroom switchyard.

% Each \coldbox has external dimensions of 14.0 \si{m} by 3.2 \si{m} by 1.3 \si{m} (H $\times$ L $\times$ W). Three layers of 100 \si{mm} thick foam insulation are used for the thermal isolation. This gives an internal dimensions of 13.4 \si{m} x 2.6 \si{m} x 0.7 \si{m} (H $\times$ L $\times$ W). Inside each \coldbox, a rail section similar to the ones used in the \dword{tco} as well as other fixed rails in the cleanroom will be mounted.This will allow the \dword{apa} to be pushed into the \coldbox{}es using the cleanroom switchyard and trolleys. A support base will be needed under the \coldbox{}es to adjust the height to mate with the cleanroom switchyard.

A \coldbox has external dimensions of 14.0 \si{m} by 3.2 \si{m} by 1.3 \si{m} (H $\times$ L $\times$ W). With three layers of 100 \si{mm} thick foam insulation,  
the internal dimensions are 13.4 \si{m} by 2.6 \si{m} by 0.7 \si{m}. A rail section similar to those used elswhere in the cleanroom will be mounted inside each \coldbox to allow the cleanroom switchyard and trolleys to push an  \dword{apa}  into a \coldbox. A support base under the \coldbox{}es will adjust the height to mate with the cleanroom switchyard.

%The \coldbox{}es will have an electronics feedthrough similar to what is used on the top of the \dword{dune} cryostat, except that short cables will be run from the \dword{wec}  to a patch panel inside the \coldbox.This will allow the cable on the \dword{apa} to be connected directly to the test readout without having to remove any cabling. Each \coldbox will be designed like the successful \dword{protodune}-\dword{sp} \coldbox. The outer shell is constructed of a stainless steel plate with reinforcing ribs welded on. The biggest difference between the \dword{dune} \coldbox{}es and the \dword{protodune}-\dword{sp} \coldbox, other than doubling the height, is that a hinged door is planned for the \dword{dune} \coldbox{}es. In \dword{protodune}, unbolting the door and lifting it off the \coldbox required significant effort. In \dword{dune}, the cleanroom does not have full crane coverage, so doors that can be opened and closed using a scissor lift are necessary.Initial estimates are that the \dword{dune} \coldbox{}es will need about 11 \si{T} of stainless steel. They must be assembled in place because the finished boxes are too big to fit down the Ross Shaft. The \coldbox base must be on Hilman rollers, so the boxes can be moved under the bridge after installation to allow installation of the cryogenic circulation pumps.
 
The \coldbox electronics feedthroughs will be  similar to what is used on the top of the \dword{dune} cryostat, except that short cables will be run from the \dword{wec}  to a patch panel inside the \coldbox. This will allow the cable on the \dword{apa} to connect directly to the test readout without having to remove any cabling. The \coldbox  design is nearly the same as the successful \dword{pdsp} \coldbox. The outer shell is similarly constructed of a stainless steel plate with reinforcing ribs welded on. The height is of course doubled, and a hinged door is planned. Unbolting the door and lifting it off the \dword{pdsp} \coldbox required significant effort, and lacking full crane coverage in this case, doors that can be opened and closed using a scissor lift are necessary. The \dword{dune} \coldbox{}es will collectively need about 11 \si{T} of stainless steel, according to initial estimates. They must be assembled in place because the finished boxes are too big to fit down the Ross Shaft. The \coldbox base must be on Hilman rollers, so the boxes can be moved under the bridge after \fixme{detector?} installation  to allow installation of the cryogenic circulation pumps.




\begin{dunefigure}[Installation \coldbox]{fig:install-coldbox}
  {\coldbox{}es used to thermally cycle the fully assembled APA pairs. }
\includegraphics[width=.5\textwidth]{graphics/install-coldbox.pdf}
\end{dunefigure}

%DAVID added cryo cold box%%
%\subsubsection{Cryogenics supporting the Cold--Boxes}
%{Cold Boxes Cryogenics}
\label{sec:fdsp-tc-cryocoldbox}

The cryogenics supporting the cold--boxes underground needs to ensure reliable and safe operations of the system used to test the \dwords{apa} that will be installed inside the cryostats. The main functional requirements are:
\begin{itemize}
\setlength\itemsep{1mm}
\setlength{\parsep}{1mm}
\setlength{\itemsep}{-5mm}
\item It needs to support three cold--boxes testing dual \dwords{apa} operating in parallel: one in cool down mode, two either in steady-state or warm-up.
\item It needs to allow personnel in the cleanroom during all phases of the purge, cool down, operation and warm-up modes 
\item It needs to test the detector modules at near \dword{lar} temperature.
\item It needs to operate 24/7 for 10 years.
\item It needs to allow remote operations.
\item It needs to be located in the vicinity of the \dword{tco}. Space is available on top of the cryogenic mezzanine on the roof of the cryostat.
\end{itemize}

It needs to fulfill the following modes of operations:

\begin{itemize}
\setlength\itemsep{1mm}
\setlength{\parsep}{1mm}
\setlength{\itemsep}{-5mm}
\item \textbf{Purge}. During this mode air is removed from the system (cold--box and cryogenic system) and replaced with dry nitrogen. The concentration of moisture is monitored and when it no longer decreases, the cool--down can commence.
\item \textbf{Cool--down}. Cold nitrogen is introduced in the system to cool it down and to cool down the detector contained inside the cold--box. It is expected to take 24 hours, during which the temperature goes from room temperature to about \SI{90}{K}. 
\item \textbf{Steady--state operations}. After reaching the nominal temperature of about \SI{90}{K}, the value is maintained for 48 hours, during which the detector is turned on and fully tested at cold. 
\item \textbf{Warm--up}. After the completion of the test, the system is slowly warmed up. It is expected to take 24 hours, during which the temperature goes from about \SI{90}{K} to room temperature.
\end{itemize}

\begin{dunetable}
[Cryogenics for cold--boxes specifications]
{cc}
{tab:table-cryo-cold-boxes}
{Table of parameters for the cold--box cryogenics.}
Parameter & Value 
\\ \toprowrule
Dual \dword{apa} thermal mass &  1,600 kg\\ \colhline
Temperature uniformity & +60 K / -0 K \\ \colhline
Electronics load & 300 W \\ \colhline
Cold--box insulation thickness &  0.3 m \\ \colhline
Target cool--down temperature &  \SI{90}{K} \\ \colhline
Target cool--down duration &  24 hr \\ \colhline
Target steady--state duration &  48 hr \\ \colhline
Target warm--up duration &  24 hr \\ \colhline
Maximum cooling power  &  \SI{13}{kW}  \\ \colhline 
Maximum liquid nitrogen consumption  &  \SI{300}{l/hr}  \\ \colhline 
\end{dunetable}

The evaporation of liquid nitrogen provides the cooling power for the system. Warm nitrogen and a heater provide the heating power. At peak consumption, the expected maximum heat load is \SI{8.5}{kW}. Assuming a 50 $\%$ margin on the refrigeration load, the cryogenic system requires \SI{13}{kW} of net cooling power at peak consumption, which equals about \SI{300}{l/hr} of evaporating liquid nitrogen.

Two layouts are currently being considered: closed loop with mechanical refrigeration, in which liquid nitrogen is generated in situ, circulated and the spent nitrogen re-condensed before being put back in the system; open loop, in which liquid nitrogen is transported underground by means of portable dewars, circulated and the spent nitrogen vented away. For the closed loop, a mechanical refrigeration capable of supplying \SI{13}{kW} of cooling is being investigated. For the open loop, it is possible to use a 2,000 l dewar, which is commercially available and transportable up and down the Ross Shaft inside the cage. To supply the required amount four trips per day are needed.

The current version of the closed loop system is presented in Figure~\ref{fig:mechanical-refrigeration}. The current version of the open loop system is presented in Figure~\ref{fig:LN2}.

\begin{dunefigure}[Coldbox cryogenic support system based on mechanical refrigeration ]{fig:mechanical-refrigeration}
  {Layout of the cryogenics supporting the \dword{apa} test facility with mechanical refrigeration.}
\includegraphics[width=.98\textwidth]{graphics/Cryo-cold-box-mechanical.pdf}
\end{dunefigure}

\begin{dunefigure}[Coldbox cryogenic support system based on LN2 ]{fig:LN2}
  {Layout of the cryogenics supporting the \dword{apa} test facility with open loop refrigeration.}
\includegraphics[width=.98\textwidth]{graphics/Cryo-cold-box-LN2.pdf}
\end{dunefigure}



%%%%%%%%%

% clear the figure buffer before starting the next section
\clearpage

%%%%%%%%%%%%%%%%%%%%%%%%%%%%
\subsection{Prototyping and Testing (QA/QC)}
\label{sec:fdsp-tc-infr-qaqc}

%The installation setup phase is where all of new equipment is installed underground. This represents a lot of new and unique work. While most of these installation procedures have already been tested during the trial assembly at Ash River, everything must be properly approved. The main \dword{apa} and \dword{cpa} towers will already be structurally approved, but load tests must be performed on all lifting fixtures, shuttle beams, crane tower connections, and \coldbox connections. 
Installing all this new equipment underground during the installation setup phase involves a lot of new techniques  and unique work. While most of the procedures will have been tested during the trial assembly at Ash River, everything must be properly approved. The main \dword{apa} and \dword{cpa} towers will already be structurally approved, but all lifting fixtures, shuttle beams, crane tower connections, and \coldbox connections must undergo load tests. 

%This is also when the \coldbox{}es and cryogenic system will be tested, so access to the cleanroom  may be restricted for several days for system checks.
The \coldbox{}es and cryogenic system will also be tested, so access to the cleanroom  may be restricted for several days for system checks.

%%%%%%%%%%%%%%%%%%%%%%%%%%%%
\subsection{Safety}
\label{sec:fdsp-tc-infr-safety}
%At this phase of construction of Detector 1, larger teams from the consortia, SDSD,  and contractors will need access to underground facilities at the same time \dword{lbnf} is completing the cold structure on Detector 1 and beginning the warm structure on Detector 2. Shift schedules now switch to 2 shifts/day because a maximum of 140 \dword{fte} can be underground at any given time.  Daily work meetings and safety coordination led by the shift supervisor at the start of each shift to coordinate work activities, hazard/mitigation reviews, and installation procedures are critical for a safe work environment. More details can be found in section 1.4.4 under the Installation \dword{esh} section. 
During this phase, as new equipment is being installed and tested, new employees and collaborators will be trained, and larger teams from the consortia, \dword{sdsd},  and contractors will need access to underground facilities.  This is at the same time that \dword{lbnf} is completing the cold structure on \dword{detmodule} \#1 and beginning the warm structure on  \dword{detmodule} \#2. Given the maximum of 140 \dword{fte} underground at any given time, we move to two shifts per day.  At the start of each shift, the shift supervisor will lead a work meeting to coordinate work activities, review hazards and mitigations, and review installation procedures to ensure a safe work environment. More details can be found in section 1.4.4 under the Installation \dword{esh} section. \fixme{anne to fix sectioning}

%The \dword{uit} will be extremely busy at this time. The crew sizes are increasing, so training new employees and consortia will be ongoing, and new equipment is being installed and tested. Proper documentation of structural calculations, assembly drawings, load tests, hazard analyses, and procedures will all have to be reviewed and approved before operational readiness is approved. This all helps prepare for a safe start to the underground installation. 
Proper documentation of structural calculations, assembly drawings, load tests, hazard analyses, and procedures will all have to be reviewed and approved before operational readiness is approved. \fixme{operational readiness? is this different from readiness to begin installation?} This all helps prepare for a safe start to the underground installation. 

%The \coldbox and cryogenic system is one of the few things in the installation process that will not be fully tested during any of the trial assembly work. While the design of the boxes are very similar to the what was done at CERN for \dword{protodune}, the major difference is the system requirement that it is safe for workers to be in the cleanroom during the cool down phase. Procedures for operating the \coldbox have not been defined, but a requirements document has been written.   
Unlike most items, the \coldbox and cryogenics system will not be fully tested during the trial assembly work. While the new \coldbox design is very similar to \dword{pdsp}'s, some new requirements are in effect. In particular, it is \fixme{now?} safe for workers to be in the cleanroom during the cool down phase. \fixme{it was permitted for protodune but is no longer safe, or it was not permitted earlier, but will be ok now?} Procedures for operating the \coldbox will be written following the established requirements. 

  
%%%%%%%%%%%%%%%%%%%%%%%%%%%%
\subsection{Costs, Schedule, and Risk Analysis}
\label{sec:fdsp-tc-infr-cost}

\fixme{Add costs when available}

{\bf Installation Setup Phase:} %This is a more difficult phase to schedule and may be adjusted often with multiple projects going forward at once. The \dword{cf} work is basically completed, which reduces the number of \dwords{fte} underground, allowing us to begin the installation infrastructure. We begin two 10 hour shifts per day as the work ramps up.  Once the cold cryostat is approximately six months into the installation schedule, floor space becomes available in the north cavern. The \coldbox construction must be started ASAP because the welding takes approximately six months. In parallel to this, the machine shop area can be set up and the bridge between the north and south sides of the cavern can be constructed.  Once the bridge is completed, work on the assembly crane, \dword{apa} cabling tower, \dword{apa} assembly tower, and \dword{cpa} assembly station can begin. 
    This is a more difficult phase to schedule and may require frequent adjustment, with multiple projects going forward at once. The \dword{cf} work is essentially completed by this point, which reduces the number of \dwords{fte} underground, allowing us to begin the installation infrastructure. 
    \fixme{Sort of a non sequitur. Reduced cf FTEs allows you to have more FTEs, but it's not clear that it's what allows you to begin the installation infra } We begin two ten-hour shifts per day as the work ramps up.  Once the  cryostat cold structure is approximately six months into the installation schedule, floor space becomes available in the north cavern. The \coldbox construction must begin immediately at this point because the welding takes approximately six months. In parallel, the machine shop area can be set up and the bridge between the north and south sides of the cavern can be constructed.  Once the bridge is completed, work on the assembly crane, \dword{apa} cabling tower, \dword{apa} assembly tower, and \dword{cpa} assembly station can begin. 
    
%Installation of the detector support system could begin during the final installation stages of the cryostat cold structure because they both require full height scaffolding for all the welding on the top of the cryostat. This was how the \dword{protodune} \dword{dss} was installed. The details have not yet been worked out with the contractor, so work may be done in stages. This requires a crew on top of the cryostat installing the \dword{dss} support feedhroughs from the top of the cryostat as shown in Figure \ref{fig:install-dss-feedthru}. 
Installation of the \dword{dss} could begin during the final installation stages of the cryostat cold structure because they both require full-height scaffolding for the welding on the top of the cryostat. The \dword{protodune} \dword{dss} was installed this way. This requires a crew on top of the cryostat installing the \dword{dss} support feedhroughs from the top, as shown in Figure~\ref{fig:install-dss-feedthru}.  The details have not yet been worked out with the contractor, so work may be done in stages. 

A detailed schedule of the installation infrastructure time period is shown in Figure~\ref{fig:Installation_Infrastructure_Schedule}.
    
\begin{dunefigure}[Schedule for the infrastructure installation]
{fig:Installation_Infrastructure_Schedule}
    {Schedule for the infrastructure installation}
\includegraphics[width=0.98\textwidth]
{Installation_Infrastructure_Schedule} 
\end{dunefigure}

\fixme{A graphical representation of the schedule is used since it is critical to convey the time sequencing and how the work is overlapping. Jim }

% clear the figure buffer before starting the next section
\clearpage


\section{Integration and Test Facility (ITF)}
\label{sec:fdsp-tc-itf}

This section describes the baseline plan to assemble the \dword{apa}, \dword{ce}, and \dword{pd} in a surface facility. The installation team is investigating in parallel the possibility of performing the work underground. A decision will be made as to what is the optimal course of action prior to the TDR being finalized.

The components of the \dword{dune} detectors will be manufactured in several different countries. 
Many of the parts can be reasonably shipped to the logistics warehouse and then underground where it can be installed. 
However, the  \dword{ce} and the \dwords{pd} are tightly coupled to the \dword{apa}s. 
The wires and filters on the \dword{apa}s form part of the electronics circuit, and the photon supports and cabling are built into the \dword{apa}s. 
Integrating the \dword{ce} and \dword{pd}s into the \dword{apa}s is a large task.  The present plan is integrate and test the components as early in the process as possible in a surface facility.  
To avoid creating integration testing facilities at each factory, one central facility will be established in South Dakota near the \dword{surf} site. 
The \dword{apa}s, \dword{ce}, and \dwords{pd} modules will arrive in this \dword{itf}, undergo initial tests, be integrated, and then undergo a set of warm tests. 

Because the \dword{ce} will be available approximately two years before installation begins, the \dword{itf} must be available on the same time scale. Other components are, in fact, available earlier. 
Table \ref{tab:specs:just:SP-TC} summarizes the specifications for the \dword{itf};  
the relevant specifications involve the quality of the cleanroom and the UV light filtering for the \dwords{pd}.



%\begin{dunetable}
%[ITF Specifications]
%{cc}
%{tab:tcps-itf-spec}
%{Summary of the high level specifications for the ITF. %The building requirements are covered separately in a separate section.}
%Parameter & Specification \\ \toprowrule
%Cleanroom & The ITF cleanroom shall meet ISO-8 standard per ISO-14644 \\ \colhline
%Filtered Lights & <520 nm for long exposure and <400 %for exposures less than 2 weeks \\ 
%\end{dunetable}



%%%%%%%%%%%%%%%%%%%%%%%%%%%%
\subsection{APA-CE-PD Integration}
\label{sec:fdsp-tc-itf-integ}

\begin{dunefigure}[ITF cleanroom layout]{fig:fdsp-tc-itf-clean}
{Conceptual layout of the cleanroom for APA-CE-PD integration and testing.}
\includegraphics[width=0.8\textwidth]{itf-cleanroom-v2}
\end{dunefigure}

Most the work in the \dword{itf} must be done in a cleanroom environment to protect the components from dust and unfiltered light. The cleanliness requirement for the detector components is ISO-8 which corresponds to filtered air with clean lab coats, clean shoes, and hair nets. 
To protect the photon detector's \dword{wls} coating, the lights must be filtered to remove wavelengths below 520nm.\cite{LBNE-docdb-8348} 
Figure \ref{fig:fdsp-tc-itf-clean} shows one possible layout of the cleanroom.
Materials enter the \dword{itf} cleanroom through the materials airlock. This area must be sufficiently large to accommodate \dword{apa} transport boxes and allow workers to move around the box to remove the dirty shipping layer and prepare the equipment for transport into the cleanroom. 
Other materials will also be brought into the cleanroom through the airlock, but they will need much less space than the \dword{apa} crates. Figure \ref{fig:fdsp-tc-itf-clean} shows an \dword{apa} transport box being moved into the airlock on the lower right.
The \dword{ce} and \dword{pd} equipment will be moved to their own work areas. 
Here the components are unpacked and tested before integration into the \dword{apa}. 
The tests performed are described in more detail in the \dword{qc}/\dword{qa} section below. 

The \dword{apa}s enter the cleanroom through the airlock and initially go to the \dword{apa} handling area where an overhead workstation or gantry crane will be available. 
The \dword{apa} will then be removed from the transport box and mounted to a process cart that can rotate the \dword{apa} horizontally. The process cart will be pushed to one of the four \dword{apa} integration areas where they are prepared for the installation of the \dword{ce} and \dword{pd}. 
During the integration process, the \dword{apa} will be held horizontal and the \dwords{pd} will be inserted into the sides while the \dword{ce} boxes are mounted to the end of the \dword{apa}. 
After integrating the components, the system will be tested and then moved back to the handling area and boxed for shipping to the logistics center for storage.  

All detector  components will arrive at the \dword{itf} either from the logistics facility or directly from the factories. Sufficient space will be needed inside the \dword{itf}, but outside the cleanroom, to store the material needed for several weeks as well as a few of the integrated \dword{apa} boxes. Additionally a changing room is needed, so workers can change into clean clothes and shoes. The changing room should have a capacity sufficient for the 10-20 workers needed in the cleanroom. 

%%%%%%%%%%%%%%%%%%%%%%%%%%%%
\subsection{ITF QA/QC}
\label{sec:fdsp-tc-itf-qaqc}
Extensive testing of the detector components will be performed inside the \dword{itf} facility as the APA-PD-CE integration takes place. These tests are a vital part of the quality control process for \dword{dune}. Details of the tests for each of the components are described below.

\subsubsection{APA}
The \dword{apa} will be unpacked from the transport box, installed on the process cart, and the protective shields removed to allow a detailed visual inspection. Then it will be transported to the integration area, where it will be held horizontally. The main test to be performed at this stage, i.e., before integration with the \dwords{pd} and \dword{ce}, is tension measurement. Ideally, all wires would be measured to ensure that no changes occurred during shipping. The limiting factor for the tension measurement will be time. In the current plan, approximately 350 wires, representing 10$\%$ of the total, will be measured, which will take 3 shifts with 2 people. All the measured values will be stored in the wire \dword{qc} database. In the current plan, the tension measurement will be performed using a laser, the same method used at the production site. This method uses a laser focused on individual wires. By plucking the wire to induce a vibration, a photodiode under the wire records the frequency of vibration, which directly translates into the tension value. While this method is robust and has been extensively used by \dword{lartpc} experiments, it is very time consuming. An alternative method, using electrical signals, is currently under development and could replace the laser method, potentially allowing all \dword{apa} wires to be measured at the \dword{itf} in less time.

The current requirement for tension values are 6$\pm$1 N. Wires measuring outside this range will be removed from the \dword{apa}. Note that the exact tolerance is currently under study with \dword{protodune} data to ensure the required acceptable range; otherwise, a channel may have to be removed.

The last test performed to ensure the quality of an \dword{apa} is the wire continuity. This checks that all wires are still intact and properly connected to the readout boards. This test can easily be done once the \dword{ce} is installed (see next sub-section).    
\subsubsection{Cold Electronics}
The \dword{qa} for the \dword{ce} is described in the \dword{ce} chapter.

\subsubsection{Photon Detectors}

\dwords{pd} will arrive at the \dword{itf} in custom designed crates.  Each crate will contain the ten photon detector modules required for a single \dword{apa}.  
Each \dword{pd} comes individually packaged in a static-resistant sealed plastic bag, filled with clean dry nitrogen.

Before each \dword{pd} is integrated into an \dword{apa}, it is removed from its shipping bag and inspected visually. 
Modules passing this initial inspection are then loaded into the optical scanner for operational testing.

The \dword{pd} optical scanner tests the operation of the photosensor readout chain to ensure all electrical connections are operational and also measures light-collection performance at several positions along the length of the module.  
Duplicate identical optical scanners are used at the module assembly facility, where a scan is the last \dword{qc} test before the module is shipped to the \dword{itf}; the module is tested again immediately before installation, allowing a sensitive test for any changes in module performance due to shipping or storage.  
This technique was used successfully in \dword{pdsp}, and will be replicated for \dword{dune}.

The optical scanner consists of a light-tight box, approximately \num{2.5}m long, with a \num{0.75} $\times$ \num{0.75}m cross section. 
The box is fabricated of aluminum and acts as a Faraday cage to minimize electrical interference with measurements. 
In the \dword{dune} configuration, two \dword{pd} modules to be tested are inserted into the station through slots on the face of the box, guided by support rails of the type used in the \dword{apa}s, representing a final mechanical check of the dimensions of the module.  
Electrical connection to the module uses an electrical connector identical to the ones in the \dword{apa} frames, allowing a final check of that crucial interface.
Following insertion into the scanner, the insertion slots are closed and optically sealed, and the scan begins. 
\dword{dune} \dword{pd} readout electronics are used to bias and read out the module photosensors, while a UV LED is scanned along the length of the modules by an automated stepper-motor driven translation stage.  
Measurements of the detector responses are made at 16 positions along the length of the module (on two sides for double-sided \dword{pd} modules), checking the performance of each of the dichroic filters. 
The response is compared to that measured in the assembly facility. 
Figure \ref{fig:fdsp-tc-pds-scanner} shows the scanner used to test the \dword{pdsp} photon detectors.


\begin{dunefigure}[Photon detector scanner]{fig:fdsp-tc-pds-scanner}
{Picture of the scanner for operational tests of the \dword{pd} modules before the modules are inserted into the \dword{apa}s.} 

\includegraphics[height=.60\textheight, angle=0]{pds-scanner}
%\includegraphics[width=1.0\textwidth, angle=-90]{pds-scanner}
\end{dunefigure}

Photon detector are inserted into the \dword{apa}s immediately follows optical scanning.  
Modules are inserted onto the \dword{pd} support rails; the connection to the cable harness, which is pre-installed in the \dword{apa} before wire wrapping, is automatic as the module is inserted.   
Immediately after the module is inserted, an electrical continuity check ensures continuity between the \dword{pd} module and the \dword{pd} cable end connector where it exits the end of the \dword{apa}.



%%%%%%%%%%%%%%%%%%%%%%%%%%%%
\subsection{Building Requirements and Infrastructure}
\label{sec:fdsp-tc-itf-req}
The \dword{itf} building requirements are summarized in DocDb 11500.\cite{bib:docdb11500} The building to be used as the \dword{itf} has not yet been designated. To help identify or design the \dword{itf} building, a set of requirements were drafted. The requirements document defines the spaces needed for integration work but does not specify the final layout of the cleanroom. This will allow cleanroom spaces to be configured as part of building footprints as candidate buildings are identified. Because the building has not been designated, the cleanroom layout shown in Figure \ref{fig:fdsp-tc-itf-clean} should be taken as a concept; the final layout may change depending on the footprint of the building chosen as the \dword{itf}. The building requirements document\cite{bib:docdb11500}  defines the minimum spaces needed for all operations inside the \dword{itf} cleanroom. It also defines the space needed for the coldbox and the related cryogenic system, and it provides guidance for the space needed for material storage outside the cleanroom. It also establishes requirements for power and other general needs. Some requirements depend on the location of the building and the facilities available in the area. For example, office space for 20 scientists working in the \dword{itf} will be needed in the area but would not necessarily need to be in the \dword{itf} building itself if other local options are available. Some local machining facilities also fall into this category. 

%%%%%%%%%%%%%%%%%%%%%%%%%%%%
\subsection{Safety}
\label{sec:fdsp-tc-itf-safety}

Information on \dword{itf} safety is identical to what is listed in Section 1.1.4 (Logistics Safety) because the \dword{itf} is also operated by SDSD as a Fermilab facility.    If the \dword{itf} and logistics facility are near each other, one safety officer can be shared by the two sites.    

%%%%%%%%%%%%%%%%%%%%%%%%%%%%
\subsection{Cost, Schedule and Risk Analysis}
\label{sec:fdsp-tc-itf-cost}

\fixme{Add costs when they are available}

{\bf \dword{itf} Time Line and \dword{apa} Integration Schedule}
The time line for starting up the \dword{itf} is late 2022 when \dwords{pd} and \dword{ce} are available to integrate into the \dword{apa}s.  This is two years 
before installation begins. This schedule has the advantage of allowing all integrated components to be tested as soon as possible and minimizing any schedule risk if problems occur. However $3/4$ of the \dword{apa} are produced before the first \dword{ce} module is available for the risk reduction applies primarily to the electronics. The integration team size is smaller than what would be required underground as the work proceeds over a longer time. 

Using four work stations and working with only a day shift, installing an \dword{apa} pair should take approximately nine shifts as shown in Figure \ref{fig:ITF-Schedule}. The 150 \dword{apa} then require 270 work days.

\begin{dunefigure}[Schedule APA Integration in ITF]
{fig:ITF-Schedule}
    {\dword{itf} Schedule}
\includegraphics[width=0.98\textwidth]
{ITF-Schedule.pdf} 
\end{dunefigure}


\fixme{A graphical representation of the schedule is used instead of the template as the duration and relative timing of the work is critical for the installation. Jim}


\section{Detector Installation}
\label{sec:fdsp-tc-inst}

\subsection{Introduction}
\label{sec:fdsp-tc-inst-intro}

The DUNE detector installation will proceed in three phases the CUC setup phase, the installation setup phase, and the detector installation phase. Figure \ref{fig:high-level-schedule} 
shows the major underground activities and gives an idea of what work will be occurring in which phase. The CUC setup phase is the first step in the installation process. The start of the CUC setup phase begins when the underground area for the North Cavern and Central Cavern become available to LBNF and DUNE. At this time the cryostat construction can begin in The North Cavern and DUNE equipment installation can begin in the central cavern referred to as the Central Utility Cavern (CUC). A top view of the underground areas is seen in Figure~\ref{fig:cavern-layout}. The main equipment from DUNE which is installed in this phase is the infrastructure in the DUNE dataroom. The dataroom is found in Figure~\ref{fig:cavern-layout} in the CUC on the West end of the excavation. The detector installation setup phase (referred to as Infrastructure Det\#1 in Figure \ref{fig:cavern-layout}) begins during the cryostat membrane installation period. In this phase the equipment needed to perform the detector installation will be erected in North Cavern. This includes the installation of the bridge across the cavern, the installation cleanroom, lifting equipment and work platforms, the cold boxes and cryogenic system for testing APA, and the DSS with related switchyard. In the third phase of the installation the detector itself will be installed. The work in each phase will be described the following sections.

\begin{dunefigure}[High level installation  schedule]{fig:high-level-schedule}
  {Overview schedule showing the main activities underground.}
\includegraphics[width=.99\textwidth]{high-level-schedule}
\end{dunefigure}

\begin{dunefigure}[Layout of the DUNE underground areas]{fig:cavern-layout}
  {Top view of the layout at the 4850 level at SURF. Shown are the three large excavations and the location of detectors in excavation \#1 and \#3. Excavation \#2 is the CUC which houses the DUNE dataroom and the underground utilities.}
\includegraphics[width=.9\textwidth]{cavern-layout}
\end{dunefigure}



%%%%%%%%%%%%%%%%%%%%%%%%%%%%
\subsection{Installation Process Description}
\label{sec:fdsp-tc-inst-proc}

\subsubsection{CUC Installation Phase}

\begin{dunefigure}[Layout of the DUNE dataroom and experimental workarea in CUC]{fig:install-cuc}
  {TOP: The overall layout of the DUNE spaces in the CUC is shown. The inner room is the DUNE dataroom which houses the underground computing and the outer area referred to as the experimental workarea is a general purpose workarea. Bottom: The first row of ten racks in the data room is shown. The first two racks are the  CF interface racks. The image was taken from the ARUP design drawing U1-FD-T-701}
\includegraphics[width=.7\textwidth]{cuc-layout}
\includegraphics[width=.9\textwidth]{cuc-cf-racks}
\fixme{need to use a figure from the 90\% submittal}
\end{dunefigure}


The first stage of the CF work ends when the outfitting of the North and Central caverns is complete. At this time the CUC is ready for DUNE outfitting and the cryostat installation can start for detector \#1. At this time DUNE will not have assess to the detector excavations as the heavy steel work for the cryostat will be ongoing. The only work planned for DUNE in this period is the outfitting of the dataroom and work area room in the CUC shown in Figure~\ref{fig:install-cuc}.  CF is providing redundant single mode fiber up the shafts to provide external connectivity.  CF also will provide the empty dataroom with an 18 \si{in} false floor, a 500 \si{kVA} power disconnect, and connections for chilled water sufficient to cool the racks. The dataroom like the adjacent CF electronics room will be outfitted with a dry fire extinguishing system. 
\fixme{When are fibers available?}

The water cooled racks, cable trays, power distribution, and water distribution are the responsibility of DUNE and will be installed once the space becomes available, as will the installation of the DAQ fiber trunk between the detector cavern and the CUC dataroom. The installation of the racks needs to be coordinated with CF as the first two racks are for CF usage and need to be in place before the underground phase one work is complete. Some small overlap will be needed between CF and DUNE at this time. The general purpose network will be installed by FNAL's SCD and connected to the shaft fiber. This is required for most further work in the underground area.

Data from the detector electronics will be transmitted over a multimode fiber trunk from the warm interface boards on top the detector to the DAQ dataroom in the CUC shown in Fig.~\ref{fig:install-cuc}.  This room will contain 60 water cooled racks: two of which are reserved for CF use, two for CISC servers, and the rest for DAQ servers and networking. Although this is the total number of racks needed for all four detector modules, the racks themselves will be installed at the beginning of the CUC commissioning phase, as they must be plumbed into the cooling water below the data room's drop floor and wired into power distribution from the ceiling.  DAQ equipment will populate the racks as needed for servicing the detector commissioning.  For the first detector module, details of this configuration will be informed by knowledge gained from DAQ vertical slice tests done at other institutions.  

At the same time the eight above ground DAQ racks which receive the data from the underground dataroom and then transmit the data to FNAL will also be installed, connected to the WAN, and connected to the single mode fiber in the shafts. With this infrastructure in place the DAQ group can begin constructing and testing the final DUNE DAQ.  The timing system is the first DAQ component needed.  Enough DAQ back-end servers to support the first APAs will be operational before the APAs are installed.  The remainder of the DAQ will grow in parallel with the APA installation.

The underground experimental work area is a general purpose area that will need to serve many purposes during the DUNE installation. Initially the area will be outfitted with office equipment for the installation team, workstations for DAQ, and a basic conference area for meetings.  The room is 4-6.4 \si{m} deep and 20 \si{m} long so it can serve several functions.

During this early installation stage the machine shop and DUNE storage area will also be setup in the detector excavation area. It is expected these facilities will be shared with the cryostat team. 

\subsubsection{Installation Setup Phase}

Once the steel structure of the cryostat is complete the remaining work by the cryostat team will be focused inside the cryostat. 
There will still be a lot of activity outside the cryostat as the 4,000 crates of foam and other materials are transported inside, but the outer structure will be in position so some DUNE work can start. 
The first piece of equipment to installed will be the bridge between the North and South drifts. 
This will allow the cryogenics equipment to travel from the North drift to the CUC and will provide part of the structure for the cleanroom. 
Construction of the cleanroom frame and related hoisting equipment can then begin. 

The largest most complex equipment that must be constructed in this phase are the cold boxes and related cryogenics system. 
Due to the size of the cold boxes these must be constructed in place. 
The layout of the installation region outside the cryostat is shown in Figure \ref{fig:install-cleanroom-layout}. 
This figure shows a cut below the bridge so the majority of the installation equipment can be seen. 
The three 15 \si{m}tall cold boxes are seen in blue next to the yellow access stairway. 

The hoisting equipment and the rail system needed to move the detector components in the installation area interfaces with the bridge, cryostat and possibly the cleanroom mechanical structure. 
Installing this early will make transporting equipment around the area outside the cryostat significantly simpler. 
 
The assembly towers for the APA assembly, APA cabling, and CPA assembly can be installed when it is most convent for the cryostat installation crew. 
Just prior to the start of detector installation the cleanroom walls will be installed, the area will be cleaned and added filters will be installed to convert the area to an ISO-8 cleanroom.

During this phase of the cryostat installation the majority of the cryostat work will be inside an at the 4910 level. 
This will allow significant work to be performed on the cryostat roof for the cryogenic system installation and the detector installation setup. 

\begin{dunefigure}[Installation of electronics crosses]{fig:install-elect-cross}
  {Installation of the crosses onto which the cold electronics warm readout and the photon detector cables are connected.}
 \includegraphics[width=.75\textwidth]{install-elect-cross}
\end{dunefigure}

The cryostat crossing tubes will be installed on the roof as required by the cryostat assembly sequence. 
These assemblies are welded to the 1 \si{cm} thick steel cryostat roof and are then additionally cross braces to the large I-Beams. 
The thick walled tubes which penetrate through the foam insulation are in place at this time. 
Once the crossing tubes are in place the large tees for the cold electronics can be installed. 
The next step in the installation requires the internal roof of the cryostat to be complete. 



Once the crossing tubes are installed and leak chased the crosses which onto which the feedthru flanges for the cold electronics and photon detector mount can also be connected. The setup is seen in Figure \ref{fig:install-elect-cross}. 
The height of the crosses is selected so a person can comfortably work on the WEC and PD flanges while standing on the cryostat roof. 
A fully assembled cross is shown on the right and a cross where the WEC are extracted to the assembly position is shown on the left. 
The present plan is to install the crosses shortly after the cryostat crossing tubes are installed. 
By doing this the large openings in the cryostat roof are sealed and cleanroom conditions can easily be achieved after the cryostat is cleaned. 
For this stage temporary seals will be used for the flanges as they will need removed during the cabling process later in the installation. At this time the cold electronics mechanical feedthrus can also be installed. 

\begin{dunefigure}[DSS feedthru installation]{fig:install-dss-feedthru}
  {The DSS support feedthru are installed using a gantry crane running on the roof of the cryostat.}
 \includegraphics[width=.95\textwidth]{graphics/dss-feedthru-install.pdf}
\end{dunefigure}

When the crossing tube installation is complete and in parallel to the installation of the cold electronics tees the DSS support feedthrus can be installed. A gantry crane on top of the cryostat is used to pick up the feedthru and feed them into the cryostat crossing tubes as shown in Figure \ref{fig:install-dss-feedthru}.
This is the first step in the \dword{dss} installation.
There are \num{20} \fdth{}s per row and five rows for a total of \num{100} \fdth{}s.  
A fixture with a tooling ball is attached to the
clevis of each \fdth.  
The $xy$ position in the horizontal plane
and the vertical $z$ position of this tooling ball is defined, then a  survey is performed to determine the location of each tooling ball center and $xy$ and $z$ adjustments are made to get the tooling
ball centers to within $\pm$\SI{3}{mm}.  
The \SI{6.4}{m} long I-beams are then raised and pinned to the clevis.  Each beam weighs roughly \SI{160}{kg} (\SI{350}{lbs}).
A lifting tripod is placed over each of the \fdth{}s supporting a beam, and a \SI{0.64}{cm} \SI{0.25}{in}  %$1/4 ^{''}$  
cable is fed through the top
flange of the \fdth down the \SI{14}{m} to the cryostat floor where it
is attached to the I-beam. The cable access port and lifting cable are shown in Figure \ref{fig:dss-beam-lifting} 
The winches on each tripod raise the beam in unison in order to get it to the correct height to be pinned to the \fdth clevis.  
Once the beams are mounted, a final survey of the beams takes place to ensure they are properly located and aligned to each other.

 \begin{dunefigure}[DSS I-Beam lifting setup]{fig:dss-beam-lifting}
  {A cable access port is included in the DSS flange. This is used to feed a cable from the roof through the flange and attacht it to the I-Beams during DSS installation.}
 \includegraphics[width=.95\textwidth]{graphics/dss-beam-lifting.pdf}
\end{dunefigure}

Once work on the feedthru are complete then the mezzanines for the cryogenic system and the detector electronics racks can be installed. Finally the cable trays,  piping, lighting, and cryostat roof flooring are installed. 
At this time the cryostat roof is ready for the start of the DAQ installation in the detector area which will proceed in parallel to the detector installation.

The last steps in the installation setup phase is the installation of the cryostat internal piping, cleaning the cryostat, installing the false floor, and filtered lighting to protect the photon detectors. 


\subsubsection{Detector Installation Phase}

\begin{dunefigure}[Top view of the installation region inside the cleanroom ]{fig:install-cleanroom-layout}
  {Top view of the cleanroom used for installation. In this view cleanroom roof and bridge are not shown. The equipment used for the installation is shown along with the material airlock layout and the location of the changing room. The cryostat is to the right of the figure and the I-Beams passing through the TCO opening are shown.}
 \includegraphics[width=\textwidth]{install-cleanroom-layout}
\end{dunefigure}

At the start of the detector installation phase the work on the cleanroom is complete, the DSS is installed in the cryostat including the switchyard near the TCO. 
The light in the cryostat and cleanroom will be filtered to protect the photon detectors and the air will be filled to reduce the dust collected during installation. 
The cryogenic piping will be in place along with the false floor. 
The false floor both gives a flat work area and protects the 1.2 \si{mm} thick stainless steel cryostat membrane. 
          
The layout of the cleanroom during the detector installation phase is shown in Figure \ref{fig:install-cleanroom-layout}. This image is a top view of the cleanroom with the roof and bridge removed; the cryostat is on the right and the open cavern on the left. In the North-East corner of the figure the access stairway is shown in yellow. This stairways is outside the cleanroom and allows people to access the 4910 level from the North drift or from the cryostat roof. An access corridor along the north wall connects the stairway to the rest of the cavern. The changing room and airlock are found on the West of the figure and are outlined in orange. The changing room is located in the the North-West corner and provided access to both the cleanroom and the airlock. South of the changing room is the material airlock. This rather large area is where the APA modules are connected together to form the 12 \si{m} tall units. The assembly station is shown in green along the south wall of the airlock. This area is also where all the materials are brought from the dirty outside cavern and are cleaned or the outer packaging removed. Roll-up doors along the West wall allow material from the outer cavern to be brought into the airlock. A section of the roof can be removed to allow the cavern bridge crane to manipulate the APA transport box and modules. The cleanroom area itself is shown on the right of the figure outlined in magenta. Inside the cleanroom is a switchyard material transport system (shown in red) similar to what is used for the DSS inside the cryostat. The switchyard is used to move the assembled detector components north-south in the cleanroom. East-west rail sections are then used to deliver the components to the required work location. In the north area of the cleanroom are the three cold boxes used to cryogenically test the fully assembled and cables APA pairs. In the south half of the cleanroom is the APA cabling tower used for cabling and testing the APA pairs. Along the South wall is the CPA assembly fixture for assembling the CPA. 

The labor effort for the Detector Installation phase has two main components: 
The UIT (Underground Installation Team) that includes a scientific lead, manager, two deputy managers (one on each shift), safety office and administrative help. They are responsible for the communication with the ITF and Logistics Facility to insure needed components are shipped underground and properly keep in the inventory system.  They also organized/plan the day to day work done underground with the lead workers and rigging teams as well as the different consortia.  Several teams of typically of 3 FTE consisting of a lead worker and riggers that are responsible for moving all the TPC components into the SAS and cleanroom, the cold box and into the cryostat. An additional team lead worker/rigger/technician work in the cryostat positioning the TPC components and deploying the Field Cages during the final stage of each new drift volume. 

The second labor effort comes from the different consortia with specific tasks related to each group. These labor estimates will be refined during the DUNE-Ash River Trial Assembly work as we do time and motion studies. Figure \ref{fig:Single-APA-Schedule} below shows the typical labor effort in the SAS, cleanroom and cryostat. 

\begin{dunefigure}[Typical APA Installation Schedule]
{fig:Single-APA-Schedule}
{Typical APA Schedule for SP Detector}
\includegraphics[width=0.95\textwidth]{Single-APA-Schedule.pdf}
\end{dunefigure}

\begin{dunefigure}[Design of the instrumentation feedthroughs]{fig:CISC-feedthru}
{The signal feedthru are integrated with the DSS support feedtrhrus. A side ports on a short spool piece in the DSS support structure allows the instrumentation cables to be fed through the cryoatat walls where needed.}
\includegraphics[width=0.95\textwidth]{CISC-feedthru.pdf}
\end{dunefigure}



The first detector equipment to be installed are the CISC thermometers,\fixme{SG: I think "CISC thermometers" is confusing since CISC has several types of thermometers. I assume here you mean static T-gradient thermometers? If so, clarify that.} one array of purity monitors, capacitive liquid level meters,\fixme{SG: I thought the capacitive level meters came after the TPC is installed!?} calibration equipment and cameras at the east end of the cryostat. 
This equipment will be used to monitor the cool down, filling and commissioning of the detector. As these components are small the installation work can be performed using a scissor lift with 12 \si{m} reach. 
At present this is the tallest battery operated (thus cleanroom compatible) scissor lift rated for use in the US that we have identified. The signals exit the cryostat using electrical feedthroughs are distributed across the cryostat roof and are integrated with the DSS support structure as shown in Figure \ref{fig:CISC-feedthru}

Static T-gradient monitors must be installed before the outer APAs so it is planned to install them during the early installation period. The thermometer system will be supported using the bolts at the top and bottom of the cryostat. To avoid any damage, the sensors will be plugged into the IDC-4 connectors later, just before moving the corresponding APA into its final position. Individual sensors on pipes and cryostat floor are installed immediately after installing the static T-gradient monitors. Cables with their corresponding supports are installed first and sensors are installed later, just before unfolding the bottom GPs, to avoid any damage. Individual sensors on the top GP must be integrated with the GPs. For each CPA (with its corresponding four GP modules) going inside the cryostat, cable and sensor supports will be anchored to the GP threaded rods. Once the CPA is moved into its final position and its top GPs are ready to be unfolded, sensors on these GPs are installed.

Installing fixed cameras is, in principle, simple but involves a considerable number of interfaces. The enclosure of each camera has exterior threaded holes to facilitate mounting on the cryostat wall, cryogenic internal piping, or DSS. Each enclosure is attached to a gas line to maintain appropriate underpressure in the fill gas, therefore an interface with cryogenic internal piping will be necessary. Camera cables will be run through cable trays to flanges on assigned instrumentation feedthroughs. 

A summary of all the cryogenic instrumentation provided by the CISC group is shown in Figure \ref{fig:cisc_devices}. 

Quartz optical fibers required for the \dword{pd} monitoring system will be pre-installed at this point as well.  Fibers will be run form the optical flange locations (which are still being finailzed) to locations on the \dword{cpa} support structures, to be connected to the diffusers mounted on the \dwords{cpa} during installation.

The residual gas analyzers which monitor the impurities in the gaseous argon system must be installed before the piston purge and gas recirculation phases of the cryostat commissioning. The exact timing of the gas analyzers will depend on the schedule of the mezzanine outfitting and installation of the gaseous argon purge piping. The instruments are installed near the tubing switch yard to minimize tubing run length and for convenience when switching the sampling points and gas analyzers. 

\begin{dunefigure}[Distribution of various instrumentation devices inside the cryostat]{fig:cisc_devices}
  {Distribution of various instrumentation devices inside the cryostat}
  \includegraphics[width=0.98\textwidth]{cisc_distribution.png}
  \includegraphics[width=0.85\textwidth]{cisc_distribution_legend.png}
\end{dunefigure}

\begin{dunefigure}[Endwall hoisting infrastructure]{fig:endwall-hoist}
  {Image showing the hoisting equipment used to lift the Endwall into position. In this image one of the Endwalls is in place and a second is being positioned.}
\includegraphics[width=.75\textwidth]{endwall-hoist}
\end{dunefigure}

The next step in detector installation is the field cage end-wall installation. 
The end-wall panels will be brought underground in custom crates which fit in the cage. 
Each of the 8 crates will hold 4 end-wall sub-panels and 8 sub-panels are needed to build one complete 12 \si{m} tall panel.  
The first step in the end-wall installation is to install a custom hoist to the end of the DSS beam. 
This will be used to lift and assemble the sub-panels in place. 
The end-wall transport crates will then be brought to the material airlock using a forklift where they are either un--crated or the crates are cleaned. 
Once clean the crates are moved into the cleanroom and placed next to the TCO. The a hoist running on the rails through the TCO is used to lift the endwalls onto the transport cart. The cart is then hoisted into the cryostat. 
Figure \ref{fig:endwall-cart} shows an Endwall panel being transferred over to the transport cart.
The top Endwall sub-panel is then attached to the installation hoist and lifted out of the cart. When the sub-panel is free of  the cart re-position so the second sub-panel can be attached to the first and the pair are then again lifted. This process is repeated until the full 12 \si{m} Endwall field cage panel is assembles and can be attached to the DSS. 
Figure \ref{fig:endwall-hoist} shows an Endwall panel being lifted into position.
At this time all the HV connections inside the panel can be tested. The process is repeated for the four Endwall panels making up the East Endwall. 
In parallel to the endwall installation the cleanroom is configured for the installation of the APA and Cathode plane systems.

\begin{dunefigure}[Installation of the first Endwall]{fig:endwall-cart}
  {The Endwalls are lifted out of the transport crates using the one of the hosts on the installation switchyard. Each panel is then placed on a custom cart which is then lifted into the cleanroom.}
\includegraphics[width=.85\textwidth]{endwall-cart}
\end{dunefigure}


The installation of the APA and Cathode with Top/Bottom Field cage modules is the most labor intensive period of the detector installation. If possible DUNE would like to finish installing one row of the APA, CPA, and FC modules every week. This represents the equipment shown in Figure \ref{fig:install-single-row}. To achieve this several separate teams will need to be working inside the cryostat positioning the equipment, connecting the cables, and deploying the field cages, while  other separate teams will be working in the outside cleanroom assembling the equipment and performing the final cold tests. The total time that would be needed to install a row of APA and CPA if all the work were to be done serially would be many weeks so performing the work as parallel  as possible will be very important for executing the installation efficiently. 


\begin{dunefigure}[Single row of APA and CPA]{fig:install-single-row}
{One row of the APA and CPA with associated field cages is shown. In this  image the field cages are deployed in the final orientation. The equipment in the figure represents $1/25$ of the total TPC.}
 \includegraphics[width=0.5\textwidth]{install-single-row}
\end{dunefigure}

When the APAs are installed the area outside the cleanroom in the North cavern is available for storage and and it has capacity to store one full month of equipment. 
Equipment will be brought into the cleanroom's materials airlock through a rollup door in the West wall using either an electric forklift or electric pallet jacks.  
The APAs enter the airlock in the transport crates which were used to bring the APAs down the Ross shaft and into the North cavern. 
Each transport box hold two APA with associated electronics and photon detectors and they enter the airlock in a horizontal or "landscape" orientation. 
Once inside the airlock the dirty outer panels of the transport box are removed, a section of the roof is opened and the bridge crane is used to lift the transport box into vertical. 
The airlock can then be cleaned and prepared for the APA assembly step where the upper and lower APA are assembled into a 12 \si{m} tall pair.
The APA assembly sequence is shown in Figure \ref{fig:apa-assembly-v3}.

After initial visual inspection the lower APA is lifted out of the transport box 
using the bridge crane and mounted on the APA assembly tower shown in green on the top row of images in in the Figure \ref{fig:apa-assembly-v3}. 
The  lower APA is supported from the bottom and guides connected to the sides of the APA provide mechanical stability while allowing the APA to be lifted using jacks integrated into the lower support. A complete test of all the lower APA Photon Detectors (PDs) is done.  Because the cable conduit is already installed if and PDs fail we must remove the APA from the tower, place it back into the shipping frame and rotate the frame into the landscape position.  This is the only way to remove the cable conduit. The APA would then be loaded into a process cart, the problem resolved, then the reverse process is done to get the APA back onto the Assembly Tower. 

Next the upper APA is lifted out of the transport box and the transport box is removed from the airlock. The same process of testing the PDs must be done with the upper APA and replaced if needed. The upper APA is connected to the transport rail above the APA assembly tower. At this point the upper APA is supported by the trolleys used for moving the APAs along the transport rails and it is stabilized using the APA assembly frame. 
The bridge crane is no longer needed after this step so airlock roof can be closed allowing the air to be purified. At this time there is a 300 \si{mm} gap between the upper and lower APA and now the photon cables between upper and lower APA can be connected and the connection from the top connectors to the SiPMs can be checked. 
In order to mechanically connect the two APA modules a metal linkage with electrical insulators is inserted into the upper APA and bolted into place. Then the lower APA is raised until the linkage can be bolted to the lower APA.  
The APA pair can then be released from the assembly tower supports and jacks and it is supported from the top APA to the transport rail system.
After the air quality in the airlock meets the requirements for ISO-8 any dust covers are removed and the APA can be moved into the cleanroom through a narrow vertical door in the cleanroom wall shown in brown in the top row of images in Figure \ref{fig:apa-assembly-v3}.


\begin{dunefigure}[APA assembly steps]{fig:apa-assembly-v3}
  {Top row from left:  \dword{apa} transport crate inside the airlock; \dword{apa} transport crate rotating to vertical position;  lower \dword{apa} placed on the \dword{apa} assembly frame; and upper \dword{apa} placed on the assembly frame. Bottom row right to left: \dword{apa} pair moving on the cleanroom transport rails; \dword{apa} in  position at the cabling tower; and the \dword{apa} being inserted into the coldbox.}
\includegraphics[width=.9\textwidth]{apa-assembly-v3}

\end{dunefigure}

The APA pair moves onto one of the shuttle beams in the cleanroom switchyard and can be moved north-south inside the cleanroom. The next assembly step is to install and test the electronics cabling. The APA is moved to line up with one of rails used to transport the APAs to the APA cabling tower. The lower left two image in Figure \ref{fig:apa-assembly-v3} shows the APA cabling tower in the cleanroom and an APA being moved into position. The APA cabling tower is capable of holding two APA pairs simultaneously allowing one APA pair to be cabled while a second APA undergoes electrical testing in parallel. In this position the cable trays and any other equipment needed during the cabling process can be installed. The electronics cables are delivered to the cleanroom on reels pre-bundled and tested. Images of the cable assemble are found in the CE section.ref{} A lower APA reel is craned up to the top of the APA cabling tower and the cable is then spooled over to a motorized deployment spool and the cable guide is attached.  The cable guide is then fed through a guide sieve and into the conduit on the side of the APA. The cable bundle is careful fed through the conduit and is anchored in place using a cryogenic compatible cable grip. The cable connected to the electronics at the bottom and is laid into the cable trays on top. This process is repeated for the second lower APA cable bundle. Finally the upper cables can be installed and prepared for transport. At this time a check of the functionality of all the electronics is performed. After the APA electrical test the APA pair is transported to a cold box where it undergoes a thermal cycle and complete system test. The APA being inserted into the coldbox is shown in the lower right image in Figure \ref{fig:apa-assembly-v3}. As the cold box is also a Faraday cage noise levels can be measure and the photon system checked for photon sensitivity. After the cold test is complete an APA will either move back to the cabling station if any repair is needed or it will be moved into the cryostat for installation. 



\begin{dunefigure}[CPA assembly steps]{fig:install-cpa-assembly}
  {The \dword{cpa} assembly steps are shown. Top row from left:  \dword{cpa} are delivered to the CPA assembly fixture in the cleanroom, the 3 \si{m} sub-panels are lifted onto the frame and connected. Bottom Row: After the CPA panel is complete it is moved into the room and the field cage modules are attached. The CPA is then moved into the cryostat.}
\includegraphics[width=.9\textwidth]{install-cpa-assemble}
\end{dunefigure}

The CPA and top field-cage modules are assembled in parallel to the APA assembly. Figure \ref{fig:install-cpa-assembly} shows the  assembly sequence. The sub-panels for the cathode plane are delivered to the airlock in crates which hold 4 \si{m} long 1.15 \si{m} segments. The crates after cleaning are brought into the cleanroom and opened. The panels inside are bagged to provide additional dust protection. The sub-panels are lifted out of the crate and place on the assembly frame using the cleanroom switchyard hoist. First one of the 1.15 \si{m} 4 
\si{m} tall sections are assembled and then the second and third ones. The 1.15 \si{m} wide section is then lifted connected to the installation switchyard and moved to the TCO beam. The second 12 \si{m} tall section is then assembled similar to the first. The two 1.15m wide panels are then connected together to make the 2.3 \si{m} wide unit.  A complete set of QC measurements are taken of all the electrical connections between panels.  The cathode assembly  is then moved to a location in the switchyard where the diffuser fibers and top field cage modules can be installed.
 The top field cage modules are now attached. In Figure \ref{fig:install-cpa-assembly} the completed assembly is shown with the lower field cage modules also attached. This is an option but present planning is to install the lower FC modules  inside the cryostat. Finally the CPA-FC assembly is moved into the cryostat.


Photon detector monitoring system optical diffusers and short optical fibers will need to be connecetd to the \dword{cpa} panels prior to installation in the cryostat.  Discussions are underway regarding the optimal site for this installation:  Either at the \dword{cpa} assembly facility prior to shipping to the site, or as part of the assembly of \dword{cpa} stacks in the underground cleanroom.  Whichever solution is adapted, quartz optical fibers will need to be routed from the diffuser to the top of the \dword{cpa} assembly, for later connection to the pre-installed fibers in the cryostat (which will occur upon final positioning of the CPA).  

Work inside the cryostat will proceed in parallel to the work in the outer cleanroom. 
The large detector components like APA pairs and CPA modules will enter the cryostat using the TCO rails which connect to the DSS switchyard.
Inside the cryostat the modules will be pushed onto one of the switchyard shuttle beams shown in  Figure \ref{fig:shuttle}. 
The DSS shuttle beam is then moved to the appropriate row of the DSS and then the module is pushed down the length of the cryostat into position. The position of the DSS beams are well defined and accurately surveyed so the APA and CPA modules can be accurately located by precisely positioning them along the DSS beams. A small correction in the height of the modules may be needed to accommodate deflections in the DSS due to load. Figure \ref{fig:install-ce-cables} shows the typical situation during the APA installation and cold electronics cabling. 

\begin{dunefigure}[Cold Electronics cabling inside the cryostat]{fig:install-ce-cables}
  {The installation of the APA and cabling of the cold electronics is described. In the left panel the situation is shown during the APA installation process. One row of APA and HV equipment is installed and a second APA is ready for the electrical cabling. In the top right image the cable trays are seen which will hold the CE cables and one of the workers in the scissor lift. In the left bottom image thework space is defined wiht the geometry of the APA the cryostat roof and the cable feedthru.}
\includegraphics[width=.9\textwidth]{install-ce-cables}
\end{dunefigure}

After the APA is moved into position the permanent support rod is connected to the DSS beam and the trolleys are removed. 
The crawler used to push the APA along the rails is then moved back through the shuttle area and can be used for the next module. 
After the APA is locked in position the Cable try to the CE feedthru in installed now CE cabling can start. 
Even if a CPA module is already in position there is over 3 \si{m} space free between the APA and CPA so a scissor lift can easily be positioned in front to the APA. 
The two right images in Figure \ref{fig:install-ce-cables} show the situation at the top of the cryostat at during the cabling period. 
The cables are not shown so one can see the cable trays and their support infrastructure. 
At the start of the cabling all the cables are in the cable trays. 
A team of two people are in the scissor lift in front of the APA and  and a team of two people are on top of the cryostat. 
The CE cables from the bottom APA emerge from the side APA side-tube and are split into two bundles in the cable tray for a total of 4 cable bundles. 
The top APA also has the CE cables organized into 4 bundles. 
The photon cables from both the top and bottom APA are bundled into two cable bundles.
During the cabling process each bundle is partially removed from the cable tray and then fed up through the cable feedthru. 
At the top of the feedthru the cables are strain relieved and then the cables are strain relieved again at the bottom of the crossing tube.
This is repeated for each of the 10 cable bundles needed for the APA pair. 
When all the cable are installed through the cable tray any excess length is returned to the cable tray at the top of the APA. 
On the roof the short individual cables are then connected to the feedthru flange and the electronics can be tested. 
When all the electronics and electrical connections are good the flange connecting the warm interface crate and can be sealed to the cryostat feedthru flange and the cable installation is complete. Similarly the photon detector warm cables are connected to the readout module and the flange sealed after testing.  Once the testing is completed the load of the excess cable tray and cable are transferred to the DSS beam.  This minimizes an uneven load on the APA pair so they hang more vertically.   
The electronics for each APA is continuously monitored after installation. 

The placement of the cables in the cable trays and exactly how the cables are routed is complex. The working 3-D model of the cable routing showing how the cables will be bundled and placed in the trays is shown in Figure \ref{fig:install-cable-routing}. It is planned to mock up the cabling configuration at BNL and to test the installation of the cables as part of the Ash River testing program.

\begin{dunefigure}[Model of the electronics and photon detector cabling]{fig:install-cable-routing}
  {Working model of the cable trays and routing of the cables in the trays.}
\includegraphics[width=.95\textwidth]{install-cable-routing}
\end{dunefigure}


\begin{dunefigure}[CPA installation]{fig:install-cpa-fieldcage}
  {The top field-cage assemblies are deployed using a custom tool that mounts to the DSS beams as seen in the top left panel. The field-cage is lifted using the electric winch controlled by an operator in the nearby scissor lift. The lower field-cage is lowered using a hoist mounted on a wheeled frame. The hoist is on a linear slide to keep it aligned above the connection point.}
\includegraphics[width=.9\textwidth]{install-cpa-fieldcage}
\end{dunefigure}

The cathode field-cage assemblies are brought into the cryostat like the APA pairs using the overhead rails through the TCO. Inside the cryostat they move into position using the DSS switchyard and DSS I-Beams. Once in position the load is transferred to the DSS beam and the trolleys are removed. The CPA will wait in position until it's APA pairs are fully tested and then the field cage modules can be deployed. The equipment for deploying the field cages is shown in Figure \ref{fig:install-cpa-fieldcage}. The top field cages are raised by connecting a cable to the module and then a pulley-winch assembly is used to lift the module which latches to the APA mounts. A scissor lift is used to connect the cable to the module and also to control the winch. After the module is in place the deployment tool is moved to the next set of APA and CPA. The lower field cage is deployed using a custom frame that can be wheeled into position. The cable from a small host  is then attached to the field cage module and the module can be lowered. The hoist it on a linear slide so the cable is always directly over the connection point. This keeps the CPA from swinging due to an induced moment. When the module is down it latches to the APA frame similarly to the upper field cage. The electrical connection to the HV bus are tested and the deployment is complete. In principal the cathode/field-cage assemblies can be constructed faster than the APAs. It will be investigated if one should
wait to deploy the lower filed cages until after all the APA are in place. This will allow access to the APA till the end of the installation and one can also clean the cryostat floor before deploying the field cages and closing the cryostat.
Keeping the CPA installation nearly synchronous to the APA installation has the benefit that the APA can be tested after any loads or mechanical disturbances from the field-cage deployment are complete. This minimizes the work on an APA after additional APAs are installed. In the event an APA is broken an needs to be removed any APAs between it and the TCO need to also be removed. This could lead to a long delays if an APA is damaged after several additional APAs are in place. The risk and benefits of the different deployment sequences will be evaluated.

\begin{dunefigure}[Installation of row 25]{fig:install-row25}
  {Detector installation during the installation of the last row of detector components. At this time the switchyard beams are removed and the temporary hoists for the end-wall are installed.}
\includegraphics[width=.9\textwidth]{install-row25}
\end{dunefigure}


The last row of detector elements are installed similar to the previous rows but the runway beams of the DSS switchyard are removed as the APA and CPA are placed in position. Figure \ref{fig:install-row25} shows the top of the detector as the last row of APA and CPA are installed. When the shuttle beam is aligned to the correct row of APA or CPA the short section of the beam is bolted to a short I-Beam section of the runway beam that is permanently fixed to the DSS support feedthrough. When the shuttle beams at both ends of a runway beam section are fixed in position a section of the runway beam is removed. The last field-cage modules are deployed. 


\begin{dunefigure}[Second endwall installation]{fig:install-ew2}
  {Installation of the final end-wall prior to closing the TCO.}
\includegraphics[width=.9\textwidth]{install-ew2}
\end{dunefigure}

The second outside end-wall is installed like the first end-wall. Now the Center APA is rolled into the cryostat, the shuttle beam is bolted to the DSS and the runway beams are removed. The two center drift volumes field cages are deployed. The last two end walls are constructed and the TPC is basically completed. As the calibration and instrumentation are not defined yet it is not clear if these will interfere with the TCO closing. If they do they will be installed after and access will only be via scaffolding. 

At this point a frame supported by the shuttle beams is covered with flame retardant plastic is installed to create cleanroom work area for the TCO.  A scaffold is set up as egress to the manhole. The cryostat will become confined space area so it must be in place before the TCO is closed up.  The top 3/4 of TCO is completed using the 14m scissor lift. The temporary cleanroom area is removed and the area cleaned. A smaller clean area is made for the bottom 3.4m TCO section.  The scissor lift must be removed at this part. The TCO work is completed and the area cleaned.

The second scaffold is set up in the space if access is required  for cleaning of calibration and instrumentation installation.

The dynamic T-gradient monitor is installed at this time. 
The monitor comes in several segments with pre--attached sensors and cabling already in place. Each segment is fed into the flange one at a time until the entire sensor carrier rod is in place. The remainder of the system (motor system that moves the sensor rod and the sensors) that goes on the top of the flange is installed using a crane. The purity monitor system will be built in modules, so it can be assembled outside the cryostat leaving few steps to complete inside the cryostat. 
The assembly itself comes into the cryostat with the three individual purity monitors mounted to support tubes which are then mounted to the brackets inside the cryostat, and the brackets attached to the appropriate elements (cables trays, DSS and bolts in cryostat corner are under consideration). Also at this time the remaining level monitors are installed.

Once all this work is completed the scaffolding is taken apart and hoisted out the man-hole along with all the remaining flooring sections. The area is cleaned and the last two FTE in the cryostat are hoisted out. 

The inspection cameras and other possible calibration instruments can be installed from the roof while the TCO is being closed.



% clear the figure buffer before starting the next section
\clearpage












%%%%%%%%%%%%%%%%%%%%%%%%%%%%
\subsection{Prototyping and Testing (QA/QC)}
\label{sec:fdsp-tc-inst-qaqc}



\subsubsection{Introduction}


Numerous times during the installation of ProtoDUNE it was clear that the experience we gained during the trial assembly period was mechanically critical but also developing all the special tooling required to do the job.  Testing access with a fake cryostat roof or walls to understand how it is physically possible to do some tasks is important. DUNE will have 1/2 the available working space on both the top and bottom of the detector.  One of the lessons learned at ProtoDUNE this was the difficult to access this area as we secured the bottom latches and tied the APAs together on the bottom.  The process of developing hazard analysis and procedure documentation for the assembly process is an important step for approval to begin installation.   

While mechanical tests at the DUNE-Trial Assembly at Ash River is a key component of the installation process.
Other important R\&D tasks are happening at other Universities, National Labs and CERN. Argonne National Lab is testing the APA shuttle beam drive system and the CPA Assembly Tower connections before it is shipped to Ash River.  
Darnsbury as constructed the new process carts from lessons to make the assemble task both more efficient but improves the safety level of the tasks preformed. 
There are numerous small steps in the assembly process that have been improved from our past experience at CERN which will be developed during this prototyping phase. 

It is the responsibility of the different consortia to develop the QC program during the Installation Phase. 
They are listed in the following QC sections. 

\subsubsection{Ash River Testing}

Full scale mechanical testing of all the TCP components including the DSS are critical for the success of the Single Phase detector as shown from the \dword{protodune} experience. 
The NOvA Far Detector Assembly area at Ash River meets the criteria of both area and available equipment but also experienced technicians that helped construct the \dword{protodune} detector. 
By definition, the installation  is on the critical path, making it vital that the work be performed efficiently and in a manner that has low risk. 
In order to achieve this, a prototype of the installation
equipment for the \dword{spmod}  will be constructed at Ash River (the \nova neutrino experiment \dword{fd} site in Ash River, Minnesota, USA), and the installation process tested with dummy detector
elements.  
In the period just prior to the start of
installation, the Ash River setup will be used as a training ground for the installation team.
Figure~\ref{fig:NOvA-Assembly-Area}
\begin{dunefigure}[NOvA Assembly Area at Ash River]
{fig:NOvA-Assembly-Area}
{NOvA-Assembly-Area}                
%\centering     %trim=left bottom right top, clip
\includegraphics[width=0.44\textwidth]{NOvA-Assembly-Area}
\includegraphics[width=0.55\textwidth,trim=0pt 0pt 0pt 0pt,clip]
{DUNE-Trial-Assembly-Ash-River}
%\includegraphics[width=0.49\textwidth,trim=0pt 300pt 180pt]{NOvA-Assembly-Area}
\end{dunefigure}




The NOvA Far Detector Lab located at Ash River is owned and operated by the University of Minnesota via grants from the DOE and Fermilab. The Trial Assembly program for both \dword{protodune} and Dune are operated here. During the \dword{protodune} detector design many of the installation and deployment concepts where tested and modified here.  It was critical that we could attempt different concepts with full-scale mechanical components. Hazard analysis and procedure documents were developed and hands on working group meetings with the different consortia were held.  The lessons learned and training from the trial assembly helped insure that the assembly and integration for \dword{protodune} went well at CERN.

While the University has jurisdiction over the safety program at Ash River, we also follow the Fermilab Safety Program and work together in a joint effort to insure a safe working environment.  One of the key attributes of the ProtoDUNE work was compiling sets of documentation from component design, to hazard analysis, the final assembly procedures for approval by CERN HSE (Health Safety and Environment). For Ash River and DUNE this is all part of the Operational Readiness Clearance (ORC) process. Documentation for both the trail assembly process at Ash River and for DUNE will be stored on EDMS at CERN. 
Many of the TPC components are similar, access equipment and working at 14m will make construction of the DUNE Single Phase detector much more challenging.   The NOvA Far Detector Assembly area Figure~\ref{fig:DUNE-Trial-Assembly-Ash-River} has both the elevation and floor space available to do both a full-scale test of both the assembly stations and a portion of the inside of the DUNE cryostat. 

\begin{dunefigure}[DUNE - Trail Assembly at Ash River] %this gets %sniffed for table of figures
{fig:DUNE-Trial-Assembly-Ash-River}  %this is figure name and is %what the above \ref:{fig:...} matches 
{DUNE-Trial-Assembly-Ash-River}  %this is caption
\centering
\includegraphics[width=0.7\textwidth,trim=0pt 0pt 0pt 0pt,clip]
{DUNE-Trial-Assembly-Ash-River}
\end{dunefigure}


At Ash River, we also have a 75 \si{ft} $\times$  100 \si{ft} loading dock and ramp access, two 10-ton cranes, a machine shop and wide assortment of tools.  Add an experienced crew of technicians that all had multiple months at CERN building ProtoDUNE and fabricating experience it makes that a good facility for these tasks. 
The main goals for the DUNE FD Trial Assembly at Ash River are:

\begin{enumerate}
\item Test all full scale TPC (Time Projection Chamber) components during assembly stages and inside the cryostat including  
\begin{itemize}
    \item APA  Assembly including manipulation of APA shipping frames, joining an APA pair together, CE (Cold Electronics) cabling, APA protection, movement on shuttle beam, cryostat cabling and final deployment in cryostat. 
    \item Integration and installation testing of \dword{pd} (Photon Detector) components, including cable harness routing and cryoegnic cable strain relief, module integration into \dword{apa} frames, and electrical connections between upper and lower \dwords{apa}.  In addition, \dword{pd} monitoring system components mounting and optical fiber routing on the CPA will be tested.
    \item DSS (Detector Support Structure) and shuttle beam system including final detector configuration 
    \item Assembly of HV system including construction of an End Wall, CPA pairs, movement on shuttle beam and final deployment in cryostat
    \item Future Assembly of Dual Phase detector components
\end{itemize}
\item Write full set of Hazard analyses and assembly procedure documents including gathering all component documentation 
\item Test access equipment (scaffold, scissor lifts, work platforms) and lifting fixtures 
\item Assembly time and motion studies including labor estimates
This facility can also be used Future as a training site for lead workers as DUNE begins setup, testing mechanical modifications.
\end{enumerate}

\begin{dunetable}
[Summary of the tests at Ash River]
{ll}
{tab:table-label}
{The full caption that appears above the table.}
Testing phase & Test Description\\ \toprowrule
FY-19 Phase 0   &  \\ \colhline
 & Test FC Deployment and Ground Plane installation \\ \colhline
 & Check Vertical cabling with a pair of APA side tubes \\ \colhline
 & Build an APA Cabling tower for full scale APA pair assembly \\ \colhline
 & Build a CPA assembly stand and test assembly process \\ \colhline
  & Test APA Shipping Frame \\ \colhline
  FY-20 21 Phase 1 &  \\ \colhline
  & Build support structure for DSS shuttle, 3 sections of DSS beam \\ \colhline
  &  Test movement of CPA and APA from cleanroom to final destination\\ \colhline
  & Test APA, CPA, Endwall and FC deployment in one drift section \\ \colhline
  & Test assembly sequence of final section of TPC \\ \colhline
  & Removal of DSS shuttle beam runway rails \\ \colhline
  & Final deployment after TCO is closed up \\ \colhline
  FY-22 thru FY23 Phase 2&  \\ \colhline
  &  Include the top of the cryostat ( no warm structure) with \fdth \\
  \colhline
  & Test DSS installation  \\  \colhline
  &  Test CE cable installation using \fdth \\  \colhline
  & Design \fdth to support Dual Phase installation test \\ \colhline
  & Test shipping and construction using first factory TPC components  \\ \colhline
  & Train lead workers for underground at SURF \\ \colhline

\end{dunetable}

\subsubsection{DAQ QC testing}

There are several stages of DAQ installation which require  testing.  The first is the installation of the dataroom infrastructure, where cooling water leak checking, rack airflow, and power distribution will be tested upon install by the professional data center building contractors doing this installation.

The detector-to-dataroom multimode fiber will be running close to its optical power budget.  It is thus important that as this fiber is routed from the WIBs on top the cryostat to the servers in the CUC, that it be installed and tested by fiber professionals early in the process.  Covered cable trays will protect it after installation.  As APAs and servers are commissioned, pre-tested fibers will be connected to the newly installed hardware.

The DAQ servers in the CUC dataroom will be initially received and integrated offsite.  Upon installation in the CUC, only a simple functionality test will be needed.  Sufficient spare capacity will be installed, and the main commissioning work will be software related and doable over the network from the surface or remotely.

\subsubsection{APA QC testing}

Once the APAs have been installed, the only relevant test for the APA quality is the tension measurement of the wires. 
Note that testing the cold electronics will inform directly the wire continuity and the full function of the channels, therefore, only the tension remains to be tested. 
The current plan would be to re-test a sub-fraction of the 350 wires that were measured at the ITF using the laser method. 
The exact number or wires to test will depend on the final schedule for the underground testing. 
Two alternative methods for the tension measurement are also currently under development. 
One using electrical signal that would allow to measure a much larger fraction of wire tension (and potentially all of them). 
The other one would be to use the cold electronics readout to extract the tension from wire movement. 
This last method would allow to measure all the wire in a short amount of time.

\subsubsection{HV QC testing}

The EndWalls are assembled in 8  panel units, 4 on each end of the TPC.  
As each of the 8 panels are removed from the shipping crate and onto the installation cart, the EndWall Panel Checklist is filled out.\cite{bib:docdb10452}
This checklist contains a visual inspection of the frames, profiles and connections, and continuity and resistance measurements of the divider boards and their connections.  
After completion of an 8-panel EndWall in the cryostat, the Complete EndWall Checklist is filled out.  
This includes hanging position and straightness measurements, and continuity checks between panels.

The CPA panels are assembled from 3 units removed from the shipping crates.  
After each unit is removed from its bag, a visual inspection is made of the structural integrity and the connections between FSSs, HV Bus pieces and Profiles if present.  
After inspection, the unit is positioned on the CPA assembly tower.  
After the 3 units are connected on the tower, the CPA Panel Checklist is filled out.  
This includes inspection of all mechanical connections, continuity checks of the FSS, RP, Profile, and HV Bus connections, and resistance measurements of the 4 mini-resistor board connections from the RP to the FSS.  
This is repeated for the second panel in a CPA plane.  Then the 2 panels are paired, each hanging from trolleys on the transport beam.  
Visual inspection of the alignment and hanging straightness are made and HV Bus connections at the top and bottom are made and checked for continuity (CPA Plane Checklist).

The FC Top and Bottom units are removed from their crates.  The FC Unit Checklist is filled out including a visual inspection of the frames, profiles, and connections, and continuity and resistance measurements of the divider boards and their connections.  
After hanging the top FC units on the CPA plane, the 4 jumpers from the first FC Profile on each side of the CPA and the CPA FSS is connected and the resistance is measured, completing the CPA/FC Top Assembly Checklist. 
The FC Bottom units are not attached to the CPA, but are taken into the cryostat independently after filling out the FC Unit Checklist described above.

The CPA/FC Top assembly is moved into its position in the cryostat.   
After deployment of the FC Top units, visual inspection of the resistor board/jumper between the FC and the CPA FSS is made.  
Also, visual inspection of the latch at the FC/APA is done.  After deployment of the FC Bottoms, visual inspection is made to verify the resistor board/jumper connection from the FC to the CPA.  
Also, visual inspection of the latch connecting the FC Bottom to the APA is made.  
These are included in the CPA/FC Cryostat Checklist.

\subsubsection{CE QC testing}
Details of the CE QC testing are found in the CE chapter.

\subsubsection{CISC QC testing}

Cryogenics instrumentation systems must undergo a series of tests to guaranty they will perform as expected:  

\begin{itemize}
\item {\bf Purity Monitors}: Each of the fully assembled purity monitors arrays is placed in its shipping tube, which serves as a vacuum chamber to test all electric and optical connections at \surf before the system is inserted into the cryostat. During insertion, electrical connections are tested continuously with multimeters and electrometers.

\item {\bf Static T-gradient Thermometers}: Right after the installation of each of the sensor's arrays, its verticality 
is checked, and the tensions in the stainless steel strings adjusted as necessary. Once cables are routed to the corresponding DSS ports the entire readout chain is tested. This allows a test of the sensor, the sensor-connector assembly, the cable-connector assemblies at both ends, and the noise level inside the cryostat.
If any sensor presents a problem, it is replaced. If the problem persists, the cable is checked and replaced as needed.

\item {\bf Dynamic T-gradient Thermometers}: The full system is tested after it is installed in the cryostat. Two aspects are particularly important, the vertical motion of the system using the step motor, which is controlled via the slow controls system; and the full readout chain which will be tested mainly to detect failures in sensors, cables and connectors inside the cryostat. 

\item {\bf Individual Sensors}: To address the quality of individual precision sensors, the same method as for
the static T-gradient monitors is used. For standard RTDs to be installed on the cryostat walls, floor, and roof, calibration is not an issue. Any QC required for associated cables and connectors is performed following the same procedure as for precision sensors.

\item {\bf Gas Analyzers}. Once the gas analyzer modules are installed at \surf and before commissioning the cryostat, they 
are checked for both \textit{zero} and the \textit{span} values using a gas-mixing instrument and two gas cylinders, one having a
zero level of the gas analyzer contaminant species and the other cylinder with a known percentage of the contaminant gas. This
 verifies the proper operation of the gas analyzers. 

\item {\bf Liquid Level Monitoring.} Once installed in the four cryostat corners, the capacitive level meters are tested in situ 
using a suitable dielectric in contact with the sensors.

\item {\bf Cameras}. After installation and connection of wiring, fixed cameras, movable inspection cameras and light-emitting system are checked for operation at room temperature. Good quality images of all cryostat and detector areas considered on the system's design should be obtained. The movable system for inspection cameras should behave as expected.  
\end{itemize}

\subsubsection{Photon QC testing}

Access to the photon detectors inside the \dwords{apa} is severely limited once the \dword{ce} cable conduit is in place, increasing the need to identify problems early in the process to minimize the schedule impact from required PD maintenance or repair due to problems detected during installation.

Upon receipt of integrated \dwords{apa} in the underground prep area (\dword{sas}) for the clean room in front of the cryostat, an electrical connectivity test between the cable end exiting the \dword{apa} to be installed and each of the PD modules in that \dword{apa} will be performed.  Discussions are underway with the \dword{apa} consortium to determine if the transport frames for shipping the integrated \dword{apa} pairs can be made sufficiently light-tight to allow for biasing the photosensors and checking that they are operating properly.  As accessing the \dword{pd} modules involves removing the \dword{ce} cable conduits from the \dword{apa} side tubes, this is the last step in the installation process where PDs can be reached for repair or replacement.  Any repairs to the \dword{pd} system discovered later in the installation process would require returning the \dword{apa} module to this point in the installation sequence.

As the upper and lower \dwords{apa} are joined on the assembly tower, photon detector cables from the upper to the lower \dword{apa} will be connected, and at that time continuity checks will be made.

Following joining of the upper and lower \dwords{apa}, the assembled unit will be moved into a cold box in front of the cryostat for final testing.  This will represent an opportunity to make a final low-temperature checkout of the complete \dword{pd} cold electronics and cabling chain prior to installation into the cryostat.  \dword{pd} FE electronics boards will be used to read out the photon system during the cold test, and results compared to previous \dword{qa} test results.

The \dword{apa} stack is rolled into position in the cryostat following the cold box test, and the \dword{pd} and \dword{ce} cables connected to the cryostat flange.  At this point, a final continuity check is made from the flange bulkhead to the \dword{pd} module.

Discussions are underway with the installation team to arrange for a one-shift dark test of the installed photon detectors in the cryostat following final installation, verifying end-to-end system operation.



%%%%%%%%%%%%%%%%%%%%%%%%%%%%
\subsection{Environmental, Safety, and Health (ESH)}
\label{sec:fdsp-tc-inst-safety}

Fermilab and DUNE are committed to supporting its research and operations by protecting the health and safety of staff, the community and the environment, as stated in the LBNF/DUNE Integrated Safety Management Plan. The Safety and Health Program is in compliance with applicable standards and Local, State and Federal legal requirements through Fermilab's Work Smart Set of Standards and the contract between Fermilab Research Alliance and the Department of Energy. Fermilab and the South Dakota Services Division have the host laboratory responsibilities for LBNF/DUNE operations at the Sanford Underground Research Facility in Lead, South Dakota.
The Fermilab facilities are further subject to the requirements of the Department of Energy (DOE) Workers Safety and Health Program 10 CFR 851. These requirements are promulgated through the Fermilab Directors Policy Manual, and the Fermilab Environment, Safety, and Health Manual (FESHM) which align with the SURF ESH Manual.

While ESH will be  a host lab (Fermilab/SDSD) responsibility, there is a  Global Safety Coordination group in place to evaluate applicable codes and standards including international code equivalency for the design, assembly, and installation of the DUNE detectors. The Global Safety Coordination group is  a team of engineering and ESH experts from within LBNF and DUNE organizations.  These requirements will be adopted by DUNE, JPO (Joint Project Office) and LBNF organizations and be used to develop manufacturing, assembly and installation processes and procedures 

 An Installation Environment, Safety and Health Plan will be developed which will define a specific set of ESH requirements and responsibilities which personnel will be required to follow to perform assembly, installation, and construction activities at the Sanford Underground Research Facility (SURF) and the Integrated Test Facility (ITF). 

{\bf Work Planning and Controls:} The goal of the work planning and hazard analysis (HA) process is to initiate thought about the hazards associated with work activities and how it can be performed safely. Careful planning of a job assures that it is performed efficiently and safely. Work planning ensures the scope of the job is understood, appropriate materials and tools are available, all hazards have been identified, mitigation efforts established, and all affected employees understand what is expected of them. Hazard analysis is a critical part of work planning.  The Work Planning and Hazard Analysis program is documented in Chapters 2060 in the FESHM.

Daily work planning meetings will be led by the shift supervisor at the start of each shift to coordinate work activities, review hazards/mitigations, and answer any questions

{\bf Documentation Approval Process:} There will be an engineering review and approval process of all required documentation including structural calculations, assembly drawings, load tests, hazard analyses, and procedural documents for typical individual tasks.  For the larger operations and systems like TPC component factories, DSS, cleanroom and assembly infrastructure this will be followed by an Operational Readiness Review. This is done by a joint safety committee that visits the sites after reviewing the documentation and watches the full operation before it is signed off on.

{\bf ESH Support:} Safety starts from the ground up both on the surface and in the underground facilities. All employees have work stop authority in support of  a safe working environment. Personnel will need to utilize the necessary PPE identified through the hazard analysis process for the task. A local ESH coordinator under the direction of the DUNE ESH Manager will provide daily support of the facilities and attend the daily work planning meetings to notify each shift of potential safety issues/constraints, to validate that  employees have the necessary ESH training, and is responsible for managing ESH related  documentation including training records, weekly safety reports, near miss/accident reports and equipment inspection.

{\bf Work Underground SDSTA will maintain:}
\begin{itemize}
    \item Site Access Control Program through Trip Action Plans (TAP)
    \item Emergency Management (EM) program which includes an Emergency Response incident command system and  Emergency Rescue Team (ERT).  Typically, a small number of employees underground (Guides) are also training first-responders to help in a medical emergency.
    
\end{itemize}

{\bf Equipment Operation:} All overhead cranes, gantry cranes, fork lifts, motorized equipment trains/carts will only be operated by trained licensed operators.. 
Other equipment like scissor lifts, pallet jacks, hand tools and shop equipment may only be operated by people trained
and certified for that particular piece of equipment.

{\bf House Cleaning, Training, Lab Access and PPE:}It is the responsibility of all workers underground to keep a clean organized work area. Limited storage of flammable items must be in proper storage cabinets, items like shipping crates and boxes must be removed and shipped back to the surface. There is very limited space underground so it is critical especially for large objects that must be slung loaded like shipping crates that hoist trips are arranged for these empty items to go up after new items come down. PPE is the responsibility of the host lab (Fermilab/SDSD) to be supplied for all workers.. All workers will be required to complete both SURF Surface and Underground Orientation classes. In addition, workers accessing the underground will be required to complete 4850 and 4910 level specific unescorted access training. A guide will be required to be stationed on all working levels and received the necessary guide training. .  e All workers requiring access to the SURF site will be required to register through the Fermilab User Office to received the necessary User training and a Fermilab ID number. In addition, all workers will be required to apply for a SURF identification badge for DUNE activities.

{\bf Requirements at Collaborating Laboratories and Institutions:}All work performed at collaborating institutions will be completed in accordance with the collaborating institutions Environmental, Safety, and Health policies and program. 
Equipment and operating procedures provided by the collaborating institution will conform to the DUNE Project ES\&H and Integrated Safety Management policies and procedures. 
The collaborating institutions ES\&H Department shall be responsibility for providing ES\&H oversight for all work activities carried out in collaborating institution facilities. 
LBNF/DUNE personnel will also follow the ES\&H Manual and procedures of the collaborative institutions.


%%%%%%%%%%%%%%%%%%%%%%%%%%%%
\subsection{Costs, Schedule and Risk Analysis}
\label{sec:fdsp-tc-inst-cost}

\fixme{Add Cost section when available}


The schedule for the Single Phase Detector 1 Installation is a complicated dance relying on numerous other entities including CF, LBNF, and SDSTA.  There is a maximum allowable limit for the number of people underground of 140 dure to the size of the mine rescue area.  This is particularly critical during the excavation of Cavern 3. As different area underground areas will give us access at different times. A detailed summary of these time periods are shown in Figure~\ref{fig:Overview_of_SinglePhase_Schedule}



\begin{dunefigure}[Overview of the single-phase schedule]
{fig:Overview_of_SinglePhase_Schedule}
{Schedule Overview of the Single Phase Detector \#1}                
\includegraphics[width=0.98\textwidth]
{Overview_of_SinglePhase_Schedule}
\end{dunefigure}

The cost, schedule and labor estimates are based on two 10 hour shifts per day, 4 days per week. It is assume that our work efficiency is a maximum of 70 percent.  Approximately 2-3 hours per day is taken up by the cage ride underground, the start of shift meeting,lunch and coffee breaks and gowning to go into the cleanroom. 

There are 3 basic schedule phases for Detector 1:

\begin{itemize}
    \item {\bf CUC Installation Phase:}
    This time period described in detail in section 1.4.2.1 can start once Acceptance for Use and Possession (AUP) has been recieved for the North Cavern and the CUC. This is also the same time that the excavation of the South Cavern and installation of the Warm structure by LBNF begins. Maximum number of FTEs will be limited to 140 people and access for hauling equipment in the shaft will minimal during this time period.  It is estimated that it will take 3 months to install the basic rack infrastructure then over a 12 month period of continued installation of DAQ equipment sections of the system will be operational. The operation of the cold boxes in the cleanroom and the testing of the readout once and APA has been installed in the cryostat are needed at the beginning of installation phase. During this time period the labor underground is minimal 5-10 FTE from the DAQ consortia and contractors plus the core UIT team as it ramps up in size. 
    
    \item {\bf Installation Setup Phase:} This phase is defined in detail in section1.3 Infrastructure. During this time period we begin working two shifts per day and the UIT team doubles in size.  This is a critical training period, getting the lead workers, riggers and equipment operators familiar with the tasks and adjusting crews to insure balanced teams.  Prior to this phase the DUNE Trial Assembly equipment at Ash River will be used as a training site begin the training process.
    
    \item {\bf Detector Installation Phase:} The final detector installation phase begins with an Operational Readiness Review to check that all documentation and procedures are in place. After the east end walls are installed there is a start-up period 1.5 months for the first two rows of TPC components. After that the installation rate is 9-10 shifts to complete each row.  To achieve this schedule we use two APA cabling stations 3 cold boxes and separate crews in the cryostat all working in parallel.  It takes 5.5 months to install rows 3-24 and about 1 month for row 25. There is approximately a 2 month period to close the TCO by the cryostat cold structure contractor. During this time there is no access to the cryostat.  Once this is completed the final instrumentation is completed and purge can begin. 
    
\end{itemize}

\begin{dunetable}
[Installation System Risk Summary]
{p{0.05\textwidth}p{0.4\textwidth}p{0.4\textwidth}}
{tab:INSTALL-risks}
{Installation Risk Summary}   
ID & Risk  & Mitigation\\ \toprowrule

1 & 
Work underground, work at heights, handling heavy equipment, tests with \dword{hv}, laser operation, and  other hazards can lead to personal injury.
& Follow FNAL safety program: procedures for work, training, use of \dword{ppe}, with levels of oversight. Use all reasonable measures to prevent workplace incidents. Still consider residual risk as critical.\\ \colhline
2 & 
Shipping delays can lead to installation delays. &
Plan one month buffer to store  materials locally.\\ \colhline
3 & 
Unavailability of components can delay assembly and installation. &
Use a detailed inventory system to verify availability of  necessary components prior to transport underground.\\ \colhline
4& 
Unavailability of international contract labor or equipment  can delay or prevent necessary work. &
Dedicated \dword{fnal} division in SD will expedite  import/export and visa related issues.\\ \colhline
5&
Limited availability of local trained workers can delay or prevent necessary work.
& Begin hiring process early; allow time for substantial training. Use Ash River installation prototype to train crew and optimize procedures.\\ \colhline
6&
Mismatch in dimensions or run-out in tolerances can lead to mechanical interferences. & 
Generate integration model of full \dword{detmodule}. Include key dimensions in integration drawings to control  interfaces.  Prototype assembly steps. Perform acceptance tests.\\ \colhline
7&
Human error in transport of heavy objects and work at heights with tools can lead to cryostat damage. & 
Construct false floor inside the cryostat for use during installation. Remove it as late as possible in process. Install temporary protection for work after its removal.\\ \colhline
8& 
Several times per year \dword{surf} closes for 1-2 days due to snow; closures can cause delays. & Include the average number of snow days in schedule.\\ \colhline
9&
Installation and cooling of components produces stress that can cause equipment failures.&
Test all components prior to delivery to SD to eliminate infant mortality failures. Cold test \dword{apa} assemblies once cabling and assembly is complete. Read out detector continuously to detect failures.\\ 
\end{dunetable}


\subsection{Detector Commissioning Phase}
\label{sec:fdsp-tc-inst-comiss}
After the \dwords{detmodule} is installed in the cryostat there remains a lot of work before it can be operated. 
First the \dword{tco} must be closed. 
This requires bringing back the cryostat manufacturer. 
First the missing panel with the steel beams and steel panel are installed to complete the cryostat's outer structural hull. 
Then the remaining foam blocks and membrane panels
are installed from the inside using the roof access holes 
to enter the cryostat. 

In parallel, the \lar pumps are installed at the ends of the cryostat and final connections are made to the recirculation plant. 
Once the pumps are installed, the cryostat is closed, and everything is leak tested, the cryogenics plant can be brought into operation. 
First the air inside the cryostat is purged by injecting pure argon gas at the bottom  at a rate such that the cryostat volume is filled uniformly but faster than the diffusion
rate. 
This produces a column of argon gas that rises through the volume and sweeps out the air. 
This process is referred to as the \textit{piston purge}. 
When the piston purge is complete the cool-down of the \dword{detmodule} can begin. 
Misting nozzles inject a liquid-gas mix into the cryostat
that cools the detector components at a controlled rate. 

Once the detector is cold the filling process can begin. 
Liquid argon stored at the surface  at \surf is vaporized and brought down the shaft in gaseous form and is then re-condensed underground. 
The \lar then flows through filters to remove any H$_2$O or O$_2$ and flows into the cryostat.
Given the very large volume of the cryostat and the limited cooling power for re-condensing, it is  expected to take \num{12} months to fill the first \dword{detmodule}. 
%During this time the detector readout electronics will be on monitoring the status of the detector. 

Several tests and a constant monitoring of the detector will be taking place from \dword{tco} closing till the end of the filling process.
Before the last manhole is sealed:
\begin{enumerate}

    \item A pedestal and RMS characterization of all cold electronic channels will be performed, to verify all \dword{apa} front-end boards are responding, and no dead channel or new noise sources arose following the \dword{tco} closing
    
    \item A noise scan of all \dword{pd} channel is performed as last check before sealing

    \item Each \dword{apa} wireplane will be checked to verify it is isolated from the \dword{apa} frame and properly connected to its HV power supply through the following steps
    
\begin{itemize}

    \item The SHV connector of each wireplane bias channel will be unplugged at the power supply and the resistance between inner conductor and ground and its capacitance will be measured. The resistance should show the wireplane electrically isolated from the ground, while the capacitance value should match that of the cold HV cable plus the capacitance of the circuit on the \dword{apa} top frame

    \item 50 V are applied to each wireplane and the current drawn checked against the expected value
    
    \item Nominal voltages are applied to each wireplane and the current drawn checked against the expected value 
    
\end{itemize}

    \item Apply a low (i.e 1-2 kiloVolts) to the cathode and measure the current drawn to ensure the integrity of the HV line

\end{enumerate}

During the \textit{piston purge} phase, periodic monitoring of both APA cold electronics and \dword{pd} system noise (pedestal, RMS) will occurr.

A number of the above mentioned tests, in addition to new ones, will instead take place during the cool-down phase:

\begin{enumerate}


    \item Each \dword{apa} wireplane isolation and proper connection to its HV power supply will be checked at regular time intervals in the same way as done before sealing the cryostat
    
    \item 1-2 kiloVolts will be hold on the cathode, and the current drawn constantly monitored to observe the behaviour with temperature of the total resistance
    
    \item Cold electronics noise figures (pedestal, RMS) will be measured at regular intervals and its trend with temperature recorded
    
    \item \dword{pd} system noise (pedestal, RMS) will be measured at regular intervals and its trend with temperature recorded
    
     \item Values of the temperature sensors deployed in several parts of the cryostat will be constantly monitored to understand the progress of the cool-down phase and related to the behaviour of the other \dword{detmodule} subsystems 
     
\end{enumerate}

Regular monitoring of CE and \dword{pd} noise, as well as check of wireplane isolation and proper connection to bias supply system will continue throughout the filling, recording the noise variations as a function of the progressively lowering temperature. In addition:

\begin{enumerate}

    \item as each purity monitor is submerged in liquid, turn it on every 8 hours to control LAr purity
    
    \item as soon as top ground planes are submerged, raise HV on the cathode up to 10 kiloVolts and check the current drawn by the system agrees with expectations.

\end{enumerate}

Once the \dword{detmodule} is full, the drift high voltage will be carefully ramp up following these steps:

\begin{enumerate}

    \item Need for a filter regeneration is evaluated before start of any operation

    \item Once filter regeneration is completed (if needed), LAr surface is looked at with cameras to verify whether it is flat or any bubble/turbolence can be spotted
    
    \item LAr recirculation is started and LAr surface is looked at again to verify whether activation of the recirculation system introduced any turbolence in the liquid.
    
    \item wait one day since begin of recirculation to stabilize the LAr flow inside the \dword{detmodule}, then start the HV ramp.
    
\end{enumerate}

Cathode voltage is raised in steps throughout the course of three days. 
On the first day, cathode voltage is raised first to 60 kV, then to 90 kV after a 2-hours waiting period, and finally to 120 kV after another 2-hours waiting period, and left at this value overnight.
On the second day, cathode voltage is raised first to 140kV, then to 160 V after 4 hours waiting period, and left at this value overnight.
On the third day, cathode voltage is raised to 170kV first, and then to the nominal operating voltage of 180 kV after a 4 hour waiting period.
During each HV ramp, all CE current draws are monitored and the procedure is stopped if any of the current draws goes out of the allowed range.
During each waiting period, regular DAQ runs are taken to monitor CE and \dword{pd} noise and response, while cathode HV and current draw stability are constantly monitored.

In ProtDUNE-SP this process took three days and the system is ready for data-taking. With a detector twenty times larger it will take longer but the turn-on time is still expected to be relatively short. 

\subsection{Conclusions}
\label{sec:fdsp-tc-inst-concl}

That's all folks
The End
\fixme{contribution from editors}

\section{Cost, Schedule and Risks}


%%%%%%%%%%%%%%%%%%%%%%%%%%%%
%\subsection{Logistics Cost and Schedule} %, and Risk Analysis}
%\label{sec:fdsp-tc-log-cost}

The costs for the logistics activities and facilities are the responsibility of \dword{lbnf} and Fermilab's \dword{sdsd}, and therefore are not listed here. %  is not a DUNE responsibility and will be covered as part of the host laboratory responsibilities.

According to the overall \dword{lbnf} and \dword{dune} schedule, the \dword{aup} for \dword{lbnf} (post-\dword{cf}) and \dword{dune} teams to the north cavern and \dword{cuc} is \cucbenocc{}.

The  \dword{sdwf} will be in place approximately one year before the warm structure installation begins, i.e., in fall 2021.  Extra storage must be available before  this facility is ready  since  \dword{apa}s, \dword{cuc} infrastructure, and equipment all begin arriving in South Dakota during summer 2021.  As an interim solution, we plan to store these early items at \dword{fnal} until the \dword{sdwf} is ready.

Figure \ref{fig:high-level-schedule} shows the overview of the schedule for the main activities for \dword{detmodule} \#1. 


%%%%%%%%%%%%%%%%%%%%%%%%%%%%
\subsection{FROM ITF FILE: Cost, Schedule and Risk Analysis}
\label{sec:fdsp-tc-itf-cost}

\fixme{Add costs when they are available (Anne adds 3/25: Cost table template coming soon; do not add any numbers yet.)}

%%%%%%%%%%%%%%%
\subsection{FROM ITF FILE: Timeline and APA Integration Schedule}

\fixme{Anne adds new standard schedule table 3/25, table~\ref{tab:sp-iic-sched}. Leave orange (dunepeach) lines as they are, add your milestones around them.}

%This is a standard table template for the TDR schedules.  It contains overall FD dates from Eric James as of March 2019 (orange) that are held in macros in the common/defs.tex file so that the TDR team can change them if needed. Please do not edit these lines! Please add your milestone dates to fit in with the overall FD schedule. 


\begin{dunetable}
[\dword{sp} Installation, Integration, and Commissioning Schedule]
{p{0.65\textwidth}p{0.25\textwidth}}
{tab:sp-iic-sched}
{\dword{sp} Installation, Integration, and Commissioning Schedule}   
Milestone & Date (Month YYYY)   \\ \toprowrule
Technology Decision Dates &      \\ \colhline
Final Design Review Dates &      \\ \colhline
Start of module 0 component production for ProtoDUNE-II &      \\ \colhline
End of module 0 component production for ProtoDUNE-II &      \\ \colhline
\rowcolor{dunepeach} Start of \dword{pdsp}-II installation& \startpduneiispinstall      \\ \colhline
\rowcolor{dunepeach} Start of \dword{pddp}-II installation& \startpduneiidpinstall      \\ \colhline
 \dword{prr} dates &      \\ \colhline
Start of  (component 1) production  &      \\ \colhline
Start of (component 2) production  &      \\ \colhline
Start of  (component 3) production  &      \\ \colhline
\rowcolor{dunepeach}South Dakota Logistics Warehouse available& \sdlwavailable      \\ \colhline
\rowcolor{dunepeach}Beneficial occupancy of cavern 1 and \dword{cuc}& \cucbenocc      \\ \colhline
\rowcolor{dunepeach} \dword{cuc} counting room accessible& \accesscuccountrm      \\ \colhline
\rowcolor{dunepeach}Top of \dword{detmodule} \#1 cryostat accessible& \accesstopfirstcryo      \\ \colhline
End of  (component 1) production  &      \\ \colhline
... & ...                       \\ \colhline
\rowcolor{dunepeach}Start of \dword{detmodule} \#1 TPC installation& \startfirsttpcinstall      \\ \colhline
\rowcolor{dunepeach}End of \dword{detmodule} \#1 TPC installation& \firsttpcinstallend      \\ \colhline
\rowcolor{dunepeach}Top of \dword{detmodule} \#2 accessible& \accesstopsecondcryo      \\ \colhline
 \rowcolor{dunepeach}Start of \dword{detmodule} \#2 TPC installation& \startsecondtpcinstall      \\ \colhline
\rowcolor{dunepeach}End of \dword{detmodule} \#2 TPC installation& \secondtpcinstallend      \\ \colhline

last item & ...                         \\
\end{dunetable}


%The timeline for starting up the \dword{itf} is late 2022 when \dwords{pd} and \dword{ce} are available to integrate into the \dword{apa}s.  This is two years before installation begins. This schedule has the advantage of allowing all integrated components to be tested as soon as possible and minimizing any schedule risk if problems occur. However $3/4$ of the \dword{apa} are produced before the first \dword{ce} module is available for the risk reduction applies primarily to the electronics. The integration team size is smaller than what would be required underground as the work proceeds over a longer time. 
The integration area underground must be ready to start operations when the \dwords{pd} and \dword{ce}  become available in late 2022; this is after three quarters of the \dword{apa}s are produced and two years before installation begins.  This schedule allows the \dword{ce} to be tested and integrated soon after production, thus minimizing the schedule risk in the case of unanticpated problems. This risk reduction applies primarily to the electronics. %The integration team size is smaller than what would be required underground as the work proceeds over a longer time. 

%Using four work stations and working with only a day shift, installing  an \dword{apa} pair should take approximately nine shifts as shown in Figure \ref{fig:ITF-Schedule}. The 150 \dword{apa} then require 270 work days.
Integrating two \dword{apa}s (to make a top-bottom pair) %that will be shipped together) 
is expected to take approximately nine shifts as shown in Figure \ref{fig:ITF-Schedule}. Using four work stations and working only day shifts, the 150 \dword{apa} pairs then require 270 work days.  

\begin{dunefigure}[Schedule for APA integration]
{fig:ITF-Schedule}
    {Schedule for APA integration}
\includegraphics[width=0.98\textwidth]
{ITF-Schedule.pdf} 
\end{dunefigure}


\fixme{A graphical representation of the schedule is used instead of the template as the duration and relative timing of the work is critical for the installation. Jim}

%%%%%%%%%%%%%%%%%%%%%%%%%%%%
\subsection{FROM INFRASTRUCTURE FILE Costs, Schedule, and Risk Analysis}
\label{sec:fdsp-tc-infr-cost}

\fixme{Add costs when available -- (anne) probably want only one cost and risk table in chapter; not sure about schedule since you have special gantt charts}

{\bf Installation Setup Phase:} %This is a more difficult phase to schedule and may be adjusted often with multiple projects going forward at once. The \dword{cf} work is basically completed, which reduces the number of \dwords{fte} underground, allowing us to begin the installation infrastructure. We begin two 10 hour shifts per day as the work ramps up.  Once the cold cryostat is approximately six months into the installation schedule, floor space becomes available in the north cavern. The \coldbox construction must be started ASAP because the welding takes approximately six months. In parallel to this, the machine shop area can be set up and the bridge between the north and south sides of the cavern can be constructed.  Once the bridge is completed, work on the assembly crane, \dword{apa} cabling tower, \dword{apa} assembly tower, and \dword{cpa} assembly station can begin. 
    This is a more difficult phase to schedule and may require frequent adjustment, with multiple projects going forward at once. The \dword{cf} work is essentially completed by this point, which reduces the number of \dwords{fte} underground, allowing us to begin the installation infrastructure. 
    \fixme{Sort of a non sequitur. Reduced cf FTEs allows you to have more FTEs, but it's not clear that it's what allows you to begin the installation infra } We begin two ten-hour shifts per day as the work ramps up.  Once the  cryostat cold structure is approximately six months into the installation schedule, floor space becomes available in the north cavern. The \coldbox construction must begin immediately at this point because the welding takes approximately six months. In parallel, the machine shop area can be set up and the bridge between the north and south sides of the cavern can be constructed.  Once the bridge is completed, work on the assembly crane, \dword{apa} cabling tower, \dword{apa} assembly tower, and \dword{cpa} assembly station can begin. 
    
%Installation of the detector support system could begin during the final installation stages of the cryostat cold structure because they both require full height scaffolding for all the welding on the top of the cryostat. This was how the \dword{protodune} \dword{dss} was installed. The details have not yet been worked out with the contractor, so work may be done in stages. This requires a crew on top of the cryostat installing the \dword{dss} support feedhroughs from the top of the cryostat as shown in Figure \ref{fig:install-dss-feedthru}. 
Installation of the \dword{dss} could begin during the final installation stages of the cryostat cold structure because they both require full-height scaffolding for the welding on the top of the cryostat. The \dword{protodune} \dword{dss} was installed this way. This requires a crew on top of the cryostat installing the \dword{dss} support feedhroughs from the top, as shown in Figure~\ref{fig:install-dss-feedthru}.  The details have not yet been worked out with the contractor, so work may be done in stages. 

A detailed schedule of the installation infrastructure time period is shown in Figure~\ref{fig:Installation_Infrastructure_Schedule}.
    
\begin{dunefigure}[Schedule for the infrastructure installation]
{fig:Installation_Infrastructure_Schedule}
    {Schedule for the infrastructure installation}
\includegraphics[width=0.98\textwidth]
{Installation_Infrastructure_Schedule} 
\end{dunefigure}

\fixme{A graphical representation of the schedule is used since it is critical to convey the time sequencing and how the work is overlapping. Jim }

% clear the figure buffer before starting the next section
\clearpage

%%%%%%%%%%%%%%%%%%%%%%%%

%%%%%%%%%%%%%%%%%%%%%%%%%%%%
\subsection{Costs and Schedule}
\label{sec:fdsp-tc-inst-cost}

\fixme{Jim S to rework this section}

\fixme{I just added the template for the Cost table. Fill in 1st column with the cost items but leave the cost and hours columns empty for now.  Risks to be treated like specs -- still to come. (Anne) }

\begin{dunetable}
[Cost Summary]
{p{0.5\textwidth}p{0.2\textwidth}p{0.2\textwidth}}
{tab:XXcostsumm}
{Cost Summary}   
Cost Item & M\&S (k\$ US) & Labor Hours \\ \toprowrule
\rowcolor{dunepeach} Design, Engineering and R\&D &  &     \\ \colhline
 (E.g., Photosensors design) &     &             \\ \colhline
 (E.g., Mechanics design) &     &             \\ \colhline
 &     &             \\ \colhline
 &     &             \\ \colhline
 &     &             \\ \colhline
 &     &             \\ \colhline
\rowcolor{dunepeach} Production Setup &  &     \\ \colhline
 (E.g., Photosensors production setup)  &     &             \\ \colhline
 &     &             \\ \colhline 
 &     &             \\ \colhline
 &     &             \\ \colhline 
 &     &             \\ \colhline
 &     &             \\ \colhline
\rowcolor{dunepeach} Production &  &     \\ \colhline
 (E.g., Photosensors production)  &     &             \\ \colhline
 &     &             \\ \colhline 
 &     &             \\ \colhline
 &     &             \\ \colhline 
 &     &             \\ \colhline
 &     &             \\ \colhline
\rowcolor{dunepeach} DUNE FD Integration \& Installation  &  &     \\ \colhline
 &     &             \\ \colhline
 &     &             \\ \colhline 
 &     &             \\ \colhline
 &     &             \\ \colhline 
 &     &             \\ \colhline
 (last line) &     &             \\
\end{dunetable}


The schedule for the \dword{spmod} 1 Installation is a complicated dance relying on many entities including \dword{cf}, \dword{lbnf}, and \dword{sdsta}.  The maximum number of people allowed underground is 144 which is the number of people that can be evacuated in one hour.  This is particularly critical during the excavation of Cavern 3. Different underground areas will allow access at different times, so a detailed summary of these times are shown in Figure~\ref{fig:Overview-of-SinglePhase-Schedule}. 



\begin{dunefigure}[Overview of the single-phase schedule]
{fig:Overview-of-SinglePhase-Schedule}
{Schedule Overview of the Single Phase Detector \#1}                
\includegraphics[width=0.98\textwidth]
{Overview-of-SinglePhase-Schedule}
\end{dunefigure}

The cost, schedule, and labor estimates are based on two 10 hour shifts per day, 4 days a week. Work efficiency should be a maximum of 70 percent.  The cage ride takes up approximately 2-3 hours per day, and shift meetings, lunch, coffee breaks, and gowning to go into the cleanroom take more time out of the shift. 

There are three basic schedule phases for Detector 1 installation:

\begin{itemize}
    \item {\bf \dword{cuc} Installation Phase:}
    This period, which is described in detail in section \ref{sec:fdsp-tc-inst-CUC}, can start once \dword{aup} has been received for the north cavern and the \dword{cuc}. This is also the same time that excavation of the south cavern and installation of the warm structure by \dword{lbnf} begins. As no more than 140 \dwords{fte} will be allowed underground at a time,  access to the underground area will be minimal during this period for \dword{dune} personnel.  Work in this period is limited to work inside the \dword{cuc}. Installing the basic rack infrastructure in the dataroom will take an estimated three months, and then installation and testing of the \dword{daq} will continue over the next 12 month period. The \dword{daq} is needed at the start of detector installation.  
    
    \item {\bf Installation Setup Phase:} This phase,  described in detail in section \ref{sec:fdsp-tc-inst-setup}, is when the majority of the infrastructure is installed. During this time, we begin working two shifts per day, and the \dword{uit} team doubles in size.  
    This is a critical training period, so getting lead-workers, riggers, and equipment operators familiar with the tasks is a priority, and adjusting crews to ensure balanced teams.  Before this phase, the \dword{dune} trial assembly equipment at Ash River will be used to begin the training process. During this phase the bridge is constructed along with the coldboxes, the cleanroom and all the equipment in the cleanroom. The \dword{dss} is also installed and surveyed.
    
    \item {\bf Detector Installation Phase:} The final detector installation phase begins with an operational readiness review to check that all documentation and procedures are in place. After the east endwalls are installed, a start-up period of 1.5 months begins for the first two rows of \dword{tpc} components. After that, the installation rate is 9-10 shifts to complete each row.  To meet this schedule, we use two \dword{apa} cabling stations, three coldboxes, and separate crews in the cryostat, all working in parallel.  It will take 5.5 months to install rows 3-24 and about 1 month for row 25. Closing the \dword{tco} will take approximately two months for the cryostat cold structure contractor. During this time, there is no access to the cryostat.  Once this is completed, the final instrumentation is completed, and the purge can begin. 
    
\end{itemize}



\subsection{Detector Commissioning Phase}
\label{sec:fdsp-tc-inst-comiss}
% Anne reviewing this section 3/20/19
%After the \dword{detmodule} is installed in the cryostat, much work remains before it can be operated. First the \dword{tco} must be closed. This requires bringing back the cryostat manufacturer. First the missing panel with steel beams and steel panel are installed to complete the cryostat's outer structural hull. Then the remaining foam blocks and membrane panelsare installed from the inside using the roof access holes to enter the cryostat. 
After the \dword{spmod} is installed in the cryostat, much work remains before it can be operated. 
The cryostat manufacturer must come back to close the \dword{tco}. 
First they install the missing panel %with steel beams and steel panel are installed 
to close it, which completes the cryostat's outer structural hull.  \fixme{correct?} 
Then the remaining foam blocks and membrane panels
are brought inside the cryostat  through the roof's access holes and installed. 


%In parallel, the \lar pumps are installed at the ends of the cryostat and final connections are made to the recirculation plant. Once the pumps are installed, the cryostat is closed, everything is leak tested, and the cryogenics plant can be brought into operation. First the air inside the cryostat is purged by injecting pure argon gas at the bottom  at a rate that fills the cryostat volume uniformly but faster than the diffusionrate. This produces a column of argon gas that rises through the volume and sweeps out the air. This process is referred to as the \textit{piston purge}. When the piston purge is complete, the cool down of the \dword{detmodule} can begin. Misting nozzles inject a liquid-gas mix into the cryostatthat cools the detector components at a controlled rate. 
In parallel, the \lar pumps are installed at the ends of the cryostat and final connections are made to the recirculation plant. Next, everything is leak tested, and the cryogenics plant can be brought into operation. The system first purges the air inside the cryostat  by injecting pure \dword{gar} at the bottom  at a rate that fills the cryostat volume uniformly but faster than the diffusion rate. This ``piston purge'' process produces a column of \dword{gar}  that rises through the volume and pushes out the air.  When the piston purge is complete, misting nozzles inject a liquid-gas mix into the cryostat that cools the detector components at a controlled rate. 

%Once the detector is cold, the filling process can begin. Liquid argon stored at the surface  at \surf is vaporized and brought down the shaft in gaseous form. The gas is then re-condensed underground. The \lar then flows through filters to remove any H$_2$O or O$_2$ and into the cryostat.Given the very large volume of the cryostat and the limited cooling power for recondensing, \num{12} months will be required to fill the first \dword{detmodule}. %During this time
Once the detector is cold, the filling process begins. \dword{lar} stored at the surface  at \dword{surf} is vaporized, brought down the shaft in gaseous form, and re-condensed underground. The \lar then flows through filters to remove any H$_2$O and O$_2$ before entering the cryostat. Given the volume of the cryostat and the limited cooling power for recondensing, \num{12} months will be required to fill the first \dword{detmodule}. the detector readout electronics will be on monitoring the status of the detector. 

Testing and constant monitoring of the detector will take place from \dword{tco} closing until the end of the filling process. 
Before the last %man
access hole is sealed:
\begin{enumerate}

    \item A pedestal and \dword{rms} characterization of all cold electronic channels will verify that all \dword{apa} front-end boards are responding and no dead channel or new noise sources arose following the \dword{tco} closing.
    
    \item A noise scan of all \dword{pd} channels is performed as last check before sealing.

    \item Each \dword{apa} wireplane will be checked to verify it is isolated from the \dword{apa} frame and properly connected to its \dword{hv} power supply through the following steps:
    
\begin{itemize}

    \item The \dword{shv} connector of each wireplane bias channel will be unplugged at the power supply, and the resistance  between inner conductor and ground and its capacitance 
    \fixme{capac betw inner cond and grd? or of sthg else?} will be measured. The resistance should show the wireplane is electrically isolated from the ground, while the capacitance value should match that of the cold \dword{hv} cable and the capacitance of the circuit on the \dword{apa} top frame.

    \item \SI{50}{V} is applied to each wireplane and the current drawn checked against the expected value.
    
    \item Nominal voltages are applied to each wireplane, and the current drawn is checked against the expected value. 
    
\end{itemize}

    \item A low \dword{hv} (i.e., \SIrange{1}{2}{kV}) is applied to the cathode, and the current drawn is checked against the expected value to ensure the integrity of the \dword{hv} line.

\end{enumerate}

During the piston purge process %phase, 
periodic monitoring of %both 
\dword{apa}, \dword{ce}, and \dword{pd} system noise (pedestal, \dword{rms}) will occur.

A number of %these 
the following tests, in addition to new ones, will instead take place during the cool-down phase:

\begin{enumerate}


    \item Each \dword{apa} wireplane isolation and proper connection to its \dword{hv} power supply will be checked at regular time intervals as was done before sealing the cryostat.
    
    \item \SIrange{1}{2}{kV} will be held on the cathode, and the current drawn will be  monitored constantly to observe the trend in temperature of the total resistance.
    
    \item \dword{ce} noise figures (pedestal, \dword{rms}) will be measured at regular intervals and their trends with temperature recorded.
    
    \item \dword{pd} system noise (pedestal, \dword{rms}) will be measured at regular intervals and its trend with temperature recorded.
    
     \item Values of the temperature sensors deployed in several parts of the cryostat will be monitored constantly to %see 
     watch the progress of the cool down phase and to relate the temperature to the behavior of the other \dword{spmod} subsystems. 
     
\end{enumerate}

Regular monitoring of \dword{ce} and \dword{pd} noise, as well as checks of wireplane isolation and proper connections to the bias supply system will continue throughout the filling, recording noise variations as a function of the progressively reduced temperature. In addition,

\begin{enumerate}

    \item As each purity monitor is submerged in liquid, it will be turned on every eight hours to control \dword{lar} purity.
    
    \item As soon as top ground planes are submerged, \dword{hv} on the cathode will be raised up to \SI{10}{V} to check that the current drawn by the system agrees with expectations.

\end{enumerate}

Once the \dword{detmodule} is full, the drift \dword{hv} will be carefully ramped up following these steps:

\begin{enumerate}

    \item The need for a filter regeneration is evaluated before starting any operation;

    \item Once filter regeneration is completed (if needed), the \dword{lar} surface is examined using cameras to verify that the surface is flat, with no bubbles or turbulence;
    
    \item \dword{lar} recirculation is started, and \dword{lar} surface is examined again to see if activating the recirculation system introduced any turbulence into the liquid;
    
    \item Wait one day after beginning  recirculation to stabilize the \dword{lar} flow inside the \dword{detmodule}, then start the \dword{hv} ramp up.
    
\end{enumerate}

Cathode voltage is raised in steps over three days. 
On the first day, cathode voltage is first raised to \SI{60}{kV}, then to \SI{90}{kV} after waiting two hours, and finally to \SI{120}{kV} after waiting another two hours, and then left at this value overnight.
On the second day, cathode voltage is first raised to \SI{140}{kV}, then to \SI{160}{kV} after waiting four hours, and then left at this value overnight. 
On the third day, cathode voltage is first raised to \SI{170}{kV} and then to the nominal operating voltage of \SI{180}{kV} after waiting four hours. 
During each \dword{hv} ramp up, all \dword{ce} current draws are monitored, and the procedure is stopped if any of the current draws go out of the allowed range. 
During each waiting period, regular \dword{daq} runs monitor \dword{ce} and \dword{pd} noise and response, while cathode \dword{hv} and current draw stability are constantly monitored.

In \dword{pdsp}, this process took three days, after which the system was  ready for data-taking. With a detector twenty times larger, the process will take longer, but the turn on time should still be relatively short. 

\subsection{Risks}


% risk table values for subsystem SP-FD-JPO
\begin{longtable}{p{0.15\textwidth}p{0.13\textwidth}p{0.13\textwidth}p{0.28\textwidth}p{0.06\textwidth}p{0.06\textwidth}p{0.06\textwidth}} 
\caption{Specification for SP-FD-JPO \fixmehl{ref \texttt{tab:specs:SP-FD-JPO}}} \\
\rowcolor{dunesky}
ID & Risk & Label & Mitigation & Prob ability & Cost Impact & Sched ule Impact \\  \colhline
RT-JPO-001 & Personnel injury & jpo-person-injury & Follow established safety plans. & M & L & H \\  \colhline
RT-JPO-002 & Shipping delays & jpo-shipping-delay & Plan one month buffer to store  materials locally. Provide logistics manual. & H & L & L \\  \colhline
RT-JPO-003 & Missing components cause delays & jpo-missing-components & Use detailed inventory system to verify availability of  necessary components.  & H & L & L \\  \colhline
RT-JPO-004 & Import, export, visa issues  & jpo-import-visa & Dedicated \dword{fnal} \dword{sdsd}division will expedite import/export and visa-related issues. & H & M & M \\  \colhline
RT-JPO-005 & Lack of available labor  & jpo-labor-avail & Hire early and use Ash River setup to train \dword{jpo} crew. & L & L & L \\  \colhline
RT-JPO-006 & Parts do not fit together & jpo-cannot-assemble & Generate \threed model, create interface drawings, and prototype detector assembly. & H & L & L \\  \colhline
RT-JPO-007 & Cryostat damage & jpo-cryostat-damage & Use cryostat false floor and temporary protection. & L & L & M \\  \colhline
RT-JPO-008 & Weather closes SURF & jpo-weather-delay & Plan for \dword{surf} weather closures & H & L & L \\  \colhline
RT-JPO-009 & Detector failure during \cooldown & jpo-cooldown-failure & Cold test individual components then cold test \dword{apa} assemblies immediately before installation. & L & H & H \\  \colhline

\label{tab:risks:SP-FD-JPO}
\end{longtable}

\section{Environmental, Safety, and Health (ES\&H)}
\label{sec:fdsp-tc-safety}

 Volume \volnumbertc{}, Section~\ref{vl:tc-ESH} of the \dword{dune} \dword{tdr} outlines the requirements and regulations that \dword{dune} work must comply with, whether (1) at \dword{fnal}, (2) in areas  leased by \dword{fnal} or the \dword{doe}, (3) in unleased space at \dword{surf}, or (4) at collaborating institutions.
 
%%%%%%%%%%%%%%%%%%%%%%%%%%%%%%%%%
\subsection{Documentation Approval Process}


\dword{dune} implements an engineering review and approval process for all required documentation, including structural calculations, assembly drawings, load tests, \dwords{ha}, and procedural documents for a comprehensive set of identified individual tasks.  For the larger operations and systems like \dword{tpc} component factories, the \dword{dss}, cleanroom, and assembly infrastructure, a joint safety committee also reviews the documentation then visits the site to conduct
 an \dword{orr}. Before signing off on the documentation for an operation, the committee watches it being performed. 
 
 Structural calculations, assembly drawings and proper documentation of  load tests, hazard analyses, and procedures for various items and activities will require review and approval before operational readiness is granted. 

 
%%%%%%%%%%%%%%%%%%%%%%%%%%%%%%%%%
\subsection{Support and Responsibilities}

An \dword{esh} coordinator, who will report to the \dword{dune} Project \dword{esh} manager, has overall \dword{esh} oversight responsibility for the \dword{dune} activities at the  \dword{sdwf} and on the \dword{surf} site. 
This person coordinates any \dword{esh} activities and facilitates the resolution of any issues that are subject to the requirements of the \dword{doe} Workers Safety and Health Program, Title 10, Code Federal Regulations (CRF) Part 851 (10 CFR 851) (see Volume~\volnumbertc{}).  The \dword{esh} coordinator facilitates training and runs weekly safety meetings. This person is also responsible for managing \dword{esh}-related  documentation including training records, \dword{ha} documents, weekly safety reports, records on materials-handling equipment, near-miss and accident reports, and equipment inspection.

All employees have work stop authority in support of  a safe working environment. 

%%%%%%%%%%%%%%%%%%%%%%%%%%%%%%%%%
\subsection{Safety Program}

Using the \dword{nova} Far Detector Laboratory as a guideline for remote facilities, several other key documents will guide the \dword{fd} installation safety program, as follows:

\begin{enumerate}
\item	Fire Safety and Building Emergency Evacuation Plan, which includes the fire evacuation plan, fire safety plan, lockdown plans, and the site plan;
\item	\dword{ha} document, which describes all typical hazards and their mediation procedures; 
\item	Safety Data Sheets (SDS), 
\item	Respiratory Plan, as required for chemical or ODH hazards, and 
\item	Training Program, which covers required certifications and  training records.
\end{enumerate}


During the installation setup phase, as new equipment is being installed and tested, new employees and collaborators will be trained, and larger teams from the consortia, \dword{sdsd},  and contractors will need access to the underground facilities.  Given the maximum of 140 \dword{fte} underground at any given time, we will move from one to two shifts per day at this point. 


Unlike most items, the \coldbox and cryogenics system will not be %fully 
tested during the trial assembly work at Ash River. 
While the new \coldbox design is very similar to \dword{pdsp}'s, it will be operated under \dword{doe} and \dword{feshm} regulations.  Procedures for operating the \coldbox will be written according to the established requirements.

\fixme{this next pgraph not needed IMO. anne} During this phase, \dword{lbnf} will be completing the cold structure on \dword{detmodule} \#1 and beginning the warm structure on  \dword{detmodule} \#2. Once \cooldown begins on module \# 1, unlike for \dword{pdsp}, it will be safe for workers to remain in the cleanroom.  

  

\fixme{the next pgraph should be in Steve's volume}

While \dword{esh} is  a host laboratory responsibility, a  \dword{gsc} will evaluate applicable codes and standards, including international code equivalency for the design, assembly, and installation of the \dword{dune} \dwords{detmodule}. The \dword{gsc} is a team of engineering and \dword{esh} experts from within \dword{lbnf} and \dword{dune} organizations.  The \dword{dune}, \dword{jpo}, and \dword{lbnf} organizations will develop manufacturing, assembly, and installation processes and procedures according to the \dword{gsc}'s recommendations. 

\dword{dune} will develop an  \dword{esh} plan for detector  installation that defines  
the \dword{esh} requirements and responsibilities for personnel during  assembly, installation, and construction of equipment at \dword{surf}. It will cover at least the following areas:

{\bf Work Planning and \dword{ha}:} The goal of the work planning and \dword{ha} process is to initiate thought about the hazards associated with work activities and plan how to perform the work. Work planning ensures the scope of the job is understood, appropriate materials and tools are available, all hazards are identified, mitigation efforts are established, and all affected employees understand what is expected of them. 
The Work Planning and Hazard Analysis program is documented in Chapters 2060 in the \dword{feshm}.

The shift supervisor and the \dword{esh} coordinator  will lead a work planning meeting at the start of each shift  to (1) coordinate the work activities, (2) notify the workers of potential safety issues, constraints, and hazard mitigations, (3) ensure that employees have the necessary \dword{esh} training and \dword{ppe}, and (4) answer any questions.

{\bf Access and training:}  All \dword{dune} workers requiring access to the \dword{surf} site must (1) register through the \dword{fnal} Users Office to receive the necessary user training and a \dword{fnal} identification number, and (2) they must apply for a \dword{surf} identification badge. 
The workers will be required to complete \dword{surf} Surface and Underground Orientation classes. Workers accessing the underground must also complete 4850L and 4910L specific unescorted access training, and obtain a \dword{tap} for each trip to the underground area; this is required as part of \dword{surf}'s Site Access Control Program. 
A properly trained guide will be stationed on all working levels. 

{\bf \dword{ppe}:} 
The host laboratory is responsible for supplying appropriate \dword{ppe} to all workers. 

{\bf \dword{em} program} The \dword{sdsta} will maintain an Emergency Response incident command system and an \dword{ert}.  The guides on each underground level will be trained as first responders to help in a medical emergency.
  
  
  {\bf House cleaning:} All workers are responsible for keeping a clean organized work area. This is particularly important underground. Flammable items must be in proper storage cabinets, and items like empty shipping crates and boxes must be removed and 
transported back to the surface to make space.


{\bf Equipment operation:} All overhead cranes, gantry cranes, fork lifts, motorized equipment, e.g., trains and carts, will be operated only by trained  operators. 
Other equipment, e.g., scissor lifts, pallet jacks, hand tools, and shop equipment, will be operated only by people trained
and certified for the particular piece of equipment.

