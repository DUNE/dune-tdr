
Figure~\ref{fig:calib-FTmap} shows the current cryostat design for the %DUNE SP FD 
\spmod with penetrations for various sub-systems. The penetrations dedicated to calibration are the highlighted black circles. The ports on the far east and far west are outside the \dword{fc}. The current plan is to use these penetrations for several different purposes. For example, the penetrations on the far east and west will be used both by laser and radioactive source systems (if deployed). In addition to these dedicated ports, the \dword{dss} and cryogenic ports (orange and blue dots in Figure~\ref{fig:calib-FTmap}) will also be used as needed to route cables for other calibration systems (e.g., fiber optic cables for the \dword{sp} \dword{pds} calibration system, which is described elsewhere, in \dword{tdr} \spchpds. \dword{dss} and cryogenic ports can be accommodated by feedthroughs with a CF63 side flange for this purpose.   

\begin{dunefigure}[Cryostat penetration map with calibration ports]{fig:calib-FTmap}
{Top view of the \spmod %DUNE SP FD 
cryostat showing various penetrations. Circles highlighted in black are multi-purpose calibration penetrations. The green dots are \dword{tpc} signal cable penetrations. The blue ports are cryogenic ports. The orange ports are \dword{dss} penetrations. The larger purple ports at the four corners of the cryostat are human access ports.}
\includegraphics[height=2.0in]{calib-FTmap.png}
\end{dunefigure}





Placement of these penetrations is largely driven by requirements for the ionization track laser. %and radioactive source system. 
The ports %that are %inwards 
toward the center of the cryostat are placed near the \dwords{apa} where the \efield is small %\fixme{check} 
to minimize any risks due to %the
\dword{hv} discharge. For the far east and west ports, \dword{hv} is not an issue because they are located outside the \dword{fc} and the penetrations are located %near 
close to mid-drift (location favorable to possible source deployment).
%to meet radioactive source requirements. 
%\fixme{The amendment above may need some work. I won't look at it until it's ready.}
Implementing the ionization track laser system as %proposed 
described in Section~\ref{sec:sp-calib-sys-las-ion} requires \num{20} feedthroughs %at 
in the four \dword{tpc} drift volumes; this arrangement allows lasers to be used for full volume calibration of the \efield and associated diagnostics (e.g., \dword{hv}). 

The distance between any two consecutive feedthrough columns in Figure~\ref{fig:calib-FTmap} is approximately \SI{15}{\m}, which is considered reasonable because experience from the \dword{microboone} laser system shows that tracks will propagate over that detector's full \SI{10}{\m} length. Assuming that the effects of Rayleigh scattering and self-focusing (Kerr effect) do not limit the laser track length, this laser arrangement could illuminate the full volume with crossing track data.  Please note that, at this time, the maximum usable track length is unknown, and it may be that the full \SI{60}{\m} \detmodule length could be covered by the laser system after optimization.
