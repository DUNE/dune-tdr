
%\fixme{KM/JM: Done; SG check?}
%KM: I am OK with this, let's see what the editors hate. \fixme{SG: done checking; 1. photoelectron laser missing from the table. Added it. 2. Updated the text to clearly state what the cost estimates include; 3. added some description for labor and a table with placeholders for actual labor hours. 4. The way it is currently split in the cost table, using the same rows will be tricky for labor hours. E.g. for the laser itself, no significant labor is associated. So, for the labor table, although it inconsistent with the cost table, I listed each of our devices as one entity. check if you are okay with this. Alternatively, we can lump everything into one system for the cost table as well so both tables are consistent which is what is expected from main editors. 5. I added a new category for labor, "physics & simulation" as those hours will also need to accounted for. }

\begin{dunetable}
[Calibration System Cost Summary]
{p{0.15\textwidth}p{0.1\textwidth}p{0.15\textwidth}p{0.4\textwidth}}
{tab:calib-cost}
{Cost estimates for different calibration subsystems.  All cost estimates include packing and shipping costs.}   
System & Quantity & Cost (under development) (k\$ US) & Description \\ \toprowrule
laser & 17 & - & high intensity laser source for laser system  \\ \colhline
laser optics module & 20 & -&  insulator ingress into detector and flange interface, mirrors, power, cables for laser system, and positioning system \\ \colhline
Photoelectron system & 4000 & - & includes targets to be mounted, both dots and strips, along with optical fibers to illuminate the photoelectric targets on two cathode planes, on both sides of each  cathode plane \\ \colhline 
DD generator device  & 2 &- & neutron source for pulsed neutron source system. \\ \colhline
Other PNS components  & 2 & - & includes moderator, surrounding shielding and neutron monitor \\ \colhline
%laser device & 17 & \num{50.0} & Laser system \\ \colhline
%feedthrough interface & 20 & \num{100.0} & Laser system; includes insulator ingress into detector and flange interface, mirrors \\ \colhline
%DD generator device  & 2 & \num{102.5} & Pulsed Neutron Source \\ \colhline
%Moderator  & 2 & \num{25.1} & Pulsed Neutron Source: materials to degrade neutrons to correct energy \\ \colhline
%Shielding & 2 & \num{24.0} & Pulsed Neutron Source: surrounding shielding \\ \colhline
% Monitor  & 2& \num{16.7} & Pulsed Neutron Source: Neutron monitor (device) and colimator (materials)  \\ \colhline
\end{dunetable}

\begin{dunetable}
[Calibration labor]
{p{0.25\textwidth}p{0.15\textwidth}p{0.1\textwidth}p{0.08\textwidth}p{0.08\textwidth}p{0.1\textwidth}p{0.08\textwidth}}
{tab:calib-labor}
{Estimate of labor hours for each category of personnel for different calibration subsystems.}
System  & Faculty/Scientist & Post-doc & Student & Engineer & Technician  &  \textbf{Total}\\ \toprowrule
& (hours) & (hours)& (hours)& (hours)& (hours)& (hours)\\ \toprowrule
Ionization Laser system & -& -& -& -& - & - \\ \colhline
Photoelectron Laser system & -& -& -& -& - & - \\ \colhline
Laser positioning system & -& -& -& -& - & - \\ \colhline
Pulsed Neutron Source system & -& -& -& -& - & - \\ \colhline
Physics \& Simulation & -& -& -& -& - & - \\ \colhline
\end{dunetable}

%\todo{SG: need quantity for photoelectron laser in the table 1.6.} KM: Quantity added from Jelena.
Table \ref{tab:calib-cost} shows the current cost estimates for the calibration subsystems. It also shows the quantity associated with each subsystem and a brief description of what is included in the cost estimate. The cost estimates only include materials and supplies (M\&S) and packing and shipping, but not labor and travel costs. To serve one SP detector module, there are 20 ports for the laser (see Figure~\ref{fig:ftmap}); 14 ports will each need one laser and one feedthrough interface, but for the 6 ports central to the cryostat, one laser will service two ports. According to simulation studies, two pulsed neutron systems are needed for one SP module.

Labor costs depend on personnel category (e.g., faculty, student, technician, post-doc, engineer) and vary by region and institution, so costs are quantified using labor hours needed to fulfill a given task. Table~\ref{tab:calib-labor} provides estimates of labor hours for each subsystem. 
%, for a total of \num{95010} labor hours for all calibration tasks. 



%The costs of equipment and materials and supplies for the baseline systems are described in Table~\ref{tab:calibcostsumm}. To serve one SP detector module, there are 20 ports for the laser (see figure~\ref{fig:ftmap}); 14 ports will each need one laser and one feedthrough interface, but for the 6 ports central to the cryostat, one laser will service two ports. Two pulsed neutron systems are needed for one SP module. % The total cost of the baseline calibration systems therefore is \$3.4M.
%\fixme{Add estimate of laser positioning system, DAQ/computers, racks? cables?}
%\fixme{Jose to get a cost from Jelena for position system, and photoelectron }