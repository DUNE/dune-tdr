%\section{Installation, Integration, and Commissioning}

This section describes the installation plans for calibration systems. Most of the hardware is to be installed outside the cryostat so, space on mezzanine surrounding each calibration port is important for powering and operating the calibration systems. However, some sub-systems have internal components which will be installed %once the associated components are installed inside the cryostat (e.g., for the laser, after the periscopes are installed)  
following a specific installation sequence, coordinated with other consortia.

\subsubsection{Ionization Laser System} 
 
Checking the alignment of the optical components is an essential step of the ionization laser system installation. The system includes a low power visible laser that can be used for the several mirror alignment operations, but before that use, both the UV and the visible lasers in the laser box need to be aligned.
Alignment of the visible and UV (Class 4) lasers requires special safety precautions and must be carried out once for each periscope/laser system before installing further \dword{tpc} components. For that reason, the laser boxes must be installed on the cryostat roof as soon at that area becomes accessible.  
 
The periscopes are the only components of the ionization laser system that will be inside the cryostat, but they will be installed from the top of the cryostat and not from the \dword{tco}, including the alternative options. However, this installation should be done very carefully in the presence of an operator inside the cryostat, to ensure there are no collisions of the long laser periscopes with other detector components, especially \dword{fc} elements and \dword{ce} cable trays.
The periscopes should be installed after the relevant structural elements, especially the top \dword{fc} modules. Installation should proceed in sequence with the assembly of other components, with the furthest from \dword{tco} assembled first.

The relevant \dword{qc} is essentially an alignment test.
The \dword{lbls} can be used to align the periscopes as they are installed, so it is important that the \dword{lbls} is also installed in the same sequence as the periscopes.




A support beam structure closest to the \dword{tco} temporarily blocks the calibration ports, but it is removed after the last \dword{tpc} component is installed. After that, the final calibration components can be installed, including the periscopes on the \dword{tco} end wall. 

\subsubsection{Laser Beam Location System}
This system has several parts that need to be installed inside the \dword{tco}, and some must be integrated with the 
\dword{hv} system during installation underground. 

The PIN diode system uses a set of diodes that fire when the laser beam hits them. Because the laser shoots from above and the diodes must be in a low voltage region, the plan is to place the diodes below the bottom \dword{fc}, facing upward, simply on a tray close to the cryostat membrane.

For the pointing measurement, the beams will pass through the \dword{fc} electrodes and hit the diodes below. There will be \num{32} of these diode clusters to be installed. The installation will consist of positioning the cluster trays in pre-determined locations, and routing the cables to the respective feedthroughs (work is still underway to decide how to route cables and which flanges to use). 

The second \dlong{lbls} consists of a set of \num{32} mirror clusters: a plastic or aluminum piece holding four to six small mirrors \SI{6}{\mm} in diameter, each at a different angle; the ionization laser will point to these mirrors to obtain an absolute pointing reference. These clusters will be attached to the bottom \dword{fc} profiles facing into the \dword{tpc}. 
%These cross bars must contain small alignment slots, matching the cluster pieces, in order for us to know the exact position of each cluster. 
This attachment/assembly of the mirror clusters on corresponding \dword{fc} profiles will be done during \dword{fc} assembly underground.



\subsubsection{Photoelectron Laser System} 
A large number of photoelectric targets (about \num{4000}) must be %fixed 
attached to the cathode. Experience from other experiments indicates that targets can be glued to the cathode surface, which can be done after cathode assembly but before the cathode is installed in the cryostat. 

Once the \dwords{cpa} are in place, the photoelectric target locations will need a high precision survey, which is necessary for the absolute calibration of the electric field with the photoelectron laser. 

The third part of the installation is  quartz optical fibers on the \dword{apa}, needed to illuminate  the photoelectric targets with light from the Nd:YAG laser. 
Fiber tips must be properly fastened and oriented for effective illumination, and fiber bundle routing will bring the fiber bundles to the outside of the cryostat where Nd:YAG laser injection points will be located. 

\subsubsection{Pulsed Neutron Source System} 
The \dword{pns} will be installed after the human access ports are closed because the source sits above the cryostat. Installing the system should take place in two stages. In the first stage, the assembly of the system would be independent of the \dword{tpc} installation. The whole system will be %installed 
assembled on the ground outside the cryostat at a dedicated radiation safe area. Once assembled, the neutron source will be lifted by crane and integrated with the cryostat structure. Final \dword{qc} testing for the system will be operating the source and measuring the flux with integrated monitor and dosimeter.



