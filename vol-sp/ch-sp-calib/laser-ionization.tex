%%%%%%%%%%%%%%%%%%%%%%%%%%%%
\subsection{Ionization Laser System}
\label{sec:sp-calib-sys-las-ion}
%%%%%%%%%%%%%%%
%\subsubsection{Physics Motivation}
%\label{sec:sp-calib-sys-las-ion-phys} 
Through its effect on drift velocity, recombination and lifetime, the \efield is a critical parameter for physics signals as it ultimately impacts the spatial resolution and energy response of the detector. The primary purpose of a laser system is to provide an independent, fine-grained estimate of the \efield in space and time. 

Since it measures spatial distortions of straight tracks, the laser system actually measures the local drift velocity field directly and helps define the detector \dword{fv}, and this in itself is an important input for the \dword{lbl} analysis. 
%, and this is in itself an input to the analysis.
%Still,
However, it is still important to use information independent from the charge in order to disentangle effects like lifetime and recombination from \efield distortions. The laser system can accomplish this, by using the position information to derive the \efield from the local velocity map, taking into account the colinearity between both vectors, and the relatively well studied relation on the magnitude (see~\cite{Li:2015rqa} and references [29, 45-58] therein). A laser system also has the intrinsic advantage of being immune to recombination effects, thus eliminating particle-dependent effects.  

There are multiple sources which may distort the \efield temporally  and/or spatially in the detector. Current simulation studies indicate that positive ion accumulation and drift (space charge) due to ionization sources such as cosmic rays or \Ar39 is small in the \dword{dune} \dword{fd}, causing \efield distortions of at most \SI{0.1}{\%}~\cite{bib:mooney2018}.
However, not enough is known yet about the fluid flow pattern in the \dword{fd} to exclude the possibility of stable eddies which may amplify the effect for both \single and \dual modules. This effect can get further amplified significantly in the \dword{dpmod} due to  accumulation in the liquid of ions created by the electron multiplication process in the gas phase.
%due to ion accumulation at the liquid-gas interface. 
Additionally, other sources in the detector (especially detector imperfections) can cause \efield distortions. For example, field cage resistor failures, non-uniform resistivity in the voltage dividers, \dword{cpa} misalignment, \dword{cpa} structural deformations, and \dword{apa} and \dword{cpa} offsets and  deviations from flatness can create localized \efield distortions. These effects are presented in Figures~\ref{fig:efield_cpa_distortions_boyu2017} and \ref{fig:efield_resistorfailure_mooney2019}, that show the impact of a few \% on the \efield from \SI{2}{\cm} \dword{cpa} position tilts and of up to \SI{4}{\%} from field cage single resistor failures.

\begin{dunefigure}[Impact on \efield of \dword{cpa} position distortions]{fig:efield_cpa_distortions_boyu2017}
{Illustration of a possible distortion of the \dword{cpa} position~\cite{bib:yu2017a}, assuming a \SI{2}{\cm} swing, and its impact on \efield (right).}
\includegraphics[width=0.8\textwidth]{efield_cpa_distortions_boyu2017.png}
\end{dunefigure}

\begin{dunefigure}[Impact on \efield of \dword{fc} resistor failures]{fig:efield_resistorfailure_mooney2019}
{Impact on \efield magnitude distortions of a single \dword{fc} resistor failure~\cite{bib:mooney2019a}.}
\includegraphics[width=0.5\textwidth]{efield_resistorfailure_mooney2019.png}
\end{dunefigure}

In both \single and \dual systems, the failure of a resistor will create significant, local \efield distortions which will need to be identified. In the \dual system, four resistors would have to fail to cause a failure across the field cage gap, but even one failure in the SP can have an impact; this may be partially mitigated by modifying the HV, but not completely. While the resistor failure will be detected temporally, its location in space is not possible to determine from slow controls monitoring data. Misalignments of detector objects or deformations may also create  \efield distortions; while individual effects may be small, it is possible to have a combined, significant effect.
Each individual \efield distortion may add in quadrature with other effects, and can aggregate up to \SI{4}{\%} under certain conditions. Understanding all these effects requires in-situ measurement of \efield for proper calibration. 

Useful secondary uses of laser include alignment (especially modes that are weakly constrained by cosmic rays),
%; see Figure~\ref{fig:apacurtainalign}),
stability monitoring, and diagnosing detector performance issues
%failures 
(e.g., \dword{hv}).  
Misalignment may include physical deformation and/or rotations of objects within the detector. Given the expected low rate of cosmic ray events (about 3500/day/10-kt) at the underground location, calibration with cosmic rays is not possible over short time scales. Certain alignment ``directions''  are difficult to assess with cosmic rays alone, such as distortions of the detector that preserve the gap widths and do not shift the \dwords{apa} in $x$ near the gaps relative to one another.
These distortions include global shifts and rotations in the locations of all detector elements, and crumpling modes where the edges of the \dwords{apa} hold together but angles are slightly different from nominal.   

%A laser system also has the intrinsic advantage of being immune to recombination, thus eliminating particle-dependent effects.  





%%%%%%%%%%%%%%%%%%%%%%%%%%%%
\subsubsection{Requirements}
\label{sec:sp-calib-laser-req}


% KM outline
%% Requirements we are held to from other systems -- EB table -- see Jose's early talk
%% Targets for SN and LBL physics -- or just remind in other subsections? \fixme{KM: right now the targets for SN and LBL physics exist in the design sections.}
%% System must operate for a long time
    % SG: physics driven calibration requirements, need a table to connect calibration requirements to high level physics requirements, not easy, but we need to try

%\fixme{guidance coming soon!}

%\fixme{KM: adjusted to be specific to laser; SG: I have made some edits as well. JM: OK, signing off.}

%The DUNE physics requirements and the high level specifications of other existing systems are the driving motivation for the specifications of the performance of the dedicated calibration systems, described below. From those, and the constraints due to detector dimensions, etc, derive also the engineering specifications of each calibration system, described in each system's respective section.

%\paragraph{\efield measurements}
The energy and position reconstruction requirements for physics measurements lead to requirements on the necessary precision of the laser calibration \efield measurement, its spatial coverage and granularity. The next sections discuss the rationale behind each requirement, which we take as the DUNE specification.
%, with ALARA (or AHARA for the coverage) as goal.

\paragraph{\efield precision}

In the \dword{lbl} and high-energy range, the DUNE IDR states that ``...calibration information needs to provide approximately 1-2\% understanding of normalization, energy, and position resolution within the detector.'' (from \cite{idr-vol-1}, p. 4-47). 
The DUNE Physics TDR (Section 4.4.1.1) indicates that a \num{1}\% bias in the lepton energy scale is significant for the \dword{lbl} sensitivity to CPV.

Since a smaller \efield leads to higher electron/ion recombination and therefore a lower collected charge, distortions of the \efield are one of the possible causes of an energy scale bias. In order to connect that requirement to a specification on the necessary precision of the \efield measurement, we note that, via recombination studies\cite{mooney2018}, we expect a \num{1}\% distortion on \efield to lead to a \num{0.3}\% bias on collected charge.

Since other effects will contribute to the lepton energy scale uncertainty budget, we consider a goal for the calibration system to measure the \efield to a precision of $\sim\num{1}\%$ so that its impact on the collected charge is well below \num{1}\%.
This is also motivated by consistency with the high level DUNE specification on field uniformity throughout the volume due to component alignment and HV system, that was set at \num{1}\%.

Together with two other high-level DUNE specifications, the APA wire spacing (4.7~mm) and the front end peaking time (1~$\mu s$), the impact of this \efield precision requirement on engineering parameters of the calibration laser system is discussed further ahead, in Section \ref{sec:sp-calib-sys-las-ion-meas}.

\paragraph{\efield measurement coverage}

In practice, measuring the \efield  ~``throughout the [whole] volume'' of the TPC will be difficult, 
%hard, 
so we must establish a goal for the coverage and granularity of the measurement. 
Until a detailed study of the propagation of the coverage and granularity into a resolution metric is available, we can already make a rough estimation of the necessary coverage in the following way. Taking \num{4}\% as the maximum \efield distortion resulting from a compounding of multiple possible effects in the DUNE FD (\cite{idr-vol-1}, page~4-53), we can then ask what would be the maximum acceptable size of the spatial region uncovered by the calibration system, if a distortion of that magnitude (systematically biased in the same direction) were present. In order to keep the overall (average) \efield distortion at the \num{1}\% level, then that region should be no larger than \num{25}\% of the total fiducial volume. Therefore, we aim to have a coverage of \num{75}\% or better.

\paragraph{\efield measurement granularity}

The IDR states that a fiducial volume uncertainty of \num{1}\% is required (ref.~\cite{idr-vol-1}, p.~4-46) and that this translates to a position uncertainty of \num{1.5}~cm in each coordinate (ref.~\cite{idr-vol-2}, p.~2-12). Also that in the $y$ and $z$ coordinates, the wire pitch of \num{4.7}~mm achieves that while in the drift ($x$) direction, the position is calculated from timing so it is claimed it should be known better.

But the position uncertainty depends also on the electric field, via the drift velocity. Since the position distortions accumulate over the drift path of the electron, it is not enough to specify an uncertainty on the field, we must accompany it by specifying the size of the spatial region of that distortion. i.e. a \num{10}\% distortion would not be relevant if it was confined to a \num{2}~cm region, for instance, and the rest of the drift region was nominal. So what matters is the product of [size of region] $\times$ [distortion]. Moreover, we should distinguish distortions of two types:
\begin{enumerate}
\item those affecting the magnitude of the field. Then the effect on the drift velocity $v$ is also a change of magnitude. According to the function provided in \cite{walkoviak2000}, close to \num{500}~V/cm, the variation of the velocity with the field is such that a \num{4}~\% variation in $E$ leads to a \num{1.5}~\% variation in $v$.
\item those affecting the direction of the field. Nominally, the field $E$ should be along $x$, so $E = E_L$ (the longitudinal component). If we consider that the distortions introduce a new transverse component $E_T$, in this case this translates directly into the same effect in the drift velocity, that gains a $v_T$ component that is $v_T=v_L  E_T/E_L $, i.e. a \num{4}~\% transverse distortion on the field leads to a \num{4}~\% transverse distortion on the drift velocity.
\end{enumerate}

So, a \num{1.5}~cm shift comes about from a constant \num{1.5}~\% distortion in the velocity field over a region of \num{1}~m. In terms of electric field, that could be from a \num{1.5}~\% distortion in $E_T$ over 1~m or a \num{4}~\% distortion in $E_L$ over the same distance.

From ref.~\cite{idr-vol-1}, page~4-53, \efield distortions can be caused by space-charge effects due to accumulation of positive ions caused by $^{39}$Ar decays (cosmic rate is low in FD), or detector defects, such as field cage resistor failures, resistivity non-uniformities, etc. These effects added in quadrature can be as high as \num{4}~\%. From ref. ~\cite{mooney2018}, the space charge effects due to $^{39}$Ar can be of the order of \num{0.1}~\% for the single phase (SP), and \num{1}~\% for the dual phase (DP), so in practice these level of 
%that kind of distortion 
distortions need to cover several meters in order to be relevant.
Other effects due to \dword{cpa} or \dword{fc} imperfections can be higher than those due to space charge, but they are also much more localized. If we assume that there are no foreseeable effects that would distort the field more than \num{4}~\%, and considering the worst case (transverse distortions), then the smallest region that would produce a \num{1.5}~cm  shift is \num{1.5}/\num{0.04}~=~\num{37.5}~cm. This provides a target for the granularity of the measurement of the \efield distortions in $x$ to be smaller than about \num{30}~cm, with of course a larger region if the distortions are smaller. Given the above considerations, then a voxel size of \num{10}x\num{10}x\num{10}~cm appears to be enough to measure the \efield with the granularity needed for a good position reconstruction precision. In fact, since the effects that can likely cause bigger \efield distortions are the problems or alignments in the CPA (or APA), or in the FC, it could be conceivable to have different size voxels for different regions, saving the highest granularity of the probing for the walls/edges of the drift volume.

\begin{comment}
\begin{dunetable}
[Calibration Requirements]
{p{0.5\textwidth}p{0.15\textwidth}p{0.15\textwidth}}
{tab:calibreq}
{Calibration Specifications and Goals}   
Requirement & Specification & Goal \\ \toprowrule
\efield measurement precision & < 1\% & ALARA \\ \colhline
\efield measurement coverage & > 75\% & AHARA \\ \colhline
\efield measurement granularity & < 30x30x30 cm & ALARA \\ \colhline
\end{dunetable}
\end{comment}



\subsubsection{Design}
\label{sec:sp-calib-sys-las-ion-des}
%\paragraph{Baseline design}

The design of the laser calibration system for \dword{dune} is largely based on the design of the system built for \dword{microboone}~\cite{microboone}, that in turn was based on several previous developments~\cite{Rossi:2009im,Zeller:2013sva,Ereditato:2014lra,Ereditato:82014tya}. A similar system was also built for CAPTAIN~\cite{Berns:2013usa} and in the near future, will be built for \dword{sbnd}~\cite{Antonello:2015lea}. Operation of the \dword{microboone} system has already taken place. A preliminary report was given in~\cite{bib:chen2018} and more details will be available in~\cite{bib:uBlaser2019}.
%\todo{link the reference once uB publishes the laser paper in 2019}

Ionization of \dword{lar} by laser can occur via a multiphoton process in which a two-photon absorption~\cite{Badhrees:2010zz} leads the atom to the excited states band, and a third photon can cause ionization. This can only occur with high photon fluxes, and so the employed lasers need to be pulsed and have pulse energies of \SI{60}{\milli\joule} or more. Contrary to muons, the laser beams do not suffer multiple scattering and travel along straight lines determined by the steering mirror optics. The basic measurement consists in %recording the laser beams 
generating laser ionization tracks in the TPC and comparing the reconstructed tracks with the direction known from the steering hardware. 
An apparent curvature of the measured track is attributed to drift velocity (and therefore \efield) distortions (either in direction or magnitude).

The first step in the analysis~\cite{bib:uBlaser2019} is to obtain a field of position displacements from the comparison of the known and reconstructed tracks. If two crossing tracks are used, the displacement vector is simply given by the vector connecting the point where the reconstructed tracks cross and the point where the known tracks cross. However, since those displacements can vary both in direction and magnitude, there will be ambiguity in that determination if only one track is used in a given spatial region. An iterative procedure was developed by the MicroBooNE collaboration~\cite{bib:chen2018,bib:uBlaser2019} in order to 
%still 
obtain a displacement map from a set of several non-crossing tracks from opposite directions. Following this, a set of ``drift velocity field lines'' (that are the same as \efield lines) can be obtained from the displacement map, assuming that all charge deposits along a field line will be collected in the same position. Using the relationship between \efield and drift velocity~\cite{Li:2015rqa,Walkowiak:2000wf}, one can then also obtain 
%as well 
the magnitude of the \efield.

%An unambiguous field map requires crossing laser tracks in every relevant "voxel" of the detector. If two tracks that enter the same spatial voxel ($10 \times 10 \times 10 ~\textrm{cm}^3$ volume) in the \dword{detmodule}, the relative position of the tracks provides an estimate of the local \threed \efield.

%With a single, steerable laser track, there would be ambiguity in the direction/magnitude of the position displacement and so the information obtained would be limited. Even if not crossing, a set of several tracks from opposite directions can still be used to obtain a displacement map via an iterative procedure~\cite{bib:chen2018,bib:uBlaser2019}.

%The laser track location placement precision is limited solely by the optics design and self-focusing effects can impact the practical range of the track (theoretically, the Rayleigh scattering length of 266~nm light is about 40~m). As discussed in Section~\ref{sec:FTs}, this is mitigated by spacing lasers about 15~m apart which is very close to the laser track range demonstrated by the MicroBooNE experiment. 
 
%\fixme{Clarify the extra degeneracy in this case, related to overall parameterization of TPC response model?}
%\fixme{Need to reference optics of system?}

%\fixme{Add a comment on how long a run takes} 

%A \phel{}-based calibration system was used in the T2K gaseous (predominantly Ar), TPCs~\cite{Abgrall:2010hi}. %Targets placed on the cathode provided dots and lines that were then imaged by the electronics, and relative distortions of the surveyed positions could be used. 
%Thin metal surfaces placed at surveyed positions on the cathode provided point-like and line sources of \phel{}s when illuminated by a laser. The T2K \phel system provided measurements of adjacent electronics modules' relative timing response, drift velocity with few \si{\nano\s} resolution of \SI{870}{\milli\m} drift distance, electronics gain, transverse diffusion, and an integrated measurement of the electric field along the drift direction. For DUNE, the system would be similarly used as on T2K to diagnose electronics or TPC response issues on demand, and provide an integral field measurement and relative distortions of $y$, $z$ positions with time, and of either $x$ or drift velocity. Ejection of \phel{}s from the direct ionization laser system has also been observed, so it is likely this is a reasonable addition to the nominal design.% and would only be considered a primary system if the intensity of the laser is problematic. 

%% JM %% from early IDR version


% Mini workshop: https://indico.fnal.gov/event/14909/

%Laser beams with lengths of \SI{10}{\m} in \dword{lar} have been observed in \dword{microboone}, and beams with \SI{20}{\m} (possibly more) are reasonably expected to be possible to obtain with a similar system. 

While the Rayleigh scattering length for \SI{266}{\nano\m}  light is about \SI{40}{\m}, additional optics effects may limit the maximum practical range of laser beams of that wavelength to a distance smaller than that. Those can include the Kerr effect,  due to the dependency of the refractive index on the \efield. In the presence of an intense field, such as that caused by the laser beam itself, the change in refractive index can lead to lensing, or focusing, that distorts the coherence of the beam\footnote{The Kerr effect is so far believed to be the cause of non-homogeneity of the ionization along the laser beam observed in \dword{microboone}, that prevents the use of the charge information. Its impact on the position measurement and \efield uncertainty has been studied by \dword{microboone}.}. 
%Still, 
Despite this, laser beams with lengths of \SI{10}{\m} in \dword{lar} have been observed in \dword{microboone}, and beams with \SI{20}{\m} lengths (possibly more) are reasonably expected to be possible to obtain with a similar system.
This has determined the choice of locating five calibration ports in the cryostat roof at \SI{15}{\m} intervals along each of the four drift volumes of the SP module, for a total of \num{20} ports. In fact, there are four ports just outside each of the \dword{fc} end-walls, and \num{12} ports located over the top \dword{fc}, close to the \dword{apa} of each drift volume, as shown in Figure~\ref{fig:FTmap}.



\begin{dunefigure}[\dword{microboone} laser calibration system schematics]{fig:uB_laser_schematic}
{Left: Schematics of the ionization laser system in one port~\cite{Antonello:2015lea}. Right: Schematics of the laser box~\cite{microboone}.}
\includegraphics[width=0.45\linewidth]{uB_laser_schematic}
\includegraphics[width=0.5\linewidth]{uB_laser_box}
\end{dunefigure}



For each of those \num{20} ports, a laser module can be schematically represented by Figure~\ref{fig:uB_laser_schematic} (left), and consists of the following elements:
\begin{itemize}
    \item A laser box (see Figure \ref{fig:uB_laser_schematic}; right) that provides
    \begin{itemize}
        \item An attenuator and a collimator to control the intensity and size of the beam;
        \item A photodiode that gives a TPC-independent trigger signal;
        \item A low-power red laser, aligned with the UV one, to facilitate alignment operations;
        \item A Faraday cage to shield the surrounding electronics from the accompanying EM pulse.
    \end{itemize}
    \item A feedthrough (see Figure \ref{fig:uB_laser_ft}; left) into the cryostat that provides:
    \begin{itemize}
        \item The optical coupling that allows the UV light to pass through into the cryostat directly into the liquid phase, avoiding distortions due to the gas-liquid interface and the gas itself;
        \item A rotational coupling that allows the whole structure to rotate while maintaining the cryostat seal;
        \item A periscope structure (see Figure~\ref{fig:uB_laser_ft}; Right) mounted under the rotating coupling that supports a mirror within the \dword{lar};
        \item The additional theta rotation of the mirror accomplished by a precision mechanism coupled to an external linear actuator;
        \item Both the rotation and linear movements of the steering mechanism read out by precision encoders.
    \end{itemize}
    
\end{itemize}


\begin{dunefigure}[\dword{microboone} laser calibration system drawings]{fig:uB_laser_ft}
{CAD drawings of the \dword{microboone} laser calibration system~\cite{microboone}. Left: calibration port feedthrough. Right: laser beam periscope. %Both figures from~\cite{microboone}.
}
\includegraphics[width=0.49\linewidth]{uB_laser_ft}
\includegraphics[width=0.248\linewidth]{uB_laser_periscope}
\end{dunefigure}


In the case of the lasers in the end-wall ports, the beams enter the \dword{fc} laterally, while in the case of the lasers in the ports over the TPC, the beams enter the TPC from the top. In both cases, the laser beam can enter the \dword{fc} only through the gaps between the \dword{fc} electrodes. These gaps are \SI{1.4}{\cm} wide and the electrodes themselves are \SI{4.6}{\cm} wide, so it's clear that the shadowed regions are very significant. Section~\ref{sec:sp-calib-laser-alter} describes the alternative designs currently being developed for the laser system. In one of the alternative designs, the top \dword{fc} is modified as to allow small openings for the bottom of the periscope to penetrate %within 
the \dword{fc} (not the ground plane), significantly increasing coverage.

For the six most central ports, the distance between them is small enough that we can consider having the same laser box serving two feedthroughs, in order to reduce the costs associated with the laser and its optics.

%a \SI{266}{\nano\m} laser would be mounted on the top of the cryostat, and service two adjacent feedthroughs. A steerable head and fiber interface would be mounted in the feedthrough, which is coated in a insulator. Two options are under investigation: (1) the \dword{fc} (but not the \dword{gp}) is penetrated, and (2) the \dword{fc} is not penetrated. In the former case, the \dword{fc} penetration has been shown to create a small distortion to the E-field, for the benefit of full volume E-field mapping. When the \dword{fc} is not penetrated, the laser shines through the \dword{fc} tubes, producing some regions that are not mappable by the laser. Unlike the ports that are inwards of the cryostat, the lasers through penetrations that are outside the \dword{fc} on the far east and west side of the cryostat will not penetrate the field cage. The photo-electron system would include a fiber and no steering; the necessity of penetrating the \dword{fc} is unlikely but has not been assessed yet.

A scan of the full detector using \SI{1}{L} volume elements would require a number of tracks on the order of \num{8e5}, 
%would 
and can take about three days. It is expected that shorter runs could be done to investigate specific regions. The sampling granularity, and therefore the amount of data taken, is dependent on \dword{daq} requirements. In fact, even to be able to record the desired \num{8e5} tracks, a dedicated data reduction algorithm will have to be devised, so that only a drift window of about \SI{100}{\micro\s}
of data is recorded, and the position of that window depends on the beam position and direction and which wires are being read out. More details on this are given in section~\ref{sec:sp-calib-daqreq}.

%The direct ionizing laser system may also be used to create \phel{}s from the cathode, even under low power operation.

%% JM %% end % from early IDR version




%%%%%%%%%%%%%%%
%\subsubsubsection{Possible Measurements}
\subsubsection{Measurement Program}
\label{sec:sp-calib-sys-las-ion-meas}


The method for \efield measurement is based on the measurement of position displacements. The laser produces straight tracks in a known position and deviations from that seen in reconstructed tracks are attributed to \efield distortions. Therefore the precision with which the \efield distortions can be measured depends on the precision with which we can know the laser track position and the TPC position reconstruction precision.
The TPC precision in the $y$, $z$ coordinates is given primarily by the wire spacings of \uvpitch and \xgpitch. It is slightly better than that (about \SI{2}{\milli\m}) on the $x$ coordinate, determined by the \fepeaktime peaking time of the front-end electronics.
Given perfect laser positioning accuracy, the smallest measurable \efield distortions would be those that cause displacements of this magnitude --- \SI{2}{\milli\m} in $x$ and \SI{5}{\milli\m} in $y$, $z$. The precision on the drift velocity distortions depends on the size of the spatial region where they are present. For distortions present in regions of \SI{0.5}{\m} and larger, drift velocity distortions can therefore be measured with an accuracy of \SI{1}{\%} in $y$, $z$ and \num{0.4}\% in $x$. In $y$, $z$, \SI{1}{\%} precision on drift velocity distortions translates to a \SI{1}{\%} precision on the transverse field distortions. Along $x$, one must consider that, at \spmaxfield, a \SI{1}{\%} change in \efield leads to a \SI{0.375}{\%} change in drift velocity. Therefore finally, this means that the smallest measurable distortions given the TPC design (wire pitch, timing precision) are of \SI{1}{\%} in \efield if they are present in regions of \SI{0.5}{\m} and larger (smaller field distortions could be in principle be measurable if they are present over larger regions, so that their effect accumulates over the drift path).
On one side, this gives us an ultimate limit to the \efield precision achievable with the laser system, but on the other side, since these TPC precision considerations apply to physics events also, it tells us that an \efield precision much better than \SI{1}{\%} should not have an impact on physics.

In principle, if we were confident about the field in one detector region and would like to probe another, we could use tracks that cross both regions and use the TPC measurements in the ''good'' region as the ``true'' track direction, without needing the hardware information on the mirror angles, etc. But in a general case, the TPC precision is only one of the components of the laser measurement precision, the other being the mechanical beam positioning accuracy. The goal of the mechanical design of the system is to achieve a precision close to that of the TPC measurements, so that no single factor is dominant in the overall systematics. The starting point of the laser beams is given by the position of the mirror in the periscope, that is known from construction drawings and cool down calculations. Warm surveys might be necessary. The angle of the beam is given by the angles ($\theta$, $\phi$) of the mirror, that are set by the periscope motors and read-out by the encoders. 
Reference~\cite{bib:chen2018} quotes a mechanical precision of \SI{0.05}{\mrad} for the \dword{microboone} system, for both angles. At \SI{10}{\m}, the maximum distance in \dword{microboone}, 
%that's \SI{0.5}{\milli\m}.
which results in \SI{0.5}{\milli\m} precision. In DUNE, we 
%count on having 
expect \SI{20}{\m} long beams, so the precision is \SI{1}{\milli\m} 
%at that distance, if we equal 
using the precision of the \dword{microboone} system. The beam itself is wider than that. In fact, with a \SI{0.5}{\mrad} divergence, we expect the beam to be \SI{1}{\cm} wide at \SI{20}{\m}. The profile is Gaussian, so the centroid of the charge creation should be more accurate. During cool down, there can be shifts that need to be measured and corrected for, so we aim to have a system that can measure the beam position in a few positions, at least one per drift volume and laser beam. Our goal is to provide the position of the beam to an accuracy of \SI{5}{\milli\m} with positioning systems located at about \SI{10}{\m} from the beam origin. A laser positioning system is described in the next section.

 
 %% JM %% discuss how to get the \efield from the measurements even if the absolute knowledge of the laser track is not perfect.
 % - need for crossing tracks
 % - use positioning system
 % - do relative measurements between tracks close by. actual advantage of the alternative system: many parallel tracks at different x
 
 % also, estimate time needed and discuss idea of having the photo-laser for a quick/rough monitoring, and the ionization laser as a detailed probe of regions identified as possibly problematic.


%The remaining studies for the laser systems to be done prior to the \dword{tdr} are: 
%\begin{itemize}
%\item Determine a nominal design for photoelectric thin metal surfaces on the cathode. A survey in cold conditions is not possible for the \single system, and the photoelectric system could provide both known positions in the detector and information complementary to a survey or cosmic data.
%\item For the \dual system, quantify the additional benefit of a photoejection system since it will be possible to survey the \dwords{crp} externally under cold conditions.
%\item  Determine whether the known classes of possible \efield distortions warrant a mechanical penetration of the \dword{fc} (versus reduced sampling from projecting laser light inward between \dword{fc} elements) and further understand sensitivity of the laser to realistic \efield distortions. 
%\item Continue to study the range of possible \efield distortions in order to further refine the estimation of overall variation of the \efield  in the \dword{detmodule}. 
%\end{itemize}
%\begin{itemize}
%\item Determine  a nominal design for photoelectric targets on the cathode, and whether  such targets would provide sufficient survey-like information.
%\item Determine if the known classes of possible \efield distortions require penetration of the \dword{fc} (versus reduced sampling from shining between the field cage). 
%\item Further understand limitations on laser location precision, practical range of propagation due to optics design and Rayleigh scattering. \fixme{KM: Is this too vague to be helpful or sets us up for failure? What specifically will be studied?}
%\item Continue to quantify the range of possible \efield distortions in the DUNE FD to further refine the estimation of overall variation of \efield (both locally and globally) in the detector.
%\end{itemize}

%\fixme{}

