

Interfaces between calibration and other consortia have been identified and appropriate documents are being developed that will be maintained in DUNE DocDB
%\todo{documents being prepared now}
. A brief summary is provided in this section. Table~\ref{tab:fdgen-calib-interfaces} lists the 
interfaces and corresponding DocDB documents where available. 
The main interfacing systems are \dword{hv}, \dword{pds} and \dword{daq} and the main issues that need to be considered are listed below.

\begin{description}
    \item[HV] Evaluate the effect of the calibration hardware, especially the laser system periscopes, on the \efield, even in case of no penetration of the \dword{fc}; \dword{fc} penetrations for laser. Evaluate the effect of the incident laser beam on the CPA material (kapton); Integrate the hardware of the %alternative 
    photoelectron laser system (targets) and the laser positioning system (diodes) within the \dword{hv} system components. Ensure the radioactive source deployment is in a safe field region and cannot do mechanical harm to the field cage.
    \item[PDS] Evaluate long term effects of laser light, even if just diffuse or reflected, on the scintillating components (TPB plates) of the PDS; Establish a laser run plan to avoid direct hits; Evaluate the impact of laser light on alternative PDS ideas, such as having reflectors on the CPAs. Validate light response model and triggering for low energy signals.
    \item[DAQ] Evaluate DAQ constraints on the total volume of calibration data that can be acquired, and develop strategies to maximize the data taking efficiency with data reduction methods; Study how to implement a way for the calibration systems to receive trigger signals from DAQ, in order to maximize supernova live time.
    %\item[Computing] Evaluate any additional semi-offline processing needs. Coordinate needs for calibration databases. 
\end{description}

Integrating and installing calibration devices will interfere with other devices and must be coordinated with the appropriate consortia as needed. Similarly, calibration will have significant interfaces at multiple levels with facilities to coordinate resources for assembly, integration, installation and commissioning (e.g. networking, cabling, safety etc.). Rack space distribution and interaction between calibration and other modules from other consortia will be managed by \dword{tc} in consultation with those consortia.

\begin{dunetable}
[Calibration system interface links]
{p{0.14\textwidth}p{0.40\textwidth}p{0.14\textwidth}}
{tab:fdgen-calib-interfaces}
{Calibration Consortium interface links}   
\small
Interfacing System & Description & Reference \\ \toprowrule
\dword{hv}	&
effect of calibration hardware (laser and radioactive source) on \efield and field cage; laser light effect on \dword{cpa} materials, field cage penetrations; attachment of positioning targets to HV supports 
& --- 
\\ \colhline
\dword{pds}	& 
effect of laser light on \dword{pds}, reflectors on the \dword{cpa}s (if any); validation of light response and triggering for low energy signals 
& ---  
\\ \colhline
\dword{daq}	& 
DAQ constraint on total volume of the calibration data; receiving triggers from DAQ
& ---  
\\ \colhline
\dword{cisc} &
multi-functional \dword{cisc}/calibration ports; space sharing around ports; fluid flow validation; slow controls and monitoring for calibration quantities 
& \citedocdb{7072} 
\\ \colhline
TPC Electronics	         &  
Noise, electronics calibration
& ---  
\\ \colhline
\dword{apa}	&
\dword{apa} alignment studies using laser and impact on calibrations
& --- 
\\ \colhline
Physics	&
tools to study impact of calibrations on physics
& --- 
\\ \colhline
Software \& Computing	  &
Calibration database design and maintenance
& ---
\\ \colhline
TC Facility              &   
Significant interfaces at multiple levels   
& ---   \\ \colhline
TC Installation     	  &     
Significant interfaces at multiple levels
& ---   \\ \colhline
%TC Integration Facility    &    
%Significant interfaces at multiple levels
%& ---   \\ 
\end{dunetable}

%\fixme{sample from HV - use as template; see Sec 3.5 of guidance doc}

%\begin{dunetable}
%[High Voltage System Interface Links]
%{p{0.4\textwidth}p{0.2\textwidth}}
%{tab:HVinterfaces}
%{High Voltage System Interface Links }   
%Interfacing System & Linked Reference \\ \toprowrule
%CISC & \href{http://docs.dunescience.org/cgi-bin/ShowDocument?docid=6787&version=1}{DocDB 6787} \cite{bib:docdb6787} \\ \colhline
%DP CRP & \cite{bib:docdb6754} \\ \colhline
%... & \cite{} \\ \colhline
%(last item)& \cite{} \\
%\end{dunetable}