
%\fixme{SG, KM: Done; JM please check. }
%To-DO: Need more information on requirements from neutron source and RA source, once we have it, we can update the text/table as needed.}

Some common design considerations for calibration devices include stability, reliability, and longevity, so calibration systems can be operated for the lifetime of the experiment (\dunelifetime). Such longevity is uncommon for any device, so the overall design permits replacing devices where possible, namely the parts that are external to the cryostat. The systems must also adhere to relevant global requirements of the \dword{dune} detector. Table~\ref{tab:specs:SP-CALIB} shows the top-level overall requirements for calibration subsystems along with global \dword{dune} requirements that are relevant for calibration. For example, \dword{dune} requires the \efield  on any instrumentation devices inside the cryostat to be less than 30 kV/cm to minimize the risk of dielectric breakdown in \dword{lar}. Another consideration important for reconstructing events is the maximum noise level the readout electronics can tolerate from calibration devices. \dword{pdsp} is evaluating this. In Table~\ref{tab:specs:SP-CALIB}, two values are quoted for most of the parameters: i) {\it specification}, which is the minimum requirement to guarantee baseline performance, and ii) {\it goal}, an ideal requirement enabling more detailed studies and for achieving improved precision.

For the ionization laser system, the energy and position reconstruction requirements for physics measurements lead to requirements for the necessary precision in measuring the \dword{tpc} \efield as well as its spatial coverage and granularity. The precision of the \efield measurement with the laser system must be about \SI{1}{\%} so that the effect from \efield on the collected charge, via the dependence of the recombination factor on \efield, is well below \SI{1}{\%}. This is also motivated by consistency with the high level \dword{dune} specification of \SI{1}{\%} on field uniformity throughout the volume for component alignment and the \dword{hv} system. For laser coverage, to keep the \efield measurement at the $\sim$\SI{1}{\%} level, we are aiming for a coverage of \SI{75}{\%} or more of the total \dword{fv}. The requirement on granularity for the laser is estimated based on the \dword{fv} uncertainty requirements (\SI{1}{\%}) and corresponding uncertainty requirements (\SI{1.5}{\cm}) in each coordinate. A specification is set for a voxel size of \num{30}$\times$\num{30}$\times$\SI{30}{\cubic\cm}, that should be sufficient to satisfy the \dword{fv} uncertainty requirements. A goal is set for \num{10}$\times$\num{10}$\times$\SI{10}{\cubic\cm}, which could allow for a refinement in precision in some detector regions. 

The laser beam location must also meet the level of reconstruction requirement in each coordinate, approximately \SI{5}{\milli\m}. In order to reach that over distances of up \SI{20}{\m}, where the latter is the maximum distance that any beam needs to travel to cover all detector voxels, this results in a stringent requirement of \ang{0.015} (or \SI{0.25}{\mrad}) on the pointing precision. The laser beam location system is also designed to check the beam location with a precision of \SI{5}{\milli\m} over distances of up \SI{20}{\m}.
%\SI{5}{\milli\m} over \num{5} to \SI{10}{\m}, where the latter is the distance between two consecutive laser ports in the beam direction. This results in a stringent requirement of \ang{0.03} (or \SI{0.5}{\mrad}).
The data volume for the ionization laser system must be at least \num{184}~TB/year/\SI{10}{\kt}, assuming \num{800}k laser pulses, \num{10}$\times$\num{10}$\times$\SI{10}{\cubic\cm} voxel sizes, a \SI{100}{\micro\s} zero suppression window, and two dedicated calibration campaigns per year.

For the \dword{pns} system, the system must provide sufficient neutron event rate to make spatially separated precision measurements across the detector of a comparable size to the voxels probed by the laser (\num{30}$\times$\num{30}$\times$\SI{30}{\cubic\cm}) for most regions of the detector (\SI{75}{\%}). 
%The laser and \dword{pns} systems are probing different detector and response parameters, so using one system to propagate the other's calibration to the rest of the detector is not possible.
% 1st draft
%For the supernova program, measurements from the \dword{pns} \fixme{This abbreviation is in neither glossary.} should demonstrate 1\% energy scale, 5\% energy resolution, and 0.5 MeV detection threshold, so each voxel should have sufficient neutron event rate to achieve this. %\todo{KM: Improve or remind connection to SN program? even though it's comparable? SG: maybe for 2nd draft?}
%rewritten for 2nd draft
For the \dword{snb} program, the sensitivity to distortions of the neutrino energy spectrum depends on the uncertainties in the detection threshold and the reconstructed energy scale and resolution. Studies discussed in the physics \dword{tdr} present target ranges for the uncertainties in these parameters~\cite{bib:docdb14068} as a function of energy. The measurements with the \dword{pns} system aim to provide response corrections and performance estimates, so those uncertainty targets are met throughout the whole volume. This ensures that each voxel has sufficient neutron event rate (percent level statistical uncertainty).

%\fixme{Insert correct reference to physics TDR ch7}
%\fixme{Put PNS in glossary and use that.}


In terms of data volume requirements, the \dword{pns} system requires at least \num{144}~TB/year/\SI{10}{\kt} assuming \num{e5} neutrons/pulse, \num{100} neutron captures/\si{\cubic\m},
%m$^{3}$ 
and \num{130} observed neutron captures per pulse, and two calibration runs per year. 

Table~\ref{tab:fdgen-calib-all-reqs} shows the full set of requirements related to all calibration subsystems. More details on each of the requirements can be found under corresponding consortia.   

%% This file is generated, any edits may be lost.
\begin{footnotesize}
%\begin{longtable}{p{0.14\textwidth}p{0.13\textwidth}p{0.18\textwidth}p{0.22\textwidth}p{0.20\textwidth}}
\begin{longtable}{P{0.12\textwidth}P{0.18\textwidth}P{0.17\textwidth}P{0.25\textwidth}P{0.16\textwidth}}
\caption{Specifications for SP-CALIB \fixmehl{ref \texttt{tab:spec:SP-CALIB}}} \\
  \rowcolor{dunesky}
       Label & Description  & Specification \newline (Goal) & Rationale & Validation \\  \colhline

   \newtag{SP-CALIB-1}{ spec:efield-calib-precision }  & Ionization laser electric field measurement precision  &  \SI{1}{\%} \newline ( $<$\SI{1}{\%} ) &  Electric field affects energy and position measurements. &  ProtoDUNE and external experiments. \\ \colhline
     % 1
   \newtag{SP-CALIB-2}{ spec:efield-calib-coverage }  & Ionization laser electric field measurement coverage  &  $>$\SI{75}{\%} \newline ( \SI{100}{\%} ) &  The necessary coverage depends on how high a distortion can be reasonably expected. &  ProtoDUNE \\ \colhline
     % 2
   \newtag{SP-CALIB-3}{ spec:efield-calib-granularity }  & Ionization laser \efield measurement  granularity  &  $~30\times 30\times 30~$\SI{}{\centi\metre\cubed} \newline ($10\times 10\times 10~$\SI{}{\centi\metre\cubed}) &  Minimum measurable region is set by the maximum expected distortion and position reconstruction requirements. &  ProtoDUNE \\ \colhline
     % 3
   \newtag{SP-CALIB-4}{ spec:laser-position-precision }  & Laser beam position precision  &  \SI{0.5}{\milli\rad} \newline ( $<\,\SI{0.5}{\milli\rad}$ ) &  The necessary spatial precision does not need to be smaller than the APA wire gap. &  ProtoDUNE \\ \colhline
     % 4
   \newtag{SP-CALIB-5}{ spec:neutron-source-coverage }  & Neutron source coverage  &  $>$\SI{75}{\%} \newline ( \SI{100}{\%} ) &  The coverage of the pulsed neutron system depends on the energy resolution requirements at low energy. &  Simulations \\ \colhline
     % 5
   \newtag{SP-CALIB-6}{ spec:data-volume-laser }  & Ionization laser DAQ rate per year (per 10 kton)  &  $>$\SI{84}{TB/yr/10 kton} \newline ( $>$\SI{185}{TB/yr/10 kton} ) &  The laser data volume must allow the needed coverage and granularity. &  protoDUNE/simulations \\ \colhline
     % 6
   \newtag{SP-CALIB-7}{ spec:data-volume-pns }  & Neutron source DAQ rate per year (per 10 kt)  &  $>\,\SI{84}{TB/yr/10 kt}$ \newline ( $>\,\SI{168}{TB/yr/10 kt}$ ) &  The pulsed neutron system must allow the needed coverage and granularity. &  Simulations \\ \colhline
     % 7


\label{tab:specs:just:SP-CALIB}
\end{longtable}
\end{footnotesize}

% This file is generated, any edits may be lost.

\begin{longtable}{p{0.14\textwidth}p{0.13\textwidth}p{0.18\textwidth}p{0.22\textwidth}p{0.20\textwidth}}
\caption{Specifications for SP-CALIB \fixmehl{ref \texttt{tab:spec:SP-CALIB}}} \\
  \rowcolor{dunesky}
       Label & Description  & Specification \newline (Goal) & Rationale & Validation \\  \colhline

    \\ \rowcolor{dunesky} \newtag{SP-FD-1}{ spec:min-drift-field } & Name: Minimum drift field \\
    Description & The drift field in the TPC shall be greater than 250 V/cm, with a goal of 500 V/cm.   \\  \colhline
    Specification (Goal) &  $>$\,\SI{250}{ V/cm}  ( $>\,\SI{500}{ V/cm}$ ) \\   \colhline
    Rationale &   Lessens impacts of e-Ar recombination, e-lifetime, e-diffusion and space charge.  \\ \colhline
    Validation & ProtoDUNE  \\
   \colhline

    \\ \rowcolor{dunesky} \newtag{SP-FD-2}{ spec:system-noise } & Name: System noise \\
    Description & The total system noise seen by each wire should be no more than 1000 enc of noise. It is expected that random noise on the FE amplifier will be the dominant contribution to the total system noise.   \\  \colhline
    Specification &  $<\,\SI{1000}{enc}$ \\   \colhline
    Rationale &   Provides $>$5:1 S/N on induction planes for  pattern recognition and two-track separation.  \\ \colhline
    Validation & ProtoDUNE and simulation  \\
   \colhline

    \\ \rowcolor{dunesky} \newtag{SP-FD-3}{ spec:light-yield } & Name: Light yield \\
    Description & The light yield shall be sufficient for measuring event time (and total intensity) of events with visible energy above 200 MeV.  Goal is to make possible a 10\% energy measurement for events with a visible energy of 10 MeV.   \\  \colhline
    Specification (Goal) &  $>\,\SI{0.5}{pe/MeV}$  ( $>\,\SI{5}{pe/MeV}$ ) \\   \colhline
    Rationale &   Rejects nucleon decay backgrounds from cosmogenic events near cathode.  \\ \colhline
    Validation &   \\
   \colhline

    \\ \rowcolor{dunesky} \newtag{SP-FD-4}{ spec:time-resolution-pds } & Name: Time resolution \\
    Description & The time resolution of the photon detection system shall be less than 1 microsecond in order to assign a unique event time.   \\  \colhline
    Specification (Goal) &  $<\,\SI{1}{\micro\second}$  ( $<\,\SI{100}{\nano\second}$ ) \\   \colhline
    Rationale &   Enables \SI{1}{mm} position resolution for \SI{10}{MeV} SNB candidate events for instantaneous rate $<\,\SI{1}{m^{-3}ms^{-1}}$.  \\ \colhline
    Validation &   \\
   \colhline

   \newtag{SP-FD-5}{ spec:lar-purity }  & Liquid argon purity  &  $<$\,\SI{100}{ppt} \newline ($<\,\SI{30}{ppt}$) &  Provides $>$5:1 S/N on induction planes for  pattern recognition and two-track separation. &  Purity monitors and cosmic ray tracks \\ \colhline
    
    \\ \rowcolor{dunesky} \newtag{SP-FD-7}{ spec:misalignment-field-uniformity } & Name: Drift field uniformity due to component alignment \\
    Description & Misalignments of the various TPC components shall not introduce drift-field nonuniformities beyond those specified in the HVS requirements.   \\  \colhline
    Specification &  $<\,1\,$\% throughout volume \\   \colhline
    Rationale &   Maintains APA, CPA,  FC orientation and shape.  \\ \colhline
    Validation & ProtoDUNE  \\
   \colhline

   
  \newtag{SP-FD-9}{ spec:apa-wire-spacing }  & APA wire spacing  &  \SI{4.669}{mm} for U,V; \SI{4.790}{mm} for X,G &  Enables 100\% efficient MIP detection, \SI{1.5}{cm} $yz$ vertex resolution. &  Simulation \\ \colhline
    
    \\ \rowcolor{dunesky} \newtag{SP-FD-11}{ spec:hvs-field-uniformity } & Name: Drift field uniformity due to HVS \\
    Description & Design of TPC cathode and FC components shall ensure uniform field.  Production tolerances shall be set so as to maintain flatness of component surfaces and, by extension, the shape of the drift field volume.   \\  \colhline
    Specification &  $<\,\SI{1}{\%}$ throughout volume \\   \colhline
    Rationale &   High reconstruction efficiency.  \\ \colhline
    Validation & ProtoDUNE and simulation  \\
   \colhline

    \\ \rowcolor{dunesky} \newtag{SP-FD-13}{ spec:fe-peak-time } & Name: Front-end peaking time \\
    Description & The FE peaking time shall be set so as to optimize vertex resolution.    \\  \colhline
    Specification (Goal) &  \SI{1}{\micro\second}  ( Adjustable so as to see saturation in less than \SI{10}{\%} of beam-produced events ) \\   \colhline
    Rationale &   Vertex resolution; optimized for \SI{5}{mm} wire spacing.  \\ \colhline
    Validation & ProtoDUNE and simulation  \\
   \colhline

   
  \newtag{SP-FD-16}{ spec:det-dead-time }  & Detector dead time  &  $<\,\SI{0.5}{\%}$ &  Meet physics goals in timely fashion. &  ProtoDUNE \\ \colhline
    
    \\ \rowcolor{dunesky} \newtag{SP-FD-22}{ spec:data-rate-to-tape } & Name: Data rate to tape \\
    Description & The DAQ shall provide capability for triggering on events of interest in order to limit the total volume for events stored on tape.   \\  \colhline
    Specification &  $<\,\SI{30}{PB/year}$ \\   \colhline
    Rationale &   Cost.  Bandwidth.  \\ \colhline
    Validation & ProtoDUNE  \\
   \colhline

   
  \newtag{SP-FD-23}{ spec:sn-trigger }  & Supernova trigger  &  $>\,\SI{90}{\%}$ efficiency for SNB within \SI{100}{kpc} &  $>\,$90\% efficiency for SNB within 100 kpc &  Simulation and bench tests \\ \colhline
    
   
  \newtag{SP-FD-24}{ spec:local-e-fields }  & Local electric fields  &  $<\,\SI{30}{kV/cm}$ &  Maximize live time; maintain high S/N. &  ProtoDUNE \\ \colhline
    
   
  \newtag{SP-FD-25}{ spec:non-fe-noise }  & Non-FE noise contributions  &  $<<\,\SI{1000}\,e^- $ &  High S/N for high reconstruction efficiency. &  Engineering calculation and ProtoDUNE \\ \colhline
    
   
  \newtag{SP-FD-26}{ spec:lar-impurity-contrib }  & LAr impurity contributions from components  &  $<<\,\SI{30}{ppt} $ &  Maintain HV operating range for high live time fraction. &  ProtoDUNE \\ \colhline
    
   
  \newtag{SP-FD-27}{ spec:radiopurity }  & Introduced radioactivity  &  less than that from $^{39}$Ar &  Maintain low radiological backgrounds for SNB searches. &  ProtoDUNE and assays during construction \\ \colhline
    

   \newtag{SP-CALIB-1}{ spec:efield-calib-precision }  & Ionization laser electric field measurement precision  &  \SI{1}{\%} \newline ( $<$\SI{1}{\%} ) &  Electric field affects energy and position measurements. &  ProtoDUNE and external experiments. \\ \colhline
    
   \newtag{SP-CALIB-2}{ spec:efield-calib-coverage }  & Ionization laser electric field measurement coverage  &  $>$\SI{75}{\%} \newline ( \SI{100}{\%} ) &  The necessary coverage depends on how high a distortion can be reasonably expected. &  ProtoDUNE \\ \colhline
    
   \newtag{SP-CALIB-3}{ spec:efield-calib-granularity }  & Ionization laser \efield measurement  granularity  &  $~30\times 30\times 30~$\SI{}{\centi\metre\cubed} \newline ($10\times 10\times 10~$\SI{}{\centi\metre\cubed}) &  Minimum measurable region is set by the maximum expected distortion and position reconstruction requirements. &  ProtoDUNE \\ \colhline
    
   \newtag{SP-CALIB-4}{ spec:laser-position-precision }  & Laser beam position precision  &  \SI{0.5}{\milli\rad} \newline ( $<\,\SI{0.5}{\milli\rad}$ ) &  The necessary spatial precision does not need to be smaller than the APA wire gap. &  ProtoDUNE \\ \colhline
    
   \newtag{SP-CALIB-5}{ spec:neutron-source-coverage }  & Neutron source coverage  &  $>$\SI{75}{\%} \newline ( \SI{100}{\%} ) &  The coverage of the pulsed neutron system depends on the energy resolution requirements at low energy. &  Simulations \\ \colhline
    
   \newtag{SP-CALIB-6}{ spec:data-volume-laser }  & Ionization laser DAQ rate per year (per 10 kton)  &  $>$\SI{84}{TB/yr/10 kton} \newline ( $>$\SI{185}{TB/yr/10 kton} ) &  The laser data volume must allow the needed coverage and granularity. &  protoDUNE/simulations \\ \colhline
    
   \newtag{SP-CALIB-7}{ spec:data-volume-pns }  & Neutron source DAQ rate per year (per 10 kt)  &  $>\,\SI{84}{TB/yr/10 kt}$ \newline ( $>\,\SI{168}{TB/yr/10 kt}$ ) &  The pulsed neutron system must allow the needed coverage and granularity. &  Simulations \\ \colhline
    
   
  \newtag{SP-CALIB-8}{ spec:rate-gammas-source }  & Rate of 9 MeV capture gamma events in the proposed radioactive source  &  $<\,\SI{1}{\kilo\hertz}$ &  The source rate must be such that there is no more than one event per drift time. &  Lab tests \\ \colhline
    


\label{tab:specs:SP-CALIB}
\end{longtable}

\begin{comment}
%comment the old "hand-made" latex table

\begin{dunetable}
[Top-level specifications for calibration subsystems]
{p{0.45\linewidth}p{0.25\linewidth}p{0.25\linewidth}}
{tab:fdgen-calib-toplevel-reqs} 
{List of Top-Level Specifications for the Calibration Subsystems. Global DUNE requirements are listed in bold.}   Quantity/Parameter	& Specification	& Goal		 \\ \toprowrule      
{\bf Noise from calibration devices}	 & $\ll$ 1000 enc   & \\ \colhline    
{\bf Max. \efield near calibration devices} & < 30 kV/cm & <15 kV/cm \\ \colhline     
Ionization Laser \efield measurement precision & 1\% & <1\% \\ \colhline
Ionization Laser \efield measurement coverage & > 75\% & 100\% \\ \colhline
Ionization Laser \efield measurement granularity & < \num{30}x\num{30}x\num{30}~cm & \num{10}x\num{10}x\num{10}~cm \\ \colhline
Laser beam location precision & 0.5 mrad & 0.5 mrad \\ \colhline
Neutron source coverage & > 75\% & 100\% \\ \colhline % neutron source
Ionization laser data volume (per 10 kton) & 90~TB/year & 185~TB/year\\ \colhline
Neutron source data volume (per 10~kton) & 84~TB/year & 168~TB/year\\ \colhline
Rate of 9~MeV capture $\gamma$-events inside the proposed radioactive source & < 1~kHz & \\ 
\end{dunetable}
\end{comment}

\begin{dunetable}
[Full specifications for calibration subsystems]
{p{0.45\linewidth}p{0.25\linewidth}p{0.25\linewidth}}
{tab:fdgen-calib-all-reqs}
{Full list of Specifications for the Calibration Subsystems.}   
Quantity/Parameter	& Specification	& Goal		 \\ \toprowrule      

Noise from calibration devices	 & $\ll$ 1000 enc   & \\ \colhline    Max. \efield near calibration devices & < 30 kV/cm & <15 kV/cm \\ \colhline                     

\textbf{Direct Ionization Laser System} &    &   \\ \colhline   
\efield measurement precision & 1\% & <1\% \\ \colhline
\efield measurement coverage & > 75\% & 100\% \\ \colhline
\efield measurement granularity & < \num{30}x\num{30}x\num{30}~cm & \num{10}x\num{10}x\num{10}~cm \\ \colhline
Top field cage penetrations (alternative design) & to achieve desired laser coverage & \\ \colhline
Data volume per 10~kton & 184 TB/year & 368 TB/year \\ \colhline
Longevity, internal parts	& \dunelifetime		& > \dunelifetime   \\    \colhline     
Longevity, external parts	& 5 years			& > \dunelifetime   \\ \colhline 
\textbf{Laser Beam Location System} & & \\ \colhline  
Laser beam location precision & 0.5~mrad & 0.5~mrad \\ \colhline
Longevity	& \dunelifetime		& > \dunelifetime   \\    \colhline     
\textbf{Photoelectron Laser System}	   &   &  \\ \colhline       Longevity, internal parts	& \dunelifetime		& > \dunelifetime   \\    \colhline 
Longevity, external parts	& 5 years			& > \dunelifetime   \\ \colhline 
\textbf{Pulsed Neutron Source System}	   &   &  \\ \colhline        
Coverage & > 75\% & 100\% \\ \colhline
Data volume per 10~kton & 144~TB/year & 288~TB/year
%DAQ rate per 10~kton & 190~TB/year & 168~TB/year
\\ \colhline 
Longevity	& 3 years			& > \dunelifetime   \\
\end{dunetable}
