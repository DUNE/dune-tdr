\dword{dune} plans to build two primary systems dedicated to %the calibration of the single-phase far detector TPC
calibrate the \dword{spmod} -- a laser system
and a \dlong{pns} system -- both of which require interfaces with the cryostat, that are described in Section~\ref{sec:sp-calib-cryostat}. 

The laser system is aimed at %the determination of 
determining the essential detector model parameters with high spatial and time granularity. The primary goal is to provide maps of the drift velocity and \efield, following a position-based technique already proven in other \dword{lartpc} experiments. %The R\&D plan in \dword{protodune2} will address the feasibility of carrying out charge-based measurements which, if successful, would open up the possibility of using the laser to measure electron lifetime.
%mapping an essential parameter for the description of the detector performance. 
Two laser sub-systems are planned. 
With high intensity coherent laser pulses, charge can be created in long straight tracks in the detector by direct ionization of \dword{lar} with the laser beams. This is described in Section~\ref{sec:sp-calib-sys-las-ion}. An auxiliary system aimed at an independent measurement and cross-check of the laser track direction is described in Section~\ref{sec:sp-calib-sys-las-loc}.
On the other hand, laser excitation of targets placed in the cathode creates additional charge from well-defined locations that can be used %to 
as a general \dword{tpc} monitor and to measure the integrated drift time. This is described in Section~\ref{sec:sp-calib-sys-las-pe}. 

The \dword{pns} system %is aimed at the performance validation in the low energy regime 
provides a ``standard candle'' neutron capture signal (\SI{6.1}{\MeV} multi-gamma cascade) across the entire \dword{dune} volume that is directly relevant to the supernova physics signal characterization thus validating the performance of the detector in the low energy regime. The \dword{pns} system is described in detail in Section~\ref{sec:sp-calib-sys-pns}.  

The physics motivation, requirements and design of these systems are described in the following subsections. Alternative designs
%Non-baseline alternatives 
for the ionization laser system, pulsed neutron source system, as well as the proposed radioactive source deployment system, are described in the Appendix, in Sections~\ref{sec:sp-calib-laser-alter}, \ref{sec:sp-calib-pns-alt}, and \ref{sec:sp-calib-sys-rsds}, respectively.