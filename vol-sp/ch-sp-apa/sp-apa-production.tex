%%%%%%%%%%%%%%%%%%%%%%%%%%%%%%%%%%%%%%%%%%%%%%%%%%%%%%%%%%%%%%%%%%%%%%%%%%%
\section{Production Plan}
\label{sec:fdsp-apa-prod}
%%%%%%%%%%%%%%%%%%%%%%%%%%%%%%%%%%%%%%%%%%%%%%%%%%%%%%%%%%%%%%%%%%%%%%%%%%%

The \dword{apa} consortium oversees the design, construction, and testing of the \dword{dune} \dword{spmod} \dword{apa}s. Production sites are being planned in the USA and UK. This approach allows the consortium to produce \dword{apa}s at the rate required to meet overall construction milestones and, at the same time, reduce risk to the project if any location has problems that slow the production pace.

The starting point for the \dword{apa} production plan for \dword{dune} \dwords{spmod} is the experience and lessons learned from \dword{pdsp} construction. For \dword{pdsp}, the \dword{apa}s have been constructed on single production lines set up at PSL in the USA and at Daresbury Laboratory in the UK.  The plan now is to construct \dword{apa}s for \dword{dune} at US and UK collaborating institutions with ten total production lines, four in the UK and six in the US.  

A production line is centered around a wire winding robot, or winder, that enables continuous wrapping of wire on a \SI{6}{m} long frame (see figures ~\ref{fig:winder} and \ref{fig:winder-photos}). 
Two process carts are needed to support the \dword{apa} during  board epoxy installation and \dword{qc} checks, among other construction processes. A means of lifting the \dword{apa} in and out of the winder is also required. A gantry-style crane was used for \dword{pdsp} construction.

The fabrication of an \dword{apa} is a  three-stage process requiring 
%Each \dword{apa} will require 
about \num{50} eight-hour shifts to complete, with a mix of engineering, technical, and scientific personnel.   The first stage, estimated at about four shifts, is a preparation stage in which  \dword{pds} cables and rails, wire mesh panels, comb bases, $X$-plane wire boards, and tension test boards are installed on the bare \dword{apa} frame. In the second and longest stage, lasting \num{38}--\num{40} shifts, the \dword{apa} occupies a winding machine.  All the wires are strung and attached in this stage, and tension and electrical tests of each channel are performed.
%includes the attachment of all wires as well as tension and electrical tests of each channel.  
The third and final stage, requiring an estimated \num{8} shifts, is completed in a process cart and involves the installation of wire harnesses, G-bias boards, and cover boards. 
%Next, p
Protection panels are then installed over the wire planes and the completed \dword{apa} is transferred to a transport frame (see Section ~\ref{sec:fdsp-apa-transport}).   During \dword{pdsp} construction, we were able to complete an \dword{apa} in \num{64} shifts, on average. Several improvements to the process and tooling have been developed since then to reduce this to the maximum allowed \num{50} shifts. 

%A key figure is t
The approximately \num{40} shifts that an \dword{apa} spends in the winding machine combined with %, as this and 
the total number of winders determines the overall pace of production since the pre- and post-winding stages can be done in parallel with winding.  %Whereas \dword{pdsp} construction involved multiple transfers back and forth between a winding machine and process cart, process improvements implemented since then allow just a single move into and out of the winder, resulting in the three clear stages described above.  
The overall production model assumes that the \dword{apa} production sites run one shift per day, that all winding machines are operated in parallel, and that two weeks per year are devoted to maintaining equipment.  The work plan at production sites further assumes a steady supply of the necessary hardware for \dword{apa} wiring, such as completed frames, grounding mesh panels, and wire boards.  Detailed planning is underway within the \dword{apa} consortium % whereby collaborating institutions contribute to the sourcing and testing of components and ensure their on-time delivery to production sites.        
for collaborating institutions to help  source and test components and ensure their on-time delivery to production sites.        

Having several \dword{apa} production sites in two different countries presents quality assurance and quality control (\dshort{qa}/\dshort{qc}) challenges. % Key among the requirements for production 
A key requirement is that every \dword{apa} be the same, regardless of where it was constructed. To achieve this goal, we are building on \dword{pdsp} experience where six identical \dword{apa}s were built, four in the US and two in the UK. The same tooling, fabrication drawings, assembly, and test procedures were used at each location, and identical acceptance criteria were established at both sites.  This uniform approach to construction %for \dword{dune} 
is necessary, and the \dword{apa} consortium is developing the necessary management structure to ensure that each production line follows the agreed-upon approach to achieve \dword{apa} performance requirements.


%%%%%%%%%%%%%%%%%%%%%%%%%%%%%%%%%%%%%%%%%%%%%%%%%%%%%%%%%%%%%%%%%%%%%%%%%%%
\subsection{Assembly Procedures and Tooling}
\label{sec:fdsp-apa-prod-tooling}


The central piece of equipment used in \dword{apa} production is the custom-designed wire winder machine, shown schematically in Figure~\ref{fig:winder} and in use in Figure~\ref{fig:winder-photos}.  An important centerpiece of the winder machine is the wiring head.  The head releases wire as motors move it up and down and across the frame, controlling the tension in the wire as it is laid.  The head then positions the wire at solder connection points for soldering by hand. The fully automated motion of the winder head is controlled by software, which is written in the widely used numerical control G programming language.  The winder also includes a built-in vision system to assist operators during winding, which is currently used at winding start-up to find a locator pin on the wire boards.  

\begin{dunefigure}[Winding machine schematic showing ongoing development]{fig:winder}
{Schematic of the custom-designed \dword{apa} wiring machine.  This shows the updated version with upper and lower supports and the spherical bearings for rotating the \dword{apa} on the winder.}
\includegraphics[width=0.95\textwidth]{sp-apa-winding-machine-design-development.jpg} 
\end{dunefigure}

\begin{dunefigure}[APA wire winding machine]{fig:winder-photos}
{Left: Partly wired \dword{pdsp} \dword{apa} on the winding machine at Daresbury Lab, UK. Right: Partly wired \dword{pdsp} \dword{apa} on the winding machine during wire tension measurements at PSL.}
\includegraphics[height=0.3\textheight,trim=25mm 0mm 4mm 0mm,clip]{sp-apa-on-winder-daresbury.jpg}
\includegraphics[height=0.3\textheight,trim=200mm 0mm 30mm 0mm,clip]{sp-apa-tension-testing-psl.jpg}
\end{dunefigure}

In the scheme used for wiring the \dword{pdsp} \dword{apa}s, an \dword{apa} moved on and off the winder machine several times for wiring, soldering, and testing. %, which is time consuming and increases risk. 
 Several design changes were developed in 2018--2019 to enable the \dword{apa} to remain on the winding machine throughout the wiring process. The design concept allows the winder head to pass from one side to the other nearly continuously. The interface frames at either end have been replaced by retractable linear guided shafts. These can be withdrawn to allow the winding head to pass around the frame over the full height of the frame. These shafts have conical ends and are in shafts fixed to the internal frame tube to provide guides to location. This design change does not alter the design of the frame itself, but it does allow for rotation in the winding machine. It is now possible to carry out board installation, gluing, and soldering all while on the winding machine. This eliminates any transfer of the \dword{apa} to the process cart for the entire production operation, making it an inherently safer and faster production method.% because it cuts down on handling of the \dword{apa}.  
 The upgraded design has been implemented on the winding machine at Daresbury, which has been used to build a new prototype, \dword{apa}~7 (Figure~\ref{fig:winder-upgrade-photos}). All winding, board installation, gluing, soldering and testing operations are being carried out in the winding machine. \dword{apa}~7 also incorporates the pre-built grounding mesh sub-frames, another new feature %since the \dword{pdsp} design 
 that saves significant time in production.  

\begin{dunefigure}[The upgraded APA wire winding machine]{fig:winder-upgrade-photos}
{Left: Upgraded winding machine with new interface arm design being used to wire APA-07. Fitted mesh panels are also shown installed. Right: The V-layer soldering process at the head end of APA-07. Soldering can now be done with the \dword{apa} in the winding machine.}
\includegraphics[height=0.25\textheight,trim=10mm 0mm 0mm 0mm,clip]{sp-apa-updated-winder.jpg} 
\includegraphics[height=0.25\textheight,trim=0mm 0mm 10mm 0mm,clip]{sp-apa-updated-winder-soldering.jpg}
\end{dunefigure}

%Another design effort since \dword{pdsp} construction has been to update the wiring head. 
The wiring head has also been updated. The upgraded design offers real-time tension feedback and control, which will save time in wiring and produce better tension uniformity across wires.  A prototype of the new head has been constructed and is undergoing extensive commissioning and qualification.   

An important element in the long-term use of the winders %for making many \dword{apa}s for \dword{dune} 
will be maintenance.  %The approach to winder maintenance during \dword{pdsp} construction was not well considered. As a result, w
During \dword{pdsp} construction winding machine problems traceable to a lack of routine maintenance occurred from time to time, shutting the production line down until repair or maintenance was performed. We will formulate a routine and preventive maintenance plan that minimizes winder downtime during \dword{apa} production for the \dword{spmod}.

The large process carts are important to the %Another key element in the 
flow of activities during production % are large process carts, with 
(Figure~\ref{fig:apa-process-cart}). The process carts are used to hold \dword{apa}s during wiring preparations,  for \dword{qc} checks after wiring, and to safely move \dword{apa}s around within the assembly facility. Process carts are fitted with specialized 360$^\circ$ rotating casters that allow the cart, loaded with a fully assembled \dword{apa}, to maneuver corners while moving the large frames between preparation, assembly, and packing/shipping areas.

\begin{dunefigure}[APA on a process cart during construction]{fig:apa-process-cart}
{A \dword{pdsp} \dword{apa} being moved around the PSL production facility on a process cart.}
\includegraphics[height=0.35\textheight,trim=50mm 10mm 70mm 250mm,clip]{sp-apa-process-cart.jpg}
\end{dunefigure}



%%%%%%%%%%%%%%%%%%%%%%%%%%%%%%%%%%%%%%%%%%%%%%%%%%%%%%%%%%%%%%%%%%%%%%%%%%%
\subsection{APA Production Sites}
\label{sec:fdsp-apa-prod-facility}
%%%%%%%%%%%%%%%%%%%%%%%%%%%%%%%%%%%%%%%%%%%%%%%%%%%%%%%%%%%%%%%%%%%%%%%%%%%

%Construction of \dword{spmod} \dword{apa}s will take place in both the USA and the UK.  
Multiple \dword{apa} production lines spread over several sites in the USA and the UK will provide some margin on the production schedule and provide backup in the event that technical problems occur at any particular site. 

The space requirements for each production line are driven by the size of the \dword{apa} frames and the winding robot used to build them. The approximate dimensions of a class \num{100000} clean space needed to house winder operations and associated tooling is \SI{175}{m$^2$}. The estimated requirement for inventory, work in progress, and completed \dword{apa}s is about \SI{600}{m$^2$}. Each facility  also needs temporary access to shipping and crating space of about \SI{200}{m$^2$}. Floor layouts at each institution are being developed, with current layouts shown in Figure~\ref{fig:factories}. Adequate space is available at each site, and the institutions have offered commitments for space for this purpose. %\dword{dune}. 

At Daresbury Laboratory in the UK, the existing single production line used for \dword{pdsp} construction will be expanded to four.  The Inner Hall on the Daresbury site has been identified as an area that is large enough to be used for \dword{dune} \dword{apa} construction. It has good access and crane coverage throughout. Daresbury Laboratory management has agreed that the area is available, and a working environment that meets \dword{dune}'s safety standards is now being prepared, starting with clearing the current area of existing facilities, obsolete cranes, and ancillary equipment. The renovation of a plant room is also in progress, so that it can be used for storage and as a shipping area. The production area is designed to hold four winding machines and associated process equipment and tooling.  A production site design review of the Daresbury facility is planned for January 2020, and a production site readiness review is anticipated for June 2020, followed by the start of \dword{apa} production for \dword{dune} \dword{detmodule} \#1 in August 2020.  


\begin{dunefigure}[Layouts of APA production sites]{fig:factories}
{Developing concepts for production site layouts at Daresbury Lab (top left), University of Chicago (top right), and Yale University (bottom left), and the existing APA production area at PSL (bottom right).}
\includegraphics[height=0.23\textheight]{sp-apa-factory-daresbury.jpg} 
\includegraphics[height=0.23\textheight]{sp-apa-factory-chicago.jpg} \\
\vspace{1mm}
\includegraphics[height=0.225\textheight]{sp-apa-factory-yale.jpg}
\includegraphics[height=0.225\textheight]{sp-apa-factory-psl.jpg} 
\end{dunefigure}

In the USA, there will be six total production lines at three sites: two at the University of Chicago, two at Yale University, and two at the University of Wisconsin's PSL, including the existing winder where the construction for \dword{pdsp} was carried out. 

The \dword{apa} production site at the University of Chicago will be housed in the Accelerator Building on the campus in Hyde Park.  The building has hosted the assembly of large apparatuses for numerous experiments over the course of its history and features an extensive high bay with an overhead crane, an indoor truck bay, clean laboratory spaces, a professional machine shop, and proximity to faculty and staff offices.  Winding will be done inside a clean room installed on the first floor-level mezzanine, where there is \SI{234}{m^2} of floor space above the machine shop.  A \SI{2}{ton} capacity bridge crane will be installed inside this clean room to move \dword{apa}s %to and from 
between the two winders and the process carts that will be located here.  \dword{apa}s will enter and exit the mezzanine by way of a loading deck external to the cleanroom.  Preparation of \dword{apa} frames, including mesh installation, will be done inside a second clean room on the basement level floor of the high bay.  Ample space, roughly \SI{170}{m^2}, between this clean room and the truck bay allows for simultaneous receiving of bare frames or other larger items, hoisting of \dword{apa}s to and from the mezzanine, and packaging of completed \dword{apa}s for outbound shipment.  When needed, additional off-site storage will be available for holding excess inventory and completed \dword{apa}s before they are transported to South Dakota.

Yale's Wright Laboratory will host another of the USA-based \dword{apa} production sites in a recently renewed area named ``The Vault'' where the nuclear accelerator operated previously.  The Vault is approximately \SI{720}{m^2} of total floor space and it satisfies all the safety and space requirements for an \dword{apa} production site. 
Indeed, the area, which is planned to be completely transformed into a cleanroom, can easily host two winders and four processing carts and has sufficient space for crating the 
\dword{apa}s for shipment and receiving and stocking all the material, e.g., bare frames, electronics boards, and mesh panels. A large high bay door at one end offers direct road access, allowing trucks to back inside the room where a \SI{10}{ton} crane operates all along the length.  Moreover, Wright laboratory has good support infrastructure, including cleanrooms and modern mechanical and prototyping workshops that are directly connected to the Vault. Faculty, researchers and postdoc offices are located upstairs in the same building. % right upstairs from the Vault.

The Physical Sciences Laboratory (PSL) Rowe Technology Center has up to \SI{1850}{m^2} (\SI{20000}{ft^2}) total space available for continued \dword{dune} activities.  A clean work area that houses the existing winding machine used for \dword{pdsp} is already in place and will be used for \dword{dune} \dword{apa} construction. A second \dword{apa} production line using the updated winder design will assembled in 2020, and the existing winder will be upgraded.  PSL will host other major activities as well, including the assembly of bare \dword{apa} frames for wiring in the USA, production of \dword{cr} boards, and fabrication of \dword{apa} pair linkage and installation hardware.

Development work relevant for local planning at each site is rapidly advancing.  Figure~\ref{fig:factories} shows current conceptual layouts for the future production setups at Daresbury, Chicago, and Yale and a photograph of the existing \dword{apa} production facility at PSL-Wisconsin.  Production Site Design Reviews of the Chicago and Yale facilities are planned for early in 2020. 

%%%%%%%%%%%%%%%%%%%%%%%%%%%%%%%%%%%%%%%%%%%%%%%%%%%%%%%%%%%%%%%%%%%%%%%%%%%
\subsection{Material Supply}  
\label{sec:fdsp-apa-prod-supply}
%%%%%%%%%%%%%%%%%%%%%%%%%%%%%%%%%%%%%%%%%%%%%%%%%%%%%%%%%%%%%%%%%%%%%%%%%%%

Ensuring the reliable supply of raw materials and parts to each \dword{apa} production site is critical to keeping \dword{apa} production on schedule through the years of construction. Here the consortium institutions are pivotal in taking responsibility for delivery of \dword{apa} sub-elements. Supplier institutions will be responsible for sourcing, inspecting, cleaning, testing, \dword{qa}, and delivery of hardware to each production site.  In particular, the critical activities to supply production sites with the minimum needed \dword{apa} components for assembly include:

\begin{itemize}

\item {\bf Frame construction:} There will be separate sources of frames in the USA and the UK. The institutions responsible will rely on many lessons learned from \dword{pdsp}. The effort requires specialized resources and skills, including a large assembly area, certified welding capability, large-scale metrology tools and experience, and large-scale tooling and crane support. We are considering two approaches for sourcing: one is to outsource to an industrial supplier; the other is to procure all the major machined and welded components and then assemble and survey in-house. Material suppliers have been identified and used with good results on \dword{pdsp}.

\item {\bf Grounding mesh supply:} The modular grounding mesh frame design allows the mesh screens to be produced outside of the \dword{apa} production sites and supplied for \dword{apa} construction.  Suitable vendors to supply the needed units (20 mesh frames per \dword{apa}) will be identified in both the USA and UK.   

\item {\bf Wire wrapping board assembly:} Multiple consortium institutions will take on the responsibility of supplying the tens of thousands of wire-wrapping boards required for each \dword{sp} \dword{detmodule}. The side and foot boards with electrical traces are procured from suppliers and a separately bonded tooth strip is installed to provide wire placement support. 
The institutions responsible for boards will work with several vendors to reduce risk and ensure quality. 


\item {\bf Wire procurement:} %Wire is a significant element in the assembly of an \dword{apa}. 
Approximately \SI{24}{km} of wire is required for %wound onto 
each unit. During \dword{pdsp} construction, %we have worked with 
an excellent supplier %that has 
worked with us to provide high-quality wire wound onto spools that we provide. These spools are then used directly on the winder head with no additional handling or re-spooling required. Wire samples from each spool are strength-tested before use.

\item {\bf Comb procurement:} Each institution will either work with our existing comb supplier or find other suppliers who can meet our requirements. The \dword{pdsp} supplier has been very reliable.

\end{itemize}


%%%%%%%%%%%%%%%%%%%%%%%%%%%%%%%%%%%%%%%%%%%%%%%%%%%%%%%%%%%%%%%%%%%%%%%%%%%
\subsection{Quality Control in APA Production}
\label{sec:fdsp-apa-prod-qc}


\Dword{qc} testing is a critical element of \dword{apa} production.  All \dword{qc} procedures are being developed by the consortium and will be implemented identically at all production sites in order to ensure a uniform quality product as well as uniform available data from all locations.  Important \dword{qc} checks are performed both at the level of components, before they can be used on an \dword{apa}, as well as on the completed \dword{apa}s, to ensure quality of the final product before leaving the production sites.  In addition, a 10\% sample of the completed \dword{apa}s produced at each of the production sites each year will be cold cycled in a cryogenic test facility available at PSL.

\subsubsection{APA Frame Acceptance Tests} 

Each \dword{apa} support frame must meet geometrical tolerances in order to produce a final \dword{apa} that meets requirements for physics. In particular, the wire plane-to-plane spacing must be within the specified tolerance of $\pm$\SI{0.5}{mm} (see Sec.~\ref{sec:fdsp-apa-design-overview}).  Flatness of the support frame, therefore, is a key feature and is defined as the minimum distance between two parallel planes that contain all the points on the surface of the \dword{apa}.  Although %there are any number of ways in which 
the frame could be distorted out of plane in several ways, the most likely causes are: %ones can be approximated by three modes: 
(1)~a curve in the long side tubes causing the frame to bow out of plane, (2)~a twist in the frame from one end to the other, or (3)~a fold down the center-line (if the ends of the ribs are not adequately square).

As detailed in Section~\ref{sec:fdsp-apa-frames}, \dword{apa} frames are constructed of 13 separate rectangular hollow steel sections.  Before machining, a selection procedure is followed to choose the sections of the steel most suited to achieving the geometrical tolerances.  After assembly, a laser survey is performed on the bare frames before they can be delivered to an \dword{apa} production site. Three sets of data are compiled into a map that shows the amount of bow, twist, and fold in the frame. A visual file is also created for each \dword{apa} from measured data. 

A study was performed to determine the tolerances on the three distortions characterized above and is documented in~\cite{bib:docdb1300}.  It was determined that a \SI{0.5}{mm} change in the final wire plane spacing could result from:
\begin{enumerate}
\item An \SI{11}{mm} out-of-flatness caused by curved long side tubes.
\item A \SI{6}{mm} out-of-flatness due to a twist in the frame.  This is assumed to be evenly distributed between each of the 5 cells of the \dword{apa} with $\sim$\SI{1.2}{mm} out-of-flatness per cell.
\item A \SI{1.2}{mm} out-of-flatness due to a fold down the middle of the \dword{apa}.
\end{enumerate}

The bow, twist, and fold extracted from the survey data will be compared against these allowable amounts before the support frame is used to build an \dword{apa}.  Later, during \dword{apa} wiring at the production sites, a final frame survey will be completed after all electrical components have been installed, and the as-built plane-to-plane separations will be measured to verify that the distance between adjacent wire planes meets the tolerances.  

Another check performed at the \dword{apa} production site before the frame is transferred to a winder will confirm sufficient electrical contact between the mesh sub-panels and the \dword{apa} support frame.  A resistance measurement is taken immediately after mesh panel installation for all \num{20} panels before wiring begins.




\subsubsection{Material Supply Inspections}


All components require inspection and \dword{qc} checks before use on an \dword{apa}.  Most of these tests will be performed at locations other than the \dword{apa} production sites by institutions within the consortium before the hardware is shipped for use in \dword{apa} construction. This distributed model for component production and \dword{qc} is key to enabling the efficient assembly of \dword{apa}s at the production sites.   The critical path components are the support frames (one per \dword{apa}), grounding mesh panels (20 per \dword{apa}), and wire carrier boards (204 per \dword{apa}). Section~\ref{sec:fdsp-apa-org} provides details about which consortium institutions in the US and UK will be responsible for each of these work packages.  

\subsubsection{Wire Tension Measurements and Channel Continuity and Isolation Checks}
\label{sec:fdsp-apa-prod-qa-tension}

The tension of every wire will be measured during production to ensure wires have a low probability of breaking or moving excessively in the detector.  Every channel on the completed \dword{apa}s will also be tested for continuity across the \dword{apa} and isolation from other channels.  The plan is to perform all tests at once, using the methods described in this section.  As will be described in Section~\ref{sec:fdsp-apa-prod-coldtest}, it is also planned that 10$\%$ of the \dword{apa}s will be shipped to PSL for a cold test, where the full \dword{apa} will be brought to \dword{ln} temperature. Following the cold test, the wire tensions and continuity will be remeasured.  Finally, for this 10$\%$ sample of \dword{apa}s, a measurement of the wire plane spacing will be performed using a Faro arm that can precisely record the position of each wire plane in space. This checks that the \dword{qc} on the flatness of the support frames remains sufficient.  

Wire tensions will be measured after each new plane of wires is installed on an \dword{apa}. The optimal target tension has been set at \SI{6}{N} based on \dword{pdsp} experience.  \Dword{pdsp} data, where the tensions did have substantial variation, is also being used to study the effects of varying tensions and finalize the allowed range of values.  


The technique  
used to measure tensions for the \dword{apa}s of \dword{pdsp} was based on a laser and a photodiode system~\cite{Acciarri:2016ugk}. In this method, the laser shines on an individual wire and its reflection is captured by the photodiode. An oscillation is produced in the measured voltage when a vibration is induced on the wire, such as by manually plucking it. This oscillation is dominated by the fundamental mode of the wire, which is set by the wire's tension. Since the length and density of the wire are known, the measured fundamental frequency can be converted into a tension value. The method works very well, but due to the necessity of aligning the laser and exciting and measuring wires individually, this technique can take tens of seconds per wire. Given the large number of wires per \dword{detmodule}, development of a faster technique represents a major opportunity for the full \dword{dune} construction.

A technique that can reduce the overall time required to measure the tension of every wire is currently being developed~\cite{Garcia-Gamez:2018frz}. In this method, DC and AC voltages are applied on the neighboring wires of a wire under test. A sine wave of the same frequency as that of the applied AC voltage is measured from the tested wire, since it is capacitively coupled to its neighbors. The amplitude of the sine wave exhibits a resonant behavior when the frequency of the AC voltage corresponds to the fundamental frequency of the wire. Thus, a frequency sweep of the AC voltage can be performed to determine at which frequency there is a resonance, from which the wire tension can be obtained. As electrical signals can be injected and measured in several wires simultaneously, this technique has the potential of measuring the tension of many wires at once.

A wire tension measurement device based on the electrical method is being developed within the context of the \dword{dune} \dword{apa}s. While the underlying principle of the electrical method has been demonstrated, its technical implementation requires consideration. The wire pitch of the \dword{apa}s requires summed input voltages on the order of \SI{500}{V} to reasonably discern resonances against noise. The head boards, cf.\ Section~\ref{sec:fdsp-apa-headboards}, have been designed to withstand temporarily such large differential voltages across neighboring channels. Additionally, the components of the \dword{cr} boards or of the \dword{ce} would interfere with the method and need to be absent.

The exact specifications of the measurement system are being finalized. It is planned to connect to one of the twenty head board stacks at a time. Within a given stack, the device is projected to inject and read out signals by groups of sixteen wires simultaneously. The device could be used to measure the tension of any wire layer at any stage of the production process, in particular after the winding of a wire layer or after all the wires are wound. The designs of the winder machine and of the \dword{apa} protection panels have clearance provisions for the usage of such a measurement device.

The measurement system design is a combination of a commercial \dword{fpga} board and a custom printed circuit board for analog signal processing. An \dword{fpga} board is used as it can produce a square wave at any frequency that is expected to be encountered while measuring a wire's fundamental frequency, i.e.,\ below \SI{5}{kHz}. In addition, the \dword{fpga} board can be used for digital signal processing of the readout signal. The analog circuitry would act as a bridge between the \dword{fpga} and the \dword{apa} wires. It is needed to filter the square wave into a sine wave, to amplify that sine wave before sending it to the wires and to digitize the readout signal before sending it to the \dword{fpga}. The analog board is also needed to provide electrical connections to the head boards. With such a design, it is expected that the concurrent tension measurement of eight wires would take on the order of ten seconds.

A prototype of the measurement device has been built. The main difference between the prototype and the planned design is that the former is restricted to three wires instead of sixteen: a single readout wire and two stimulus wires. The prototype has been employed on a test bench in which wires with the same physical properties as those that will be used in the \dword{apa}s have been wound. The wires were wound according to these wire parameters, which are similar to those of the \dword{apa}s: \SI{6}{N} tension, \SI{6}{m} length, \SI{4.7}{mm} pitch. The applied voltages were \SI{400}{V} DC and \SI{26}{V} for the AC peak amplitude. The results obtained are shown in Figure~\ref{fig:electrical-tension}. The expected resonant frequency is \SI{16.1}{Hz}. The observed resonant frequency is obtained from the raw data by offline data analysis using numerical algorithms that can be implemented directly in the \dword{fpga}. The value obtained is \SI{16.0}{Hz}, corresponding to a tension value of \SI{5.9}{N}, which is within a few percent of the physical value.

\begin{dunefigure}[Observed resonant frequency in electrical wire tension method]{fig:electrical-tension}
{Amplitude of the readout signal as a function of the stimulus frequency, as used in the electrical wire-tension method. The vertical line is located at the observed resonant frequency. The raw digitized signal values corresponding to the first data point of the main plot are shown in the inset plot.}
\includegraphics[height=0.33\textheight]{sp-apa-electrical-tension.pdf}
\end{dunefigure}

In this test bench setup, no wire support combs, cf.\ Section~\ref{sec:combs}, are present. Their presence shortens the wire length that needs to be considered in this method, resulting in several higher resonant frequencies per wire. A similar effect happens for the wire channels that wrap around the \dword{apa} frame. Although they are a succession of wire segments electrically connected, the segments are mechanically independent and can have different tension values. Several resonant frequencies can be present per readout channel, possibly corresponding to different tension values.

In addition to measuring tension values, the measurement device is envisioned to be able to test wires for electrical isolation and electrical continuity, given the flexibility of the \dword{fpga} and provisions put in place in the design of the analog circuitry. Injecting a signal in a readout channel and detecting it in a different channel would indicate that these channels are not electrically isolated, for example, due to a solder bridge. The electrical continuity could be tested by sending a pulse down a channel and measuring the time it takes to travel through the wire and back to the measurement device.  If the measured time is shorter than expected, this could indicate cold solder joints, for example.

A final review of the electrical tension measuring system design will take place in spring 2020. Once completed, mobile \dword{apa} test stands will be built for each of the \dword{apa} production sites, the \dword{sdwf}, and \dword{surf}.  The introduction of the electrical testing methods for \dword{apa}s presents a fantastic opportunity for more efficient \dword{apa} fabrication and more flexible testing during the integration and installation phases.    


\subsubsection{Cold Testing of APAs}
\label{sec:fdsp-apa-prod-coldtest}

The six \dword{apa}s produced for \dword{pdsp} have demonstrated clearly that the \dword{apa} design, materials, and fabrication methods are sufficiently robust to operate at \dword{lar} temperature.  No damage or change in performance due to cold have been identified during \dword{pdsp} running.  Nevertheless, over a five year construction effort, it is prudent to cold cycle a sample of the \dword{apa}s produced to ensure steady fabrication quality.  A cold testing facility sized for \dword{dune} \dword{apa}s exists at PSL and can be used for such tests. Throughout the construction project, it is anticipated that 10\% of the produced \dword{apa}s will be shipped to PSL for cold cycling.  This amounts to about 1 APA per year per production site during the project.  It is planned that all APAs will still be cold tested during integration at SURF and before installation in the DUNE cryostats.      


\subsubsection{Documentation} 
\label{sec:fdsp-apa-prod-doc}

Each \dword{apa} is delivered with a traveler document in which specific assembly information is gathered, initially by hand on a paper copy, then entered into an electronic version for longer term storage.  The traveler database contains a detailed log of the production of each \dword{apa}, including where and when the \dword{apa} was built and the origin of all parts used in its construction. 

Assembly issues that arise during the construction of an \dword{apa} are gathered in an issue log for each \dword{apa}, and separate short reports provide details of what caused the occurrence, how the issue was immediately resolved, and what measures should be taken in the future to ensure the specific issue has a reduced risk of occurring.  
