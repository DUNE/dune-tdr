%%%%%%%%%%%%%%%%%%%%%%%%%%%%%%%%%%%%%%%%%%%%%%%%%%%%%%%%%%%%%%%%%%%
\section{Cost, Schedule, and Risks}
\label{sec:fdsp-apa-cost-sched}

\subsection{Cost}

Labor and M\&S cost estimates for \dword{apa} production are based on extensive previous experience at PSL in the USA and at Daresbury Laboratory in the UK in building six full-scale \dword{apa}s during 2017-2018 for \dword{pdsp}, now operating at \dword{cern}.  Table~\ref{tab:apa-costs} summarizes the estimated labor hours and material costs required to construct 300 \dword{dune} \dword{apa}s, enough for two \dword{fd} \nominalmodsize \dwords{spmod}.

Production setup includes all the setup costs necessary for the production and installation of \dword{apa}s. These costs are incurred only once, independently from the number of detector modules. \dword{apa} components production setup includes all the tooling necessary for the production and test of \dword{apa} components, e.g. jigs to assemble and test \dword{apa} boards. \dword{apa} assembly production setup refers to the costs of setting up the \dword{apa} wiring sites, including clean tents, winding machines, process carts, and jigs and test hardware. \dword{apa} integration and installation setup includes all \dword{apa} specific fixtures necessary for handling the \dword{apa} transport boxes at \dword{surf} and assembling \dword{apa} pairs. % doublets.

Production includes all costs for the construction of 300 \dword{apa}s, plus two spare \dword{apa}s for each detector module. \dword{apa} components production includes material costs and labor hours for procurement and fabrication of the \dword{apa} components.  \dword{apa} components production costs are dominated by the production of boards and frames. \dword{apa} assembly production includes all labor hours necessary for the assembly of the \dword{apa}s, as well as shipping costs. Production represents the driving material costs and labor resources for the \dword{apa}s.

\dword{apa} assembly production requires a large amount of labor hours. Each wiring facility will need personnel to stage all materials and tooling necessary for \dword{apa} assembly and conduct the actual \dword{apa} assembly/production process steps.  As each \dword{apa} is completed, it will be readied and packed for shipment to the \dword{sdwf} in South Dakota. The following personnel resource allocations are based on \dword{pdsp} experience: each production line will require 5.5 \dword{fte}, including 2.5 \dword{fte} of technicians (certified operators, material handlers), 1.0 \dword{fte} of engineer (winding processes, maintenance), 1.0 \dword{fte} of student (operator), 0.5 \dword{fte} of post-doctoral researcher (QA/QC manager) and 0.5 \dword{fte} of research physicist (production manager) on average over time.

Integration and installation include all labor hours required at \dword{surf}, as well as travel support.

\fixme{new standard cost table will be coming in early April - for auto-generating latex. Anne}

%\fixme{Table~\ref{tab:Xsched} is a standard table template for the TDR schedules.  It contains overall FD dates from Eric James as of March 2019 (orange) that are held in macros in the common/defs.tex file so that the TDR team can change them if needed. Please do not edit these lines! Please add your milestone dates to fit in with the overall FD schedule. Please set captions and label appropriately. Anne}

\subsection{Schedule}
%%%%%% SCHEDULE
\begin{comment}
The high-level milestones between 2018 and 2026 are given in Table~\ref{tab:apa-milestones}. The final design of the \dwords{apa} proposed in the \dword{tdr} will be informed by \dword{pdsp} \dword{apa} production and performance, which will be reviewed in early 2019. Additional design considerations that cannot be directly tested through \dword{pdsp}, like the two-\dword{apa} assembly and related cabling issues, will require a full test with cabling of a two-\dword{apa} assembly, also set for early 2019. The production schedule, the required number of assembly lines, and the location of the production factories will depend on improvements in the wire winding procedures, which will be formally reviewed in early 2019. The post-\dword{tdr} milestones are driven by high-level international project milestones and are based on a schedule that includes one year of factory preparation and about two years and a half of \dword{apa} construction.
\end{comment}


The high-level milestones between 2019 and 2026 are given in Table~\ref{tab:apa-milestones}.
\begin{dunetable}
[APA Consortium Schedule]
{p{0.65\textwidth}p{0.25\textwidth}}
{tab:apa-milestones}
{Schedule milestones for the production and installation of \dwords{apa} for two \dword{sp} \dword{dune} far detector modules.}   
Milestone & \bfseries{Date}    \\ \toprowrule
%\rowcolor{lightgray} \multicolumn{2}{c}{Design completion, reviews, and \dword{pdsp}} \\ \colhline
Decision on the necessity and design of electron diverters & June 2019     \\ \colhline
\dword{apa} Production Site Design Review - UK & July 2019  \\ \colhline
\dword{apa} Production Site Design Review - USA & September 2019  \\ \colhline
Completion of winding machine modifications and tests & September 2019 \\ \colhline
Completion of \dword{apa} doublet frame assembly \& cabling at \dword{ashriver} & October 2019 \\ \colhline
Start of pre-production \dword{apa} for integration test at CERN & November 2019 \\ \colhline
Decision on the wire tension measurement method & March 2020    \\ \colhline
Completion of \dword{apa} integration test with CE and PDS at CERN & April 2020 \\ \colhline
Final Design Review & May 2020 \\ \colhline
Start of \dword{apa} Components Production & June 2020 \\ \colhline
Production Site Readiness Review  - UK & June 2020 \\ \colhline
Start of component production for \dword{pdsp}-II & July 2020 \\ \colhline
Start of \dword{apa} Production - \dword{detmodule} \#1 - UK & August 2020 \\ \colhline 
Production Sites Readiness Review  - USA &  December 2020    \\ \colhline
End of component production for \dword{pdsp}-II & December 2020 \\ \colhline
Start of \dword{apa} Production - \dword{detmodule} \#1 - USA  & January 2021  \\ \colhline
\rowcolor{dunepeach} Start of \dword{pdsp}-II installation& \startpduneiispinstall \\ \colhline
\rowcolor{dunepeach}South Dakota Logistics Warehouse available& \sdlwavailable \\ \colhline
\rowcolor{dunepeach}Beneficial occupancy of cavern 1 and \dword{cuc}& \cucbenocc \\ \colhline
\rowcolor{dunepeach} \dword{cuc} counting room accessible& \accesscuccountrm  \\ \colhline
End of \dword{apa} Production - \dword{detmodule} \#1  & September 2023 \\ \colhline
Start of \dword{apa} Production - \dword{detmodule} \#2  & October 2023 \\ \colhline
\rowcolor{dunepeach}Top of \dword{detmodule} \#1 cryostat accessible& \accesstopfirstcryo \\ \colhline
\rowcolor{dunepeach}Start of \dword{detmodule} \#1 TPC installation& \startfirsttpcinstall \\ \colhline
\rowcolor{dunepeach}Top of \dword{detmodule} \#2 accessible& \accesstopsecondcryo \\ \colhline
\rowcolor{dunepeach}End of \dword{detmodule} \#1 TPC installation& \firsttpcinstallend \\ \colhline
\rowcolor{dunepeach}Start of \dword{detmodule} \#2 TPC installation& \startsecondtpcinstall \\ \colhline
End of \dword{apa} Production - \dword{detmodule} \#2  & April 2026  \\ \colhline
\rowcolor{dunepeach}End of \dword{detmodule} \#2 TPC installation& \secondtpcinstallend \\
\end{dunetable}

Analysis of the \dword{pdsp} data will inform the decision on the electron diverters by June 2019.  Additional design considerations that cannot be directly tested through \dword{pdsp}, like the \dword{apa} doublet assembly and related cabling issues, will require a full test with cabling of an \dword{apa} doublet frame assembly, planned to be performed at \dword{ashriver} by the end of Summer 2019.  Also planned is the construction of a pre-production \dword{apa} for an integration test with \dword{ce} and \dword{pds} at \dword{cern} in Spring 2020, which will fully test all interface aspects. This test will inform the final design review of the \dword{apa} system in May 2020.

Design reviews of \dword{apa} production sites in the UK and USA, to determine the layout of the production lines, are planned for summer 2019, together with the finalization of the winding machine modifications.  Production site readiness reviews are planned in June 2020 and December 2020, respectively in the UK and USA.

Production of three \dword{apa}s for a final test in \dword{pdsp}-II is foreseen in the second half of 2020. The pre-production \dword{apa} used for the integration test at \dword{cern} in Spring 2020 could be used for installation in \dword{pdsp}-II, if no additional modifications are required.

Dates are also provided in Table~\ref{tab:apa-milestones} for the start and end of \dword{apa} production for \dword{detmodule}s \#1 and \#2. Steady-state production rates are 24 \dword{apa}s/year at Daresbury Laboratory with 4 production lines, 12 \dword{apa}s/year both at Yale and Chicago, each with 2 production lines, and 6 \dword{apa}s/year at PSL, with one production line. The production time for \dword{detmodule} \#1 takes into account a gradual start-up of the production lines, and the different start dates and number of production lines in the UK and USA. The end of \dword{apa} production for \dword{detmodule} \#1 happens comfortably ten months before the start of installation. In the UK, with four assembly lines, in order to meet the installation date for detector module \#2 the \dword{apa} production time would need to be reduced by about 7 months. This could be achieved by a reduction of the \dword{apa} assembly time, an opportunity mentioned in Subsection~\ref{sec:fdsp-apa-cost-sched-risks}, and, if necessary, by increasing the number of working shifts/week.
%

\begin{comment}
\begin{dunetable}
[APA Production Schedule]
{lccccccc}
{tab:apa-production-schedule}
{Estimated schedule of \dword{apa} production at each of the \dword{apa} wiring locations.}
Production Site & \textbf{2020} & \textbf{2021} & \textbf{2022} & \textbf{2023} & \textbf{2024} & \textbf{2025} & \textbf{Total} \\ \colhline
Daresbury Laboratory & 6 & 24 & 26 & 32 & 32 & 32 & \color{red} 152 \\ \colhline
University of Chicago & 0 & 10 & 13 & 13 & 13 & 11 & \color{red} 60 \\ \colhline
Yale University & 0 & 10 & 13 & 13 & 13 & 11 & \color{red} 60 \\ \colhline
University of Wisconsin-PSL & 0 & 3 & 7 & 7 & 7 & 8 & \color{red} 32 \\ \Xhline{3\arrayrulewidth}
Running Total & 6 & 53 & 112 & 177 & 242 & 304 &  \\
\end{dunetable}
\end{comment}

%%%%% RISKS
\subsection{Risks}
\label{sec:fdsp-apa-cost-sched-risks}
Risks have been identified for the finalization of the \dword{apa} design and the prototyping phase, for the setup of the production sites, for the production of \dword{apa}s, and for installation at \dword{surf}.  Risks are summarized in Table~\ref{tab:risks:SP-FD-APA}.


% risk table values for subsystem SP-FD-APA
\begin{footnotesize}
%\begin{longtable}{p{0.18\textwidth}p{0.20\textwidth}p{0.32\textwidth}p{0.02\textwidth}p{0.02\textwidth}p{0.02\textwidth}}
\begin{longtable}{P{0.18\textwidth}P{0.20\textwidth}P{0.32\textwidth}P{0.02\textwidth}P{0.02\textwidth}P{0.02\textwidth}} 
\caption[APA risks]{APA risks (P=probability, C=cost, S=schedule) The risk probability, after taking into account the planned mitigation activities, is ranked as 
L (low $<\,$\SI{10}{\%}), 
M (medium \SIrange{10}{25}{\%}), or 
H (high $>\,$\SI{25}{\%}). 
The cost and schedule impacts are ranked as 
L (cost increase $<\,$\SI{5}{\%}, schedule delay $<\,$\num{2} months), 
M (\SIrange{5}{25}{\%} and 2--6 months, respectively) and 
H ($>\,$\SI{20}{\%} and $>\,$2 months, respectively). \fixmehl{ref \texttt{tab:risks:DP-FD-CISC}}} \\
\rowcolor{dunesky}
ID & Risk & Mitigation & P & C & S  \\  \colhline
RT-SP-APA-01 & Loss of key personnel & Implement succession planning and formal project documentation & L & L & M \\  \colhline
RT-SP-APA-02 & Delay in finalisation of APA frame design & Close oversight on prototypes and interface issues & L & L & M \\  \colhline
RT-SP-APA-03 & One additional pre-production APA may be necessary & Close oversight on approval of designs, commissioning of tooling and assembly procedures & L & L & L \\  \colhline
RT-SP-APA-04 & APA winder construction takes longer than planned & Detailed plan to stand up new winding machines at each facility & M & L & M \\  \colhline
RT-SP-APA-05 & Poor quality of APA frames and/or inaccuracy in the machining of holes and slots & Clearly specified requirements and seek out backup vendors & L & L & M \\  \colhline
RT-SP-APA-06 & Insufficient scientific manpower at APA assembly factories & Get institutional commitments for requests of necessary personnel in research grants & M & M & L \\  \colhline
RT-SP-APA-07 & APA production quality does not meet requirements & Close oversight on assembly procedures & L & M & M \\  \colhline
RT-SP-APA-08 & Materials shortage at factory & Develop and execute a supply chain management plan & M & L & L \\  \colhline
RT-SP-APA-09 & Failure of a winding machine - Drive chain parts failure & Regular maintenance and availability of spare parts & L & L & L \\  \colhline
RT-SP-APA-10 & APA assembly takes longer time than planned  & Estimates based on protoDUNE. Formal training of every tech/operator & L & M & M \\  \colhline
RT-SP-APA-11 & Loss of one APA due to an accident & Define handling procedures supported by engineering notes & M & L & L \\  \colhline
RT-SP-APA-12 & APA transport box inadequate & Construction and test of prototype transport boxes & L & L & M \\  \colhline
RO-SP-APA-01 & Reduction of the APA assembly time & Improvements in the winding head and wire tension mesurements & M & M & M \\  
 &  &  &  &  &  \\  \colhline

\label{tab:risks:SP-FD-APA}
\end{longtable}
\end{footnotesize}


Risks with larger probability and/or larger impact are discussed in more detail below:
\begin{itemize}
\item RT-SP-APA-01, Loss of key personnel:
\begin{itemize}\setlength\itemsep{-0.8em}
\item \textit{Description}: If loss of key personnel happens, it will cause delays as knowledge is lost and new team members will need to come up to speed.
\item \textit{Mitigation}: Implement succession planning and formal project documentation at all stages. All key tasks to be shared between multiple people, including factory management.
\item \textit{Probability and impact}: While the post-mitigation probability is low, below 10\%, if the risk is realized, the impact on the schedule could range from a couple of months to a half-year.
\end{itemize}

\item RT-SP-APA-02, Delay in finalization of \dword{apa} frame design:
\begin{itemize}\setlength\itemsep{-0.8em}
\item \textit{Description}: If problems are encountered with the \dword{apa} doublet frame assembly and cabling tests at \dword{ashriver}, or with the integration test of the pre-production \dword{apa} at \dword{cern}, this will delay the finalization of the \dword{apa} frame design.
\item \textit{Mitigation}: Oversight of the \dword{apa} Consortium on the schedule of components procurement for the \dword{ashriver} tests and close coordination with the \dword{ashriver} team. Close coordination with \dword{ce} and \dword{pds} Consortia on all interface issues, to be formalized in the Interface Documents.
\item \textit{Probability and impact}: On the basis of the work done up to now we believe that the probability of this risk is low. Anyhow, if materialized, it would imply a delay in the start of \dword{apa} production from a couple of months to a half-year.
\end{itemize}

\item RT-SP-APA-04, \dword{apa} winders construction takes longer than planned:
\begin{itemize}\setlength\itemsep{-0.8em}
\item \textit{Description}: If the construction of the winding machines takes longer than planned, the schedule for \dword{apa} production will be delayed, and additional labor for winders production will be needed. We plan the construction of four additional winders in the USA and the modification of the present winder at PSL, and the construction of three additional winders in the UK, in addition to the modification and relocation of the present winder at Daresbury Lab. The estimated time for the production of the additional winders is approximately one year, both in the UK and the USA.
\item \textit{Mitigation}: Get commitments from the relevant institutions for the necessary resources for winder production, both for space and skilled manpower availability. Develop and execute a detailed plan to set up new winding machines at each facility. This plan will include contingencies in the event that technical problems cause schedule delays.  
\item \textit{Probability and impact}: Winders are complex machines, and we estimate a Medium probability for this risk, of less than 25\%. The impact on the schedule is also Medium, with possible delays up to a half-year.
\end{itemize}

\item RT-SP-APA-05, Poor quality of \dword{apa} frames and/or inaccuracy in the machining of holes and slots:
\begin{itemize}\setlength\itemsep{-0.8em}
\item \textit{Description}: \dword{apa} frames are constructed from structural stainless steel tubing. The quality of the material provided by the vendor may change with time and be outside the required tolerances. Problems with QA during machining of holes and slots may result in unusable products. If this happens, it may delay the supply of frames of sufficient quality and it would delay the \dword{apa} construction schedule.
\item \textit{Mitigation}: All requirements must be clearly specified in the purchase contracts. We will establish a well managed relationship with a vendor to provide the stainless steel tubing and the machining for the components of an \dword{apa} frame. In addition, through our prototyping efforts we will seek out at least one solid backup vendor for material supply and machining in both the USA and UK.
\item \textit{Probability and impact}: While the post-mitigation probability is low, below 10\%, if the risk is realized, the impact on the schedule is Medium, since finding a new vendor may take up to a half-year.
\end{itemize}

\item RT-SP-APA-06, Insufficient scientific manpower at \dword{apa} assembly factories:
\begin{itemize}\setlength\itemsep{-0.8em}
\item \textit{Description}: Out of 5.5 \dword{fte} necessary at each production line, 2 \dword{fte} are assumed to be uncosted scientific manpower. If it is not possible to recruit scientific resources, costed professional manpower is needed and costs will increase. This risk applies to the USA project only, as in the UK the required scientific staff is costed and awarded on project.
\item \textit{Mitigation}: Proactively contact institutions and get their commitments for inclusion of the necessary personnel in their research grants.
\item \textit{Probability and impact}: This is a Medium probability risk, we estimate a 20\% probability that 50\% of the scientific resources may be missing. The cost impact is also Medium, up to 20\%.
\end{itemize}

\item RT-SP-APA-07, \dword{apa} production quality does not meet requirements:
\begin{itemize}\setlength\itemsep{-0.8em}
\item \textit{Description}: If wire planes are outside the required tolerances, they will need to be reworked and the \dword{apa} production schedule will be affected. A point of concern is to stay within limits for the tension of the wires. 
\item \textit{Mitigation}: The overall quality of each constructed \dword{apa} will be ensured by following detailed procedures for every step of the assembly process (e.g. mesh installation, board placement and gluing, soldering, wire winding, etc.).  These procedures already exist from our \dword{pdsp} work and are in the process of being modified to the final design of the DUNE \dword{fd} \dword{apa}s. For critical steps, an operator and quality control representative will record information in travelers for each \dword{apa}. 
\item \textit{Probability and impact}: Given the \dword{pdsp} experience and the steps outlined in the mitigation strategy, we can keep this risk probability low, below 10\%. If realized, we assume a maximum impact on cost and schedule of 20\%, corresponding to a Medium impact.
\end{itemize}

\item RT-SP-APA-08, Materials shortage at factory:
\begin{itemize}\setlength\itemsep{-0.8em}
\item \textit{Description}: A material shortage at a factory would delay production.
\item \textit{Mitigation}: As part of our comprehensive production strategy we are in the process of developing and executing a supply chain management plan. This plan will include the details of material source, delivery logistics, critical milestones, and personnel resources required to meet factory needs for efficient \dword{apa} production. All suppliers (vendors, labs, academic institutions) will be included in the implementation of the supply chain plan. A key part of this plan will be the establishment of supplier metrics that will be gathered and reported to DUNE management by the \dword{apa} production manager. These metrics will serve as an early warning of potential problems and trigger mitigation efforts early in the cycle. 
\item \textit{Probability and impact}: Even with mitigation, this is a realistic risk with an estimated probability of up to 25\%. Delays on the schedule would probably not exceed a couple of months, making the impact Low.
\end{itemize}

\item RT-SP-APA-10, \dword{apa} assembly takes longer than planned:
\begin{itemize}\setlength\itemsep{-0.8em}
\item \textit{Description}: If the labor for \dword{apa} assembly is underestimated, it will correspondingly lengthen the time to produce \dword{apa}s. We estimate an upper limit on the additional required labor of 10\%. 
\item \textit{Mitigation}: \dword{apa} assembly time estimates have been based on \dword{pdsp} experience and improvements to the winding machine. Formal training of every technician/operator of the winding machine in order to maintain a high production efficiency. 
\item \textit{Probability and impact}: We believe that this risk probability is low, below 10\%, anyhow even a 10\% increase in the required labor would have a Medium impact on both cost and schedule.
\end{itemize}

\item RT-SP-APA-11, Loss of an \dword{apa} due to an accident:
\begin{itemize}\setlength\itemsep{-0.8em}
\item \textit{Description}: If during \dword{apa} assembly or integration/installation an accident happen, this may  cause the destruction of an \dword{apa}. We already plan to build two spare \dword{apa}s for each detector module. In addition, we assume a probability of up to 10\% to lose an additional \dword{apa} during assembly, or integration/installation. 
\item \textit{Mitigation}: We define procedures to handle \dword{apa}s at all stage of fabrication, integration/installation, together with associated engineering notes for all modes of handling. Safety is a primary concern, and the probability of an accident involving personnel is negligible. On the other hand wire planes are delicate, and once damaged they would not be repairable. 
\item \textit{Probability and impact}: This is a marginally Medium risk, with low impact both on cost and schedule.
\end{itemize}

\item RT-SP-APA-12, \dword{apa} transport box inadequate:
\begin{itemize}\setlength\itemsep{-0.8em}
\item \textit{Description}: The \dword{apa} transport box needs to provide safe and affordable transport of \dword{apa}s from production sites in the UK and USA to \dword{surf}. It will also need to ensure safe transportation of \dword{apa}s from the surface site at \dword{surf} down the Ross shaft to the detector underground location. If the transport box will not provide enough mechanical protection for the safe transportation of the \dword{apa}s, or if the size of the box is inadequate for transfer to the underground location, it will impact the schedule.
\item \textit{Mitigation}: Construction and test of prototype transport boxes. Tests at \dword{ashriver} and \dword{surf} of all handling steps. Close coordination with the team at \dword{surf} and \dword{apa} Consortium oversight on the transport box effort.
\item \textit{Probability and impact}: Given the mitigation steps, we estimate a probability for this risk of less than 10\%. In case of redesign, it may have a Medium impact on the schedule.
\end{itemize}

\item RO-SP-APA-01, Reduction of the \dword{apa} assembly time:
\begin{itemize}\setlength\itemsep{-0.8em}
\item \textit{Description}: If the new winding head will provide a better uniformity in wire tension, it will reduce the time necessary for re-tensioning of the wires. If the new electrical method for wire tension measurements will work as planned, it will reduce substantially the time required for wire tension measurements. A saving of up to 10\% in \dword{apa} assembly time will be possible with these improvements. 
\item \textit{Opportunity}: This is an opportunity that the \dword{apa} Consortium wants to pursue with close oversight on new winding head tests and electronic tension measurements effort. \dword{apa} boards have been already redesigned to allow for electronic tension measurements. 
\item \textit{Probability and impact}: Given the preliminary results obtained up to now, we estimate a Medium probability for this opportunity. A saving of 10\% in the \dword{apa} assembly time could be realized, corresponding to a Medium impact for both cost and schedule.
\end{itemize}

\end{itemize}



