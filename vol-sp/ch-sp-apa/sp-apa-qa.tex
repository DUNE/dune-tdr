%%%%%%%%%%%%%%%%%%%%%%%%%%%%%%%%%%%%%%%%%%%%%%%%%%%%%%%%%%%%%%%%%%
\section{Quality Assurance}
\label{sec:fdsp-apa-qa}


The most important and complete information for assuring the quality of the \dword{apa} design, components, materials, and construction methods comes from the construction and operation of \dword{pdsp}.  We have learned much about the design and fabrication procedures that has informed the detailed design and plans for the DUNE \dword{apa} construction project. \dword{pdsp} included six full-scale \dword{dune} \dword{apa}s instrumenting two drift regions around a central cathode.  Four of the \dword{pdsp} \dword{apa}s were constructed in the USA at the University of Wisconsin-PSL, and two were made at Daresbury Laboratory in the UK. All were shipped to \dword{cern}, integrated with \dwords{pd} and \dword{ce}, and tested in a \coldbox prior to installation into the \dword{pdsp} cryostat.  Figure~\ref{fig:sp-apa-pd-photo} shows one of the drift regions in the fully constructed \dword{pdsp} detector.

\begin{dunefigure}[A drift region in ProtoDUNE-SP with installed APAs]{fig:sp-apa-pd-photo}
{One of the two drift regions in the \dword{pdsp} detector at \dword{cern} showing the three installed \dword{apa}s on the left.}
\includegraphics[width=0.9\textwidth]{sp-apa-protodune-photo.jpg}
\end{dunefigure}


\subsection{Results from ProtoDUNE-SP Construction}
\label{sec:fdsp-apa-qa-protodune-const}

A thorough set of \dword{qc} tests were performed and documented throughout the fabrication of the \dword{pdsp} \dword{apa}s.  The positive outcomes give great confidence in the quality of the overall \dword{apa} design  and construction techniques.  Here we summarize some of the testing that was done for \dword{pdsp} and the results.   


After each wire layer was applied to an \dword{apa}, electrical continuity between the head and foot boards was checked for each wire.  This test is most useful for the $U$ and $V$ layers, where metal traces on the side boards can be damaged during construction. All boards were visually inspected as construction proceeded.

Channels were checked for isolation.  In the beginning, wire-to-wire isolation was measured over a long period of time, and no problems arose.  In the end, each wire was individually hipot tested (a dielectric withstand test) at \SI{1}{kV}. No failures were ever seen. However, leakage currents were seen to be highly dependent on relative humidity.  The surface of the epoxy has some affinity for moisture in the air and provides a measurable leakage path when relative humidity exceeds about \SI{60}{\%}. Tests have confirmed that in a dry environment, such as the \dword{lar} cryostat, these leakage currents disappear.

Wire tension was measured for all wires at production.  Figure~\ref{fig:sp-apa-pd-tension-prod} displays the measured tensions for wires on the instrumented wire planes ($X, U, V$) for the six \dword{pdsp} \dword{apa}s, four constructed at PSL in the US and two at Daresbury Laboratory in the UK.  %The number of wires below and above the tension specification are shown in Table \ref{tab:nwires}. 
In total, \SI{4.4}{\%} of the \num{14972} wires considered for this analysis had a tension below \SI{4}{N}, and \SI{22.5}{\%} were above \SI{6}{N}. 
A wire which has a tension higher than specification should not impact the physics in any meaningful way. Wires with tension lower than specification could move slightly out of position and impact detector function primarily through modifying the local \efield. \efield modification can lead to the number of ionization electrons being incorrectly reconstructed in the deconvolution process or alter the transparency so that less than \SI{100}{\%} of the ionization electrons reach the collection plane. Because these processes change the amount of reconstructed charge, they would alter the reconstruction of the energy deposited by charged particles near these wires. A further complication from very low-tension wires might be an increase in noise level, introduced by wire vibrations, which can lead to vortex shedding.  Each of these impacts is expected to be quite small, but to confirm, cosmic muon tracks in \dword{pdsp} data are now being used to test if differences in response can be seen on wires with particularly low tension.  The target tension for \dword{dune} \dword{apa}s has already been increased to \SI{6}{N}, and these \dword{pdsp} studies will quantitatively inform a minimum tension requirement, but no challenges in meeting specifications are foreseen based on current knowledge from \dword{pdsp} construction.   



\begin{dunefigure}[ProtoDUNE-SP wire tensions as measured during production]{fig:sp-apa-pd-tension-prod}
{Distributions of wire tensions in the \dword{pdsp} \dword{apa}s for wires longer than \SI{70}{cm}, as measured during production at PSL and Daresbury. For the $X$-plane, every wire has the same length (\SI{598.39}{cm}), and so every wire is included.  
The histograms for the six \dword{apa}s are stacked. 
}
\includegraphics[height=0.28\textheight,trim=0mm 0mm 0mm 0mm,clip]{graphics/sp-apa-X-layer-tensions.pdf}
\includegraphics[height=0.28\textheight,trim=0mm 0mm 0mm 0mm,clip]{graphics/sp-apa-V-layer-tensions.pdf}
\includegraphics[height=0.28\textheight,trim=0mm 0mm 0mm 0mm,clip]{graphics/sp-apa-U-layer-tensions.pdf}
\end{dunefigure}

Wire tension measurements were also performed for a subset of wires on each \dword{apa} after arriving at \dword{cern}. Figure~\ref{fig:sp-apa-pd-tension-cern} shows the comparison of tension values measured at \dword{cern} versus at the production site for the selected subset of wires from each wire plane. Based on the traveler documents provided by the production sites, wires having outlier tension values were selected from each \dword{apa} for re-measurement at \dword{cern}. In addition, a set of randomly selected wires from each plane was measured. In total, for six \dword{apa}s, $\sim$1500 wires had their tension re-measured at \dword{cern}. Measurements took place in the clean room with \dword{apa}s hanging vertically, the first time the tensions were sampled in this orientation. Tension measurements were performed by using the laser-photodiode based method, the same as at the production sites. 

\begin{dunefigure}[ProtoDUNE-SP wire tension measurement plots]{fig:sp-apa-pd-tension-cern}
{Comparison of wire tensions upon arrival at \dword{cern} versus at the production sites for a sample of wires on each of the \dword{pdsp} \dword{apa}s.}
\mbox{\includegraphics[height=0.23\textheight]{sp-apa-PD-tension-prod-cern-G.pdf} %\hspace{1mm}
\includegraphics[height=0.23\textheight]{sp-apa-PD-tension-prod-cern-U.pdf}} \\
\vspace{3mm}
\mbox{\includegraphics[height=0.23\textheight]{sp-apa-PD-tension-prod-cern-V.pdf} %\hspace{1mm}
\includegraphics[height=0.23\textheight]{sp-apa-PD-tension-prod-cern-X.pdf}}
\end{dunefigure}

Finally, to test if a cold cycle had any effect on the wire tension, samples of wires were measured again after the \coldbox tests at \dword{cern}. This is the only tension data we have after a cold-cycle for \dword{pdsp} \dword{apa}s. Figure~\ref{fig:sp-apa-pd-tension-cold} presents the results, showing no significant change in the resonant frequency of the wires, indicating cold cycle does not have a significant effect on wire tension.

\begin{dunefigure}[ProtoDUNE-SP wire tension before and after cold tests]{fig:sp-apa-pd-tension-cold}
{Comparison of wire tensions after the \coldbox test versus before at \dword{cern} for a sample of wires on each of the \dword{pdsp} \dword{apa}s.}
\includegraphics[height=0.3\textheight]{sp-apa-PD-tension-aftercold.pdf} 
\end{dunefigure}

\subsection{Results from ProtoDUNE-SP Operation}
\label{sec:fdsp-apa-qa-protodune-ops}


Analysis of \dword{pdsp} data is ongoing and will continue through 2019.  Several useful analyses for evaluating the \dword{apa} design have been carried out including monitoring the number of non-responsive or disconnected channels in the detector, studying the impact of the electron diverters on reconstruction and calorimetry, and measuring the change in electron transparency with wire bias voltage.  The status of these studies is presented below.   
\fixme{still ongoing?}


\subsubsection{Disconnected Channels}
\label{sec:fdsp-apa-qa-protodune-ops-dead-channels}



\dword{apa} channels with a ``broken connection'' can be identified in \dword{pdsp} data by comparing channels that do not record hits during detector runs against channels that do respond to the internal calibration pulser system on the \dword{femb}s.  If pulser signals are seen on a channel with no hits, this most likely points to a mechanical failure in the wire path to the electronics.  The failure could be, for example, at a bad solder connection, a damaged trace on a wire board, or a faulty connection between a wire, \dword{cr}, and \dword{ce} adapter boards. Studies have been done using data throughout the \dword{pdsp} run, looking for channels non-responsive to ionization. Note that this analysis is insensitive to the $X$-plane wires that face the cryostat walls since no ionization arrives at those wires.

The results show a very low count of permanently disconnected channels in the \dword{pdsp} \dword{apa}s (28 channels out of 12,480 channels facing the drift volume). In addition, we identified 21 channels that are intermittently not responsive, most probably due to \dword{apa} problems. This is summarized in tables~\ref{tab:deadchan1} and \ref{tab:deadchan2}.  
The fractions of disconnected and intermittent channels are low, 0.22\% and 0.17\%, respectively. 

\begin{dunetable}[Disconnected channel counts in ProtoDUNE-SP]{lcccccc}{tab:deadchan1}{Summary of disconnected channels per plane in \dword{pdsp} due to mechanical failures in the \dword{apa}s.}
 & $U$-plane & $V$-plane & $X$-plane & Total Channels & ~~~Rate~~~ & Total \\ \toprowrule
Disconnected & 16 & 8 & 4 & \multirow{2}{*}{\num{12480}} & 0.22\% & \multirow{2}{*}{0.39\%} \\
Intermittent & 7 & 7 & 7 &  & 0.17\% & \\
\end{dunetable}

\begin{dunetable}[Dead channel counts in ProtoDUNE-SP]
{lcccccc}
{tab:deadchan2}
{Summary of disconnected channels per \dword{apa} in \dword{pdsp} due to mechanical failures in the \dword{apa}s.}
& APA 1 & APA 2 & APA 3 & APA 4 & APA 5 & APA 6 \\ \toprowrule
Disconnected & 4 & 5 & 8 & 3 & 1 & 7  \\
Intermittent & 10 & 0 & 1 & 4 & 3 & 3  \\
\end{dunetable}

So far, analysis of data throughout the \dword{pdsp} run shows no evidence of increasing numbers of disconnected or intermittent channels.


\subsubsection{Effect of Electron Diverters on Charge Collection}
\label{sec:fdsp-apa-qa-protodune-ops-electron-diverters-charge}

Active strip-electrode electron diverters were installed in \dword{pdsp} between \dword{apa}s~1 and~2 (ED12) 
and between \dword{apa}s~2 and~3 (ED23), which are both on the beam-right side of \dword{pdsp} for 
the 2018-2019 run.  The two inter-\dword{apa} gaps on the beam-left side did not have electron diverters in them. ED12 developed an electrical short early in the run, and as a consequence, both ED12 and ED23 were left unpowered for the beam run and all but a small number of test runs after the beam run.  A voltage divider on the electron diverter \dword{hv} distribution board provided a path to ground, and so the electron diverter strips were effectively grounded.  Since they protrude into the drift volume in front of the \dword{apa}s, the grounded diverters collect nearby drifting charge instead of diverting it towards the active apertures of the \dword{apa}s, %.  Event displays from \dword{pdsp} show 
leading to broken tracks with charge loss in the gaps.  When powered properly, charge is primarily displaced away from the gap, and tracks that are more isochronous provide good measurements of the charge arrival time delays due to the longer drift paths of diverted charge. Figure~\ref{fig:sp-apa-diverterevd100} shows the collection-plane view of the readout of \dword{apa}s~3 and~2 for a test run in which ED23 was powered at its nominal voltage. Figure~\ref{fig:sp-apa-nodiverterevd} shows the collection-plane view of a track crossing the drift volumes read out by \dword{apa}s~6 and~5, which do not have an electron diverter installed between them.  Timing and spatial distortions in the absence of diverters appear minimal.

The impact of charge distortions can be seen in Figure~\ref{fig:sp-apa-qc-diverterdqdx}, which shows the average $dQ/dx$ distributions for \dword{pdsp} run 5924, which has ED12 at ground voltage, ED23 at nominal voltage, and no diverters on the beam-left side of the detector between \dword{apa}s~4, 5, and~6.  Pronounced drops in the charge collected near ED12 (grounded diverter) are seen, while much smaller distortions are seen elsewhere.  Run 5924 was taken while the grid plane in \dword{apa}~3 was charging up, resulting in artifacts in the $dQ/dx$ measurements with a period of three wires.  \dword{apa}~2 has an artifact from an \dword{asic} with a slightly different gain reading out channels near the boundary with \dword{apa}~1, causing even and odd channels to be offset.

\begin{dunefigure}[\dshort{pdsp} event display; impact of a grounded electron diverter]{fig:sp-apa-diverterevd100}{Collection-plane charge signals in ProtoDUNE-SP for a single readout window in \dword{apa}s~3 (left) and~2 (right) for a test run in which ED23 was powered at its nominal voltage.  The horizontal axis is wire number, arranged spatially along the beam direction, and the vertical axis is readout time.  The event is run 5924, event 275.}
    \includegraphics[width=\textwidth,trim=85mm 0mm 85mm 0mm,clip]{graphics/sp-apa-diverter100percent-R5924-E275-T1T5.png}
\end{dunefigure}

\begin{dunefigure}[\dshort{pdsp} event display; track crossing gap without electron diverter]{fig:sp-apa-nodiverterevd}{Collection-plane event display for \dword{apa}s~6 (left) and~4 (right). No electron diverter was installed between these two \dword{apa}s.  The event is run 5439, event 13.}
    \includegraphics[width=\textwidth,trim=85mm 0mm 85mm 0mm,clip]{graphics/sp-apa-nodiverterevd.png}
\end{dunefigure}

\begin{dunefigure}[$dQ/dx$ distributions for \dshort{pdsp} with different diverter conditions]
{fig:sp-apa-qc-diverterdqdx}{The $dQ/dx$ distributions as a function of the collection wire number zoomed in near the gaps, using cosmic ray muons in \dword{pdsp} run 5924. The electron diverters are only instrumented for the gaps at the beam right side ($x<0$). The electron diverter between \dword{apa}~2 and \dword{apa}~3 was running at the nominal voltage while the electron diverter between \dword{apa}1 and \dword{apa}~2 was turned off. }
\includegraphics[width=0.48\textwidth]{graphics/sp-apa-dQdxAPA12.png}
\includegraphics[width=0.48\textwidth]{graphics/sp-apa-dQdxAPA32.png}
\includegraphics[width=0.48\textwidth]{graphics/sp-apa-dQdxAPA46.png}
\includegraphics[width=0.48\textwidth]{graphics/sp-apa-dQdxAPA56.png}
\end{dunefigure}
    
%\fixme{Should we add an event display with the diverters grounded?}



\subsubsection{Effect of Wire Support Combs on Charge Collection}
\label{sec:fdsp-apa-qa-protodune-ops-combs-charge}

Inclusive distributions of charge deposition on each channel can be made with \dword{pdsp} data using the cosmic-ray tracks.  Tracks that cross from the cathode to the anode have unambiguous times even without association with \dwords{pd}, and thus distance-dependent corrections to the lifetime can be made.  The reconstruction of tracks in three dimensions makes use of the charge deposited in each of the three wire planes.  Maps of the median $dQ/dx$ response have been made for each plane in each \dword{apa} in the $(y,z)$ plane, the plane in which the \dword{apa} resides.  The granularity of these maps is the wire spacing, in both dimensions, and so the charge response of small segments of wires is measured.  These maps are projected onto the $U$, $V$, $y$, and $z$ coordinate axes in order to visualize more easily the impacts of localized detector inhomogeneities.

The wire-support combs are approximately evenly spaced in the $y$ coordinate.  In order to investigate the impact of the wire combs on charge collection and induction signals, the average of the median binned $dQ/dx$ values as a function of $y$ is shown for $U$, $V$, and collection-plane ($Z$) wires in Figure~\ref{fig:sp-apa-pd-comb-charge-impact}.  \dword{apa}~6, which is in the middle of the detector and thus is minimally affected by features on the neighboring field cages, is chosen so the effects of the combs are most visible, though similar effects are seen in all six \dword{apa}s in \dword{pdsp}.  Localized dips of the order of 2\% in the average signals can be seen at the locations of the combs in the $U$ and $V$ views, while the collection-plane channels show smaller dips and other features.  Charge is expected to divert around the dielectric combs after they charge up, and if the diversion is purely in the vertical direction, then the impact on the collection-plane response is expected to be suppressed.  The induction-plane response may be understood as the result of the dielectric comb locally polarizing in the field of the drifting charge, thus modifying the \efield at the wires.  This analysis well exhibits the uniformity of the response of the \dword{pdsp} \dword{apa}s as well as the level of detail that can be extracted from \dword{tpc} data for the precise calibration of the \dwords{spmod}.

\begin{dunefigure}[Average charge deposition on tracks vs. height]{fig:sp-apa-pd-comb-charge-impact}
{Average $dQ/dx$ on the $U$, $V$, and collection-plane ($Z$) wires in \dword{apa}~6 as a function of the height $y$ from the bottom of the \dword{pdsp} detector. }
\includegraphics[width=0.32\textwidth]{graphics/sp-apa-comb-apa6sUbY.pdf}
\includegraphics[width=0.32\textwidth]{graphics/sp-apa-comb-apa6sVbY.pdf}
\includegraphics[width=0.32\textwidth]{graphics/sp-apa-comb-apa6sZbY.pdf}   
\end{dunefigure}

\subsubsection{Wire Bias Voltage Scans and Electron Transparency}
\label{sec:fdsp-apa-qa-protodune-ops-bias-scans}

%\fixme{can we update this plot to go to R=1.0?}

A set of dedicated runs were taken at \dword{pdsp} in order to confirm the bias voltage settings calculated by the COMSOL software %~\cite{COMSOL} 
and presented in Section~\ref{sec:fdsp-apa-design-overview}. In particular, the bias voltages in the $G$ (grid), (induction) $U$, and (collection) $X$ wire plane were uniformly reduced from 5\% to 30\% relative to the nominal settings. For each wire plane, the transparency condition depends on the ratio of the \efield before and after the wire plane. Therefore, in the situation of uniform reduction of the bias voltages, some ionization electrons are expected to be collected by the grid plane, leading to a loss of ionization electrons collected by the $X$ wires. Figure~\ref{fig:protodune-bias-voltage-scan} shows the results from each of six \dword{apa}s in \dword{pdsp}. The ratio ``R'', ranging from 0.7 (30\% reduction) to 0.95 (5\% reduction), represents the different bias voltage settings used in these runs. ``T'' represents the transparency of the ionization electrons, which is proportional to the number of ionization electrons collected by the $X$ wire plane. As a result of the significant space charge effect in \dword{pdsp}, the sources of ionization electrons (presumably dominated by cosmic muons) are different for different \dword{apa}s. To facilitate the comparison among different \dword{apa}s, the transparency at each bias voltage setting is normalized by the transparency at the highest bias voltage setting (R=0.95). Except for \dword{apa}~3, all \dword{apa}s show a similar trend in the change of transparency. The spread represents the uncertainty in calculating the transparency. The grid plane of \dword{apa}~3 was found to be disconnected since December 2018, which led to incorrect bias voltage settings in these runs. This explained the abnormal behavior in its transparency data. Two sets of predictions (COMSOL vs. Garfield) are compared with the \dword{pdsp} data. The ranges of R in these predictions are different from that of the \dword{pdsp} data, since these two predictions were obtained prior to the \dword{pdsp} data taking. The COMSOL prediction is clearly confirmed by the \dword{pdsp} data, which also validates the nominal bias voltage settings listed in Section~\ref{sec:fdsp-apa-design-overview}. The incorrect prediction from the Garfield simulation is attributed to inaccurate \efield calculations near the boundary of the wires (\SI{152}{$\mu$m} diameter), which is much smaller than the wire pitch ($\sim$\SI{4.79}{mm}). 

\begin{dunefigure}[ProtoDUNE bias voltage scan data]{fig:protodune-bias-voltage-scan}
{The transparency results from the bias voltage scan in \dword{pdsp}. "R", the ratio to the nominal bias voltages, represents different bias voltage settings. "T" represents the transparency of the ionization electrons, which is proportional to the number of ionization electrons collected by the $X$ wire plane. The prediction of COMSOL (Garfield) is confirmed (refuted) by the \dword{pdsp} data. The abnormal behavior of \dword{apa}~3 is a result of incorrect bias voltage settings. See Section~\ref{sec:fdsp-apa-qa-g-plane} for more discussion.}
\includegraphics[width=0.75\textwidth]{graphics/protodune-bias_voltage_scan_group1.pdf}   
\end{dunefigure}

\subsubsection{Abnormal Behavior of G-plane on APA 3}
\label{sec:fdsp-apa-qa-g-plane}

Dedicated studies of $dQ/dx$, the recorded ionization charge per unit path length from cosmic muon tracks, have been performed for each \dword{apa}. For runs immediately after periods when the cathode \dword{hv} was off for an extended length of time, of the order of a few days, the average of the $dQ/dx$ distribution on \dword{apa}~3 collection and induction planes was found to be systematically lower than for the other \dword{apa}s.  The $dQ/dx$ would then slowly increase with time.  Detailed investigations showed that this behavior is explained by the assumption that the $G$-plane on \dword{apa}~3 is not connected to a proper reference voltage. When the cathode \dword{hv} is turned on after a long off period, the $G$-plane, initially at a floating potential close to ground, slowly charges up towards a negative \dword{hv},  re-establishing transparency for the ionization electrons towards the signal planes. It takes about 100 hours for the $G$-plane to reach a negative potential close to the nominal value that allows full transparency.

We are presently evaluating more accessible locations for the connection of the bias \dword{hv} cables from the cryostat feedthroughs to the \dword{apa}s, to minimize connection problems with the \dword{shv} connectors. In addition, during installation, we will include as part of the standard checkout procedure either a direct confirmation of the bias connection between a wire plane and its bias input on the feedthrough flange, or an indirect measurement of the connection by recording the charging current in the bias line when increasing the bias voltage to its nominal value.  The construction and integration tests with a pre-production \dword{apa}, described below in Sec.~\ref{sec:fdsp-apa-qa-prototyping}, will fully test any changes to the \dword{shv} system. 


\subsection{Final Design Prototyping and Test Assemblages}
\label{sec:fdsp-apa-qa-prototyping}


To confirm modifications made to the \dword{apa} design and production process since \dword{pdsp} and to work through the multi-\dword{apa} assembly procedures, several prototypes are planned for 2019--2020.
\fixme{dates}

A seventh \dword{pdsp}-like \dword{apa} was completed at Daresbury Laboratory by utilizing an upgraded winding machine with the new interface arm design (see Section~\ref{sec:fdsp-apa-prod-tooling}). This \dword{apa} was shipped to \dword{cern} in March 2019 for a test of the \dword{ce} in the \coldbox, expected to be performed in 2019. In addition, work is in progress to implement a new winding head on the \dword{apa} winding machines, with automatic tension feedback and control on the wires. These same upgrades will be implemented, by end of summer 2019, on the winding machine at PSL. 
\fixme{dates}

A top and bottom version of the new supporting \dword{apa} frame design were built in spring 2019 at PSL and shipped to \dword{ashriver}.  A full test of the \dword{apa} pair assembly procedure was successfully completed in early October 2019 (see Figure 1.27). The procedure of routing the \dword{ce} cables along the side tubes of the \dword{apa} pair was also successfully tested. In addition, a preliminary test of the installation of \dword{pds} prototype cables inside the \dword{apa} frames and the mating of cable connections between the lower and upper \dword{apa} was performed.  See Section 1.4.3 for more information on cable routing in the \dword{apa} frames. 


Also planned is the construction of a pre-production \dword{apa} for an integration test with the \dword{ce} and \dword{pds} systems at \dword{cern}, which will fully test all interface aspects. This test will inform the final design review of the \dword{apa} system in May 2020. 

In addition, three fully wound \dword{apa}s with pre-installed \dword{pds} cables, will be built by the end of 2020 for deployment in \dword{pdsp}-II, replacing the detectors of one of the drift volumes.  This will allow a test of all \dword{apa} components, including the larger size frames and geometry boards and a final tuning of the winding machines. The three \dword{apa}s will be shipped to \dword{cern}, integrated with \dword{ce} boxes and \dword{pds} detectors, and tested in the \coldbox before installation in \dword{pdsp}.  The pre-production \dword{apa} mentioned above could serve as one of the three if no design modifications are required.  These final prototyping activities will serve to test all critical aspects of the \dword{apa} design before starting \dword{dune} \dword{apa} production in 2020.  

\begin{dunefigure}[Ash River APA pair integration tests]{fig:ashriver-integration}
{\dword{apa} pair assembly and integration tests at \dword{ashriver}.}
\includegraphics[height=0.5\textheight]{graphics/sp-apa-ash-river-ladder.jpg} \quad
\includegraphics[height=0.5\textheight]{graphics/sp-apa-ash-river-doublet.png} 
\end{dunefigure}