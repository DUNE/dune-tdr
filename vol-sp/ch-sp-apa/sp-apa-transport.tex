%%%%%%%%%%%%%%%%%%%%%%%%%%%%%%%%%%%%%%%%%%%%%%%%%%%%%%%%%%%%%%%%%%%%
\section{Handling and Transport to SURF} % and Receiving at the DUNE Integration Facility}
\label{sec:fdsp-apa-transport}

Completed \dwords{apa} are shipped from the \dword{apa} production sites to South Dakota for integration with the \dword{tpc} \dword{fe} electronics and \dwords{pd} followed by installation in the \dword{dune} cryostats.  %The \dword{itf} location is not yet confirmed, but facilities near the \surf site are being considered.   
Extensive \dword{qc} testing will be performed before installation to ensure the fully integrated \dword{apa}s function properly.  %Once checked, the \dword{apa}s are repackaged for final transport to \surf.  
For details on installation activities at \surf, see Chapter~\ref{ch:sp-install}. 

%%%%%%%%%%%%%%%%%%%%%%%%%%%%%%%%%%%%%%%%%%%%%%%%%%%%%%%%%%%%%%%%%%%%
\subsection{APA Handling}
\label{sec:fdsp-apa-transport-handling}

The handling of the \dword{apa}s throughout their lifetime must be carefully considered to ensure their safety.  Several lifting and handling fixtures will be employed for transferring and manipulating the \dword{apa}s during fabrication, integration, and installation.  At the production sites a fixture called the edge lift kit will be used to transfer the \dword{apa} to and from the process cart and the winder, as well as to the transport containers.  The lift kit is shown schematically in Figure~\ref{fig:apa-edge-lift}.  It is essential that the fixture connects to the \dword{apa} along an outer edge since once wires are attached to the support frame, it can no longer be grabbed anywhere on the front or back face of the frame. 
%The edge lift kit will also be used to transfer the \dword{apa} from the transport frame to a process cart there.  

\begin{dunefigure}[APA edge lifting fixture]{fig:apa-edge-lift}
{A custom lifting fixture is used to pick up an \dword{apa} from the long edge and safely handle it during the various construction steps at the production sites.}  
\includegraphics[width=0.8\textwidth]{sp-apa-edge-lift-kit.png} 
\end{dunefigure}



%%%%%%%%%%%%%%%%%%%%%%%%%%%%%%%%%%%%%%%%%%%%%%%%%%%%%%%%%%%%%%%
\subsection{APA Transport Frame and Shipping Strategy}
\label{sec:fdsp-apa-transport-container}


The transport packaging for the \dword{apa}s is designed to safely transport them from the production sites to the underground clean room in the detector hall at \dword{surf}. 
The design of the packaging is shown in Figure~\ref{fig:apa-transport-frame}. Light rigid metalized foam protective panels are attached via clamps affixed to the \dword{apa} frames and provide the primary protection for the wire planes. Pairs of \dword{apa}s (one upper and one lower in an \dword{apa} pair) are loaded onto welded structural steel transport frames at the factory. The \dword{apa} frames are bolted to mounts on the transport frames that incorporate shock attenuating coil springs designed to reduce possible accelerations on the \dword{apa} frames to less than $4g$. The \dword{apa}s and transport frames will be instrumented with accelerometers to understand if the \dword{apa}s were subject to shocks above their specifications. Removable side frames, made from aluminum, are then bolted to the transport frames providing a structure around the \dword{apa}s, and this whole structure is then wrapped in sealed plastic sheeting. 

The packaged transport frames from the US sites will be covered in wooden panels, loaded on custom pallets, and shipped via truck from the APA factories to the South Dakota Warehouse Facility (\dword{sdwf}) near Rapid City. The packaged transport frames from the UK will be packed, in pairs, inside wooden crates for shipping. They then will be trucked to the near-by port in Liverpool, transported by ship to the port of Baltimore, and then shipped by truck to \dword{sdwf}. In some cases, the APAs will be stored for three years or longer at the SDWF until required underground. 

\begin{dunefigure}[APA transport frame]{fig:apa-transport-frame}
{The current design of the APA shipping frame (maroon) and removable side frames (green) with two APAs covered with protective panels (shown in grey and tan). The external wooden packaging is not shown in this view.}  
\includegraphics[width=0.9\textwidth]{graphics/sp-apa-transport-box-drawing.png} 
\end{dunefigure}
% No more ITF - changed to SDWF

The size of the packaging and rigging hardware is constrained by the headframe at the Ross Shaft, and over-road shipping requirements in the US. The design of the protective panels and the side frames allow for temporally removing a portion of the shipping packaging and protective panels to access the \dword{apa} head boards for wire tension, isolation, and continuity tests after shipment and after transport underground. 

When required underground, the crates will be stripped of their wooden crating and then transported via Conestoga-type trailer to the headframe area. Near the headframe, the crates will be moved by forklift onto a cart on a rail system and rolled into the headframe. The inside portion of the headframe will have rigging gear attached to hard points on both short ends of the crate. One end (upper end) will be used for attaching to the hoist below the cage and will be used to lift the crate from horizontal to vertical and pull it into the shaft. The shipping frame is designed to clear the headframe during this operation. The other end (lower end) will be used to attach a horizontal tugger that will control the crate as it is pulled into the shaft station (Figure~\ref{fig:apa-transport-shaft}). When in the shaft, fixtures on the sides of the crate will engage wooden guides in the shaft to keep the crate from swinging or rotating while being lowered down the shaft. This operation is consistent with standard slung-load transport procedures at \dword{surf}. When the crate arrives underground, it will be pulled out of the shaft by reversing the shaft rigging operation; it will land on the opposite long edge of the crate that was used on the surface. The crate is placed on a transport cart and pulled down the drift to the cavern. When in the cavern, the \dword{apa}s will be rotated to vertical by the cavern crane, mounted on a vertically oriented cart, tested, and stored temporarily (few weeks) in the cavern adjacent to the clean room prior to final integration and installation. The transport frames and carts have been designed to be stable in each of these configurations. 

\begin{dunefigure}[APA loading into the mine shaft]{fig:apa-transport-shaft}
{Motion study of loading an \dword{apa} frame into the shaft.}  
\includegraphics[height=0.9\textheight,trim = 0mm 0mm 0mm 0mm, clip]{graphics/sp-apa-transport-shaft.png} 
\end{dunefigure}
% No more ITF - changed to SDWF

\begin{comment}
Figure~\ref{fig:apa-transport-frame} shows the current concept for a transport frame, which comprises of a central support and a left- and right-hand bolted-on cage.  The \dword{apa}s are connected to the central support by means of two sets of vertical spring struts. These are sprung to allow for anti-vibration and reduce any shock loads to the \dword{apa}s when being transported.  

Work is ongoing to finalize the transport crate design and plan the details of the transport procedures.

The spring system is shown in Figure~\ref{fig:apa-transport-springs}. It comprises two vertical struts with three springs per pair. The ``wire rope \fixme{isolator?}'' springs are %Cavoflex and are 
manufactured by Vibrostop~\footnote{
Vibrostop\textregistered{} Cavoflex, \url{https://www.vibrostop.it/en/product/wire-rope-isolators/cavoflex/}
}. 
Vibrostop will perform the relevant calculations based on input data, such as methods of travel (by road or by sea) and lifting operations when loading onto a trailer or being manipulated down the shaft at \dword{surf}.

\begin{dunefigure}[APA transport frame spring system]{fig:apa-transport-springs}
{Transport frame designed to transport two \dword{apa}s between production sites and the \dword{sdwf} then to the Ross Headframe at  \dword{surf}.}  \fixme{anne changed caption - no more ITF}
\includegraphics[height=0.19\textheight]{sp-apa-transport-spring.png} \qquad
\includegraphics[height=0.2\textheight]{sp-apa-transport-detail.png} 
\end{dunefigure}
\end{comment}




\begin{comment}
%%%%%%%%%%%%%%%%%%%%%%%%%%%%%%%%%%%%%%%%%%%%%%%%%%%%%%%%%%%%%%%%%%%%
\subsection{APA-to-CPA Assembly and Installation in the Cryostat}
\label{sec:fdsp-apa-install-cryostat}

Once underground, there will be a small storage area for stockpiling \dword{apa}s (see Figure~\ref{fig:handling}). When ready for installation, each \dword{apa} is extracted from its crate, inspected and rotated to be lowered into the area just outside of the \dword{tco} in the cryostat. Two \dwords{apa} are lowered in front of the \dword{tco} where they are linked and cabled. 
\fixme{or they come in through the airlock rollup door - decided?}
The details of the cabling are still being finalized, but the main option is currently to pass all the cables inside the \dword{apa} frame tubes (see Section~\ref{sec:fdsp-apa-intfc-apa}).



Finally, when %the two \dword{apa}s are 
an \dword{apa} pair is fully cabled, %they are 
it is placed onto the \dword{dss} inside the cryostat (see bottom right of Figure~\ref{fig:handling}) and moved to %their 
its location in the cryostat where final integration tests are performed.  For more information on the \dword{dss} and installation into the cryostat, see %the Technical Coordination 
Chapter~\ref{ch:sp-install}. \fixme{some changes above. anne}

\begin{dunefigure}[Underground handling of the APAs]{fig:handling}{(Top row) Handling of an \dword{apa} in the underground storage area where the \dword{apa}s are extracted from the crates, inspected, and readied for installation in the cryostat. (Bottom row) A pair of \dword{apa}s is brought into the space just outside the \dword{tco} to be linked and cabled, then connected to the \dword{dss} and moved into its final position inside the cryostat.}
\setlength{\fboxsep}{0pt}
\setlength{\fboxrule}{0.5pt}
\centering
\fbox{\includegraphics[height=0.195\textheight,trim=8mm 8mm 20mm 4mm,clip]{sp-apa-install-1.png}} 
\fbox{\includegraphics[height=0.195\textheight,trim=8mm 4mm 20mm 4mm,clip]{sp-apa-install-2.png}} 
\fbox{\includegraphics[height=0.195\textheight,trim=4mm 4mm 4mm 4mm,clip]{sp-apa-install-3.png}} 
\\ \vspace*{1.5mm}
\hspace*{-.25mm}
\fbox{\includegraphics[height=0.37\textheight,trim=4mm 4mm 4mm 4mm,clip]{sp-apa-install-4.png}}
\hspace*{1.mm}
\fbox{\includegraphics[height=0.37\textheight,trim=4mm 4mm 4mm 4mm,clip]{sp-apa-install-5.png}}
\end{dunefigure}
\end{comment}

%%%%%%%%%%%%%%%%%%%%%%%%%%%%%%%%%%%%%%%%%%%%%%%%%%%%%%%%%%%%%%%%%%%%
\subsection{APA Quality Control During Integration and Installation} % at the DUNE Integration Facility}
\label{sec:fdsp-apa-transport-qc}


\begin{comment}
The \dword{qc} of integration and installation has two main testing campaigns: one %at the Integration Facility (\dword{itf}) 
in the cleanroom just outside the cryostat and another once the \dwords{apa} are installed into the cryostat. The installation and integration team in the \dword{apa} consortium are still developing some details. 

Keeping track of all components for all \dword{apa}s at the different stages of  integration and installation requires a dedicated database for \dword{qc}. For efficient integration, a simple and practical way of tagging critical parts in the \dword{apa} is also being developed. This is described in Chapter~\ref{ch:sp-install}. \fixme{anne added ref}

The \dword{qa} for integration and installation relies heavily on the \dword{pdsp} experience, and at this point, no dedicated \dword{qa} protocol is developed. The full development of the protocol is being handled by the installation and integration team in the \dword{apa} consortium. 

%%%%%%%%%%%%%%%%%%%%%
\subsubsection{Quality Control at the Underground Integration Facility}
\label{sec:fdsp-apa-install-qc_if}
\end{comment}

%\fixme{anne added underground to title}

All active detector components are shipped to the \dword{sdwf} before final transport to \dword{surf}. During the storage period, the wire tensions will be measured on all \dword{apa}s to ensure that the transport has not damaged the wires. The advantage of the electrical method (see Sec.~\ref{sec:fdsp-apa-prod-qa-tension}) for measuring tension is that the \dword{apa}s can stay in the crates during the measurements. 


\begin{comment}
\begin{dunetable}[QC List]{l|c|c}{tab:qclist}{List of tests performed for \dword{qc} upon receipt at the underground cleanroom} %ption at the integration facility}   
Test to perform   &  Number of wires/channels & Acceptable values, action\\ 
Visual inspection & All & > 99\% intact \\
Wire tension      & 10$\%$ sample & 5 $\pm$ \SI{1}{N}\\
Wire continuity   & All & $-$\\
Current leakage   & All & < X $\mu$A \\
Electronics connections & All & Perfect (> 99$\%$)\\
%Noise             & All &  $\pm$ XX (> 99$\%$)\\
Cold test         & All & All intact (> 99$\%$)\\
\textbf{Overall}  & \textbf{All} & \textbf{At least 99$\%$ fully operational}\\
\end{dunetable}
\end{comment}

After unpacking an \dword{apa}, a visual inspection will be performed and wire continuity and tension measurements will be made again. 
Tension values will be recorded in the database and compared with the original tension measurements performed at the production sites, as was done for \dword{pdsp} and shown in Figure~\ref{fig:sp-apa-pd-tension-cern}. Definite guidance for the acceptable tension values will be available to inform decisions on the quality of the \dword{apa}. Clear pass/fail criteria % I think slash ok in this construction; leaving it in (anne)
will be provided as well as clear procedures to deal with individual wires lying outside the acceptable values. %Exact relation between lower or higher tension and the acceptance of a channel still needs to be worked out. 
This guidance will be informed also by the \dword{pdsp} experience. %, where the tension of some wires changed during the production to installation process. 
In addition, a continuity test and a leakage current test is performed on all channels and the data recorded in the database. 


When all tests are successful, %and more than \num{99}\,\% of the channels are confirmed functional,
%\fixme{99\% is threshold for success? clarify. anne} 
the \dword{apa} can be prepared 
for integration with the other components.  Integration and final testing will take place underground. %; there we will have more time available to perform tests. 
This step is critical for ensuring high performance of the integrated \dword{apa}s. The procedures for \dword{apa} transport to the 4,850~ft level at \dword{surf}, integration with the \dword{pds} and \dword{ce}, and the schedule for testing the integrated \dword{apa} will be addressed in %the Technical Coordination Chapter of the \dword{dune} TDR.
Chapter~\ref{ch:sp-install} of the \dword{dune} TDR. \dword{apa} consortium personnel will play direct and key roles throughout the integration and installation activities.  


\begin{comment}
%%%%%%%%%%%%%%%%%%%%%%%%%%
\subsubsection{Quality Control Underground}
\label{sec:fdsp-apa-install-qc_underground}

We have three opportunities to test the \dword{apa}s underground: in the storage area, \fixme{really? anne} once secured in front of the \dword{tco}, %or 
and once %positioned at their final location
in their final positions in the cryostat. The last is the most important, and we may save time by performing the final tests once the full \textit{APA-CPA-APA-CPA-APA} wall \fixme{wall? mean one row?} is installed (\dwords{cpa} are described in Chapter~\ref{ch:fdsp-hv}). This does risk, if serious problems are found, moving the \dwords{apa} out again, which may be harder and more time-consuming.

%\paragraph{Tests underground in the storage/unpacking area}
The \dword{apa}s are unpacked in the underground storage area (see Figure~\ref{fig:storage}). Space in this area is very limited, so only visual inspection is performed during unpacking. If clear defects are visible, the \dword{apa} is returned to  \fixme{no ITF. returned where?} %the \dword{itf} 
for further investigation. 

\begin{dunefigure}[Schematics of the underground storage area; full A-C-A-C-A wall in the cryostat]{fig:storage}{Left: A schematic of the layout for the underground storage and unpacking area. Right: A schematic of the layout of a full APA-CPA-APA-CPA-APA wall installed in the cryostat.}
\setlength{\fboxsep}{0pt}
\setlength{\fboxrule}{0.5pt}
\fbox{\includegraphics[height=0.26\textheight]{sp-apa-underground-storage.png}} 
\fbox{\includegraphics[height=0.26\textheight,trim=0mm 2mm 2mm 0mm,clip]{sp-apa-acaca-wall.png}}
\end{dunefigure}

%\paragraph{Tests underground in the \dword{tco} area}
Pairs of \dwords{apa} (top and bottom) are lowered in front of the \dword{tco} to be linked and cabled. Once the cabling is finished, a connection test is performed to ensure adequate cabling. The space near the \dword{tco} is very restricted (see Figure~\ref{fig:handling}), so no additional tests other than visual inspection are performed at that time, and the cabled and linked \dwords{apa} are positioned in their final location in the cryostat.


The current goal is to install a full APA-CPA-APA-CPA-APA wall every week (see Figure~\ref{fig:storage}, right). After each wall is installed, the night crew has time for final testing of the installed \dword{apa}. Currently, we have two testing models: one where the night crew tests \dword{apa} pairs as they are installed (every two days), and the other where the night crew tests the full wall at once. The decision will be made when accurate estimates of the time needed for testing become available.



The tests are %performed will be 
the same as described above at the \dword{sdwf}. Tension on a small set of wires is measured ($\sim$5\%, more if a quicker tension testing method is developed) to ensure that the installation operations did not alter the \dwords{apa}. With the complete integration now done, a full readout test can be performed. Short runs are taken with the \dword{daq} system to ensure that the readout is fully operational. The details of these tests must still be developed to provide efficient assessment of the integrated \dwords{apa}. If an \dword{apa} appears to have more than \num{1}\,\% of the channels not functioning, the \dword{apa} is sent back to the \dword{sdwf}.


\end{comment}
