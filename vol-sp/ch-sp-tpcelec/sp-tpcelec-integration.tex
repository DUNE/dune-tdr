\section{Integration, Installation, and Commissioning}
\label{sec:fdsp-tpcelec-integration}

Chapter~\ref{ch:sp-install} provides a complete discussion of the plans for 
integrating, installing, and commissioning the detector.
Here, we briefly discuss the responsibilities of the \dword{tpc} electronics
consortium for the activities taking place at the
at \dword{surf}, with the exception of the \dword{qc} process that
is discussed in detail in Section~\ref{sec:fdsp-tc-inst-qaqc}. We also discuss the timeline
and the resources for the integration and installation activities.
Finally, we conclude with a discussion of the commissioning
of the \dword{tpc} electronics detector components that take place while
the cryostat is being filled and immediately after. 

%%%%%%%%%%%%%%%%%%%%%%%%%%%%%%%%%%%
\subsection{Timeline and Resources}
\label{sec:fdsp-tpcelec-integration-timeline}

The current \dword{tpc} electronics consortium plan is to receive all detector 
components at the \dword{sdwf}, where they are stored temporarily
prior to being transported to \dword{surf} for integration and 
installation. Only the \dwords{femb} will undergo a reception test,
either in a laboratory on the surface at \dword{surf} or in a nearby
institution, prior to integration with the other \dword{dune} \dword{sp}
detector components. All other integration takes place at \dword{surf} in the clean room
in front of the detector cryostat. After a pair of \dword{apa}s are 
connected and moved inside the clean room, the \dword{ce} cables
for the bottom \dword{apa} are routed through the \dword{apa} frames.
The cables are then connected to the \dwords{femb}, and the bundles
of cables are placed in the trays on the top of the \dword{apa} pair.
At this point, the pair of \dword{apa}s is moved into one of the cold
boxes, and the cables are connected to a patch panel inside the cold box
to save the time that would be required for routing the cables through the cryostat
penetration of the cold box and connecting them to the end flange.
The \dword{ce} electronics is then tested at both room temperature
and at a temperature close to that of \lntwo, much like
what was done for the \dword{apa}s installed in the \dword{pdsp} detector.

Later, the pair of \dword{apa}s is moved to its final position 
inside the cryostat. The \dword{ce}
and \dword{pds} cables are routed through the cryostat penetration and
connected to the corresponding warm flanges, and final leak tests are performed
on the cryostat penetration. At this point, the \dword{wiec} is attached
to the warm flange and all the cables and fibers required to provide 
power and control signals to the \dword{tpc} electronics and for data
readout are connected. This permits additional testing with the full 
\dword{daq} readout chain and the final power and controls distribution
system. Once initial tests are completed successfully, more \dword{apa}s
can be installed, and the \dword{apa}s can remain accessible until the \dword{fc} are
deployed.

This installation sequence assumes that all the \dword{tpc} electronics detector 
components required for readout of a pair of \dword{apa}s  on top of the cryostat 
are installed before \dword{apa}s are inserted into the cryostat. This 
includes the \dwords{wiec} with their boards, the power supplies in 
the racks, and all cables and fibers required to distribute power and
control signals as well as for detector readout. Installation should occur at least two weeks before
the \dword{apa}s are inserted into the cryostat to allow time for some checks. 

One exception
is installing the cryostat penetrations with the warm flanges
for both the \dword{ce} and the \dword{pds}. The cryostat
penetrations should be installed as soon as possible, at the latest
simultaneous to installing the detector support structure
inside the cryostat. This ensures the cryostat is almost 
completely sealed to minimize the amount of dust 
entering the cryostat. During the routing of the \dword{ce} and
\dword{pds} cables through the cryostat penetrations, dust entering the cryostat will be minimized by having a small
over-pressure inside the cryostat and by isolating each penetration
from the cavern using a tent mounted over 
the work area.

The schedule of activities at \dword{surf} is designed so all 
\dword{apa}s can be installed in the cryostat on a timescale of eight
months, proceeding at a rate of one row of six \dword{apa}s per week and
allowing for a ramp-up period at the beginning of the process. This
requires that personnel from the \dword{tpc} electronics consortium be available
for two shifts per day at \dword{surf} at all times, including weekends. A
total of 25 FTEs/week will be needed to install and test the \dword{apa}s. 

The installation of all the other \dword{tpc} electronics detector components
takes place on the top of the cryostat. The cryostat penetrations are installed
ahead of the installation of the \dword{apa}s inside the cryostat, ideally as
soon as the welding of the cold membrane is completed in part of the cryostat. 
This activity requires at team of 8 FTEs, split over two shifts per day, for a 
period of one month. A similar amount of time and personnel is also required
to install the power supplies in the racks, attaching the \dwords{wiec} to
the \dword{ce} flanges, and routing and connecting the warm cables.

%%%%%%%%%%%%%%%%%%%%%%%%%%%%%%%%%%%
\subsection{Internal Calibration and Initial Commissioning}
\label{sec:fdsp-tpcelec-integration-calib}

While the cryostat is being closed (and any time there is welding 
on the cryostat), the electronics should be turned off and all 
cables between the detector racks, including the low voltage
and bias voltage, fans, and heater power should be disconnected 
from the \dwords{wiec}. Once the cryostat is closed, the baseline 
and noise of all channels should be measured. Dead electronics channels 
should be identified by measuring the response of all channels to 
the internal electronics calibration pulser at a nominal setting, 
such as $\pm\SI{600}{mV}$, which distinguishes between induction 
and collection channels. The noise should be measured with the wire bias 
voltages fully enabled on the G, U, and X planes of the \dword{apa}s. 
It should also be measured with the cathode high voltage on at a very 
low value, e.g. \SI{50}{V}. The non-responsive channels, identified
as having very low noise, and the channels that have noise that 
significantly exceeds the average should be flagged and recorded.
Sources of noise that exceeds the expectations should be identified
and, if possible, fixed. Any warm electronics components with
issues should be replaced with spares.

% While the cryostat is being filled with gaseous argon, the electronics 
% should be powered off. 
Once the cryostat is filled with gaseous
argon, the baseline and noise of all the channels will be measured
again, and any new non-responsive channels in the electronics 
should be identified by injecting $\pm\SI{600}{mV}$ with the 
internal calibration pulser. As the cryostat is cooled down, the 
temperature at the electronics and the noise of all channels should 
be monitored periodically. Any new non-responsive channels should 
be flagged and excess noise sources that are exposed as the 
electronics cool should be identified, and if possible, fixed.

Once the electronics is fully submerged in \dword{lar}, a full 
set of electronics diagnostic tests should be run, including: 
baseline and noise measurement, and a full gain calibration on 
all channels with the internal calibration pulser at settings 
up to the saturation of the \dword{fe} inputs. The shaping time 
should be measured on all channels by injecting the $\pm\SI{600}{mV}$
internal pulser at each of the four settings and fitting the 
pulse shape. Any new non-responsive channels during the pulser 
runs should be flagged. Any new disconnected channels should be 
flagged and excess noise sources should be identified. These tests 
can be performed on the electronics installed on the bottom
\dword{apa}s even while the corresponding wires are in 
the gaseous argon.
