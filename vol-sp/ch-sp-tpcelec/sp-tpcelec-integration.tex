\section{Integration, Installation, and Commissioning}
\label{sec:fdsp-tpcelec-integration}

\fixme{The material contained in this section could be moved entirely into the Technical Coordination chapter of the Single Phase volume. Jim Stewart seems to prefer this. The CE consortium would like to retain the testing aspects into the CE chapter. We agree that there may be a few paragraphs from this section that need to be moved into the TC chapter.}

%%%%%%%%%%%%%%%%%%%%%%%%%%%%%%%%%%%
\subsection{Timeline and Resources}
\label{sec:fdsp-tpcelec-integration-timeline}

A complete discussion of the plans for the Integration, Installation
and Commissioning of the detector is discussed in Chapter~\ref{ch:sp-tc}.
In this Section we discuss briefly the responsibilities of the \dword{ce}
consortium for the activities that take place at the
\dword{itf} and at \surf. The current plan for the \dword{ce}
consortium is to receive all detector components at the \dword{itf},
store them temporarily and, if needed, change the packaging prior to
the final transport to \surf. In some cases this will entail simply
dividing a shipment received at the \dword{itf} into the number of
units that are required for a specific installation job at \surf,
while in other cases this may also require changing the material
used for packaging, to avoid contamination inside the clean room
at \surf or in the cryostat. One specific case is that of the \dword{ce}
cables that may be received on separate spools carrying 
individual bias voltage, signal and control, and power cables, and that 
need to be rearranged in bundles of different cables that are enclosed
in a mesh and ready for installation on the \dword{apa}s. 

The \dword{ce} consortium plans to perform a quick test on all
the \dwords{femb} upon reception at the \dword{itf}, to ensure that
the boards and \dwords{asic} have not been damaged in any way during the transport 
from the \dword{femb} \dword{qc} sites. The main activity that takes 
place at the \dword{itf} for the \dword{ce} consortium is the 
installation of the \dwords{femb} on the \dword{apa}s, followed by 
a second check of the functionality of the boards and of the 
corresponding \dwords{asic} and a check of the connection to the 
\dword{apa}'s wires. 

All other integration activities take place at \surf in the clean room
in front of the detector cryostat. After the pair of \dword{apa}s are 
connected and moved inside the clean room, the \dword{ce} cables
for the bottom \dword{apa} are routed through the \dword{apa}s' frames.
The cables are then connected to the \dwords{femb} and the bundles
of cables are placed in the trays on the top of the \dword{apa}s' pair.
At this point, the pair of \dword{apa}s is moved into one of the cold
boxes and the cables are connected to a patch panel inside the cold box,
to save the time required for routing the cables through the cryostat
penetration of the cold box and for connecting them to the end flange.
The \dword{ce} electronics is then tested at both room temperature
and at a temperature close to that of \lntwo, similarly to
what was done for the \dword{apa}s installed in the \dword{pdsp} detector.

Later, the pair of \dword{apa}s is moved to its final position 
inside the cryostat the \dword{ce}
and \dword{pds} cables are routed through the cryostat penetration and
connected to the corresponding warm flanges, and final leak tests are performed
on the cryostat penetration. At this point, the \dword{wiec} is attached
to the warm flange and all the cables and fibers required to provide 
power and control signals to the \dword{ce} and for the data
readout are connected. This allows for further testing with the full 
\dword{daq} readout chain and the final power and controls distribution
system. Once initial tests are completed successfully further \dword{apa}s
can be installed, and the \dword{apa}s remain accessible until the 
deployment of the field cages.

This installation sequence assumes that all the \dword{ce} detector 
components (the \dwords{wiec} with their boards, the power supplies in 
the racks, all the cables and fibers required to distribute power and
control signals and for the detector readout) on the top of the cryostat 
required for the readout of a pair of \dword{apa}s are installed prior 
to the insertion of those \dword{apa}s in the cryostat. The plan is 
that this installation is performed at least two weeks prior to the 
insertion of the \dword{apa}s to allow for some checks. One exception
is the installation of the cryostat penetrations with the warm flanges
for both the \dword{ce} and the \dword{pds}. The cryostat
penetrations should be installed as soon as possible, at the latest
simultaneously with the installation of the detector support structure
inside the cryostat. This is to ensure that the cryostat is almost 
completely sealed, in order to minimize the amount of dust that can 
enter the cryostat. During the routing of the \dword{ce} and
\dword{pds} cables through the cryostat penetrations the amount
of dust entering the cryostat will be minimized by having a small
over-pressure inside the cryostat and by isolating the penetration
that is being worked on from the cavern with a tent mounted over 
the working area.

We expect that at most three months will be needed to commission
all the test systems at the \dword{itf} and the current planning
assumes that eighteen months are required to integrate the \dword{ce}
and the \dword{pds} with two work stations, i.e.
a rate of one pair of \dword{apa}s per week. Given that the \dword{apa}
production will start well in advance of the \dwords{asic}'s fabrication
and of the assembly of \dwords{femb}, initial integration tests will 
be performed with \dword{femb} prototypes and the overall schedule
for the integration will be driven by the availability of the \dword{ce}
components. The work to be performed by the \dword{ce}
consortium at the \dword{itf} requires 2 FTEs/week for the installation 
of the \dwords{femb} on the \dword{apa}s and their testing, and for 
the \dword{femb} reception tests. We expect these shifts to be covered 
by scientific personnel from the \dword{ce} consortium. In addition, 
we plan to have one engineer and one technician available at the 
\dword{itf} to support the activities on the consortium, including 
the preparation of components for shipment to \surf.
% The integration of the \dwords{femb} on the \dword{apa}s for a
% second Single Phase TPC Far Detector would proceed after the
% completion of the first detector and would require an additional
% 26 months. This period of time is longer than the corresponding
% one for the first detector and it is driven by the availability
% of the \dword{apa}s that take longer to build compared to the
% \dword{ce} components. The personnel requirements will be
% similar to the ones for the first detector.

The schedule for the activities at \surf is designed such that all the
\dword{apa}s can be installed in the cryostat on a timescale of seven
months, proceeding at a rate of one row of \dword{apa}s per week, allowing
for a ramp-up period at the beginning of the process. This
requires that personnel from the \dword{ce} consortium is available
for two shifts per day at \surf at all times, including week-ends. A
total of 15 FTEs/week is required for the installation and testing 
activities related to the \dword{apa}s. An additional 3 FTEs/week are
required for the installation of the other detector components on
top of the cryostat, that needs to proceed in parallel, but slightly
ahead of schedule, relative the integration and installation of the
APAs. The installation of the cryostat penetrations, that will take
place ahead of all these activities, requires 4 FTEs working full time
for one month. We expect most of this to be scientific personnel from
the \dword{ce} consortium and like for the \dword{itf} we plan
to have one engineer and one technician available at \surf to support
the integration and installation activities. For the installation of
the cryostat penetrations a team of 8 FTEs is required, split over
two shifts per day, for a period of one month. The installation of
the cryostat penetration will take place at the latest during the 
installation of the detector support structure, but it could also be
scheduled at an earlier date, possibly starting as early
as the welding of the cold membrane is completed in part of the cryostat.

% The activities at \surf for a second Single Phase TPC Far Detector
% will have a similar timeline and similar personnel requirements as
% for the first detector.

%%%%%%%%%%%%%%%%%%%%%%%%%%%%%%%%%%%
\subsection{Quality Control}
\label{sec:fdsp-tpcelec-integration-qc}

Many of the activities of the \dword{ce}
consortium at the \dword{itf} and at \surf have the goal of 
ensuring that the detector will be fully functional once the cryostat
is filled with liquid Argon. All the detector components provided
by the \dword{ce} consortium that arrive at the \dword{itf}
and at \surf have gone through a qualification process to ensure
that they are fully functional and that they meet the DUNE 
specifications. Additional tests and checks are performed at the
\dword{itf} and at \surf to ensure that the components have not
been damaged during the transport or during the installation itself,
and most importantly that all the parts are properly connected.

\dwords{femb} are tested multiple times in this process. 
The \dwords{femb} undergo first a reception test and a test after 
installation on the \dword{apa}s at the \dword{itf}, and further 
tests are performed at \surf before and after installation of the 
\dword{apa}s in the cryostat, using the final cables for the 
connection between the \dwords{femb} and the detector flanges. 
Results from the tests performed at
the \dwords{itf} and at \surf are compared with the results of the
tests performed during the qualification of \dwords{asic} and
\dwords{femb} to detect possible deviations that could signal 
damage in the boards or problems in the connections. All tests 
results will be stored in the same database system used for
the results obtained during the qualification of components.

Both the reception test at the \dword{itf} and the test after the
installation on the \dword{apa}s are performed at room temperature 
by connecting up to four \dwords{femb} to a \dword{wib} that is 
connected directly to a laptop computer for readout over 1~Gbps
Ethernet, with power provided by a portable 12~V supply. For 
the reception test the \dwords{femb} are attached to a capacitive 
load to simulate the presence of wires, allowing for connectivity 
tests, the measurement of the baseline and RMS of the noise for 
each channel. Dead channels are identified using the calibration 
pulse internal to the \dword{fe} \dword{asic}. Overall the reception 
test and the test performed after attaching the \dwords{femb} to the 
\dword{apa}s each require approximately half an hour per motherboard, 
including  the time for connecting and disconnecting test cables.
The \dword{ce} consortium plans to have a \dword{cts} available
at the \dword{itf} for performing checks at \lntwo temperature
of \dwords{femb} that fail the quality control procedures at \surf,
and eventually for sample checks on the \dwords{femb} upon reception
at the \dword{itf}. It is desirable to have a cold box at the \dword{itf}
in such a way that a fraction of the \dword{apa}s can be tested at
\lntwo temperature after the integration with the \dword{pds}
and with the \dwords{femb} prior to the test at 
\surf. The purpose of this additional cold test on the surface is to
detect possible problems with the integration of \dword{apa}s, \dwords{femb},
and \dword{pds} components before the availability of the cold
boxes in the \surf cavern.

At \surf, an initial test of the readout is performed at room temperature to
ensure the proper connection of the cables. This test is similar to the one
performed at the \dword{itf} except that the power, control, and readout
connections are made through a system identical to the one used during
operations in the cryostat instead of being made with a portable system.
During these tests the internal calibration system will be used to check
that the response of the \dword{adc} has not changed relative to prior
tests. Fast Fourier transforms of the noise measurements will be inspected
for indications of coherent noise. All \dword{fe} gain and shaping time
settings will be exercised and the gain will be measured using the
integrated pulser circuit in the \dword{fe} \dword{asic}. 
The connectivity and noise measurements, as well as the check for dead
channels are repeated later after the cool down of the \dword{apa}s pair
to a temperature close to that of \lntwo in the cold box, 
checking also for the possible impact of variations of the bias voltage
on the \dword{apa}s' wires and of the status of the \dword{pds}.
Results from all these tests will be compared with results obtained 
in earlier \dword{qc} tests.  If problems are found, it will be possible 
to fix them by re-seating cables or replacing individual \dwords{femb}.
Noise levels are also monitored during the cool-down and warm-up 
operations of the cold boxes. These tests also ensure that the connection
of the power, control, and readout cables has been performed correctly
on the \dword{femb} side and that this connection will withstand temperature 
cycles. This addresses one of the problems observed during the integration,
installation and commissioning of the \dword{protodune}--\dword{sp} 
detector. While the connection between the cables and the \dword{femb}
has been redesigned to minimize the problems observed with \dword{protodune},
we think that it is important to repeat these tests during the integration
and installation of the detector, since a single connection problem would
result in the loss of one entire \dword{femb}. In addition, the tests 
performed in the cold box at \surf demonstrate that the power, control, and
readout cables for the bottom \dword{apa}s are not damaged when they are routed 
through the \dword{apa}s' frames. Additional measurements of the noise
level inside the cryostat will be performed on a regular basis by closing 
temporarely the \dword{tco} with an RF shield that is electrically connected 
to the cryostat steel. 

All readout tests are repeated again after the \dword{apa}s are put
in their final location inside the cryostat and after the power, control, and
readout cables are connected to the warm flange attached to the cryostat
penetration. At this point, the connection between the cables and the flange
is validated and the entire power, control, and readout chain, including the
final \dword{daq} back-end used during normal operations are exercised. The
installation plan for the detector components inside the cryostat (\dword{apa}s,
\dwords{cpa}, and deployment of the field cages) is such that it will still
be possible to make minor repairs on some \dwords{femb} without extracting
the \dword{apa}s from the cryostat. The testing of all the detector components
will continue throughout the installation of all the elements of the TPC, 
until the cryostat is ready to be filled with liquid Argon.  When the 
\dword{apa}s are in their final position the replacement of \dwords{femb} 
or cold cables will be more difficult and may require that the \dword{apa}s 
are extracted from the cryostat. This operation will be performed only in 
case of major problems with the \dwords{femb}.

In addition to measurements performed to demonstrate that the \dword{apa}
readout is working as planned, the \dword{ce} consortium will also
be performing tests of the bias voltage system together with the \dword{apa}
and \dword{cpa} consortia. These tests are aimed at demonstrating that
the cables that provide the bias voltage to the \dword{apa} wires, the
field cage termination electrodes and the electron diverters are connected
properly and that there are no short circuits. For safety reasons, these
tests will be performed with limited voltage (50--100 V) while the access
to the inside of the cryostat (or to the area close to the \dword{apa} being
tested) is limited. These tests will be performed as soon as an \dword{apa}
is in its final position after connecting the bias voltage cables to the 
\dword{shv} boards on the \dword{apa}, using a resistive load in the case of the 
field cage termination electrodes and of the electron diverters. This ensures
the continuity of the bias voltage distribution system from the bias voltage
supplies to the \dword{apa}s. The test has to be repeated for the field cage
termination electrodes and the electron diverters after the deployment of
the field cages.

Additional tests are going to be performed on the other detector components
provided by the \dword{ce} consortium prior to the insertion of the
\dword{apa}s. After the cryostat penetrations are put in place leak checks
will be performed by spraying Helium inside the cryostat penetration and
having a leak detector outside the detector. These tests will be repeated 
after all the cables have been routed through the cryostat penetration.
As soon as the bias voltage and the power supplies are installed on the detector
mezzanine and cables are put in place between the corresponding racks and
cryostat penetrations, tests will be performed to ensure that the proper 
power and bias voltage can be delivered to the \dwords{wiec} before these
are installed. Even before connecting the \dwords{wiec} to the warm flanges
tests will be performed to ensure that they can be properly powered up, controlled,
and readout by the \dword{daq} back-end. Tests will be performed on the
readout fiber plant to ensure that all fiber connections are functional
and properly mapped. Additional tests will be performed on the slow control
system and on the Detector Safety System at multiple times during the
installation of the detector. These tests will take place before the
corresponding \dword{apa}s are installed, after their installation and
the connection of all the corresponding cables and fibers, and finally
during all the integrated tests that take place prior to the closure of
the \dword{tco} and the filling of the cryostat. Negative results in any
of these tests will halt the integration, installation, and
commissioning activities, and will be used as input in reviews that will
take place before authorizing the closure of the \dword{tco}, the beginning
of the liquid Argon filling operation, and eventually of the detector 
commissioning after the cryostat has been filled.

%%%%%%%%%%%%%%%%%%%%%%%%%%%%%%%%%%%
\subsection{Internal Calibration and Initial Commissioning}
\label{sec:fdsp-tpcelec-integration-calib}

\fixme{The material in this section has been moved to the previous one.
This section should contain a summary of the calibrations performed 
once the cryostat is completely filled with LAr and a brief plan for the
commissioning of the detector at that point, based on the experience
gained with promoting-SP, including an estimate of the time required
before tracks from cosmics can be observed in the detector.}
