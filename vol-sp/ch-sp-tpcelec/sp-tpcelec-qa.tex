\section{Quality Assurance}
\label{sec:fdsp-tpcelec-qa}

%%%%%%%%%%%%%%%%%%%%%%%%%%%%%%%%%%%
\subsection{Introduction}
\label{sec:fdsp-tpcelec-qa-introduction}

The \dword{ce} consortium is developing a \dword{qa} plan that is consistent
with the principles discussed in the Technical Coordination part of this 
\dword{tdr} (Volume 7 Chapter 9). %Commented out because it breaks the automatic build system %(\refch{tc}{vl:tc-QA}).
The goal of the \dword{qa} plan is to maximize the number of functioning
readout channels in the detector that achieve the performance specifications
for the detector that have been discussed in Section~\ref{sec:fdsp-tpcelec-overview-design},
in particular in terms of noise. Minimizing the noise in the detector requires that all
system aspects are taken into consideration starting from the design phase, and in this
respect the experience gained with the \dword{pdsp} prototype is extremely
valuable and has been used to inform the design changes in the detector 
components. The lessons learned during the construction of \dword{pdsp},
the commissioning of the detector, and the initial data taking period have
already been discussed in Section~\ref{sec:fdsp-tpcelec-design}. Further
operation of \dword{pdsp} in 2019 will continue to inform the final design
of the detector components.

Apart from the number of channels the most important difference
between \dword{pdsp} and DUNE is the projected lifetime of the detector. This
is relevant because significant fraction of the detector components provided 
by the \dword{ce} consortium are installed inside the cryostat and cannot 
be accessed or repaired during the operational lifetime of the detector. The 
Graded Approach to \dword{qa} implies that particular care needs to be applied
to the \dword{ce} components that are installed inside the cryostat.

A complete \dword{qa} plan starts with ensuring that the design of all
detector components fulfills the specification criteria, considering
also system aspects, i.e. how the various detector components interact
among themselves and with the detector components provided by other 
consortia. We discuss the validation of the design in 
Section~\ref{sec:fdsp-tpcelec-qa-initial} and the facilities that we use
to investigate the interaction between different detector components
in Section~\ref{sec:fdsp-tpcelec-qa-facilities}. 

The other aspects of the \dword{qa} plan involve documenting the 
assembly and testing processes, storing and analyzing the information
collected during the \dword{qc} process, training and qualifying the
personnel from the consortium, monitoring the procurement of 
components from external vendors, and assessing whether the
\dword{qc} procedures are being applied in a uniform way across
the various sites involved in the detector construction, integration,
and installation. The plan of the \dword{ce} consortium involves
having multiple sites performing the same \dword{qc} procedures,
with the possibility of a significant turnaround in the personnel
performing these tasks. To avoid problems during the bulk of the
production phase we plan to emphasize the training and documentation
aspects of the \dword{qa} plan. Reference parts will be tested at
multiple sites to ensure that consistent results are produced. At
a single site the same parts will be tested repeatedly to ensure
that there are no changes in the response of the apparatus and
that new personnel involved in testing detector components is 
as proficient as more experienced personnel. All the data from
the \dword{qc} process will be stored in a common database and
the yields of the production will be centrally monitored and 
compared among different sites. The procedures that will be adopted
during the detector construction will evolve from the experience
gained with \dword{pdsp}. A first version of the testing procedures
will be put in place in 2019 and 2020, while the final design of
the detector components is being completed and new prototypes are
being tested. The \dword{qc} procedures will be then reviewed
during the Engineering Design Review that precedes the beginning
of the pre-production. Lessons learned during the pre-production
will be analyzed and a final and improved \dword{qc} process will be 
developed prior to the Production Readiness Review that triggers
the beginning of the production. During the production the results
of the \dword{qc} process will be reviewed at regular interval
in Production Progress Reviews. In case of problems, production
will be stopped and an analysis of the issues will be performed
and will result in changes to the procedures if necessary.

%%%%%%%%%%%%%%%%%%%%%%%%%%%%%%%%%%%
\subsection{Initial Design Validation}
\label{sec:fdsp-tpcelec-qa-initial}

As described in Section~\ref{sec:fdsp-tpcelec-design}, four \dwords{asic} 
are being developed for the DUNE far detector single-phase TPC readout 
(\dword{larasic}, \dword{coldadc}, \dword{coldata}, and \dword{cryo}). 
When a new prototype \dword{asic} is produced, the first tests of 
\dword{asic} functionality and performance are done by the groups 
responsible for the \dword{asic} design. These tests may use either 
packaged parts or dice mounted directly on a printed circuit board 
and wire bonded to the board.  The goal of these tests is to determine 
the extent to which the \dword{asic} functions as intended, both at room 
temperature and at \lntwo temperature.  For all chips these tests 
include exercising digital control logic and all modes of operation. Tests 
of \dword{fe} \dwords{asic} include measurements of the noise as a function 
of input capacitance, baseline recovery from large pulses, cross-talk, linearity, 
and dynamic range. Tests of \dwords{adc} include effective noise and 
differential and integral non-linearity. Tests of \dword{coldata} and \dword{cryo} 
include verification of both the control and high speed data output links with 
cables at least as long at the longest cables needed in the DUNE far detectors 
(currently estimated to be \SI{23.5}{m}). After the initial functionality
tests performed by the groups that have designed the \dwords{asic} further
tests will be performed by other independent groups and then the \dwords{asic}
will be mounted on \dwords{femb} such that noise measurements can be repeated
with real \dword{apa}s attached to the readout, including the investigation
of possible system issues generated by the interaction with different 
detector components.

%%%%%%%%%%%%%%%%%%%%%%%%%%%%%%%%%%%
\subsection{Integrated Test Facilities}
\label{sec:fdsp-tpcelec-qa-facilities}

%%%%%%%%%%%%%%%%%%
\subsubsection{ProtoDUNE-SP}
\label{sec:fdsp-tpcelec-qa-facilities-pdune}

\dword{pdsp} is designed as a full slice of the \dword{spmod} as close as 
possible to the final DUNE-\single components. It contains six full-size 
DUNE \dwords{apa} instrumented with \num{20} \dwords{femb} each for a 
total readout channel count of \num{15360} digitized sense wires. Critically, 
the wires on each \dword{apa} are read out via a full \dword{ce} readout 
system, including a \dword{ce} flange and \dword{wiec} with five \dwords{wib} 
and one \dword{ptc}. Each combined \dword{apa} and \dword{ce} readout follows 
the grounding guidelines described in Section~\ref{sec:fdsp-tpcelec-design-grounding} 
to operate as a fully-isolated readout unit.

\dword{pdsp} took beam data in the CERN Neutrino Platform in 2018 and will continue
to take cosmic data throughout 2019. As described in Section~\ref{sec:fdsp-tpcelec-overview-pdune}, 
the live channel count (99.7\%) and average noise levels on the collection and induction wires 
(ENC$<$600e$^-$ and 800$^-$, respectively) satisfy the DUNE-\single requirements described 
in Section~\ref{sec:fdsp-tpcelec-overview-scope}. Several lessons learned from the production 
and testing of the \dword{ce} and the \dword{pdsp} beam data run will be incorporated into the 
next iteration of the system design for the \dword{spmod}.

\fixme{Even if the lessons learned are mentioned explicitly in the previous section
we should probably have a table here with the main points and references to the 
previous section.}

Five of the six \dword{apa}s were tested in the \dword{pdsp} cold box, which is described in
Section~\ref{sec:fdsp-tpcelec-qa-facilities-additional}, prior to installation in the cryostat. 
These tests were critical in identifying issues with \dword{ce} components after installation 
on the \dword{apa}. Therefore, a very similar set of cold box tests are being planned at SURF 
with the fully-instrumented DUNE \dword{apa}s. A seventh \dword{apa} will be delivered
at CERN in January 2019 and equipped at first with \dword{pdsp} \dwords{femb}. This
\dword{apa} will be characterized in the cold box like the \dword{apa}s installed
in \dword{pdsp}, establishing a reference point for further tests that will be
performed after replacing half of the \dwords{femb} with new prototypes that will 
be equipped with the new \dwords{asic} that are being developed for DUNE. 

The DUNE \dwords{apa} and the readout electronics will be different from the ones used 
in \dword{pdsp}. For this reason, plans are being made for re-opening the \dword{pdsp} cryostat 
and replacing three of the six \dwords{apa} with final DUNE prototypes that will also include 
the final versions of the \dwords{asic} and \dwords{femb}. A second period of data-taking 
with this new configuration of \dword{pdsp} is being planned for 2021-2022. This will also 
allow for another opportunity to check for interference between the readout of the \dword{apa} 
wires and either the \dword{pds} or other cryogenic instrumentation.

\fixme{Results from the actual commissioning of the 7th ProtoDUNE APA could be
added here in Spring 2019, and possibly comparisons with the COTS solution for the
ASICs}

%%%%%%%%%%%%%%%%%%
\subsubsection{Small Test TPC (ICEBERG)}
\label{sec:fdsp-tpcelec-qa-facilities-testtpc}

The main drawbacks of the tests in the cold box at CERN are that the \dword{apa}
operates in gas and that there is no provision for an electrical field that 
would provide signals from tracks passing in the detector. To overcome these 
issues Fermilab has built a new cryostat (\dword{iceberg}) that will be used for 
\dword{lar} detector R\&D and for system tests of the \dword{ce} prototypes. 
The \dword{iceberg} cryostat, shown in Figure~\ref{fig:ICEBERG-cryotee} is installed
at the Proton Assembly Building at Fermilab. It has an inner diameter of \SI{152}{cm}
and can hold about 35,000 liters of \dword{lar}, which is sufficient to house a
\dword{tpc} with dimensions \SI{125}{cm}~$\times$~\SI{125}{cm}~$\times$~\SI{60}{cm}. For DUNE 
purposes this cryostat will house a 1,280-channel \dword{tpc} shown in
Figure~\ref{fig:ICEBERG-tpcdaq}, with an \dword{apa} and two \dwords{fc} each 
enclosing a sensitive volume with a maximum drift distance of \SI{25}{cm}. The \dword{apa} 
has been built using the same wire boards and anchoring elements of  \dword{pdsp} 
described in Section~\ref{sec:fdsp-apa-boards}. It has dimensions of 
1/10$^{th}$ of a DUNE \dword{apa}. The \dword{apa} mechanics is designed to accommodate 
two half-length \dword{pdsp} bars with dimensions and connectors that already 
include the design modifications planned for DUNE. Power, readout, and controls 
use equipment identical to those used for \dword{pdsp}. The interface between the 
\dwords{femb} and the \dword{apa} wires is realized using the same CR boards used 
for \dword{pdsp} and described in Section~\ref{sec:fdsp-apa-boards}. The \dword{tpc} is
readout via a \dword{daq} system (also shown in Figure~\ref{fig:ICEBERG-tpcdaq})
identical to that of \dword{pdsp}. The power and signal cables for the detector 
are routed through a Tee installed on the center port of a movable flange on the 
top of the cryostat that is also used to support the \dword{tpc}. The movable 
flange contains fourteen additional ports that are available for different utilities 
including \dword{hv} feedthrough, purity monitor, cryogenic controls, and visual inspection. 
A condenser, and \dword{lar} fill and vacuum ports are on the side of the cryostat, 
providing easy access to the detector.

\begin{dunefigure}
	[\dword{iceberg} cryostat and top plate Tee]
  {fig:ICEBERG-cryotee}
	{\dword{iceberg} cryostat (left) and top plate Tee (right).}
  \includegraphics[width=0.26\linewidth]{sp-tpcelec-ICEBERG-cryostat.jpg}
  \includegraphics[width=0.61\linewidth]{sp-tpcelec-ICEBERG-Tee.jpg}
\end{dunefigure}

\begin{dunefigure}
	[\dword{iceberg} TPC and DAQ rack]
  {fig:ICEBERG-tpcdaq}
	{\dword{iceberg} TPC (left) and DAQ rack (right).}
  \includegraphics[angle=270,width=0.43\linewidth]{sp-tpcelec-ICEBERG-TPC.jpg}
  \includegraphics[angle=270,width=0.43\linewidth]{sp-tpcelec-ICEBERG-DAQ.jpg}
\end{dunefigure}

The \dword{fc} for the \dword{tpc} is constructed using printed circuit boards and is 
designed to provide 25 cm of drift space on both sides of the \dword{apa}. The cathode 
plane is made of a printed circuit board coated with copper and will be powered with 
\SI{-15}{kV} DC power. There is a \SI{1}{G$\Omega$} resistance between the strips of the \dword{fc}
creating a gradient field changing from \SI{-15}{kV} at the cathode to \SI{-1}{kV} near the 
\dword{apa}. The sides of the \dwords{fc} are terminated on the \dword{apa}
ground with \SI{156}{M$\Omega$} resistors. 

The \dword{iceberg} power system that provides power to the detector, electronics, 
\dword{daq}, and cryogenics controls has been designed with an extreme care to 
isolate the detector and building grounds, following the same principles adopted
for \dword{pdsp}, including a new 480~V transformer. The impedance between the detector
and building grounds is continuously monitored. The distribution panel, which is 
at detector ground, provides 208 and 120~V power for the \dword{ce} rack, that
provides both the low voltage power to the \dword{ce} components and the bias 
voltage to the \dword{apa} wire planes, through the \dword{wiec} and \dword{shv}
connectors located on the cryostat penetration. A single WIENER MPOD  provides 
\SI{-665}{V}, \SI{-370}{V}, \SI{0}{V}, and \SI{820}{V} to the grid, U, V, and Y planes of the \dword{apa},
respectively. It is also used to provide the \SI{-15}{kV} to the cathode plane. A Wiener 
PL506 provides \SI{18}{V}, \SI{48}{V}, \SI{12}{V} and \SI{-12}{V} to the \dword{ptc} located in the Tee at the top of the cryostat. 

The \dword{daq} for \dword{iceberg} is a copy of the system used for the readout
of five of the \dword{pdsp} \dword{apa}s at CERN. The core of the \dword{daq} system 
consists of two Linux PCs which communicate over 10 Gbps optical fiber
to processing units called RCEs (Reconfigurable Cluster Element), which are 
housed on industry-standard ATCA shelves on COB (cluster-on-board) motherboards.
The RCEs can perform data compression or zero-suppression. They also buffer
the data and then send it to the Linux PCs where the data can be analyzed 
using the artDAQ framework. A pair of scintillators located at the top and bottom 
of the cryostat generates a cosmic trigger for the DAQ  using a Trigger Logic Unit (TLU).
The system is modular and could be upgraded to follow the overall DUNE \dword{daq} 
development. 

As for the tests that will be performed in the cold box at CERN using the seventh
\dword{pdsp} \dword{apa} the baseline performance of the \dword{iceberg} \dword{tpc}
will be established using \dword{pdsp} \dwords{femb}. Later, new prototypes, equipped
with the new \dwords{asic} under development, will be tested. The main advantage
relative to the tests performed at CERN in the cold box is that the performance of the
\dword{ce} prototypes will be measured with actual tracks from cosmic rays 
reconstructed within the \dword{tpc}. In addition, system studies with new 
prototypes of the \dword{pds} will be performed, since the \dword{apa} is
mechanically compatible with the new design of the \dword{pds}, which is not
the case for the seventh \dword{pdsp} \dword{apa}.

\fixme{Results from the actual commissioning of the ICEBERG TPC could be
added here in Spring 2019, and possibly comparisons with the COTS solution for the
ASICs}

%%%%%%%%%%%%%%%%%%
\subsubsection{40\% APA at BNL}
\label{sec:fdsp-tpcelec-qa-facilities-fortypercent}

One additional facility where the \dword{femb} prototypes can be connected to
an \dword{apa} inside a shielded environment is the \num{40}\,\% \dword{apa} 
test stand at BNL. The \num{40}\,\% \dword{apa} at BNL is a \SI{2.8}{m}~$\times$~\SI{1.0}{m} 
three-plane \dword{apa} with two layers of \num{576} wrapped ($U$ and $V$) wires 
and one layer of \num{448} straight ($X$) wires. It is read out by up to eight 
\dwords{femb} with the full \SI{7}{m} \dword{pdsp} length data and \dword{lv} power 
cables, four on the top and four on the bottom. The readout uses the full \dword{ce} 
system, with \dword{ce} flange and \dword{wiec}, as shown in Figure~\ref{fig:tpcelec_40apa}. 
Detailed integration tests of the \dword{pdsp} \dword{ce} readout performance while 
following the DUNE grounding and shielding guidelines were done at the \num{40}\,\% 
\dword{apa}. This system was also used for initial studies of the COTS \dword{adc}
option described in Section~\ref{sec:fdsp-tpcelec-design-femb-alt-cots}, and will
be used again for new \dwords{femb} prototypes.

\begin{dunefigure}
[One side of the \num{40}\,\% \dword{apa} with four \dwords{femb} and the full \dword{ce} \fdth and flange.]
{fig:tpcelec_40apa}
{Left: one side of the \num{40}\,\% \dword{apa} with four \dwords{femb}.  Right: the full \dword{ce} \fdth and flange.}
\includegraphics[width=0.72\linewidth]{sp-tpcelec-40-apa.png}
\hspace{3mm}
\includegraphics[width=0.2\linewidth]{sp-tpcelec-40-apa-ft.png}
\end{dunefigure}

Each of the three set-ups that can be used for system tests has advantages
and disadvantages. Only the \dword{iceberg} \dword{tpc} can be used to 
perform measurements with tracks, but the size of the \dword{apa} is 
much smaller than that of the DUNE (which is also an advantage because it
allows for the determination of the ultimate performance of the electronics,
as the detector capacitance is reduced). The \dword{iceberg} \dword{tpc}
is for the moment the only set-up that is compatible with the new \dword{pds}
design. Tests performed in the cold box at CERN and with the \num{40}\,\% \dword{apa} 
at BNL are limited to noise measurements. These tests are not
performed at the temperature of \lar in the CERN setup. The advantage of
both set-ups and in particular of tests performed using the \dword{pdsp}
\dword{apa} is that the detector size is the one used in DUNE and not
much smaller like in the \dword{iceberg} case. During the development
of new \dwords{asic} and \dwords{femb} we plan to continue using all
three set-ups.

%%%%%%%%%%%%%%%%%%
\subsubsection{Additional Test Facilities}
\label{sec:fdsp-tpcelec-qa-facilities-additional}

For the \dword{ce} development, testing single prototypes at both room and
cryogenic temperature is the first step, as many problems can be identified
quickly without a full or partial \dword{tpc} wire readout. A test dewar
design developed by Michigan State University, referred to as the
Cryogenic Test System (\dword{cts}), allows for testing of the \dwords{femb} and
\dwords{asic} at both room temperature and submerged in \lntwo. Several \dword{cts} units
were deployed at BNL for the \dword{pdsp} production \dword{femb} QC and SBND \dword{asic} QC.
Several others have already been deployed at the institutions involved in
the \dword{asic} development to test the first prototypes of \dwords{asic}
and \dwords{femb}.  The \dword{cts} cooling process avoids the condensation
of water from air that can otherwise interfere with the tests or damage the
test equipment; two \dword{cts} units in operation at BNL are shown in Figure~\ref{fig:CTS}.

\begin{dunefigure}
[The Cryogenic Test System (\dword{cts})]
{fig:CTS}
{Cryogenic Test System: an insulated box is mounted on top of a commercial \lntwo dewar.  Simple controls allow the box to be purged with nitrogen gas and \lntwo to be moved from the dewar to the box and back to the dewar.}
\includegraphics[width=0.4\linewidth]{sp-tpcelec-CTS2.jpeg}
\end{dunefigure}

%%%%%%%%%%%%%%%%%%%%%%%%%%%%%%%%%%%
\subsection{Reliability Studies}
\label{sec:fdsp-tpcelec-qa-reliability}

The TPC cold electronics system of the DUNE single phase far detector has to meet 
stringent requirements, such as low ($<<$ 1\%) failure rate of components installed 
on the detector, inside the cryostat, without easy access during the \dunelifetime of 
detector operation. Reliability of all the components must be incorporated in the 
design and a dedicated analysis of all the possible failure mechanisms is required 
before finalizing the design of all \dwords{asic}, printed circuit boards, cables, 
connectors, and their supports, all of which are housed inside the DUNE far detector 
cryostat. 

There are a few HEP detectors that have been operated without intervention for a 
prolonged period of time, with limited losses of readout channels, and in extreme 
conditions like those of the DUNE cryostats:
\begin{itemize}
	\item The NA48/NA64 liquid Krypton (LKr) calorimeter has 13,212 channels 
	of JFET pre-amplifiers installed on detector. It has been kept at LKr temperature 
	since 1998. The failure rate is $<$ 0.2\% for 20 years of operation so far.
	\item The ATLAS \dword{lar} accordion EM Barrel (EMB) calorimeter has 
	$\sim$110,000 readout signal channels, with up to seven connections and different 
	circuit boards populated with resistors and diodes inside the cryostat. The EMB
        calorimeter has been cold since 2004, for 14 years of operation.  So far the
        failure rate of readout channels is $\sim$0.02\%.
	\item The ATLAS \dword{lar} Hadronic Endcap Calorimeter (HEC) has $\sim$5,600 
	readout channels through $\sim$35,000 cold preamplifier channels designed in GaAs 
        technology on preamplifier and summing boards (PSB). The HEC cold electronics has 
	been in cold operation since 2004, with $\sim$0.37\% failure rate during 14 years of operation. 
\end{itemize}
In addition, FERMI/GLAST is an example of a joint project between NASA and HEP groups 
that had a minimum mission requirement of five years and is on its way to achieving a 
stretch goal of ten years of operations in space. As the requirements can be rather 
different, it is important to examine and understand the various strategies for a space 
flight project compared to those of DUNE. Specifically launch "Shake and Bake" tests, 
redundancy requirements on critical infrastructure, and extensive test to failure, etc., 
may be inapplicable.

A preliminary list of reliability topics to be studied for the TPC electronics operated 
in \dword{lar} environment are:
\begin{itemize}
	\item The custom \dwords{asic} proposed for use in DUNE (\dword{larasic}, 
	\dword{coldadc}, \dword{coldata}, \dword{cryo}) incorporate design rules 
	aimed at minimizing the hot carrier effect\cite{Li:CELAr,Hoff:2015hax}, 
	which is recognized as the main failure mechanism for integrated circuits 
	operating at \dword{lar} temperature.
	\item For COTS components, accelerated lifetime testing, a standard test 
	methodology used by the semiconductor industry, shall be devised to verify 
	the expected lifetimes for operation in cryogenic temperature. A COTS ADC has 
	undergone this procedure, to be qualified as a solution for the SBND 
	experiment\cite{Chen:2018zic}.
	\item Print circuit board assemblies designed and fabricated to survive 
	repeated immersions in \lntwo.
	\item Capacitors operating voltage with sufficient margin, e.g. V$_{OP}<$ WVDC/2.
	\item Connectors and cables, usually major sources of detector channel 
	failures, impose a testing challenge.
        \item A formal \dword{qa} process for all TPC electronics components to be installed inside the cryostat.
\end{itemize}
The \dword{ce} consortium has formed a working group tasked with studying reliability issues 
of these components and is preparing recommendations for the choice of \dwords{asic}, 
the design of printed circuit boards, and testing. This working group will review the 
segmentation of the cold electronics to understand which failures are going to have the 
largest impact on data taking; revisit recommendations for the \dword{asic} design, 
beyond those aimed at minimizing the hot carrier effect; revisit the industry and 
NASA standards for the design and fabrication of printed circuit boards, connectors, 
and cables, and make recommendations for the \dword{qc} procedures to be adopted during 
the fabrication of the cold electronics components. The working group will also review
system aspects, to understand where it is desirable, necessary, and feasible to implement 
redundancy in the system, to minimize data losses due to single component failures. 
