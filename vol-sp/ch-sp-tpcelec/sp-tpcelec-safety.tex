\section{Safety}
\label{sec:fdsp-tpcelec-safety}

%%%%%%%%%%%%%%%%%%%%%%%%%%%%%%%%%%%
\subsection{Personnel Safety During Construction}
\label{sec:fdsp-tpcelec-safety-personnel}

Personnel safety during the construction, testing, integration,
and installation of the Cold Electronics components for the DUNE
Single Phase far detector is a crucial element for the success
of the project. The members of Cold Electronics consortium will
respect the safety rules of the institutions where the work is
performed, which can be one of the national laboratories, \surf,
or one of the universities participating in the project.  The
main risks for the personnel of the consortium are exposure to
liquid Nitrogen (used for cooling down components during testing),
oxygen deficiency hazard (which could be possibly caused by leaks
of either liquid Nitrogen or liquid Argon from test setups),
electrical shocks, and falls from heights. The leadership of the
Cold Electronics consortium will work with the LBNF / DUNE
ES\&H Manager and with the relevant responsible personnel at the
various institutions to ensure that all the members of the
consortium receive the appropriate training for the work they
are performing, and that all the preventive measures to minimize
the risk of safety accidents are put in place. Where appropriate
we will try to adopt the strictest standard and requirements among
those of different institutions. Hazard analyses will be performed,
and the level of personal protective equipment (PPE) will be determined
appropriately for each task. Example of PPE include the use of
appropriate gloves for handling liquid Nitrogen dewars, fall
protection equipment for work at heights, steel toed shoes and
hard hats for integration work with the \dword{apa}. Oxygen
monitors should be used for areas with large concentration of
cryogenic gases.

ES\&H plans for the activities to be performed at various institutions
will be  reviewed as part of the various reviews (Final Design,
Engineering Design, Production Readiness, Production Progress)
that will take place during the construction of the detector.

%%%%%%%%%%%%%%%%%%%%%%%%%%%%%%%%%%%
\subsection{Detector Safety During Construction}
\label{sec:fdsp-tpcelec-safety-detcon}


In addition to the personnel safety during the detector
construction, including all testing activities, and the
integration, installation, and commissioning, we have 
considered also the need for protecting the detector
components and ways to minimize the chance of damaging
them during handling. We consider two main safety risks 
for the detector during the construction and one during
operations. The most important risk during construction is damage 
induced by electrostatic discharges (\dword{esd}) in the 
electronic components. The second one is the mechanical damage of 
parts during the transport and handling. For operations we
need to consider the risk of damage to the electronics 
caused by the accumulation of dust inside the components
installed on the top of the detector. In this section 
we discuss these three risks and ways to minimize their possible 
impact. In the next section we will discuss how to prevent
damage during operations to the Cold Electronics components 
by using the interlocks of the detector safety system.

Electrostatic discharges can damage any of the electronics
components mounted on the \dwords{femb}, \dwords{wiec},
the bias voltage supplies, or the power supplies. If the
the damage occurs early in the construction process, 
the outcome is a reduction
of the yield for some of components, that needs to be
addressed with a sufficient number of spares to prevent
schedule delays associated with new procurements. \dword{esd}
damage on the \dwords{femb} after the \dwords{apa} have been
installed inside the cryostat could result in a permanent
reduction of the fraction of operating channels in the
detector. Even if most components, including the custom 
\dwords{asic} designed for use in the \dwords{femb}, contain 
some level of protection  against \dword{esd}, the occurance 
of this kind of damage cannot be discounted and appropriate 
preventive measures need to be taken during the 
assembly, testing, installation, and shipping of all the detector 
components provided by the Cold Electronics consortium. These 
measures include using appropriate ESD-safe packing materials, 
appropriate clothing and gloves, conducting wrist straps 
and / or foot-straps to prevent high voltages from accumulating 
on workers' bodies, anti-static mats to conduct harmful electric 
charges away from the work area, and humidity control. These
measure will be implemented in all laboratories where the
detector components provided by the Cold Electronics consortium,
including at the \dword{itf} and at \surf. Additional measures
include the use of custom made termination for the
power and the control and read-out cold cables when these
are being routed through the \dword{apa} frames or through the
cryostat penetrations. Storage cabines where \dwords{asic} and
\dwords{femb} are stored should have \dword{esd} mats
on the shelves and humidity control. Most importantly all personnel needs 
to be trained to take the appropriate preventive measures. We 
will require that all the personnel working on the Cold Electronics 
consortium components take a training class originally developed 
at Fermilab for handling the CCDs of the DES experiment (the 
material for this training class can also be used at remote 
sites). Monitoring of the training of the personnel at universities, 
at the \dword{itf}, and at \surf will be one of the responsibilities 
of the scientists in charge of the Cold Electronic activities in
each of the remote sites.

Most of the damage to detector components happens during their
transport among the various sites where assembly, testing, integration,
and installation take place. When appropriate, measures to prevent
ESD damage have to be taken also for shipments. Appropriate 
packaging will be used to ensure that parts are not damaged
during transport. We will perform reception tests for the 
\dwords{femb} and integration tests for the cold cables as
part of the quality control process discussed in 
Section~\ref{sec:fdsp-tpcelec-integration-qc}, to ensure the 
full functionality of these parts that are very hard to replace 
after the detector integration and installation. For other 
components that are on the top of the cryostat and that
can be replaced if damaged during transport we will be
performing integration tests after the installation. 
In addition to damage during shipping we also consider the
possibility of damage caused by handling of the detector parts.
Additional protections are being considered for operations where
the risk of damage to Cold Electronics detector components is
considered high. In the case of testing \dwords{asic} and \dwords{femb} 
in liquid Nitrogen this has resulted in the development of the
\dword{cts} discussed in Section~\ref{sec:fdsp-tpcelec-qa-facilities-additional}
to prevent condensation on the components after they are extracted from
the LN at the end of a test. In the case of the cold cables this 
has included modifying the size of the \dword{apa} frames' tubes, 
the addition of a conduit inside the frame and of a mesh around 
the cables. Special tooling will be designed for arranging the
cold cables on the spools used when the cables are routed through
the \dwords{apa}' frames. Similarly tooling will be developed to 
suport the cold cables while thye are being routed through the 
cryostat penetrations. All cables and fibers will
be installed in cable trays on top of the cryostat, and in the case
of fibers additional protections in the form of sleeves or tubes 
are also being considered. 

To ensure that the DUNE detector will be operational for a long
period of time we are also considering ways to minimize the
damage that could happen to detector components inside the 
experimental cavern. These can have two sources: incorrect
operation of the detector and environmental conditions. We
will discuss the first in the next section. The environmental
conditions inside the cryostat are extremely stable once the
filling with liquid Argon is complete. The experience from 
previous experiments that have operated electronics inside the
\dword{lar} indicates that apart from initial problems there
is hardly any loss of read-out channels over long periods of
time. Therefore the main worry is the electronics that is
installed on the top of the cryostat. There the main problem
is dust accumulation on the detector components. In the long
term the dust could lead to damage of the cooling fans used
in the \dwords{wiec} and the Cold Electronics racks, to breakdown 
on the surface of diodes that are used in bias voltage supplies,
and, if the dust has any amount of salt and if there is sufficient humidity
in the air, to the growth of dendrites that could create 
shorts between traces on a printed circuit board. While the
experimental cavern is expected to be a very dry environment
protections will be put in place to prevent water from 
dripping on the \dwords{wiec} and on the racks containing the
Cold Electronics supplies. HEPA filters will be added on the
air supplies used to cool the \dwords{wiec} and the Cold
Electronics power and bias voltage supplies, to minimize
the accumulation of dust. The air humidity in the cavern 
will be controlled to prevent condensation from occurring.

%%%%%%%%%%%%%%%%%%%%%%%%%%%%%%%%%%%
\subsection{Detector Safety During Operations}
\label{sec:fdsp-tpcelec-safety-detops}

In this Section we discuss areas where we will be using
the Detector Safety System described in Chapter~\ref{fd:ZZZAAA}
to avoid unsafe conditions for the Cold Electronics detector 
during operations. Hardware interlocks will be put in place
for major installations to prevent the operation of some 
components or to shut-down the power to those components 
unless the conditions are safe
both for the detector and for personnel. Interlocks will be
used on all the low voltage power and on the bias voltage 
supplies, including inputs from environmental monitors both
inside and outside the cryostat. Examples of interlocks that
will be used include turning off the power to the \dwords{wiec}
if the corresponding cooling fans are not operational or
if the temperature inside the crates exceeds a pre-set value.
Similar interlocks will be used for the low voltage power
and the bias voltage supplies in the Cold Electronics racks.
Interlocks may be required connecting the value of the 
bias voltage on the field cage termination electrodes to the
high voltage applied on the TPC cathode. Interlocks will turn 
off the transmitters on the \dwords{wiec} if the readout fibers 
bundles are cut. One of the problems we have to address is 
the connection between the \dword{plc} used by the Detector 
Safety System and the \dwords{wiec}, to avoid introducing noise 
inside the detector. We can easily decouple the environmental 
sensors required by the Detector Safety System inside the 
\dwords{wiec} by following the appropriate grounding rules. 
The connection used to provide the enable/disable signals 
from the \dword{plc} to the \dwords{wiec} will require optical 
fibers to avoid possible ground loops. Interlocks connected
to the Detector Safety System will also be used during tests 
of the \dwords{apa} in the cold boxes at the \dword{itf} and 
at \surf. The \dword{cts} has its own interlock system to
prevent condensation from forming on the \dwords{femb} once
they are warmed up to room tempearature. We cannot exclude at
this point that for some of the smaller test stands we will 
have to rely on software interlocks for the detector safety,
but their number will be kept to a minimum and no software
interlock should be in use at the \dword{itf} and at \surf.
