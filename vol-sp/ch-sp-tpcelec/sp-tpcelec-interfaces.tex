\section{Interfaces}
\label{sec:fdsp-tpcelec-interfaces}

%%%%%%%%%%%%%%%%%%%%%%%%%%%%%%%%%%%
\subsection{Overview}
\label{sec:fdsp-tpcelec-interfaces-overview}

Table~\ref{tab:CEinterfaces} contains a brief summary of all the interfaces
between the \dword{ce} consortium and other consortia or other groups,
with references to the current version of the interface documents. More
details are given about each interface in the following Sections.
In some cases the interface documents actually involve more than one 
consortium (one example is the bias voltage distribution system where
the interface involves both the \dword{apa} and the \dword{hv} consortium).
Every attempt has been made to have all the corresponding interface documents 
consistent in such cases. At this stage most of the interface documents are
not yet complete, and all interfaces are mostly defined at the conceptual
level. Drawings of the mechanical interfaces and diagrams
of the electrical interfaces are still under development. It is expected 
that further refinements of the interface documents will take place in the
remainder of 2019 and in the first half of 2020 prior to the Engineering Design
Reviews of the detector. All the interface documents specify the responsibility
of different consortia or group during all the phases of the experiment
including during the design and prototyping, the integration, the installation,
and the commissioning.

\fixme{The interfaces with the Calibration consortium, and with the Physics
and the Software and Computing groups have yet to be defined. Once they are
defined some text should be added in this Section and in the Table below.
References to documents in DocDB need to be fixed. }

\begin{dunetable}
[\dword{hv} system interfaces]
{p{0.25\textwidth}p{0.5\textwidth}l}
{tab:CEinterfaces}
{High Voltage System Interface Links }
Interfacing System & Description & Linked Reference 
\\ \toprowrule
\dword{apa} & Mechanical (cable trays, cable routing, connections of CE boxes and 
frames) and electrical (bias voltage, \dword{femb}--CR boards connection, grounding 
scheme) & \cite{bib:docdb-6670}
\\ \colhline
\dword{daq} & Data output from the \dword{wib} to the \dword{daq} back-end, clock distribution,
controls and data monitoring responsibilities & \cite{bib:docdb-6742}
\\ \colhline
CISC & Rack layout, controls and data monitoring & \cite{bib:docdb-6745}
\\ \colhline
\dword{hv} & Grounding, bias voltage distribution, installation and testing & \cite{bib:docdb-6739}
\\ \colhline
\dword{pds} & Electrical (cable routing and installation), cold flange & \cite{bib:docdb-6718}
\\ \colhline
Facility & Cable trays inside the cryostat, cryostat penetrations, rack layout and
power distribution on the detector mezzanine, cable and fiber trays on top of the
cryostat& \cite{bib:docdb-6973}
\\ \colhline
Installation Team & Sequence of integration and installation activities at \surf,
equipment required for  \dword{ce} consortium activities & \cite{bib:docdb-7000}
\\ \colhline
\dword{itf} & Material handling and testing activities of  the \dword{ce} 
consortium at the \dword{itf}, cold box for \dword{apa} tests& \cite{bib:docdb-7027}
\\ \colhline
Calibration & TBD & \cite{bib:docdb-7027}
\\ \colhline
Physics & TBD & \cite{bib:docdb-7027}
\\ \colhline
Software \& Computing & TBD & \cite{bib:docdb-7027}
\\
\end{dunetable}

%%%%%%%%%%%%%%%%%%%%%%%%%%%%%%%%%%%
\subsection{APA}
\label{sec:fdsp-tpcelec-interfaces-apa}

The most important interface is that between the \dword{ce}
and the \dword{apa} consortia. The design of the \dwords{femb}
and of the \dword{apa}s are strictly intertwined, both from the
mechanical and the electrical point of view. The \dword{ce}
boxes housing the \dwords{femb} are supported by the \dword{apa}
and at the same time attached to the CR boards of the \dword{apa}
through a connector that passes all the signals from the wires to
the \dword{fe} amplifiers. The cable trays that house both the
\dword{ce} and the \dword{pds} cold cables are initially
attached to the yoke of the \dword{apa}. The \dword{ce}
cables for the bottom \dword{apa} need to be routed through the 
frames of both the bottom and top \dword{apa}s. The \dword{ce}
consortium provides the bias voltage for the \dword{apa}
wires, as well as for the electron diverters and the field
cage termination electrodes (the latter are a responsibility of
the \dword{hv} consortium), using the \dword{shv} boards that are mounted
on the \dword{apa}s. The grounding requirements discussed in
Section~\ref{sec:fdsp-tpcelec-design-grounding} inform the
design of all the mechanical and electrical interfaces between
the \dword{ce} components and the \dword{apa}s and also the
design of the connections between the top and bottom \dword{apa}s
and between the top \dword{apa} and the \dword{dss}. All the
integration and installation activities that take place both at
the \dword{itf} and at \surf need to be carefully coordinated between
the two consortia, and where appropriate also with the \dword{pds}
consortium and Technical Coordination.

%%%%%%%%%%%%%%%%%%%%%%%%%%%%%%%%%%%
\subsection{DAQ}
\label{sec:fdsp-tpcelec-interfaces-daq}

The \dword{daq} is responsible for receiving the data produced by the
\dword{ce} detector components and further processing it to
form trigger decisions and later for transferring it to 
permanent storage for analysis. The 
\dword{daq} is also responsible for delivering the clock and control
signals to the \dwords{wiec}. The interfaces are realized 
through optical fibers, to ensure that no noise is fed into
the \dwords{wiec}. One fiber per \dword{wiec} delivers the
clock and the control signals to the \dword{ptc} that then
rebroadcasts this information to the \dwords{wib} in that 
crate. Each \dword{wib} reads out the data from four \dwords{femb}
and transmits it through two 10~Gbps links to the \dword{daq} back-end.
The data signals are carried on multi--mode fibers, compatible with 
either the OM3 or OM4 standards. Fibers will be formed into trunks with 
12 or 24 fibers and terminated into MTP connectors. The feasibility
of this data transmission scheme with fibers with a length of up
to \SI{300}{m} has been demonstrated at CERN in Summer 2018 using 
\dword{pdsp} components. The data format to be used for 
\dword{dune} will be an evolution of that adopted for
\dword{pdsp}, taking into account the need for an 
extended address space to accommodate the larger number of
\dwords{femb} in the detector. The \dword{daq} consortium
is also responsible for providing the software environment
used for downloading the detector configuration.

%%%%%%%%%%%%%%%%%%%%%%%%%%%%%%%%%%%
\subsection{HV}
\label{sec:fdsp-tpcelec-interfaces-hv}

The interface with the \dword{hv} consortium
is driven by the fact that the \dword{ce} flange provides the return path for
the small current that flows from the high voltage power 
supply through the cathode panels, the field cages and then
the termination electrodes. The hardware interface takes place through
the \dword{shv} boards that are mounted on the \dword{apa}s, that
are  the responsibility of the \dword{apa} consortium. The
\dword{shv} boards also distribute the bias voltage to the field
cage termination electrodes. The \dword{ce} consortium
is responsible for bringing the bias voltage for the field
cage termination electrodes to the \dword{shv} boards. Appropriate
rules for avoiding ground loops are also included in the 
interface document.

%%%%%%%%%%%%%%%%%%%%%%%%%%%%%%%%%%%
\subsection{Technical Coordination}
\label{sec:fdsp-tpcelec-interfaces-tc}

In this section we consider the interfaces with the Facility (LBNF)
and the DUNE Technical Coordination, as well as the interfaces with
the Integration and Test Facility and that with the Underground
Installation Team. The \dword{ce} consortium has several
interfaces with the Facility, namely the cable trays inside the
cryostat, the cryostat penetrations used by the \dword{ce}
and by the \dword{pds} consortia, and the racks and trays on top
of the cryostat. The \dword{ce} consortium is responsible
for the design, procurement, and installation of the cable trays
inside the cryostat and of the cryostat penetrations. The DUNE
Technical Coordination will be responsible for providing the racks,
including their power, cooling and monitoring systems, and the interlocks,
where the low voltage power supplies and the bias voltage supplies
for the \dword{ce} detector components will be installed, as well as the trays
connecting these racks to the corresponding \dword{wiec}. Technical
Coordination will also be responsible for the network switches that
are used to connect the controls for the \dword{ce} to the
\dword{daq} and \dword{cisc} back-ends. Technical Coordination will also be responsible
for providing the Detector Safety System that will be used to protect
the \dword{ce} detector components. The \dword{ce}
consortium will work together with Technical Coordination to establish
the action matrix for the Detector Safety System and the hardware
interlocks.

The \dword{ce} consortium will work with the teams responsible
for the Integration and Test Facility and for the Underground Installation
to plan all the activities that take place at the \dword{itf} and at
\surf. This will include developing plans to outline the responsibilities
of the consortium and those of the \dword{itf} personnel or of 
Technical Coordination for the activities at \surf, including all the
shipment, transport, and logistic related activities, as well as all
the integration and installation activities. Cold boxes for testing
the \dword{apa}s after integration with the \dwords{femb} will be 
provided by the \dword{itf} and by Technical Coordination at \surf.
All the other testing equipment will be provided either by the \dword{ce}
consortium or by other consortia. Equipment required to
minimize the risk of \dword{esd} Damage to the detector components
will  be provided by the \dword{ce} consortium. 

%%%%%%%%%%%%%%%%%%%%%%%%%%%%%%%%%%%
\subsection{Other Interfaces}
\label{sec:fdsp-tpcelec-interfaces-other}

The interface with the \dword{pds} consortium is relatively simple.
The \dword{pds} detector component should be isolated from the \dword{ce}
detector component except for sharing a common reference 
voltage point (ground) at the chimneys. Inside the cryostat the 
\dword{pds} and \dword{ce} cables will be housed together in
cable trays that are the responsibility of the \dword{ce}
consortium. The \dword{ce} consortium will also take over
the responsibility for routing the \dword{pds} cables through the
cryostat penetration and for connecting them to the \dword{pds}
flange. The flange itself will be designed and build by the \dword{pds}
consortium, but its test and final integration on the spool piece
of the cryostat penetration will be a responsibility of the \dword{ce}
consortium. The \dword{ce} consortium may also
take the responsibility of performing the connections of the 
\dword{pds} cables between the flange the and mini-racks housing
the \dword{pds} warm electronics on the top of the cryostat, as
well as connecting the clock distribution system for the mini-racks.

The \dword{cisc} consortium provides the software infrastructure for the slow
control and monitoring of the status of the \dword{ce} components.
The \dword{cisc} and \dword{ce} consortia may also have an
hardware interface as they may share the same racks on the top of the
cryostat. The most important aspect of the interface between these
two consortia is the requirement from the \dword{ce} consortium
to have all the relevant part of the slow control and monitoring
equipment functional already at the time of the pre-production of
the detector components for the \dword{itf} and at the beginning
of the installation for \surf. 
