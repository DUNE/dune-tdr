%\begin{table}[htp]
  \caption{Specification for TOP-11 \fixmehl{ref \texttt{tab:spec:hvs-field-uniformity}}}
  \centering
  \begin{tabular}{p{0.2\textwidth}p{0.75\textwidth}} 
     \rowcolor{dunesky}
    \newtag{TOP-11}{ spec:hvs-field-uniformity } 
                & Name: Drift field uniformity due to HVS    \\ 
    Description & Design of TPC cathode and FC components shall ensure uniform field.  Production tolerances shall be set so as to maintain flatness of component surfaces and, by extension, the shape of the drift field volume.   \\  \colhline
    
    Specification &  < \num{1}\% throughout volume \\   \colhline
    
    Rationale &  { Non-uniformity of \efield affects 3D reconstruction due to introduction of non-constant electron drift velocity, including residual transverse components with respect to the nominal drift direction. } \\ \colhline
    Validation &{ Space charge in ProtoDUNE will complicate the analysis of the field uniformity. Local effects close to the field cage electrodes could be in principle be disentagled, exploiting through-going muon tracks. Simulation will be used to determine fiducial volume cuts, with input from the ProtoDUNE data. } \\    
   \colhline
  \end{tabular}
  \label{tab:spec:hvs-field-uniformity}
\end{table} % 11
%\input{generated/req-TOP-hv-ps-ripple.tex} % 12
%\begin{table}[htp]
  \caption{Specification for TOP-16 \fixmehl{ref \texttt{tab:spec:det-live-time}}}
  \centering
  \begin{tabular}{p{0.2\textwidth}p{0.75\textwidth}} 
     \rowcolor{dunesky}
    \newtag{TOP-16}{ spec:det-live-time } 
                & Name: Detector live time    \\ 
    Description & The detector shall operate stably at least 90\% of the time in order to collect the required physics data.   \\  \colhline
    
    Specification &  > \num{90} \% \\   \colhline
    
    Rationale &  { In order to collect the required physics data, the detector must operate stably over long time periods. } \\ \colhline
    Validation &{ Operation of the ProtoDUNE detector will provide information on the stability of the HVS over time.   } \\    
   \colhline
  \end{tabular}
  \label{tab:spec:det-live-time}
\end{table} % 16
%\input{generated/req-TOP-cathode-resistivity.tex} % 17
%\input{generated/req-TOP-local-e-fields.tex} % 24

\begin{longtable}{p{0.25\textwidth}p{0.7\textwidth}}   
\caption{Top-level SP FD specification that apply to HV } \\

    \rowcolor{dunesky}
    \newtag{SP-FD-11}{ spec:hvs-field-uniformity } 
                & Name: Drift field uniformity due to HVS    \\ 
    Description & Design of TPC cathode and FC components shall ensure uniform field.  Production tolerances shall be set so as to maintain flatness of component surfaces and, by extension, the shape of the drift field volume.   \\  \colhline
    Specification (Goal) &  $<\,\SI{1}{\%}$ throughout volume  ( ALARA ) \\   \colhline
    
    Rationale &   Non-uniformity of \efield affects 3D reconstruction due to introduction of non-constant electron drift velocity, including residual transverse components with respect to the nominal drift direction.  \\ \colhline
    Validation & Space charge in ProtoDUNE will complicate the analysis of the field uniformity. Local effects close to the field cage electrodes could be in principle be disentagled, exploiting through-going muon tracks. Simulation will be used to determine fiducial volume cuts, with input from the ProtoDUNE data.  \\
   \colhline
   
        \rowcolor{dunesky}
     \rowcolor{dunesky}
    \newtag{SP-FD-12}{ spec:hv-ps-ripple } 
                & Name: Cathode HV power supply ripple contribution to system noise    \\ 
    Description & Power supply ripple shall be adequately attenuated to guarantee that its contribution to the overall system electronics noise  is negligible.   \\  \colhline
    Specification (Goal) &  $<\,\SI{100}{enc}$  ( ALARA ) \\   \colhline
    
    Rationale &     \\ \colhline
    Validation & Effect of HV system on the baseline electronics noise should be easily measured by switching on and off the HV power supply.  Equivalent full circuit simulation of the coupling between the HVS and the FE electronics will be performed to compare against ProtoDUNE data and validate the FD design.  \\
   \colhline
   
      \newtag{SP-FD-16}{ spec:det-dead-time } 
                & Name: Detector dead time    \\ 
    Description & The down time of the detector should be such that data taking interruptions affecting all active cryostats are kept below $<$0.5\% with a goal of ALARA   \\  \colhline
    Specification (Goal) &  $<\,\SI{0.5}{\%}$  ( ALARA ) \\   \colhline
    
    Rationale &   In order to collect the required physics data, the detector must operate stably over long time periods. The specification is driven by the risk of missing a supernova burst if all operating cryostats are offline.  \\ \colhline
    Validation & Operation of the ProtoDUNE detector will provide information on the stability of the HVS over time.    \\
   \colhline
        \rowcolor{dunesky}
   \newtag{SP-FD-17}{ spec:cathode-resistivity } 
                & Name: Cathode resistivity    \\ 
    Description & The cathode resistivity shall ensure that in the event of an HV discharge, the release of the large stored energy is spread out over time.    \\  \colhline
    Specification (Goal) &  $>\,\SI{1}{\mega\ohm/square}$  ( $>\,\SI{1}{\giga\ohm/square}$ ) \\   \colhline
    
    Rationale &   This prevents damage to detector components. The goal resistivity is > \SI{1}{\giga\ohm}/sq  \\ \colhline
    Validation & In ProtoDUNE the resistivity of the CPA panels  is in the $M\Omega/sq$ range due to the detector being operated on the surface.  There will be opportunities to test if the front-end electronics is adequately protected against discharges (hopefully at the end of beam operations).  Existing discharge simulations can be tuned based on ProtoDUNE data to better validate far detector designs.  \\
   \colhline
   
       \rowcolor{dunesky}
    \newtag{SP-FD-24}{ spec:local-e-fields } 
                & Name: Local electric fields    \\ 
    Description & The integrated detector design shall minimize potential pathways for HV discharges.   \\  \colhline
    
    Specification &  <\SI{30}{kV/cm} \\   \colhline
    
    Rationale &  { HV discharges limit detector livetime and have the potential for damaging detector components. Keeping the local fields to this value is necessary in order to reach the desired drift fields. The minimum \efield requirement is based on the minimum drift-field goal of \SI{500}{V/cm}. } \\ \colhline
    Validation &{  } \\    
   \colhline
\end{longtable} 
