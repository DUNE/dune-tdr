\chapter{Data Acquisition}
\label{ch:sp-daq}

\fixme{\textbf{Authors:}

  See \url{https://wiki.dunescience.org/wiki/Technical_Design_Report}
  for general guidance. 

  While this chapter is still in outline, \textbf{check that it hits all
  the required points} some of which are:

  We are to describe a \textbf{baseline} or \textbf{process to
  decide a baseline}.

  \textbf{BE SUCCINCT} $TDR \approx IDR + 10\%$, goal is 50 pages for
  this chapter. 

  You are encouraged to produce \textbf{tech notes} with any
  supporting verbosity which may be referenced.

  State requirements and demonstrate how they are met, use
  standardized requirements table.

  Emphasize safety and professionalism (projectisms: cost, schedule,
  risks, interfaces).}

\metainfo{Some sections of this chapter must be written generically
  and without any reference to module-specific terms. They are marked
  with an orange ``fixme'' box. 
  Yellow info boxes like this one provide guidance for the content. 
  This guidance is not comprehensive so authors may provide additional
  information but retaining \textbf{conciseness} and \textbf{not
    repeating} info in other section is required.}

\section{Introduction}
\label{sec:fd-daq:introduction}
\fixme{module-generic}

\metainfo{A brief introduction to this chapter describing what will be
  described.  This is \textbf{not} an overview of the DAQ itself.
  Keep it brief. Do \textbf{not} write a conceptual overview here,
  that is below, reference it. 
  Do \textbf{not} use module-specific language but \textbf{do}
  describe how commonalities are described in text shared by both
  SP/DP volumes and specialized sections appear only in their
  respective volume. 
  \textbf{Do} describe the lexicographical convention used to demark
  shared sections (this needs coordination with other chapters in the
  same boat).}

\section{DAQ Requirements and Parameters}
\label{sec:fd-daq:req-and-params}
\fixme{module-generic}

\metainfo{Sentence explaining requirements are generic to a module and parameters are specific to a module}

\subsection{Requirements}
\label{sec:fd-daq:requirements}
\fixme{module-generic}

\metainfo{Include rows of top-level requirements table (``Schmitz''
  table) here. 
  Augment that with any additional requirements of our determining. 
  Eg: accept data from detector electronics, perform reduction to
  satisfy output rate limit, allow for cross-module triggering,
  collect beam activity with XX\%, SNB requirements, noise level,
  total thermal and space envelop, etc....}

\subsection{Parameters}
\label{sec:sp-daq:requirements}
\fixme{single-phase module}

\metainfo{Include a table which lists all important parameters driving
  the design.  Sampling rate and resolution, channel count}

\section{DAQ Design}
\label{sec:fd-daq:design}
\fixme{module-generic}

\metainfo{One sentence to introduce the section. 
  Listing the subsections and describing which are generic and which
  are X-phase specific.}

\subsection{Overview}
\label{sec:fd-daq:design-overview}
\fixme{module-generic}

\begin{dunefigure}{fig:daq-conceptual-overview}{DAQ Conceptual
    Subsystem Overview. 
    Concrete hardware or software systems may span multiple subsystems
    or cover only portions of a subsystem. High level walk through of
    and referencing the remaining subsections in this section. }
  \includegraphics[width=0.8\textwidth]{daq-toplevel-conceptual.pdf}
\end{dunefigure}

\metainfo{This is the \textbf{only} place to describe the conceptual
  overview. 
  Do \textbf{not} repeat this info in sections below. 
  \textbf{Do} use \textbf{module-neutral} terms.
  \textbf{Do} describe major interface between each subsystem (the edges between the circles in Fig~\ref{fig:daq-conceptual-overview}).
  \textbf{Do} mention that concrete systems span portions of the
  conceptual subsystems and how the following subsections are defined
  along these concrete deliverable lines.}

\metainfo{Include components summary and table here.}

\metainfo{For the remaining sections, place information about
  validations into section~\ref{sec:sp-daq:design-validation}.}
  

\subsection{Message Passing System}
\label{sec:fd-daq:design-messages}
\fixme{module-generic}

\metainfo{Describe how all the nodes in the DAQ ``bow tie'' graph talk
  to each other by message passing.
  Do \textbf{not} describe detailed functionality of the nodes
  themselves, that comes next.  
  Do describe the scope in terms of what parts described below
  participate in the message passing. Include table of running
  patterns (nominal self trigger, pulser FE calibration, external
  triggered calibration (eg laser, rad source).}

\subsection{Run Control and Monitoring}
\label{sec:fd-daq:design-run-control}
\fixme{module-generic}

\metainfo{Describe run control and DAQ operation monitoring. 
  How it makes use of the Message Passing System. 
  What is an ``epoch''.
  How Epoch Change Requests lead to zero down time reconfiguration. 
  Public key based ``iron'' authentication between run control
  processes and the controlled processes. 
  Describe how partitioning, reconfiguration, run changes, node
  discovery, configuration, logging, startup, shutdown and failures
  are handled. Describe how RC will support detector electronics configuration.}

\subsection{Data Receiver Hardware}
\label{sec:sp-daq:design-front-end}
\fixme{single-phase module}

The front-end hardware provides the Data Receiver subsystem and front
end portions of the  Buffer and Trigger subystems.

\metainfo{One or two introductory paragraphs that intentionally
  consider all the multiple options are being pursued for the Data
  Receiver role of the SP. 
  It should describe our strategy for settling on an option. 
  This section is module specific.}.

\subsection{Data Handling}
\label{sec:sp-daq:design-fe-data-processing}

\metainfo{One or two sentences that positions the two processing
  patterns in this section as options. 
  Say firmware option is used for ``baseline costing''.
  Processing here includes RAM buffer, SNB buffer, and possibly
  compression. 
  It doesn't include the trigger primitive algorithms themselves (for
  example, that's below in section~\ref{sec::sp-daq:design-selection-algs}).}


% \subsubsection{FELIX+FPGA}
\subsubsection{Firmware-based Option}
\label{sec:sp-daq:design-felix-fpga}
\fixme{single-phase module}

\metainfo{Include full hardware scope starting at fibers from CE and
  ending at the output of trigger processors and the interface between
  buffer and the Data Selector.
  Describe the per-APA multiplicity of computers, CPU cores, host
  system RAM, host system storage, FELIX boards, DPM components (RAM,
  SSD). 
  Include thermal estimates itemized by components.}

% \subsubsection{FELIX+CPU}
\subsubsection{Software-based Option}
\label{sec:sp-daq:design-felix-cpu}
\fixme{single-phase module}

\metainfo{Include full hardware scope starting at fibers from CE and
  ending at the output of trigger processors and the interface between
  buffer and the Data Selector. 
  Describe the per-APA multiplicity of computers, CPU cores, host
  system RAM, SSD and FELIX boards. 
  Include thermal estimates itemized by components.}


\subsubsection{DP data ingest via UDP}
\label{sec:dp-daq:design-udp-ingest}
\fixme{dual-phase module, move to DP-DAQ chapter eventually}
\metainfo{This is a DP section and will be only in the DP volume. 
  It should describe the ``Bump On Wire'' from the IDR unless we can
  come up with any new/better ideas.}
\metainfo{Include full hardware scope starting at fibers from CE and
  ending at the output of trigger processors and the interface between
  buffer and the Data Selector.
  Describe the per-APA multiplicity of computers, CPU cores, host
  system RAM, host system storage, FELIX boards, DPM components (RAM,
  SSD). 
  Include thermal estimates itemized by components.}  


\subsection{Data Selection}
\label{sec:sp-daq:design-selection-algs}
\fixme{single-phase module}

Data Selection must make the decision about what data will be
transferred to the Egress Subsystem. 
It does by providing the core payload software which runs in message
passing nodes. 
In particular it provides the trigger primitive, candidate and
command hierarchy.

\metainfo{One sentence that briefly describes the ``bow tie''
  hierarchy: primitive, candidate, command, query. 
  Do not repeat too much what is in the design overview. 
}


\subsubsection{Trigger Primitives}
\label{sec:sp-daq:design-trigger-primitives}
\fixme{single-phase module}

\metainfo{Describe what they are, how they are formed, storage size on
  disk, message schema.}
\metainfo{Reference the section that provides validation.}



\subsection{Data Egress System}
\label{sec:fd-daq:design-data-egress}
\fixme{module-generic}

\metainfo{This subsystem starts with receiving trigger commands from
  MTL, based on their content it queries Data Selectors on all front
  end computers, forms events, writes files to offline buffer disk. 
  It may perform ``offline'' type processing along the way.}

\subsubsection{Event builder}
\label{sec:fd-daq:design-event-builder}
\fixme{module-generic}

\metainfo{Explain artDAQ, handling of trigger commands by
  asynchronous, parallel queries to front end Data Selector (but take
  care not to duplicate between here and in the overview).}

\subsubsection{Data Model}
\label{sec:fd-daq:design-data-model}
\fixme{module-generic}

\metainfo{Describe the data model. 
  This isn't a strict schema just things like how various parts of the
  detector readout map to files, etc.}

%% no wishlists
% \subsection{Egress Processing}
% \label{sec:fd-daq:design-egress-processing}
% \fixme{module-generic}
% \metainfo{This may be a place to describe additional data
%   reduction/selection/identification, data quality/condition
%   monitoring.}


\subsection{Timing Distribution}
\label{sec:sp-daq:design-timing}
\fixme{single-phase module}
\fixme{Is it indeed still single-phase specific?}

\metainfo{Hardware, consumers, links.}

\subsection{Design Validation}
\label{sec:sp-daq:design-validation}
\fixme{single-phase module}

\metainfo{One sentence to describe our validation strategy: exploit
  ProtoDUNE, use simulation and develope vertical slice tests. 
  Put each validation study (performed or future) in a subsubsection
  and describe either \textbf{how it justifies a decision} or
  \textbf{how its outcome will be used to make a decision in the
    future}.}

\subsubsection{FELIX Throughput Demonstration at ProtoDUNE-SP}
\label{sec:sp-daq:validation-pdune-felix}
\fixme{single-phase module}

\metainfo{Describe how the FELIX DAQ at ProtoDUNE-SP demonstrates a
  FELIX+CPU approach. 
  Describe the elements that are same or similar (full-rate to host
  RAM buffer) and different (higher-rate but external trigger).}


\subsubsection{RCE Throughput Demonstration at ProtoDUNE-SP}
\label{sec:sp-daq:validation-pdune-rce}
\fixme{single-phase module}

\metainfo{Describe how RCE DAQ at ProtoDUNE-SP demonstrates an FPGA
  approach with DUNE.}

\subsubsection{Trigger Primitives in Software}
\label{sec:sp-daq:validation-software-trigger-primitives}
\fixme{single-phase module}
 
\metainfo{Succinctly describe algorithm, include physics and computing
  performance numbers.}

\subsubsection{Trigger Primitives in Firmware}
\label{sec:sp-daq:validation-firmware-trigger-primitives}
\fixme{single-phase module}

\metainfo{Succinctly describe algorithm, include physics and computing
  performance numbers.}

\subsubsection{Vertical Slice Demonstrations}
\label{sec:sp-daq:validation-demonstrators}
\fixme{single-phase module}

\metainfo{Describe VST demonstrators and why we must build them.}

\subsubsection{Prototype Message Passing System}
\label{sec:fd-daq:validation-demonstrators}
\fixme{module-generic}

\metainfo{This is actually module-generic. 
  Very briefly describe the prototype message passing. 
  This will mostly refer to a tech note.}

\subsubsection{Another validation....}
\fixme{write me, as needed}

\section{Cost, Schedule, and Risk Summary.}
\label{sec:sp-daq:cost}
\metainfo{
Include cost summary and table here.
Include schedule summary and table here.
Include risk summary and table here.}

\section{Production, Assembly and Development}
\label{sec:sp-daq:production}

\metainfo{Describe how hardware, firmware and software will produced. 
  Discussion is probably needed what this means.}



\section{Installation, Integration and Assembly}
\label{sec:sp-daq:installation}

\metainfo{Describe how we get stuff in place underground (and in the
  ITF), how we will put it all together and make sure it works. 
  What can we do to minimize the effort needed underground both in
  terms of physical work but also in working out the bugs both in
  individual processes and in emergent behavior of the system as a
  whole?}

\section{Interfaces}
\label{sec:sp-daq:interfaces}
\metainfo{Include interface summary and table here.}
\subsection{TPC Cold Electronics}
\metainfo{Data reception physical and logical, configuration information delivery.}
\subsection{PDS Readout System}
\metainfo{Data reception physical and logical, configuration information delivery.}
\subsection{Computing}
\metainfo{Buffer disk. 
  Agreement on sysadmin support and computer procurement, ssh
  gateways, non data networks.}
\subsection{CISC}
\subsection{Timing}
\metainfo{This is an ``outgoing'' interface from DAQ Timing to others.}