%\chapter{Detector Installation}
%\label{ch:sp-tc}
% \fdth{}s
% \cite{bib:docdb8255}

This chapter covers all the work and infrastructure required to install the \dword{sp} detector module. 
 
We first provide a reminder of the scale of the task, beginning with the two facts that drive all others: A \dword{dune} \dword{fd} module is enormous, with outer cryostat dimensions  of 
\cryostatlen{}(L) $\times$ \cryostatwdth{}(W) $\times$ \cryostatht{}(H) m$^{3}$; 
%(L W H) $62\times 19\times 18$ m$^{3}$; 
and every piece of a \dword{detmodule} must descend 
\SI{1500}{m} down the Ross shaft to the 4850-foot level of \dword{surf} and be transported to a detector cavern.


The \dword{spmod}'s 150 \dwords{apa}, each $6.0$ m high and $2.3$ m wide, and  weighing $600$ kg with $3500$ strung sense and shielding wires, must be taken down the shaft as special ``slung loads'' and moved to the area just outside the \dword{dune} cryostat. 
The \dword{apa}s are moved into a \SI{30 x 19}{m} clean room (a portion of which is \SI{17}{m} high) where they are outfitted  with \dword{pd} units and passed through a series of qualification tests.
Here two \dword{apa}s are linked into a vertical \SI{12}{m} high double unit and connected to readout electronics. 
They receive a cold-test in place, then move into the cryostat to be connected at the proper location on the previously installed \dword{dss} and and have their cabling connected to \fdth{}s. 
Additional systems are installed in parallel with the \dword{apa}s such as the \dword{fc} and their HV connections, elements of the \dword{cisc}, and detector calibration systems; the cathode plane, field cage and \dword{apa} together define the TPC active volume.


After twelve months of detector component installation, which follows twelve months of detector infrastructure installation, the cryostat closes (with the last installation steps occurring in a confined space accessed through a narrow human access port). 
Following leak checks, final electrical connection tests, and installation of the neutron calibration source, the process of filling the cryostat with $17,000,000$ kg of \dword{lar} begins.

The installation requires meticulous planning and execution of thousands of tasks by well trained teams of technicians, riggers, and detector specialists. 
High-level requirements for these tasks are spelled out in Table~\ref{tab:specs:just:SP-INST}
\footnote{\dword{apa}s are produced well in advance of their installation date. They are shipped to the storage facility immediately after fabrication and testing in order to control the risk of damage in shipping.} 
%This produced the requirement for \dword{apa} storage space.} 
and the text that follows it. 
In all the planning and future work, the pre-eminent requirement in the installation process is safety.
\dword{dune}'s goal is zero accidents resulting in personal injury, damage to detector components, or harm to the environment.


%% This file is generated, any edits may be lost.
\begin{footnotesize}
%\begin{longtable}{p{0.14\textwidth}p{0.13\textwidth}p{0.18\textwidth}p{0.22\textwidth}p{0.20\textwidth}}
\begin{longtable}{P{0.12\textwidth}P{0.18\textwidth}P{0.17\textwidth}P{0.25\textwidth}P{0.16\textwidth}}
\caption{Specifications for SP-INST \fixmehl{ref \texttt{tab:spec:SP-INST}}} \\
  \rowcolor{dunesky}
       Label & Description  & Specification \newline (Goal) & Rationale & Validation \\  \colhline

   
  \newtag{SP-INST-1}{ spec:logistics-material-handling }  & Compliance with the SURF Material Handling Specification for all material transported underground  &  SURF Material Handling Specification &  Loads must fit in the shaft be lifted safely. &  Visual and documentation check \\ \colhline
     % 1
   
  \newtag{SP-INST-2}{ spec:logistics-shipping-coord }  & Coordination of shipments with CMGC; DUNE to schedule use of Ross Shaft  &  2 wk notice to CMGC &  Both DUNE and CMGC need to use Ross Shaft &  Deliveries will be rejected \\ \colhline
     % 2
   
  \newtag{SP-INST-3}{ spec:logistics-materials-buffer }  & Maintain materials buffer at logistics facility in SD   &  $>1$ month &  Prevent schedule delays in case of shipping or customs delays &  Documentatation and progress reporting \\ \colhline
     % 3
   
  \newtag{SP-INST-4}{ spec:apa-storage-sd }  & APA stroage at logistics facility in SD  &  700 m$^2$ &  Store APAs during lag between production and installation &  Agree upon space needs \\ \colhline
     % 4
   
  \newtag{SP-INST-5}{ spec:cleanroom-specification }  & Installation cleanroom Specificaiton  &  ISO 8 &  Reduce dust (contains U/Th) to prevent induced radiological background in detector &  Monitor air purity \\ \colhline
     % 5
   
  \newtag{SP-INST-6}{ spec:cleanroom-uv-filters }  & UV filter in installation cleanrooms for PDS sensor protection  &  filter $<\SI{400}{nm}$ for $>$ 2 wk exp; $<\SI{520}{nm}$ all else &  Prevent damage to PD coatings  &  Visual or spectrographic inspection \\ \colhline
     % 6


\label{tab:specs:just:SP-INST}
\end{longtable}
\end{footnotesize}
\input{generated/req-longtable-SP-INST.tex}

Installation of the \dword{spmod} presents a multitude of hazards that include  manipulation of heavy loads in the tight spaces at the 4850 level and in the \dword{detmodule},  working at considerable heights above the floor, repeated utilization of large volumes of cryogens, multiple tests with \dword{hv}, commissioning of a class IV laser system, and deployment of a high-activity neutron source. Mitigation of these hazards begins with the strong professional on-site \dword{esh} teams of the \dword{sdsd} and \dword{surf}.
All installation team members, both at the surface and underground, will undergo rigorous formal safety training. Daily safety meetings will ensure that all workers are aware of the scope of the planned underground work and any related safety considerations. Any team member can stop work at any time for safety purposes. The overall \dword{dune} safety plan is described in   
 Volume~\volnumbertc~\voltitletc, Section~\ref{sec:far_site_safety} of this \dword{tdr}.  Individual sections within this chapter provide details on the evolving safety plan for installation. This plan has been informed by the successful safety experience of \dword{surf} with other underground experiments (e.g., \dword{lux}, \dword{mjdemo}, \dword{lz}), \dword{dune} members in executing projects at other underground locations (e.g., \dword{minos} at Soudan, Minnesota, USA), at other locations remote from major international laboratories (e.g., \dword{dayabay}, China and \dword{nova} Far Detector (Ash River, Minnesota, USA), and at the home laboratories of both \dword{fnal} and \dword{cern}).


% risk table values for subsystem SP-FD-INSTALL
\begin{footnotesize}
%\begin{longtable}{p{0.18\textwidth}p{0.20\textwidth}p{0.32\textwidth}p{0.02\textwidth}p{0.02\textwidth}p{0.02\textwidth}}
\begin{longtable}{x{0.18\textwidth}x{0.20\textwidth}x{0.32\textwidth}x{0.02\textwidth}x{0.02\textwidth}x{0.02\textwidth}} 
\caption[Risks for SP-FD-INSTALL]{Risks for SP-FD-INSTALL (P=probability, C=cost, S=schedule) More information at \dword{riskprob}. \fixmehl{ref \texttt{tab:risks:SP-FD-INSTALL}}} \\
\rowcolor{dunesky}
ID & Risk & Mitigation & P & C & S  \\  \colhline
RT-INSTALL-01 & Personnel injury & Follow established safety plans. & M & L & H \\  \colhline
RT-INSTALL-02 & Shipping delays & Plan one month buffer to store  materials locally. Provide logistics manual. & H & L & L \\  \colhline
RT-INSTALL-03 & Missing components cause delays & Use detailed inventory system to verify availability of  necessary components.  & H & L & L \\  \colhline
RT-INSTALL-04 & Import, export, visa issues  & Dedicated \dword{fnal} \dword{sdsd}division will expedite import/export and visa-related issues. & H & M & M \\  \colhline
RT-INSTALL-05 & Lack of available labor  & Hire early and use Ash River setup to train \dword{jpo} crew. & L & L & L \\  \colhline
RT-INSTALL-06 & Parts do not fit together & Generate \threed model, create interface drawings, and prototype detector assembly. & H & L & L \\  \colhline
RT-INSTALL-07 & Cryostat damage & Use cryostat false floor and temporary protection. & L & L & M \\  \colhline
RT-INSTALL-08 & Weather closes SURF & Plan for \dword{surf} weather closures & H & L & L \\  \colhline
RT-INSTALL-09 & Detector failure during \cooldown & Cold test individual components then cold test \dword{apa} assemblies immediately before installation. & L & H & H \\  \colhline

\label{tab:risks:SP-FD-INSTALL}
\end{longtable}
\end{footnotesize}

As part of the \dword{dune} design process the detector components and the \dword{tpc} have been prototyped at various stages. \dword{pdsp}, which was assembled from full-scale %dune 
components has recently been completed and has taken data. 
This process has been extremely important in planning the \dword{spmod} %far detector 
installation. A detailed list of lessons learned from \dword{pdsp} construction and installation was compiled\cite{bib:docdb8255}. 
These lessons %learned 
and other experience from the team planning the installation was used to develop a list of \textquotedblleft risks \textquotedblright for the %\dword{dune} 
\dword{spmod} installation and to formulate mitigation strategies to reduce the risks.
The highest-impact risks -- those requiring a mitigation strategy -- are listed in Table~\ref{tab:risks:SP-FD-INSTALL}. 
These mitigation strategies and all the lessons learned from \dword{pdsp} will be factored into the detailed installation plan. A description of each of the high level risks follows.

Personal Injury:
The installation of the detector requires on the order of fifty of man-years of effort. Substantial work at heights, rigging of heavy equipment, use or custom tooling, and some work in confined space is necessary. It is critical that all safety measures are implemented and proper oversight is in place. The safety plan is to follow the FNAL safety program, and if any additional measures are needed to comply with the SURF program they will be adopted. However even with and excellent safety program there remains a residual risk that someone will be injured. Given the large number of hours the risk of injury remains significant for a project of this scale and needs to be accounted for in the project risk evaluation.

Shipping Delays:
Delays in shipping and availability of components was a serious problem at ProtoDUNE. The delays were such that the installation plan was driven by the availability of parts and not by technical limitations. To avoid this in DUNE a one month buffer of equipment is required form the consortia. The one month was determined by the maximum delay in customs from a shipment for ProtoDUNE which was 3 weeks. In addition a detail shipping manual will be prepared to provide guidance to collaborators and the LBNF/DUNE logistics manager will be available to provide direct assistance. The residual risk that components are delayed is still considered high but the total schedule impact is expected to be on the few week scale.

Missing components cause delay:
Often in ProtoDUNE parts would arrive at CERN but small pieces would be missing like brackets or hardware. For DUNE detailed interface drawings will be generated the define the interface clearly and the work packages will clearly list all parts. A part breakdown structure will be defined so the ownership if each part is clear and the location of all hardware can be tracked. With these systems in place it is expected that the number of events where equipment arrives but not all the pieces are available when the installation is being executes will be minimized. The residual risk is considered  highly likely but with minimal impact.

Import export and VISA issues:
FNAL has established a new South Dakota Services to expedite customs and VISA issues. This risk will need re-evaluated after the new division has had time to evaluate the issues.

Lack of available labor:
The unemployment in the Lead area remains low. At the time work is ramping up it may be difficult to find local people with the requisite skills. To mitigate the risk the plan is to hire the core team early and assume they will need trained at Ash River. This will enlarge the possible pool of candidates by relaxing the requirements and allow a longer hiring period if needed. The residual risk is considered low.

Parts do not fit together:
Integration is a critical component of any complex project. DUNE has implemented a process to generate a complete 3-D model of the detector which can be used to detect conflicts. Interface drawings are being generated to clearly define the interfaces between components. Beyond this an installation prototype of the full assembly is being planed at the NOvA far detector building in Ash River. This installation prototype will test the installation of the detector components using full scale mechanical mockups. For PrototDUNE the Ash River test was critical in finding mis-matches between components and identifying installation issues where it was physically difficult to perform an operation due to limited space. After all the installation steps have been tested it is expected the residual risk will be low. It is highly likely that some small conflict will be found but the impact of the overall schedule will be low.

Cryostat Damage:
The cryostat membrane is a 1.2mm thick stainless steel membrane. A screw driver dropped from 12 m will likely cause damage and much larger pieces of equipment will be used. To protect the cryostat a fall floor will be constructed which is also used to move equipment. When the false floor is removed measures will be taken to prevent items from dropping on the membrane. Where possible all bolts, brackets, and components will be attached to nearby structures so they cannot be dropped. Teh residual risk of damaging the cryostat is considered small but in the unlikely event it occurs it would have a moderate schedule impact.

Weather closes SURF: 
Weather events in South Dakota occur several times each winter. This leads to closing of the facility for typically 1-2 days. This risk is accepted and the average number of snow days is added to the project schedule.

Detector failure during cool down: 
As the detector cools thermal stresses will develop that if the design is not sufficiently robust could cause failure. This risk is particularly critical as it occurs at the end of the project after all components are installed. This risk is minimized by thoroughly tested each component in the cold. Additional the APA-PD-CE assembly is also cold tested just prior to moving into the cryostat. This test of the final assembled components is considered critical in reducing the risk of failure during cool down. The residual risk is classed with low probability but would have a high impact if it should occur.

The remainder of this chapter is divided into three main sections. 
The first section describes how material will be delivered to the South Dakota region and forwarded to the Ross Headframe on the \dword{surf} site. 
The second section describes the infrastructure needed to install and operate the detector. This includes a cleanroom and its contents, and also electronics racks, cable trays, storage facilities and machining facilities. 
The third section describes the installation process itself, which is divided into three phases: the \dword{cuc} setup phase, the installation setup phase, and the detector installation phase. These are summarized in Section~\ref{sec:sp-iic-sched}.
