\section{Schedule}
\label{sec:sp-inst-sched}

The detector installation planning hinges on the date that the \dword{jpo} is permitted to begin work underground. According to the \dword{dune} \dword{cf} schedule, the \dword{jpo} receives the \dword{aup} for  the north cavern and \dword{cuc} in \cucbenocc{}.  The  \dword{sdwf} will be in place approximately six months before the warm structure installation begins, i.e., in spring 2022. Building the schedule for the \dword{detmodule} \#1 installation after \dword{aup} is  complicated and depends on many entities including \dword{cf}, \dword{lbnf}, and \dword{sdsta}.  The maximum number of people allowed underground is 144, which 
is based on several factors, including (1) time to reach the underground refuge, (2) capacity of the underground refuge, and (3) the time needed to evacuate all the underground personnel using the slowest path. At present the limiting factor is the time to evacuate using  the Yates cage. 
%is the number of people that can be evacuated in one hour.  
This number places a hard bound on how much work can be performed underground at any time and is particularly critical during the excavation of the third cavern when \dword{cf} is still active. Figure~\ref{fig:Overview-of-SinglePhase-Schedule} shows the main activities for the \dword{detmodule} \#1 installation and  the high-level milestones are shown in Table~\ref{tab:sp-iic-sched}.

The cost, schedule, and labor estimates are based on two \SI{10}{hour} shifts per day, four days a week (Monday through Thursday). Work efficiency should be a maximum of \SI{70}{\%}.  The cage ride, shift meetings, lunch, coffee breaks, and cleanroom gowning takes %up approximately 2-3 
up to three hours per day. Some low level of effort is planned on Friday, Saturday, and Sunday to monitor the \coldbox{}es and take data. 

\fixme{from TC vol}

\begin{dunetable}
[\dshort{dune} schedule milestones]
{p{0.75\textwidth}p{0.18\textwidth}}
{tab:DUNE_schedule}
{\dword{dune} schedule milestones for first two far detector modules. Key DUNE dates and milestones, defined for planning purposes in this TDR, are shown in orange.  Dates will be finalized following establishment of the international project baseline.}
Milestone & Date   \\ \toprowrule
Final design reviews  & 2020 \\ \colhline
Start of APA production & August 2020 \\ \colhline
Start photosensor procurement & July 2021 \\ \colhline
Start TPC electronics procurement  & December 2021 \\ \colhline
Production readiness reviews  &  2022    \\ \colhline
\rowcolor{dunepeach} South Dakota Logistics Warehouse available& \sdlwavailable      \\ \colhline
Start of ASIC/FEMB production   & May 2022   \\ \colhline
Start of DAQ server procurement &September 2022    \\ \colhline
\rowcolor{dunepeach} Beneficial occupancy of cavern 1 and \dshort{cuc}& \cucbenocc      \\ \colhline
Finish assembly of initial PD modules (80)      &March 2023    \\ \colhline
\rowcolor{dunepeach} \dshort{cuc} \dshort{daq} room available& \accesscuccountrm      \\ \colhline
Start of DAQ installation&      May 2023   \\ \colhline
Start of FC production for \dshort{detmodule} \#1       &September 2023   \\ \colhline
Start of CPA production for \dshort{detmodule} \#1&     December 2023   \\ \colhline
\rowcolor{dunepeach} Top of \dshort{detmodule} \#1 cryostat accessible& \accesstopfirstcryo      \\ \colhline
Start TPC electronics installation on top of \dshort{detmodule} \#1     & April 2024   \\ \colhline
Start FEMB installation on APAs for \dshort{detmodule} \#1 &    August 2024    \\ \colhline
\rowcolor{dunepeach}Start of \dshort{detmodule} \#1 \dshort{tpc} installation& \startfirsttpcinstall      \\ \colhline
\rowcolor{dunepeach} Top of \dshort{detmodule} \#2 cryostat accessible& \accesstopsecondcryo      \\ \colhline %1/25
Complete FEMB installation on APAs for \dshort{detmodule} \#1   &March 2025    \\ \colhline
End DAQ installation    &May 2025    \\ \colhline
\rowcolor{dunepeach} End of \dshort{detmodule} \#1 \dshort{tpc} installation& \firsttpcinstallend      \\ \colhline %8/24
\rowcolor{dunepeach}Start of \dshort{detmodule} \#2 \dshort{tpc} installation& \startsecondtpcinstall      \\ \colhline
End of FC production for \dshort{detmodule} \#1 &January 2026     \\ \colhline
End of APA production for \dshort{detmodule} \#1        &April 2026    \\ \colhline
\rowcolor{dunepeach} End \dshort{detmodule} \#2 \dshort{tpc} installation& \secondtpcinstallend      \\  \colhline
\rowcolor{dunepeach}Start detector module \#1 operations & July 2026 \\
\end{dunetable}
\fixme{end from TC volume- these match what I find in SP APA, PD, etc}


\begin{dunetable}
[\dshort{sp} Installation, Integration, and Commissioning Milestones]
{p{0.65\textwidth}p{0.25\textwidth}}
{tab:sp-iic-sched}
{\dword{sp} Installation, Integration, and Commissioning Schedule}   
Milestone & Date (Month YYYY)   \\ \toprowrule

Ash River phase 0 complete &  March 2020    \\ \colhline
\rowcolor{dunepeach} Start of \dword{pdsp}-II installation& \startpduneiispinstall      \\ \colhline
Ash River phase 1 complete & June 2021     \\ \colhline
 \dword{cuc} \dword{prr} & July 2021     \\ \colhline
Installation Preliminary Design Review & August 2021     \\ \colhline

\rowcolor{dunepeach} Start of \dword{pddp}-II installation& \startpduneiidpinstall      \\ \colhline
\rowcolor{dunepeach}South Dakota Logistics Warehouse available& \sdlwavailable      \\ \colhline
Begin procurement of \dword{cuc} equipment  &   April 2022   \\ \colhline

Ash River phase 2 complete &  July 2022    \\ \colhline
Installation \dword{prr} &  August 2022    \\ \colhline
Start production of installation infrastructure Detector \#1 & August 2022     \\ \colhline
Installation Final Design Review  & September 2022     \\ \colhline
\rowcolor{dunepeach}Beneficial occupancy of cavern 1 and \dword{cuc}& \cucbenocc      \\ \colhline
Start construction warm structure cryostat \#1   & October 2022     \\ \colhline
Start outfitting of \dword{cuc}  &  October 2022    \\ \colhline

\rowcolor{dunepeach} \dword{cuc} counting room accessible& \accesscuccountrm      \\ \colhline
Start installation of  cold structure Cryostat\#1 &  August 2023    \\ \colhline
Start installing Detector\#1 infrastructure  &  August 2023    \\ \colhline
\rowcolor{dunepeach}Top of \dword{detmodule} \#1 cryostat accessible& \accesstopfirstcryo      \\ \colhline
Begin Detector\#1 installation   &   June 2024   \\ \colhline

\rowcolor{dunepeach}Start of \dword{detmodule} \#1 TPC installation& \startfirsttpcinstall      \\ \colhline
\rowcolor{dunepeach}Top of \dword{detmodule} \#2 accessible& \accesstopsecondcryo      \\ \colhline
\rowcolor{dunepeach}End of \dword{detmodule} \#1 TPC installation& \firsttpcinstallend      \\ \colhline
TCO detector \#1 closed  &  July 2025    \\ \colhline
Start of cryogenic operation for detector \#1  &  August 2025    \\ \colhline
 \rowcolor{dunepeach}Start of \dword{detmodule} \#2 TPC installation& \startsecondtpcinstall      \\ \colhline
\rowcolor{dunepeach}End of \dword{detmodule} \#2 TPC installation& \secondtpcinstallend      \\ \colhline
Start of  detector\#1 commissioning  & January 2027     \\ \colhline                        \\
\end{dunetable}


\begin{dunefigure}[Overview of the single-phase schedule]
{fig:Overview-of-SinglePhase-Schedule}
{Schedule Overview of the \dword{sp} \dword{detmodule} \#1}                
\includegraphics[angle=90, width=0.55\textwidth]
{Overview-of-SinglePhase-Schedule}
\end{dunefigure}

We have defined three schedule phases for installation of the first \dword{spmod}:

\begin{itemize}

   \item {\bf \dword{cuc} Installation Phase:}
    This period,described in detail in Section~\ref{sec:fdsp-tc-inst-CUC}, starts once \dword{aup} has been received for the north cavern and the \dword{cuc}. This is the same time that excavation of the south cavern and installation of the warm structure by \dword{lbnf} begins. Since the \dwords{fte} underground is limited to 144 at a time, access will be minimal for \dword{dune} personnel and their work will only take place inside the \dword{cuc} and surface dataroom. Installing the basic rack infrastructure in the dataroom will take an estimated three months. Installation and testing of the \dword{daq} (required at the start of detector installation) will continue over the next 12 month period. Ten \dword{daq} workers are planned on each shift in this period.
     
 
   
    \item {\bf Installation Setup Phase:}  Described in detail in Section~\ref{sec:fdsp-tc-inst-setup}, this period includes installation of  the majority of the infrastructure. The setup phase is a critical training period, so getting lead workers, riggers, and equipment operators familiar with the tasks is a priority, as is  adjusting the crews to ensure balanced teams. The training process will have begun already at \dword{ashriver}. 
    There are many parallel underground activities planned in this phase making it a difficult phase to schedule, and frequent schedule adjustments may be required. Immediately after the cryostat warm structure is complete the north-south bridge is constructed. Following this the bridge crane under the bridge can be installed. A few months after the cryostat warm structure is complete the \dword{cf} work will also complete. Eighty of the 144 underground workers will become available to the \dword{jpo}, and the \dword{uit} team doubles in size. The \dword{jpo} will start two \SI{10}{hour} shifts per day.  Due to space constraints, peripheral work only on the cleanroom structure and assembly towers can begin. Once the  cryostat cold structure is approximately six months into its installation schedule, most of the foam will have been installed and floor space becomes available in the north cavern. The \coldbox construction must begin immediately at this point because the welding takes approximately six months. In parallel, the machine shop area can be set up. As the membrane installation nears completion, the walls of the cleanroom can be installed as can the remaining equipment. 
    
    Installation of the \dword{dss} could begin during the final installation stages of the cryostat cold structure because they both require full-height scaffolding for the welding on the top of the cryostat. The \dword{pdsp} \dword{dss} was installed this way. This requires a crew on top of the cryostat installing the \dword{dss} support feedthroughs from the top, as shown in Figure~\ref{fig:install-dss-feedthru}.  The details have not yet been worked out with the contractor, and work may be done in stages.

   
    \item {\bf Detector Installation Phase:} The final detector installation phase begins with an \dword{orr} to check that all documentation and procedures are in place. After the east \dwords{ewfc} are installed, a start-up period of 1.5 months begins for the first two rows of \dword{tpc} components.  To meet this schedule, three assembly lines, three \coldbox{}es, and separate crews in the cryostat, all working in parallel, are needed.  It will take 5.5 months to install rows 3 through 24 and about one month for row 25. Closing the \dword{tco} will take approximately two months for the cryostat cold structure contractor; during this time, there is no access to the cryostat.  Once this is completed, we can complete the final instrumentation, and the purge can begin. During this period, up to 50 people will be working in the cleanroom and cryostat.
    
\end{itemize}

The total time to install the detector including the time for the setup phase is two years. Coincidentally, this was roughly the time needed to install \dword{minos} and \dword{nova}.

  