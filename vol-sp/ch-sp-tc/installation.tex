\section{Detector Installation}
\label{sec:fdsp-tc-inst}

\subsection{Introduction}
\label{sec:fdsp-tc-inst-intro}

The DUNE detector installation will proceed in three phases the CUC setup phase, the installation setup phase, and the detector installation phase. Figure \ref{fig:high-level-schedule} 
shows the major underground activities and gives an idea of what work will be occurring in which phase. The CUC setup phase is the first step in the installation process. The start of the CUC setup phase begins when the underground area for the North Cavern and Central Cavern become available to LBNF and DUNE. At this time the cryostat construction can begin in The North Cavern and DUNE equipment installation can begin in the central cavern referred to as the Central Utility Cavern (CUC). A top view of the underground areas is seen in Figure~\ref{fig:cavern-layout}. The main equipment from DUNE which is installed in this phase is the infrastructure in the DUNE dataroom. The dataroom is found in Figure~\ref{fig:cavern-layout} in the CUC on the West end of the excavation. The detector installation setup phase (referred to as Infrastructure Det\#1 in Figure \ref{fig:cavern-layout}) begins during the cryostat membrane installation period. In this phase the equipment needed to perform the detector installation will be erected in North Cavern. This includes the installation of the bridge across the cavern, the installation cleanroom, lifting equipment and work platforms, the cold boxes and cryogenic system for testing APA, and the DSS with related switchyard. In the third phase of the installation the detector itself will be installed. The work in each phase will be described the following sections.

\begin{dunefigure}[High level installation  schedule]{fig:high-level-schedule}
  {Overview schedule showing the main activities underground.}
\includegraphics[width=.99\textwidth]{high-level-schedule}
\end{dunefigure}

\begin{dunefigure}[Layout of the DUNE underground areas]{fig:cavern-layout}
  {Top view of the layout at the 4850 level at SURF. Shown are the three large excavations and the location of detectors in excavation \#1 and \#3. Excavation \#2 is the CUC which houses the DUNE dataroom and the underground utilities.}
\includegraphics[width=.9\textwidth]{cavern-layout}
\end{dunefigure}



%%%%%%%%%%%%%%%%%%%%%%%%%%%%
\subsection{Installation Process Description}
\label{sec:fdsp-tc-inst-proc}

\subsubsection{CUC Installation Phase}

\begin{dunefigure}[Layout of the DUNE dataroom and experimental workarea in CUC]{fig:install-cuc}
  {TOP: The overall layout of the DUNE spaces in the CUC is shown. The inner room is the DUNE dataroom which houses the underground computing and the outer area referred to as the experimental workarea is a general purpose workarea. Bottom: The first row of ten racks in the data room is shown. The first two racks are the  CF interface racks. The image was taken from the ARUP design drawing U1-FD-T-701}
\includegraphics[width=.7\textwidth]{cuc-layout}
\includegraphics[width=.9\textwidth]{cuc-cf-racks}
\fixme{need to use a figure from the 90\% submittal}
\end{dunefigure}


The first stage of the CF work ends when the outfitting of the North and Central caverns is complete. At this time the CUC is ready for DUNE outfitting and the cryostat installation can start for detector \#1. At this time DUNE will not have assess to the detector excavations as the heavy steel work for the cryostat will be ongoing. The only work planned for DUNE in this period is the outfitting of the dataroom and work area room in the CUC shown in Figure~\ref{fig:install-cuc}.  CF is providing redundant single mode fiber up the shafts to provide external connectivity.  CF also will provide the empty dataroom with an 18 \si{in} false floor, a 500 \si{kVA} power disconnect, and connections for chilled water sufficient to cool the racks. The dataroom like the adjacent CF electronics room will be outfitted with a dry fire extinguishing system. 
\fixme{When are fibers available?}

The water cooled racks, cable trays, power distribution, and water distribution are the responsibility of DUNE and will be installed once the space becomes available, as will the installation of the DAQ fiber trunk between the detector cavern and the CUC dataroom. The installation of the racks needs to be coordinated with CF as the first two racks are for CF usage and need to be in place before the underground phase one work is complete. Some small overlap will be needed between CF and DUNE at this time. The general purpose network will be installed by FNAL's SCD and connected to the shaft fiber. This is required for most further work in the underground area.

Data from the detector electronics will be transmitted over a multimode fiber trunk from the warm interface boards on top the detector to the DAQ dataroom in the CUC shown in Fig.~\ref{fig:install-cuc}.  This room will contain 60 water cooled racks: two of which are reserved for CF use, two for CISC servers, and the rest for DAQ servers and networking. Although this is the total number of racks needed for all four detector modules, the racks themselves will be installed at the beginning of the CUC commissioning phase, as they must be plumbed into the cooling water below the data room's drop floor and wired into power distribution from the ceiling.  DAQ equipment will populate the racks as needed for servicing the detector commissioning.  For the first detector module, details of this configuration will be informed by knowledge gained from DAQ vertical slice tests done at other institutions.  

At the same time the eight above ground DAQ racks which receive the data from the underground dataroom and then transmit the data to FNAL will also be installed, connected to the WAN, and connected to the single mode fiber in the shafts. With this infrastructure in place the DAQ group can begin constructing and testing the final DUNE DAQ.  The timing system is the first DAQ component needed.  Enough DAQ back-end servers to support the first APAs will be operational before the APAs are installed.  The remainder of the DAQ will grow in parallel with the APA installation.

The underground experimental work area is a general purpose area that will need to serve many purposes during the DUNE installation. Initially the area will be outfitted with office equipment for the installation team, workstations for DAQ, and a basic conference area for meetings.  The room is 4-6.4 \si{m} deep and 20 \si{m} long so it can serve several functions.

During this early installation stage the machine shop and DUNE storage area will also be setup in the detector excavation area. It is expected these facilities will be shared with the cryostat team. 

\subsubsection{Installation Setup Phase}

Once the steel structure of the cryostat is complete the remaining work by the cryostat team will be focused inside the cryostat. 
There will still be a lot of activity outside the cryostat as the 4,000 crates of foam and other materials are transported inside, but the outer structure will be in position so some DUNE work can start. 
The first piece of equipment to installed will be the bridge between the North and South drifts. 
This will allow the cryogenics equipment to travel from the North drift to the CUC and will provide part of the structure for the cleanroom. 
Construction of the cleanroom frame and related hoisting equipment can then begin. 

The largest most complex equipment that must be constructed in this phase are the cold boxes and related cryogenics system. 
Due to the size of the cold boxes these must be constructed in place. 
The layout of the installation region outside the cryostat is shown in Figure \ref{fig:install-cleanroom-layout}. 
This figure shows a cut below the bridge so the majority of the installation equipment can be seen. 
The three 15 \si{m}tall cold boxes are seen in blue next to the yellow access stairway. 

The hoisting equipment and the rail system needed to move the detector components in the installation area interfaces with the bridge, cryostat and possibly the cleanroom mechanical structure. 
Installing this early will make transporting equipment around the area outside the cryostat significantly simpler. 
 
The assembly towers for the APA assembly, APA cabling, and CPA assembly can be installed when it is most convent for the cryostat installation crew. 
Just prior to the start of detector installation the cleanroom walls will be installed, the area will be cleaned and added filters will be installed to convert the area to an ISO-8 cleanroom.

During this phase of the cryostat installation the majority of the cryostat work will be inside an at the 4910 level. 
This will allow significant work to be performed on the cryostat roof for the cryogenic system installation and the detector installation setup. 

\begin{dunefigure}[Installation of electronics crosses]{fig:install-elect-cross}
  {Installation of the crosses onto which the cold electronics warm readout and the photon detector cables are connected.}
 \includegraphics[width=.75\textwidth]{install-elect-cross}
\end{dunefigure}

The cryostat crossing tubes will be installed on the roof as required by the cryostat assembly sequence. 
These assemblies are welded to the 1 \si{cm} thick steel cryostat roof and are then additionally cross braces to the large I-Beams. 
The thick walled tubes which penetrate through the foam insulation are in place at this time. 
Once the crossing tubes are in place the large tees for the cold electronics can be installed. 
The next step in the installation requires the internal roof of the cryostat to be complete. 



Once the crossing tubes are installed and leak chased the crosses which onto which the feedthru flanges for the cold electronics and photon detector mount can also be connected. The setup is seen in Figure \ref{fig:install-elect-cross}. 
The height of the crosses is selected so a person can comfortably work on the WEC and PD flanges while standing on the cryostat roof. 
A fully assembled cross is shown on the right and a cross where the WEC are extracted to the assembly position is shown on the left. 
The present plan is to install the crosses shortly after the cryostat crossing tubes are installed. 
By doing this the large openings in the cryostat roof are sealed and cleanroom conditions can easily be achieved after the cryostat is cleaned. 
For this stage temporary seals will be used for the flanges as they will need removed during the cabling process later in the installation. At this time the cold electronics mechanical feedthrus can also be installed. 

\begin{dunefigure}[DSS feedthru installation]{fig:install-dss-feedthru}
  {The DSS support feedthru are installed using a gantry crane running on the roof of the cryostat.}
 \includegraphics[width=.95\textwidth]{graphics/dss-feedthru-install.pdf}
\end{dunefigure}

When the crossing tube installation is complete and in parallel to the installation of the cold electronics tees the DSS support feedthrus can be installed. A gantry crane on top of the cryostat is used to pick up the feedthru and feed them into the cryostat crossing tubes as shown in Figure \ref{fig:install-dss-feedthru}.
This is the first step in the \dword{dss} installation.
There are \num{20} \fdth{}s per row and five rows for a total of \num{100} \fdth{}s.  
A fixture with a tooling ball is attached to the
clevis of each \fdth.  
The $xy$ position in the horizontal plane
and the vertical $z$ position of this tooling ball is defined, then a  survey is performed to determine the location of each tooling ball center and $xy$ and $z$ adjustments are made to get the tooling
ball centers to within $\pm$\SI{3}{mm}.  
The \SI{6.4}{m} long I-beams are then raised and pinned to the clevis.  Each beam weighs roughly \SI{160}{kg} (\SI{350}{lbs}).
A lifting tripod is placed over each of the \fdth{}s supporting a beam, and a \SI{0.64}{cm} \SI{0.25}{in}  %$1/4 ^{''}$  
cable is fed through the top
flange of the \fdth down the \SI{14}{m} to the cryostat floor where it
is attached to the I-beam. The cable access port and lifting cable are shown in Figure \ref{fig:dss-beam-lifting} 
The winches on each tripod raise the beam in unison in order to get it to the correct height to be pinned to the \fdth clevis.  
Once the beams are mounted, a final survey of the beams takes place to ensure they are properly located and aligned to each other.

 \begin{dunefigure}[DSS I-Beam lifting setup]{fig:dss-beam-lifting}
  {A cable access port is included in the DSS flange. This is used to feed a cable from the roof through the flange and attacht it to the I-Beams during DSS installation.}
 \includegraphics[width=.95\textwidth]{graphics/dss-beam-lifting.pdf}
\end{dunefigure}

Once work on the feedthru are complete then the mezzanines for the cryogenic system and the detector electronics racks can be installed. Finally the cable trays,  piping, lighting, and cryostat roof flooring are installed. 
At this time the cryostat roof is ready for the start of the DAQ installation in the detector area which will proceed in parallel to the detector installation.

The last steps in the installation setup phase is the installation of the cryostat internal piping, cleaning the cryostat, installing the false floor, and filtered lighting to protect the photon detectors. 


\subsubsection{Detector Installation Phase}

\begin{dunefigure}[Top view of the installation region inside the cleanroom ]{fig:install-cleanroom-layout}
  {Top view of the cleanroom used for installation. In this view cleanroom roof and bridge are not shown. The equipment used for the installation is shown along with the material airlock layout and the location of the changing room. The cryostat is to the right of the figure and the I-Beams passing through the TCO opening are shown.}
 \includegraphics[width=\textwidth]{install-cleanroom-layout}
\end{dunefigure}

At the start of the detector installation phase the work on the cleanroom is complete, the DSS is installed in the cryostat including the switchyard near the TCO. 
The light in the cryostat and cleanroom will be filtered to protect the photon detectors and the air will be filled to reduce the dust collected during installation. 
The cryogenic piping will be in place along with the false floor. 
The false floor both gives a flat work area and protects the 1.2 \si{mm} thick stainless steel cryostat membrane. 
          
The layout of the cleanroom during the detector installation phase is shown in Figure \ref{fig:install-cleanroom-layout}. This image is a top view of the cleanroom with the roof and bridge removed; the cryostat is on the right and the open cavern on the left. In the North-East corner of the figure the access stairway is shown in yellow. This stairways is outside the cleanroom and allows people to access the 4910 level from the North drift or from the cryostat roof. An access corridor along the north wall connects the stairway to the rest of the cavern. The changing room and airlock are found on the West of the figure and are outlined in orange. The changing room is located in the the North-West corner and provided access to both the cleanroom and the airlock. South of the changing room is the material airlock. This rather large area is where the APA modules are connected together to form the 12 \si{m} tall units. The assembly station is shown in green along the south wall of the airlock. This area is also where all the materials are brought from the dirty outside cavern and are cleaned or the outer packaging removed. Roll-up doors along the West wall allow material from the outer cavern to be brought into the airlock. A section of the roof can be removed to allow the cavern bridge crane to manipulate the APA transport box and modules. The cleanroom area itself is shown on the right of the figure outlined in magenta. Inside the cleanroom is a switchyard material transport system (shown in red) similar to what is used for the DSS inside the cryostat. The switchyard is used to move the assembled detector components north-south in the cleanroom. East-west rail sections are then used to deliver the components to the required work location. In the north area of the cleanroom are the three cold boxes used to cryogenically test the fully assembled and cables APA pairs. In the south half of the cleanroom is the APA cabling tower used for cabling and testing the APA pairs. Along the South wall is the CPA assembly fixture for assembling the CPA. 

The labor effort for the Detector Installation phase has two main components: 
The UIT (Underground Installation Team) that includes a scientific lead, manager, two deputy managers (one on each shift), safety office and administrative help. They are responsible for the communication with the ITF and Logistics Facility to insure needed components are shipped underground and properly keep in the inventory system.  They also organized/plan the day to day work done underground with the lead workers and rigging teams as well as the different consortia.  Several teams of typically of 3 FTE consisting of a lead worker and riggers that are responsible for moving all the TPC components into the SAS and cleanroom, the cold box and into the cryostat. An additional team lead worker/rigger/technician work in the cryostat positioning the TPC components and deploying the Field Cages during the final stage of each new drift volume. 

The second labor effort comes from the different consortia with specific tasks related to each group. These labor estimates will be refined during the DUNE-Ash River Trial Assembly work as we do time and motion studies. Figure \ref{fig:Single-APA-Schedule} below shows the typical labor effort in the SAS, cleanroom and cryostat. 

\begin{dunefigure}[Typical APA Installation Schedule]
{fig:Single-APA-Schedule}
{Typical APA Schedule for SP Detector}
\includegraphics[width=0.95\textwidth]{Single-APA-Schedule.pdf}
\end{dunefigure}

\begin{dunefigure}[Design of the instrumentation feedthroughs]{fig:CISC-feedthru}
{The signal feedthru are integrated with the DSS support feedtrhrus. A side ports on a short spool piece in the DSS support structure allows the instrumentation cables to be fed through the cryoatat walls where needed.}
\includegraphics[width=0.95\textwidth]{CISC-feedthru.pdf}
\end{dunefigure}



The first detector equipment to be installed are the CISC thermometers,\fixme{SG: I think "CISC thermometers" is confusing since CISC has several types of thermometers. I assume here you mean static T-gradient thermometers? If so, clarify that.} one array of purity monitors, capacitive liquid level meters,\fixme{SG: I thought the capacitive level meters came after the TPC is installed!?} calibration equipment and cameras at the east end of the cryostat. 
This equipment will be used to monitor the cool down, filling and commissioning of the detector. As these components are small the installation work can be performed using a scissor lift with 12 \si{m} reach. 
At present this is the tallest battery operated (thus cleanroom compatible) scissor lift rated for use in the US that we have identified. The signals exit the cryostat using electrical feedthroughs are distributed across the cryostat roof and are integrated with the DSS support structure as shown in Figure \ref{fig:CISC-feedthru}

Static T-gradient monitors must be installed before the outer APAs so it is planned to install them during the early installation period. The thermometer system will be supported using the bolts at the top and bottom of the cryostat. To avoid any damage, the sensors will be plugged into the IDC-4 connectors later, just before moving the corresponding APA into its final position. Individual sensors on pipes and cryostat floor are installed immediately after installing the static T-gradient monitors. Cables with their corresponding supports are installed first and sensors are installed later, just before unfolding the bottom GPs, to avoid any damage. Individual sensors on the top GP must be integrated with the GPs. For each CPA (with its corresponding four GP modules) going inside the cryostat, cable and sensor supports will be anchored to the GP threaded rods. Once the CPA is moved into its final position and its top GPs are ready to be unfolded, sensors on these GPs are installed.

Installing fixed cameras is, in principle, simple but involves a considerable number of interfaces. The enclosure of each camera has exterior threaded holes to facilitate mounting on the cryostat wall, cryogenic internal piping, or DSS. Each enclosure is attached to a gas line to maintain appropriate underpressure in the fill gas, therefore an interface with cryogenic internal piping will be necessary. Camera cables will be run through cable trays to flanges on assigned instrumentation feedthroughs. 

A summary of all the cryogenic instrumentation provided by the CISC group is shown in Figure \ref{fig:cisc_devices}. 

Quartz optical fibers required for the \dword{pd} monitoring system will be pre-installed at this point as well.  Fibers will be run form the optical flange locations (which are still being finailzed) to locations on the \dword{cpa} support structures, to be connected to the diffusers mounted on the \dwords{cpa} during installation.

The residual gas analyzers which monitor the impurities in the gaseous argon system must be installed before the piston purge and gas recirculation phases of the cryostat commissioning. The exact timing of the gas analyzers will depend on the schedule of the mezzanine outfitting and installation of the gaseous argon purge piping. The instruments are installed near the tubing switch yard to minimize tubing run length and for convenience when switching the sampling points and gas analyzers. 

\begin{dunefigure}[Distribution of various instrumentation devices inside the cryostat]{fig:cisc_devices}
  {Distribution of various instrumentation devices inside the cryostat}
  \includegraphics[width=0.98\textwidth]{cisc_distribution.png}
  \includegraphics[width=0.85\textwidth]{cisc_distribution_legend.png}
\end{dunefigure}

\begin{dunefigure}[Endwall hoisting infrastructure]{fig:endwall-hoist}
  {Image showing the hoisting equipment used to lift the Endwall into position. In this image one of the Endwalls is in place and a second is being positioned.}
\includegraphics[width=.75\textwidth]{endwall-hoist}
\end{dunefigure}

The next step in detector installation is the field cage end-wall installation. 
The end-wall panels will be brought underground in custom crates which fit in the cage. 
Each of the 8 crates will hold 4 end-wall sub-panels and 8 sub-panels are needed to build one complete 12 \si{m} tall panel.  
The first step in the end-wall installation is to install a custom hoist to the end of the DSS beam. 
This will be used to lift and assemble the sub-panels in place. 
The end-wall transport crates will then be brought to the material airlock using a forklift where they are either un--crated or the crates are cleaned. 
Once clean the crates are moved into the cleanroom and placed next to the TCO. The a hoist running on the rails through the TCO is used to lift the endwalls onto the transport cart. The cart is then hoisted into the cryostat. 
Figure \ref{fig:endwall-cart} shows an Endwall panel being transferred over to the transport cart.
The top Endwall sub-panel is then attached to the installation hoist and lifted out of the cart. When the sub-panel is free of  the cart re-position so the second sub-panel can be attached to the first and the pair are then again lifted. This process is repeated until the full 12 \si{m} Endwall field cage panel is assembles and can be attached to the DSS. 
Figure \ref{fig:endwall-hoist} shows an Endwall panel being lifted into position.
At this time all the HV connections inside the panel can be tested. The process is repeated for the four Endwall panels making up the East Endwall. 
In parallel to the endwall installation the cleanroom is configured for the installation of the APA and Cathode plane systems.

\begin{dunefigure}[Installation of the first Endwall]{fig:endwall-cart}
  {The Endwalls are lifted out of the transport crates using the one of the hosts on the installation switchyard. Each panel is then placed on a custom cart which is then lifted into the cleanroom.}
\includegraphics[width=.85\textwidth]{endwall-cart}
\end{dunefigure}


The installation of the APA and Cathode with Top/Bottom Field cage modules is the most labor intensive period of the detector installation. If possible DUNE would like to finish installing one row of the APA, CPA, and FC modules every week. This represents the equipment shown in Figure \ref{fig:install-single-row}. To achieve this several separate teams will need to be working inside the cryostat positioning the equipment, connecting the cables, and deploying the field cages, while  other separate teams will be working in the outside cleanroom assembling the equipment and performing the final cold tests. The total time that would be needed to install a row of APA and CPA if all the work were to be done serially would be many weeks so performing the work as parallel  as possible will be very important for executing the installation efficiently. 


\begin{dunefigure}[Single row of APA and CPA]{fig:install-single-row}
{One row of the APA and CPA with associated field cages is shown. In this  image the field cages are deployed in the final orientation. The equipment in the figure represents $1/25$ of the total TPC.}
 \includegraphics[width=0.5\textwidth]{install-single-row}
\end{dunefigure}

When the APAs are installed the area outside the cleanroom in the North cavern is available for storage and and it has capacity to store one full month of equipment. 
Equipment will be brought into the cleanroom's materials airlock through a rollup door in the West wall using either an electric forklift or electric pallet jacks.  
The APAs enter the airlock in the transport crates which were used to bring the APAs down the Ross shaft and into the North cavern. 
Each transport box hold two APA with associated electronics and photon detectors and they enter the airlock in a horizontal or "landscape" orientation. 
Once inside the airlock the dirty outer panels of the transport box are removed, a section of the roof is opened and the bridge crane is used to lift the transport box into vertical. 
The airlock can then be cleaned and prepared for the APA assembly step where the upper and lower APA are assembled into a 12 \si{m} tall pair.
The APA assembly sequence is shown in Figure \ref{fig:apa-assembly-v3}.

After initial visual inspection the lower APA is lifted out of the transport box 
using the bridge crane and mounted on the APA assembly tower shown in green on the top row of images in in the Figure \ref{fig:apa-assembly-v3}. 
The  lower APA is supported from the bottom and guides connected to the sides of the APA provide mechanical stability while allowing the APA to be lifted using jacks integrated into the lower support. A complete test of all the lower APA Photon Detectors (PDs) is done.  Because the cable conduit is already installed if and PDs fail we must remove the APA from the tower, place it back into the shipping frame and rotate the frame into the landscape position.  This is the only way to remove the cable conduit. The APA would then be loaded into a process cart, the problem resolved, then the reverse process is done to get the APA back onto the Assembly Tower. 

Next the upper APA is lifted out of the transport box and the transport box is removed from the airlock. The same process of testing the PDs must be done with the upper APA and replaced if needed. The upper APA is connected to the transport rail above the APA assembly tower. At this point the upper APA is supported by the trolleys used for moving the APAs along the transport rails and it is stabilized using the APA assembly frame. 
The bridge crane is no longer needed after this step so airlock roof can be closed allowing the air to be purified. At this time there is a 300 \si{mm} gap between the upper and lower APA and now the photon cables between upper and lower APA can be connected and the connection from the top connectors to the SiPMs can be checked. 
In order to mechanically connect the two APA modules a metal linkage with electrical insulators is inserted into the upper APA and bolted into place. Then the lower APA is raised until the linkage can be bolted to the lower APA.  
The APA pair can then be released from the assembly tower supports and jacks and it is supported from the top APA to the transport rail system.
After the air quality in the airlock meets the requirements for ISO-8 any dust covers are removed and the APA can be moved into the cleanroom through a narrow vertical door in the cleanroom wall shown in brown in the top row of images in Figure \ref{fig:apa-assembly-v3}.


\begin{dunefigure}[APA assembly steps]{fig:apa-assembly-v3}
  {Top row from left:  \dword{apa} transport crate inside the airlock; \dword{apa} transport crate rotating to vertical position;  lower \dword{apa} placed on the \dword{apa} assembly frame; and upper \dword{apa} placed on the assembly frame. Bottom row right to left: \dword{apa} pair moving on the cleanroom transport rails; \dword{apa} in  position at the cabling tower; and the \dword{apa} being inserted into the coldbox.}
\includegraphics[width=.9\textwidth]{apa-assembly-v3}

\end{dunefigure}

The APA pair moves onto one of the shuttle beams in the cleanroom switchyard and can be moved north-south inside the cleanroom. The next assembly step is to install and test the electronics cabling. The APA is moved to line up with one of rails used to transport the APAs to the APA cabling tower. The lower left two image in Figure \ref{fig:apa-assembly-v3} shows the APA cabling tower in the cleanroom and an APA being moved into position. The APA cabling tower is capable of holding two APA pairs simultaneously allowing one APA pair to be cabled while a second APA undergoes electrical testing in parallel. In this position the cable trays and any other equipment needed during the cabling process can be installed. The electronics cables are delivered to the cleanroom on reels pre-bundled and tested. Images of the cable assemble are found in the CE section.ref{} A lower APA reel is craned up to the top of the APA cabling tower and the cable is then spooled over to a motorized deployment spool and the cable guide is attached.  The cable guide is then fed through a guide sieve and into the conduit on the side of the APA. The cable bundle is careful fed through the conduit and is anchored in place using a cryogenic compatible cable grip. The cable connected to the electronics at the bottom and is laid into the cable trays on top. This process is repeated for the second lower APA cable bundle. Finally the upper cables can be installed and prepared for transport. At this time a check of the functionality of all the electronics is performed. After the APA electrical test the APA pair is transported to a cold box where it undergoes a thermal cycle and complete system test. The APA being inserted into the coldbox is shown in the lower right image in Figure \ref{fig:apa-assembly-v3}. As the cold box is also a Faraday cage noise levels can be measure and the photon system checked for photon sensitivity. After the cold test is complete an APA will either move back to the cabling station if any repair is needed or it will be moved into the cryostat for installation. 



\begin{dunefigure}[CPA assembly steps]{fig:install-cpa-assembly}
  {The \dword{cpa} assembly steps are shown. Top row from left:  \dword{cpa} are delivered to the CPA assembly fixture in the cleanroom, the 3 \si{m} sub-panels are lifted onto the frame and connected. Bottom Row: After the CPA panel is complete it is moved into the room and the field cage modules are attached. The CPA is then moved into the cryostat.}
\includegraphics[width=.9\textwidth]{install-cpa-assemble}
\end{dunefigure}

The CPA and top field-cage modules are assembled in parallel to the APA assembly. Figure \ref{fig:install-cpa-assembly} shows the  assembly sequence. The sub-panels for the cathode plane are delivered to the airlock in crates which hold 4 \si{m} long 1.15 \si{m} segments. The crates after cleaning are brought into the cleanroom and opened. The panels inside are bagged to provide additional dust protection. The sub-panels are lifted out of the crate and place on the assembly frame using the cleanroom switchyard hoist. First one of the 1.15 \si{m} 4 
\si{m} tall sections are assembled and then the second and third ones. The 1.15 \si{m} wide section is then lifted connected to the installation switchyard and moved to the TCO beam. The second 12 \si{m} tall section is then assembled similar to the first. The two 1.15m wide panels are then connected together to make the 2.3 \si{m} wide unit.  A complete set of QC measurements are taken of all the electrical connections between panels.  The cathode assembly  is then moved to a location in the switchyard where the diffuser fibers and top field cage modules can be installed.
 The top field cage modules are now attached. In Figure \ref{fig:install-cpa-assembly} the completed assembly is shown with the lower field cage modules also attached. This is an option but present planning is to install the lower FC modules  inside the cryostat. Finally the CPA-FC assembly is moved into the cryostat.


Photon detector monitoring system optical diffusers and short optical fibers will need to be connecetd to the \dword{cpa} panels prior to installation in the cryostat.  Discussions are underway regarding the optimal site for this installation:  Either at the \dword{cpa} assembly facility prior to shipping to the site, or as part of the assembly of \dword{cpa} stacks in the underground cleanroom.  Whichever solution is adapted, quartz optical fibers will need to be routed from the diffuser to the top of the \dword{cpa} assembly, for later connection to the pre-installed fibers in the cryostat (which will occur upon final positioning of the CPA).  

Work inside the cryostat will proceed in parallel to the work in the outer cleanroom. 
The large detector components like APA pairs and CPA modules will enter the cryostat using the TCO rails which connect to the DSS switchyard.
Inside the cryostat the modules will be pushed onto one of the switchyard shuttle beams shown in  Figure \ref{fig:shuttle}. 
The DSS shuttle beam is then moved to the appropriate row of the DSS and then the module is pushed down the length of the cryostat into position. The position of the DSS beams are well defined and accurately surveyed so the APA and CPA modules can be accurately located by precisely positioning them along the DSS beams. A small correction in the height of the modules may be needed to accommodate deflections in the DSS due to load. Figure \ref{fig:install-ce-cables} shows the typical situation during the APA installation and cold electronics cabling. 

\begin{dunefigure}[Cold Electronics cabling inside the cryostat]{fig:install-ce-cables}
  {The installation of the APA and cabling of the cold electronics is described. In the left panel the situation is shown during the APA installation process. One row of APA and HV equipment is installed and a second APA is ready for the electrical cabling. In the top right image the cable trays are seen which will hold the CE cables and one of the workers in the scissor lift. In the left bottom image thework space is defined wiht the geometry of the APA the cryostat roof and the cable feedthru.}
\includegraphics[width=.9\textwidth]{install-ce-cables}
\end{dunefigure}

After the APA is moved into position the permanent support rod is connected to the DSS beam and the trolleys are removed. 
The crawler used to push the APA along the rails is then moved back through the shuttle area and can be used for the next module. 
After the APA is locked in position the Cable try to the CE feedthru in installed now CE cabling can start. 
Even if a CPA module is already in position there is over 3 \si{m} space free between the APA and CPA so a scissor lift can easily be positioned in front to the APA. 
The two right images in Figure \ref{fig:install-ce-cables} show the situation at the top of the cryostat at during the cabling period. 
The cables are not shown so one can see the cable trays and their support infrastructure. 
At the start of the cabling all the cables are in the cable trays. 
A team of two people are in the scissor lift in front of the APA and  and a team of two people are on top of the cryostat. 
The CE cables from the bottom APA emerge from the side APA side-tube and are split into two bundles in the cable tray for a total of 4 cable bundles. 
The top APA also has the CE cables organized into 4 bundles. 
The photon cables from both the top and bottom APA are bundled into two cable bundles.
During the cabling process each bundle is partially removed from the cable tray and then fed up through the cable feedthru. 
At the top of the feedthru the cables are strain relieved and then the cables are strain relieved again at the bottom of the crossing tube.
This is repeated for each of the 10 cable bundles needed for the APA pair. 
When all the cable are installed through the cable tray any excess length is returned to the cable tray at the top of the APA. 
On the roof the short individual cables are then connected to the feedthru flange and the electronics can be tested. 
When all the electronics and electrical connections are good the flange connecting the warm interface crate and can be sealed to the cryostat feedthru flange and the cable installation is complete. Similarly the photon detector warm cables are connected to the readout module and the flange sealed after testing.  Once the testing is completed the load of the excess cable tray and cable are transferred to the DSS beam.  This minimizes an uneven load on the APA pair so they hang more vertically.   
The electronics for each APA is continuously monitored after installation. 

The placement of the cables in the cable trays and exactly how the cables are routed is complex. The working 3-D model of the cable routing showing how the cables will be bundled and placed in the trays is shown in Figure \ref{fig:install-cable-routing}. It is planned to mock up the cabling configuration at BNL and to test the installation of the cables as part of the Ash River testing program.

\begin{dunefigure}[Model of the electronics and photon detector cabling]{fig:install-cable-routing}
  {Working model of the cable trays and routing of the cables in the trays.}
\includegraphics[width=.95\textwidth]{install-cable-routing}
\end{dunefigure}


\begin{dunefigure}[CPA installation]{fig:install-cpa-fieldcage}
  {The top field-cage assemblies are deployed using a custom tool that mounts to the DSS beams as seen in the top left panel. The field-cage is lifted using the electric winch controlled by an operator in the nearby scissor lift. The lower field-cage is lowered using a hoist mounted on a wheeled frame. The hoist is on a linear slide to keep it aligned above the connection point.}
\includegraphics[width=.9\textwidth]{install-cpa-fieldcage}
\end{dunefigure}

The cathode field-cage assemblies are brought into the cryostat like the APA pairs using the overhead rails through the TCO. Inside the cryostat they move into position using the DSS switchyard and DSS I-Beams. Once in position the load is transferred to the DSS beam and the trolleys are removed. The CPA will wait in position until it's APA pairs are fully tested and then the field cage modules can be deployed. The equipment for deploying the field cages is shown in Figure \ref{fig:install-cpa-fieldcage}. The top field cages are raised by connecting a cable to the module and then a pulley-winch assembly is used to lift the module which latches to the APA mounts. A scissor lift is used to connect the cable to the module and also to control the winch. After the module is in place the deployment tool is moved to the next set of APA and CPA. The lower field cage is deployed using a custom frame that can be wheeled into position. The cable from a small host  is then attached to the field cage module and the module can be lowered. The hoist it on a linear slide so the cable is always directly over the connection point. This keeps the CPA from swinging due to an induced moment. When the module is down it latches to the APA frame similarly to the upper field cage. The electrical connection to the HV bus are tested and the deployment is complete. In principal the cathode/field-cage assemblies can be constructed faster than the APAs. It will be investigated if one should
wait to deploy the lower filed cages until after all the APA are in place. This will allow access to the APA till the end of the installation and one can also clean the cryostat floor before deploying the field cages and closing the cryostat.
Keeping the CPA installation nearly synchronous to the APA installation has the benefit that the APA can be tested after any loads or mechanical disturbances from the field-cage deployment are complete. This minimizes the work on an APA after additional APAs are installed. In the event an APA is broken an needs to be removed any APAs between it and the TCO need to also be removed. This could lead to a long delays if an APA is damaged after several additional APAs are in place. The risk and benefits of the different deployment sequences will be evaluated.

\begin{dunefigure}[Installation of row 25]{fig:install-row25}
  {Detector installation during the installation of the last row of detector components. At this time the switchyard beams are removed and the temporary hoists for the end-wall are installed.}
\includegraphics[width=.9\textwidth]{install-row25}
\end{dunefigure}


The last row of detector elements are installed similar to the previous rows but the runway beams of the DSS switchyard are removed as the APA and CPA are placed in position. Figure \ref{fig:install-row25} shows the top of the detector as the last row of APA and CPA are installed. When the shuttle beam is aligned to the correct row of APA or CPA the short section of the beam is bolted to a short I-Beam section of the runway beam that is permanently fixed to the DSS support feedthrough. When the shuttle beams at both ends of a runway beam section are fixed in position a section of the runway beam is removed. The last field-cage modules are deployed. 


\begin{dunefigure}[Second endwall installation]{fig:install-ew2}
  {Installation of the final end-wall prior to closing the TCO.}
\includegraphics[width=.9\textwidth]{install-ew2}
\end{dunefigure}

The second outside end-wall is installed like the first end-wall. Now the Center APA is rolled into the cryostat, the shuttle beam is bolted to the DSS and the runway beams are removed. The two center drift volumes field cages are deployed. The last two end walls are constructed and the TPC is basically completed. As the calibration and instrumentation are not defined yet it is not clear if these will interfere with the TCO closing. If they do they will be installed after and access will only be via scaffolding. 

At this point a frame supported by the shuttle beams is covered with flame retardant plastic is installed to create cleanroom work area for the TCO.  A scaffold is set up as egress to the manhole. The cryostat will become confined space area so it must be in place before the TCO is closed up.  The top 3/4 of TCO is completed using the 14m scissor lift. The temporary cleanroom area is removed and the area cleaned. A smaller clean area is made for the bottom 3.4m TCO section.  The scissor lift must be removed at this part. The TCO work is completed and the area cleaned.

The second scaffold is set up in the space if access is required  for cleaning of calibration and instrumentation installation.

The dynamic T-gradient monitor is installed at this time. 
The monitor comes in several segments with pre--attached sensors and cabling already in place. Each segment is fed into the flange one at a time until the entire sensor carrier rod is in place. The remainder of the system (motor system that moves the sensor rod and the sensors) that goes on the top of the flange is installed using a crane. The purity monitor system will be built in modules, so it can be assembled outside the cryostat leaving few steps to complete inside the cryostat. 
The assembly itself comes into the cryostat with the three individual purity monitors mounted to support tubes which are then mounted to the brackets inside the cryostat, and the brackets attached to the appropriate elements (cables trays, DSS and bolts in cryostat corner are under consideration). Also at this time the remaining level monitors are installed.

Once all this work is completed the scaffolding is taken apart and hoisted out the man-hole along with all the remaining flooring sections. The area is cleaned and the last two FTE in the cryostat are hoisted out. 

The inspection cameras and other possible calibration instruments can be installed from the roof while the TCO is being closed.



% clear the figure buffer before starting the next section
\clearpage












%%%%%%%%%%%%%%%%%%%%%%%%%%%%
\subsection{Prototyping and Testing (QA/QC)}
\label{sec:fdsp-tc-inst-qaqc}



\subsubsection{Introduction}


Numerous times during the installation of ProtoDUNE it was clear that the experience we gained during the trial assembly period was mechanically critical but also developing all the special tooling required to do the job.  Testing access with a fake cryostat roof or walls to understand how it is physically possible to do some tasks is important. DUNE will have 1/2 the available working space on both the top and bottom of the detector.  One of the lessons learned at ProtoDUNE this was the difficult to access this area as we secured the bottom latches and tied the APAs together on the bottom.  The process of developing hazard analysis and procedure documentation for the assembly process is an important step for approval to begin installation.   

While mechanical tests at the DUNE-Trial Assembly at Ash River is a key component of the installation process.
Other important R\&D tasks are happening at other Universities, National Labs and CERN. Argonne National Lab is testing the APA shuttle beam drive system and the CPA Assembly Tower connections before it is shipped to Ash River.  
Darnsbury as constructed the new process carts from lessons to make the assemble task both more efficient but improves the safety level of the tasks preformed. 
There are numerous small steps in the assembly process that have been improved from our past experience at CERN which will be developed during this prototyping phase. 

It is the responsibility of the different consortia to develop the QC program during the Installation Phase. 
They are listed in the following QC sections. 

\subsubsection{Ash River Testing}

Full scale mechanical testing of all the TCP components including the DSS are critical for the success of the Single Phase detector as shown from the \dword{protodune} experience. 
The NOvA Far Detector Assembly area at Ash River meets the criteria of both area and available equipment but also experienced technicians that helped construct the \dword{protodune} detector. 
By definition, the installation  is on the critical path, making it vital that the work be performed efficiently and in a manner that has low risk. 
In order to achieve this, a prototype of the installation
equipment for the \dword{spmod}  will be constructed at Ash River (the \nova neutrino experiment \dword{fd} site in Ash River, Minnesota, USA), and the installation process tested with dummy detector
elements.  
In the period just prior to the start of
installation, the Ash River setup will be used as a training ground for the installation team.
Figure~\ref{fig:NOvA-Assembly-Area}
\begin{dunefigure}[NOvA Assembly Area at Ash River]
{fig:NOvA-Assembly-Area}
{NOvA-Assembly-Area}                
%\centering     %trim=left bottom right top, clip
\includegraphics[width=0.44\textwidth]{NOvA-Assembly-Area}
\includegraphics[width=0.55\textwidth,trim=0pt 0pt 0pt 0pt,clip]
{DUNE-Trial-Assembly-Ash-River}
%\includegraphics[width=0.49\textwidth,trim=0pt 300pt 180pt]{NOvA-Assembly-Area}
\end{dunefigure}




The NOvA Far Detector Lab located at Ash River is owned and operated by the University of Minnesota via grants from the DOE and Fermilab. The Trial Assembly program for both \dword{protodune} and Dune are operated here. During the \dword{protodune} detector design many of the installation and deployment concepts where tested and modified here.  It was critical that we could attempt different concepts with full-scale mechanical components. Hazard analysis and procedure documents were developed and hands on working group meetings with the different consortia were held.  The lessons learned and training from the trial assembly helped insure that the assembly and integration for \dword{protodune} went well at CERN.

While the University has jurisdiction over the safety program at Ash River, we also follow the Fermilab Safety Program and work together in a joint effort to insure a safe working environment.  One of the key attributes of the ProtoDUNE work was compiling sets of documentation from component design, to hazard analysis, the final assembly procedures for approval by CERN HSE (Health Safety and Environment). For Ash River and DUNE this is all part of the Operational Readiness Clearance (ORC) process. Documentation for both the trail assembly process at Ash River and for DUNE will be stored on EDMS at CERN. 
Many of the TPC components are similar, access equipment and working at 14m will make construction of the DUNE Single Phase detector much more challenging.   The NOvA Far Detector Assembly area Figure~\ref{fig:DUNE-Trial-Assembly-Ash-River} has both the elevation and floor space available to do both a full-scale test of both the assembly stations and a portion of the inside of the DUNE cryostat. 

\begin{dunefigure}[DUNE - Trail Assembly at Ash River] %this gets %sniffed for table of figures
{fig:DUNE-Trial-Assembly-Ash-River}  %this is figure name and is %what the above \ref:{fig:...} matches 
{DUNE-Trial-Assembly-Ash-River}  %this is caption
\centering
\includegraphics[width=0.7\textwidth,trim=0pt 0pt 0pt 0pt,clip]
{DUNE-Trial-Assembly-Ash-River}
\end{dunefigure}


At Ash River, we also have a 75 \si{ft} $\times$  100 \si{ft} loading dock and ramp access, two 10-ton cranes, a machine shop and wide assortment of tools.  Add an experienced crew of technicians that all had multiple months at CERN building ProtoDUNE and fabricating experience it makes that a good facility for these tasks. 
The main goals for the DUNE FD Trial Assembly at Ash River are:

\begin{enumerate}
\item Test all full scale TPC (Time Projection Chamber) components during assembly stages and inside the cryostat including  
\begin{itemize}
    \item APA  Assembly including manipulation of APA shipping frames, joining an APA pair together, CE (Cold Electronics) cabling, APA protection, movement on shuttle beam, cryostat cabling and final deployment in cryostat. 
    \item Integration and installation testing of \dword{pd} (Photon Detector) components, including cable harness routing and cryoegnic cable strain relief, module integration into \dword{apa} frames, and electrical connections between upper and lower \dwords{apa}.  In addition, \dword{pd} monitoring system components mounting and optical fiber routing on the CPA will be tested.
    \item DSS (Detector Support Structure) and shuttle beam system including final detector configuration 
    \item Assembly of HV system including construction of an End Wall, CPA pairs, movement on shuttle beam and final deployment in cryostat
    \item Future Assembly of Dual Phase detector components
\end{itemize}
\item Write full set of Hazard analyses and assembly procedure documents including gathering all component documentation 
\item Test access equipment (scaffold, scissor lifts, work platforms) and lifting fixtures 
\item Assembly time and motion studies including labor estimates
This facility can also be used Future as a training site for lead workers as DUNE begins setup, testing mechanical modifications.
\end{enumerate}

\begin{dunetable}
[Summary of the tests at Ash River]
{ll}
{tab:table-label}
{The full caption that appears above the table.}
Testing phase & Test Description\\ \toprowrule
FY-19 Phase 0   &  \\ \colhline
 & Test FC Deployment and Ground Plane installation \\ \colhline
 & Check Vertical cabling with a pair of APA side tubes \\ \colhline
 & Build an APA Cabling tower for full scale APA pair assembly \\ \colhline
 & Build a CPA assembly stand and test assembly process \\ \colhline
  & Test APA Shipping Frame \\ \colhline
  FY-20 21 Phase 1 &  \\ \colhline
  & Build support structure for DSS shuttle, 3 sections of DSS beam \\ \colhline
  &  Test movement of CPA and APA from cleanroom to final destination\\ \colhline
  & Test APA, CPA, Endwall and FC deployment in one drift section \\ \colhline
  & Test assembly sequence of final section of TPC \\ \colhline
  & Removal of DSS shuttle beam runway rails \\ \colhline
  & Final deployment after TCO is closed up \\ \colhline
  FY-22 thru FY23 Phase 2&  \\ \colhline
  &  Include the top of the cryostat ( no warm structure) with \fdth \\
  \colhline
  & Test DSS installation  \\  \colhline
  &  Test CE cable installation using \fdth \\  \colhline
  & Design \fdth to support Dual Phase installation test \\ \colhline
  & Test shipping and construction using first factory TPC components  \\ \colhline
  & Train lead workers for underground at SURF \\ \colhline

\end{dunetable}

\subsubsection{DAQ QC testing}

There are several stages of DAQ installation which require  testing.  The first is the installation of the dataroom infrastructure, where cooling water leak checking, rack airflow, and power distribution will be tested upon install by the professional data center building contractors doing this installation.

The detector-to-dataroom multimode fiber will be running close to its optical power budget.  It is thus important that as this fiber is routed from the WIBs on top the cryostat to the servers in the CUC, that it be installed and tested by fiber professionals early in the process.  Covered cable trays will protect it after installation.  As APAs and servers are commissioned, pre-tested fibers will be connected to the newly installed hardware.

The DAQ servers in the CUC dataroom will be initially received and integrated offsite.  Upon installation in the CUC, only a simple functionality test will be needed.  Sufficient spare capacity will be installed, and the main commissioning work will be software related and doable over the network from the surface or remotely.

\subsubsection{APA QC testing}

Once the APAs have been installed, the only relevant test for the APA quality is the tension measurement of the wires. 
Note that testing the cold electronics will inform directly the wire continuity and the full function of the channels, therefore, only the tension remains to be tested. 
The current plan would be to re-test a sub-fraction of the 350 wires that were measured at the ITF using the laser method. 
The exact number or wires to test will depend on the final schedule for the underground testing. 
Two alternative methods for the tension measurement are also currently under development. 
One using electrical signal that would allow to measure a much larger fraction of wire tension (and potentially all of them). 
The other one would be to use the cold electronics readout to extract the tension from wire movement. 
This last method would allow to measure all the wire in a short amount of time.

\subsubsection{HV QC testing}

The EndWalls are assembled in 8  panel units, 4 on each end of the TPC.  
As each of the 8 panels are removed from the shipping crate and onto the installation cart, the EndWall Panel Checklist is filled out.\cite{bib:docdb10452}
This checklist contains a visual inspection of the frames, profiles and connections, and continuity and resistance measurements of the divider boards and their connections.  
After completion of an 8-panel EndWall in the cryostat, the Complete EndWall Checklist is filled out.  
This includes hanging position and straightness measurements, and continuity checks between panels.

The CPA panels are assembled from 3 units removed from the shipping crates.  
After each unit is removed from its bag, a visual inspection is made of the structural integrity and the connections between FSSs, HV Bus pieces and Profiles if present.  
After inspection, the unit is positioned on the CPA assembly tower.  
After the 3 units are connected on the tower, the CPA Panel Checklist is filled out.  
This includes inspection of all mechanical connections, continuity checks of the FSS, RP, Profile, and HV Bus connections, and resistance measurements of the 4 mini-resistor board connections from the RP to the FSS.  
This is repeated for the second panel in a CPA plane.  Then the 2 panels are paired, each hanging from trolleys on the transport beam.  
Visual inspection of the alignment and hanging straightness are made and HV Bus connections at the top and bottom are made and checked for continuity (CPA Plane Checklist).

The FC Top and Bottom units are removed from their crates.  The FC Unit Checklist is filled out including a visual inspection of the frames, profiles, and connections, and continuity and resistance measurements of the divider boards and their connections.  
After hanging the top FC units on the CPA plane, the 4 jumpers from the first FC Profile on each side of the CPA and the CPA FSS is connected and the resistance is measured, completing the CPA/FC Top Assembly Checklist. 
The FC Bottom units are not attached to the CPA, but are taken into the cryostat independently after filling out the FC Unit Checklist described above.

The CPA/FC Top assembly is moved into its position in the cryostat.   
After deployment of the FC Top units, visual inspection of the resistor board/jumper between the FC and the CPA FSS is made.  
Also, visual inspection of the latch at the FC/APA is done.  After deployment of the FC Bottoms, visual inspection is made to verify the resistor board/jumper connection from the FC to the CPA.  
Also, visual inspection of the latch connecting the FC Bottom to the APA is made.  
These are included in the CPA/FC Cryostat Checklist.

\subsubsection{CE QC testing}
Details of the CE QC testing are found in the CE chapter.

\subsubsection{CISC QC testing}

Cryogenics instrumentation systems must undergo a series of tests to guaranty they will perform as expected:  

\begin{itemize}
\item {\bf Purity Monitors}: Each of the fully assembled purity monitors arrays is placed in its shipping tube, which serves as a vacuum chamber to test all electric and optical connections at \surf before the system is inserted into the cryostat. During insertion, electrical connections are tested continuously with multimeters and electrometers.

\item {\bf Static T-gradient Thermometers}: Right after the installation of each of the sensor's arrays, its verticality 
is checked, and the tensions in the stainless steel strings adjusted as necessary. Once cables are routed to the corresponding DSS ports the entire readout chain is tested. This allows a test of the sensor, the sensor-connector assembly, the cable-connector assemblies at both ends, and the noise level inside the cryostat.
If any sensor presents a problem, it is replaced. If the problem persists, the cable is checked and replaced as needed.

\item {\bf Dynamic T-gradient Thermometers}: The full system is tested after it is installed in the cryostat. Two aspects are particularly important, the vertical motion of the system using the step motor, which is controlled via the slow controls system; and the full readout chain which will be tested mainly to detect failures in sensors, cables and connectors inside the cryostat. 

\item {\bf Individual Sensors}: To address the quality of individual precision sensors, the same method as for
the static T-gradient monitors is used. For standard RTDs to be installed on the cryostat walls, floor, and roof, calibration is not an issue. Any QC required for associated cables and connectors is performed following the same procedure as for precision sensors.

\item {\bf Gas Analyzers}. Once the gas analyzer modules are installed at \surf and before commissioning the cryostat, they 
are checked for both \textit{zero} and the \textit{span} values using a gas-mixing instrument and two gas cylinders, one having a
zero level of the gas analyzer contaminant species and the other cylinder with a known percentage of the contaminant gas. This
 verifies the proper operation of the gas analyzers. 

\item {\bf Liquid Level Monitoring.} Once installed in the four cryostat corners, the capacitive level meters are tested in situ 
using a suitable dielectric in contact with the sensors.

\item {\bf Cameras}. After installation and connection of wiring, fixed cameras, movable inspection cameras and light-emitting system are checked for operation at room temperature. Good quality images of all cryostat and detector areas considered on the system's design should be obtained. The movable system for inspection cameras should behave as expected.  
\end{itemize}

\subsubsection{Photon QC testing}

Access to the photon detectors inside the \dwords{apa} is severely limited once the \dword{ce} cable conduit is in place, increasing the need to identify problems early in the process to minimize the schedule impact from required PD maintenance or repair due to problems detected during installation.

Upon receipt of integrated \dwords{apa} in the underground prep area (\dword{sas}) for the clean room in front of the cryostat, an electrical connectivity test between the cable end exiting the \dword{apa} to be installed and each of the PD modules in that \dword{apa} will be performed.  Discussions are underway with the \dword{apa} consortium to determine if the transport frames for shipping the integrated \dword{apa} pairs can be made sufficiently light-tight to allow for biasing the photosensors and checking that they are operating properly.  As accessing the \dword{pd} modules involves removing the \dword{ce} cable conduits from the \dword{apa} side tubes, this is the last step in the installation process where PDs can be reached for repair or replacement.  Any repairs to the \dword{pd} system discovered later in the installation process would require returning the \dword{apa} module to this point in the installation sequence.

As the upper and lower \dwords{apa} are joined on the assembly tower, photon detector cables from the upper to the lower \dword{apa} will be connected, and at that time continuity checks will be made.

Following joining of the upper and lower \dwords{apa}, the assembled unit will be moved into a cold box in front of the cryostat for final testing.  This will represent an opportunity to make a final low-temperature checkout of the complete \dword{pd} cold electronics and cabling chain prior to installation into the cryostat.  \dword{pd} FE electronics boards will be used to read out the photon system during the cold test, and results compared to previous \dword{qa} test results.

The \dword{apa} stack is rolled into position in the cryostat following the cold box test, and the \dword{pd} and \dword{ce} cables connected to the cryostat flange.  At this point, a final continuity check is made from the flange bulkhead to the \dword{pd} module.

Discussions are underway with the installation team to arrange for a one-shift dark test of the installed photon detectors in the cryostat following final installation, verifying end-to-end system operation.



%%%%%%%%%%%%%%%%%%%%%%%%%%%%
\subsection{Environmental, Safety, and Health (ESH)}
\label{sec:fdsp-tc-inst-safety}

Fermilab and DUNE are committed to supporting its research and operations by protecting the health and safety of staff, the community and the environment, as stated in the LBNF/DUNE Integrated Safety Management Plan. The Safety and Health Program is in compliance with applicable standards and Local, State and Federal legal requirements through Fermilab's Work Smart Set of Standards and the contract between Fermilab Research Alliance and the Department of Energy. Fermilab and the South Dakota Services Division have the host laboratory responsibilities for LBNF/DUNE operations at the Sanford Underground Research Facility in Lead, South Dakota.
The Fermilab facilities are further subject to the requirements of the Department of Energy (DOE) Workers Safety and Health Program 10 CFR 851. These requirements are promulgated through the Fermilab Directors Policy Manual, and the Fermilab Environment, Safety, and Health Manual (FESHM) which align with the SURF ESH Manual.

While ESH will be  a host lab (Fermilab/SDSD) responsibility, there is a  Global Safety Coordination group in place to evaluate applicable codes and standards including international code equivalency for the design, assembly, and installation of the DUNE detectors. The Global Safety Coordination group is  a team of engineering and ESH experts from within LBNF and DUNE organizations.  These requirements will be adopted by DUNE, JPO (Joint Project Office) and LBNF organizations and be used to develop manufacturing, assembly and installation processes and procedures 

 An Installation Environment, Safety and Health Plan will be developed which will define a specific set of ESH requirements and responsibilities which personnel will be required to follow to perform assembly, installation, and construction activities at the Sanford Underground Research Facility (SURF) and the Integrated Test Facility (ITF). 

{\bf Work Planning and Controls:} The goal of the work planning and hazard analysis (HA) process is to initiate thought about the hazards associated with work activities and how it can be performed safely. Careful planning of a job assures that it is performed efficiently and safely. Work planning ensures the scope of the job is understood, appropriate materials and tools are available, all hazards have been identified, mitigation efforts established, and all affected employees understand what is expected of them. Hazard analysis is a critical part of work planning.  The Work Planning and Hazard Analysis program is documented in Chapters 2060 in the FESHM.

Daily work planning meetings will be led by the shift supervisor at the start of each shift to coordinate work activities, review hazards/mitigations, and answer any questions

{\bf Documentation Approval Process:} There will be an engineering review and approval process of all required documentation including structural calculations, assembly drawings, load tests, hazard analyses, and procedural documents for typical individual tasks.  For the larger operations and systems like TPC component factories, DSS, cleanroom and assembly infrastructure this will be followed by an Operational Readiness Review. This is done by a joint safety committee that visits the sites after reviewing the documentation and watches the full operation before it is signed off on.

{\bf ESH Support:} Safety starts from the ground up both on the surface and in the underground facilities. All employees have work stop authority in support of  a safe working environment. Personnel will need to utilize the necessary PPE identified through the hazard analysis process for the task. A local ESH coordinator under the direction of the DUNE ESH Manager will provide daily support of the facilities and attend the daily work planning meetings to notify each shift of potential safety issues/constraints, to validate that  employees have the necessary ESH training, and is responsible for managing ESH related  documentation including training records, weekly safety reports, near miss/accident reports and equipment inspection.

{\bf Work Underground SDSTA will maintain:}
\begin{itemize}
    \item Site Access Control Program through Trip Action Plans (TAP)
    \item Emergency Management (EM) program which includes an Emergency Response incident command system and  Emergency Rescue Team (ERT).  Typically, a small number of employees underground (Guides) are also training first-responders to help in a medical emergency.
    
\end{itemize}

{\bf Equipment Operation:} All overhead cranes, gantry cranes, fork lifts, motorized equipment trains/carts will only be operated by trained licensed operators.. 
Other equipment like scissor lifts, pallet jacks, hand tools and shop equipment may only be operated by people trained
and certified for that particular piece of equipment.

{\bf House Cleaning, Training, Lab Access and PPE:}It is the responsibility of all workers underground to keep a clean organized work area. Limited storage of flammable items must be in proper storage cabinets, items like shipping crates and boxes must be removed and shipped back to the surface. There is very limited space underground so it is critical especially for large objects that must be slung loaded like shipping crates that hoist trips are arranged for these empty items to go up after new items come down. PPE is the responsibility of the host lab (Fermilab/SDSD) to be supplied for all workers.. All workers will be required to complete both SURF Surface and Underground Orientation classes. In addition, workers accessing the underground will be required to complete 4850 and 4910 level specific unescorted access training. A guide will be required to be stationed on all working levels and received the necessary guide training. .  e All workers requiring access to the SURF site will be required to register through the Fermilab User Office to received the necessary User training and a Fermilab ID number. In addition, all workers will be required to apply for a SURF identification badge for DUNE activities.

{\bf Requirements at Collaborating Laboratories and Institutions:}All work performed at collaborating institutions will be completed in accordance with the collaborating institutions Environmental, Safety, and Health policies and program. 
Equipment and operating procedures provided by the collaborating institution will conform to the DUNE Project ES\&H and Integrated Safety Management policies and procedures. 
The collaborating institutions ES\&H Department shall be responsibility for providing ES\&H oversight for all work activities carried out in collaborating institution facilities. 
LBNF/DUNE personnel will also follow the ES\&H Manual and procedures of the collaborative institutions.


%%%%%%%%%%%%%%%%%%%%%%%%%%%%
\subsection{Costs, Schedule and Risk Analysis}
\label{sec:fdsp-tc-inst-cost}

\fixme{Add Cost section when available}


The schedule for the Single Phase Detector 1 Installation is a complicated dance relying on numerous other entities including CF, LBNF, and SDSTA.  There is a maximum allowable limit for the number of people underground of 140 dure to the size of the mine rescue area.  This is particularly critical during the excavation of Cavern 3. As different area underground areas will give us access at different times. A detailed summary of these time periods are shown in Figure~\ref{fig:Overview_of_SinglePhase_Schedule}



\begin{dunefigure}[Overview of the single-phase schedule]
{fig:Overview_of_SinglePhase_Schedule}
{Schedule Overview of the Single Phase Detector \#1}                
\includegraphics[width=0.98\textwidth]
{Overview_of_SinglePhase_Schedule}
\end{dunefigure}

The cost, schedule and labor estimates are based on two 10 hour shifts per day, 4 days per week. It is assume that our work efficiency is a maximum of 70 percent.  Approximately 2-3 hours per day is taken up by the cage ride underground, the start of shift meeting,lunch and coffee breaks and gowning to go into the cleanroom. 

There are 3 basic schedule phases for Detector 1:

\begin{itemize}
    \item {\bf CUC Installation Phase:}
    This time period described in detail in section 1.4.2.1 can start once Acceptance for Use and Possession (AUP) has been recieved for the North Cavern and the CUC. This is also the same time that the excavation of the South Cavern and installation of the Warm structure by LBNF begins. Maximum number of FTEs will be limited to 140 people and access for hauling equipment in the shaft will minimal during this time period.  It is estimated that it will take 3 months to install the basic rack infrastructure then over a 12 month period of continued installation of DAQ equipment sections of the system will be operational. The operation of the cold boxes in the cleanroom and the testing of the readout once and APA has been installed in the cryostat are needed at the beginning of installation phase. During this time period the labor underground is minimal 5-10 FTE from the DAQ consortia and contractors plus the core UIT team as it ramps up in size. 
    
    \item {\bf Installation Setup Phase:} This phase is defined in detail in section1.3 Infrastructure. During this time period we begin working two shifts per day and the UIT team doubles in size.  This is a critical training period, getting the lead workers, riggers and equipment operators familiar with the tasks and adjusting crews to insure balanced teams.  Prior to this phase the DUNE Trial Assembly equipment at Ash River will be used as a training site begin the training process.
    
    \item {\bf Detector Installation Phase:} The final detector installation phase begins with an Operational Readiness Review to check that all documentation and procedures are in place. After the east end walls are installed there is a start-up period 1.5 months for the first two rows of TPC components. After that the installation rate is 9-10 shifts to complete each row.  To achieve this schedule we use two APA cabling stations 3 cold boxes and separate crews in the cryostat all working in parallel.  It takes 5.5 months to install rows 3-24 and about 1 month for row 25. There is approximately a 2 month period to close the TCO by the cryostat cold structure contractor. During this time there is no access to the cryostat.  Once this is completed the final instrumentation is completed and purge can begin. 
    
\end{itemize}

\begin{dunetable}
[Installation System Risk Summary]
{p{0.05\textwidth}p{0.4\textwidth}p{0.4\textwidth}}
{tab:INSTALL-risks}
{Installation Risk Summary}   
ID & Risk  & Mitigation\\ \toprowrule

1 & 
Work underground, work at heights, handling heavy equipment, tests with \dword{hv}, laser operation, and  other hazards can lead to personal injury.
& Follow FNAL safety program: procedures for work, training, use of \dword{ppe}, with levels of oversight. Use all reasonable measures to prevent workplace incidents. Still consider residual risk as critical.\\ \colhline
2 & 
Shipping delays can lead to installation delays. &
Plan one month buffer to store  materials locally.\\ \colhline
3 & 
Unavailability of components can delay assembly and installation. &
Use a detailed inventory system to verify availability of  necessary components prior to transport underground.\\ \colhline
4& 
Unavailability of international contract labor or equipment  can delay or prevent necessary work. &
Dedicated \dword{fnal} division in SD will expedite  import/export and visa related issues.\\ \colhline
5&
Limited availability of local trained workers can delay or prevent necessary work.
& Begin hiring process early; allow time for substantial training. Use Ash River installation prototype to train crew and optimize procedures.\\ \colhline
6&
Mismatch in dimensions or run-out in tolerances can lead to mechanical interferences. & 
Generate integration model of full \dword{detmodule}. Include key dimensions in integration drawings to control  interfaces.  Prototype assembly steps. Perform acceptance tests.\\ \colhline
7&
Human error in transport of heavy objects and work at heights with tools can lead to cryostat damage. & 
Construct false floor inside the cryostat for use during installation. Remove it as late as possible in process. Install temporary protection for work after its removal.\\ \colhline
8& 
Several times per year \dword{surf} closes for 1-2 days due to snow; closures can cause delays. & Include the average number of snow days in schedule.\\ \colhline
9&
Installation and cooling of components produces stress that can cause equipment failures.&
Test all components prior to delivery to SD to eliminate infant mortality failures. Cold test \dword{apa} assemblies once cabling and assembly is complete. Read out detector continuously to detect failures.\\ 
\end{dunetable}


\subsection{Detector Commissioning Phase}
\label{sec:fdsp-tc-inst-comiss}
After the \dwords{detmodule} is installed in the cryostat there remains a lot of work before it can be operated. 
First the \dword{tco} must be closed. 
This requires bringing back the cryostat manufacturer. 
First the missing panel with the steel beams and steel panel are installed to complete the cryostat's outer structural hull. 
Then the remaining foam blocks and membrane panels
are installed from the inside using the roof access holes 
to enter the cryostat. 

In parallel, the \lar pumps are installed at the ends of the cryostat and final connections are made to the recirculation plant. 
Once the pumps are installed, the cryostat is closed, and everything is leak tested, the cryogenics plant can be brought into operation. 
First the air inside the cryostat is purged by injecting pure argon gas at the bottom  at a rate such that the cryostat volume is filled uniformly but faster than the diffusion
rate. 
This produces a column of argon gas that rises through the volume and sweeps out the air. 
This process is referred to as the \textit{piston purge}. 
When the piston purge is complete the cool-down of the \dword{detmodule} can begin. 
Misting nozzles inject a liquid-gas mix into the cryostat
that cools the detector components at a controlled rate. 

Once the detector is cold the filling process can begin. 
Liquid argon stored at the surface  at \surf is vaporized and brought down the shaft in gaseous form and is then re-condensed underground. 
The \lar then flows through filters to remove any H$_2$O or O$_2$ and flows into the cryostat.
Given the very large volume of the cryostat and the limited cooling power for re-condensing, it is  expected to take \num{12} months to fill the first \dword{detmodule}. 
%During this time the detector readout electronics will be on monitoring the status of the detector. 

Several tests and a constant monitoring of the detector will be taking place from \dword{tco} closing till the end of the filling process.
Before the last manhole is sealed:
\begin{enumerate}

    \item A pedestal and RMS characterization of all cold electronic channels will be performed, to verify all \dword{apa} front-end boards are responding, and no dead channel or new noise sources arose following the \dword{tco} closing
    
    \item A noise scan of all \dword{pd} channel is performed as last check before sealing

    \item Each \dword{apa} wireplane will be checked to verify it is isolated from the \dword{apa} frame and properly connected to its HV power supply through the following steps
    
\begin{itemize}

    \item The SHV connector of each wireplane bias channel will be unplugged at the power supply and the resistance between inner conductor and ground and its capacitance will be measured. The resistance should show the wireplane electrically isolated from the ground, while the capacitance value should match that of the cold HV cable plus the capacitance of the circuit on the \dword{apa} top frame

    \item 50 V are applied to each wireplane and the current drawn checked against the expected value
    
    \item Nominal voltages are applied to each wireplane and the current drawn checked against the expected value 
    
\end{itemize}

    \item Apply a low (i.e 1-2 kiloVolts) to the cathode and measure the current drawn to ensure the integrity of the HV line

\end{enumerate}

During the \textit{piston purge} phase, periodic monitoring of both APA cold electronics and \dword{pd} system noise (pedestal, RMS) will occurr.

A number of the above mentioned tests, in addition to new ones, will instead take place during the cool-down phase:

\begin{enumerate}


    \item Each \dword{apa} wireplane isolation and proper connection to its HV power supply will be checked at regular time intervals in the same way as done before sealing the cryostat
    
    \item 1-2 kiloVolts will be hold on the cathode, and the current drawn constantly monitored to observe the behaviour with temperature of the total resistance
    
    \item Cold electronics noise figures (pedestal, RMS) will be measured at regular intervals and its trend with temperature recorded
    
    \item \dword{pd} system noise (pedestal, RMS) will be measured at regular intervals and its trend with temperature recorded
    
     \item Values of the temperature sensors deployed in several parts of the cryostat will be constantly monitored to understand the progress of the cool-down phase and related to the behaviour of the other \dword{detmodule} subsystems 
     
\end{enumerate}

Regular monitoring of CE and \dword{pd} noise, as well as check of wireplane isolation and proper connection to bias supply system will continue throughout the filling, recording the noise variations as a function of the progressively lowering temperature. In addition:

\begin{enumerate}

    \item as each purity monitor is submerged in liquid, turn it on every 8 hours to control LAr purity
    
    \item as soon as top ground planes are submerged, raise HV on the cathode up to 10 kiloVolts and check the current drawn by the system agrees with expectations.

\end{enumerate}

Once the \dword{detmodule} is full, the drift high voltage will be carefully ramp up following these steps:

\begin{enumerate}

    \item Need for a filter regeneration is evaluated before start of any operation

    \item Once filter regeneration is completed (if needed), LAr surface is looked at with cameras to verify whether it is flat or any bubble/turbolence can be spotted
    
    \item LAr recirculation is started and LAr surface is looked at again to verify whether activation of the recirculation system introduced any turbolence in the liquid.
    
    \item wait one day since begin of recirculation to stabilize the LAr flow inside the \dword{detmodule}, then start the HV ramp.
    
\end{enumerate}

Cathode voltage is raised in steps throughout the course of three days. 
On the first day, cathode voltage is raised first to 60 kV, then to 90 kV after a 2-hours waiting period, and finally to 120 kV after another 2-hours waiting period, and left at this value overnight.
On the second day, cathode voltage is raised first to 140kV, then to 160 V after 4 hours waiting period, and left at this value overnight.
On the third day, cathode voltage is raised to 170kV first, and then to the nominal operating voltage of 180 kV after a 4 hour waiting period.
During each HV ramp, all CE current draws are monitored and the procedure is stopped if any of the current draws goes out of the allowed range.
During each waiting period, regular DAQ runs are taken to monitor CE and \dword{pd} noise and response, while cathode HV and current draw stability are constantly monitored.

In ProtDUNE-SP this process took three days and the system is ready for data-taking. With a detector twenty times larger it will take longer but the turn-on time is still expected to be relatively short. 

\subsection{Conclusions}
\label{sec:fdsp-tc-inst-concl}

That's all folks
The End
\fixme{contribution from editors}