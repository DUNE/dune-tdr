\section{Logistics}
\label{sec:fdsp-tc-log}

%%%%%%%%%%%%%%%%%%%%%%%%%%%%
\subsection{Introduction}
\label{sec:fdsp-tc-log-intro}
The transportation of equipment and people underground is one of the more challenging aspects of the LBNF/DUNE endeavor. The access underground is via the mile long Ross shaft, which is now undergoing renovation. The shaft is outfitted with a single cage for people and materials and two skips which are needed for removing the rock underground. Planning the usage of the cage is one of the most important exercises in making LBNF/DUNE a success. Given the enormous cost of the conventional facilities (CF) contracts and the large cost of any in-efficiencies in construction the overall coordinator of the Ross Shaft for LBNF/DUNE will be the CF Central Manager General Contractor (CMGC). Independent of CF both LBNF and DUNE have a large number of contractors, institutions, and scientists who will need to bring equipment and materials underground. In order to facilitate the flow of material and people to the underground area a logistics organization will be established in South Dakota near SURF. This organization will be responsible for receiving all goods for LBNF (except CF) and DUNE, it will be responsible for coordinating the transport of this material underground with the CF-CMGC, and it will coordinate all personnel usage of the Ross cage with the CF-CMGC. 


\begin{dunetable}
[Logistics Specifications]
{cc}
{tab:table-Log-Req}
{Table of logistics specifications (needs to move to offical spot).}
Specifications &  \\ \toprowrule
Material Handling & Comply with the SURF Material Handling Specificaiton \\ \colhline
CMGS coordination & Provide CMGC with 2 week notice of shipments to SURF \\ \colhline
Stage DUNE Shipments & Provide a 1 month local buffer of DUNE materials \\
\colhline
APA Storage & Provide storage space for 150 APA in a clean enviroment \\
\colhline
Inventory & Provide an inventory system accessable to the collaboration \\
\end{dunetable}
 
 
 



%%%%%%%%%%%%%%%%%%%%%%%%%%%%
\subsection{Logistics Planning}
\label{sec:fdsp-tc-logPln}
The LBNF/DUNE Logistics scope includes overseeing the transportation of the cryostat (steel, foam, and membrane), the cryogenic system, the detector, and all related infrastructure not provided by facilities. The LBNF scope consists of the cryostat and cryogenics and it will not be discussed in detail in this TDR but as the LBNF material dominates the logistics needs a brief summary here is required. The cryostat steel structure for each cryostat requires bringing roughly 1,800 individual steel pieces underground some which weigh up to 7.5t. 125t of bolts are needed to assemble the steel pieces. The internal structure which includes the foam insulation and the thin stainless steel membrane will require transporting roughly 4,000 boxes each roughly 1.5 $\times$ 3.5 $\times$ 1.2 m$^3$ in dimension. The plan for cryostat installation at present calls for all the components to be warehoused in SD prior to the start of installation. This means that the logistics operation will need roughly 5,000 m$^2$ of warehouse space roughly 2 years prior to the start of DUNE detector installation. By the time DUNE detector components start arriving most the cryostat boxes will have been removed from the warehouse so there will be ample space for the detector and cryogenics components. Additional space may be required if the boxes for the second cryostat arrive before the detector\#1 installation is complete, but several buildings of the required size are available in the area if it is decided expansion is required.


\begin{dunefigure}[Ross Cage]{fig:fdsp-tc-Cage}
  {Simplified model of the new Ross Cage.}
\includegraphics[width=.5\textwidth]{Cage-view}
\fixme{combine with table?}
\end{dunefigure}
%
\begin{dunetable}
[Ross Cage Specifications]
{cc}
{tab:table-Ross-Cage}
{Table of parameters for the Ross hoist and cage.}
Ross Cage Parameters &  
\\ \toprowrule
Inside height &  3.6 m\\ \colhline
Inside depth & 3.7 m \\ \colhline
Inside width & 1.38 m \\
\colhline
Weight limit&  5,897 kg \\
\colhline
Round trip time & 17 min\\ \colhline
\end{dunetable}

All material brought underground must conform to the Sanford Underground Research Facility's (SURF) \href{http://docs.dunescience.org/cgi-bin/ShowDocument?docid=328}{FACILITY ACCESS SPECIFICATIONS}\cite{bib:docdb328}
\fixme{reference does not seem to work}
. This document defines the limitations on dimensions and weights for all materials to be transported underground.  The most important limitations which are described in detail in the specification document are related to the Ross shaft and Ross cage. It is possible to bring material down the shaft outside the cage as a slung load but this is a much slower process and requires careful planning, detailed procedures and review. The DUNE APA for example need this special handling as they are too tall to fit in the cage. Most material should be brought underground in the cage. Figure \ref{fig:fdsp-tc-Cage} shows an image of the new Ross cage and Table \ref{tab:table-Ross-Cage} summarizes its parameters. As a comparison the round trip travel time for the Ross cage is 17 minutes where most the time needed to load and unload the cage and any slung load will take over an hour round trip as both the loading/unloading and travel times are longer. 

Many other factors need to be considered when planning the DUNE logistics. There is no loading dock at the Ross headframe so all materials will need transported using a flatbed or curtain-sided truck. The equipment can then be removed using a forklift at SURF. In general one should plan on loading the trucks at the logistics warehouse in the same manner as they will be unloaded at SURF to ensure that there will be no unexpected difficulties. The CF-CMGC has to coordinate all the loads through the Ross shaft and to do this they will need to know 2 weeks in advance what is planned for DUNE in order to make the overall hoist schedule. It will also be forbidden for DUNE collaborators to ship equipment directly to the Ross Shaft unless this is coordinated by the DUNE logistics team. A central inventory/shipping system will be needed to monitor the shipments of goods to the warehouse in South Dakota. This system needs to be such that all DUNE institutions can enter the shipping data and the logistics team can monitor the progress. In ProtoDUNE-SP delays in shipping and customs caused up to 3 weeks delay in arrival of some parts which caused significant re-planning of the installation work. In order to prevent this from being a much larger problem in DUNE a one month buffer of materials will be planned. With this the underground work can be planned well in advance knowing all the materials will be available. This will require that sufficient space be available in the warehouse and underground at SURF to house the material buffer. Many small parcels will arrive at the warehouse and these will need to be consolidated into larger boxes/crates to make efficient use of the hoist. 




\begin{dunefigure}[Material for Setup]{fig:fdsp-tc-setup}
  {Image showing the cavern on end opposite of the detector. During the installation setup phase half the space will be used for the cryostat work and half as storage for the detector infrastructure. The material outside the cavern must be stored in the logistics warehouse.}
\includegraphics[width=.9\textwidth]{Material-Setup}
\end{dunefigure}
%

In order to understand how much space is needed for storage and how much of the hoist time must be dedicated to DUNE a detailed inventory of all the detector equipment and DUNE infrastructure is needed. The list of all the materials was solicited from all the consortia and technical coordination. The entries in the inventory spreadsheet are organized as "Loads" for the Ross shaft where a load is a crate or set of boxes which will either be transported underground in the hoist or as a slung load. \href{http://docs.dunescience.org/cgi-bin/ShowDocument?docid=8426}{DUNE load Spreadsheet} \cite{bib:docdb8426}
\fixme{reference does not seem to work}
 Information captured in the Load spreadsheet includes the number of hoist trips, nature of the trip (slung load or cage), the package dimensions, weight and type of package (crate, pallet, pallet box). The load list at present accounts for 1,600 hoist trips and accounts for roughly 2 months of cage time most of which is spread over one year. The installation operation for the single-phase detector will span over 2 years so it is prudent to divide the logistics planning in several phases. The load information is divided into the CUC setup phase, the installation setup phase, and the detector installation phase. In each phase a model was generated to show how much material can be stored underground outside the work area and also how much material needs stored on the surface. These models are used to set the space requirements for the logistics effort on the surface. The phase with the largest amount of material to transport is the detector setup phase and the model of the underground area and the required boxes for surface storage for the first $1/3$\ of the setup is shown in Figure \cite{bib:docdb8426}. This represents the first month of installation setup and shows that roughly 1,000 m$^2$\ of warehouse space will be needed for DUNE at this time. In addition to the space needed for receive and ship equipment underground the warehouse will also need space to store up to 150 APA. This adds an additional 700 m$^2$\ to the needed warehouse area. 


%%%%%%%%%%%%%%%%%%%%%%%%%%%%
\subsection{Quality Assurance and Control}
\label{sec:fdsp-tc-itf-qaqc}

ProtoDUNE-SP lessons learned:

1) Lack of a central inventory system made it impossible to track shipments.

2) Delays in shipping meant that the installation work could not be planned and parts were installed as they arrived. 

Effective inventory management will be essential for all aspects of \dword{dune} detector development, construction, installation and operation.  While its relevance and importance go beyond the \dword{itf}, the \dword{itf} is the location at which \dword{lbnf}, \dword{dune} project management, consortium scientific personnel and \surf operations will interface.  We therefore will develop standards and protocols for inventory management as part of the \dword{itf} planning.  A critical requirement for the project is that the inventory management system for procurement, construction and installation be compatible with future \dword{qa}, calibration and detector performance database systems.  Experience with past large detector projects, notably \nova, has demonstrated that the capability to track component-specific information is extremely valuable throughout installation, testing, commissioning and routine operation. Compatibility between separate inventory management and physics information systems will be maintained for effective operation and analysis of \dword{dune} data.

The logistics facility will ensure that all normal materials to be shipped to the Ross headframe will fit in the cage. If a slung load is needed the facility will confirm that the necessary procedures are in place and approved before any material is transported to SURF.

%%%%%%%%%%%%%%%%%%%%%%%%%%%%
\subsection{Safety}
\label{sec:fdsp-tc-itf-safety}

The LBNF/DUNE Logistics Facility is operated as a Fermilab Facility, but because of the international presence, we also follow CERN HSE, Fermilab ES\&H and SURF ES\&H regulations.  Work is in progress to combine all of this into a coherent list of codes and requirements to follow. The \dword{dune} Project ES\&H Coordinator has overall ES\&H oversight responsibility for the DUNE Project.  This person coordinates any activities and facilitates the resolution of any issues that cut across various Divisions and institutions. It is subject to the requirements of the DOE Workers Safety and Health Program, Title 10, Code Federal Regulations (CRF) Part 851 (10 CFR 851). These requirements are promulgated through the Fermilab Directors Policy Manual and Fermilab ES\&H manual (FESHM) which align with the SURF ES\&H Manual. 
Using the NOvA Far Detector Lab as a guideline for remote facilities there are several other key documents that guide the Logistics Center Safety Program.  The Building Safety Plan combines all of the building specific documents in a single folder:

\begin{enumerate}
\item	Fire Safety and Building Emergency Evacuation Plan- Fire evacuation plan, fire safety plan and lockdown plans, site plan
\item	Hazard Analysis- Describes all the typical hazards and their mediation including procedures 
\item	SDS- Safety Data Sheets
\item	Respiratory Plan- As required due to chemical or ODH hazards
\item	Training Program- Covers certifications required and  training records
\end{enumerate}

The current Technical Coordination facilities management plan has a joint safety officer used between the ITF and Logistics facility. This safety officer would facilitate training, write Hazard Analysis documents, run the weekly safety meetings, and keep documentation records on materials handling equipment and personnel. 


%%%%%%%%%%%%%%%%%%%%%%%%%%%%
\subsection{Cost, Schedule and Risk Analysis}
\label{sec:fdsp-tc-itf-cost}

\fixme{use templates from cost-risk-sched.tex file. Anne}
