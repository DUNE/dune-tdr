%\fixme{SG/JM done! Needs updated chart from David replacing CE to CRP.}
The calibration consortium was formed in November 2018 as a joint \dword{sp} and \dword{dp} consortium, with a consortium leader and a technical leader. Figure~\ref{fig:orgchart} shows the organization. % of the consortium. 
The calibration consortium board currently comprises institutional representatives from 11 institutions, as shown in Table~\ref{tab:gen-calib-org}. The consortium leader is the spokesperson for the consortium and is responsible for the overall scientific program and managing the group. The technical leader %of the consortium 
manages the project for the group. 

The calibration consortium's initial mandate is the design and prototyping of a laser calibration system, a neutron generator, and possibly a radioactive source system. The consortium is therefore organized into three working groups, each dedicated to one system. Each group has a designated working group leader.
%The \dword{tdr} editors are responsible for the overall editing and delivery of the \dword{tdr} document.

\begin{dunefigure}[Organizational chart for the calibration consortium]{fig:orgchart}
{Organization chart for the calibration consortium.}
\includegraphics[height=3.0in]{graphics/orgchart-dp-v2.png}
\end{dunefigure}

\begin{dunetable}
[Calibration Consortium Institutions]
{p{0.7\textwidth}c}
{tab:gen-calib-org}
{Current Calibration Consortium Board Institutional Members and Countries.}
Member Institute     &  Country       \\
Laboratorio de Instrumentacao e Fisica Experimental de Particulas (LIP) & Portugal \\ \colhline
University of Bern (Bern) & Switzerland \\ \colhline
Boston University (BU) & USA \\ \colhline
University of California, Davis (UC Davis)& USA \\ \colhline
Colorado State University (CSU) & USA \\ \colhline
University of Hawaii (Hawaii) & USA \\ \colhline
University of Iowa & USA \\ \colhline
Michigan State University (MSU) & USA \\ \colhline
University of Pittsburgh (Pitt) & USA \\ \colhline
South Dakota School of Mines and Technology (SDSMT) & USA \\ \colhline
Los Alamos National Lab (LANL) & USA \\ 
\end{dunetable}

In addition, Figure~\ref{fig:orgchart} shows several liaison roles  currently being established to facilitate connections with other groups and activities:
\begin{itemize}
    \item detector integration and installation,
    \item electrical and safety issues,
    \item \dword{daq},
    \item computing,
    \item cryogenic instrumentation and slow controls (CISC),
    %\item Cold electronics,
    \item \dword{crp},
    \item \dword{hv}, and 
    \item photon detection system.
\end{itemize}

%\fixme{KM adjusted above from "planned" to state will exist by Summer 2019}

Currently, new institutions are added to the consortium following an expression of interest from the institution and upon obtaining consensus from the current consortium board members.

%if the consortium board members agree, new members are added from other institutions if the new institutions express an interest through a petition to the board.