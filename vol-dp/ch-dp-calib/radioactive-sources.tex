%%%%%%%%%%%%%%%%%%%%%%%%%%%%%%%

%\fixme{KM: done for now, but am not satisfied with the end of it-- Juergen did not provide measurements! SG/JM: please briefly check to make sure it is coherent? SG: done. Sent an email to Juergen with some items to be addressed.}

%\subsubsection{Physics Motivation}

Radioactive source deployment provides an in situ source of physics signals at a known location and with a known activity that can be chosen to produce only one calibration event per drift time window. The baseline source design probes de-excitation products (gamma-rays) that are directly relevant for detecting supernova neutrinos and \isotope{B}{8}/hep solar neutrinos. The \dword{rsds} is the only calibration system that can probe the capability to detect single, isolated solar neutrino events and study how well radiological backgrounds can be suppressed. The trigger efficiency could be studied %versus 
as a function of threshold. 

%KMTDRREADME: I think this paragraph is too long
Secondary measurements from the baseline source deployment include electro-magnetic \fixme{"EM" is already used as an abbreviation for emergency management. I am deleting this abbreviation here.} shower characterization for long-baseline $\nu_e$ CC events; electron lifetime and \efield as a function of \dword{detmodule} vertical position, i.e., along the drift direction);
%\todo{SG: the vertical position is now drift direction, so needs update}
individual light detector response; and determination of radiative components of the Michel electron energy spectrum from muon decay. Aside from the baseline \isotope{Ni}{58}(n,$\gamma$)\isotope{Ni}{59} \SI{9}{\MeV} gamma-ray source, other sources could be deployed with the same multi-purpose system like a \isotope{Ar}{40}($\alpha,\,\gamma$)\isotope{Ca}{44} gamma-ray source, a \isotope{Cf}{252}, and/or AmBe neutron source that probe the effect of various radiological backgrounds, like radon ($\alpha,\,\gamma$) or radiological neutrons, or simply measure the neutron tagging efficiency, useful for improved calorimetry of beam neutrino interactions. \fixme{The previous sentence is too long and complex. It should be broken into two sentences.} In contrast to the baseline \isotope{Ni}{58}(n,$\gamma$)\isotope{Ni}{59} source with \SI{9}{\MeV} gamma-rays, an \isotope{Ar}{40}($\alpha,\,\gamma$)\isotope{Ca}{44} source producing gamma-ray energies around 15~MeV could even be deployed outside the cryostat to probe the upper visible energy range and trigger efficiency for \isotope{B}{8}/hep solar neutrinos. 

Both the \dword{rsds} and the pulsed neutron source systems are needed to address the integrated response of the detector for low energy physics: 
the \dword{rsds} primarily for trigger efficiency and the \dword{pns} mostly for uniformity. Response in Ar may change rapidly as a function of photon energy because of underlying nuclear physics mechanisms. A combination of \SI{6}{\MeV} (direct neutron capture response), \SI{9}{\MeV} (peak visible $\gamma$-energy of interest to \dword{snb} and \isotope{B}{8}/hep solar neutrinos), \SI{15}{\MeV} (upper visible energy of \isotope{B}{8}/hep solar neutrinos), and decay electrons ($\sim$\SI{30}{\MeV}) is needed to map the low energy response. In terms of complementarity, radioactive sources provide a known position, known-energy single photon events that could be triggered, while the pulsed neutron source provides a simple, potentially non-invasive design with an externally triggered multi-photon energy signature that is visible across the entire detector with a known time signature.


%%%%%%%%%%%%%%%%
\subsubsection{Design Considerations}

A composite source would comprise \isotope{Cf}{252}, a strong neutron emitter, and \isotope{Ni}{58}, which, via the \isotope{Ni}{58}(n,$\gamma$)\isotope{Ni}{59} process, converts one of the $^{252}$Cf fission neutrons, suitably moderated, to a monoenergetic \SI{9}{\MeV} photon~\cite{Rogers:1996ks}. 
The source would be inside a cylindrical moderator with a mass of approximately \SI{15}{kg} and a diameter of \SI{20}{\cm}, so it can be deployed via the proposed calibration ports discussed in Section~\ref{sec:calib-ports}. The activity of the radioactive source is chosen so no more than one \SI{9}{\MeV} capture $\gamma$-event occurs during a single drift period
%\todo{SG: does this still apply now that the drift window is much longer?}.
This is the main requirement for this system because it allows using the arrival time of the measured light as a $t0$ and then measures the average drift time of the corresponding charge signal(s). Table~\ref{tab:fdgen-calib-all-reqs-rsds} lists the full set of requirements for the \dword{rsds}.

The sources would be deployed outside the \dword{fc} within the cryostat to avoid regions with high \efield{}s, approximately \SI{30}{\cm} from the \dword{fc}. The $\gamma$-ray would need to travel two attenuation lengths (including the \SI{10}{\cm} radius of the source body). Such high $\gamma$-energies are typically only achieved by thermal neutron capture, which invokes a neutron source surrounded by a large amount of moderator, thus making such an externally deployed (n, $\gamma$) source \SI{20}{\cm}  to \SI{50}{\cm} % large
in diameter. 

\begin{dunetable}
[Full specifications for the radioactive source deployment system]
{p{0.45\linewidth}p{0.25\linewidth}p{0.25\linewidth}}
{tab:fdgen-calib-all-reqs-rsds}
{Full list of specifications for radioactive source deployment system.}   
Quantity/Parameter	& Specification	& Goal		 \\ \toprowrule   
%\textbf{Proposed Radioactive Source System}	   &   &  \\ \colhline  
Distance of the source from the field cage & 30 cm & \\ \colhline
Rate of 9~MeV capture $\gamma$-events inside the source {(\it top-level requirement)} & < 1kHz & \\ \colhline 
Data volume per 10~kton & 120~TB/year & 240~TB/year \\ \colhline 
Longevity	& \dunelifetime			& > \dunelifetime   \\ \colhline    
%Stability & Match precision requirement at all places/times	&  \\ \colhline  Reliability	& Measurements as needed & Measurements as needed \\ \colhline

\end{dunetable}

In J.~G.~Rogers et al., 
%Ref.~\cite{Rogers:1996ks}, \fixme{This needs to be restated: In Jones [4]... (with Jones being the name of the author).}
%\cite{Triumf:Nickelsource} 
%\todo{SG: reference needs fixing},
an $^{58}$Ni (n,$\gamma$) source, triggered by an AmBe neutron source,
was successfully built, yielding high $\gamma$-energies of \SI{9}{\MeV}. \dword{dune} %We
proposes to use a $^{252}$Cf (or AmLi as backup) neutron source with lower
neutron energies, which requires less than half of the surrounding
moderator, making the $^{58}$Ni (n, $\gamma$) source \SI{20}{\cm} or less in diameter. A cryostat feedthrough at either end of the cryostat (east and west sides) outside the \dword{fc} with an outer diameter of \SI{25}{\cm} will be suitable for this. The exact locations of ports is being explored.
%The multipurpose instrumentation feedthroughs at either end of the cryostat are sufficient for this, and have an inner diameter of \SI{25}{\cm}.  
The moderator material chosen for \dword{dune} is Delrin, which has enough
density to avoid flotation. Further, the end caps of the source
body are round to avoid distorting the \efield and to
eliminate the risk of the source becoming stuck during deployment. 
%JM 19May2019: I commented out the figure and text below since it's for SP only; SG: put it back in since it serves as a nice example for both SP and DP.
Figure~\ref{fig:RadioactiveSource_zm40cm_xp220cm} depicts the baseline source design of a cylindrical Delrin moderator \SI{20}{\cm} in diameter and \SI{40}{\cm} high, including half-spheres at either end with
radius of \SI{10}{\cm}, deployed at $z$=\SI{40}{\cm} outside the \dword{fc} and leaving a gap of \SI{30}{\cm} from the \dword{fc}. Deployed along the full vertical height (at each feedthrough location) the 9~MeV $\gamma$-ray source can illuminate the full drift length from bottom cathode to \dword{crp}.

\begin{dunefigure}[Fish-line deployment scheme in DUNE for an encapsulated radioactive source]
{fig:RadioactiveSource_zm40cm_xp220cm}
{Fish-line deployment scheme in \dword{dune} for a radioactive source encapsulated inside a cylindrical Delrin moderator body \SI{20}{\cm} in diameter and \SI{40}{\cm} high, including half-spheres with a radius of \SI{10}{\cm} at either end. A $^{252}$Cf neutron source and a natural Ni target are sealed inside at the center. The fish-line is deployed \SI{40}{\cm} outside of the \dword{fc}  (depicted here for the \dword{sp} but also valid for \dword{dp}).
%and \SI{220}{\cm} away from the \dword{apa} (red plane).
}
\includegraphics[width=1.0\linewidth]{RadioactiveSource_zm40cm_xp220cm.jpg}
\end{dunefigure}

% Here it is, but I don't have privileges for the bibliography

%@Misc{Triumf:Nickelsource,
%  author =   {J. Rogers, M. Andreaco, C. Moisan},
%  title =    {A $7-9$\,MeV isotropic gamma ray source for detector testing},
%  howpublished = {TRIUMF TRI-PP-96-7},
%  month =    {Apr},
%  year =     {1996},
%}

%\fixme{KM: From Juergen but I think covered by above: The activity of the radioactive source is chosen such that no more than one \SI{9}{\MeV} capture $\gamma$-ray spills into the active \dword{lartpc} region during a single \SI{2.2}{\milli\s} drift period. This allows one to use the arrival time of the measured light as $t_{0}$ and then measure the average drift time of the corresponding charge signal(s). The resulting drift velocity in turn yields the electric field strength, averaged over the variations encountered during the drifting of the charge(s). This can be repeated for each single \SI{9}{\MeV} capture $\gamma$-event that occurs during a \SI{2.2}{\milli\s} drift period and where visible $\gamma$-energy is deposited inside the active volume of the TPC. Pile-up and data read-out considerations restrict the maximally permissible event rate to less than \SI{10}{\hertz} and in turn the \SI{9}{\MeV} capture $\gamma$-rate occurring inside the radioactive source body to less than \SI{140}{\hertz}, given a spill-in efficiency into the active \dword{lar} of about \num{7}\%.}

A multipurpose fish-line calibration
system successfully used
%~\cite{bib:deKerret2012} \todo{SG: Juergen to provide proper reference. JM: Juergen says there's no alternative public ref.} 
% JM, April 2018: We will not refer to this since the LBNC will not have access to it
%\fixme{need to add reference: "The Double Chooz Near Detector Technical Design Report" 
%Double Chooz Collaboration (H. de Kerret (APC, Paris) et al.) Oct 1, 2012. EDMS ID:I-028812 ; DocDB ID: 3403-V5 Pages 198 - 223
%(Anne can't find proper ref for this)}
 for the Double Chooz
reactor neutrino experiment has become available for \dword{dune} after
Double Chooz was decommissioned in 2018. The system can be
easily refitted for use in \dword{dune}. The system will be housed inside
a purge-box connected via a neck to a multipurpose
calibration feedthrough \fixme{I note that feedthrough is spelled out in this section of source code while in the rest of the manuscript an abbreviation was used.} with a closed gate valve on top of the
cryostat. Before deployment, the source will be gently cooled-down by blowing liquid argon boil-off onto it inside a sealed purge-box. After the source has reached 
%near liquid argon 
near-\dword{lar} temperature, the purge-box will be evacuated by a vacuum pump to remove any residual oxygen and nitrogen, which is monitored at the ppm level. Then, the entire purge-box interior is purged with boil-off liquid argon, and the pressure equalized with the gas pressure inside the detector. The gate-valve is then opened and deployment can begin. This procedure ensures that no significant impurities are introduced into the detector during deployment and that no significant amount of liquid argon is boiled off from the detector. 
%Also, if the source is in close proximity of an \dword{apa} wire frame, lower energetic radiological backgrounds become problematic as the source light and charge yield is reduced exponentially with distance. 

%Deployed near mid-drift (in each TPC module) the \SI{9}{\MeV}
%$\gamma$-ray source can illuminate the full drift length from
%\dword{apa} to \dword{cpa}. 
The sources are retrieved from the
detector after each deployment and stored outside the cryostat following approved safety protocols, and the gate-valves are kept closed after deployment. More details on radiation safety and handling procedures are in Section~\ref{sec:sp-calib-rsds-safety}.

\subsubsection{Development Plan}
The major development plans for the \dword{rsds} include
\begin{itemize}
\item Continued development of relevant simulation tools, including geometry representation of the source deployment system and effects of various radiological contaminants on detector response; 
\item Studies to suppress radiological backgrounds for the calibration source;
\item Simulation studies to understand data and trigger rates;
\item A baseline design source with Delrin moderator, $^{252}$Cf neutron source, and natural nickel target, both sealed inside at the moderator's center;
\item Validation of \SI{9}{\MeV} capture $\gamma$-ray yield of source using spectroscopic measurements with the RABBIT germanium detector at the South Dakota School of Mines and Technology, which has an assay chamber large enough for the bulky moderator; 
%\fixme{Check this please. SG: check what?}
\item Validation with $^{3}$He based hodoscope at South Dakota School of Mines and Technology to ensure that the flux of neutrons escaping the moderator is not an issue; otherwise use lower energetic AmLi neutron source instead and/or more moderator material, and/or different geometric configuration of nickel target; 
\item Test gentle GAr cooling of source and validate material integrity; measure tensile strength of braided SS-304 wire-rope at cryogenic temperatures and ensure a safety factor of one order of magnitude by adjusting number of steel braids and their diameters; validate cryogenic shrinkage of sectional teflon sleeves that enclose the braided steel wire-rope and electrically insulate it from the \dword{fc}; 
\item Validation that anticipated fluid flow in \dword{lar} does not cause oscillations of the source; otherwise design vertical guide wires to be pre-installed during detector installation 
%and that they 
that will keep source stable during deployment along the vertical (drift) axis;
%\item A mechanical test of the Double Chooz fish-line deployment system with both an \dword{lar} and liquid nitrogen mock-up column in the high bay lab at South Dakota School of Mines and Technology. The ultimate test of the system will be done at ProtoDUNE. 
\item Explore other radioactive sources beyond the baseline $^{58}$Ni(n,$\gamma$)$^{59}$Ni 9 MeV gamma-ray source, such as a non-invasive external $^{40}$Ar($\alpha,\,\gamma$)$^{44}$Ca 15~MeV gamma-ray source with $^{241}$Am like that currently being assembled at South Dakota School of Mines and Technology and that could, from outside the cryostat, probe the upper visible energy range and trigger efficiency of $^{8}B$/hep solar neutrinos. Furthermore, investigate hybrid neutron sources ($^{252}$Cf and AmBe) that emulate the kinetic neutron energy spectrum of radiological neutrons and probe neutron tagging efficiency, which is useful for improving the calorimetry of beam neutrino interactions and to clarify radiological neutron backgrounds. 
\end{itemize}

%%%%%%%%%%%%%%%%%%%%%%%%%%%%
\subsubsection{Measurement Program}
\label{sec:sp-calib-sys-src-dep-meas}
%new text from Juergen on DP
The proposed baseline 9 MeV single $\gamma$ source could also be used to test the $\gamma$ aspect of the \dword{snb} and $^{8}B$/hep solar neutrino signal 
along the full vertical drift, but only in the \endwall regions of the detector because the detector does not have sufficient space on the long sides of the detector. 
The source may also be used to determine the relative charge and light extraction efficiency in the vertical direction for measuring energy resolution and energy scale. The electron lifetime and the \efield strength can be directly measured with vertical deployments that are not far from the \dwords{pmt} at the bottom, so the instant light signal can still be detected and used as t$_0$. Because of the very long vertical drift in the \dword{dp} detector, the \dword{rsds} should provide the most unambiguous measurement of electron lifetime in the \dword{dp} detector.

%SP text, commented out
%The proposed baseline 9~MeV single $\gamma$ source may also be used to test the $\gamma$ aspect of the \dword{snb} and $^{8}B$/hep solar neutrino signal along the full drift but only in the endwall regions of the detector. The source may also be used to determine the relative charge and light extraction efficiency in the vertical direction for measurements of energy resolution and energy scale. 

%SG: this is repetition
%The \dword{rsds} is the only calibration system that could probe the detection capability for single isolated solar neutrino events and study how well radiological backgrounds can be suppressed. 
%Figure~\ref{fig:rsds-fig1}
%{fig:9MeVgamma_withBG_LArSoft_v08_14_00_hiLY_originCHARGEcollected_TopView}
%depicts in a top view of the detector the simulated charge extraction efficiency for the baseline $^{58}$Ni(n,$\gamma$)$^{59}$Ni 9~MeV $\gamma$-ray source deployed \SI{40}{\cm} outside of the \dword{fc}, near mid-drift i.e., \SI{220}{\cm} away from the \dword{apa} in the $x$ direction, in the presence of expected background before (Fig.~\ref{fig:rsds-fig1}(a)) and after (Fig.~\ref{fig:rsds-fig1}(b)) applying selection cuts. The selection cuts discard each collection wire hit with less than 90~ADCU, limit wire hits to be within the width of the outer \dword{apa}, use induction wire hits to discard collection wire hits occurring further than \SI{2.2}{\m} away from the vertical deployment height, and requiring at least one occurrence per \SI{2.2}{\ms} drift-time window of a simultaneously extracted number of photoelectrons of more than 40~pes (within a \SI{50}{\ns} time window).
%Figure~\ref{fig:rsds-fig1}(b)
%Figure~\ref{fig:9MeVgamma_withBG_LArSoft_v08_14_00_hiLY_originCHARGEcollected_TopView} (b) 
%demonstrates that the selection cuts can reject charge from radiological backgrounds almost entirely, such that almost pure 9~MeV gamma-ray events are selected. Thus, the baseline \dword{rsds} would allow the trigger efficiency for isolated solar neutrino events to be studied, and even its dependence on the applied threshold measured. 

%SG: already stated under physics motivation, no need to repeat
%Moreover, the small electronic pulses from 9~MeV gamma-ray induced charge and light collections can be utilized to better characterize electro-magnetic (EM) showers for long-baseline $\nu_e$ CC events, as well as to determine the radiative components of the Michel electron energy spectrum from muon decays. 
%Further secondary measurements from the baseline 9 MeV gamma-ray source deployment include measuring the electron lifetime and electric field as a function of \dword{detmodule} vertical position, as well as the individual light detector response. 

%Figure~\ref{fig:rsds-fig2}
%{9MeVgamma_withBG_LArSoft_v08_14_00_hiLY_DriftTimesEfield_ChargeSpectrumElectronLifetime_cut} 
%shows exemplary simulated baseline \dword{rsds} measurements of the electric field strength (Fig.~\ref{fig:rsds-fig2}(a)) and of the electron lifetime (Fig.~\ref{fig:rsds-fig2}(b)), each for three different scenarios. The time difference between an extracted number of photoelectrons of more than 40~pes and the recorded hit times on collection wires, passing the selection cuts, defines the plotted drift-times. Precise quantities can be extracted from fitting the measured drift-time distribution to the simulation. For this, the simulated collection wire charge (before selection cuts) is re-weighted with respect to 
%(a) 
%the electric field strength, and 
%(b) 
%the electron lifetime, respectively. After each re-weighting of the simulation the selection cuts are then applied. Figure~\ref{fig:rsds-fig2}(a)
%{9MeVgamma_withBG_LArSoft_v08_14_00_hiLY_DriftTimesEfield_ChargeSpectrumElectronLifetime_cut} (a) 
%illustrates that with this method the electric field strength could be measured at $\sim 1\%$ precision at each vertical deployment position at the endwalls. Likewise, %Figure~\ref{fig:rsds-fig2}(b)
%Figure~\ref{9MeVgamma_withBG_LArSoft_v08_14_00_hiLY_DriftTimesEfield_ChargeSpectrumElectronLifetime_cut} (b) 
%illustrates that the electron lifetime could be measured at few $\%$ precision at each vertical deployment position at the endwalls. 

%Figure~\ref{fig:rsds-fig2}(c)
%{9MeVgamma_withBG_LArSoft_v08_14_00_hiLY_DriftTimesEfield_ChargeSpectrumElectronLifetime_cut} (c) 
%illustrates that with a recorded charge spectrum (after selection cuts) it is not convincingly possible to unambiguously measure both the electron lifetime and the electric field strength. Both parameters for most part simply shift the upper falling edge of the charge spectrum further up or down. However, at each vertical \dword{rsds} deployment position at the endwalls, it is still possible to self-consistently measure both the electron lifetime and the electric field strength at $\sim 1\%$ precision: First, the electric field strength is precisely inferred from the drift-time distribution at each deployment position, second the corresponding electron lifetime is then precisely deduced from the charge distribution by using the measured electric field strength as precise constraint. 

%Figure~\ref{fig:rsds-fig2plus} visualizes the simulated change of individual light detector response as a function of \dword{detmodule} vertical position when the baseline source is deployed at one of the possible endwall locations. The uniformity of the optical detection response can be probed with a full vertical scan. Wrong channel mapping, dead or noisy optical detectors could be easily identified. 

Aside from the baseline $^{58}$Ni(n,$\gamma$)$^{59}$Ni 9 MeV $\gamma$-ray source, other sources could be deployed with the same multi-purpose system, such as an $^{40}$Ar($\alpha,\,\gamma$)$^{44}$Ca gamma-ray source, a $^{252}$Cf and/or AmBe neutron source that probes the effects of various radiological backgrounds like radon ($\alpha,\,\gamma$) or radiological neutrons, or simply measures neutron tagging efficiency, useful for improving calorimetry of beam neutrino interactions. 

In contrast to the baseline $^{58}$Ni(n,$\gamma$)$^{59}$Ni source with 9~MeV $\gamma$-rays, an $^{40}Ar(\alpha,\,\gamma)^{44}$Ca source producing $\gamma$-ray energies around 15~MeV could even be deployed outside the cryostat and probe the upper visible energy range and trigger efficiency for $^{8}B$/hep solar neutrinos. 

%Figure~\ref{fig:rsds-fig3}
%{5MeValphaSource_zx_TDR_TopView} 
%shows in a top view of the detector the widely spread interaction locations of about half a dozen $^{40}Ar(\alpha,\,\gamma)^{44}$Ca 15 MeV $\gamma$-ray events in terms of detected light (Fig.~\ref{fig:rsds-fig3}(a)) and  detected charge (Fig.~\ref{fig:rsds-fig3}(b)). These simulation plots depict that 15~MeV $\gamma$-rays can travel \SI{7}{\m} in liquid argon, i.e., about half the width and half the height of the detector. It is therefore possible to deploy such a $^{40}Ar(\alpha,\,\gamma)^{44}$Ca 15~MeV $\gamma$-ray source conveniently and non-invasive from the outside of the cryostat, from each side and from the top (assuming that underneath the cryostat there is no accessible space). This would allow for calibrating large parts of the full detector with this source. 

An external \dword{protodune} deployment can show the feasibility of a non-invasive $^{40}Ar(\alpha,\,\-\gamma)^{44}$Ca 15~MeV $\gamma$-ray source despite the lack of overburden to shield cosmic rays. In contrast to cosmic muons, 15~MeV $\gamma$-ray induced hit clusters will start inside the detector volume and are not tracks that begin at the detector edges. Thus, the \dword{rsds} calibration events could be easily selected and the detected charge can be analyzed. The detected light, however, will be obscured by the high light level in each drift period from cosmic muons hitting \dword{protodune}. 

%\begin{dunefigure}[Detected charge from 9 MeV gamma-ray source with radiological backgrounds]
%{fig:9MeVgamma_withBG_LArSoft_v08_14_00_hiLY_originCHARGEcollected_TopView}
%{fig:rsds-fig1}
%{In LArSoft backtracked origins of detected charge (a) without cuts and (b) with selection cuts for a simulated 9~MeV $\gamma$-ray source deployed at $z=$\SI{-40}{\cm} outside of the \dword{fc}, $x=$\SI{220}{\cm} away from the \dword{apa}, and $y=$\SI{300}{\cm} half-height of an upper endwall \dword{apa} with simulated expected radiological background, that gets almost eliminated by selection cuts.}
%\centering
%   (a)
%   \includegraphics[width=1.0\linewidth]{9MeVgamma_withBG_LArSoft_v08_14_00_hiLY_originCHARGEcollected_TopView_TDR.png}
%   (b)
%   \includegraphics[width=1.0\linewidth]{9MeVgamma_withBG_LArSoft_v08_14_00_hiLY_originCHARGEcollected_TopView_cut_TDR.png}

%    \subfigure[a]{\includegraphics[width=1.0\linewidth]{9MeVgamma_withBG_LArSoft_v08_14_00_hiLY_originCHARGEcollected_TopView_TDR.png}}
%    \subfigure[b]{\includegraphics[width=1.0\linewidth]{9MeVgamma_withBG_LArSoft_v08_14_00_hiLY_originCHARGEcollected_TopView_cut_TDR.png}}
%\end{dunefigure}

%\begin{dunefigure}[Measurements of electric field strength and electron lifetime from 9 MeV gamma-ray source with radiological backgrounds]
%{fig:rsds-fig2}
%{fig:9MeVgamma_withBG_LArSoft_v08_14_00_hiLY_DriftTimesEfield_ChargeSpectrumElectronLifetime_cut}
%{Simulated measurements of (a) electric field strength from drift-time distribution, (b) electron lifetime from drift-time distribution, and (c) electron lifetime from charge distribution when electric field is unambiguously known from drift-time distribution. All spectra were created with applied selection cuts for a simulated 9~MeV $\gamma$-ray source with radiological backgrounds deployed at $z=$\SI{-40}{\cm} outside of the \dword{fc}, $x=$\SI{220}{\cm} away from the \dword{apa}, and $y=$\SI{300}{\cm} half-height of an upper endwall \dword{apa}. (Colors of histograms are matching colors of corresponding labels in each histogram.)}
%\centering
%    (a)
%    \includegraphics[width=0.45\linewidth]{9MeVgamma_withBG_LArSoft_v08_14_00_hiLY_EfieldDriftTimes_cut_TDR.png}
%    (b)
%    \includegraphics[width=0.45\linewidth]{9MeVgamma_withBG_LArSoft_v08_14_00_hiLY_eLifeTimeDriftTimes_cut_TDR.png}
%    (c)\includegraphics[width=1.0\linewidth]{9MeVgamma_withBG_LArSoft_v08_14_00_hiLY_chargeSpectrum_eLifeTimeEfield_cut_TDR.png}
    
%        \subfigure[a]{\includegraphics[width=0.45\linewidth]{9MeVgamma_withBG_LArSoft_v08_14_00_hiLY_EfieldDriftTimes_cut_TDR.png}}
%    \subfigure[b]{\includegraphics[width=0.45\linewidth]{9MeVgamma_withBG_LArSoft_v08_14_00_hiLY_eLifeTimeDriftTimes_cut_TDR.png}}
%    \subfigure[c]{\includegraphics[width=1.0\linewidth]{9MeVgamma_withBG_LArSoft_v08_14_00_hiLY_chargeSpectrum_eLifeTimeEfield_cut_TDR.png}}
%\end{dunefigure}

%\begin{dunefigure}[Frequency of optical channel hits with a radioactive gamma-ray source with radiological backgrounds]
%{fig:rsds-fig2plus}
%{fig:5MeValphaSource_zx_TDR_TopView}
%{In LArSoft simulated change in frequency of optical channel hits with a simulated 9~MeV $\gamma$-ray source deployed at $z=$\SI{-40}{\cm} outside of the \dword{fc}, $x=$\SI{220}{\cm} away from the \dword{apa}, and $y=$\SI{300}{\cm} half-height of an upper endwall \dword{apa} with simulated expected radiological background.}
%  \includegraphics[width=0.6\linewidth]{ophit_opchannels_9MeVgammaSource_wBGs_LArSoft_v08_14_00_geo_v4_TDR.png}
%\end{dunefigure}


%\begin{dunefigure}[Detected light and charge from 15 MeV gamma-ray source without radiological backgrounds]
%{fig:rsds-fig3}
%{fig:5MeValphaSource_zx_TDR_TopView}
%{In LArSoft backtracked origins of detected (a) photoelectrons and (b) charge for a simulated $^{40}$Ar($\alpha,\,\gamma$)$^{44}$Ca 15~MeV $\gamma$-ray source deployed at $z=$\SI{-40}{\cm} outside of the \dword{fc}, $x=$\SI{220}{\cm} away from the \dword{apa}, and $y=$\SI{300}{\cm} half-height of an upper endwall \dword{apa} without simulated expected radiological background.}
%\centering
%   (a)
%   \includegraphics[width=0.455\linewidth]{5MeValphaSource_zx_pe_TDR.png}
%   (b)
%   \includegraphics[width=0.455\linewidth]{5MeValphaSource_zx_adc_TDR.png}
%\end{dunefigure}



\subsubsection{RSDS Design Validation}
\dword{pddp} provides the ultimate test to validate the design, operation, and performance of the system and decide whether to deploy this system for the \dword{fd}. The cosmic induced background rate at \dword{protodune} is too high at the surface to detect responses to the \dword{dune} gamma source; however, the known deployment location along the vertical drift will allow a clear measurement of the electron lifetime and will help separate background events from calibration events, unlike in the \dword{pdsp} detector. Additionally, tests of functionality, reliability, and safety of the mechanical deployment system are needed to show the source can be deployed and retrieved with no issues.

%The source can be deployed to test the detector response and analysis method. 
%a higher intensity source could be deployed to test the detector response and analysis method. However, tests of functionality, reliability, and safety of the mechanical deployment system are needed to show the source can be deployed and retrieved with no issues. While these aspects can be largely tested in \dword{pdsp}-II, there are many \dual-specific aspects (e.g., 12~m long drift, sampling along the drift direction) that can benefit from a validation in \dword{pddp}-II.

Table~\ref{tab:calib-rsds-sched} shows a schedule with the main steps for \dword{protodune2} deployment.

\begin{dunetable}
[Key Milestones for Commissioning RSDS in ProtoDUNE-DP-II.]
{p{0.65\textwidth}p{0.25\textwidth}}
{tab:calib-rsds-sched}
{Key milestones towards commissioning the \dword{rsds} in \dword{pddp}-II.}  
Milestone & Date (Month YYYY)   \\ \toprowrule
Baseline \dword{rsds} design validation & January 2021 \\ \colhline 
\dword{rsds} mock-up deployment test at SDSMT & March 2021 \\ \colhline 
\dword{rsds} Design review  & May 2021 \\ \colhline
\dword{rsds} Production readiness review (PRR) & July 2021 \\ \colhline
Start of module 0 \dword{rsds} component production for ProtoDUNE-DP-II & September 2021      \\ \colhline
End of module 0 \dword{rsds} component production for ProtoDUNE-DP-II &  February 2022    \\ \colhline
\textbf{Start of ProtoDUNE-DP-II installation} & \textbf{March 2022} \\ \colhline
Start of \dword{rsds} installation &  April 2022    \\ \colhline
\dword{rsds} demonstration test at ProtoDUNE-DP-II  & April 2023\\ \colhline
\end{dunetable}

\subsubsection{DAQ Requirements}
Section~\ref{sec:dp-calib-daqreq} provides an overall discussion of calibration and \dword{daq} interface. 

Here, the \dword{daq} requirements for the \dword{rsds} are discussed. The radioactive source will not be triggered by the \dword{mlt}, and the \isotope{Cf}{252} rate should be too high for a source tag to be useful.

%Rather, it will deliver a tag to the \dword{mlt} and that tag will include a time stamp that can be used by the \dword{mlt} to issue a trigger command to the \dword{fe} readout.  The trigger command will have a standard readout window size of \SI{16.4}{\milli\s}, but to keep data rates manageable, the command will only be send to \dword{fe} readout buffers that are expected to be illuminated by the source. The localization of trigger commands thus reduces the data volume by \num{150}, if only one %\dword{apa} 
%\dword{crp} is read out.

%\fixme{SG: In the previous sentence, I nominally changed APA to CRP but the data volume reduction for DP might need to be updated.}
%\fixme{SG: the next paragraph needs updating for DP. \\ 
%JM: Also because the source isn't tagged and the text assumes it is.}

If the rate of such a source is anywhere close to one per \SI{16.4}{\milli\s}, the detector would run continuously in the current scheme. Therefore, we assume that the interaction rate in the detector is \SI{10}{\hertz} or less. 

%The tag from the source will likely be much higher than this, because not all $\gamma$s interact in the active \dword{tpc} volume. Thus the radioactive source trigger will be a coincidence in the \dword{mlt} between a low-energy trigger candidate from the illuminated %\dword{apa}
%\dword{crp}, and a source tag with a relevant time stamp.  

With this rate, and with localization of events to one %\dword{apa}
\dword{crp} (\dpchpercrp channels), the total data volume would be

\begin{equation}
\num{8}~{\rm hours} \times \num{4}~{\rm FTs} \times \SI{10}{\hertz} \times \num{1.5}~{\rm Bytes}\times \SI{2}{\mega\hertz}\times \SI{16.4}{\milli\s}\times \num{1920}~{\rm channels} = \num{114}~{\rm TB/scan}.
\end{equation}

%SP
%\begin{equation}
%\num{8}~{\rm hours} \times \num{4}~{\rm FTs} \times \SI{10}{\hertz} \times \num{1.5}~{\rm Bytes}\times \SI{2}{\mega\hertz}\times \SI{5.4}{\milli\s}\times \num{2560}~{\rm channels} = \num{50}~{\rm TB/scan}.
%\end{equation}

\fixme{Table 1.13 is not referenced in the text. It should be.}

\begin{dunetable}
[Calibration DAQ summary for RSDS]
{p{0.2\textwidth}p{0.15\textwidth}p{0.5\textwidth}}
{tab:calib-daq-rsds}
{Estimated uncompressed \dword{daq} data volume per year per \SI{10}{\kt} for the radioactive source system.}   
System & Data Volume (TB/year) & Assumptions  \\ \toprowrule
Proposed Radioactive Source System & \num{240} & Source rate < \SI{10}{\hertz}; single %\dword{apa} 
\dword{crp} readout,  lossless readout; \num{2} times/year   \\ 
\end{dunetable}   

 Running this calibration twice per year would yield \num{240}~TB of data in \SI{10}{\kt} per year. By selecting only a fraction of the drift time range, it should be possible to reduce this estimate, or by increasing the calibration frequency, the total data volume can be kept. 

\subsubsection{Risks}
The risks associated with the radioactive source system are described in Table~\ref{tab:risks:DP-FD-CAL-RSDS} along with appropriate mitigation strategies and the effect (low, medium or high risk levels) on probability, cost, and schedule post-mitigation.

%KMTDRREADME: We need to add the following risk for SP and DP. A thin conducting metal guidewire will penetrate the liquid/gas interface. Does this pose any safety or stability issues? How does this situation change as the source is 'dunked'. Is there any evidence iof surface charge collection?


%\fixme{uncomment when risk table available}

% risk table values for subsystem DP-FD-CAL-RSDS
\begin{footnotesize}
%\begin{longtable}{p{0.18\textwidth}p{0.20\textwidth}p{0.32\textwidth}p{0.02\textwidth}p{0.02\textwidth}p{0.02\textwidth}}
\begin{longtable}{P{0.18\textwidth}P{0.20\textwidth}P{0.32\textwidth}P{0.02\textwidth}P{0.02\textwidth}P{0.02\textwidth}} 
\caption[DP radioactive source calibration system risks]{Risks for DP-FD-CAL-RSDS (P=probability, C=cost, S=schedule) More information at \dshort{riskprob}. \fixmehl{ref \texttt{tab:risks:DP-FD-CAL-RSDS}}} \\
\rowcolor{dunesky}
ID & Risk & Mitigation & P & C & S  \\  \colhline
RT-DP-CAL-10 & Radioactive source swings into detector elements & Constrain the system with guide-wires & L & L & L \\  \colhline
RT-DP-CAL-11 & Radioactivity leak & Obtain rigorous source certification under high pressure and cryogenic temperatures & L & L & M \\  \colhline
RT-DP-CAL-12 & Source stuck or lost & Safe engineering margins, stronger fish-line and a torque limit in deployment system & L & M & L \\  \colhline
RT-DP-CAL-13 & Oxygen and nitrogen contamination & Leak checks before deployments & L & M & M \\  \colhline
RT-DP-CAL-14 & Light leak into the detector through purge-box & Light-tight purge box with an infrared camera for visual checks & L & L & L \\  \colhline
RT-DP-CAL-15 & Activation of the cryostat insulation & Activation studies and simulations & L & L & L \\  \colhline

\label{tab:risks:DP-FD-CAL-RSDS}
\end{longtable}
\end{footnotesize}


\begin{comment}
%SG: Old table of risks for RSDS.
\begin{dunetable}
[Calibration risks3]
{p{0.03\linewidth}p{0.4\linewidth}p{0.05\linewidth}p{0.4\linewidth}}
{tab:fdgen-calib-risks3}
{Possible risk scenarios for the radioactive source system along with mitigation strategies. The level of risk is indicated by letters ``H'', ``M'', and ``L'' corresponding to high, medium and low level risks.}   
No. & Risk  & Risk Level & Mitigation Strategy  \\ \toprowrule

10 & The deployed radioactive source can potentially swing into detector elements if not controlled or if large currents exist in the \dword{lar} & M & Guide-wires mitigate this risk.\\ \colhline

11 & Radioactivity could leak into the detector during a deployment. & L & Rigorous source certification under large pressure and cryogenic temperatures mitigates this risk.\\ \colhline

12 & The source could get stuck or lost in the detector. & L & Fish-line an order of magnitude stronger than needed to hold the weight, round edges of the moderator and a torque limit of the stepper motor will mitigate this risk.\\ \colhline

13 & Oxygen and nitrogen could get into the \dword{lar} in case the purge-box has a small leak. & M & Leak checks before deployments, purge-box in under-pressure inside w/r to the detector, will mitigate this risk.\\ \colhline

14 & Light could couple into the detector. & M &
Light-tight purge-box, internally equipped with an infra-red camera for visual checks will mitigate this risk.\\ \colhline

15 & The source activity can activate the cryostat insulation. & L & Detailed simulations/activation measurements can say what is a tolerable activity and the source activity can be chosen to be below that. \\ \colhline
\end{dunetable}
\end{comment}

\subsubsection{Installation, Integration, and Commissioning}
Before the \dword{tpc} elements are installed, the first elements of the radioactive source guide system are installed on the \endwall farthest from the \dword{tco}; it is the last system installed
%concurrent and coordinated with the alternative laser system (if any deployed), 
before the \dword{tpc} is installed and before closing the \dword{tco}. The radioactive source deployment system is installed at the top of the cryostat and can be installed when \dword{dune} becomes operational.

The commissioning plan for the source deployment system will include a dummy source deployment (within 2 months of the commissioning) followed by first real source deployment (within 3 to 4 months of the commissioning) and a second real source deployment (within 6 months of the commissioning). Assuming stable detector conditions, the radioactive source will be deployed every six months. Ideally, a deployment should occur before and after a run period, so at least two data points are available for calibration. This also provides a check on whether the state of the detector
has changed before and after the physics data run.
%If stability fluctuates for any reason (e.g., electronic response changes over time) at a particular location, one would want to deploy the source at that location once a month, or more often, depending on how bad the stability is.
It is estimated that it will take a few hours (e.g., 8 hours) to deploy the system at one feedthrough location and a full radioactive source calibration campaign might take %at least 
a week.

%\newpage
\subsubsection{Quality Control}
A mechanical test of the Double Chooz fish-line deployment system with a \dword{lar} mock-up column will be done in the high bay laboratory at South Dakota School of Mines and Technology. The ultimate test of the system will be done at \dword{protodune}. Safety checks will also be done for the source and for appropriate storage on the surface and underground. 

\subsubsection{Safety}
\label{sec:sp-calib-rsds-safety}
A composite source is used for the radioactive source system that consists  of \isotope{Cf}{252}, a strong neutron emitter, and \isotope{Ni}{58}, which, via the \isotope{Ni}{58}(n,$\gamma$)\isotope{Ni}{59} process, converts one of the \isotope{Cf}{252} fission neutrons, suitably moderated, to a monoenergetic \SI{9}{\MeV} gamma. This system also poses a radiation risk, which will be mitigated with a purge-box for handling and a shielded storage box and an area with lockout-tagout procedures, also applied to the gate-valve on top of the cryostat. Material safety data sheets will be submitted to \dword{dune} \dword{esh}, and specific procedures will be developed for storing and handling sources to meet FRCM \fixme{FRCM is not in the common glossary. Should it be there?} requirements. These procedures will be reviewed and approved by \dword{surf} and \dword{fnal} radiation safety officers. Deployed sources will be checked monthly to ensure they are not leaking. A designated shielded storage area will be assigned for sources, and proper handling procedures will be reviewed periodically. A custodian will be assigned to each shielded source.
