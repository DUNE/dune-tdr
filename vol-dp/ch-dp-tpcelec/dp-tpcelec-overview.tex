%%%%%%%%%%%%%%%%%%%%%%%%%%%%%%%%%
%converted dpele to dc-tpcelec
\section{System Overview}
\label{sec:dp-tpcelec-overview}

%%%%%%%%%%%%%%%%%%%%%%%%%%%%%%%%%
\subsection{Introduction}
\label{ssec:dp-tpcelec-intro}


The function of the \dword{dp} \dword{tpc} electronics system is to collect and digitize the signals from the \dwords{crp} (see chapter~\ref{ch:dp-crp}) and \dwords{pds} (see chapter~\ref{ch:dp-pds}). The first task is done by the \dword{cro} and the second by the \dword{lro} sub-systems. The design of the system incorporates components already developed for \dword{pddp} through R\&D begun in 2006. One of the key objectives of that R\&D program has been designing an electronics system that is easily scalable and cost-effective to meet the needs of the large-scale neutrino \dword{lar} detector. \fixme{This first paragraph should serve as the introduction to the section (1.1, System Overview) and should be placed after the main heading. A new heading title will be necessary for 1.1.1.}

While a single \dword{dpmod} has \num{20} more readout (both charge and light) channels than \dword{pddp}, simply scaling up the number of the components used in the prototype design is sufficient without any additional R\&D. A small-scale version of the \dword{tpc} electronics system was used in the \dword{wa105} at CERN, a preliminary \dual \lartpc prototype with an active volume  of \SI[product-units=power]{3x1x1}{m} (\dword{crp} area of \SI[product-units=power]{3x1}{m}) that took data in the summer-fall of 2017. 
Operation of the \dword{wa105} validated the design choices and provided checks on various performance markers like noise. 

\begin{dunefigure}[Schematic layout of the \dword{dp} \dword{cro} sub-system]{fig:dp-tpcelec-crosystem-sketch}
{Schematic layout of the \dword{dp} \dword{cro} sub-system.}
\includegraphics[width=0.6\textwidth]{dp-tpcelec-crosystem-sketch}
\end{dunefigure}

The \dword{cro} electronics system, illustrated in the Figure~\ref{fig:dp-tpcelec-crosystem-sketch}, is designed to provide continuous, non-zero-suppressed, and losslessly compressed digital signals by reading the charge collected on the \SI{3}{m} long \dword{crp} strips arranged in two collection views, all with a pitch of \SI{3.125}{mm}. The system consists of the \dword{fe} analog electronics operating at cryogenic temperatures that amplifies and shapes the signals from \dword{crp} strips and the digital electronics working in the warm environment outside the cryostat that digitizes the analog signals, compresses the resulting digitized data, and transmits them to the \dword{daq} system.  The cryogenic \dshort{fe} analog electronics uses an application-specific integrated circuit (\dword{asic}) chip with a large dynamic range (up to \SI{1200}{fC}) to cope with the charge amplification in the \dwords{crp}. The analog \dword{fe} cards are housed in dedicated \dwords{sftchimney} accessible from the outside even after the \dword{dpmod} is operating, thus removing any significant risks associated with their long-term survivability. The \dwords{sftchimney} are approximately \SI{2.3}{m} long stainless steel pipes that traverse the entire insulation layer of the cryostat allowing placement of the \dword{fe} electronics close to the \dwords{crp} to minimize cable capacitance (noise).  In addition, their metallic structure shields the \dword{fe} cards from any interference from the warm digital electronics and ambient environment. The analog signals are digitized by \dwords{amc}, which are housed in the commercial \dword{utca} crates on top of the cryostat near the \dwords{sftchimney}. 

The \dword{cro} data are sampled at the rate of \SI{2.5}{MHz} with \SI{12}{bit} resolution. This frequency, traditionally used in \lartpc experiments, is well matched to the \SI{1}{\micro\second} pulse-shaping time of the \dword{fe} electronics and the detector response times determined by the electron drift velocity in the \lar. The corresponding sampling resolution along the drift coordinate is better than \SI{1}{\mm}. 

The \dword{lro} electronics system, illustrated in the Figure~\ref{fig:dp-tpcelec-lrosystem-sketch},  collects and digitizes signals from the \dword{pds}, which consists of \dword{tpb}-coated \num{8}\,in \dwords{pmt} (Hamamatsu\footnote{Hamamatsu\texttrademark R5912-02-mod, \url{http://www.hamamatsu.com/}.} R5912-02-mod) beneath the \dword{tpc} cathode. The \dword{lro} electronics facilitate detecting the primary scintillation signals, which provide the absolute time reference for interaction events. The same electronics also enable recording light signals generated by photons from the so-called \textit{proportional scintillation component}, the light created by the electrons extracted and amplified in the gaseous phase. A fraction of this light can reach the \dwords{pmt} after traversing the entire detector volume. The \dword{lro} electronics, consisting of analog and digital stages, is housed in the \dword{utca} crates on top of the cryostat structure (like the \dword{cro} electronics). The \dword{lro} \dword{amc} card design shares a similar architecture with the \dwords{amc} for the charge readout. The \dword{lro} \dword{utca} crates are connected to specific \dword{lro} signal \fdth flanges on top of the cryostat (see chapter~\ref{ch:dp-pd}).

\begin{dunefigure}[Schematic layout of the \dword{dp} \dword{lro} sub-system]{fig:dp-tpcelec-lrosystem-sketch}
{Schematic layout of the \dword{dp} \dword{lro} sub-system.}
\includegraphics[width=0.6\textwidth]{dp-tpcelec-lrosystem-sketch}
\end{dunefigure}

Each \dword{utca} crate for either charge or light readout is connected to the \dword{daq} system via an optical fiber link that supports at least \SI{10}{Gbit/s}. Every crate also contains a \dword{wrmch} for time synchronization of the digital electronics. This timing slave unit is connected via \SI{1}{Gbit/s} optical fiber to a master node that serves as a synchronization reference for all connected slave nodes on the network. This system for time synchronization is built around the commercially available components from the \dword{wr} project\footnote{\url{https://www.ohwr.org/projects/white-rabbit}.} with ad-hoc hardware and firmware development. The system performs automatic and continuous self-calibrations to account for any propagation delays and can provide sub-\si{\nano\s} accuracy for timing synchronization.



%%%%%%%%%%%%%%%%%%%%%%%%%%%%%%%%%
\subsection{Design Considerations}
\label{ssec:dp-tpcelec-requir}

The \dword{cro} electronics design covers the analog \dword{fe} cards containing pre-amplifier \dwords{asic} operating at cryogenic temperatures and digitization cards with the relevant system for their synchronization working in the warm environment outside of the cryostat. The system reads and digitizes signals from \num{153600} channels (per one \dword{dpmod}) and can continuously stream the collected and losslessly compressed data to the \dword{daq} without any zero suppression. 

The designed \dshort{cro} electronics system fits the following requirements:
\fixme{to be replaced with a standardized table of specifications}
\begin{itemize}
\item{The \dword{cro} electronics must measure signals up to \SI{1200}{\femto\coulomb} without saturation; this has been optimized following \dword{mc} studies on the maximal occupancy per channel in shower events~\cite{WA105-TDR}. For a nominal \dword{crp} gain of \num{20}, a \dword{mip} signal should be approximately \SI{30}{fC}, the lowest limit that assumes a particle travelling horizontally with an azimuthal angle of \num{0} degrees, giving a maximal operational range of \num{40} \dwords{mip}.}

\item{The electronic noise in the \dshort{cro} analog electronics must be \SI{< 2500}{e^{-}}. This condition can be derived from the requirement on the minimal \dword{s/n}, which should be more than \num{5}:\num{1} once the charge attenuation is taken into account. Given the maximum drift distance of \SI{12}{\meter}, the largest attenuation factor due to electro-negative impurities, assuming the \SI{3}{\milli\second} (minimally required) electron lifetime and the drift field of \SI{0.5}{\kilo\volt/\cm}, is \num{0.08}. The smallest \dword{mip} signal with the \dword{crp} effective gain of \num{20} is, therefore, \SI{2.5}{\femto\coulomb} (\SI{15600}{e^{-}}).}

\item{The peaking time of the \dword{fe} analog amplifiers must be \SI{1}{\micro\second} because optimal vertex resolution is necessary. This resolution is determined in turn by the single track resolution and the power to separate two or more tracks close to one another.}

\item{The sampling frequency must be \SI{2.5}{\MHz} to match the peaking time of the \dword{fe} electronics.}

\item{The power dissipated by the \dword{fe} analog electronics must be below \SI{50}{\milli\watt/channel} to minimize heat input to the cryostat volume.}

\item{The \dword{fe} analog electronics must be replaceable without contaminating the main \lar volume to guarantee the long-term reliability of the system.}

\item{The \dword{adc} resolution must be such that the noise is at the level of an \dword{adc} given a dynamic range wide enough to match the response of the \dword{fe} amplifier. This can be achieved with a \num{12} bit \dword{adc}.}

\item{The digital electronics placed outside the cryostat in the warm environment must be capable of standard industrial components and solutions to keep costs low and to benefit from technological evolution, e.g., higher network speeds.}
\end{itemize}
As described in subsequent sections, the performance of the final system is significantly better than many of the listed requirements. 

The magnitude of the noise also affects the quality of the lossless compression of the raw data. A compression factor of approximately \num{10} is achieved with an \rms noise level less than \SI{1}{\dword{adc}}. To give an idea of the compression efficiency dependency on the noise level, a compression factor of four is obtained with noise at approximately \SI{1.5}{\dword{adc}} counts. 

The primary objective of the \dword{lro} system is to detect signals from a minimum of one \phel on one \dword{pmt}, giving a precise timestamp that can be used in conjunction with the charge signals to determine the absolute event time ($T_0$). Precise measurements of \dword{lro} signal charge allow  continual monitoring of the \dword{pmt} gain at a single \phel level, as well as determining the number of photons in each scintillation event.  In addition, an \dword{adc}  continuously streams data, downsampled to \SI{400}{ns} as for the \dword{cro} signals,  which, among other things, allows the scintillation time profile to be measured. The \dword{lro} system also reads \num{20} channels from reference SiPM sensors from the \dword{pd} calibration system.


The cryogenic analog electronics for the \dword{cro} are housed in the dedicated \dwords{sftchimney}. Its design enables access to the \dword{fe} card so it can be replaced without contaminating the pure \lar in the main cryostat volume. The chimneys have a cooling system that can control the temperature around the \dword{fe} cards to roughly \SI{110}{\kelvin} to reach their optimal noise level, and that compensates for the heat input from the chimneys into the cryostat volume. 

The digital electronics for both charge and light readout is in the warm environment on the top of the cryostat support structure and is easily accessible. This removes any constraints associated with  accessibility and operation in cryogenic environments, allowing use of standard components and industrial solutions in the design. Digital electronics must be continuously and automatically synchronized to more than \SI{400}{\nano\s} to ensure the correct temporal alignment of the \dword{adc} samples from all of the readout channels. This is a minimal requirement dictated by a sampling rate of \SI{2.5}{\MHz}.  

\fixme{I just noted that the header in the PDF reads LIST OF TABLES.} Key parameters for the electronics system design are summarized in Table~\ref{tab:dp-tpcelec-physicsparams}. 

\begin{dunetable}
[Parameters for the TPC electronics system design]
{lr}
{tab:dp-tpcelec-physicsparams}
{Parameters for the  TPC electronics system design. The values given for the total number of channels and data rate are for one \dword{dpmod}.}   
Parameter & Value  \\ \toprowrule
  \dword{cro} channels    &  \num{153600}            \\ \colhline
  \dword{cro} continuous sampling rate & \SI{2.5}{\MHz}\\ \colhline
  \dword{cro} \dword{adc} resolution & \num{12}\,bit           \\ \colhline
  \dword{cro} data compression factor   & \num{10}    \\ \colhline 
  \dword{cro} data flow  & \num{430}\,Gibit/s          \\ \colhline 
  \dword{lro} channels       & \num{720}               \\ \colhline
  \dword{lro} continuous sampling rate & \SI{2.5}{\MHz} \\ \colhline
  \dword{lro} \dword{adc} resolution & \num{14}\,bit            \\ \colhline
  \dword{lro} data compression factor  & \num{1}       \\ \colhline
  \dword{lro} data flow   & \num{24}\,Gibit/s          \\ 
\end{dunetable}


%%%%%%%%%%%%%%%%%%%%%%%%%%%%%%%%%
\subsection{Scope}
\label{ssec:dp-tpcelec-scope}


The scope of the \dword{tpc} electronics system covers procurement, production, testing, validation, installation, and commissioning of all components necessary to ensure the complete readout of the charge and light signals from a given \dword{dpmod}. This includes 
\begin{itemize}
\item{Cryogenic analog \dword{fe} cards for charge readout;}
\item{\dword{amc} cards for charge and light readout;}
\item{The \dword{wrmch} cards for \dword{amc} clock synchronization;}
\item{\dword{utca} crates;}
\item{Switches for the \dword{wr} network;}
\item{\dwords{sftchimney};}
\item{Low-voltage power supplies, distribution, and filtering system for the \dword{fe} cards;}
\item{Flat cables connecting the \dword{fe} cards to the warm flange interface of the \dwords{sftchimney};}
\item{VHDCI cables connecting the warm flange interface of the \dwords{sftchimney} to \dwords{amc}.}
\end{itemize}

The total number of components to be procured to instrument one \dword{dpmod} are given in Table~\ref{tab:dp-tpcelec-num-components}.

\begin{dunetable}
[Numbers for \dual electronics components to procure]
{lr} {tab:dp-tpcelec-num-components}
{Numbers for \dual electronics components to procure for one \dword{dpmod}.}
Name & Number  \\ \toprowrule
\dword{cro} cryogenic \dwords{asic} (\num{16} ch) & \num{9600} \\ \colhline
\dword{cro} cryogenic analog \dword{fe} cards (\num{64} ch) & \num{2400} \\ \colhline
\dword{cro} \dwords{amc} & \num{2400} \\ \colhline
\dwords{sftchimney} & \num{240} \\ \colhline
Flat cables for \dword{sftchimney} (\num{68} ch) & \num{2400} \\ \colhline
Flat cables for \dword{sftchimney} (\num{80} ch) & \num{2400} \\ \colhline
VHDCI cables (\num{32} ch) & \num{4800} \\ \colhline
\dword{lro} \dwords{amc} with analog \dword{fe} & \num{45} \\ \colhline
\dword{utca} crates & \num{245} \\ \colhline
\dword{wrmch} units & \num{245} \\ \colhline
WR switches (\num{18} ports) & \num{16} \\ 
\end{dunetable}

