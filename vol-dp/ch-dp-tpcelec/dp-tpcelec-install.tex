%%%%%%%%%%%%%%%%%%%%%%%%%%%%%%%%%
\section{Installation, Integration, and Commissioning}
\label{sec:dp-tpcelec-install}
\fixme{some parts of this section should go to Installation chapter}

Installing the \dword{tpc} electronics systems requires several stages. To cable the \dwords{crp} to the \dwords{sftchimney}, the chimneys are installed first, before starting the \dword{crp} installation inside the cryostat. Next the \dword{fe} cards are mounted on the blades and inserted. The installation of the digital electronics and \dword{utca} crates is postponed until all the heavy work is done on top of the cryostat. This prevents damage to fragile components (like the optical fibers) due to movement of material and traffic. 
Once the \dword{utca} crates are installed and all the digital cards are inserted, the \dwords{amc} are cabled to the warm flanges of the \dwords{sft} for the charge readout and are connected to the \dword{pmt} signal cables for the light readout. Finally, to complete the installation and integrate the system with the \dword{daq}, the \SI{10}{Gbit/s} and \SI{1}{Gbit/s} optical links to the \dword{daq} and \dword{wr} timing network are connected. At this stage, the full system is ready for commissioning. 

%%%%%%%%%%%%%%%%%%%%%%%%%%%%%%%%%
\subsection{SFT Chimneys}
\label{sec:dp-tpcelec-install-sft}

Installing the \dwords{sftchimney} requires a compact gantry crane with movable supports along the length of the cryostat. The crane itself moves along the transverse direction. The crates containing the \dwords{sftchimney} are placed along the edges of the cryostat roof. An unpacked chimney is hoisted and transported to the appropriate penetration crossing pipe for installation. Once in place, the chimney is fastened to the flange on the crossing pipe. Enough overhead room to accommodate a chimney's \SI{2.4}{m} length is required to allow the crane to freely move the chimney along the direction transverse to the beam axis. 

In parallel with the \dword{sftchimney} installation, the \dword{fe} cards are unpacked on top of the cryostat and mounted on the blades before they are inserted in the chimneys. With \dwords{sftchimney} secured in the cryostat structure, the blades with mounted \dword{fe} cards are inserted before sealing the chimney. At this stage, the \lar and gas nitrogen delivery pipes are already installed, making it  possible to connect to the pipes. The pressure probes and temperature sensors are also connected to the slow control system.

%%%%%%%%%%%%%%%%%%%%%%%%%%%%%%%%%
\subsection{Digital $\mu$TCA Crates}
\label{ssec:dp-tpcelec-install-utca}

Installing the \dword{utca} crates with the digital electronics occurs in the final stage of the \dword{dpmod} to avoid damaging the fragile equipment. The crates are placed in their designated positions on the cryostat and connected to the power distribution network. The \dword{amc} cards and \dword{wrmch} modules are inserted into their slots. The VHDCI cables are then attached, connecting the \dword{cro} \dwords{amc} to the warm flange interface of the \dwords{sftchimney}.  The fibers from the timing system are connected to \dword{wrmch}. 

%%%%%%%%%%%%%%%%%%%%%%%%%%%%%%%%%
\subsection{Integration within the DAQ}
\label{ssec:dp-tpcelec-install-daq}

Integrating the \dual TPC electronics with the \dword{daq} system requires connecting the \SI{10}{Gbit/s} fiber links to each of the \num{245} \dword{utca} crates. Connecting the timing system to the synchronization \dword{wrgm} is done via a single \SI{1}{Gbit/s} fiber link. 

The necessary software for the \dword{daq} to read and decode the data packets sent by each \dword{utca} crate is provided by the electronics consortium.  

%%%%%%%%%%%%%%%%%%%%%%%%%%%%%%%%%
\subsection{Integration with the Photon Detection System}
\label{ssec:dp-tpcelec-install-pmt}

The cables carrying the \dword{pmt} signals from the splitter boxes are connected to the \dword{lro} analog electronics in each \dword{utca} crate. The positioning of the crates is optimized with respect to the layout of \dword{pmt} cables. In addition, the calibration system of the \dword{pds} is connected to specified inputs on the cards.


%%%%%%%%%%%%%%%%%%%%%%%%%%%%%%%%%
\subsection{Commissioning}
\label{ssec:dp-tpcelec-install-comission}

The \dwords{sftchimney} are commissioned as a first step. This consists of evacuating them \fixme{to vacuum?} and then filling them with nitrogen gas at slight overpressure. The leak rate must be checked when the chimney is under vacuum, and the nitrogen pressure must be monitored once the chimney is filled to verify that no damage occurred to the flange interfaces during installation.

The electronics system is commissioned after completing the installation of the \dword{utca} crates with the \dwords{amc}, and the timing system. \fixme{The previous sentence is not clear. Where does the timing system fit into this? Is it part of the commissioning or part of the installation of the crates and AMCs?} Functionality of the full \dword{daq} system is not strictly required at this stage. The data from each crate is read with a portable computer connected to the crate using either the \dword{mch} \SI{10}{Gbit/s} or \SI{1}{Gbit/s} interface. Non-functioning channels are identified by pulsing the \dword{crp} strips, and the data quality is examined to ensure the correct functioning of the digital electronics and the temporal alignment of the data segments.   
