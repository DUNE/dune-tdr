\section{Risks}
\label{sec:dp-tpcelec-risks}

\begin{dunetable}
[TPC Electronics System Risk Summary]
{p{0.15\textwidth}p{0.75\textwidth}}
{tab:dp-tpcelec-risks}
{TPC Electronics System Risk Summary.}
ID & Risk \\ \toprowrule
1 & Obsolescence of electronic components over the period of experiment \\ \colhline
2 & Modification to \dword{fe} electronics due to evolution in design of \dwords{pd} \\ \colhline
3 & Damage to electronics due to \dword{hv} discharges or other causes \\ \colhline
4 & Problems with \dword{fe} card extraction due to insufficient clearance \\ \colhline
5 & Overpressure in the \dwords{sftchimney} \\ \colhline
6 & Leak of nitrogen inside the \dword{dpmod} via cold flange \\ \colhline
7 & Data flow increase due to inefficient compression caused by higher noise \\ \colhline
8 & Damage to \dword{utca} crates due to presence of water on the roof of the cryostat \\ \colhline
9 & Clogging ventilation system of \dword{utca} crates due to bad air quality \\ \colhline
\end{dunetable}

To mitigate their impact, the design of the \dual electronics system takes into account several risk factors, summarized in Table~\ref{tab:dp-tpcelec-risks}. The obsolescence of electronic components (ID 1) can be addressed by creating a sufficient stock of spare elements (preferably complete cards rather than components). The \dword{dp} \dword{tpc} electronics consortium will also monitor the market in the period prior to the system production for \dword{dune} to ensure any single component parts (e.g., \dword{fpga}) will not go out of production. Any elements identified as at risk will be replaced in the executive design with functionally equivalent parts. While this may necessitate some minor design adaptations, the overall functionality of the system will remain unchanged. Market survey will also be undertaken for the \dword{fe} \dword{asic} manufacturing process, so that, if a real risk is identified, the \dword{asic} production can be re-qualified for a different production process. Similarly a strict and timely follow-up of the \dword{fe} requirements from the \dual \dword{pds} should allow any potential modification of the \dword{lro} \dword{fe} electronics design to be addressed (ID 2).

The \dword{fe} cards should have suitable protection to prevent damage caused by \dword{crp} discharges (ID 3). TVS diodes used in the current design provide sufficient protection for the \dword{fe} electronics in \dword{wa105}. By design, the \dword{fe} cards remain accessible and could be replaced. However, the \dwords{sftchimney} require sufficient overhead clearance (ID 4) so that the blades holding the \dword{fe} cards can be extracted and re-inserted. This should be addressed by imposing an appropriate requirement for \dword{lbnf}.

The \dwords{sftchimney} are equipped with safety valves that vent excess gas if pressure suddenly rises (ID 5). The over-pressure threshold must be set low enough that the flanges suffer no significant damage. The pressure of the nitrogen inside the \dwords{sftchimney} should be monitored to detect potential leaks. In the event of a leak towards the inner volume of the \dword{dpmod} via the cold flange (ID 6), the chimney volume should be filled with argon gas to mitigate the effect nitrogen contamination carries for quenching the scintillation light.  

To ensure the continuous flow of digitized data to \dword{daq} (ID 7), the data rate must stay below the \dword{mch} bandwidth, which is a potential risk if the data compression becomes less efficient because of higher noise. To mitigate this risk, we currently have a margin (a factor of \num{5}) in the available bandwidth: \SI{10}{Gbit/s} \dword{mch} compared to the expected rate of \SI{1.8}{Gbit/s} (Table~\ref{tab:dp-utcabandwidth}).

Potential damage to the \dword{utca} crates due to water condensation on the roof of the cryostat (ID 8) or poor air quality (ID 9) should be mitigated by the \dword{lbnf} requirements that the top of the cryostat remain dry and the air quality similar to any industrial environment (e.g., at CERN or \dword{fermilab}). Large quantities of dust in the detector caverns at SURF should be avoided after the \dword{utca} crates are deployed.
