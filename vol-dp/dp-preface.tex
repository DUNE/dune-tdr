\cleardoublepage
\chapter*{Preface}

This document describes the design of a DUNE Far Detector module based on the dual-phase LArTPC technology. The document was initially drafted in 2017 with the intention of publishing it as a separate volume of the DUNE Far Detector Technical Design Report (TDR). An earlier, shorter version was made public in 2018 as ``The DUNE Far Detector Interim Design Report, Volume 3: Dual-Phase Module,'' (\url{https://arxiv.org/abs/1807.10340}). Editing was then completed in 2019 and the document was submitted to the TDR review process run by the Long Baseline Neutrino Committee (LBNC). 

Given the short timescale envisaged for the construction of the second DUNE Far Detector module, affecting both the R\&D activities and the review process, the LBNC recommended conversion to a technology that will be easier to implement. Consequently, the dual-phase Far Detector module design document did not complete the steps to be officially published as a DUNE TDR volume.
In the fall of 2020 the dual-phase design evolved into the Vertical Drift concept. The Vertical Drift incorporates many of the design features developed for the dual-phase, in particular, the use of perforated anode printed circuit boards. The new design replaces the Large Electron Multiplier (LEM) micro-pattern detectors operating in the argon gas phase with charge collection strips on the PCBs immersed in the liquid argon. 

We believe that the present document, in the state it had reached at the time of the LBNC review process, is of scientific and historical interest for the neutrino community, and therefore is important to publish. It provides the global, integrated view of the intensive R\&D and design efforts carried out over many years on this detector technology, which had been a candidate for implementation as a Far Detector module since DUNE's inception in 2015. A set of dedicated papers will document specific subdetector systems and the performance of the ProtoDUNE dual-phase detector at CERN. 

If the remaining R\&D activities needed to bring this technology to the level of maturity required for the construction phase can be completed, the dual-phase technology may still be a viable option for the third or fourth DUNE Far Detector module.
