\section{Risks}
\label{sec:dp-pds-risks}

Table \ref{tab:risks:DP-FD-PDS} summarizes the risks associated with the \dual \dword{pds}. Severity level assigned for each risk is indicated as L (low), M (medium), and H (high). Severity levels are split into three columns: probability for a risk to occur (P), and impact in terms of cost (C) and schedule (S) that the materialization of a risk would have. Below, we discuss these risks and mitigation plans for each risk item.


% risk table values for subsystem DP-FD-PDS
\begin{longtable}{p{0.18\textwidth}p{0.20\textwidth}p{0.32\textwidth}p{0.02\textwidth}p{0.02\textwidth}p{0.02\textwidth}} 
\caption{Risks for DP-FD-PDS \fixmehl{ref \texttt{tab:risks:DP-FD-PDS}}} \\
\rowcolor{dunesky}
ID & Risk & Mitigation & P & C & S  \\  \colhline
RT-DP-PDS-01 & Insufficient light yield due to inefficient PDS design & Increase coverage of \dword{pmt} photo-cathodes and/or \dword{wls} reflector foils. & L & M & L \\  \colhline
RT-DP-PDS-02 & Poor coating quality for \dword{tpb} coated surfaces and \dword{lar} contamination by \dword{tpb} & Test quality and ageing properties of \dword{tpb} coating techniques. Improve techniques if needed. & L & L & L \\  \colhline
RT-DP-PDS-03 & \dword{pmt} channel loss due to faulty \dword{pmt} base design & Optimize clustering algorithms. Improve \dword{pmt} base design from analysis of possible failure modes in \dword{pddp}. & L & L & L \\  \colhline
RT-DP-PDS-04 & Bad \dword{pmt} channel due to faulty connection between \dword{hv}/signal cable and \dword{pmt} base & Optimize clustering algorithms. Test connectivity in \lntwo prior to installation. & L & L & L \\  \colhline
RT-DP-PDS-05 & \dword{pmt} signal saturation & Tune \dword{pmt} gain. In worst case, redesign \dword{fe} to adjust to analog input range of \dword{adc}. & M & L & L \\  \colhline
RT-DP-PDS-06 & Excessive electronics noise to distinguish \dword{lar} scintillation light & Measure noise levels during commissioning prior to \lar filling. Modify grounding, shielding, or power distribution schemes. & M & L & L \\  \colhline
RT-DP-PDS-07 & Availability of resources for work at the installation/integration site less than planned & Bring consortium members to the integration/installation site. & L & L & L \\  \colhline
RT-DP-PDS-08 & Damage of \dwords{pmt} during shipment to the experiment site & Use special packaging to avoid damage during shipment. Plan for 10\% spare  \dwords{pmt}. & L & L & L \\  \colhline
RT-DP-PDS-09 & Damage of optical fibers during installation & Install fibers last. Provide detailed installation procedures.   & L & L & L \\  \colhline
RT-DP-PDS-10 & Excessive exposure to ambient light of \dword{tpb} coated surfaces, resulting in degraded performance & Cover \dword{tpb} coated surfaces until cryostat closing. Include in installation procedure: minimize exposure to ambient light. & L & L & L \\  \colhline
RT-DP-PDS-11 & \dword{pmt} implosion during \dword{lar} filling & No mitigation necessary, per \dword{pmt} \SI{7}{bar} pressure rating and experience with similar \dwords{pmt} in  large liquid detectors. & L & L & L \\  \colhline
RT-DP-PDS-12 & Insufficient light yield due to poor \dword{lar} purity & Procure \dword{lar} with less than 3 ppm nitrogen. & M & L & L \\  \colhline
RT-DP-PDS-13 & \dword{pmt} channel or \dword{pds} sector loss due to failures in \dword{hv}/signal rack & Access/replace components easily outside cryostat and provide spares for all components for at least one \dword{hv}/signal rack. & L & L & L \\  \colhline
RT-DP-PDS-14 & Unstable response of the photon detection system over the lifetime of the experiment & Correct channel-level instabilities via light calibration system. Correct detector-level instabilities via cosmic-ray muon calibration data. & L & L & L \\  \colhline

\label{tab:risks:DP-FD-PDS}
\end{longtable}


%%%%%%%%%%%%%%%%%%%%%%%%%%%%%%%%%%%%%%%%%%%%%%%%%%%%%%%%%%%%%%%%%%%%

\subsection{Design and Construction Risks}
\label{sec:dp-pds-risks_design}

\begin{itemize}

\item Because of the long drift distance and the position of the cathode and ground grid on top of the \dwords{pmt}, the number of photons detected by the \dwords{pmt} might not be sufficient at some geometrical acceptances (RT-DP-PDS-001). This is estimated as low-probability risk, given that the current \dword{pds} design has been optimized based on detailed simulations. The \dword{pds} baseline design includes \dword{wls} reflector foils precisely to address the risk of insufficient light levels, see Sec.~\ref{sec:dp-pds-simulation}. The detailed understanding of the light levels observed in the \dword{wa105} and particularly in \dword{pddp} will further mitigate this risk. The \dune \dword{fd} \dword{pds} design is essentially a large-scale replica of the recently installed \dword{pddp} \dword{pds}. The most notable exception is the \dword{wls} reflector foils, that is considered to be tested in a \dword{pddp} \dword{pds} upgrade. The largest variation in light yield occurs along the drift direction, where the extrapolation from \dword{pddp} size to the \dune \dword{fd} module size is only a factor of two. If higher \dword{pds} detection efficiency turns out to be necessary, the number of \dwords{pmt} may be increased and/or the surface area of \dword{wls} reflector foils may be increased.

\item Past experience shows that the \dword{tpb} coating might not be sufficiently stable (RT-DP-PDS-002). \dword{tpb} emanation from coated surfaces (\dword{pmt} photo-cathodes and \dword{wls} reflector foils) into the bulk \dword{lar} may cause \dword{wls} behaviour of the contaminated \dword{lar} bulk, or create a loss in optical performance of the coatings over time. This is estimated as medium-probability risk. Certain coating techniques have shown to emanate \dword{tpb} up to tens of parts per billion into argon by mass \cite{Asaadi:2018ixs}, although such impurity levels are not expected to be problematic. In addition, other \dword{lar} experiments using \dword{tpb} coatings, most notably DarkSide-50 \cite{Agnes:2018fwg} and DEAP-3600 \cite{Ajaj:2019imk}, have already shown percent-level stability in \dword{pds} light yields over timescales of 1--2 years. Nevertheless, different coating techniques vary greatly in terms of coating quality and stability. As risk mitigation measure, we plan to test the quality and ageing properties of the exact coating procedure to be followed for the \dword{pmt} photo-cathodes and \dword{wls} reflector foils in \dword{pddp} and in dedicated R\&D setups. We will elaborate improved coating techniques if needed.

\item An inadequate \dword{pmt} base design may result in dead \dword{pmt} channels during the long lifetime of the experiment (RT-DP-PDS-003). Such channel losses could not be recovered. This risk is estimated as low-probability. On the one hand, the loss of single \dword{pmt} channels is expected to have a minimal impact on detector performance. Current \dword{pmt} clustering algorithms (see Sec.~\ref{sec:dp-pds-performance}) already allow for non-responsive \dwords{pmt} along one row or column within one optical cluster. On the other hand, the \dword{pmt} base design is a mature and simple design already tested in the \dword{wa105} and to be tested in \dword{pddp}. \dword{pddp} operations will provide additional risk mitigation. If \dword{pddp} experience with \dword{pmt} bases proves to be unsatisfactory, design modifications will be introduced and tested.

\item \dword{pmt} cables are soldered to the \dword{pmt} base and tested before installation. If the soldering is poorly done, some channels could show a bad waveform or no signal at all (RT-DP-PDS-004). Such noisy or dead channels could not be repaired. This risk is estimated as low-probability. On the one hand, and as discussed above, the loss of single \dword{pmt} channels is expected to have a minimal impact on detector performance. On the other hand, the \dword{pmt} with final base plus soldered cable will be tested in \lntwo during the \dword{pmt} characterization tests, see Sec.~\ref{sec:dp-pds-selection-characterization}. In addition, several tests will be done in the base before the \dwords{pmt} are installed: impedance, \dword{hv} tests in gas Ar to avoid sparks, and full test of the \dword{pmt} to verify signals are correct.

\item Choosing the wrong feedthrough means we could have higher noise levels or oscillations in the signals (RT-DP-PDS-005). Using the wrong parts might cause mechanical problems. Special care must be taken during the design of the screws and \dword{hv} feedthroughs needed for cabling in the detector. \dword{pddp} experience will be used to mitigate this risk, estimated to be low-probability.

\item \dword{pmt} signal saturation at the front-end input (RT-DP-PDS-006) may occur as a result of operation at a higher-than-anticipated \dword{pmt} gain, for example to compensate for a poor signal-to-noise ratio. \dword{pmt} signals may also saturate because of an incorrect signal amplitude estimate. This risk is estimated to be medium-probability, also considering the highly non-uniform \dword{pds} spatial response. \dword{pmt} signal saturation would have minimal impact on \dword{pds} $t_0$ reconstruction and triggering capabilities. It would have more impact on \dword{pds} energy reconstruction capabilities, although some amount of saturation can be tolerated also in this case (see Sec.~\ref{sec:dp-pds-performance}). To mitigate this risk, we will avoid operating the \dwords{pmt} at very high voltages. A balance between gain and saturation will be chosen.

\item Excessive noise on \dword{pmt} waveforms could appear because of poor grounding design, insufficient cable shielding or noisy power distribution (RT-DP-PDS-007). The risk is estimated as medium-probability. This problem could be detected during \dword{pds} commissioning. Doing so prior to \dword{lar} filling is essential, in case necessary modifications to grounding, shielding, or power distribution  schemes affect components inside the cryostat.

\end{itemize}

%%%%%%%%%%%%%%%%%%%%%%%%%%%%%%%%%%%%%%%%%%%%%%%%%%%%%%%%%%%%%%%%%%%%

\subsection{Risks During Installation}
\label{sec:dp-pds-risks_installation}

\begin{itemize}

\item Availability of resources for work at the installation/integration site may be less than planned (RT-DP-PDS-008). This risk is estimated to be medium-probability. The \dword{fd} construction cost estimate assumes that qualified local labor can be identified for certain activities.  The cost will increase if external labor is required.  Labor costs might need to increase to attract qualified candidates. Labor resources from laboratories may need to be housed and used. We will ensure that sufficient funding is available to move people temporarily from institutions involved in the \dword{pds} consortium to the integration/installation site.

\item If \dwords{pmt} are not packaged properly, they could be damaged during shipment (RT-DP-PDS-009). In this case, the \dword{pmt} cannot be used in the detector. This risk is estimated to be low-probability. We will use special packaging to avoid possible damage to the \dwords{pmt} during shipment. In any case, we will have a \num{10}\% contingency spare \dwords{pmt}.

\item Fibers are fragile and could be broken during installation because of the high number of fibers in the detector (RT-DP-PDS-010). This risk is estimated to be low-probability. Personnel in charge of assembling the different parts of the detector must take special care during fiber installation.

\item \dword{tpb} coated surfaces (\dword{pmt} photo-cathodes and \dword{wls} reflector foils) are sensitive to ultraviolet light exposure, which may cause degraded optical performance (RT-DP-PDS-011), see for example \cite{Jones:2012hm}. This risk is estimated to be low-probability. \dword{tpb} coated surfaces will be covered with plastic bags or filters to avoid photo-degradation until closing of the cryostat. The installation procedure for \dwords{pmt} and \dword{wls} reflector foils will be defined in great detail, to minimize exposure to ambient light. 

\end{itemize}

%%%%%%%%%%%%%%%%%%%%%%%%%%%%%%%%%%%%%%%%%%%%%%%%%%%%%%%%%%%%%%%%%%%%

\subsection{Risks During Commissioning}
\label{sec:dp-pds-risks_commissioning}

\begin{itemize}

\item Filling the detector with \dword{lar} could become a critical issue in the occurrence of a \dword{pmt} implosion (RT-DP-PDS-012). If this happens, a chain reaction may develop, as in the Super-Kamkionande detector accident of 2001, destroying several other \dwords{pmt}. If a \dword{pmt} implosion occurs, the detector must be emptied, and the \dwords{pmt} would have to be reinforced or removed. This risk is estimated to be low-probability. The Hamamatsu R5912 \dwords{pmt} are rated for \SI{7}{bar} pressure and the hydrostatic pressure on the cryostat floor will be about \SI{2}{bar}. Similar 8-inch tubes, with same pressure rating, have been successfully used in other large detectors, such as MiniBooNE, SNO, ICARUS T600 and DEAP-3600, and in similar pressure conditions. In addition, a computational model to calculate the shock wave pressure resulting from tube implosion was developed for the MiniBooNE experiment, based upon the stored energy in an evacuated tube at a given depth \cite{Brice:2006ny}. It should be noted that the energy stored in an \SI{8}{inch} \dword{pmt} is over an order of magnitude less than in a \SI{20}{inch} Super-K \dword{pmt}. That study indicated a large safety factor against a chain reaction of \dword{pmt} implosions, and for the more densely packed MiniBooNE \dwords{pmt}.

\end{itemize}

%%%%%%%%%%%%%%%%%%%%%%%%%%%%%%%%%%%%%%%%%%%%%%%%%%%%%%%%%%%%%%%%%%%

\subsection{Risks During Operation}
\label{sec:dp-pds-risks_operation}

\begin{itemize}

\item A high level of \dword{lar} contamination may result in an unacceptably short absorption length for argon scintillation light, thus in an unacceptably low detected light yield despite a satisfactory \dword{pds} design (RT-DP-PDS-013). This risk is estimated as medium-probability. A particularly harmful contaminant, as far as light detection is concerned, is nitrogen. The simulation studies presented in this chapter assume an argon scintillation light absorption length of \SI{20}{m}, corresponding to about \SI{3}{ppm} of N$_2$ impurity concentration. Such argon purity specification has already been achieved in large \dword{lartpc} detectors.

\item The \dword{hv}/signal racks will contain the \dword{hv} crates, \dword{hv}/signal splitters, the \dword{utca} crates for the front-end electronics, and the calibration \dword{led} driver and the associated electronics for \num{36} \dwords{pmt}. Failure in some of these components may result in failure of a single \dword{pmt} channel or of an entire sector of \num{36} \dwords{pmt} (RT-DP-PDS-014). This risk is estimated as low-probability. The loss of an entire \dword{pmt} sector would have a major impact on detector performance. However, all of these components are located outside the cryostat and are easily replaceable. Spares will be available for all components of at least one \dword{hv}/signal rack.

\item The \dword{pds} response may vary over time (RT-DP-PDS-015) as a result of a variety of causes, such as time variations in the \dword{lar} purity, in the quality of \dword{tpb} coatings, or in the \dword{pmt} gains. This risk is estimated as low-probability. The quantum efficiency of the \dword{tpb}-coated \dwords{pmt} and their gain will be continuously monitored using the light calibration system (see Sec.~\ref{sec:dp-pds-calibration}). Once such channel-level time variations are corrected for, possible global changes in \dword{lar} purity or in \dword{wls} reflector foils optical response will be addressed using cosmic-ray muon calibration data. 

\end{itemize}
