\section{Mechanics}
\label{sec:dp-pds-mechanics}

%\subsection{\dshort{pmt} Support Structure}
%\label{subsec:dp-pds-mechanics-pmtsupport}

A uniform array of \dpnumpmtch cryogenic Hamamatsu R5912-MOD20 \dwords{pmt}, below the transparent cathode structure, will be fixed on the membrane floor in the areas between the membrane corrugations. The arrangement of the \dwords{pmt} accommodates the cryogenic piping on the membrane floor, and other elements installed in this area.

The mechanics for the attachment of the \dwords{pmt} were carefully studied. The \dword{pmt} buoyancy must be counteracted while avoiding stress to the \dword{pmt} glass due to differentials in the thermal contraction between the support and the \dword{pmt} itself. The attachment is done via a stainless steel support base point-glued to the membrane via four adhesive injection holes. The weight of the support and \dword{pmt} (approximately \SI{7}{\kg}) exceeds the buoyancy force of the system. Given the large standing surface of the stainless steel plate, these supports also ensure stability against possible lateral forces acting on the \dwords{pmt} due to the liquid flow. Figure \ref{fig:dppd_3_2} shows the \dword{pmt} with its support base attached to the bottom of the \dword{pddp} cryostat.

%\begin{dunefigure}[Cryogenic Hamamatsu R5912-MOD20 \dword{pmt} %fixed on the membrane floor.]{fig:dppd_3_2}
%{Cryogenic Hamamatsu R5912-MOD20 \dword{pmt} fixed on the membrane floor, with the optical fiber of the calibration system.}
%\includegraphics[width=0.42\textwidth]{dppd_3_2}
%\end{dunefigure}

\begin{dunefigure}[Picture of the Hamamatsu R5912-MOD20 \dword{pmt} fixed on the membrane floor of \dword{pddp}.]{fig:dppd_3_2}
{Picture of the cryogenic Hamamatsu R5912-MOD20 \dword{pmt} fixed on the membrane floor of \dword{pddp}. The optical fiber of the calibration system is also visible.}
\includegraphics[width=0.42\textwidth]{dppd_3_n2}
\end{dunefigure}

%\fixme{Update Fig.~\ref{fig:dppd_3_2} of PMT support structure.}

The \dword{hv} and \dual \dword{pds} consortia is studying an alternative to the table-like assembly of the ground grid: the concept of individual ground grids for each \dword{pmt}. In this option, the grids would be mechanically attached to the support structure. This will increase the acceptance of the \dwords{pmt} at larger angles.

The support frame structure mainly comprises \num{304}L stainless steel with some small Teflon (PTFE) pieces assembled by A4 stainless steel screws that minimize the mass while securing the \dword{pmt} support to the cryostat membrane. The design %was done taking 
takes into account the shrinking of the different materials during the cooling process to avoid breaking the \dword{pmt} glass.
Over-pressure tests were carried out for \dword{pddp}, and further tests ensure correct performance under pressure. An individual \dword{pmt} mount has been designed and tested in the  \dword{wa105} prototype~\cite{Zambelli:2017dkg}, and the same design is used for \dword{pddp}.

Installation of an individual ground grid mechanically attached to the \dword{pmt} support structure is also under consideration (See Appendix~\ref{sec:dp-pds-appendix-grid}).

%%%%%%%%%%%%%%%%%%%%%%%%%%%%%%%%%%%%%%%%%%%%%%%%%%%%%%%%%%%%%%%%%%%%

%\subsection{\dshort{wls} Reflector Foils}
%\label{subsec:dp-pds-mechanics-foils}

%\fixme{Add description here for foils on field cage walls for 2nd draft if included in baseline design. Otherwise, describe this in appendix.}