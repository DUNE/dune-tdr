\section{Photosensor System}
\label{sec:dp-pds-photosensors}

%%%%%%%%%%%%%%%%%%%%%%%%%%%%%%%%%
\subsection{Photodetector Selection and Procurement}
\label{sec:dp-pds-selection-procurement}

The baseline photodetector for the light readout is the Hamamatsu R5912-MOD20 \dword{pmt}. This is the same model as is used in \dword{pddp}. The Hamamatsu R5912-MOD20, depicted in  Fig.~\ref{fig:dppd_2_1}, is an 8-inch diameter, 14-stage, high gain \dword{pmt} (nominal gain of \num{e9}). The maximum quantum efficiency of the R5912-MOD20 \dword{pmt} is about \SI{20}{\%} at \SI{400}{\nano\m}. In addition, this \dword{pmt} was designed to work at cryogenic temperatures by adding a thin platinum layer between the photocathode and the borosilicate glass envelope to preserve the conductance of the photocathode at low temperatures. This particular \dword{pmt} has proven reliability in other cryogenic detectors. The same or similar \dwords{pmt} have been successfully operated in other \lar experiments like MicroBooNE~\cite{microboone}, MiniCLEAN \cite{miniclean}, ArDM, ICARUS T600 \cite{icarus}, as well as in \dword{pddp}~\cite{protoDUNDP-tdr}. Discussions with other manufacturers such as Electron Tubes Limited (UK) \cite{electrontubeslim} and HZC (China) \cite{hzc} are on-going to engage them in the program.

\begin{dunefigure}[Picture of the Hamamatsu R5912-MOD20 \dword{pmt}.]{fig:dppd_2_1}
{Picture of the Hamamatsu R5912-MOD20 \dword{pmt} \cite{hamamatsu-5912}.}
\includegraphics[width=0.3\textwidth]{dppd_2_1}
\end{dunefigure}

The baseline number of \dwords{pmt} is \dpnumpmtch plus \num{80} spares.  There are several operations and tests that have to be performed with \dwords{pmt} before the installation. The \dwords{pmt} have to be ordered with sufficient lead time to complete the following planned operations: assembly of the voltage divider circuit, mounting on the support structure, testing at room and cryogenic temperatures, packing and shipment to the \dword{itf}. The \dword{tpb} coating of the \dword{pmt} windows will be performed at the \dword{itf}. The \dwords{pmt} will be re-tested for validation of basic functionality at the \dword{itf} and at \surf before installation (see Section~\ref{sec:dp-pds-installation}). Considering the large number of \dwords{pmt} required by \dual \dword{pds}, the purchase order must be completed at least two years ahead of installation. A staged or staggered order with a steady supply of \dwords{pmt} would be most convenient, and will be negotiated with the manufacturer.

%%%%%%%%%%%%%%%%%%%%%%%%%%%%%%%%%
\subsection{Photodetector Characterization}
\label{sec:dp-pds-selection-characterization}

Prior to installation, the most important characteristics of the \dword{pmt} response have to be determined with two goals: to possibly reject under-performing \dwords{pmt}, and to store the characterization information in a database for later use during the \dword{dpmod} commissioning and operation.

The basic and most important parameters to characterize are the dark count rate versus high voltage and the gain versus high voltage. Both parameters must be measured at room and  cryogenic temperatures. Although with the baseline \dword{pmt} model, the rates of pre-pulsing and after-pulsing are expected to be negligible, they will be measured as part of the testing procedure. 

From the mechanical point of view, the test setup requires a light-tight dark vessel filled with a cryogenic liquid (argon or nitrogen) plus the infrastructure for filling and operating the vessel with temperature and liquid-level controls. For \dword{pddp}, \num{10} \dwords{pmt} were tested at a time over the course of a week as cryogenic tests of \dwords{pmt} require several days for the \dword{pmt} thermalization \cite{Belver:2018erf}. Figure~\ref{fig:dppd_2_2a} shows the  \dword{pddp} \dwords{pmt} being installed in the testing vessel.
Increasing the capacity of the vessel, and thus the number of \dwords{pmt} to test simultaneously,
could reduce the duration of the characterization test per \dword{pmt}.

\begin{dunefigure}[Picture of the \dwords{pmt} being installed in the testing vessel]{fig:dppd_2_2a}
{Picture of the \dwords{pmt} being installed in the testing vessel used for the \dword{pddp} \dwords{pmt}.}
\includegraphics[width=0.3\textwidth]{dppd_2_2a}
\end{dunefigure}

Figure~\ref{fig:dppd_2_2b} shows the sketch of the envisaged setup for \dword{pmt} characterization tests. From the electronics point of view, the test setup requires an \dword{hv} power supply, a discriminator, a counter for the dark rate measurements, a pulsed light source, and a charge-to-digital or analog-to-digital converter for the \dword{pmt} gain versus voltage measurements. All those instruments must allow computer control to automate the data acquisition.

\begin{dunefigure}[Sketch of the setup for \dword{pmt} characterization tests.]{fig:dppd_2_2b}
{Sketch of the setup for \dword{pmt} characterization tests.}
\includegraphics[width=0.7\textwidth]{dppd_2_2b}
\end{dunefigure}


%%%%%%%%%%%%%%%%%%%%%%%%%%%%%%%%%
\subsection{High Voltage System}
\label{sec:dp-pds-HV}

Based on the experience with the \dword{wa105} prototype, the A7030 power supply modules from CAEN~\footnote{CAEN\texttrademark{}, \url{http://www.caen.it/csite/CaenFlyer.jsp?parent=222}} are selected as the baseline power supply of the \dword{pmt} \dword{hv} system. 
These modules provide up to \SI{3}{kV} with a maximal output current of \SI{1}{mA} and a common floating ground to minimize the noise. Module versions with \num{12}, \num{24}, \num{36}, or \num{48} \dword{hv} channels are available. The \dword{hv} polarity can be chosen for each module. According to the baseline \dword{pmt} powering scheme, modules with positive \dword{hv} polarity will be acquired for the experiment. Modules with \num{36} \dword{hv} channels and Radiall \num{52}\footnote{Radiall\texttrademark{}, \url{https://www.radiall.com/}.}  connectors are under consideration. The corresponding \dword{hv} cable connects the modules with the \dword{hv} splitters, described in Section~\ref{sec:fddp-pd-4.2}. For \dpnumpmtch \dwords{pmt}, \num{20} A7030 modules (+ \num{2} spares) are needed. These \num{20} \dword{hv} modules will be installed in mainframes from CAEN.

Each \dword{pmt} is powered individually.  This allows the gain of all \dwords{pmt} to be set individually by adjusting their operating high voltage.
This will be sofware-controlled. The software will interface to the \dword{pmt} calibration system and its database to extract the gain curves needed for setting and/or equalizing gains.

%%%%%%%%%%%%%%%%%%%%%%%%%%%%%%%%%
\subsection{Wavelength Shifting}
\label{sec:dppd-wls}

The \dual \dword{pds} requires wavelength-shifting of the \SI{127}{nm} scintillation photons towards visible wavelengths that can overlap with the photocathode luminous sensitivity. Coating the \dword{pmt} glass bulbs over the photocathode area with a thin film of \dword{tpb}  has already been validated~\cite{tpb} and is adopted as the baseline plan. 

\dword{tpb} is a wavelength shifter that has high efficiency for converting \lar scintillation \dword{vuv} light towards the light for which the \dword{pmt} photocathode is more sensitive. 

A thin layer of \dword{tpb} is deposited on the \dword{pmt} glass by means of a thermal evaporator which consists of a vacuum chamber with two copper crucibles (Knudsen cells) placed at the bottom of the chamber, see Fig.~\ref{fig:dppd_11_4} in Section~\ref{sec:dp-pds-installation}. A \dword{pmt} is mounted on a rotating support that ensures a uniform coating layer. It is placed at the top of the evaporator with the  \dword{pmt} window pointing downwards. The crucibles, filled with the \dword{tpb}, are heated to \SI{220}{\degreeCelsius}. At this temperature, the \dword{tpb} evaporates through a split in the crucible lid into the vacuum chamber, eventually reaching the \dword{pmt} window.

Several tests were performed to tune evaporator's parameters, e.g., the coating thickness (\dword{tpb} surface density) and the deposition rate. A \dword{pmt} mock up covered with mylar foils was used for these tests. A \dword{tpb} surface density of \SI{0.2}{mg/cm^2} -- the value for which the \dword{pmt} efficiency is stable as a function of the surface density -- was chosen for \dword{pddp}. Efficiency measurements were performed using a \dword{vuv} monochromator and by comparing the cathode current of a coated \dword{pmt} with the current value of a calibrated photodiode. From these efficiency tests it was concluded that about \SI{0.8}{g} of \dword{tpb} must be placed in the crucibles for each evaporation to achieve the desired \dword{pmt} coating surface density. %The best deposition rate was fixed to about 6.5\,\AA/s. 
This value optimizes the quantity of \dword{tpb} used per evaporation while keeping the coating surface density fluctuations below \num{5}$\%$.  
Two to four \dwords{pmt}  with these specifications can be coated per day at a single coating station. 
Multiple coating stations will be required to maintain the installation and testing schedule (see Section~\ref{sec:dp-pds-installation} for details). An average qantum efficiency of \SI{12}{\%} at \SI{127}{\nano\m} has been measured for TPB-coated \dwords{pmt} \cite{Bonesini:2018ubd}.

In coordination with the \dword{hvs} consortium, the installation of wavelength shifting films on the inner surfaces of the field cage is under consideration. This option will increase both light yield and response uniformity and is routinely used in \dual \lartpc{}s searching for dark matter, such as the ArDM~\cite{Boccone:2009zz} experiment. It is also under investigation for the \dword{spmod} concept, building on the experience of the \lariat experiment, and the design for SBND. The same \dword{wls} compound used for coating of the \dword{pmt} windows, \dword{tpb}, could be vacuum-evaporated on foils. The shifted longer wavelength light emitted by the foils would have a greater chance of reaching the \dword{pmt} windows compared to \SI{127}{nm} light since it has better reflective properties. 
 %
 %This concept would need to be demonstrated satisfactorily for performance and stability on the timescale of the experiment duration before it could become a part of the \dword{pds} system.
