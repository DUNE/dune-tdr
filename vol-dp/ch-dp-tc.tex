\chapter{Detector installation}
\label{ch:dp-installation}

This chapter covers the work and the infrastructure required to install the \dword{dune} \dword{dp} \dword{fd} module.
The infrastructure and the installation sequence have been defined assuming that the cryostat of the \dword{dpmod} is installed in an empty cavern.
This allows for more flexible installation of the needed infrastructure.
If the \dword{dpmod} is the third module installed, a different installation procedure and infrastructure should be defined.
This is out of the scope of this chapter.

A \dword{dune} \dword{fd} module has outer cryostat dimensions of $65.8 \times 18.9 \times 17.8$~m$^3$.
Every piece of the cryostats and the detectors must descend about 1500~m down the Ross shaft to the 4850-foot level of \dword{surf} and be transported to the detector cavern.
The installation of the detector is done in close collaboration with \dword{lbnf}, that is in charge of  excavating the caverns and providing the cryostats for all the \dword{dune} modules.

Inside the cryostat the \dword{dp} detector consists of:
\begin{itemize}
\item 80 $3 \times 3$~m$^2$ \dwords{crp} installed across the liquid/vapour argon interface.
\item  12 $12 \times 12$~m$^2$ \dword{fc} modules installed vertically along the walls.
\item  15 cathode modules  installed parallel to the \dwords{crp} and hanging from the bottom of the \dword{fc}.
\item  720 \dwords{pmt} installed on the cryostat floor under the cathode and a protection ground grid.
\item  A set of instruments to optimally operate the \dword{tpc} (purity monitors, temperature sensors, level meters, cryogenic cameras).
\end{itemize}

Compared to the \dword{spmod}, the installation of the \dword{dpmod} is significantly simpler inside the cryostat and it requires less complex infrastructure.
The sequence of installation is driven by the tests of the \dwords{crp}.
The work on the \dword{dpmod} can profit from all the infrastructure already in place for the installation on the \dword{spmod}, like the \dword{cuc}, the underground mechanical workshop, the storage facility, the logistic, and so on.

Some of the detector components arrive at \dword{surf} fully/mostly assembled, like the \dwords{crp}.
Others are mostly assembled underground, like the field cage modules.
The majority of the activities happen in a clean room, set up in front of the \dword{tco} entrance and inside the cryostat.
The clean room is equipped with four cold boxes where the \dwords{crp} go through the final cryogenic and HV tests.

After about twelve months of detector component installation, which follows twelve months of detector infrastructure installation, the cryostat is closed (with the last installation steps occurring in a confined space accessed through access ports on the roof, called man-holes).
Following leak checks, final electrical connection tests, the process of filling the cryostat with 17~kton of \dword{lar} begins.
The installation of \dword{dpmod} requires meticulous planning and execution of all the tasks by well trained teams of technicians, riggers, and detector specialists.

The high level requirements for the \dword{dpmod} installation are summarised in the table~\ref{tab:requirements}.

%\begin{dunetable}[Installation requirements]
%{ccccc}
%&&&&\\

%&&&&\\
%{tab:requirements}
%{Installation requirements.}
%\end{dunetable}

In all the planning and future work, the preeminent requirement in the installation process is safety.
The installation of the \dword{dp} module presents a multitude of hazards that includes manipulation of heavy loads, work in tight areas and confined spaces, working at heights above the floor, use of chemicals (for instant as cleaning agents), use of cryogens, tests involving \dword{hv}.

Mitigation of these hazards begins with the strong review efforts and definition of specific safety rules by \dword{esh} teams of \dword{sdsd} and \dword{surf}.
All installation team members, both on surface and underground, will undergo rigorous formal safety training.
Any team member can stop work at any time for safety purposes.
Further details of the overall \dword{dune} safety plan are provided in the Technical Coordination volume of \dword{tdr}.
In addition, each section of this chapter provides further details on the evolving safety plan for the \dword{dp} module installation.


The high level risks associated to the \dword{dpmod} installation are summarised in the table~\ref{tab:risks}.

%\begin{dunetable}[Installation risks]
%{ccccc}
%&&&&\\
%&&&&\\
%{tab:risks}
%{Installation risks.}
%\end{dunetable}

Access to the underground installation area for \dword{lbnf}, \dword{dune}, and \dword{jpo} personnel, as well as for \dword{lbnf} and \dword{dune} materials and equipment, will be provided by a single shaft, the mile-deep Ross Shaft.
Coordinating transport and ensuring on-time delivery of all items are therefore among the more challenging aspects of the \dword{lbnf} and \dword{dune} endeavor.
The \dword{jpo} will establish a logistics organization for \dword{lbnf} and \dword{dune}, operated under the \dword{sdsd}, to verify deliveries to the point of receipt in South Dakota and coordinate transport of materials from there to the Ross Headframe.

Due to the cost of the \dword{cf} contracts and the risk of increased construction costs due to delays in delivery of materials, the shaft scheduling must be tightly controlled by \dword{lbnf}-\dword{cf} during construction.
The shaft is outfitted with a hoist that controls the cage and the skips.
The cage is used to transport people, equipment and materials, and the skips are used to bring up muck and transport over-sized equipment and materials.
The \dword{lbnf}-\dword{cf} \dword{cmgc} will coordinate overall usage of the Ross Shaft during this period.
To facilitate the flow of materials and equipment not related to \dword{cf} through the Ross Headframe, the \dword{jpo} will lease a warehouse facility within a maximum one-day roundtrip from \dword{surf} by truck.
It is expected that the lease of this facility, referred to as the \dword{sdwf}, will include warehouse space, personnel and \dword{wms} to inventory all incoming materials and equipment.
The location and the premisses  of the facility have not yet been identified.
Most detector and materials and equipment will be shipped to the \dword{sdwf}.
\dword{cf} material and cryogenics equipment are exceptions and will ship directly to \dword{surf}.

The \dword{sdsd} logistics organization will receive and inventory all goods shipped to the \dword{sdwf}, coordinate with the \dword{cmgc} and transport the material to the Ross Headframe in a just-in-time manner, and transport it underground and into the cavern.
The \dword{sdsd} logistics team oversees transportation of the cryostat (steel, foam, and membrane), the cryogenics system, the detector, and all related infrastructure not provided by the \dword{cf}.
\dword{lbnf} specifically oversees the cryostat and cryogenics system, which are discussed in detail in the \dword{lbnf} \dword{tdr}.
The space in the \dword{sdwf} must be shared between the detector components for the \dword{dpmod} and the \dword{lbnf} material to construct the next cryostat.

The \dword{surf} Facility Access Specification~\cite{bib:docdb328} defines the limitations on dimensions and weights for all materials to be transported underground, the most stringent of which are set by the Ross Shaft and Cage.
It is possible to bring material down the shaft underneath the cage or in the skip compartment as a slung load, but this is a much slower process and requires careful planning and review of detailed procedures for each trip.
In some cases this is the only possibility.
Most material will be brought underground inside the cage.
Figure~\ref{fig:shaft-cage} illustrates the new Ross Cage and summarizes its main dimensions.
\begin{dunefigure}[Ross Shaft cage]{fig:shaft-cage}
{Preliminary model of the new Ross Shaft cage.}
\includegraphics[width=1.\textwidth]{shaft-cage.png}
\end{dunefigure}

The roundtrip travel time for the Ross cage is 17 minutes dominated by loading and unloading time (the actual travel time is 3.6 minutes each way).
Slung loads will require more than an hour round trip.
The Ross Headframe has no loading dock so careful planning of material loading and unloading of shipments is required.
All materials must arrive on a flatbed or curtain-sided chassis, and a
forklift will be available for unloading.
All deliveries, either from the \dword{sdwf} or direct to the Ross Headframe, require (1) coordination with the \dword{sdsd} logistics organization, and (2) minimum two weeks prior notice, per an advance delivery plan.
Logistics will provide a shipping manual to \dword{dune} institutions that will specify guidelines on required shipping data and cargo consignment such that the logistics organization can monitor shipping progress and no delays occur due to incomplete or missing documentation.
To prevent delays due to missing deliveries a minimum one month buffer of materials from detector components is foreseen.
This buffer will allow advance planning for the underground work, with confidence that all materials will be available as needed.
Sufficient space must be made available at the \dword{sdwf} and in the underground area to house this material.
To determine the storage space requirements and how much hoist time must be dedicated a detailed inventory of all detector equipment and infrastructure is needed.
The \dword{sdwf} staff in some occasion may de-consolidate or consolidate arriving cargo into appropriately sized boxes and crates, as needed, for delivery to \dword{surf}, to make the most efficient use of available trucks and the Ross Shaft.

The \dword{uit} supports the consortia installation effort and coordinates all the \dword{dune} underground work.
Each consortium is at all times responsible for the detector components and equipment that they provide to \dword{dune}.
The \dword{uit} supports the consortium installation work by providing the needed expertise to operate all cranes, platforms, hoists, and other specialized rigging equipment needed to move consortia equipment.
Special lifting devices are expected to be needed to install \dwords{crp}, \dword{hv} system \dword{fc} and cathode, and electronics.
Special devices may need to go through approval procedures and tests prior their use (tests that should be done on surface in the \dword{sdwf}).
The \dword{uit} leader is the responsible for supervising the \dword{uit} members, but the ultimate responsibility for all detector components will remain within the consortia even while the underground team is rigging or transporting these components.
This will be critical in case parts are damaged during transport or installation, as the consortia need to judge the necessary actions.
For this reason, a consortium representative or point of contact must be present when any work is performed on their equipment.
The consortium is also responsible for certifying that each installation step is properly performed.
The \dword{uit} is responsible for planning the work underground and communicating this plan to the consortia.
The consortia are responsible for planning the individual installation steps for their equipment and communicating this to the \dword{uit}.
This includes the reviewed and approved Consortium Installation Plan and all of the approved detailed Installation Procedures.

%\cleardoublepage

\section{Detector infrastructure}
\label{ch:dp-tc-infrastructure}

ACCESS TO THE ROOF

The detector infrastructure for the \dword{dpmod} comprises the equipment used by several groups to install and operate the detector.
It includes the electronics racks on top of the mezzanine on the cryostat roof and the related networking, electrical and cooling power, cable trays, cryostat penetrations and flanges, detector ground isolation transformers and monitors.
Inside the cryostat, the detector infrastructure includes the gas and liquid argon distribution manifolds, the cryostat false floor to protect the corrugated membrane and allow safe circulation of material and people inside the cryostat, movable scaffolding and man lifts for activities at height.
A clean room equipped with cold boxes for the functional tests of the \dwords{crp} is also present in front of the \dword{tco}.
This clean room and the cryostat itself are the places where some detector components are assembled before installation.

\subsection{Cryostat roof}
A drawing of the full cryostat roof with its penetrations is shown in figure~\ref{fig:DP-roof-penetration}.
A zoom of the same image of a corner on the opposite side of the \dword{tco} is shown in figure~\ref{fig:DP-roof-penetration-zoom}.
The same figure shows also the \threed model of the roof with equipment on the \dword{crp} support and \dword{sgft} penetrations. \fixme{what is sgft?}
This image highlights the high density of the equipment on the \dword{dpmod} roof, even more important considering that each of these flanges will be reached by several cables.

\begin{dunefigure}[Drawing of \dword{dpmod} roof]{fig:DP-roof-penetration}
{Drawing of the cryostat roof with its penetration.}
\includegraphics[width=1.\textwidth]{DP-roof-penetration.png}
\end{dunefigure}

\begin{dunefigure}[Zoom of the \dword{dpmod} roof]{fig:DP-roof-penetration-zoom}
{Zoom of the \dword{dpmod} roof drawing (left) and 3D model of the same portion of the roof with the \dword{sgft} and \dword{crp} suspension chimneys instrumented.}
\includegraphics[width=0.5\textwidth]{DP-roof-penetration-zoom.png}
\includegraphics[width=0.5\textwidth]{DP-roof-penetration-3D.png}
\end{dunefigure}

The role of the penetrations is to \emph{connect} the ultra-pure argon volume to the atmosphere through the roof insulation, while minimising the heat input.
All the penetrations are on the roof of the cryostat except the ones for the liquid argon re-circulation that are at the bottom of the \dword{tco} wall.
The sketch of a typical roof penetration cross-section is shown in figure~\ref{fig:penetration-cross-section-sketch}.
\begin{dunefigure}[Sketch of a typical roof penetration]{fig:penetration-cross-section-sketch}
{Sketch of a typical roof penetration.}
\includegraphics[width=1.\textwidth]{penetration-cross-section-sketch.png}
\end{dunefigure}
The thermal insulation is installed around a vertical stainless steel pipe that is mechanically connected to the walls of the warm structure.
The corrugated membrane is welded vacuum tight to the pipes.
The excess pipe inside the cryostat volume may be used to weld a mechanical support, for instance to attach cable trays.
On the atmospheric side a \dword{uhv} CF flange is welded vacuum tight to the pipe. \fixme{Define UHV in glossary}
Except for the four man holes, all the flanges on the chimneys follow the \dword{uhv} CF flange standard ISO 3669:2017.
These flanges may need to be supported in case they must sustain some load, as for example in the case of the penetrations for the \dword{crp} and field cage supports.
These flanges carry out the role of interface between the cryostat and the feedthroughs.
The scope of the detector infrastructure ends at the flange.
Copper gaskets, bolts, and nuts are in the scope of the mating flange.
The holes on the cryostat flanges are not threaded.
In order to allow an effective air purge, each feedthrough must foresee an exhaust positioned preferably at the maximum hight in order to avoid air pockets.
For reasons of space, the purge lines for the \dword{sgft} are embedded in the stainless steel pipe of the penetrations.
COMMENT ON THE VALVES

CONSTRAINTS ON THE LINEARITY OF THE PIPE, ON THE VERTICALITY OF THE PIPE AND ON THE HORIZONTALITY OF THE FLANGE AND ON THE ORIENTATION OF THE FLANGE.
REQUIREMENT ON THE TIGHTNESS

There are several kinds of penetrations that serves different needs.
The table~\ref{tab:penetrations} summarises the them.
\begin{dunetable}[Types of penetrations through the roof]
{ccccccc}
{tab:penetrations}
{Types of penetrations through the roof.}
Type & Number & CF & Pipe OD (mm) & Pipe ID (mm) & Ext. Sup. & Int. Sup.\\
\toprowrule \dword{crp} sup. & 240 & CF100 & 104 & 100 & yes & no\\
\colhline Field cage sup. & 24 & CF150 & 154 & 150 & yes & no\\
\colhline \dword{crp} inst. & 42 & CF250 & 254 & 250 & no & yes\\
\colhline Tank inst. & 20 & CF250 & 254 & 250 & no & yes\\
\colhline \dword{sgft} & 240 & CF275 & 273.2 & 267 & no & no\\
\colhline VHV & 1 & CF250 & 254 & 250 & no & yes\\
\colhline Manhole & 4 & custom & 706 & 700 & no & no\\
\colhline Cryo & 39 & several & several & several & several & several\\
\end{dunetable}

TABLE WITH THE WEIGHTS THAT THE MECHANICAL FEEDTHROUGHS MUST SUPPORT
\begin{dunetable}[Load for mechanical support penetrations]
{ccccc}
{tab:load-penetration}
{Load for mechanical support penetrations.}
Type & Number & Installation (kg) & Empty (kg) & Full (kg) \\
\toprowrule \dword{crp} sup. & 240 & 150 & 150 & 150\\
\colhline Field cage sup. & 24 & XXX & XXX & XXX \\
\end{dunetable}

BRIEF DESCRIPTION OF THE \dword{crp} SUPPORTS. NOTE THAT THE THERE ARE TWO TYPES OF TRIANCLES. NOTE ALLOW SHRINKAGE AND RE-ADJUSTMENT IN 4X4 CRPS ISLANDS

BRIEF DESCRIPTION OF THE FIELD CAGE SUPPORT. NOTE THAT ON THE SHORT WALLS THE EXTERNAL I-BEAM ARE VERY CLOSE TO THE PENETRATOINS. NOTE ALLOW SHRINKAGE.

BRIEF DESCRIPTION OF THE \dword{crp} INSTRUMENTATION. NOTE THAT THEY SERVE TWO CRPS AND NOT ALSWAYS WAS POSSIBLE TO PUT THEM IN THE RIGHT PLACE. NEED TO DEVELOP AND APPROPRIATE CABLE SUPPORTING SYETEM.

BRIEF DESCRIPTION OF THE TANK INSTRUMENTATION. NOTE THAT NO CABLE TRAYS ARE ALOWED TO BE ATTACHED TO THE CORRUGATED MEMBRANE. DEDICATED CABLE SUPPORTS AND USE THE CRYOGENIC PIPES AS CABLE TRAY WHEREVER POSSIBLE.

BRIEF DESCRIPTION OF THE VHV PENETRATION. NOTE THAT IT IS IN THE ACTIVE VOLUME. THAT CRP IS DIFFERENT.

The \dword{sgft} were developed first for the 3x1x1 detector at \dword{cern} and then applied to the \dword{pddp} with a slightly different design.
Their role is to allow the electronics to be near the anode electrodes (minimise the cable lengths) and in cold (reduce the intrinsic noise of the electronics) while still having the possibility to extract the electronics without compromising the liquid argon purity.
The drawing of the \dword{sgft} used in \dword{pddp} is shown in figure~\ref{fig:sgft-drawing}.
\begin{dunefigure}[\dword{sgft} drawing]{fig:sgft-drawing}
{Drawing of the \dword{sgft} used in \dword{pddp}.}
\includegraphics[width=1.\textwidth]{sgft-drawing.png}
\end{dunefigure}

The \dword{sgft} are tubes through the cryostat roof insulation.
Their inner volume is isolated form the argon and from the atmosphere.
The electronics is installed inside the tube at a level few cm below the corrugated membrane of the cryostat roof.
At the cold side of the \dword{sgft}, a flange hosts electrical connectors that connect the cables to the anode at one side and the cold electronics at the other (see figure~\ref{fig:sgft-picture}).
A system of guiding rails along the length of the \dword{sgft} ensures that the electronics can be correctly plugged into the cold flange.
\begin{dunefigure}[\dword{sgft} picture]{fig:sgft-picture}
{Image of the flange and the tube of the \dword{sgft} used in \dword{pddp} prior assembly.}
\includegraphics[width=.5\textwidth]{sgft-picture.jpg}
\end{dunefigure}
This flange is crucial because at cold it isolates the ultra-pure argon.
The \dword{sgft} volume is filled with nitrogen or argon to avoid moisture condensation on the electronics.
The temperature at the bottom of the \dword{sgft} is not low enough to allow the condensation of argon.
Having argon at 1~bar as the atmosphere inside the \dwords{sgft} would reduce the possible \dword{lar} contamination consequent to the development of a leak in the cold flange.
Each \dword{sgft} before installation must be electrically tested and helium leak must be proven to be better than $10^{-8}$~mbar$\times$l/s at room temperature.
At least the cold flanges must be leak tested at cryogenic conditions too.

DESCRIPTION OF THE LAYOUT OF THE CABLE TRAYS. NOT YET FULLY DEFINED.

DESCRIPTION OF THE LAYOUT OF THE ELECTRONICS RACKS. NOT YET FULY DEFINED.

BRIEF DESCRIPTION OF THE PURGE PIPES. SCOPE OF THE CRYOGENICS. NOTE POSSIBLE INTERFERENCES.

\subsection{Cryostat inside}
The internal cryogenics comprises manifold of pipes to ensure the proper detector purge, cool-down, filling and re-circulation of \dword{lar}.
A 3D view of the inside of the \dword{dpmod} with the cryogenic pipes is shown in figure~\ref{fig:internal-cryogenics}
All pipes enter the cryostat from the cryo penetrations on the roof.
The \dword{gar} distribution system consists of a set of pipes installed on the cryostat floor (shown in red in figure~\ref{fig:internal-cryogenics}).
These pipes are used only prior to filling to remove the air in the cryostat.
They have either a longitudinal slit or calibrated holes to distribute \dword{gar} uniformly along the length of the cryostat.
The piston purge effect was successfully tested in \dword{pdsp}.
In order to efficiently extract the air, turbulence must be avoided while purging with warm \dword{gar}.
The optimal filling speed is estimated to be 1.2~m/h (vertical height in the cryostat).
\begin{dunefigure}[Internal cryogenic piping]{fig:internal-cryogenics}
{3D model view of the internal cryogenic piping.}
\includegraphics[width=1.\textwidth]{internal-cryogenics.png}
\end{dunefigure}

The \dword{lar} distribution system consists of a set of pipes again installed on the floor of the cryostat (shown in blue in figure~\ref{fig:internal-cryogenics}).
They are required for flowing \dword{lar} over a broad range of flow rates.
These pipes are used to fill the cryostat and, during steady state operations, to return the \dword{lar} from the purification system.
These pipes have calibrated holes to return the LAr uniformly throughout the length of the cryostat.
This is fundamental to maintain uniform purity along the full length of the detector.
Four pumps circulate the \dword{lar} inside the cryostat, all of which operate initially to achieve the desired purity.
Once the target purity is achieved, only one or two pumps remain in service.

A system of cool-down sprayers is installed along the two long walls of the cryostat just below the ceiling.
One set distributes \dword{lar} using liquid sprayers that generate a conical profile of small droplets of liquid.
The other set of sprayers distributes \dword{gar} to move the \dword{lar} droplets inside and cool down the detector and cryostat uniformly.
These sprayers are being tested \dword{pddp} and are a modified version of the ones used in \dword{pdsp}.
They are a variation of those implemented in ProtoDUNE-SP.
The cooling system design is being updated now, therefore the solution that will actually be implemented may differ from what described here.

Other infrastructure inside the cryostat includes the cryostat false floor, the UV-filtered lighting, and the battery-operated scissor lifts.
The false floor must ensure a flat surface above the internal cryogenic pipes to allow the safe operation of the man lift and the positioning of the scaffolding.
The heaviest load that the false floor must support is the 12~m high man lift, though the requirements for the false floor are not yet fully defined.
Following the experience of \dwords{protodune} the false floor can be made out of plywood panels fixed on wood blocks that defined the false floor height.
The wood blocks are simply standing on the corrugated membrane, with some plastic protection to avoid damage to the corrugated membrane.
The wood must be painted to minimise wood dust in the cryostat and it must also be treated with fire retardant agents.
The floor must be composed of section that can be independently and easily dismounted according to the needs of the installation.
This is fundamental when installing the \dword{pd} and the ground grid, since they occupy the space taken by the false floor.

The cryostat lighting, using UV-filtered LED lamps, is expected to be fairly simple.
Options for the lighting will be developed further during tests at the Ash River facility.
Floor-mounted lights and lights entering from some roof penetrations will be investigated.

At least one commercially-available battery-operated scissor lift with a 12~m reach is used to work on the \dword{crp} installation and cabling as well as for the installation of the field cage and the cryogenic instrumentation.
Tests at Ash River will verify the stability of the lift at height.
If the lift is determined to be suitable, then the remaining issue to resolve is how to insert and remove it from the cryostat.
Commercially-available scissor lifts are too wide to fit easily through the \dword{tco} opening where one of the large cryostat support I-beams protrudes above the floor level.
Custom lifting equipment is needed to insert and remove the lifts into the cryostat.
At the end of the installation process, the last lift may require partial dismantling before it can be removed from the cryostat.

VENTILATION

\subsection{Clean room}
The underground clean room in front of the \dword{tco} is the main place where the detector components are unpacked, assembled and tested prior the installation inside the cryostat.
The experience in building \dword{pdsp} and \dword{pddp} showed that the role of the clean room is fundamental in several aspects.
The cleanliness of the air and the absence powder and dust is fundamental for all the detector components, and in particular for the \dwords{lem} in the \dwords{crp}.
The clean room should ensure the adequate cleanliness of the air where the detector components are unpacked and tested, and it should meet ISO-8 clean room class standards.
The clean room also isolates and protects the inside of the cryostat from the dirty environment of the cavern, which has exposed exposed rocks and non-protected floor.
In order to achieve this, a ventilation and an air filtration system well dimensioned is required.
The requirements for work in an ISO-8 cleanroom are a cleanroom lab coat, clean shoes, or shoe-covers, and nets for hair and beards.
This is in addition to a clean hard hat and gloves for safety reasons.

The clean room must provide space and tools to test the detector equipment, most notably the \dwords{crp} in the cold boxes and the \dword{pmt} tests in the dark box.
Compared to the \dword{spmod} the clean room is smaller in all the directions, since most components do not need to be assembled prior installation, and the major assembly work of the cathode and field cage is done directly inside of the cryostat.

Two view of the 3D model of the clean room are shown in figure~\ref{fig:cleanroom}.
The clean room surrounds and seals against the \dword{tco}, to isolate the cryostat.
In addition, the cryostat \dword{tco} can be temporary closed with plastic curtains, to isolate it from possible dirty jobs inside the clean room.

THE DP TCO IS SMALLER BUT THE MODEL IS NOT YET UPDATED

The mechanical structure for the clean room is a self sustained steel skeleton where the walls, made out of rigid aluminium-foam-aluminium sandwich panels, are fixed.
The size of the clean room are about $16 \mathrm{(L)} \times 16 \mathrm{(L)} \times 9 \mathrm{(H)}$~m$^3$.
The floor must be covered with resin or other material that is simple to keep clean.
The clean room floor and the objects inside the clean room must undergo regular cleaning in order to ensure all the time the ISO-8 class standard.

\begin{dunefigure}[Internal cryogenic piping]{fig:cleanroom}
{Two views of the 3D model of the clean room in front of the \dword{tco}.}
\includegraphics[width=1.\textwidth]{cleanroom1.png}
\includegraphics[width=1.\textwidth]{cleanroom2.png}
\end{dunefigure}

The clean room layout can be seen in figure~\ref{fig:cleanroom} (bottom).
Around the \dword{tco} enough space is left in order to install and access the four liquid argon pumps that stay on the cryostat level on the two sides of the \dword{tco}.
The clean room is equipped with four cold boxes, overhead mobile cranes, a material airlock, a personnel airlock, space for the cryogenics to operate the cold boxes, and a room to test the \dwords{pmt} in a dark box.
Additional space may be required to work on detector components that need to be modified/fixed/refurbished and for which the work does not justify to bring the component on surface.
An additional $5\times6$~m$^2$ equipped with two gantry cranes to lift one \dword{crp} may be needed and can be foreseen.
The material airlock is large enough to house one \dword{crp} box, that is the widest component that needs to enter.
The material access is done at the cryostat level through two 3.8~m large rolling doors.
Unlike the \dword{pdsp} and \dword{pddp}, in this cans the airlock does not need to have an openable roof.
Nonetheless, this option can be easily implemented in the present design of the clean room.
In this airlock the material is cleaned and if needed unpacked before entering into the actual clean room.
For the components that are longer than the length of the airlock, as possibly the I beam that support the filed cage, both the doors can be temporary and shortly opened to allow the access of the material without compromise the quality of the air in the clean room.
In case of longer operation that require both rolling doors to be opened, temporary plastic curtains can be installed to limit the powder and dust from the cavern to enter into the clean room.
The airlock is large enough so that a standard fork-lift can enter and maneuver.
Inside the cleanroom, instead, due to space constraints only manual forklift can be used.

The clean room hosts four cold boxes and the cryogenic infrastructure in order to operate them.
The cold boxes are arranged in two rows along the walls of the clean room, one in front of the other.
The central region is used to manipulate the \dword{crp} box and to attach the \dword{crp} below the cold box roof.
This region is also used to manipulate all the elements prior insertion in the cryostat.

The cryogenics for the cold boxes is not fully defined and detailed yet.
The tall cryostats shown in figure~\ref{fig:cleanroom} are mostly a place-holder.
If more space is required for the cryogenic installation, the area outside the clean room can be used, since the cryogenics does not need to be installed in a clean environment.

Along the width of the clean room and above the cold boxes there are I-beams equipped with trolleys and electrical hoists that serve as 2~ton over-head cranes to open and close the cold boxes.
These I beams may be connected to the cavern walls or the steel structure of the clean room.
A third I-beam connects the \dword{tco} structure to the clran room structure and in the same way it is used as crane to bring material inside the cryostat.
The heaviest detector component to be brought inside the cryostat is the \dword{crp} in its box.
Its total weight is below 800~kg.
No major activities at height are needed inside the clean room.
For the lifting of the \dword{crp} box and in order to insert it inside the cryostat there will be available at least one 8~m-tall compact man lift.

The \dword{crp} in its box is not the heavies object to enter into the cryostat.
In fact, despite not having yet identified the 12~m tall man lift, it will certainly be heavier and bigger than the \dword{crp} box.
A custom tool to insert heavy object inside the cryostat must be developed and possibly to be used in conjunction to the cavern crane.
For this reason, the clean room roof in front of the \dword{tco} and the I-beam that enters into the cryostat can be removed.
This operation will be necessary also to complete the closure of the external structure of the \dword{tco}.

CABLE TRAYS

\subsection{Cold boxes}
An electrical and mechanical test of each entire \dword{crp} in realistic cryogenic thermodynamic conditions and in reasonably pure argon vapor must be performed in order to access the final \dword{qc} of the \dword{lem} and \dword{crp} assembly before the installation in the cryostat.

The tests are performed in four dedicated cryostats, also referred to as \emph{cold boxes}, installed in the clean room in front of the \dword{tco} and with dedicated cryogenic infrastructure that allows to operate each of them independently.
The objectives of the global test of the \dwords{crp} are:
\begin{itemize}
\item the characterization of the \dword{hv} operation of each \dword{lem} independently and as well the \dword{hv} operation of all the 36~\dwords{lem} forming a \dword{crp} at the same time,
\item the characterization of the \dword{hv} operation of the extraction grid,
\item the test of all the \dword{hv} contacts from the feedthroughs to the \dword{lem} and grid connectors,
\item the measurement of the flatness of the \dword{crp} during the test and the capability to align the \dword{crp} to the \dword{lar} level.
\item calibration of the level measurement from the capacitance between the extraction grid and the bottom electrode of the \dword{lem}.
\end{itemize}

The requirements for the environment of this test are defined as following:
\begin{itemize}
\item the \dword{lar} surface must be flat withing 2~mm in order to allow the \dword{lem} to be in the vapor phase and the grid in liquid phase,
\item the vapor pressure does not need to be stabilised, but it must constantly be monitored,
\item the vapor temperature should not be controlled, but constantly monitored,
\item the liquid argon purity should be of the order of at most 100~ppm (O$_2^{eq}$), in order to be comparable with the tests at 3.3~bar and room temperature done for each \dword{lem} during the \dwords{qc} before reaching \surf,
\item the \dword{crp} vertical position and horizontality should be adjustable, at least before the filling of the cold box.
\end{itemize}

Figure~\ref{fig:NP02-ColdBoxSketch} shows a sketch of the cold box.
The same concept of cold box was successfully used several times at \dword{cern} in 2018 to characterise the four \dwords{crp} installed in \dword{pddp}.
\begin{dunefigure}[Sketch of the cold box]{fig:NP02-ColdBoxSketch}
{Schematic representation of one cold box for the test of the \dwords{crp}.}
\includegraphics[width=1.\textwidth]{NP02-ColdBoxSketch.png}
\end{dunefigure}
The external structure is a self-sustained box made out of 1~cm-thick stainless steel plates.
The internal membrane is made out of 2~mm-thick stainless steel plates.
The internal dimensions are 1~m in height and 3.9~m in the other directions.
The thermal insulation is passive and it consists of four layers each of 10~cm-thick polyurethane foam (30~kg/m$^3$).
The insulation layers are separated by polyethylene foils to minimise the convection phenomena and the consequent increase of heat input.
The insulation region is flushed with gas nitrogen, as it is done for all the corrugated membrane cryostats.
The cold boxes are too large to be transported underground, therefore they must be produced underground during the construction of the clean room.
A custom made RGA-based dense gas analyser monitors the composition of the gas nitrogen circulating in the insulation space.
The same system may be used to monitor the quality of argon boil-off with a sensitivity of about 50~ppm of O$_2$ and N$_2$.

The \dwords{crp} is attached to a portion of the roof (shown in blue in figure~\ref{fig:NP02-ColdBoxSketch}) that can be opened.
The horizontality and the vertical position of the \dword{crp} can be adjusted withing $\pm 20$~mm around the nominal position.
The adjustment is done manually from the hanging system installed on the roof.
In order to empty the box once the test is concluded, a 10~kW heater is used to evaporate the liquid argon, that is then re-condensed by the cryogenic system and stored in a standard insulated cryostat.

The liquid argon level in the cold box is kept constant by compensating the losses due to the evaporation with new liquid argon.
The cold boil-off vapour is sent to the cryogenic systems via vacuum insulation pipes to be re-condensed.

DESCRIPTION OF THE CRYO SYSTEM REQUIREMENTS

CYCLE DESCRIPTION

The operation of the \dword{pddp} cold box proved that the requirements can be met and exceeded.
Figure\ref{fig:NP02-cold-box} shows the \dword{pddp} cold box prior being closed for a test.
A \dwords{crp} is hanging from the roof and it is being closed inside the box.
\begin{dunefigure}[Photo of the \dword{pddp} cold box]{fig:NP02-cold-box}
{Photo of the \dword{pddp} cold box open and with a \dwords{crp} hanging from the roof.}
\includegraphics[width=1.\textwidth]{NP02-cold-box.jpg}
\end{dunefigure}

During the tests, the contamination of nitrogen and oxygen was always well below 100~ppm and the measurement was limited by the sensitivity of the instrument.
The level was monitored with a coaxial capacitive level meter and it was also monitored by means of eight capacitive level meter installed around the \dword{crp} with a precision of 250~$\mu$m.
The level was automatically and constantly adjusted to the nominal value refiling the cold box with new liquid argon.
The liquid argon evaporation resulted in a level decrease of about 0.7~mm/h\footnote{The evaporation rate is a simple method to evaluate the heat input due to the cold box only which results in less than 700~W. In the case of the \dword{pddp} cold box, the argon was let evaporate and replaced with new argon to maintain the level.}.
A simplified version of the \dword{pddp} final suspension system allows to adjust the position of the CRP with respect to the liquid argon level using three manual winches changing the height of the CRP at the level of the anchoring points by step of 0.4~mm.
To complete the cold box instrumentation, three cryogenic cameras were used to control the \dword{crp} from below and above.
One camera was positioned to give a side view of the \dword{crp} with a field of view of the order of 40~cm.
In figure~\ref{fig:NP02-ColdBoxCamera} one can appreciate the flatness of the liquid argon level, which is much better than 2~mm even when the regulation of the liquid argon level is functioning.
\begin{dunefigure}[Image from inside the \dword{pddp} cold box]{fig:NP02-ColdBoxCamera}
{Photo from the inside of the \dword{pddp} cold box at \dword{cern} during a \dword{crp} test.}
\includegraphics[width=1.\textwidth]{NP02-ColdBoxCamera.jpg}
\end{dunefigure}

\subsection{Alternative position of the cold boxes}
IN CASE THE DP IS NOT THE SECOND DETECTOR THERE IS NO SPACE IN FRONT OF THE CRYOSTAT AND ADIFFERENT SPACE MUST BE ARRANGED FOR THE CLEAN ROOM AND THE COLD BOXES.
\begin{dunefigure}[]{fig:}
{.}
\includegraphics[width=1.\textwidth]{image001.png}
\end{dunefigure}


%\cleardoublepage

\section{Detector installation}
\label{ch:dp-tc-installation}

DESCRIPTION OF THE INSTALLATION OF THE INFRASTRUCTURE NEEDED BEFORE STARTING THE INSTALLATION OF THE DETECTOR (ROOF, CLEAN ROOM AND COLD BOXES)

The \dword{dp} detector is a single active volume device with the \dwords{crp}\footnote{The amplification and charge sensitive device.} plane parallel to and at the level of the liquid argon surface, the field cage walls installed vertically to delimit the active volume, the cathode plane hanging from the field cage at the bottom.
The photo-detectors are installed below the cathode, and they are protected from the strong electric field by a ground grid that covers them.
The \dword{dpmod} is completed with a set of cryogenic instrumentation devices that permits the proper operation of the detector and by the cold electronics for the charge readout that is installed on the roof of the cryostat.
A number of additional items, like the safety, slow control, \dword{daq} racks complement the detector.

The access to insert large detector components into the cryostat is the \dword{tco}, a large opening in one of the two short walls of the cryostat.
Given the structure of the \dword{tpc}, the installation starts from the opposite side of the \dword{tco}.
The \dwords{crp}, the field cage, and the cathode hang from the cryostat roof.
The natural installation sequence for the \dword{tpc} components is the field cage and \dwords{crp} first, followed by the cathode, and finally by the \dwords{pmt} and the ground grid.

The installations of the \dwords{crp} and the field cage are mostly independent.

CRP AND FIELD CAGE MOSTLY INDEPENDENT
NOT POSSIBLE TO INSTALL AT THE SAME THE FIELD CAGE AND THE CRP NEXT TO IT
POSITIONING OF THE CRP FAST <0.5 DAYS 3-4 PEOPLE TOGETHER
MORE COMPLEX CABLING WORKING AT HEIGHT POSITION OF THE PENETRATION NOT ALWAYS IN THE OPTIMAL POSITION.
OPTIMISING THE BUNDLES:
- SIGNAL CABLES >0.5 DAYS 2 PEOPLE.
- CRP INSTRUMENTATION >0.5 2 PEOPLE.
IDEALLY 2 PEOPLE ON ONE SCISSOR LIFT REACHING THE CEILING.
SCAFFOLDING MORE STABLE BUT MORE DIFFICULT TO MOVE (WHILE LIFTING THE CRP SCAFFOLDING CANNOT BE IN PLACE).
SPECIAL SCAFFOLDING MAY BE THE ONLY POSSIBILITY TO REACH AT HEIGHT OUTSIDE THE FIELD CAGE ALONG THE LONG WALLS.
STRATEGY MAKE THE FIELD CAGE AND THE CRP INSTALLATION THE MOST INDEPENDENT AS POSSIBLE: DIFFEREMNT PEOPLE, DIFFERENT REGION OF THE CRYOSTAT, ...

MUCH MORE COMPLEX, BUT IN MY OPINION ABSOLUTELY NECESSARY, TESTING THE LEMS ONCE INSTALLED.
DISCHARGE DEPENDS ON THE HUMIDITY.
POSSIBLY CONTROLLING THE CRYOSTAT HMIDITY IS MANDATORY.
HV ON THE LEMS WHILE INSTALLATION OF NEXT CRP IS ONGOING.
TEST MUST BE DONE BEFORE INSTALLATION OF THE CATHODE SO THAT THE LEMS ARE STILL REACHABLE WITHOUT DISMOUNTING MAJOR INSTALLATION.
INSTALLATION OF AN ISLAND (4X4 CRPS AND TWO OPPOSITE FIELD CAGE)
BEFORE INSTALLATION OF THE CATHODE AND THE PMTS
MEASUREMENT OF THE POSITION AND THE PLANARITY OF THE CRP.
MEASURE EACH CRP ONCE HANGING FOR THE FIRST TIME.
BEFORE LIFTING NEED TO MAKE IT FLAT WITHIN +/- 1MM.
2 PEOPLE FOR THE GEOMETRY
2 PEOPLE TO ALIGHT THE CRP
TIME NEEDED 0.5 DAYS.
TEST OF THE POSITIONIN SYSTEM OF THE CRP AS SOON AS IT IS LIFTED.
ONCE 4X4 CRPS ARE INSTALLED GLOBAL POSITIONIN TEST.
AFTER INSTALLATION OF THE CATHODE.

CRP TEST PER COLD BOX TAKES 10 DAYS IN THE BEST CASE.
THERE WILL BE 4 COLD BOXES.
INSTALLATION OF 1 CRP EVERY 2.5 DAYS.
POSSIBLY PILE UP DUE TO DELAY/ADVANCEMENT OF THE INDEPENDENT CRP TESTS.
STRATEGY: PEOPLE TESTING THE CRPS MAY NEED TO HELP INSTALLING AND CABLING
1 PERSON CAN FOLLOW THE 4 DAYS OF TEST OF TWO CRPS AND TAKE CARE OF INTERFACING WITH CRYO AND UIT FOR THE COOLIGN AND WARMING UP OF THE COLD BOX AND OPENING AND CLOSING THE COLD BOX.
1 PERSON CONSTANTLY IN CHARGE OF 2 CRP ENOUGH?
NEED SPECIALISED PERSON ABLE TO TEST HV FUNCTIONALITY.
MAY NOT BE THE SAME PERSON WHO INSTALL.
IF NEEDED MUST BE ABEL TO INSTALL AND CABLE.

FIELD CAGE ASSEMBLY REQUIRES A SURFACE AT THE ENTRANCE OF THE CRYOSTAT FREE AND LARGE ENOUGH TO ASSEMBLE AND STOCK THE MODULES OF 4X2 M2.
ONE SUPER MODULE FORMED BY 18 MODULES.
2 PEOPLE CAN ASSEMBLE 4 MODULES IN ONE DAY.
ASSEMBLY OF THE MODULES DOES NOT REQUIRE WORKING AT HEIGHT (REACH 2.5 M AT MOST)
MODULES IMPLY THE INSTALLATION OF THE RESISTOR DIVIDER, THE REFLECTOR FOILS AND THE TEST OF THE CONTACTS AND THE TEST OF THE RESISTOR DIVIDER.
THE REFLECTOR FOILS ARE FRAGILE MUST ARRIVE IN BOXES WELL PACKED.
SMALL PANELS ATTACHED WITH SCREWS WILL NOT NEED SPECIAL TOOLS.
PANELS FRIGILE AND DELICATE (TPB COATING OR THIN REFLECTOR) VERY CAREFUL MANIPULATION.
MOST OF THE MODULES ARE THE SAME.
5 DAYS TO COMPLETE ONE SUPER MODULE.
ENOUGH SPACE TO STORE THE 18 MODULES.
ASSEMBLY OF THE SUPER MODULE DONE IN ONE SHOT.
10-12 M LONG I-BEAM MAY BE ASSEMBLED FROM SHORTER SECTIONS INSIDE THE CRYOSTAT.
DETAILS OF THE DESIGN NOT YET DEFINED.
ENVISAGE SHORT BEAM CONNECTED TOGETHER WITH BOLTS (1 DAY SAME PEOPLE DOING THE ASSEMBLY OF THE FIELD CAGE MODULES.
MOVE THE I-BEAM IN THE RIGHT POSITION WITH 2 TRANSPALLET.
CONNECTION TO THE 14M LONG WIRE AND LIFT OF THE BEAM.
LIFT THE I BEAM, CONNECT THREE MODULES, LIFT THE BEAM CONNECTS THREE MODULES, TILL REACHING 12M.
ONCE COMPLETED NEED SUPERMODULE IMPORTANT TO CHECH THE CONNECTVITY.
NEED TO LIFT AND CHANGE THE LOAD TO THE FINAL CIXTURE.
SIMPLE IF THE CRP IS NOT YET INSTALLED, ACCESS WITH SHISSOR LIFT.
NOT NECESSARELY THE CASE. NEED TO FIND A SOLUTION WITH THE EXTERNAL SCAFFOLDING.
THIN AND TALL SCAFFOLDING.
SPECIAL SOLUTION NOT YET FOUND.
PROTODUNE SIMILAR ISSUE, FOUND THE SCAFFOLDING, BUT THE WIDTH WAS THE SAME AND THE HEIGHT WAS HALF...
SOLUTION MUST BE FOUND BECAUSE IT IS IMPERATIVE TO BE ABLE TO REACH THE ROOF AROUND THE FIELD CAGE:
NEED TO CONNECT THE FISRT FIELD SHAPER TO THE SAFETY GRPOUND AND THE BIAS VOLTAGE FEEDTHROUGH.
OPERATION THAT NEED TO BE DONE AS SOON AS THE  INSTALLATION OF THE SUPERMODULE IS COMPLETE AND THE LOAS IS TRANSFERRED TO THE FINAL SUPPORT BECAUSE THE SUPERMODULE MUST BE HV TESTED BEFORE "LOOSING THE ACCESS TO IT" 
LATEST DESIGN DOES NOT NEED TO CLIP PROFILES BETWEEN SUPERMODULES.
BIG SIMPLIFICATIONIN THE INSTALLATION.
TIME TO ASSEMBLE A MODULE DEPENDS ON THE NUMBER OF CLIPS  TO BE INSTALLED (AT MOST AT 2.5M) AND THE NUMBER OF FIELD CAGES.
EVALUATION STILL ONGOING. ANYWAY CAN BE KEPT IN THE 5 DAYS/MODULE ADDING ONE PERSON IN THE ASSEMBLY PROCESS.
ACTIVITY CAN BE PARALELISED (MAINLY AT THE BEGINNING, WHEN THE SPACE IS MORE) IF MORE SPEED IS NEEDED.

LIFTING TOOLS FOR CRPS AND FIELD CAGE MUST BE THERE.
INSTALLATION DOES NOT NEED TO WAIT COMPLETION OF THE CLEAN ROOM.
TO GAIN TIME 240 CRP SUSPENSION SYSETM IS CRITICAL. MUST BE DONE BEFORE CRP TESTING STARTS.
42 FIELD CAGE SUPPORT LESS CRITICAL. BETER DO IT IN ADVANCE.
MATERIAL LIFTING CAN BE DONE WITH MANUAL WHINCHES BOTH FOR CRPS AND FOR FIELD CAGES.
ELECTRIC MOTOR MAKE THINGS SAFER AND SIMPLER IN BOTH CASES.
SOME SOLUTION FOUNF BUT NEED TO GO THROUGH A REVIEW ALSO SAFETY.
MANUAL IMPLIES OMMUNICATION WITH PEOPLE ON THE ROOF.
PEOPLE'S TIME MISUSED TO ONLY LIFT SICRONOUSLY TWO OR THREE WHINCHES.
MAY STILL BE FASTER DOING IT MANUALLY.
LEAVE OPEN THE POSSIBILITY.

POLICY FOR LIFTING EQUIPMENT, MAINLY ABOUT CUSTOM MADE TOOLS? NEED FOR TESTS? FORESEE WELL IN ADVANCE AN DALWAYS INVOLE THE SDSD EXPERTS FROM THE BEGINNING.
NEED OF A LOT OF CUSTOM AND CRYTICAL DEVICES...
ALL THE MECHANICAL AND SUPPORT FEEDTHROUGHS ARE CUSTOM MADE.
SAFETY REVIEW AND INSPECTION POSSIBLY DONE NOT UNDERGROUND.
NEED TO TEST EACH SINGLE PIECE, OR SAMLES ARE SUFFICIENT?
PARALELIZE THE TEST ON SURFACE IF POSSIBLE (240 + 42 PIECES AT LEAST...)

LOGISTIC OF THE CRPS AND FIELD CAGE ASSEMBLY:
CRPS
CRPS ARRIVE IN THEIR BOX, THEY ARE STORED IN THE SDSF AND DELIVERED TO THE CAVERN IN SMALL BATCHES.
REQUIRED TO HAVE AT LEAST 4 CRPS OUTSIDE THE CLEAN ROOM READY TO BE TESTED.
BOXES CAN STAND VERTICALLY AND THEY ROLL ON WEELS.
THE FLOOR OF THE CAVERN IN NOT FLAT ENOUGH, DEDICATED MANUAL TRANSPORT TOOL TO LIMIT THE ACCELERATION AND SHACKING OF THE CRPS.
ONE TRANSPORT TOOL SUFFICIENT TO MOVE THE CRPS UNDERGROUND.
STORAGE NO PROBLEM, THE BOX CAN BE STORED VERTIACLLY (SURFACE 3M X 0.5M).
EMPTY CRP BOX GO TO SURFACE AND POSSIBLY SHIPPED TO PRODUCTION FACTORY IF N-CRP > N-BOX TO BE DEFINED BY CRP CONSORTIUM
NO SPACE TO STORAGE CRP BOX DOWN, THEY NEED TO GO TO SURFACE.
TEAM OF TWO CRANE DRIVER TO OPENA DN CLOSE THE COLD BOXES: 1 OPERATION/DAY ON AVERAGE
INSERT THE CRP BOX: ~0.5/DAY
EXTRACT THE EMPTY BOX: 0.5/DAY
ARRANGE THE OUTSIDE THE CLEAN ROOM: BOXES WHEN NEEDED
MOVE THE BOX INSIDE THE CRYOSTAT.
MOVE THE SCAFFOLDING, MOUNT AND DISMOUNT IT.
HELP MANEUVERING THE HEAVY LOAD INSIDE THE CRYOSTAT.
HELP IN LIFTING THE CRPS AND SWITCHING TO THE FINAL SUPPORT SYSTEM: 0.5/DAY
ONLY HELP BECAUSE CRP EXPERTS IN CHARGE.

FIELD CAGE
PROFILES ARRIVE IN 4.?M LONG BOXES
VERY FRAGILE AND IN EACH BOX SEVERAL UNDERDS PROFILES THAT SHOULD NOT BE SCRATCHED. FUNDAMENTAL THE PROPER PACKING AT THE COMPANY.
SHIPPED TO THE SDSF AND AS USUAL ENOUGH MATERIAL TO PRODUCE 1 SUPERMODULE MUST BE UNDERGROUND OUTSIDE THE CLEAN ROOM.
STORAGE IS NOT A PROBLEM, BOXES CAN BE PILED.
FUNDAMENTAL THE CRAND DRIVER TO RE-SHUFFLE THE BOXES (CRPS, PMTS, FIELD CAGES, CATHODE, ...)
PROFILES DON'T NEED TO BE CLEANED.
SIMPLE TO INSERT THEM INTO THE CRYOSTAT
FIELD CAGE BEAM ARE SIMPLE TOO AND ARE FEWER.
THEY MUST BE STORED IN A CLEAN SPACE AND THEY NEED TO BE CLEAN ED BEFORE INSTALLATION.
REFLECTOR FOILS ARRIVE. FRAGILE, IN BOXES SPECIAL MANIPULATION TOOL?
ONLY VISUAL INSPECTION PRIOR INSTALLATION.
CRYOSTAT AND CLEAN ROOM STORAGE.

ARRIVING TOWARDS THE CRP THE SPACE LEFT FOR ASSEMBLY OF THE CATHODE AND FIELD CAGE IS LESS AND LESS.
FIELD CAGES MUST BE COMPLETED BEFORE.
FIELD CAGES CAN BE STORED ON THE SIDES OF THE FIELD CAGE.
ALL THE FIELD CAGE, EXCEPT THE ONE IN FRONT OF THE CTO MUST BE INSTALLED BEFORE THE CRP INSTALLATION REACHES THE ISLAND NUMBER 4.
ASSEMBLY TEST AND INSTALLATION OF THE FIELD CAGE MUCH FASTER THAN THE TESTING INSTALLATION AND CABLING OF THE CRP. IT SHOULD BE FEASIBLE.
PART OF THE LAST WALL IN FRONT OF THE TCO MUST BE INSTALLED AFTER THE TCO CLOSURE TO FACILITATE THE TCO CLOSURE.
TCO IS "SMALL", SO SPACE REQUIRED INSIDE IS "SMALL"
TCO CAN BE CLOSED IN TWO STEPS (HALVES) TO STORE THE LEAST AMOUNT OF MATERIAL INSIDE
STRATEGY USED WITH SUCCESS IN PDSP.


The \dword{tpc} installation is meant to allow test

The installation of the entire \dword{dpmod} is driven by the rhythm of testing and installation of the \dwords{crp}.


\begin{dunefigure}[]{fig:}
{.}
\includegraphics[width=1.\textwidth]{hv-assembly-area.png}
\end{dunefigure}

\begin{dunefigure}[]{fig:}
{.}
\includegraphics[width=1.\textwidth]{crp-testing-schedule.png}
\end{dunefigure}

\subsection{Charge readout plane installation}
\subsection{High voltage system installation}
\subsection{Light readout installation}
\subsection{Cryo-instrumentation installation}
\subsection{Electronics installation}


%\cleardoublepage

\section{Cost and Schedule}
\label{ch:dp-tc-costschedrisk}
\subsection{Cost}
\subsection{Schedule}

\fixme{new standard risks table for auto-generating latex as of 3/25. I will send email. Anne}

\fixme{new standard cost table will be coming in early April - for auto-generating latex. Anne}

\fixme{Table~\ref{tab:Xsched} is a standard table template for the TDR schedules.  It contains overall FD dates from Eric James as of March 2019 (orange) that are held in macros in the common/defs.tex file so that the TDR team can change them if needed. Please do not edit these lines! Please add your milestone dates to fit in with the overall FD schedule. Please set captions and label appropriately. Anne}



\begin{dunetable}
[Consortium X Schedule]
{p{0.65\textwidth}p{0.25\textwidth}}
{tab:Xsched}
{Consortium X Schedule}   
Milestone & Date (Month YYYY)   \\ \toprowrule
Technology Decision Dates &      \\ \colhline
Final Design Review Dates &      \\ \colhline
Start of module 0 component production for ProtoDUNE-II &      \\ \colhline
End of module 0 component production for ProtoDUNE-II &      \\ \colhline
\rowcolor{dunepeach} Start of \dword{pdsp}-II installation& \startpduneiispinstall      \\ \colhline
\rowcolor{dunepeach} Start of \dword{pddp}-II installation& \startpduneiidpinstall      \\ \colhline
 \dword{prr} dates &      \\ \colhline
Start of  (component 1) production  &      \\ \colhline
Start of (component 2) production  &      \\ \colhline
Start of  (component 3) production  &      \\ \colhline
\rowcolor{dunepeach}South Dakota Logistics Warehouse available& \sdlwavailable      \\ \colhline
\rowcolor{dunepeach}Beneficial occupancy of cavern 1 and \dword{cuc}& \cucbenocc      \\ \colhline
\rowcolor{dunepeach} \dword{cuc} counting room accessible& \accesscuccountrm      \\ \colhline
\rowcolor{dunepeach}Top of \dword{detmodule} \#1 cryostat accessible& \accesstopfirstcryo      \\ \colhline
End of  (component 1) production  &      \\ \colhline
... & ...                       \\ \colhline
\rowcolor{dunepeach}Start of \dword{detmodule} \#1 TPC installation& \startfirsttpcinstall      \\ \colhline
\rowcolor{dunepeach}End of \dword{detmodule} \#1 TPC installation& \firsttpcinstallend      \\ \colhline
\rowcolor{dunepeach}Top of \dword{detmodule} \#2 accessible& \accesstopsecondcryo      \\ \colhline
 \rowcolor{dunepeach}Start of \dword{detmodule} \#2 TPC installation& \startsecondtpcinstall      \\ \colhline
\rowcolor{dunepeach}End of \dword{detmodule} \#2 TPC installation& \secondtpcinstallend      \\ \colhline

last item & ...                         \\
\end{dunetable}

